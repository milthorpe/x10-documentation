\chapter{A First Look At \Xten's Types}\label{chp:types}

\Xten's type system is not quite the same as any other object-oriented
language's, but it is not really all that different, either. We'll look at
some of the main features here. We don't assume that you are familiar with any
particular language already, but we do assume that you have had some
experience with at least one object-oriented language. We'll make references
to Java and C++, since they are so widely taught as a first language, but only
to help compare \Xten{} with them.

\section{What's In An {\tt .x10} Source File?}
An {\tt .x10} source file comes in three parts:
\begin{itemize}
\item first, an optional ``package'' statement
\item then zero or more ``import''  statements
\item and finally, one or more type declarations, at most one of which is declared to be public.
\end{itemize}
Let's look at each of these parts in detail.
\begin{quote}
{\bf Optional package name:}  The first line in the file, other than blank lines and
comments, can be a ``{\tt package}'' statement.  The syntax is
\begin{quote}{\tt package {\em packageName};}\end{quote}
The package name, by convention, is a ``{\tt .}''-separated  series of
lower-case identifiers, \eg{} {\tt com.ibm.math}.  The identfiers can only
consist of letters, underscores (``{\tt \_}''), and numbers. It must begin with
a letter.

You don't need to have a package spec, and our examples will generally not
bother with them, although we give some examples later for those unfamiliar
with what is at stake.   When no {\tt package} declaration appears, the
declaration in the file is said to be ``in the default package''.

Packages were introduced to allow large projects to organize
their classes.  They provide firewalls behind which to hide classes that are
needed but should not be generally visible, and they provide a systematic scheme
for avoiding name clashes. Before we explain all this, we need the next
ingredient in the file, {\tt import} statements.

{\bf Optional Import Statements}  After the package statement, if there is one,
come the {\tt import} statements:
\begin{quote}\hspace*{2em}{\tt import {\em fullPackageName.properName};}\\
or\\
\hspace*{2em}{\tt import {\em fullPackageName.}*;}
\end{quote}

The full package name is, as we've said, a sequence of ``{\tt .}'' separated
lower-case names. The final part, ``{\tt properName}'', should {\em always} be a
capitalized name: for example, ``{\tt import x10.util.Ran\-dom}''. 

The import
statement tells the compiler where to find whatever it is you are importing.  ''{\tt
x10.}{\tt util.}{\tt Ran\-dom}'', for example, means that in the file system
where {\tt Random.x10} was created, there will be a directory named {\tt x10}
with a subdirectory {\tt util} that contains the source file {\tt Random.x10}.
Which brings up an important point: how does the compiler know where to look
for these directories?   What you have supplied is a {\em relative path}
for the class's {\tt .x10} file.  
Someone has to provide the absolute path that serves as the relative
path's starting point.
The command-line option {\tt -classpath} does this job:
\begin{quote}
  If
no class path is specified, the compiler will look in the current directory
and in the \Xten{} standard library.  If it needs to look elsewhere, you need
to say so by using \\
\hspace*{3em}{\tt -classpath path1;path2;...}\\
in your {\tt x10c} or {\tt x10c++} commenad line.  The value of the
option uses the same system-dependent syntax that you use for the
{\tt PATH} environment variable.  We have shown the Unix style here.
\end{quote}
We don't want to belabor the point here: \Xten{} is consistent with
Java in its use of class paths, and you can find a thorough
discussion of that in 
\href{http://en.wikipedia.org/wiki/Classpath_%28Java%29}
{Wikipedia's article on class paths} or any introduction to Java.

The second form of the {\tt import} statement---the one with an asterisk,
rather than a class name, as its final
part---instructs the compiler to import every class in the package.
`` {\tt import x10.util.*}.'' gets you every last class in the {\tt x10.util}
package.
When to use the catch-all and when it is better to do individual imports is a
matter of taste.

You don't {\em have} to import {\tt x10.util.Random} to use it in your code,
but if you don't, you need to refer to everywhere you use it as
{\tt x10.util.Ran\-dom}---you can't just write {\tt Random}.\footnote{This
should explain why, even though a package spec is a proxy for a file system path, dots and not
slashes are used to punctuate it: first, the dot is file-system independent, and
second, it is consistent with the notation for selecting a field within an object.}

Another situation where you might have to use the fully qualified name is the
rare, but not unknown, case of having two classes with the same proper
name.  
Suppose, for instance, we needed to use both \Xten's class {\tt Random}
and another class named {\tt Random} that was supplied by some other
library.  How do we tell the compiler which one to use for any given call?
Use package names in the code at every reference:
\begin{quote}
\begin{verbatim}
val ours = new x10.util.Random();
val theirs = new com.foo.bar.Random();
\end{verbatim}
\end{quote}
When we said the packages help avoid name clashes, it is this sort of
example that we had in mind.

{\bf One or more type declarations:}  After the optional package and import 
statements come one or more of three possible types of ``type'' declarations:
either classes, interfaces or structs.  We assume you're familiar with classes,
and we'll get to the other two in sections \ref{sec:mata} and \ref{sec:guitge}.
The overwhelming majority of the
time, exactly one appears, but there is no hard and fast rule.  One rule does
apply: there can be at most one type in the file that is declared {\tt public}, 
and the name of that type must be the same as the file's name, minus the
``{\tt .x10}'' file type.
Public classes are usable by anyone, anywhere. 
A type that is declared without that keyword, \eg{} just ``{\tt class X \{...\}}'',
is accessible only for use by classes {\em in exactly the same package.}
This is one of the protection features packages provide. 

If you want to put several type declarations in one file, make sure that the first has the
same name as the source file.  Be kind, though: remember that when someone who
is reading some \Xten{} code sees a reference to a class {\tt C} and wants to
look at the source for that class, they really expect
to find {\tt C} in {\tt C.x10} in the appropriate directory, whether it is
private to a package or not.

{\bf That's it:}  Yes, nothing more, other than comments, appears in an \Xten{} source file.
\end{quote}
\section{What's In A Class?}

The best way to see what's in a class is an example that says it all. {\tt
Ship} in Figure \ref{fig:nark}, on page \pageref{fig:nark}, is it.
We work our way through the code
line-by-line. First, some terminology:

\begin{quote}
A {\bf field} of a class holds a piece of data.\\
A {\bf method} is a piece of code (a function)\\
The {\bf members} of a class are the fields, methods and other types defined within it. 
\end{quote}
\begin{figure}[!htbp]
\hrulefill

%%START X10: Ship.x10 ship
\begin{verbatim}
 1  public class Ship {
 2      public static DEFAULT_SIZE = 500;
 3      public  val name: String;
 4      private var passengers: Array[String](1);
 5      private var onBoard: Int = 0;
 6      
 7      public def this(name: String) {
 8         this(name, DEFAULT_SIZE);
 9      } 
10      public def this(name:String, initialCapacity:Int) {
11         this.name = name;  
12         passengers = new Array[String](1..initialCapacity);
13      }
14  
15      static def resize(size:Int, na:Ship) { /*...*/ }
16  
17      public def addPassenger(name: String) { /*...*/ }
18      def throwOverboard(name: String, why: String) { /*...*/ }
19      protected def showPassengers() { /*...*/ }
20      
21      public static def main(args: Array[String](1)) { /*...*/ } 
22      
23      static class FareClasses { /*...*/ }   
24      protected class Galley { 
25          private val platesPerPerson = 4;
26          public def platesRequiredToday() {
27             return Ship.this.onBoard * this.platesPerPerson;
28          }
29          def this(){}
30       }
31      private var galley : Galley = null ; 
32      public def galley() = this.galley; 
33      public def makeGalley() {
34         this.galley = this.new Galley(); 
35      }
36  
37     }
\end{verbatim}
%%END X10: Ship.x10 ship

\hrulefill
\caption{A class with lots of stuff.}\label{fig:nark}
\end{figure}
\begin{description}
\item {\bf line \xlref{ship-a}{2}:} 
\xline{ship-a}{public static DEFAULT\_SIZE = 500;}
  A ``{\tt static}'' field of a class is a constant
associated with the class. Its lifetime is the same as that of the class: it
is available once the class is available (\ie{} once the class is ``loaded'')
and stays available until the application terminates.  We could have said ``{\tt
static val DEFAULT\_SIZE}'',  but the ``{\tt val}'' is implied.  The initial
value need not be a literal constant, as it is here---it can even be a run-time
calculation, but the expression should not involve any non-static methods or data.

There are no variable static members.  This may seem a bit harsh to Java and
C++ programmers, but there is a good reason.  \Xten, unlike these languages,
has been intended from the start to make full use of a multi-processor system. 
By restricting the data associated with a class to that which is put in place
at the time the class is loaded, the \Xten{} runtime is free to copy the loaded
class from {\tt Place} to {\tt Place} as needed, with every {\tt Place} always
seeing a consistent view of the class.

\item{\bf line \xlref{ship-b}{3}:} a public {\tt val} instance member.\\
An instance field that is a {\tt val} names a constant that
must be computed either as part of its declaration or by the constructor.
The whole point of an {\em
instance} {\tt val} is that it is a {\em per instance} constant, not a {\em per
class} one, so the constructor would normally use information from its
arguments---the values that make a particular instance what it is---to compute
the  field's value.

We've shown this instance member as {\tt public}.  As with classes, we could use nothing
here, just ``{\tt val name:String}''.  The member could then only be used by
classes in exactly the same package (which normally would mean {\em from the
same directory}). 

There are two other access keywords available: {\tt protected} and
{\tt private}.  A {\em protected} member is usable by any  class in the same package,
and by any class that {\em extends} the class in which the member is declared.
See section \ref{sec:inh}, page \pageref{sec:inh}. A {\em private member} is
just that: usable only in the class it is declared.  Thus, from most widely
available to least:
\begin{quote}{\tt public}, {\tt protected}, {\em no spec at all}, {\tt private}\end{quote}

\item{\bf lines \xlref{ship-c}{4}-\xlref{ship-d}{5}:} {\tt var} instance members.\\
Here are two instance fields that are variables.  Both
are private for good reason: {\tt onBoard} is the number of valid entries in
the {\tt passengers} array, and the last thing we want to do is to allow the
general public complete, direct access to either the array or its vital
statistics. {\tt onBoard} will be updated
every time we add a passenger, or throw one overboard. {\tt passengers} will be
reset when we are at capacity and want to add another passenger.  We 
have guaranteed that code responsible is part of {\em this} class.

\item{\bf lines \xlref{ship-ctor}{7}-\xlref{ship-endctor}{13}:}  Constructors.
We show a couple of constructors here.  We chose to
make both public, but we need not have done so: like any other method, we can
restrict the availability of a constructor.  We could have made one private and
one protected if that is really what we needed, unusual though that would be.

Notice the way that the first constructor invokes the second via {\tt this}.
{\em It is important that this call be the first executable line in the constructor's body.}

\item{\bf line \xlref{ship-resize}{15}:} {\tt static def resize}\\
Static methods are ``class methods'', in the sense that they are properties
of the class, not properties of an instance of the class.
Static methods have access to all of the static fields and methods
of the class, but only an instance of the class may refer to instance fields and
methods.  

This method has the default access: no access keyword ({\tt public}, {\tt
  protected}, or {\tt private}) is specified. Default access means that only
things in the exact same package have access. Now we didn't specify a package
for {\tt Ship}, so it is in the default package. If {\tt DieselShip} is
another class and is also in the default package, it can invoke {\tt resize}.
The syntax it would have to use is {\tt Ship.resize(\ldots)}. Inside {\tt
  Ship} itself, you don't need the ``{\tt Ship.}'' part. {\tt resize(\ldots)}
by itself will do. The moral: static methods {\em really belong to the class
  in which they are defined.}

\item{\bf lines \xlref{ship-defs}{17}-\xlref{ship-enddefs}{19}:} instance
  method declarations with varying access\\ Here we have three instance
  methods with varying degrees of protection. We'll allow anyone to add a
  passenger, because if the boat threatens to sink under the weight, we can
  always say no. We're a little more careful about who can throw a passenger
  overboard: this is pretty drastic, and different kinds of ships may have
  different policies. Finally, respecting its customers' privacy, {\tt Ship}
  reserves to its package and its subclasses the right to prescribe how
  instances show who is on board.

\item{\bf line \xlref{ship-main}{21}:} {\tt main} again\\ You've seen {\tt
  main} enough times that all we feel we have to say here is that you don't
  need one except in a class that will begin a program's execution.

\item{\bf line \xlref{ship-farecls}{23}:} A static class field is a value
  associated with the class, like any other, except that it names a class.
  Within the body of {\tt Ship}, you write ``{\tt new FareClasses()}'' in
  order to construct an instance. From a class {\em not} declared within {\tt
    Ship}, you would need to write ``{\tt new Ship.FareClasses()}''. In this
  example, we opted to give the class the default accessibility: same package
  can use, others not. We could have chosen any of the other three.

\item{\bf line \xlref{ship-galley}{24}:} A class that is an instance
  member.\\ 
\begin{finepoint} {\tt Galley} is an example of what is often called an
    ``inner class''. If you are not familiar with the notion, we'll give a
    brief discussion of it here. To be honest, though, it is something you are
    not likely to need very often, at least as an \Xten{} programmer, so you
    might wish to skip over this.

%%INNER%% {\tt Galley} is an inner class. That means first and foremost that only an
%%INNER%% instance of {\tt Ship} can refer directly to it. In line
%%INNER%% \xlREMOVED{ship-mkgalley}{32}, for example, a private instance field is created by
%%INNER%% calling {\tt Galley}'s constructor. If {\tt myBoat} is a {\tt Ship}, it can
%%INNER%% create other {\tt Galleys} by executing ``{\tt myBoat.new Galley();}''. This
%%INNER%% is a pretty reasonable locution if you think about it: it says: ``use the {\tt
%%INNER%%   new} that knows about {\tt Ship} to create this {\tt Galley}''.
%%INNER%% 
%%INNER%% Every instance of {\tt Galley} is associated, then, with an instance of
%%INNER%% {\tt Ship}---its guardian angel.  The {\tt Galley} can refer to the 
%%INNER%% instance members of its guardian angel.  Line 26 shows how this is done.
%%INNER%% The expression ``{\tt Ship.this}'' that you see there is \Xten's syntax
%%INNER%% for this {\tt Galley}'s guardian angel.
%%INNER%% The total number of plates the {\tt Galley} needs is computed by
%%INNER%% multiplying the {\tt Galley}'s instance field {\tt plates\-Per\-Person} 
%%INNER%% by the field {\tt onboard} of its guardian angel. 
%%INNER%% 
%%INNER%% The tight linkage to guardian angels aside, inner classes are classes
%%INNER%% like any other.   
%%INNER%% 

An instance of an inner class, like \xcd`Galley`, depends on an instance of
its outer class, here \xcd`Ship`, for its very existence.  It doesn't make
sense to have a galley without a boat, since a galley is defined as ``a
kitchen in a boat''.\footnote{Or an airplane or a camper, as is known to
pedants, pilots, vacationers, and other such people of uncertain character.}   
So, on line \xlref{ship-newgalley}{34}, when we construct a \xcd`Galley`, we
have to give it a \xcd`Ship` to build the galley in. That's why we write the
constructor as \xcd`this.new Galley()` rather than simply 
\xcd`new Galley()`.\footnote{Honestly, \xcd`new Galley()` would work too; X10
will use 
\xcd`this` automatically, just like it does for method calls.}


Every instance of {\tt Galley} is associated, then, with an instance of
{\tt Ship}.  
The {\tt Galley} can refer to its \xcd`Ship`, and to the 
instance members of its \xcd`Ship`.  Line \xlref{ship-galleyplate}{27} shows how
this is done. 
The expression ``{\tt Ship.this}'' is \Xten's syntax
for this {\tt Galley}'s \xcd`Ship`. 
The total number of plates the {\tt Galley} needs is computed by
multiplying the {\tt Galley}'s instance field {\tt plates\-Per\-Person} 
by the field {\tt onboard} of its ship. 

The tight linkage to ships (or whatever the required container is) aside,
inner classes are classes like any other.   


In our discussion of inheritance hierarchies (see ``extra credit:''
on page \pageref{ssec:pinc}), we'll say some more
about how instances of inner classes can be used.
For very subtle reasons, we can't initialize \xcd`galley` when the \xcd`Ship`
is being constructed.  We'll explain this more in \Sref{NoThisEscapes}.  

\end{finepoint}
\end{description}

We wish we could tell you that this is all there is to a class declaration.  It isn't, but it
is more than enough to get you going.

\subsection{Inheritance And More}\label{sec:inh}

\Xten{} supports simple (single) inheritance: all \Xten{} classes constitute a
tree, in which the root is the class {\tt x10.lang.Object}, and a class {\tt
  Parent} is the parent of a class {\tt Child} when {\tt Child}'s declaration
says that it extends {\tt Parent}:
\begin{quote}{\tt class Child extends Parent \{ ... \}}\end{quote}
For example:

%%START X10: DieselShip.x10 diesel
\begin{verbatim}
 1    public class DieselShip extends Ship {
 2      public val fuelTankCapacity: Double; 
 3      public var gallonsRemaining: Double;
 4      
 5      public def this(name:String, maxPsgrs:Int, tankSize:Double) {
 6         super(name, maxPsgrs);
 7         fuelTankCapacity = gallonsRemaining = tankSize;
 8      }
 9      public def throwOverboard(name: String, why: String) {
10        super.throwOverboard(name, why);
11        /* more stuff here */
12      }
13      /* ... more methods can go here */
14   }
\end{verbatim}
%%END X10: DieselShip.x10 diesel

A {\tt DieselShip} is a {\tt Ship} whose method of propulsion we know.  It is
in that sense that {\tt DieselShip} is an extension of {\tt Ship}: it is a
{\tt Ship} and more.  ``{\tt DieselShip} {\em is a direct
subclass of} {\tt Ship}'' is another common way of saying the same thing,
and we'll use both.
 \begin{quote}
More generally, a class {\tt S} is a {\em subclass} of another class {\tt C} if
there is a chain of classes {\tt S}, {\tt S$_{1}$}, ..., {\tt S$_{n}$}, {\tt C}
each of which is a {\em direct} subclass of the next.
\end{quote}
In declarations in which no {\tt extends}
clause appears, like those for the classes {\tt MontePi} or {\tt HelloWorld}, 
the class extends {\tt x10.lang.Object} by default.

Because it extends {\tt Ship}, an instance of {\tt DieselShip} will have
instance fields {\tt name}, {\tt passengers}, and {\tt on\-Board}, and methods
{\tt resize}, {\tt add\-Pas\-sen\-ger}, and so on.  They are said to be {\em
inherited from {\tt Ship}}.  The call to {\tt super} in line 6 invokes the
super-class {\tt Ship}'s constructor to make sure that the instance of that
class on which the {\tt DieselShip} instance is built has these fields correctly
initialized. 

The call to {\tt super} must be the first executable statement in the
constructor, just as shown here.  If your code does not call {\tt super}
explicitly like this, the compiler will insert a call to {\tt super()}---notice:
the call is made with no arguments---so in this situation the compiler must be able to
find a constructor for the super-class with no arguments. The good news is that
if you provide no constructor at all in your class declaration, the compiler will
create this default constructor (needs no arguments, does nothing) for you.

Notice that {\tt DieselShip} provides its own
version of the method {\tt throw\-Over\-board}, so rather than inheriting that
method, it {\em overrides it}.  The expression {\tt super(name,why)} in the
body of the override invokes the parent's version of {\tt throw\-Over\-board}.
\begin{quote}
Be careful! While every instance of {\tt DieselShip} inherits the private members
of {\tt Ship}, {\em only the methods already in {\tt Ship} have access
to them.} Thus, {\tt DieselShip} can affect the array {\tt passengers} only by
invoking methods like {\tt addPassenger} that it inherits from {\tt Ship}
and has permission to call.
\end{quote}
Subclasses are free to override any of the methods they inherit.  The only
limitation is that the subclass cannot make a method more narrowly available
(\eg{} {\tt protected} rather than {\tt public}) than it is in its parent.

If a method {\tt m} expects an argument that is an instance of a class {\tt Ship},
then it is okay to call {\tt m} with an
instance of {\tt DieselShip}.
Suppose, for example, we have a class {\tt CruiseLine} whose instances have
fleets of ships that sail the world with lots of passengers.  A {\tt CruiseLine}
instance might want to add a {\tt Ship} to its fleet, so it has a method
\begin{verbatim}
     public def addShip(ship: Ship) { ... }
\end{verbatim}
Because a {\tt DieselShip} {\em is} a {\tt Ship}, we can call {\tt addShip}
with a {\tt DieselShip} as the argument.

A similar rule holds for return values.  If a method is supposed to return a 
{\tt Ship}, it can return one that just happens also to be a {\tt
DieselShip}.

\finepoint{\label{ssec:pinc}
This gets us to the point where we can give an example of how an instance
of an inner class might find itself being used outside the class in which it
was declared.  
Subclassing gives us a way to make an otherwise invisible class visible:   
\begin{itemize}
\item Suppose
there is a public class {\tt Kitchen}, and suppose that in our declaration of
{\tt Ship} we had written ``{\tt protected class Galley extends Kitchen \{ ... \}}'' .
\item Let's add to {\tt Ship}   a public method {\tt getGalley()} that returns 
the private field {\tt galley}.
\item If a  class has a method whose signature is ``{\tt isWellEquipped(k: Kitchen)}'',
and if {\tt anArk} is a {\tt Ship}, then the call
\begin{verbatim}
     isWellEquipped(anArk.getGalley())
\end{verbatim}
is perfectly okay, because a {\tt Galley}, wherever it came from, is a {\tt Kitchen}.
In effect, {\tt galley} has ``leaked out'' to stand on its own in whatever class the
method {\tt isWellEquipped} is declared.
\end{itemize}

 Bottom line: you need to be 
a {\tt Ship} to construct a {\tt Galley}, {\em but not necessarily just to
use one.} 

}

\subsection{Concrete classes versus wishful thinking}
Our declaration of {\tt Ship} is an example of wishful thinking, because
if you are the least bit realistic about it, a ship without any means of
propulsion isn't going to get you very far.  On the other hand, there
are a large number of very different propulsion systems:  we've got {\tt DieselShip}
already, and {\tt SteamArk}, {\tt SailArk}, and {\tt RowedArk} are also good possibilities.

So, {\tt Ship}, lacking, as it does, any means of getting anywhere, is, in a
very clear sense, not a {\em concrete} class: you will never see an instance
that is not actually an instance of some subclass. Folks just don't build
boats that go nowhere, excepting of course for certain tourist-trap seafood
restaurants and the occasional glorified liferaft.

But that is not to say that {\tt Ship} can't capture a lot of concrete 
information--passenger lists, fare schedules, itinerary, and so on--that
all ships share, no matter their means of propulsion.  That makes {\tt Ship} a
good example of what is  
called an {\em abstract} class: a class that one does not intend to
instantiate, but 
which is the ultimate parent of a set of classes that one does.

To convert {\tt Ship} to being an abstract class, one just adds the keyword
``{\tt abstract}'' to its declaration.  We'll call this version
\xcd`AbstractShip`.


%%START X10: AbstractShip.x10 absship
\begin{verbatim}
1  abstract public class AbstractShip {
2      public static DEFAULT_SIZE = 500;
3      public  val name: String;
4      // and all the other code as from Ship.x10
\end{verbatim}
%%END X10: AbstractShip.x10 absship


An abstract class is permitted to declare method signatures that it itself
does not 
implement, but that any of its subclasses that
are not themselves abstract {\em must} implement.  Here is an example. 
Notice that there is no function body: there is just a trailing semicolon.
%%START X10: AbstractShip.x10 absshipabs
\begin{verbatim}
1    public abstract def costPerPassengerMile(): Double;
\end{verbatim}
%%END X10: AbstractShip.x10 absshipabs

It is clearly unreasonable to talk about cost per mile without knowing how
your ship is  propelled, and it is equally reasonable to demand that if you
want to really function as a ship, you must provide this information.


A concrete class like {\tt DieselShip}
will have to implement {\tt cost\-Per\-Passen\-ger\-Mile}.
In {\tt DieselShip}, its declaration will
look like:

%%START X10: DieselShip2.x10 costPerPassengerMile
\begin{verbatim}
1    public def costPerPassengerMile(): Double = 0.018;
\end{verbatim}
%%END X10: DieselShip2.x10 costPerPassengerMile
The keyword {\tt abstract} does {\em not} appear, because this version of
\xcd`costPerPassengerMile` is not abstract. 
The function body appears instead.

\section{Interfaces: More Abstract Than Abstract}\label{sec:mata}

While {\tt Ship} may not be as complete as a real ark should be, it can implement a
lot of important functionality that all its progeny can use directly: a passenger list is a 
passenger list, no matter what else is different!  There are situations though, where all
that one wants to specify is functionality that has no common implementation, but
that a variety of classes will implement, each in its own way.  This sort of specification
is called an {\em interface}.  
\begin{quote}
Those of you who are coming from a background in C++ will want to work carefully through
our discussion of interfaces, because they constitute the closest approximation
to multiple inheritance that \Xten{} provides.
\end{quote}

A nice example of an interface is {\tt Arithmetic}:\footnote{For extra
slickness, we could have programmed it so that the operations use the usual
symbols: \xcd`a*b` instead of \xcd`a.multiply(b)`.  See \Sref{Operators} for
how to do this.}
%%START X10: Arithmetic.x10 arith
\begin{verbatim}
1  public interface Arithmetic[T] { 
2     def add(t: T): T; 
3     def subtract(t: T): T;
4     def multiply(t: T): T;
5     def divide(t: T): T;
6     public static VERSION = "1.1";   
7  
8  }
\end{verbatim}
%%END X10: Arithmetic.x10 arith

In detail: 
\begin{description}
\item[line {\bf \xlref{arith-decl}{1}:}] 
Like a class, a top-most interface is either public or accessible only in the
package of which it is a part. An interface declared within the body of a
class is {\em always} static (it is never per-instance and cannot refer to
instance members or methods in the surrounding context), but it can have any
access limit: public, protected, default, or private.

The ``{\tt [T]}'' following {\tt Arithmetic} says that the interface is
parameterized by a type {\tt T}. We've seen this before, in the argument {\tt
  args} for {\tt main()}, whose type is {\tt Array[String]}. The lingo of the
trade is that {\tt Arithmetic} is a {\em generic interface}, and {\tt Array}
is a {\em generic class}. There is nothing sacred about a generic class or
interface being parameterized by one type: one can use as many as needed:
``{\tt class X[T1, T2, T3,...] \{...\}}''.


\item{\bf lines \xlref{arith-startdefs}{2}-\xlref{arith-enddefs}{6}:} An interface may only declare abstract methods, so (unlike the story
with abstract classes), you don't need the keyword {\tt abstract}.  What line 2 says is that
\begin{quote}
if a class {\tt A} implements the interface {\tt Arithmetic[B]},\\
if {\tt a} is an {\tt A}, and {\tt b} is a {\tt B}, \\
then the expression {\tt a.add(b)} produces a result that is a {\tt B}.
\end{quote}
In real life, we might have two classes {\tt Monomial} and {\tt Polynomial} to 
implement polynomials with {\tt Double} coefficients.  It would be natural, then,
for {\tt Monomial} to implement {\tt Arithmetic[Polynomial]}: when you add a monomial
to a polynomial, and the result is a polynomial.

When you add two or multiply \xcd`Polynomial`s, you get another
\xcd`Polynomial`.  So, we want \xcd`Polynomial` to implement
\xcd`Arithmetic[Polynomial]`.  This isn't a weird recursive thing, though it's
got the imposing-sounding name of ``F-bounded quantification'' (which most
programmers should remember purely for the sake of sounding erudite at job
interviews).
It just  says that polynomials can perform arithmetic with polynomials and the
results will be polynomials -- which is just what you'd expect. 

X10 has a few standard interfaces that are often used in this F-bounded way.  
For example, there's \xcd`Comparable[T]`.  A class implementing
\xcd`Comparable[T]` can be compared to a value of type \xcd`T`.  Most of the
time, you want to compare things of the {\em same} type.  So, classes tend to
implement \xcd`Comparable[`themselves\xcd`]`. 

Another important point is that the methods declared by an interface are always
{\tt public}.  So even though the keyword {\tt public} does not appear here, it
is implied.  We'll say more about this when we discuss how you use interfaces.

\item {\bf line \xlref{arith-enddefs}{6}:} Interfaces may also declare static
      constants.  That's the only kind of constant they can declare --
      \xcd`val` means \xcd`public static val` in an interface, just the way
      that \xcd`def` means \xcd`public def`.  

In fact, some interfaces do nothing but declare constants: 
%%START X10: MathConst.x10 mathconst
\begin{verbatim}
1  public interface MathConst {
2    val PI = 3.1415926535897932384626433;
3    val E  = 2.7182818284590452353602874;
4    val GAMMA = 0.57721566490153286060651;
5    val SOFA = 2.219531668871; 
6    val DOUBLE_DIGITS = 15.95; 
7    val DOUBLE_E_MAX  = 307.95;
8  } 
\end{verbatim}
%%END X10: MathConst.x10 mathconst

Some of these, like \xcd`PI` and \xcd`E`, are useful mathematical constants
that will be the same for all X10 programs.   
Others, like \xcd`DOUBLE_DIGITS` and \xcd`DOUBLE_E_MAX`, might change from
computer to computer: these are the right values for IEEE double-precision
numbers, but they would be different on a machine that used quadruple
precision for the \xcd`Double` type.


Another common use is to give names to the legal values for
the arguments of one or more of the methods.  Here are some names for some
standard colors:
%%START X10: Color.x10 color
\begin{verbatim}
1  interface Color {
2    val RED = 0xFF0000;
3    val BLUE = 0x0000FF;
4    val PEACOCK = 0x33A1C9;
5  }
\end{verbatim}
%%END X10: Color.x10 color
We can refer to these as \xcd`Color.RED` and so on: 

%%START X10: Discoloration.x10 discoloration
\begin{verbatim}
1  class Discoloration {
2    static def example() {
3      Console.OUT.println("RED = " + Color.RED);
4    }
5  }
\end{verbatim}
%%END X10: Discoloration.x10 discoloration

If we're inside of a class that implements \xcd`Color`, we can leave off the
\xcd`Color.` and refer to \xcd`Color.RED` as just plain \xcd`RED`:

%%START X10: Colorizer colorizer
\begin{verbatim}
1  public class Colorizer implements Color {
2     static def example() {
3        Console.OUT.println("Red = " + RED); 
4     }
5  }
\end{verbatim}
%%END X10: Colorizer colorizer

A class implementing \xcd`Color` has to use \xcd`Color`'s value for \xcd`RED`.  
Would you 
want someone to implement the interface with its own value for {\tt RED}? That
would ruin code reuse. 


\end{description}

Finally, one interface can extend another. They can do so better than classes
can, even: one interface can extend several other interfaces, but a class can
only extend a single other class. 

%%START X10: ExtendingInterfaces.x10 extendinginterfaces
\begin{verbatim}
1  interface Colored {
2    def color(): Int;
3  }
4  interface Shape{}
5  interface Shaped {
6    def shape():Shape;
7  }
8  interface ColorForm extends Colored, Shaped {}
\end{verbatim}
%%END X10: ExtendingInterfaces.x10 extendinginterfaces


One could
say ``{\tt ColorForm} is a sub-interface of {\tt Colored} and {\tt Shaped}'', though some
people think that 
``{\tt ColorForm} {\em refines} {\tt Colored} and {\tt Shaped}'' sounds better. 

\subsection{How interfaces get used}

Let's implement a class of fractions, \xcd`Frac`.  Fractions are useful
because you can express some common values exactly.  {$\frac{1}{3}$} is a
fraction that is exactly one-third.  The 
closest that X10 can come to this value is 
\xcd`0.3333333333333333`, which is pretty close, but off by more than
{$3\times 10^-16$}.  If you're not careful, these little errors can really
mess up your computation; the discipline of scientific computing teaches how
to minimize the problem.  Using fractions, which have exactly the right value
with no round-off error, is one way.

The syntax 
\begin{verbatim}
class Frac implements Arithmetic[Frac] { ... }
\end{verbatim}
asserts that the class {\tt Frac} implements all of the methods 
declared in the {\tt Arith\-me\-tic[Frac]} interface--with exactly the 
argument types and return types shown there.
In plain English, this is saying that you can add, subtract, multiple and divide one
{\tt Frac} by another and get an {\tt Frac} result.


While {\tt Frac} is one candidate for the class that
appears as the {\tt T} in the {\tt Arithmetic[T]}, we'd probably also like to
add {\tt Frac}s to other kinds of numbers -- at least \xcd`Double`s, and if we
were doing a full-strength version we'd have \xcd`Long`s, \xcd`Int`s, and
perhaps other types as well.  


%%START X10: Frac.x10 fracdef
\begin{verbatim}
1  class Frac implements Arithmetic[Frac], 
2                        Arithmetic[Double] 
3  {
\end{verbatim}
%%END X10: Frac.x10 fracdef



At first blush, this is a daunting declaration: think how many methods must be
implemented, especially for the library version!
But the truth is that none of these methods is all that involved, so it is just a matter of
some (necessary!) drudgery.  Here the end more than justifies the means.

The implementation is straightforward.  The constructor for
\xcd`Frac`s always puts fractions in lowest terms, so that 
\xcd`new Frac(10,20)` is the same as  \xcd`new Frac(1,2)`: 

%%START X10: Frac.x10 fracctor
\begin{verbatim}
1    public def this(var n:Int, var d:Int) {
2      if (d < 0) { n = -n; d = -d; }
3      if (d == 0) throw new Exception("Division by zero!");
4      var gcd : Int = gcd(n,d);
5      this.n = n / gcd;
6      this.d = d / gcd;
7    }
\end{verbatim}
%%END X10: Frac.x10 fracctor

Fraction arithmetic uses the familiar rules.  Since the constructor reduces
everything to lowest terms, we don't need to do any extra work to do so: 

%%START X10: Frac.x10 fraconfrac
\begin{verbatim}
1    
2    public def add(f:Frac) = new Frac(n * f.d + f.n * d, d * f.d);
3    public def negate() = new Frac(-n, d);
4    public def subtract(f:Frac) = this.add(f.negate());
5    public def multiply(f:Frac) = new Frac(n * f.n, d * f.d);
6    public def divide(f:Frac) = new Frac(n * f.d, d * f.n);  
7    
\end{verbatim}
%%END X10: Frac.x10 fraconfrac

The \xcd`Frac`-to-\xcd`Double` versions are easily implemented using 
a method to convert a fraction to a \xcd`Double`: 

%%START X10: Frac.x10 fracondouble
\begin{verbatim}
1    public def asDouble():Double = (n as Double) /(d as Double);
2    public def add(dbl: Double) = this.asDouble() + dbl;
3    public def subtract(dbl: Double) = this.asDouble() - dbl;
4    public def multiply(dbl: Double) = this.asDouble() * dbl;
5    public def divide(  dbl: Double) = this.asDouble() / dbl;
\end{verbatim}
%%END X10: Frac.x10 fracondouble


Notice that in this class, the method {\tt add} is explicitly declared to be
{\tt public}.  While {\tt public} is all there is in an interface declaration,
it is {\em not} all there is in a class declaration, so you must be specific
here.  The compiler will complain if you do not.

Now suppose that somewhere there is a class that has a method ``{\tt doIt(x:
Arith\-me\-tic[Frac])}''. Since {\tt Frac} implements {\tt Arithmetic[Frac]},
we can use a {\tt Frac} as the argument to {\tt doIt}, \eg:

%%START X10: Frac.x10 fracdoit
\begin{verbatim}
1      doIt(new Frac(1,3));
\end{verbatim}
%%END X10: Frac.x10 fracdoit



We end this section with some ``side bars'' that somehow didn't quite fit in 
the flow of the rest of this section, but are worth a moment's thought.

\subsubsection{ Anything you can do\ldots}\label{subsub:type:anything}
The 1946 Broadway musical {\em Annie Get Your
Gun} has a wonderful song that begins ``Anything you can do, I can do better. 
I can do anything better than you.''  \Xten{} has an operator that says the
same thing much more briefly: if {\tt T} and {\tt U} are two types, then the
expression {\tt T<:U} is a boolean that, if true, means that a value of type
{\tt T} may be used anywhere a value of type {\tt U} may be, or in other words,
that {\tt T} is either {\tt U} itself or a {\em subtype} of {\tt U}. 

\bard{This isn't a good explanation of \xcd`<:`, and I think that the
explanation goes elsewhere and should be used a lot more.}
One use of the ``operator'' {\tt <:} is in {\tt val} declarations. 
Saying ``{\tt val t <: T}'' declares {\tt t}'s type to be a subtype of {\tt T}. 
That way, if {\tt t} is initialized in one of several arms of an {\tt if}
statement, differing types may appear in the various assignments, so long as all
are subtypes of {\tt T}.  Also users of {\tt t} should only count on getting a
{\tt T}.

For example, if we have an interface like \xcd`Caloried`, which tells how many
calories something has: 
%%START X10: NeedsNutrients.x10 needsnutrCaloried
\begin{verbatim}
1  interface Caloried {
2    def calories():Int;
3  }
\end{verbatim}
%%END X10: NeedsNutrients.x10 needsnutrCaloried

We can write a class which manages a list of ingredients, and can count their
calories.  Note that this wouldn't even compile without the constraint 
on \xcd`T`.  It is only that constraint that gives us permission to 
call the \xcd`calories()` method in line
\xlref{needsnutrRecipe-usecalories}{7}.  
%%START X10: NeedsNutrients.x10 needsnutrRecipe
\begin{verbatim}
1  class Recipe[T]{T <: Caloried} {
2     val ingredients: List[T] = new ArrayList[T]();
3     public def add(t:T) { ingredients.add(t); }
4     public def totalCals() {
5        var s : Int = 0;
6        for(ingredient in ingredients) 
7            s += ingredient.calories(); 
8        return s;
9     }
10  }
\end{verbatim}
%%END X10: NeedsNutrients.x10 needsnutrRecipe

A class, like \xcd`Flour`, can implement \xcd`Caloried` in the ordinary way: 
%%START X10: NeedsNutrients.x10 needsnutrFlour
\begin{verbatim}
1  class Flour implements Caloried {
2    private val name: String, cal:Int ;
3    public def this(name:String, cal:Int) 
4      { this.name = name; this.cal = cal; }
5    public def calories() = this.cal;
6  }
\end{verbatim}
%%END X10: NeedsNutrients.x10 needsnutrFlour

Then, we can use \xcd`Flour` in a \xcd`Recipe`: 

%%START X10: NeedsNutrients.x10 needsnutrExample
\begin{verbatim}
1  static def example(): void = {
2    val flours <: Recipe[Flour] = new Recipe[Flour]();
3    flours.add(new Flour("1 cup bread flour", 495));
4    flours.add(new Flour("1 cup whole-wheat", 520));
5    flours.add(new Flour("1 cup rye", 360));
6    assert flours.totalCals() == 1375;
7  }
\end{verbatim}
%%END X10: NeedsNutrients.x10 needsnutrExample



(This is different from writing a class which simply manages a list of
\xcd`Caloried` values.  The difference doesn't matter much in this example,
but it would matter if the constraint were \xcd`T <: Arithmetic[T]` -- in
which case we couldn't even talk about a \xcd`List[Arithmetic]`, since X10
insists on knowing what type of thing the arithmetic is being done on.)


\subsubsection{ Why interface methods are public:}  Suppose a class {\tt A} implements an
interface {\tt I}, and suppose that {\tt B} is a subclass of {\tt A}.  Because
an instance of {\tt B} might appear anywhere an instance of {\tt A} is 
expected, we see that {\tt B} {\em must also implement} {\tt I}. This is one
reason why the methods in an interface must be declared {\tt public}.

There is another reason that is more pragmatic than principled.  The point of an
interface is to tell the outside world what some object is capable of doing.
A {\tt private} method, on the other hand, is just a {\em hidden} means to
accomplish a class's goals.  Why should anyone care whether a class implements
a method that can only be called from within that class?

An interface itself may declared with any access limit that makes sense in the
context: if the interface is being declared inside a class (and it definitely
can be), then protected, default and even private access make sense.  For
example, a private interface might describe the type of an argument of
one or more of the class's private methods, and there might be several static
or inner classes declared inside the class that implement that interface: not 
a common situation, but not unreasonable.

\subsubsection{ A matter of convention: }
There is a sort of convention that interface names should begin with ``{\tt
I}'',  Suppose for example, that we had used {\tt INumber}, instead of {\tt
Arithmetic}, for our interface name.  We could then have declared  a class
named {\tt Number} that implements it.  A matter of taste of course, but not an
unreasonable convention, particularly in situations where there is a
``standard'' or ``canonical'' implementation of the interface.  A variation on
this theme is to append ``{\tt Impl}'' to the name of an interface to create
the name of a concrete class that implements it: \eg{} interface {\tt
XmlParser} is implemented by {\tt XmlParserImpl}.  You will see both
conventions widely used.

\subsubsection{ For C++ programmers: } 
We said that interfaces provide a sort of
``multiple inheritance.''  What we meant is that a class may declare that it
implements any number of interfaces.  That is certainly reasonable: all you are
saying is that the class provides several named sets of methods.  It is, of
course, a weaker notion than multiple inheritance, because,  when a class
implements an interface, no instance members or methods are actually inherited:
all of the code is in the class, or is in one of its ancestors up the
inheritance chain.

\section{Giving Up Inheritance To Get Efficiency}\label{sec:guitge}
\subsection{The Problem}

Supporting inheritance does not come without costs when a program runs.
Consider the class {\tt DblDbl}, a model of innocence:
\begin{verbatim}
 1 public def class DblDbl {
 2    public val x: Double;
 3    public val y: Double;
 4    public def this(a: Double, b: Double) {
 5       this.x = a; this.y = b;
 6    }
 7    public def normSq() = x*x + y*y;
 8 }
\end{verbatim} 
Let ``{\tt public def doIt(p:DblDbl) \{...\}}'' be a method declared in some
other class, and suppose that in the body of {\tt doIt}
we find the call {\tt p.normSq()}.  What does this call really cost?
The problem is that the value passed to {\tt doIt} for the argument {\tt p}
might really be a reference to an instance of some proper subclass
of {\tt DblDbl}. We only know the precise class of {\tt p} once we start
executing the body of {\tt doIt}.  Because {\tt p}'s class can have its own
implementation of {\tt normSq}, making sure we are calling the correct
{\tt normSq} means looking up its address in {\tt p}'s class's 
descriptor {\em each time the call is executed}.

The moral is that an instance {\tt obj} of a class that may have subclasses
must carry with it a reference to the actual class of the instance {\tt obj}.
Applying this to instances of {\tt DblDbl}, we see that in addition to
the 16 bytes required to store the two instance fields, we have to keep
at least 4 bytes for the reference that allows us to find its class.
If there are only a couple of hundred {\tt DblDbl} instances around, and
if we only call {\tt normSq} a couple of thousand times in any
given run of the code, who cares?  But what if we had an array a million
{\tt DblDbls} and were traversing it repeatedly calling methods that
some subclass might override?

Think about it.  The million {\tt DblDbls} have
16 megabytes of data that we're really interested in, but are costing us
at least another 4 megabytes or more for type information.  A million
calls to {\tt normSq}, moreover, require a million method lookups that
we'd really love to avoid.  Now, if
there really are subclasses of {\tt DblDbl}, and they really need their own
implementation of {\tt normSq}, then fine: we'll gladly pay the cost in time
and space, because we are getting something in return. But if not? 

These considerations are one reason why a performance-oriented language avoids
using the normal class hierarchy for primitive types like 32-bit integers: no language
that expects high performance can allow a 4-byte integer to consume 8 bytes or
more and to be accessed indirectly.  

There are two different approaches one can
take, as a language designer, to solving this problem.  One is to keep the
primitive value types to a handful close to the physical reality of the CPU
and treat them specially in the language itself.  This is what Java chose to do.
The other approach is to keep the implementations of the primitive
types as close to that of classes as possible, which is what \Xten's designers
did.

The question \Xten{} had to face is: how much of what classes give
us can we keep without incurring any substantial overhead? 
There are two sets of  issues here: serial issues like ``don't waste space'',
and ``don't waste time looking stuff up'', and distribution issues like ``don't
create the possibility of inconsistencies between what one processor thinks the
value of the ``thing'' is and what another thinks, so that communication
between the processors is required to maintain storage integrity''.

\subsection{The Solution}\label{sec:ecs}
To begin with, we have already seen that we need instances whose types are
exactly what they are declared to be: no inheritance.  To achieve the storage
integrity, all of the instance fields must be {\tt vals}, so that an
instance once constructed is immutable and hence can be copied freely
whereever its value is useful.  
\Xten{} calls these things ``{\tt struct}s'', to distinguish them from
classes.
\footnote{
This is a rather unfortunate choice of terminology, because \Xten{} structs  bear about the same
relation to the familiar {\em C} {\tt struct} as \Xten{} classes do.
}  
\Xten{} made one other critical decision: when an instance of a
struct appears as a field or is assigned to an identifier, it is the
data in the instance fields that is stored, and not a reference to that data.
For primitives like {\tt Int}, this is a critical efficiency consideration:
what's stored for an {\tt Int} its value, not to the address of its value, so
you don't have ``follow a pointer'' to get to the value.

Struct declarations are essentially the same as {\tt class} declarations.
For example, we can
change the class {\tt DblDbl} to be a struct just by replacing the keyword
``{\tt class}'' with ``{\tt struct}'' in its declaration:
\begin{verbatim}
 1  public struct DblDbl {
 2    public val x: Double;
 3    public val y: Double;
 4    public def this(a: Double, b: Double) {
 5       this.x = a;  this.y = b
 6    }
 7    public def normSq() = x*x + y*y;
 9 }
\end{verbatim}
A few comments:
\begin{description}
\item{\bf line 1:} The accessibility rules for a struct are the same as for a
class.  We chose ``{\tt public}'' for the example, but default, {\tt protected} 
and {\tt private} access may be assigned as appropriate.

Structs may be generic---that is, they may, like {\tt Array}, have type
parameters---although this one does not.

\item{\bf lines 2 and 3:}  As we noted earlier, the instance members of a struct
are always {\tt val}s. For that reason, you don't need to say ``{\tt
val}'' as we did here.  It is okay if you do. If you are not sure whether in
the long run you are going to want this type to be a struct or a class,
throwing in the {\tt val} makes it easier to go back and forth.

We've only shown instance members in this example, but
structs can have static member fields as well.
Indeed, one can also define classes, interfaces, and structs in the
body of a struct, just as one can in a class.
\item 
 
\item{\bf line 7:} Unlike their C namesakes, \Xten{} structs can declare
methods, both instance methods and static methods. We've shown a public
method here, but both private and default access can be used. Structs
{\em never} have subclasses, so ``protected,'' which just broadens the default
access by allowing access to subclasses, adds nothing in this context.
\end{description}

The keyword {\tt new} is {\em not} required when creating
 a new instance of a struct---in contrast to the case for a class.
Here is an example that assigns a {\tt DblDbl} value:
\begin{verbatim}
        var p: DblDbl = DblDbl(1.2, 3.4);
\end{verbatim}
The ``{\tt new}'' is gone from the right-hand side.
If you put the ``{\tt new}'' in, though, the compiler will happily accept it.

{\bf Important: } One important difference between struct and class
implementation is the treatment of ``{\tt ==}''.   If {\tt s1} and {\tt s2} are
two instances of the same struct, then ``{\tt s1 == s2}'' is true exactly when
when their instance fields have identical values.  If, for example, 
{\tt s1} and {\tt s2} are {\tt DblDbls}, then {\tt s1==s2}
is equivalent to {\tt (s1.x==s2.x) \& (s1.y==s2.y)}.  Remember that for
instances of a class, ``{\tt ==}'' tests for references to same instance,
not equality of values. But: there are {\em no} references to instances of a
struct, just the instance data itself.  Since the data is all there is, that's
all you operate on---with ``{\tt ==}'' or anything else.

We close this section with some fine points about structs that are interesting
and certainly part of the expert's toolkit, but are not critical for a first
reading.

\finepoint {\bf  what all structs and classes share.}\label{ssec:wasacs}
Both structs and classes can implement an interface, and there is one interface,
called {\tt x10.lang.Any} that is the minimal interface that all structs and
classes must implement.  As you would guess, it is not particularly rich:
\begin{verbatim}
public interface Any {
   def toString():String;
   def typeName():String;
   def equals(that:Any):Boolean;
   def hashCode():Int;
}
\end{verbatim}
The first three are the ones that most programmers care about.  The purpose of
the fourth method, the hash code, is to give a compact way of identifying an
object.

The method {\tt toString} returns
a representation of an instance as a {\tt String}.  Because {\tt toString} is
always available, conversion to {\tt String} is possible in any context where a
{\tt String} is known by the compiler to be required. 

The method {\tt typeName} returns the precise name of the type as it is
understood by the \Xten{} compiler.  It is most useful as a learning aid and
debugging tool.  If ever you are in doubt about what the compiler thinks the
type of some object ``{\tt e}'' is, the call

{\tt  \ \ \ Console.OUT.println(e.typeName());}

will tell you.  Try a few, for instance: {\tt 3.typeName()}.

The method {\tt equals} tests whether its invoker and its argument ``have the
same value.''  It should be thought of as a user-overridable version of the
comparison operator ``{\tt ==}".  Therefore, {\tt a.equals(b)} is the same as
{\tt a==b} by default.  Remember, though, that, as we just remarked above,
``{\tt ==}'' has a different meaning for class instances than for
struct instances.
 
Many classes override {\tt equals}.  A familiar example is the {\tt
String} class: if {\tt a} and {\tt b} are {\tt Strings}, then {\tt a.equals(b)}
is {\tt true} when the contents of their underlying byte arrays are the same.

\finepoint{\bf structs really are values.}
The rules for what can be part of a struct are designed so that the compiler
can determine the size of an instance from the declaration.  The reason is
storage efficiency:  once the size is known, no more space need be allocated
than is needed to store the value in memory, and at runtime, we don't have to
carry around size information for every instance.

One consequence of the storage rules is that you cannot define the member
fields of a struct in terms of the same struct or collections of that
struct.  If, for instance we try to declare a struct {\tt DblDblList\-Item} 
\begin{verbatim}
public struct DblDblListItem {
   public p: DblDbl;            // the point
   public next: DblDblListItem; // !!! no: illegal
}
\end{verbatim}
so that we can form lists of {\tt DblDbl}s, the compiler will complain.
To understand why, ask yourself, ``How big is a {\tt DblDbl\-List\-Item}?''
Call the answer ``{\tt sizeOfDDI}.''
\begin{quote}
A {\tt DblDbl}
itself is 16 bytes. So adding the sizes of {\tt p} and {\tt
next}, we get a total of {\tt sizeOfDDI+16} bytes.
But that has to be the same as the size of the whole {\tt DblDblList\-Item} in
which these are the instance fields. That whole value is only {\tt sizeOfDDI},
just like the size of {\tt next}.
\end{quote}
We cannot have {\tt  sizeOfDDI == sizeOfDDI+16}.  Something is wrong.

The point is that we really wanted {\tt next} to be a reference to the next
item: an address, and not the item itself---and if {\tt DblDbl\-List\-Item} were a
class rather than a struct, that is exactly what {\tt next} would be: a
reference. C++ programmers would use a pointer in this sort of situation, a
physical address: {\tt DblDbl\-List\-Item *next}.  Its size, like the size
of any object reference, is some small number of bytes that is independent of
what its target is. But in \Xten, there is no such thing as reference to a
struct, nor can you take its ``address.''  A struct is a value. Period!

The short answer, then, to ``How {\em does} one design an \Xten{}
struct that is a list?'' is that {\em one shouldn't try}.  
\footnote{
We wrote out some code for you to show that if you are really desperate, you 
{\em can} do it, but be warned, this is very, very bad \Xten.  Do not use this
as a model for {\em anything}  The use of the {\tt Any} interface there is
particularly ill-advised. See 
\href{http://dist.codehaus.org/x10/documentation/guide/src/list/DblDblList.x10}{list/DblDblList.x10}.
}

\finepoint{a final word about performance.}
By exposing the mechanism that it uses to implement
its primitive types, \Xten{} allows programmers to eke out similar
efficiencies.  But there is no simple answer as to when structs are really
to be preferred over classes.  The problem is that there is a potential down
side to structs.  Consider the assignment: 
\begin{verbatim}
   var x: T;
   var y: T;
   ...
   x = y;
\end{verbatim}
When {\tt T} is a class, the assignment copies a reference that is {\tt y}'s
current value, to {\tt x}.  When {\tt T} is a struct, however, what gets copied
is the entire current value of {\tt y}, not a pointer to the value.  To the
extent that the  size of the value is greater than the size of a reference,
copying the value is more expensive than the copying the reference.
Our {\tt DblDbl} is a good example: its size is 16 bytes, and a reference
could use as few as 4 bytes, if only an object id is kept.

An analogous problem comes up when passing parameters:
\begin{verbatim}
   public def doIt(t: T) { ... }
   var x: T;
   doIt(x);
\end{verbatim}
When {\tt T} is a class, the value passed to {\tt doIt} is the reference.  If
{\tt T} is a struct, the whole value of {\tt x} has to be copied.
 
To summarize: the benefits of structs are:
\begin{quote}
- no cost for memory allocation,\\
- no cost for garbage collection,\\
- faster method invocation, and\\
- better storage utilization for large aggregates.
\end{quote}
and the costs are:
\begin{quote}
- assignments and argument passing may be more expensive, and\\
- you lose the benefits of the class hierarchy for organizing common
functionality.
\end{quote}  

The situation is further complicated by the tricks code optimization can use to
minimize the assignment costs, so even knowing that there might be a problem, it
may be difficult to tease out the cost, if any, short of running the code to
see. In addition, when you are working with classes, you can avoid some of the
cost of method invocation by declaring a class to be ``{\tt final}''---\eg{}
``{\tt public final class DblDblListItem}''.  Final classes cannot be
subclassed, which implies that their methods cannot be overridden, and this, in
turn, allows the compiler to reduce the cost of method invocation.
