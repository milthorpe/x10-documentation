\chapter{Change Log}

\section{Changes from \Xten{} v2.0.6}

This document summarizes the main changes between X10 2.0.6 and X10 2.1.  The
descriptions are intended to be suggestive rather than definitive; see the
language specification for full details.

\subsection{Object Model}

\begin{enumerate}
\item Objects are now local rather than global.
   
    \begin{enumerate}
    \item The \Xcd{home} property is gone.
    \item \Xcd{at(P)S} produces deep copies of all objects reachable from
          lexically exposed variables in \xcd`S` 
          when it executes \Xcd{S}.  ({\bf Warning:} They are copied even in  
          \Xcd{at(here)S}.)
    \end{enumerate}

\item The \Xcd{GlobalRef[T]} struct is the only way to produce or manipulate
      cross-place references.
    \begin{enumerate}
    \item \Xcd{GlobalRef}'s have a \Xcd{home} property.
    \item Use \Xcd{GlobalRef[Foo](foo)} to make a new global reference.
    \item Use \Xcd{myGlobalRef()} to access the object referenced; this
          requires \Xcd{here == myGlobalRef.home}. 
    \end{enumerate}


\item  The \xcd`!` type modifier is no longer needed or present.

\item \Xcd{global} modifiers are now gone:
    
    \begin{enumerate}
    \item \Xcd{global} methods in {\em interfaces} are now the default. 
    \item \Xcd{global} {\em fields} are gone.  In some cases object copying
          will produce the same effect as global fields.  In other cases code
          must be rewritten.  It may be desirable to mark nonglobal fields
          \Xcd{transient} in many cases.
    \item \Xcd{global} {\em methods} are now marked \Xcd{@Global} instead.  
          Methods intended to be non-global may be marked \Xcd{@Pinned}.
    \end{enumerate}


\end{enumerate}

\subsection{Constructors}


\begin{enumerate}
\item \Xcd{proto} types are gone.
\item Constructors and the methods they call must satisfy a number of static
      checks.  
    
    \begin{enumerate}
    \item Constructors can only invoke \Xcd{private} or \Xcd{final} methods, 
          or methods annotated \xcd`@NonEscaping`.  
    \item Methods invoked by constructors cannot read fields before they are
          written. 
    \item The compiler ensures this with a detailed protocol. 
    \end{enumerate}

\item It is still impossible for X10 constructors to leak references to
      \Xcd{this} or observe uninitialized fields of an object.  Now, however,
      the mechanisms enforcing this are less obtrusive than in 2.0.6; the
      burden is largely on the compiler, not the programmer.
\end{enumerate}




%REF> \subsection{Call by Reference}
%REF> 
%REF> A very limited form of call-by-reference is now available.
%REF> 
%REF> 
%REF> \begin{enumerate}
%REF> 
%REF> \item Formal parameters to functions and methods may be \Xcd{ref} rather than
%REF>       \Xcd{var} or \Xcd{val}.  
%REF> \item Assignment to a \Xcd{ref} parameter \Xcd{x} changes the original
%REF>       location that the \Xcd{ref} refers to.  \eg, 
%REF>       \xcd`def inc(ref x:Int) { x ++; }`
%REF>       allows a call \Xcd{inc(n)} to increment a local \Xcd{var} \Xcd{n}.
%REF> \item Only local variables or \Xcd{ref} parameters can be passed as actual
%REF>       \Xcd{ref} parameters.  Fields, array elements, and other variable-like
%REF>       items cannot be. 
%REF> \item External \Xcd{ref} variables cannot be captured in closures. However,
%REF>       closures may have \Xcd{ref} parameters.
%REF> \item \Xcd{ref}s are {\em not} first-class objects in X10. They cannot be
%REF>       returned from functions, stored in data structures, etc.
%REF> \item These restrictions limit the possibilities of aliasing and the need for
%REF>       boxing of \Xcd{ref} parameters.  \Xcd{ref}s to stack locations cannot,
%REF>       with these restrictions, live past the death of the location's
%REF>       containing stack frame.      
%REF> \item This allows the implementation of many core constructs as syntactic
%REF>       sugar on library calls.   Programmers may use it, but mutability should
%REF>       generally be encapsulated inside objects rather than \Xcd{ref}
%REF>       parameters. 
%REF> \end{enumerate}
%REF> 

%ACC> \subsection{Accumulator Variables}
%ACC> 
%ACC> Accumulator variables generalize and make explicit collecting \Xcd{finish} in
%ACC> X10 2.0.6.  An \Xcd{acc} variable is declared: 
%ACC> \begin{xten}
%ACC> acc(r) A;
%ACC> \end{xten}
%ACC> where \Xcd{r} is a {\em reducer} (much as in 2.0.6): 
%ACC> \begin{xten}
%ACC> struct Reducer[T](zero:T, apply:global (T,T)=>T){}
%ACC> \end{xten}
%ACC> 
%ACC> Usage of \Xcd{A} is restricted in ways that make it determinate in the
%ACC> intended case of a pure, associative, commutative \Xcd{apply} with unit
%ACC> element \Xcd{zero}.  
%ACC> 
%ACC> \begin{enumerate}
%ACC> \item \Xcd{A} is initialized to \Xcd{r.zero}.  
%ACC> \item Multiple activities can {\em write} into \Xcd{A}.  In particular, the
%ACC>       ``assignment'' \Xcd{A = v} is approximately interpreted as 
%ACC>       \xcd`atomic{A = r.apply(A, v)}` --- that is, it accumulates \Xcd{v} into
%ACC>       \Xcd{A} using \Xcd{r.apply.}
%ACC> \item {\em Reading} of \Xcd{A} is restricted to situations where it makes
%ACC>       sense.  Specifically, only the activity in which \Xcd{A} is declared can
%ACC>       read from it, and it can only do so when all asyncs which it has spawned
%ACC>       have terminated -- \eg, outside of the scope of all \Xcd{async}s and
%ACC>       \Xcd{finish}es.  
%ACC> \item Formal parameters of functions may be marked \Xcd{acc x:T}.  The reducer
%ACC>       \Xcd{r} must not be specified; it is passed as an implicit parameter
%ACC>       going with the actual \Xcd{acc} variable.  
%ACC> \item X10 provides protocols for indexed collections of \Xcd{acc} variables,
%ACC>       presented as objects.
%ACC> \end{enumerate}
%ACC> 


\subsection{Implicit clocks for each finish}


Most clock operations can be accomplished using the new implicit clocks.

\begin{enumerate}
\item A \Xcd{finish} may be qualified with \Xcd{clocked}, which gives it a
      clock.
\item An \Xcd{async} in a \Xcd{clocked finish} may be marked \Xcd{clocked}.
      This registers it on the same clock as the enclosing \Xcd{finish}.  
\item \xcd`clocked async S` and \xcd`clocked finish S` may use \xcd`next` in
      the body of \Xcd{S} to advance the clock.
\item When the body of a \Xcd{clocked finish} completes, the \Xcd{clocked
      finish} is dropped form the clock.  It will still wait for spawned
      asyncs to terminate, but such asyncs need to wait for it.
\end{enumerate}


%CLOCAL>\subsection{Clocked local variables}
%CLOCAL>
%CLOCAL>Local \Xcd{val} and \Xcd{acc} variables may be \Xcd{clocked}.  They are
%CLOCAL>associated with the clock of the surrounding \Xcd{clocked finish}.  
%CLOCAL>Clocked variables have a {\em current} value and an {\em upcoming} value.  The
%CLOCAL>current value may be read at suitable times; the upcoming value may be
%CLOCAL>updated.  The \Xcd{next} phase makes the upcoming value current.

\subsection{Asynchronous initialization of val}

\Xcd{val}s can be initialized asynchronously.   As always with \Xcd{val}s,
they can only be read after it is guaranteed that they have been initialized.
For example, both of the \Xcd{print}s below are good.  However, the
commented-out \Xcd{print} in the \Xcd{async} is bad, since it is possible that
it will be executed before the initialization of \Xcd{a}. 
\begin{xten}
val a:Int;
finish {
  async {
     a = 1; 
     print("a=" + a);
  }
  async {
     // WRONG: print("a=" + a);
  }
}
print("a=" + a);
\end{xten}



\subsection{Main Method}

The signature for the \Xcd{main} method is now: 
\begin{xten}
           def main(Array[String]) {..}
\end{xten}
or, if the arguments are actually used, 
\begin{xten}
           def main(argv: Array[String](1)) {..}
\end{xten}

\subsection{Assorted Changes}


\begin{enumerate}
\item The syntax for destructuring a point now uses brackets rather than
      braces: \Xcd{for( [i] in 1..10 )}, rather than the prior \Xcd{(i)}.  
\end{enumerate}

\subsection{Safety of atomic and when blocks}


\begin{enumerate}
\item Static effect annotations (\Xcd{safe}, \Xcd{sequential},
      \Xcd{nonblocking}, \Xcd{pinned}) are no longer used. They have been
      replaced by dynamic checks.
\item Using an inappropriate operation in the scope of an \Xcd{atomic} or
      \Xcd{when} construct will throw \Xcd{IllegalOperationException}.  
      The following are inappropriate:      
      \begin{itemize}
      \item \Xcd{when}
      \item \Xcd{resume()} or \Xcd{next} on clocks
      \item async
      \item \Xcd{Future.make()}, or \Xcd{Future.force()}.
      \item \Xcd{at}
      \end{itemize}

\end{enumerate}


\subsection{Removed Topics}

The following are gone: 

\begin{enumerate}
\item \Xcd{foreach} is gone.
\item All \Xcd{var}s are effectively \Xcd{shared}, so \Xcd{shared} is gone.
\item The place clause on \Xcd{async} is gone.  \Xcd{async (P) S} should be
      written \Xcd{at(P) async S}.
\item Checked exceptions are gone.
\item \Xcd{future} is gone.
\item \Xcd{await ... or ... } is gone.
\item \Xcd{const} is gone.

\end{enumerate}

\subsection{Deprecated}

The following constructs are still available, but are likely to be replaced in
a future version: 


\begin{enumerate}
\item \Xcd{ValRail}.
\item \Xcd{Rail}.
\item \xcd`ateach`
\item \xcd`offers`.  \index{offers}  The \xcd`offers` concept was experimental
      in 2.1, but was determined inadequate.  It has not been removed from the
      compiler yet, but it will be soon.  In the meantime, traces of it are
      still visible in the grammar.  They should not be used and can safely be ignored.
\end{enumerate}

\section{Changes from \Xten{} v2.0}

\begin{itemize}
\item \Xcd{Any} is now the top of the type hierarchy (every object,
  struct and function has a type that is a subtype of
  \Xcd{Any}). \Xcd{Any} defines \Xcd{home}, \Xcd{at}, \Xcd{toString},
  \Xcd{typeName}, \Xcd{equals} and \Xcd{hashCode}. \Xcd{Any} also defines the methods
  of \Xcd{Equals}, so \Xcd{Equals} is not needed any more.
\item Revised discussion of incomplete types.
\item The manual has been revised and brought into line with the current implementation. 
\end{itemize}
\section{Changes from \Xten{} v1.7}

The language has changed in the following way:
\begin{itemize}

\item {\bf Type system changes}: There are now three kinds of entities
  in an \Xten{} computation: objects, structs and functions. Their
  associated types are class types, struct types and function
  types. 

  Class and struct types are called {\em container types} in that they
  specify a collection of fields and methods. Container types have a
  name and a signature (the collection of members accessible on that
  type). Collection types support primitive equality \Xcd{==} and may
  support user-defined equality if they implement the {\tt
    x10.lang.Equals} interface. 

  Container types (and interface types) may be further qualified with
  constraints.

  A function type specifies a set of arguments and their type, the
  result type, and (optionally) a guard. A function application
  type-checks if the arguments are of the given type and the guard is
  satisfied, and the return value is of the given type.  A function
  type does not permit \Xcd{==} checks. Closure literals create
  instances of the corresponding function type.

  Container types may implement interfaces and zero or more function
  types.

  All types support a basic set of operations that return a string
  representation, a type name, and specify the home place of the entity.

  The type system is not unitary. However, any type may be used to
  instantiate a generic type. 

  There is no longer any notion of \Xcd{value} classes. \Xcd{value}
  classes must be re-written into structs or (reference) classes. 

\item {\bf Global object model}: Objects are instances of
  classes. Each object is associated with a globally unique
  identifier. Two objects are considered identical \Xcd{==} if their
  ids are identical. Classes may specify \Xcd{global} fields and
  methods. These can be accessed at any place. (\Xcd{global} fields
  must be immutable.)

\item{\bf Proto types.} For the decidability of dependent type
  checking it is necessary that the property graph is acyclic. This is
  ensured by enforcing rules on the leakage of \Xcd{this} in
  constructors. The rules are flexible enough to permit cycles to be
  created with normal fields, but not with properties.

\item{Place types.} Place types are now implemented. This means that
  non-global methods can be invoked on a variable, only if the
  variable's type is either a struct type or a function type, or a
  class type whose constraint specifies that the object is located in
  the current place.

  There is still no support for statically checking array access
  bounds, or performing place checks on array accesses.

\end{itemize}
