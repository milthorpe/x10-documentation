\chapter{Variables}\label{XtenVariables}\index{variables}

%%OLDA variable is a storage location.  \Xten{} supports seven kinds of
%%OLDvariables: constant {\em class variables} (static variables), {\em
%%OLD  instance variables} (the instance fields of a class), {\em array
%%OLD  components}, {\em method parameters}, {\em constructor parameters},
%%OLD{\em exception-handler parameters} and {\em local variables}.

A {\em variable} is an X10 identifier associated with a value within some
context. Variable bindings have these essential properties:
\begin{itemize}
\item {\bf Type:} What sorts of values can be bound to the identifier;
\item {\bf Scope:} The region of code in which the identifier is associated
      with the entity;
\item {\bf Lifetime:} The interval of time in which the identifier is
      associated with the entity.
\item {\bf Visibility:} Which parts of the program can read or manipulate the
      value through the variable.
\end{itemize}



X10 has many varieties of variables, used for a number of purposes. They will
be described in more detail in this chapter.  
\begin{itemize}
\item Class variables, also known as the static fields of a class, which hold
      their values for the lifetime of the class.  
\item Instance variables, which hold their values for the lifetime of an
      object;
\item Array elements, which are not individually named and hold their values
      for the lifetime of an array;
\item Formal parameters to methods, functions, and constructors, which hold
      their values for the duration of method (etc.) invocation;
\item Local variables, which hold their values for the duration of execution
      of a block.
\item Exception-handler parameters, which hold their values for the execution
      of the exception being handled. 
\end{itemize}
A few other kinds of things are called variables for historical reasons; \eg,
type parameters are often called type variables, despite not being variables
in this sense because they do not refer to X10 values.  Other named entities,
such as classes and methods, are not called variables.  However, all
name-to-whatever bindings enjoy similar concepts of scope and visibility.  

In the following example, \xcd`n` is an instance variable, and \xcd`next` is a
local variable defined within the method \xcd`bump`.\footnote{This code is
unnecessarily turgid for the sake of the example.  One would generally write
\xcd`public def bump() = ++n;`.   }
%~~gen
% package Vars.For.Squares;
%~~vis
\begin{xten}
class Counter {
  private var n : Int = 0;
  public def bump() : Int {
    val next = n+1;
    n = next;
    return next;
    }
}
\end{xten}
%~~siv
%
%~~neg
Both variables have type \xcd`Int` (or
perhaps something more specific).    The scope of \xcd`n` is the body of
\xcd`Counter`; the scope of \xcd`next` is the body of \xcd`bump`.  The
lifetime of \xcd`n` is the lifetime of the \xcd`Counter` object holding it;
the lifetime of \xcd`next` is the duration of the call to \xcd`bump`. Neither
variable can be seen from outside of its scope.

\label{exploded-syntax}
\label{VariableDeclarations}
\index{variable declaration}


Variables whose value may not be changed after initialization are said to be
{\em immutable}, or {\em constants} (\Sref{FinalVariables}), or simply
\xcd`val` variables. Variables whose value may change are {\em mutable} or
simply \xcd`var` variables. \xcd`var` variables are declared by the \xcd`var`
keyword. \xcd`val` variables may be declared by the \xcd`val` keyword; when a
variable declaration does not include either \xcd`var` or \xcd`val`, it is
considered \xcd`val`. 


%~~gen
%package Vars.For.Bears.In.Chairs;
%class VarExample{
%static def example() {
%~~vis
\begin{xten}
val a : Int = 0;               // Full 'val' syntax
b : Int = 0;                   // 'val' implied
val c = 0;                     // Type inferred
var d : Int = 0;               // Full 'var' syntax
var e : Int;                   // Not initialized
var f : Int{self != 100} = 0;  // Constrained type
\end{xten}
%~~siv
%}}
%~~neg







\section{Immutable variables}
\label{FinalVariables}
\index{variable!immutable}
\index{immutable variable}

Immutable variables can be given values (by initialization or assignment) at
most once, and must be given values before they are used.  Usually this is
achieved by declaring and initializing the variable in a single statement.
%~~gen
% package Vars.In.Snares;
% class ABitTedious{
% def example() {
%~~vis
\begin{xten}
val a : Int = 10;
val b = (a+1)*(a-1);
\end{xten}
%~~siv
%}}
%~~neg
\xcd`a` and \xcd`b` cannot be assigned to further.

In other cases, the declaration and assignment are separate.  One such
case is how constructors give values to \xcd`val` fields of objects.  The
\xcd`Example` class has an immutable field \xcd`n`, which is given different
values depending on which constructor was called. \xcd`n` can't be given its
value by initialization when it is declared, since it is not knowable which
constructor is called at that point.  
%~~gen
% package Vars.For.Cares;
%~~vis
\begin{xten}
class Example {
  val n : Int; // not initialized here
  def this() { n = 1; }
  def this(dummy:Boolean) { n = 2;}
}
\end{xten}
%~~siv
%
%~~neg

Another common case of separating declaration and assignment is in function
and method call.  The formal parameters are bound to the corresponding actual
parameters, but the binding does not happen until the function is called.  In
the code below, \xcd`x` is initialized to \xcd`3` in the first call and
\xcd`4` in the second.
%~~gen
%package Vars.For.Swears;
%class Examplement {
%static def whatever() {
%~~vis
\begin{xten}
val sq = (x:Int) => x*x;
x10.io.Console.OUT.println("3 squared = " + sq(3));
x10.io.Console.OUT.println("4 squared = " + sq(4));
\end{xten}
%~~siv
%}}
%~~neg





%%IMMUTABLE%% An immutable variable satisfies two conditions: 
%%IMMUTABLE%% \begin{itemize}
%%IMMUTABLE%% \item it can be assigned to at most once, 
%%IMMUTABLE%% \item it must be assigned to before use. 
%%IMMUTABLE%% \end{itemize}
%%IMMUTABLE%% 
%%IMMUTABLE%% \Xten{} follows \java{} language rules in this respect \cite[\S
%%IMMUTABLE%% 4.5.4,8.3.1.2,16]{jls2}. Briefly, the compiler must undertake a
%%IMMUTABLE%% specific analysis to statically guarantee the two properties above.
%%IMMUTABLE%% 
%%IMMUTABLE%% Immutable local variables and fields are defined by the \xcd"val"
%%IMMUTABLE%% keyword.  Elements of value arrays are also immutable.
%%IMMUTABLE%% 
%%IMMUTABLE%% \oldtodo{Check if this analysis needs to be revisited.}

\section{Initial values of variables}
\label{NullaryConstructor}\index{nullary constructor}

Every assignment, binding, or initialization to a variable of type \xcd`T{c}`
must be an instance of type \xcd`T` satisfying the constraint \xcd`{c}`.
Variables must be given a value before they are used. This may be done by
initialization, which is the only way for immutable (\xcd`val`) variables and
one option for mutable (\xcd`var`) ones: 

%~~gen
%package Vars.For.Bears;
%class VarsForBears{
%def check() {
%~~vis
\begin{xten}
  val immut : Int = 3;
  var mutab : Int = immut;
  val use = immut + mutab;
\end{xten}
%~~siv
%}}
%~~neg
Or, for mutable variables, it may be done by a later assignment.  

%~~gen
%package Vars.For.Stars;
%class VarsForStars{
%def check() {
%~~vis
\begin{xten}
  var muta2 : Int;
  muta2 = 4;
  val use = muta2 * 10;
\end{xten}
%~~siv
%}}
%~~neg


Every class variable must be initialized before it is read, through
the execution of an explicit initializer or a static block. Every
instance variable must be initialized before it is read, through the
execution of an explicit initializer or a constructor.
\bard{Revise this in light of initial values}
Mutable instance variables of class type are initialized to 
to \xcd"null".
Mutable instance variables of struct type are 
assumed to have an initializer that sets the value to the
result of invoking the nullary constructor on the class. 
An initializer is required if the default initial value of the variable's type
is not
assignable to the variable's type, \eg, \xcd`Int` variables are initialized to
zero, but that doesn't work for \xcd`val x:Int{x!=0}`.

Each method and constructor parameter is initialized to the
corresponding argument value provided by the invoker of the method. An
exception-handling parameter is initialized to the object thrown by
the exception. A local variable must be explicitly given a value by
initialization or assignment, in a way that the compiler can verify
using the rules for definite assignment \cite[\S~16]{jls2}.


\section{Destructuring syntax}
\index{variable declarator!destructuring}
\index{destructuring}
\Xten{} permits a \emph{destructuring} syntax for local variable
declarations and formal parameters of type \xcd`Point`, \Sref{point-syntax}.
(Future versions of X10 may allow destructuring of other types as well.) 
A point is a sequence of {$r \ge 0$} \xcd`Int`-valued coordinates.  
It is often useful to get at the coordinates directly, in variables. 

The following code makes an anonymous point with one coordinate \xcd`11`, and
binds \xcd`i` to \xcd`11`.  Then it makes a point with coordinates \xcd`22`
and \xcd`33`, binds \xcd`p` to that point, and \xcd`j` and \xcd`k` to \xcd`22`
and \xcd`33` respectively.
%~~gen
% package Vars.For.Glares;
% class DestructuringEx1 {
% def whyJustForLocals() {
%~~vis
\begin{xten}
val [i] : Point = Point.make(11);
val p[j,k] = Point.make(22,33);
\end{xten}
%~~siv
%}}
%~~neg

A useful idiom for iterating over a range of numbers is: 
%~~gen
%package Vars.For.Bears;
% class ForBear {
% def forbear() {
%~~vis
\begin{xten}
var sum : Int = 0;
for ([i] in 1..100) sum += i;
\end{xten}
%~~siv
% ; } } 
%~~neg
\noindent
The brackets in \xcd`[i]` introduce destructuring, making X10 treat \xcd`i`
as an \xcd`Int`; without them, it would be a \xcd`Point`.  

In general, a pattern of the form \xcdmath"[i$_1$,$\ldots$,i$_n$]" matches a
point with {$n$} coordinates, binding \xcdmath"i$_j$" to coordinate {$j$}.  
A pattern of the form \xcdmath"p[i$_1$,$\ldots$,i$_n$]" does the same, , but
also binds \xcd`p` to the point.

\section{Formal parameters}

Formal parameters are the variables which hold values transmitted into a
method or function.  
They are always declared with a type.  (Type inference is not
advisable, because there is no single expression to deduce a type from.)
The variable name can be omitted if it is not to be used in the
scope of the declaration, as in the type of the method 
\xcd`static def main(Rail[String]):Void` executed at the start of a program that
does not use its command-line arguments.

\begin{grammar}
Formal
        \: FormalModifier\star \xcd"var" VarDeclaratorWithType \\
        \| FormalModifier\star \xcd"val" VarDeclaratorWithType \\
        \| FormalModifier\star VarDeclaratorWithType \\
        \| Type \\
FormalModifier \: Annotation \\
              \| \xcd"shared" \\
\end{grammar}

\xcd`var`, \xcd`val`, and \xcd`shared` behave just as they do for local
variables, \Sref{local-variables}.  In particular, the following \xcd`inc`
method is allowed, but, unlike some languages, does {\em not} increment its
actual parameter.  \xcd`inc(j)` creates a new local 
variable \xcd`i` for the method call, initializes \xcd`i` with the value of
\xcd`j`, increments \xcd`i`, and then returns.  \xcd`j` is never changed.
%~~gen
% package Vars.For.Squares.Of.Mares;
% class Ink {
%~~vis
\begin{xten}
static def inc(var i:Int) { i += 1; }
\end{xten}
%~~siv
%}
%~~neg


\section{Local variables}\label{local-variables}
Local variables are declared in a limited scope, and, dynamically, keep their
values only for so long as the scope is being executed.  They may be \xcd`var`
or \xcd`val`.  
They may have 
initializer expressions: \xcd`var i:Int = 1;` introduces 
a variable \xcd`i` and initializes it to 1.
If the variable is immutable
(\xcd"val")
the type may be omitted and
inferred from the initializer type (\Sref{TypeInference}).
Variables marked \xcd`shared` can be used by many activities at once; see
\Sref{Shared}.

The variable declaration \xcd`val x:T=e;` confirms that \xcd`e`'s value is of
type \xcd`T`, and then introduces the variable \xcd`x` with type \xcd`T`.  For
example, 
%~~gen
% package Vars.Local;
% class Tub{
% def example() {
%~~vis
\begin{xten}
   val t : Tub = new Tub();
\end{xten}
%~~siv
%}}
%~~neg
\noindent
produces a variable \xcd`t` of type \xcd`Tub`, even though the expression
\xcd`new Tub()` produces a value of type \xcd`Tub!` -- that is, a \xcd`Tub`
located \xcd`here`.    This can be inconvenient if, \eg, it is desired to make
method calls upon \xcd`t`.  

Including type information in variable declarations is generally good
programming practice: it explains to both the compiler and human readers
something of the intent of the variable.  However, including types in 
\xcd`val t:T=e` can obliterate helpful information.  So, X10 allows a {\em
documentation type declaration}, written \xcd`val t <: T = e`.  This 
has the same effect as \xcd`val t = e`, giving \xcd`t` the full type inferred
from \xcd`e`; but it also confirms statically that that type is at least
\xcd`T`.  For example, the following gives \xcd`t` the type \xcd`Tub!` as
desired: 
%~~gen
% package Vars.Local;
% class TubBounded{
% def example() {
%~~vis
\begin{xten}
   val t <: Tub = new Tub();
\end{xten}
%~~siv
%}}
%~~neg
\noindent
However, replacing \xcd`Tub` by \xcd`Int` would result in a compilation error. 

Variables do not need to be initialized at the time of definition -- not even
\xcd`val`s. They must be initialized by the time of use, and \xcd`val`s may
only be assigned to once. The X10 compiler performs static checks guaranteeing
this restriction. The following is correct, albeit obtuse: 
%~~gen
%package Vars.Local;
% class NotInitVal {
%~~vis
\begin{xten}
static def main(r: Rail[String]):Void {
  val a : Int;
  a = r.length;
  val b : String;
  if (a == 5) b = "five?"; else b = "" + a + " args"; 
  // ... 
\end{xten}
%~~siv
%} }
%~~neg


\bard{Give grammar}

\section{Fields}

Like most other kinds of variables in X10, 
the fields of an object can be either \xcd`val` or \xcd`var`. 
Fields can be \xcd`static`,\xcd`global`, or \xcd`property`; see
\Sref{FieldDefinitions} and \Sref{PropertiesInClasses}.
Field declarations may have optional
initializer expressions, as for local variables, \Sref{local-variables}.
\xcd`var` fields without an initializer are initialized with the default value
of their type. \xcd`val` fields without an initializer must be initialized by
each constructor.


For \xcd`val` fields, as for \xcd`val` local variables, the type may be
omitted and inferred from the initializer type (\Sref{TypeInference}).
\xcd`var` files, like \xcd`var` local variables, must be declared with a type.

\begin{grammar}
FieldDeclaration
        \: FieldModifier\star \xcd"var" FieldDeclaratorsWithType \\&& ( \xcd"," FieldDeclaratorsWithType )\star \\
        \| FieldModifier\star \xcd"val" FieldDeclarators \\&& ( \xcd"," FieldDeclarators )\star \\
        \| FieldModifier\star FieldDeclaratorsWithType \\&& ( \xcd"," FieldDeclaratorsWithType )\star \\
FieldDeclarators
        \: FieldDeclaratorsWithType \\
        \: FieldDeclaratorWithInit \\
FieldDeclaratorId
        \: Identifier  \\
FieldDeclaratorWithInit
        \: FieldDeclaratorId Init \\
        \| FieldDeclaratorId ResultType Init \\
FieldDeclaratorsWithType
        \: FieldDeclaratorId ( \xcd"," FieldDeclaratorId )\star ResultType \\
FieldModifier \: Annotation \\
                \| \xcd"static" \\ \| \xcd`property` \\ \| \xcd`global` \\
\end{grammar}



