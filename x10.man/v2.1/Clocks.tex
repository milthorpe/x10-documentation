\chapter{Clocks}\label{XtenClocks}\index{clock}

Many concurrent algorithms proceed in phases: in phase {$k$}, several
activities work independently, but synchronize together before proceeding on
to phase {$k+1$}. X10 supports this communication structure (and many
variations on it) with a generalization of barriers 
called {\em clocks}. Clocks are designed so that programs which follow a
simple syntactic discipline will not have either deadlocks or race conditions.


The following minimalist example of clocked code has two worker activities A
and B, and three phases. In the first phase, each worker activity says its
name followed by 1; in the second phase, by a 2, and in the third, by a 3.  
So, if \xcd`say` prints its argument, 
\xcd`A-1 B-1 A-2 B-2 B-3 A-3`
would be a legitimate run of the program, but
\xcd`A-1 A-2 B-1 B-2 A-3 B-3`
(with \xcd`A-2` before \xcd`B-1`) would not.

The program creates a clock \xcd`cl` to manage the phases.  Each participating
activity does
the work of its first phase, and then executes \xcd`next;` to signal that it
is finished with that work. \xcd`next;` is blocking, and causes the participant to
wait until all participant have finished with the phase -- as measured by the
clock \xcd`cl` to which they are both registered.  
Then they do the second phase, and another \xcd`next;` to make sure that
neither proceeds to the third phase until both are ready.  This example uses
\xcd`finish` to wait for both particiants to finish.  


%%TODO -- put the 'atomic' back in when that's legal.

%~~gen
%package Clocks.For.Spock;
%~~vis
\begin{xten}
class ClockEx {
  static def say(s:String) = 
     { atomic{x10.io.Console.OUT.println(s);} }
  public static def main(argv:Rail[String]) {
    finish async{
      val cl = Clock.make();
      async clocked(cl) {// Activity A
        say("A-1");
        next;
        say("A-2");
        next;
        say("A-3"); 
      }// Activity A

      async clocked(cl) {// Activity B
        say("B-1");
        next;
        say("B-2");
        next;
        say("B-3"); 
      }// Activity B
    }
\end{xten}
%~~siv
%  }
% }
%~~neg

This chapter describes the syntax and semantics of clocks and
statements in the language that have parameters of type \xcd"Clock". 

The key invariants associated with clocks are as follows.  At any
stage of the computation, a clock has zero or more {\em registered}
activities. An activity may perform operations only on those clocks it
is registered with (these clocks constitute its {\em clock set}). 
An attempt by an activity to operate on a clock it is not registered with
will cause a 
\xcd"ClockUseException"\index{clock!ClockUseException}
to be thrown.  
An activity is registered with zero or more clocks when it is created.
During its lifetime the only additional clocks it can possibly be registered with
are exactly those that it creates. In particular it is not possible
for an activity to register itself with a clock it discovers by
reading a data structure.

The primary operations that an activity \xcd`a` may perform on a clock \xcd`c`
that it is registered upon are: 
\begin{itemize}
\item It may spawn and simultaneously  {\em register} a new activity on
      \xcd`c`, with the statement       \xcd`async clocked(c)S`.
\item It may {\em unregister} itself from \xcd`c`, with \xcd`c.drop()`.  After
      doing so, it can no longer use most primary operations on \xcd`c`.
\item It may {\em resume} the clock, with \xcd`c.resume()`, indicating that it
      has finished with the current phase associated with \xcd`c` and is ready
      to move on to the next one.
\item It may {\em wait} on the clock, with \xcd`c.next()`.  This first does
      \xcd`c.resume()`, and then blocks the current activity until the start
      of the next phase, \viz, until all other activities registered on that
      clock have called \xcd`c.resume()`.
\item It may {\em block} on all the clocks it is registered with
      simultaneously, by the command \xcd`next;`.  This, in effect, calls
      \xcd`c.next()` simultaneously 
      on all clocks \xcd`c` that the current activity is registered with.
\item Other miscellaneous operations are available as well; see the
      \xcd`Clock` API.
\end{itemize}

%%CLOCK%% An activity may perform the following operations on a clock \xcd"c".
%%CLOCK%% It may {\em unregister} with \xcd"c" by executing \xcd"c.drop();".
%%CLOCK%% After this, it may perform no further actions on \xcd"c"
%%CLOCK%% for its lifetime. It may {\em check} to see if it is unregistered on a
%%CLOCK%% clock. It may {\em register} a newly forked activity with \xcd"c".
%%CLOCK%% %% It may {\em post} a statement \xcd"S" for completion in the current phase
%%CLOCK%% %% of \xcd"c" by executing the statement \xcd"now(c) S". 
%%CLOCK%% Once registered and "active" (see below), it may also perform the following operations.
%%CLOCK%% It may {\em resume} the clock by executing \xcd"c.resume();". This
%%CLOCK%% indicates to \xcd"c" that it has finished posting all statements it
%%CLOCK%% wishes to perform in the current phase. Finally, it may {\em block}
%%CLOCK%% (by executing \xcd"next;") on all the clocks that it is registered
%%CLOCK%% with. (This operation implicitly \xcd"resume"'s all clocks for the
%%CLOCK%% activity.) It will resume from this statement only when all these
%%CLOCK%% clocks are ready to advance to the next phase.

%%CLOCK%% A clock becomes ready to advance to the next phase when every activity
%%CLOCK%% registered with the clock has executed at least one \xcd"resume"
%%CLOCK%% operation on that clock and all statements posted for completion in
%%CLOCK%% the current phase have been completed.

%%OLIVIER-DENIES%% Though clocks introduce a blocking statement (\xcd"next") an important
%%OLIVIER-DENIES%% property of \Xten{} is that clocks -- when used with the \xcd`next;` {\em
%%OLIVIER-DENIES%%   statement} only, without the \xcd`c.next()` method call -- cannot introduce
%%OLIVIER-DENIES%% deadlocks. That is, the system cannot reach a quiescent state (in which no
%%OLIVIER-DENIES%% activity is progressing) from which it is unable to progress. For, before
%%OLIVIER-DENIES%% blocking each activity resumes all clocks it is registered with. Thus if a
%%OLIVIER-DENIES%% configuration were to be stuck (that is, no activity can progress) all clocks
%%OLIVIER-DENIES%% will have been resumed. But this implies that all activities blocked on
%%OLIVIER-DENIES%% \xcd"next" may continue and the configuration is not stuck. The only other
%%OLIVIER-DENIES%% possibility is that an activity may be stuck on \xcd"finish". But the
%%OLIVIER-DENIES%% interaction rule between \xcd"finish" and clocks
%%OLIVIER-DENIES%% (\Sref{sec:finish:clock-rule}) guarantees that this cannot cause a cycle in
%%OLIVIER-DENIES%% the wait-for graph. A more rigorous proof may be found in \cite{X10-concur05}.

\section{Clock operations}\label{sec:clock}
\index{clock!operations on}
There are two language constructs for working with clocks. 
\xcd`async clocked(cl) S` starts a new activity registered on one or more
clocks.  \xcd`next;` blocks the current activity until all the activities
sharing clocks with it are ready to proceed to the next clock phase. 
Clocks are objects, and have a number of useful methods on them as well.

\subsection{Creating new clocks}\index{clock!creation}\label{sec:clock:create}

Clocks are created using a factory method on \xcd"x10.lang.Clock":


%~~gen
% package Clocks.For.Spocks;
%class Clockuser {
% def example() {
%~~vis
\begin{xten}
val c: Clock = Clock.make();
\end{xten}
%~~siv
%}}
%~~neg

%%CLOCKVAR%% \eat{All clocked variables are implicitly \xcd`val`. The initializer for a
%%CLOCKVAR%% local variable declaration of type \xcd"Clock" must be a new clock
%%CLOCKVAR%% expression. Thus \Xten{} does not permit aliasing of clocks.
%%CLOCKVAR%% Clocks are created in the place global heap and hence outlive the
%%CLOCKVAR%% lifetime of the creating activity.  Clocks are structs, hence may be freely
%%CLOCKVAR%% copied from place to 
%%CLOCKVAR%% place. (Clock instances typically contain references to mutable state
%%CLOCKVAR%% that maintains the current state of the clock.)
%%CLOCKVAR%% }

The current activity is automatically registered with the newly
created clock.  It may deregister using the \xcd"drop" method on
clocks (see the documentation of \xcd"x10.lang.Clock"). All activities
are automatically deregistered from all clocks they are registered
with on termination (normal or abrupt).

\subsection{Registering new activities on clocks}
\index{clock!clocked statements}\label{sec:clock:register}

The statement 

%~~gen
%package Clocks.For.Jocks;
%class Qlocked{
%static def S():void{}
%static def flock() { 
% val c1 = Clock.make(), c2 = Clock.make(), c3 = Clock.make();
%~~vis
\begin{xten}
  async clocked (c1, c2, c3) S
\end{xten}
%~~siv
%();
%}}
%~~neg
starts a new activity, initially registered with
clocks \xcd`c1`, \xcd`c2`, and \xcd`c3`, and  running \xcd`S`. The activity running this code must
be registered on those clocks. 
Violations of these conditions are punished by the throwing of a
\xcd"ClockUseException"\index{clock!ClockUseException}. 

% An activity may transmit only those clocks that are registered with and
% has not quiesced on (\Sref{resume}). 
% A \xcd"ClockUseException"\index{clock!ClockUseException} is
%thrown if (and when) this condition is violated.

If an activity {$a$} that has executed \xcd`c.resume()` then starts a
new activity {$b$} also registered on \xcd`c` (\eg, via \Xcd{async
clocked(c) S}), the new activity {$b$} starts out having also resumed
\xcd`c`, as if it too had executed \xcd`c.resume()`.  
%~~gen
% package Clocks.For.Jocks.In.Clicky.Smocks;
%class Example{
%static def S():void{}
%static def a_phase_two():void{}
%static def b_phase_two():void{}
%static def example() {
%~~vis
\begin{xten}
// a
val c = Clock.make();
c.resume();
async clocked(c) {
  // b
  c.next();
  b_phase_two();
}
c.next();
a_phase_two();
\end{xten}
%~~siv
%} }
%~~neg
In the proper execution, {$a$} and {$b$} both perform
\xcd`c.next()` and then their phase-2 actions.  
However, if {$b$} were not
initially in the resume state for \xcd`c`, there would be a race condition;
{$b$} could perform \xcd`c.next()` and proceed to \xcd`b_phase_two`
before {$a$} performed \xcd`c.next()`.


An activity may check that it is registered on a clock \xcd"c" by
%~~exp~~`~~`~~c:Clock ~~
the predicate \xcd`c.registered()`


\begin{note}
\Xten{} does not contain a ``register'' operation that would allow an activity
to discover a clock in a datastructure and register itself on it. Therefore,
while a clock \xcd`c` may be stored in a data structure by one activity
\xcd`a` and read from it by another activity \xcd`b`, \xcd`b` cannot do much
with \xcd`c` unless it is already registered with it.  In particular, it
cannot register itself on \xcd`c`, and, lacking that registration, cannot
register a sub-activity on it with \xcd`async clocked(c) S`.
\end{note}


\subsection{Resuming clocks}\index{clock!resume}\label{resume}\label{sec:clock:resume}
\Xten{} permits {\em split phase} clocks. An activity may wish
to indicate that it has completed whatever work it wishes to perform
in the current phase of a  clock \xcd"c" it is registered with, without
suspending altogether. It may do so  by executing 
%~~exp~~`~~`~~c:Clock ~~
\xcd`c.resume()`.



An activity may invoke \xcd`resume()` only on a clock it is registered with,
and has not yet dropped (\Sref{sec:clock:drop}). A
\xcd"ClockUseException"\index{clock!ClockUseException} is thrown if this
condition is violated. Nothing happens if the activity has already invoked a
\xcd"resume" on this clock in the current phase.
%%OLIVIER-DENIES%%  Otherwise, \xcd`c.resume()`
%%OLIVIER-DENIES%% indicates that the activity will not transmit \xcd"c" to an 
%%OLIVIER-DENIES%% \xcd"async" (through a \xcd"clocked" clause), 
%%OLIVIER-DENIES%% until it terminates, drops \xcd"c" or executes a \xcd"next".

%%OLIVIER-DENIES%% \bard{The following is under investigation}
%%OLIVIER-DENIES%% \begin{staticrule*}
%%OLIVIER-DENIES%% It is a static error if any activity has a potentially
%%OLIVIER-DENIES%% live execution path from a \xcd"resume" statement on a clock \xcd"c"
%%OLIVIER-DENIES%% to a
%%OLIVIER-DENIES%% %\xcd"now" or
%%OLIVIER-DENIES%% async spawn statement (which registers the new
%%OLIVIER-DENIES%% activity on \xcd"c") unless the path goes through a \xcd"next"
%%OLIVIER-DENIES%% statement. (A \xcd"c.drop()" following a \xcd"c.resume()" is legal,
%%OLIVIER-DENIES%% as is \xcd"c.resume()" following a \xcd"c.resume()".)
%%OLIVIER-DENIES%% \end{staticrule*}

\subsection{Advancing clocks}\index{clock!next}\label{sec:clock:next}
An activity may execute the statement
\begin{xten}
next;
\end{xten}

\noindent 
Execution of this statement blocks until all the clocks that the
activity is registered with (if any) have advanced. (The activity
implicitly issues a \xcd"resume" on all clocks it is registered
with before suspending.)

\xcd`next;` may be thought of as calling \xcd`c.next()` in parallel for all
clocks that the current activity is registered with.  (The parallelism is
conceptually important: if activities {$a$} and {$b$} are both
registered on clocks \xcd`c` and \xcd`d`, and {$a$} executes
\xcd`c.next(); d.next()` while {$b$} executes \xcd`d.next(); c.next()`,
then the two will deadlock.  However, if the two clocks are waited on in
parallel, as \xcd`next;` does, {$a$} and {$b$} will not deadlock.)

Equivalently, \xcd`next;` sequentially calls \xcd`c.resume()` for each
registered clock \xcd`c`, in arbitrary order, and then \xcd`c.wait()` for each
clock, again in arbitrary order.  


%%OLIVIER-DENIES%% An \Xten{} computation is said to be {\em quiescent} on a clock
%%OLIVIER-DENIES%% \xcd"c" if each activity registered with \xcd"c" has resumed \xcd"c".
%%OLIVIER-DENIES%% Note that once a computation is quiescent on \xcd"c", it will remain
%%OLIVIER-DENIES%% quiescent on \xcd"c" forever (unless the system takes some action),
%%OLIVIER-DENIES%% since no other activity can become registered with \xcd"c".  That is,
%%OLIVIER-DENIES%% quiescence on a clock is a {\em stable property}.

%%OLIVIER-DENIES%% Once the implementation has detected quiescence on \xcd"c", the system
%%OLIVIER-DENIES%% marks all activities registered with \xcd"c" as being able to progress
%%OLIVIER-DENIES%% on \xcd"c". 
%%OLIVIER-DENIES%% 
An activity blocked on \xcd"next" resumes execution once
it is marked for progress by all the clocks it is registered with.

\subsection{Dropping clocks}\index{clock!drop}\label{sec:clock:drop}
%~~exp~~`~~`~~ c:Clock~~
An activity may drop a clock by executing \xcd`c.drop()`.



\noindent{} The activity is no longer considered registered with this
clock.  A \xcd"ClockUseException" is thrown if the activity has
already dropped \xcd"c".

\section{Deadlock Freedom}

In general, programs using clocks can deadlock, just as programs using loops
can fail to terminate.  However, programs written with a particular syntactic
discipline {\em are} guaranteed to be deadlock-free, just as programs which
use only bounded loops are guaranteed to terminate.  The syntactic discipline
is: 
\begin{itemize}
\item The \xcd`next()` {\bf method} may not be called on any clock. (The
      \xcd`next;` statement is allowed.)
\item Inside of \xcd`finish{S}`, all clocked \xcd`async`s must be in the scope
      an unclocked \xcd`async`.  
\end{itemize}


The second clause prevents the following deadlock.  
%~~gen
% package Clocks.Finish.Hates.Clocks;
% class Example{
% def example() {
%~~vis
\begin{xten}
val c:Clock = Clock.make();
async clocked(c) {                // (A) 
      finish async clocked(c) {   // (B) Violates clause 2
            next;                 // (Bnext)
      }
      next;                       // (Anext)
}
\end{xten}
%~~siv
% } } 
%~~neg
\xcd`(A)`, first of all, waits for the \xcd`finish` containing \xcd`(B)` to
finish.  
\xcd`(B)` will execute its \xcd`next` at \xcd`(Bnext)`, and then wait for all
other activities registered on \xcd`c` to execute their \xcd`next`s.
However, \xcd`(A)` is registered on \xcd`c`.  So, \xcd`(B)` cannot finish
until \xcd`(A)` has proceeded to \xcd`(Anext)`, and \xcd`(A)` cannot proceed
until \xcd`(B)` finishes. Thus, this causes deadlock.


\section{Program equivalences}
From the discussion above it should be clear that the following
equivalences hold:

\begin{eqnarray}
 \mbox{\xcd"c.resume(); next;"}       &=& \mbox{\xcd"next;"}\\
 \mbox{\xcd"c.resume(); d.resume();"} &=& \mbox{\xcd"d.resume(); c.resume();"}\\
 \mbox{\xcd"c.resume(); c.resume();"} &=& \mbox{\xcd"c.resume();"}
\end{eqnarray}

Note that \xcd"next; next;" is not the same as \xcd"next;". The
first will wait for clocks to advance twice, and the second
once.  

%\notinfouro{\subsection{Implementation Notes}
Clocks may be implemented efficiently with message passing, e.g.{} by
using short-circuit ideas in \cite{SaraswatPODC88}.  Recall that every
activity is spawned with references to a fixed number of clocks. Each
reference should be thought of as a global pointer to a location in
some place representing the clock. (We shall discuss a further
optimization below.) Each clock keeps two counters: the total number
of outstanding references to the clock, and the number of activities
that are currently suspended on the clock.

When an activity $A$ spawns another activity $B$ that will reference a
clock $c$ referenced by $A$, $A$ {\em splits} the reference by sending
a message to the clock. Whenever an activity drops a reference to a
clock, or suspends on it, it sends a message to the clock. 

The optimization is that the clock can be represented in a distributed
fashion. Each place keeps a local counter for each clock that is
referenced by an activity in that place. The global location for the
clock simply keeps track of the places that have references and that
are quiescent. This can reduce the inter-place message traffic
significantly.
}
%\notinfouro{\section{Clocked types}\index{types!clocked}

We allow types to specify clocks, via a {\cf clocked(c)} modifier,
e.g.{}

\begin{x10}
  clocked(c) int r;
\end{x10}

This declaration asserts that {\cf r} is accessible
(readable/writable) only by those statements that are clocked on {\cf
c}. Thus propagation of the clock provides some control over the
``visibility'' of {\cf r}.

The declaration 

\begin{x10}
  clocked(c) final int l = r;
\end{x10}

\noindent asserts additionally that in each clock instant {\cf l} is final, 
i.e.{} the value of {\cf l} may be reset at the beginning of each phase
of {\tt c} but stays constant during the phase.

This statement terminates when the computation of {\tt r} has
terminated and the assignment has been performed.

\todo{Generalize the syntax so that aggregate variables can be clocked with an aggregate clock of the same shape.}

\subsection{Clocked assignment}\index{assignment!clocked}
We expect that most arrays containing application data will be
declared to be {\cf clocked final}. We support this very powerful type
declaration by providing a new statement:
{\footnotesize
\begin{verbatim}
  next(c) l = r; 
\end{verbatim}}


\noindent 
for a variable $l$ declared to be clocked on $c$. The statement
assigns $r$ to the {\em next} value of $l$. There may be multiple such
assignments before the clock advances. The last such assignment
specifies the value of the variable that will be visible after the
clock has advanced.  If the variable is {\cf clocked final} it is
guaranteed that {\em all} readers of the variable throughout this
phase will see the value $r$.

The expression {\tt r} is implicitly treated as {\tt now(c) r}. That
is, the clock {\tt c} will not advance until the computation of {\tt r} has
terminated.

}
%\notinfouro{\section{Examples}
\todo{Bring in other examples from Concur paper.}
Consider the core of the ASCI Benchmark Sweep3D program for computing
solutions to mass transport problems.

In a nutshell the core computation is a triply nested sequential loop
in which the value of a variable in the current iteration is dependent
on the values of neighboring variables in a past iteration. Such a
problem can be parallelized through pipelining. One visualizes a
diagonal wavefront sweeping through the array. An MPI version of the
program may be described as follows. There is a two dimensional grid
of processors which performs the following computation
repeatedly. Each processor synchronously receives a value from the
processor to its west, then to its north, then computes some function
of these values and computes a new value to be sent to the processor
to its east and then to its south.  Ignoring the behavior of the
boundary processors for the moment such a computation may be described
by the following \Xten{} program:

\begin{x10}
region R = [1..n0,1..m0];
clock[R] W,N;
clock(W) final double [cyclic(R)] A; 
for (int t : 1..TMax) \{
  ateach( i,j:A) 
    clock (W[i-1,j],N[i,j-1],W[i,j],N[i,j]) \{
      double west = now (W[i-1,j]) future\{A[i-1,j]\}; 
      W[i-1,j].continue();           
      double north = now (N[i,j-1]) future\{A[i,j-1]\}; 
      N[i,j-1].continue();
      next(W[i,j]) A[i,j] = compute(west, north);
      next W[i-1,j],N[i,j-1],W[i,j],N[i,j];
  \}
\}
\end{x10}
}

\section{Clocked Finish}
\index{finish!clocked}
\index{async!clocked}
\index{clocked!finish}
\index{clocked!async}
\label{ClockedFinish}

In the most common case of a single clock coordinating a few behaviors, X10
allows coding with an implicit clock.  \xcd`finish` and \xcd`async` statements
may be qualified with \xcd`clocked`.  

A \xcd`clocked finish` introduces a new clock.  It executes its body in the
usual way that a \xcd`finish` does--- except that, when its body completes,
the activity executing the \xcd`clocked finish` drops the clock, while it
waits for asynchronous spawned \xcd`async`s to terminate.  

A \xcd`clocked async` registers its async with the implicit clock of
the surrounding \xcd`clocked finish`.   

Both the \xcd`clocked finish` and \xcd`clocked async` may use the \xcd`next`
statement to advance implicit clock.  Since the implicit clock is not
available in a variable, it cannot be manipulated directly. (If you want to
manipulate the clock directly, use an explicit clock.)

The following code starts two activities, each of which perform their first
phase, wait for the other to finish phase 1, and then perform their second
phase.  
%~~gen
%package Clocks.ClockedFinish;
%class Example{
%static def phase(String, Int) {}
%def example() {
%~~vis
\begin{xten}
clocked finish {
  clocked async {
     phase("A", 1);
     next;
     phase("A", 2);
  }
  clocked async {
     phase("B", 1);
     next;
     phase("B", 2);
  }
}
\end{xten}
%~~siv
%}}
%~~neg


\index{finish!nested clocked}
\index{clocked finish!nested}

Clocked finishes may be nested.  The inner \xcd`clocked finish` operates in a
single phase of the outer one.  
