\chapter{Classes}
\label{XtenClasses}\index{class}
\label{ReferenceClasses}

\section{Principles of X10 Objects}\label{XtenObjects}\index{Object}

\subsection{Basic Design}

Objects are instances of classes: the most common and most powerful sort of
value in X10.  The other kinds of values, structs and functions, are more
specialized, better in some circumstances but not in all.
\xcd"x10.lang.Object" is the most general class; all other classes inherit
from it, directly or indirectly. 


Classes are structured in a single-inheritance code
hierarchy, may implement multiple interfaces, may have static and
instance fields, may have static and instance methods, may have
constructors, 
%% vj: We dont support initializers.
% may have static and instance initializers, 
may have static and instance nested classes and interfaces. \Xten{} does not
permit mutable static state.

\Xten{} objects do not have locks associated with them.
Programmers should use atomic blocks (\Sref{AtomicBlocks}) for mutual
exclusion and clocks (\Sref{XtenClocks}) for sequencing multiple parallel
operations.

An object exists in a single location: the place that it was created.  One
place cannot directly refer to an object in a different place.   A
special type, \Xcd{GlobalRef[T]}, allows explicit cross-place references. 

The basic operations on objects are:
\begin{itemize}

{}\item Field access (\Sref{FieldAccess}). 
The static and instance fields of an object can be retrieved; \xcd`var` fields
can be set.  
%%ACCUM%% Accumulator fields can be updated, but only in limited contexts. 

{}\item Method invocation (\Sref{MethodInvocation}).  
Static and instance methods of an object can be invoked.

{}\item Casting (\Sref{ClassCast}) and instance testing with \xcd`instanceof`
(\Sref{instanceOf}) Objects can be cast or type-tested.  

\item The equality operators \xcd"==" and \xcd"!="
Objects can be compared for equality with the \Xcd{==} operation.  This checks
object {\em identity}: two objects are \Xcd{==} iff they are the same object.

\end{itemize}

  
 
\subsection{Class Declaration Syntax}

The {\em class declaration} has a list of type \params, properties, a
constraint (the {\em class invariant}), a single superclass, zero or more
interfaces, and a class body containing the the definition of fields,
properties, methods, and member types. Each such declaration introduces a
class type (\Sref{ReferenceTypes}).

\bard{Redo grammar}

%%OW  
%%OW  \begin{grammar}
%%OW  NormalClassDeclaration \:
%%OW        ClassModifiers\opt \xcd"class" Identifier  \\
%%OW     && TypeParameterList\opt PropertyList\opt Guard\opt \\
%%OW     && Super\opt Interfaces\opt ClassBody \\
%%OW  \\
%%OW  TypeParameterList     \:  \xcd"[" TypeParameters \xcd"]" \\
%%OW  TypeParameters        \:  TypeParameter ( \xcd"," TypeParameter )\star \\
%%OW  TypeParameter         \:  Variance\opt Annotation\star Identifier     \\
%%OW  Variance \: \xcd"+" \\
%%OW           && \xcd"-" \\
%%OW  \\
%%OW  PropertyList     \:  \xcd"(" Properties \xcd")" \\
%%OW  Properties       \:  Property ( \xcd"," Property )\star \\
%%OW  Property         \:  Annotation\star \xcd"val"\opt Identifier \xcd":" Type \\
%%OW  \\
%%OW  Super \: \xcd"extends" ClassType \\
%%OW  Interfaces \: \xcd"implements" InterfaceType ( \xcd"," InterfaceType)\star \\
%%OW  \\
%%OW  ClassBody \: ClassMember\star \\
%%OW  ClassMember \: ClassDeclaration \\
%%OW              \| InterfaceDeclaration \\
%%OW              \| FieldDeclaration \\
%%OW              \| MethodDeclaration \\
%%OW              \| ConstructorDeclaration \\
%%OW  \end{grammar}
%%OW  
 




\section{Fields}
\label{FieldDefinitions}

Objects may have {\em instance fields}, or simply {\em fields}: places to
store data that is pertinent to the object. Fields, like variables, may be
mutable (\xcd`var`) or immutable (\xcd`val`)
%%ACC%% , or accumulator (\Xcd{acc})
.

Class may have {\em static fields}, which store data pertinent to the
entire class of objects.  
See \Sref{StaticInitialization} for more information.

No two fields of the same class may have the same name.

To avoid an ambiguity, it is a static error to invoke  a field with a function
type (\Sref{FunctionTypes}) that has 
the same name and signature  as a method of the same class.  
(Consider the class 
\begin{xten}
class Crash {
  val f : (Int) => Boolean = (Int)=>true;
  def f(Int) = false;
}
\end{xten}
\noindent
Then \xcd`crash.f(3)` might either mean ``call the function \xcd`crash.f` on
argument \xcd`3`'', or ``invoke the method \xcd`f` on argument \xcd`3`''.)

\subsection{Field Initialization}

Fields may be given values via {\em field initialization expressions}:
\xcd`val f1 = E;` and \xcd`var f2 : Int = F;`. Other fields of \xcd`this` may
be referenced, but only those that {\em precede} the field being initialized.
For example, the following is correct, but would not be if the fields were
reversed:

%~~gen
%package Classes.Are.For.Lasses.Structs.Are.For.Bucks;
%~~vis
\begin{xten}
class Fld{
  val a = 1;
  val b = 2+a;
}
\end{xten}
%~~siv
%
%~~neg


\subsection{Field hiding}

A subclass that defines a field \xcd"f" hides any field \xcd"f"
declared in a superclass, regardless of their types.  The
superclass field \xcd"f" may be accessed within the body of
the subclass via the reference \xcd"super.f".

%~~gen
% package classes.fields.primus;
%~~vis
\begin{xten}
class Super{ 
  val f = 1; 
}
class Sub extends Super {
  val f = true;
  def superf() : Int = super.f; // 1
}
\end{xten}
%~~siv
%
%~~neg

With inner classes, it is occasionally necessary to 
write \xcd`Cls.super.f` to get at a hidden field \xcd`f` of an outer class
\xcd`Cls`, as in 
%~~gen
% package classes.fields.secundus;
%~~vis
\begin{xten}
class A {
   val f = 3;
}
class B extends A {
   val f = 4;
   class C extends B {
      // C is both a subclass and inner class of B
      val f = 5;
      def foo()
         = f          // 5
         + super.f    // 4
         + B.this.f   // 4 (the "f" of the outer instance)
         + B.super.f; // 3 (the "super.f" of the outer instance)
    }
}
\end{xten}
%~~siv
%
%~~neg


\subsection{Field qualifiers}
\label{FieldQualifier}
\index{Qualifier!field}

The behavior of a field may be changed by a field qualifier, such as
\xcd`static` or \xcd`transient`.  


\subsubsection{\Xcd{static} qualifier}

A \xcd`val` field may be declared to be {\em static}, as described in
\Sref{FieldDefinitions}. 

\subsection{\Xcd{transient} Qualifier}
\index{transient}
\index{field!transient}

A field may be declared to be {\em transient}.  Transient fields are excluded
from the deep copying that happens when information is sent from place to
place in an \Xcd{at} statement.    The value of a transient field of a copied
object is the default value of its type, regardless of the value of the field
in the original.  If the type of a field has no
default value, it cannot be marked \Xcd{transient}.
%~~gen
% package Classes.Transient.Example;
%~~vis
\begin{xten}
class Trans { 
   val copied = "copied";
   transient val transy : String = "a very long string";
   def example() {
      at (here) { // causes copying
         assert(this.copied == "copied");
         assert(this.transy == null);
      }
   }
}
\end{xten}
%~~siv
%
%~~neg

%%CLOCKED%% \subsection{\Xcd{clocked} Qualifier}
%%CLOCKED%% 
%%CLOCKED%% Clocked fields are discussed in \Sref{ClockedFields}.
%%CLOCKED%% 


\section{Properties}
\label{PropertiesInClasses}

The properties of an object (or struct) are  public \xcd`val` fields
usable at compile time in constraints.\footnote{In many cases, a 
\xcd`val` field can be upgraded to a \xcd`property`, which 
entails no compile-time or runtime cost.  Some cannot be, \eg, in cases where
cyclic structures of \xcd`val` fields are required.} 
For example,  every array has a \xcd`rank` telling
how many subscripts it takes.  User-defined classes can have whatever
properties are desired. 

Properties are defined in parentheses, after the name of the class.  They are
given values by the \xcd`property` command in constructors.
%~~gen
% package Classes.Toss.Freedom.Disk2;
%~~vis
\begin{xten}
class Proper(t:Int) {
  def this(t:Int) {property(t);}
}
\end{xten}
%~~siv
%
%~~neg




\begin{staticrule*}
It is a compile-time error for a class
defining a property \xcd"x: T" to have an ancestor class that defines
a property with the name \xcd"x".  
\end{staticrule*}

A property \xcd`x:T` induces a field with the same name and type, 
as if defined with: 
%~~gen
% package Classes.For.Masses.Of.NevermindTheRest;
% class Exampll[T] {
%~~vis
\begin{xten}
public val x : T;
\end{xten} 
%~~siv
% def this(y:T) { x=y; }
% }
%~~neg
\noindent It also defines a nullary getter method, 
%~~gen
% package Classes.For.Masses.Of.NevermindTheRest;
% class Exampllll[T] {
% public val x : T;
% def this(y:T) { x=y; }
%~~vis
\begin{xten}
public final def x()=x;
\end{xten}
%~~siv
%}
%~~neg

\noindent (As noted in \Sref{DepType:Interface}, interfaces can define
properties too. They define the same nullary getter methods, though they do
not require fields.)


\begin{staticrule*}
It is a compile-time error for a class or
interface defining a property \xcd"x :T" to have an existing method with
the signature \xcd"x(): T".
\end{staticrule*}


Properties are initialized by the invocation of a special \Xcd{property}
statement, which must be performed by each constructor of the class:
\begin{xten}
property(e1,..., en);
\end{xten}
The number and types of arguments to the \Xcd{property} statement must match
the number and types of the properties in the class declaration.  
Every constructor of a class with properties must invoke \xcd`property(...)`
precisely once; it is a static error if X10 cannot prove that this holds.

The requirement to use the \xcd`property` statement means that all properties
must be given values at the same time.  

By construction, the graph whose nodes are values and whose edges are
properties is acyclic.  \Eg, there cannot be values \xcd`a` and \xcd`b` with
properties \xcd`c` and \xcd`d` such that \xcd`a.c == b` and \xcd`b.d == a`.


\index{property!call}
\index{property!initialization}
\label{PropertyCall}







\section{Methods}

As is common in object-oriented languages, objects can have {\em methods}, of
two sorts.  {\em Static methods} are functions, conceptually associated with a
class and defined in its namespace.  {\em Instance methods} are parameterized
code bodies associated with an instance of the class, which execute with
convenient access to that instance's fields. 

Each method has a {\em signature}, telling what arguments it accepts, what
type it returns, what precondition it requires. Method definitions may be
overridden by subclasses; the overriding definition may have a declared return
type that is a subtype of the return type of the definition being overridden.
Multiple methods with the same name but different signatures may be provided
on a class (called ``overloading'' or ``ad hoc polymorphism''). Methods may be
declared \Xcd{public}, \Xcd{private}, \Xcd{protected}, or given default package-level access
rights.


%%GRAMMAR%%  \begin{grammar}
%%GRAMMAR%%  MethodDeclaration \: MethodHeader \xcd";" \\
%%GRAMMAR%%                    \| MethodHeader \xcd"="\opt ClosureBody \\
%%GRAMMAR%%  MethodHeader \:  
%%GRAMMAR%%    MethodModifiers\opt \xcd"def" Identifier TypeParameters\opt \\
%%GRAMMAR%%  && \xcd"(" 
%%GRAMMAR%%    FormalParameterList\opt \xcd")" Guard\opt \\
%%GRAMMAR%%    && ReturnType\opt \\
%%GRAMMAR%%  \end{grammar}

\index{parameter!var}
\index{parameter!val}
A formal parameter may have a \xcd"val", \xcd"var", or \Xcd{ref}
modifier; \xcd`val` is the default.
The body of the method is executed in an environment in which 
each formal parameter corresponds to a local variable (\xcd`var` iff the
formal parameter is \xcd`var`)
and is initialized with the value of the actual parameter.
%%REF%% Call-by-reference, \Xcd{ref} parameters, allows passing in variables for a
%%REF%% method to update, as described in \Sref{RefParameters}. 
%%REF%% 
%%REF%%  \subsection{\Xcd{ref} Parameters}
%%REF%%  \label{RefParameters}
%%REF%%  \index{ref}
%%REF%%  \index{parameter!ref}
%%REF%%  
%%REF%%  A \Xcd{ref} parameter allows a method to modify a \Xcd{var} variable that is
%%REF%%  available to the caller.  
%%REF%%  
%%REF%%  %~TODO~gen
%%REF%%  % package Classes.RefParameters.Primo.Examplo;
%%REF%%  % class Example {
%%REF%%  %~TODO~vis
%%REF%%  \begin{xten}
%%REF%%    def incr(ref n:Int):void {
%%REF%%       n += 1;
%%REF%%    }
%%REF%%    def caller() {
%%REF%%       var a : Int = 0;
%%REF%%       incr(a);
%%REF%%       assert(a == 1);
%%REF%%    }
%%REF%%  \end{xten}
%%REF%%  %~TODO~siv
%%REF%%  % }
%%REF%%  %~TODO~neg
%%REF%%  

\subsection{Method Guards}
\label{MethodGuard}

Often, a method will only make sense to invoke under certain
statically-determinable conditions.  For example, \xcd`example(x)` is only
well-defined when \xcd`x != null`, as \xcd`null.toString()` throws a null
pointer exception: 
%~~gen
% package Classes.Make.Passes.At.Girls.Who.Wear.Glasses;
%~~vis
\begin{xten}
class Example {
   var f : String = "";
   def example(x:Object){x != null} = {
      this.f = x.toString();
   }
}
\end{xten}
%~~siv
%
%~~neg
\noindent
(We could have used a constrained type \xcd`Object{self!=null}` instead; in
most cases it is a matter of personal preference or convenience of expression
which one to use.) 

The requirement of having a method guard is that callers must demonstrate to
the X10
compiler that the guard is satisfied.  (As usual with static constraint
checking, there is no runtime cost.  Indeed, this code can be more efficient
than usual, as it is statically provable that \xcd`x != null`.)
This may require a cast: 
%~~gen
% package Classes.Make.Asses.Of.Girls.With.Fake.Passes;
% class Example {var f : String = ""; def example(x:Object){x != null} = {this.f = x.toString();}}
% class Eyample {
%~~vis
\begin{xten}
  def exam(e:Example, x:Object) {
    if (x != null) 
       e.example(x as Object{x != null});
    // WRONG: if (x != null) e.example(x);
  }
\end{xten}
%~~siv
%}
%~~neg

The guard \xcd`{c}` 
in a guarded method 
\xcd`def m(){c} = E;`
specifies a constraint \xcd"c" on the
properties of the class \xcd"C" on which the method is being defined. The
method exists only for those instances of \xcd"C" which satisfy \xcd"c".  It is
illegal for code to invoke the method on objects whose static type is
not a subtype of \xcd"C{c}".

\begin{staticrule*}
    The compiler checks that every method invocation
    \xcdmath"o.m(e$_1$, $\dots$, e$_n$)"
    is type correct. Each argument
    \xcdmath"e$_i$" must have a
    static type \xcdmath"S$_i$" that is a subtype of the declared type
    \xcdmath"T$_i$" for the $i$th
    argument of the method, and the conjunction of the constraints on the
    static types 
    of the arguments must entail the guard in the parameter list
    of the method.

    The compiler checks that in every method invocation
    \xcdmath"o.m(e$_1$, $\dots$, e$_n$)"
    the static type of \xcd"o", \xcd"S", is a subtype of \xcd"C{c}", where the method
    is defined in class \xcd"C" and the guard for \xcd"m" is equivalent to
    \xcd"c".

    Finally, if the declared return type of the method is
    \xcd"D{d}", the
    return type computed for the call is
    \xcdmath"D{a: S; x$_1$: S$_1$; $\dots$; x$_n$: S$_n$; d[a/this]}",
    where \xcd"a" is a new
    variable that does not occur in
    \xcdmath"d, S, S$_1$, $\dots$, S$_n$", and
    \xcdmath"x$_1$, $\dots$, x$_n$" are the formal
    parameters of the method.
\end{staticrule*}

\limitation{
Using a reference to an outer class, \xcd`Outer.this`, in a constraint, is not supported.
}


\subsection{Property methods}

Property methods are methods that can be evaluated in constraints.  
For example, the \xcd`eq()` method below tells if the \xcd`x` and \xcd`y`
properties are equal; the \xcd`is(z)` method tells if they are both equal to
\xcd`z`.  These can be used in constraints, as illustrated in the
\xcd`example()` method.
%~~gen
%package Classes.PropertyMethods;
%~~vis
\begin{xten}
class Example(x:Int, y:Int) {
   def this(x:Int, y:Int) { property(x,y); }
   property eq() = (x==y);
   property is(z:Int) = x==z && y==z;
   def example( a : Example{eq()}, b : Example{is(3)} ) {}
}
\end{xten}
%~~siv
%
%~~neg


A method declared with the modifier \xcd"property" may be used
in constraints.  A property method declared in a class must have
a body and must not be \xcd"void".  The body of the method must
consist of only a single \xcd"return" statement or a single
expression.  It is a static error if the expression cannot be
represented in the constraint system. 

The expression may contain invocations of other property methods. It is the
responsibility of the programmer to ensure that the evaluation of a property
terminates at compile-time, otherwise the type-checker will not terminate and
the program will fail to compile in a potentially most unfortunate way.

Property methods in classes are implicitly \xcd"final"; they cannot be
overridden.

A nullary property method definition may omit the formal parameters and
the \xcd"def" keyword.  That is, the following are equivalent:



%~~gen
% package classes.not.harpoons.instead.of.python;
% class Waif(rect:Boolean, onePlace:Place, zeroBased:Boolean) {
%~~vis
\begin{xten}
property def rail(): Boolean = rect && onePlace == here && zeroBased;
\end{xten}
%~~siv
%}
%~~neg
and
%~~gen
% package classes.not.muskrats.instead.of.blueberry.cake.with.honey;
% class Waif(rect:Boolean, onePlace:Place, zeroBased:Boolean) {
%~~vis
\begin{xten}
property rail: Boolean = rect && onePlace == here && zeroBased;
\end{xten}
%~~siv
%}
%~~neg

Similarly, nullary property methods can be inspected in constraints without
\xcd`()`.  
%~~longexp~~`~~`
% package classes.not.weasels;
% class Waif(rect:Boolean, onePlace:Place, zeroBased:Boolean) {
%   def this(rect:Boolean, onePlace:Place, zeroBased:Boolean) 
%          :Waif{self.rect==rect, self.onePlace==onePlace, self.zeroBased==zeroBased}
%          = {property(rect, onePlace, zeroBased);}
%   property rail: Boolean = rect && onePlace == here && zeroBased;
%   static def zoink() {
%      val w : Waif{
%~~vis
\xcd`w.rail`, with either definition above, 
% }= new Waif(true, here, true);
% }}
%~~pxegnol
is equivalent to 
%~~longexp~~`~~`
% package classes.not.ferrets;
% class Waif(rect:Boolean, onePlace:Place, zeroBased:Boolean) {
%   def this(rect:Boolean, onePlace:Place, zeroBased:Boolean) 
%          :Waif{self.rect==rect, self.onePlace==onePlace, self.zeroBased==zeroBased}
%          = {property(rect, onePlace, zeroBased);}
%   property rail: Boolean = rect && onePlace == here && zeroBased;
%   static def zoink() {
%      val w : Waif{
%~~vis
\xcd`w.rail()`
% }= new Waif(true, here, true);
% }}
%~~pxegnol



\subsection{Method overloading, overriding, hiding, shadowing and obscuring}
\label{MethodOverload}


The definitions of method overloading, overriding, hiding, shadowing
and obscuring in \Xten{} are the same as in \Java, modulo the following
considerations motivated by type parameters and dependent types.

Two or more methods of a class or interface may have the same
name if they have a different number of type parameters, or
they have formal parameters of different types.  \Eg, the following is legal: 

%~~gen
% package Classes.Mful;
%~~vis
\begin{xten}
class Mful{
   def m() = 1;
   def m[T]() = 2;
   def m(x:Int) = 3;
   def m[T](x:Int) = 4;
}
\end{xten}
%~~siv
%
%~~neg

\XtenCurrVer{} does not permit overloading based on constraints. That is, the
following is {\em not} legal, although either method definition individually
is legal:
\begin{xten}
   def n(x:Int){x==1} = "one";
   def n(x:Int){x!=1} = "not";
\end{xten}


The definition of a method declaration \xcdmath"m$_1$" ``having the same signature
as'' a method declaration \xcdmath"m$_2$" involves identity of types. 

The {\em constraint erasure} of a type \xcdmath"T" is defined as follows.
The constraint erasure of  (a)~a class, interface or struct type \xcdmath"T" is 
\xcdmath"T"; (b)~a type \xcdmath"T{c}" is the constraint erasure of 
\xcdmath"T"; (b)~a type \xcdmath"T[S$_1$,$\ldots$,S$_n$]" 
is \xcdmath"T'[S$_1$',$\ldots$,S$_n$']" where each primed type is the erasure of 
the corresponding unprimed type.
 Two methods are said to have {\em the
  same signature} if (a) they have the same number of type parameters,
(b) they have the same number of formal (value) parameters, and (c)
for each formal parameter the constraint erasure of its types are equivalent. It is a
compile-time error for there to be two methods with the same name and
same signature in a class (either defined in that class or in a
superclass).

\begin{staticrule*}
  A class \xcd"C" may not have two declarations for a method named \xcd"m"---either
  defined at \xcd"C" or inherited:
\begin{xtenmath}
def m[X$_1$, $\dots$, X$_m$](v$_1$: T$_1$, $\dots$, v$_n$: T$_n$){tc}: T {...}
def m[X$_1$, $\dots$, X$_m$](v$_1$: S$_1$, $\dots$, v$_n$: S$_n$){sc}: S {...}
\end{xtenmath}
\noindent
if it is the case that the constraint erasures of the types \xcdmath"T$_1$",
\dots, \xcdmath"T$_n$" are
equivalent to the constraint erasures of the types \xcdmath"S$_1$, $\dots$, T$_n$"
respectively.
\end{staticrule*}

In addition, the guard of a overriding method must be 
no stronger than the guard of the overridden method.   This
ensures that any virtual call to the method
satisfies the guard of the callee.

\begin{staticrule*}
  If a class \xcd"C" overrides a method of a class or interface
  \xcd"B", the guard of the method in \xcd"B" must entail
  the guard of the method in \xcd"C".
\end{staticrule*}

A class \xcd"C" inherits from its direct superclass and superinterfaces all
their methods visible according to the access modifiers
of the superclass/superinterfaces that are not hidden or overridden. A method \xcdmath"M$_1$" in a class
\xcd"C" overrides
a method \xcdmath"M$_2$" in a superclass \xcd"D" if
\xcdmath"M$_1$" and \xcdmath"M$_2$" have the same signature with constraints erased.
Methods are overriden on a signature-by-signature basis.

% Yoav says that the following needs complete revision.  I think the method 
% resolution section is that complet revision.
%%SEE-METHOD-RESOLUTION%% A method invocation \xcdmath"o.m(e$_1$, $\dots$, e$_n$)"
%%SEE-METHOD-RESOLUTION%% is said to have the {\em static signature}
%%SEE-METHOD-RESOLUTION%% \xcdmath"<T, T$_1$, $\dots$, T$_n$>" where \xcd"T" is the static type of
%%SEE-METHOD-RESOLUTION%% \xcd"o", and
%%SEE-METHOD-RESOLUTION%% \xcdmath"T$_1$",
%%SEE-METHOD-RESOLUTION%% \dots,
%%SEE-METHOD-RESOLUTION%% \xcdmath"T$_n$"
%%SEE-METHOD-RESOLUTION%% are the static types of \xcdmath"e$_1$", \dots, \xcdmath"e$_n$",
%%SEE-METHOD-RESOLUTION%% respectively.  
%%SEE-METHOD-RESOLUTION%% It must be the case that the compiler can determine a single
%%SEE-METHOD-RESOLUTION%% method defined on \xcd"T" with argument type
%%SEE-METHOD-RESOLUTION%% \xcdmath"T$_1$", \dots \xcdmath"T$_n$"; otherwise, a
%%SEE-METHOD-RESOLUTION%% compile-time error is declared. However,  the \Xten{} type \xcd"T"
%%SEE-METHOD-RESOLUTION%% may be a dependent type \xcd"C{c}". Therefore, given a class definition for
%%SEE-METHOD-RESOLUTION%% \xcd"C" we must determine which methods of \xcd"C" are available at a type
%%SEE-METHOD-RESOLUTION%% \xcd"C{c}". But the answer to this question is clear: exactly those methods
%%SEE-METHOD-RESOLUTION%% defined on \xcd"C" are available at the type \xcd"C{c}"
%%SEE-METHOD-RESOLUTION%% whose guard \xcd"d" is implied by \xcd"c".


%%OUTDATED%% \section{Instance Initialization}
%%OUTDATED%% 
%%OUTDATED%% \section{Constructors}
%%OUTDATED%% 
%%OUTDATED%% Constructors allow the initialization of objects by the execution of
%%OUTDATED%% almost-arbitrary code.  Like methods, constructors can have formal parameters,
%%OUTDATED%% a constraint, a return type, 
%%OUTDATED%% and a body. The formals and constraint
%%OUTDATED%%  are identical to those for a method.
%%OUTDATED%% A constructor is declared by \xcd`def this()...`; that is, as if it were a
%%OUTDATED%% method whose name were the reserved word \xcd`this`.  
%%OUTDATED%% 
%%OUTDATED%% 
%%OUTDATED%% \subsection{Constructor Return Types}
%%OUTDATED%% 
%%OUTDATED%% The return type of a constructor describes the values that constructor can
%%OUTDATED%% create.  While all constructors for class \xcd`C` create objects of base class
%%OUTDATED%% \xcd`C`, some individual constructors may construct objects with more specific
%%OUTDATED%% constraints.    For example, in
%%OUTDATED%% %~x~gen
%%OUTDATED%% % package Classes.Are.For.Grasses;
%%OUTDATED%% %~x~vis
%%OUTDATED%% \begin{xten}
%%OUTDATED%% class Crate(n:Int) {
%%OUTDATED%%   def this() : Crate{self.n==0} = { property(0); }
%%OUTDATED%%   def this(b:Boolean) : Crate{self.n==1} = { property(1); }
%%OUTDATED%% }
%%OUTDATED%% \end{xten}
%%OUTDATED%% %~x~siv
%%OUTDATED%% %
%%OUTDATED%% %~x~neg
%%OUTDATED%% \noindent
%%OUTDATED%% the nullary constructor call \xcd`new Crate()` will return a value of type 
%%OUTDATED%% \xcd`Crate{self.n == 0}`--- the \xcd`n` field is zero and the compiler knows
%%OUTDATED%% it.  The unary Boolean constructor will return an object of type 
%%OUTDATED%% \xcd`Crate{self.n==1}`.
%%OUTDATED%% A less trivial example might be a specialized constructor for a square matrix, 
%%OUTDATED%% which returned type \xcd`Matrix{self.rows==self.cols}`.  
%%OUTDATED%% 
%%OUTDATED%% If the constructor type is omitted, the constructor returns 
%%OUTDATED%% the type of its class, constrained by the actual parameters to the
%%OUTDATED%% \xcd`property` call in the constructor.  That is, the first constructor 
%%OUTDATED%% call above could be abbreviated:
%%OUTDATED%% %~x~gen
%%OUTDATED%% % package Classes.Are.For.Grasses.In.Mountain.Passes;
%%OUTDATED%% % class Crate(n:Int) {
%%OUTDATED%% %~x~vis
%%OUTDATED%% \begin{xten}
%%OUTDATED%%   def this() { property(0); }
%%OUTDATED%%   // And to prove that the nullary constructor knows n==0: 
%%OUTDATED%%   static def confirm() {
%%OUTDATED%%     val v : Crate{self.n == 0} = new Crate();
%%OUTDATED%%   }
%%OUTDATED%% \end{xten}
%%OUTDATED%% %~x~siv
%%OUTDATED%% % }
%%OUTDATED%% %~x~neg
%%OUTDATED%% 
%%OUTDATED%% \subsection{Constructor Bodies}
%%OUTDATED%% 
%%OUTDATED%% Constructors have many restrictions, designed to ensure that objects behave
%%OUTDATED%% sanely while being constructed. Constructors initialize fields of objects and
%%OUTDATED%% establish object invariants.  It should never be possible to observe an
%%OUTDATED%% uninitialized object outside of a constructor call.  However, without some
%%OUTDATED%% restrictions, it would be possible to do so quite easily.  A constructor could
%%OUTDATED%% put \Xcd{this} into a data structure before initializing it, and a concurrent
%%OUTDATED%% activity could read it before the initialization is finished.  If the 
%%OUTDATED%% line marked \Xcd{ILLEGAL} were uncommented, the following program would
%%OUTDATED%% exhibit the problem: 
%%OUTDATED%% %~x~gen
%%OUTDATED%% % package Classes.ConstructorBodies.Methane.Ammonia.Atmosphere;
%%OUTDATED%% % import x10.util.*;
%%OUTDATED%% %~x~vis
%%OUTDATED%% \begin{xten}
%%OUTDATED%% class Escaper {
%%OUTDATED%%    static val all = new ArrayList[Escaper]();
%%OUTDATED%%    val fld : String;
%%OUTDATED%%    def this() {
%%OUTDATED%%       //ILLEGAL: all.add(this);
%%OUTDATED%%       this.fld = "initialized";
%%OUTDATED%%    }
%%OUTDATED%%    def Main(Array[String](1)):void {
%%OUTDATED%%       finish{
%%OUTDATED%%         async {new Escaper();}
%%OUTDATED%%         async {
%%OUTDATED%%           for (e in Escaper.all) 
%%OUTDATED%%             assert(e.fld == "initialized");
%%OUTDATED%%         }
%%OUTDATED%%       }
%%OUTDATED%%    }
%%OUTDATED%% }
%%OUTDATED%% \end{xten}
%%OUTDATED%% %~x~siv
%%OUTDATED%% %
%%OUTDATED%% %~x~neg
%%OUTDATED%% 
%%OUTDATED%% \subsubsection{Initialization of Fields}
%%OUTDATED%% 
%%OUTDATED%% All fields of an object must be given a value before they are used.  There are
%%OUTDATED%% several ways that this can be accomplished: 
%%OUTDATED%% \begin{itemize}
%%OUTDATED%% \item A \Xcd{var} field of a type with a default value, if not otherwise
%%OUTDATED%%       initialized, 
%%OUTDATED%%       will have its type's default value. (\Xcd{val} fields may not use this
%%OUTDATED%%       option; they must be given values explicitly, either by an initializer
%%OUTDATED%%       or by a constructor.)  For example, \Xcd{(new C()).zero}
%%OUTDATED%%       in the example below will be zero, the default value of an \Xcd{Int}.
%%OUTDATED%% %~x~gen
%%OUTDATED%% % package Classes.Initofields.One;
%%OUTDATED%% %~x~vis
%%OUTDATED%% \begin{xten}
%%OUTDATED%% class C {
%%OUTDATED%%   var zero : Int;
%%OUTDATED%% }
%%OUTDATED%% \end{xten}
%%OUTDATED%% %~x~siv
%%OUTDATED%% %
%%OUTDATED%% %~x~neg
%%OUTDATED%% 
%%OUTDATED%% \item A field may have an explicit initializer, in which case its value is the
%%OUTDATED%%       value of the initializer.  \Xcd{(new D()).one} is one.
%%OUTDATED%% %~x~gen
%%OUTDATED%% % package Classes.Initofields.Two;
%%OUTDATED%% %~x~vis
%%OUTDATED%% \begin{xten}
%%OUTDATED%% class D {
%%OUTDATED%%   val one = 1;
%%OUTDATED%% }
%%OUTDATED%% \end{xten}
%%OUTDATED%% %~x~siv
%%OUTDATED%% %
%%OUTDATED%% %~x~neg
%%OUTDATED%% 
%%OUTDATED%% \item A field may be given a value in a constructor, or in a method called
%%OUTDATED%%       from a constructor.  \Xcd{(new E()).x} is two; \Xcd{(new E(3)).x} is
%%OUTDATED%%       three. 
%%OUTDATED%% %~x~gen
%%OUTDATED%% % package Classes.Initofields.Three;
%%OUTDATED%% %~x~vis
%%OUTDATED%% \begin{xten}
%%OUTDATED%% class E {
%%OUTDATED%%    var x : Int;
%%OUTDATED%%    def this() { x = 2; }
%%OUTDATED%%    def this(n: Int) { 
%%OUTDATED%%       this.setX(n);
%%OUTDATED%%    }
%%OUTDATED%%    private final def setX(n:Int) {
%%OUTDATED%%      this.x = n;
%%OUTDATED%%    }
%%OUTDATED%% }
%%OUTDATED%% \end{xten}
%%OUTDATED%% %~x~siv
%%OUTDATED%% %
%%OUTDATED%% %~x~neg
%%OUTDATED%% 
%%OUTDATED%% 
%%OUTDATED%% \end{itemize}
%%OUTDATED%% 
%%OUTDATED%% \subsubsection{Calling Another Constructor}
%%OUTDATED%% 
%%OUTDATED%% A constructor for class \Xcd{C} can call another constructor, either of
%%OUTDATED%% \Xcd{C} itself or of \Xcd{C}'s supertype.  The call to the other constructor
%%OUTDATED%% must be the first statement of the constructor body.
%%OUTDATED%% 
%%OUTDATED%% %~x~gen
%%OUTDATED%% % package Classical.Constructorical.Calling.Another.One;
%%OUTDATED%% %~x~vis
%%OUTDATED%% \begin{xten}
%%OUTDATED%% class C {
%%OUTDATED%%   val a : Int;
%%OUTDATED%%   def this(a:Int) { this.a = a; }
%%OUTDATED%%   def this() { 
%%OUTDATED%%      this(0); // call another constructor of C
%%OUTDATED%%   }
%%OUTDATED%% }
%%OUTDATED%% class D extends C {
%%OUTDATED%%   val b : Int; 
%%OUTDATED%%   def this(a:Int, b:Int) {
%%OUTDATED%%      super(a); // call superclass constructor
%%OUTDATED%%      this.b = b;
%%OUTDATED%%   }
%%OUTDATED%% }
%%OUTDATED%% \end{xten}
%%OUTDATED%% %~x~siv
%%OUTDATED%% %
%%OUTDATED%% %~x~neg
%%OUTDATED%% 
%%OUTDATED%% No fields of \Xcd{this} may be read before the optional \Xcd{this}- or
%%OUTDATED%% \Xcd{super}-call. For example, having \Xcd{super(this.x);} as the
%%OUTDATED%% constructor-call is illegal, since it reads \Xcd{this.x} before the
%%OUTDATED%% constructor-call.
%%OUTDATED%% 
%%OUTDATED%% \subsubsection{\Xcd{property} Statement}
%%OUTDATED%% 
%%OUTDATED%% The \Xcd{property} statement sets the properties of \Xcd{this}; note that
%%OUTDATED%% properties may not be assigned to directly.  For example, 
%%OUTDATED%% \Xcd{(new C(10)).x == 10}.
%%OUTDATED%% %~x~gen
%%OUTDATED%% % package Classes.Propertystatement.Squid.Stuffing;
%%OUTDATED%% %~x~vis
%%OUTDATED%% \begin{xten}
%%OUTDATED%% class C(x:Int, y:Boolean) {
%%OUTDATED%%    def this(x:Int) {
%%OUTDATED%%       property(x, true);
%%OUTDATED%%    }
%%OUTDATED%% }
%%OUTDATED%% 
%%OUTDATED%% \end{xten}
%%OUTDATED%% %~x~siv
%%OUTDATED%% %
%%OUTDATED%% %~x~neg
%%OUTDATED%% 
%%OUTDATED%% The properties of \Xcd{this} may be read after the \Xcd{property} statement,
%%OUTDATED%% but not before it: 
%%OUTDATED%% %~x~gen
%%OUTDATED%% % package Classes.Propertystatement.Squid.Stuffing.Pillows;
%%OUTDATED%% %~x~vis
%%OUTDATED%% \begin{xten}
%%OUTDATED%% class C(x:Int, y:Boolean) {
%%OUTDATED%%    val z : Int;
%%OUTDATED%%    def this(x:Int) {
%%OUTDATED%%       // x cannot be read here.
%%OUTDATED%%       property(x, true);
%%OUTDATED%%       this.z = x+3;
%%OUTDATED%%    }
%%OUTDATED%% }
%%OUTDATED%% \end{xten}
%%OUTDATED%% %~x~siv
%%OUTDATED%% %
%%OUTDATED%% %~x~neg
%%OUTDATED%% 
%%OUTDATED%% 
%%OUTDATED%% \subsubsection{Use of \Xcd{this} in a Constructor Body}
%%OUTDATED%% 
%%OUTDATED%% After the optional constructor-call and optional \Xcd{property} statement,
%%OUTDATED%% \Xcd{this} {\em can} be used, subject to certain restrictions. 
%%OUTDATED%% We say that a method is {\em constructor-like} if it is \Xcd{private},
%%OUTDATED%% \Xcd{final}, and obeys the following restrictions.
%%OUTDATED%% 
%%OUTDATED%% \begin{itemize}
%%OUTDATED%% \item \Xcd{this} may not be assigned to anything. This prevents problematic
%%OUTDATED%%       cases such as: 
%%OUTDATED%% %~x~gen
%%OUTDATED%% % package Classes.Usethisinactorbodytwo;
%%OUTDATED%% %~x~vis
%%OUTDATED%% \begin{xten}
%%OUTDATED%% class Outer {
%%OUTDATED%%    var leak : Inner;
%%OUTDATED%%    class Inner {
%%OUTDATED%%       def this() {
%%OUTDATED%%         //ILLEGAL: Outer.this.leak = this;
%%OUTDATED%%       }
%%OUTDATED%%    }
%%OUTDATED%% }
%%OUTDATED%% \end{xten}
%%OUTDATED%% %~x~siv
%%OUTDATED%% %
%%OUTDATED%% %~x~neg
%%OUTDATED%% (It also prevents harmless cases, such as \Xcd{val nonleak = this;}
%%OUTDATED%% 
%%OUTDATED%% \item \Xcd{this} may only be used as an argument or receiver of a
%%OUTDATED%%       constructor-like method call.  The following class definition is legal.
%%OUTDATED%% %~x~gen
%%OUTDATED%% % package Classes.Usethisinactorbodyone; 
%%OUTDATED%% %~x~vis
%%OUTDATED%% \begin{xten}
%%OUTDATED%% class C {
%%OUTDATED%%   private final def ctorLike() {
%%OUTDATED%%     x10.io.Console.OUT.println("constructed");
%%OUTDATED%%   }
%%OUTDATED%%   def this() {
%%OUTDATED%%     this.ctorLike();
%%OUTDATED%%   }
%%OUTDATED%% }
%%OUTDATED%% \end{xten}
%%OUTDATED%% %~x~siv
%%OUTDATED%% %
%%OUTDATED%% %~x~neg
%%OUTDATED%%       It would not be legal if \Xcd{private} or \Xcd{final} were omitted, or
%%OUTDATED%%       if \Xcd{ctorLike} called \Xcd{x10.io.Console.OUT.println(this +
%%OUTDATED%%       "constructed");}.  (The implicit \Xcd{this.toString()} call is a call to
%%OUTDATED%%       a non-constructor-like method.)
%%OUTDATED%% \end{itemize}
%%OUTDATED%% 
%%OUTDATED%% Constructor code may refer to fields of \Xcd{this} only after they have been
%%OUTDATED%% definitely assigned.\Sref{DefiniteAssign}  For example: 
%%OUTDATED%% %~x~gen
%%OUTDATED%% %package Classes.ctordefassignuse;
%%OUTDATED%% %~x~vis
%%OUTDATED%% \begin{xten}
%%OUTDATED%% class C {
%%OUTDATED%%   val d : Int, e:Int; var f: Int;
%%OUTDATED%%   def this() {
%%OUTDATED%%     // ILLEGAL: f = d - 1; 
%%OUTDATED%%     d = 1;
%%OUTDATED%%     e = d + 2;
%%OUTDATED%%     
%%OUTDATED%%   }
%%OUTDATED%% }
%%OUTDATED%% \end{xten}
%%OUTDATED%% %~x~siv
%%OUTDATED%% %
%%OUTDATED%% %~x~neg
%%OUTDATED%% 
%%OUTDATED%% 
%%OUTDATED%% The bodies of constructor-like methods called from constructors may only read
%%OUTDATED%% fields of \Xcd{this} which have been initialized at the point that the method
%%OUTDATED%% was called.  They differ from true constructor bodies in that only 
%%OUTDATED%% constructor bodies, not constructor-like methods, may assign to \Xcd{val}
%%OUTDATED%% fields. 
%%OUTDATED%% %~x~gen
%%OUTDATED%% % package Classes.Usethisinactorbodythree;
%%OUTDATED%% %~x~vis
%%OUTDATED%% \begin{xten}
%%OUTDATED%% class C {
%%OUTDATED%%   val m: Int;
%%OUTDATED%%   var n:Int; 
%%OUTDATED%%   private final def ctorLike() {
%%OUTDATED%%      n = m + 3;
%%OUTDATED%%      }
%%OUTDATED%%   def this() { 
%%OUTDATED%%      //ILLEGAL: ctorLike();
%%OUTDATED%%      m = 7;
%%OUTDATED%%      ctorLike();
%%OUTDATED%%      }
%%OUTDATED%% }
%%OUTDATED%% \end{xten}
%%OUTDATED%% %~x~siv
%%OUTDATED%% %
%%OUTDATED%% %~x~neg
%%OUTDATED%% 
%%OUTDATED%% 
%%OUTDATED%% 
\section{Static initialization}
\label{StaticInitialization}
\index{initialization!static}
The \Xten{} runtime implements the following procedure to ensure
reliable initialization of the static state of classes.


Execution commences with a single thread executing the
\emph{initialization} phase of an \Xten{} computation at place \Xcd{0}. This
phase must complete successfully before the body of the \Xcd{main} method is
executed.

The initialization phase must be thought of as if it is implemented in
the following fashion: (The implementation may do something more
efficient as long as it is faithful to this semantics.)

\begin{xten}
Within the scope of a new finish
for every static field f of every class C 
   (with type T and initializer e):
async {
  val l = e; 
  ateach (Dist.makeUnique()) {
     assign l to the static f field of 
         the local C class object;
     mark the f field of the local C 
         class object as initialized;
  }
}
\end{xten}

During this phase, any read of a static field \Xcd{C.f} (where \Xcd{f} is of type \Xcd{T})
is replaced by a call to the method \Xcd{C.read\_f():T} defined on class \Xcd{C}
as follows

\begin{xten}
def read_f():T {
   await (initialized(C.f));
   return C.f;
}
\end{xten}
 

If all these activities terminate normally, all static fields have values of
their declared types, 
and the \Xcd{finish} terminates normally. If
any activity throws an exception, the \Xcd{finish} throws an
exception. Since no user code is executing which can catch exceptions
thrown by the finish, such exceptions are printed on the console, and
computation aborts.

If the activities deadlock, the implementation deadlocks.

In all cases, the main method is executed only once all static fields
have been initialized correctly.

Since static state is immutable and is replicated to all places via 
the initialization phase as described above, it can be accessed from
any place.



\section{User-Defined Operators}

It is often convenient to have methods named by symbols rather than words.
For example, suppose that we wish to define a \xcd`Poly` class of
polynomials -- for the sake of illustration, single-variable polynomials with
\xcd`Int` coefficients.  It would be very nice to be able to manipulate these
polynomials by the usual operations: \xcd`+` to add, \xcd`*` to multiply,
\xcd`-` to subtract, and \xcd`p(x)` to compute the value of the polynomial at
argument \xcd`x`.  We would like to write code thus: 
%~~gen

% package Classes.In.Poly101;
% // Integer-coefficient polynomials of one variable.
% class Poly {
%   public val coeff : Array[Int](1);
%   public def this(coeff: Array[Int](1)) { this.coeff = coeff;}
%   public def degree() = coeff.size()-1;
%   public  def  a(i:Int) = (i<0 || i>this.degree()) ? 0 : coeff(i);
%
%   public static operator (c : Int) as Poly = new Poly([c]);
%
%   public def apply(x:Int) {
%     val d = this.degree();
%     var s : Int = this.a(d);
%     for( [i] in 1 .. this.degree() ) {
%        s = x * s + a(d-i);
%     }
%     return s;
%   }
%
%   public operator this + (p:Poly) =  new Poly(
%      new Array[Int](
%         Math.max(this.coeff.size(), p.coeff.size()),
%         (i:Int) => this.a(i) + p.a(i)
%      ));
%   public operator this - (p:Poly) = this + (-1)*p;
%
%   public operator this * (p:Poly) = new Poly(
%      new Array[Int](
%        this.degree() + p.degree() + 1,
%        (k:Int) => sumDeg(k, this, p)
%        )
%      );
%
%
%   public operator (n : Int) + this = (n as Poly) + this;
%   public operator this + (n : Int) = (n as Poly) + this;
%
%   public operator (n : Int) - this = (n as Poly) + (-1) * this;
%   public operator this - (n : Int) = ((-n) as Poly) + this;
%
%   public operator (n : Int) * this = new Poly(
%      new Array[Int](
%        this.degree()+1,
%        (k:Int) => n * this.a(k)
%      ));
%   private static def sumDeg(k:Int, a:Poly, b:Poly) {
%      var s : Int = 0;
%      for( [i] in 0 .. k ) s += a.a(i) * b.a(k-i);
%        // x10.io.Console.OUT.println("sumdeg(" + k + "," + a + "," + b + ")=" + s);
%      return s;
%      };
%   public final def toString() = {
%      var allZeroSoFar : Boolean = true;
%      var s : String ="";
%      for( [i] in 0..this.degree() ) {
%        val ai = this.a(i);
%        if (ai == 0) continue;
%        if (allZeroSoFar) {
%           allZeroSoFar = false;
%           s = term(ai, i);
%        }
%        else
%           s +=
%              (ai > 0 ? " + " : " - ")
%             +term(ai, i);
%      }
%      if (allZeroSoFar) s = "0";
%      return s;
%   }
%   private final def term(ai: Int, n:Int) = {
%      val xpow = (n==0) ? "" : (n==1) ? "x" : "x^" + n ;
%      return (ai == 1) ? xpow : "" + Math.abs(ai) + xpow;
%   }
%
%   public static def Main(ss:Array[String](1)) = main(ss);
%


%~~vis
\begin{xten}
  public static def main(Array[String](1)):void {
     val X = new Poly([0,1]);
     val t <: Poly = 7 * X + 6 * X * X * X; 
     val u <: Poly = 3 + 5*X - 7*X*X;
     val v <: Poly = t * u - 1;
     for( [i] in -3 .. 3) {
       x10.io.Console.OUT.println(
         "" + i + "	X:" + X(i) + "	t:" + t(i) 
         + "	u:" + u(i) + "	v:" + v(i)
         );
     }
  }

\end{xten}
%~~siv
%}
%~~neg

Writing the same code with method calls, while possible, is far less elegant: 
%~~gen

%package Classes.In.Poly101;
% // Integer-coefficient polynomials of one variable.
% class UglyPoly {
%   public val coeff : Array[Int](1);
%   public def this(coeff: Array[Int](1)) { this.coeff = coeff;}
%   public def degree() = coeff.size()-1;
%   public  def  a(i:Int) = (i<0 || i>this.degree()) ? 0 : coeff(i);
%
%   public static operator (c : Int) as UglyPoly = new UglyPoly([c]);
%
%   public def apply(x:Int) {
%     val d = this.degree();
%     var s : Int = this.a(d);
%     for( [i] in 1 .. this.degree() ) {
%        s = x * s + a(d-i);
%     }
%     return s;
%   }
%
%   public operator this + (p:UglyPoly) =  new UglyPoly(
%      new Array[Int](
%         Math.max(this.coeff.size(), p.coeff.size()),
%         (i:Int) => this.a(i) + p.a(i)
%      ));
%   public operator this - (p:UglyPoly) = this + (-1)*p;
%
%   public operator this * (p:UglyPoly) = new UglyPoly(
%      new Array[Int](
%        this.degree() + p.degree() + 1,
%        (k:Int) => sumDeg(k, this, p)
%        )
%      );
%
%
%   public operator (n : Int) + this = (n as UglyPoly) + this;
%   public operator this + (n : Int) = (n as UglyPoly) + this;
%
%   public operator (n : Int) - this = (n as UglyPoly) + (-1) * this;
%   public operator this - (n : Int) = ((-n) as UglyPoly) + this;
%
%   public operator (n : Int) * this = new UglyPoly(
%      new Array[Int](
%        this.degree()+1,
%        (k:Int) => n * this.a(k)
%      ));
%   private static def sumDeg(k:Int, a:UglyPoly, b:UglyPoly) {
%      var s : Int = 0;
%      for( [i] in 0 .. k ) s += a.a(i) * b.a(k-i);
%        // x10.io.Console.OUT.println("sumdeg(" + k + "," + a + "," + b + ")=" + s);
%      return s;
%      };
%   public final def toString() = {
%      var allZeroSoFar : Boolean = true;
%      var s : String ="";
%      for( [i] in 0..this.degree() ) {
%        val ai = this.a(i);
%        if (ai == 0) continue;
%        if (allZeroSoFar) {
%           allZeroSoFar = false;
%           s = term(ai, i);
%        }
%        else
%           s +=
%              (ai > 0 ? " + " : " - ")
%             +term(ai, i);
%      }
%      if (allZeroSoFar) s = "0";
%      return s;
%   }
%   private final def term(ai: Int, n:Int) = {
%      val xpow = (n==0) ? "" : (n==1) ? "x" : "x^" + n ;
%      return (ai == 1) ? xpow : "" + Math.abs(ai) + xpow;
%   }
%
%   def mult(p:UglyPoly) = this * p;
%   def mult(n:Int) = n * this;
%   def plus(p:UglyPoly) = this + p;
%   def plus(n:Int) = n + this;
%   def minus(p:UglyPoly) = this - p;
%   def minus(n:Int) = this - n;
%   static def const(n:Int) = n as UglyPoly;
%
%   public static def Main(x:Rail[String]) = main(x);
%   public static def main(Rail[String]):void {
%      val X = new UglyPoly([0,1]);
%      val t <: UglyPoly = 7 * X + 6 * X * X * X;
%
%      val u <: UglyPoly = 3 + 5*X - 7*X*X;
%      val v <: UglyPoly = t * u - 1;
%      for( [i] in -3 .. 3) {
%        x10.io.Console.OUT.println(
%          "" + i + "	X:" + X(i) + "	t:" + t(i) + "	u:" + u(i) + "	v:" + v(i)
%          );
%      }
%      uglymain();
%   }
%


%~~vis
\begin{xten}
  public static def uglymain() {
     val X = new UglyPoly([0,1]);
     val t <: UglyPoly = X.mult(7).plus(X.mult(X).mult(X).mult(6));  
     val u <: UglyPoly = const(3).plus(X.mult(5)).minus(X.mult(X).mult(7));
     val v <: UglyPoly = t.mult(u).minus(1);
     for( [i] in -3 .. 3) {
       x10.io.Console.OUT.println(
         "" + i + "	X:" + X.apply(i) + "	t:" + t.apply(i) 
          + "	u:" + u.apply(i) + "	v:" + v.apply(i)
         );
     }
  }
\end{xten}
%~~siv
%}
%~~neg

The operator-using code can be written in X10, though a few variations are
necessary to handle such exotic cases as \xcd`1+X`.

\subsection{Binary Operators}

Defining the sum \xcd`P+Q` of two polynomials looks much like a method
definition.  It uses the \xcd`operator` keyword instead of \xcd`def`, and
\xcd`this` appears in the definition in the place that a \xcd`Poly` would
appear in a use of the operator.  So, 
\xcd`operator this + (p:Poly)` explains how to add \xcd`this` to a
\xcd`Poly` value.
%~~gen
% package Classes.In.Poly102;
%~~vis
\begin{xten}
class Poly {
  public val coeff : Array[Int](1);
  public def this(coeff: Array[Int](1)) { this.coeff = coeff;}
  public def degree() = coeff.size()-1;
  public def  a(i:Int) = (i<0 || i>this.degree()) ? 0 : coeff(i);

  public operator this + (p:Poly) =  new Poly(
     new Array[Int](
        Math.max(this.coeff.size(), p.coeff.size()),
        (i:Int) => this.a(i) + p.a(i)
     )); 
  // ... 
\end{xten}
%~~siv
%   public operator (n : Int) + this = new Poly([n]) + this;
%   public operator this + (n : Int) = new Poly([n]) + this;
% 
%   def makeSureItWorks() {
%      val x = new Poly([0,1]);
%      val p <: Poly = x+x+x;
%      val q <: Poly = 1+x;
%      val r <: Poly = x+1;
%   }
%     
% }
%~~neg


The sum of a polynomial and an integer, \xcd`P+3`, looks like
an overloaded method definition.  
%~~gen
% package Classes.In.Poly103;
% class Poly {
%   public val coeff : Array[Int](1);
%   public def this(coeff: Array[Int](1)) { this.coeff = coeff;}
%   public def degree() = coeff.size()-1;
%   public def  a(i:Int) = (i<0 || i>this.degree()) ? 0 : coeff(i);
% 
%   public operator this + (p:Poly) =  new Poly(
%      new Array[Int](
%         Math.max(this.coeff.size(), p.coeff.size()),
%         (i:Int) => this.a(i) + p.a(i)
%      ));
%    public operator (n : Int) + this = new Poly([n]) + this;
%~~vis
\begin{xten}
   public operator this + (n : Int) = new Poly([n]) + this;
\end{xten}
%~~siv
% 
%   def makeSureItWorks() {
%      val x = new Poly([0,1]);
%      val p <: Poly = x+x+x;
%      val q <: Poly = 1+x;
%      val r <: Poly = x+1;
%   }
%     
% }
%~~neg


However, we want to allow the sum of an integer and a polynomial as well:
\xcd`3+P`.  It would be quite inconvenient to have to define this as a method
on \xcd`Int`; changing \xcd`Int` is far outside of normal coding.  So, we
allow it as a method on \xcd`Poly` as well.


%~~gen
% package Classes.In.Poly104o;
% class Poly {
%   public val coeff : Array[Int](1);
%   public def this(coeff: Array[Int](1)) { this.coeff = coeff;}
%   public def degree() = coeff.size()-1;
%   public def  a(i:Int) = (i<0 || i>this.degree()) ? 0 : coeff(i);
% 
%   public operator this + (p:Poly) =  new Poly(
%      new Array[Int](
%         Math.max(this.coeff.size(), p.coeff.size()),
%         (i:Int) => this.a(i) + p.a(i)
%      ));
%~~vis
\begin{xten}
   public operator (n : Int) + this = new Poly([n]) + this;
\end{xten}
%~~siv
% 
%   public operator this + (n : Int) = new Poly([n]) + this;
%   def makeSureItWorks() {
%      val x = new Poly([0,1]);
%      val p <: Poly = x+x+x;
%      val q <: Poly = 1+x;
%      val r <: Poly = x+1;
%   }
%     
% }
%~~neg

Furthermore, it is sometimes convenient to express a binary operation as a
static method on a class. 
The definition for the sum of two
\xcd`Poly`s could have been written:
%~~gen
% package Classes.In.Poly105;
% class Poly {
%   public val coeff : Array[Int](1);
%   public def this(coeff: Array[Int](1)) { this.coeff = coeff;}
%   public def degree() = coeff.size()-1;
%   public def  a(i:Int) = (i<0 || i>this.degree()) ? 0 : coeff(i);
%~~vis
\begin{xten}
  public static operator (p:Poly) + (q:Poly) =  new Poly(
     new Array[Int](
        Math.max(q.coeff.size(), p.coeff.size()),
        (i:Int) => q.a(i) + p.a(i)
     ));
\end{xten}
%~~siv
%
%   public operator (n : Int) + this = new Poly([n]) + this;
%   public operator this + (n : Int) = new Poly([n]) + this;
% 
%   def makeSureItWorks() {
%      val x = new Poly([0,1]);
%      val p <: Poly = x+x+x;
%      val q <: Poly = 1+x;
%      val r <: Poly = x+1;
%   }
%     
% }
%~~neg


This requires the following syntax:\\ 
\begin{grammar}
MethodHeader \:
  \xcd`operator` TypeParameterList\opt \xcd`this` BinOp \xcd`(`  FormalParameter \xcd")" \\
  && Guard\opt ReturnType\opt  \\
MethodHeader \:
  \xcd`operator` TypeParameterList\opt \xcd`(`  FormalParameter \xcd")" BinOp \xcd`this`  \\
  && Guard\opt ReturnType\opt   \\
MethodHeader \:
  \xcd`operator` TypeParameterList\opt \xcd`(`  FormalParameter \xcd")" BinOp  \xcd`(`  FormalParameter \xcd")"  \\
  && Guard\opt ReturnType\opt  \\
\end{grammar}

When X10 attempts to typecheck a binary operator expression like \xcd`P+Q`, it
first typechecks \xcd`P` and \xcd`Q`. Then, it looks for operator declarations
for \xcd`+` in the types of \xcd`P` and \xcd`Q`. If there are none, it is a
static error. If there is precisely one, that one will be used. If there are
several, X10 looks for a {\em best-matching} operation, \viz{} one which does
not require the operands to be converted to another type. For example,
\xcd`operator this + (n:Long)` and \xcd`operator this + (n:Int)` both apply to
\xcd`p+1`, because \xcd`1` can be converted from an \xcd`Int` to a \xcd`Long`.
However, the \xcd`Int` version will be chosen because it does not require a
conversion. If even the best-matching operation is not uniquely determined,
the compiler will report a static error.

The main difference between expressing a binary operation as an instance
method (with a \xcd`this` in the definition) and a static one (no \xcd`this`)
is that instance methods don't apply any conversions, while static methods
attempt to convert both arguments. 
\bard{give an example}

\bard{List the operators which this works for, in precedence order}

\subsection{Unary Operators}

Unary operators are defined in a similar way, with \xcd`this` appearing in the
\xcd`operator` definition where an actual value would occur in a unary
expression.  The operator to negate a polynomial is: 

%~~gen
% package Classes.In.Poly106;
% class Poly {
%   public val coeff : Array[Int](1);
%   public def this(coeff: Array[Int](1)) { this.coeff = coeff;}
%   public def degree() = coeff.size()-1;
%   public def  a(i:Int) = (i<0 || i>this.degree()) ? 0 : coeff(i);
%~~vis
\begin{xten}
  public operator - this = new Poly(
    new Array[Int](coeff.size(), (i:Int) => -coeff(i))
    );
\end{xten}
%~~siv
%   def makeSureItWorks() {
%      val x = new Poly([0,1]);
%      val p <: Poly = -x;
%   }
% }
%~~neg

The syntax for unary operators is:

\begin{grammar}
MethodHeader \:
  \xcd`operator` PrefixOp \xcd`this`    Guard\opt ReturnType\opt  
\end{grammar}

The rules for typechecking a unary operation are the same as for methods; the
complexities of binary operations are not needed.

\bard{List the operators which this works for, in precedence order}


\subsection{Type Conversions}

Explicit type conversions, \xcd`e as T{c}`, can be defined as operators on
class \xcd`T`.

%~~gen
% package Classes.And.Type.Conversions.For.Sea.Urchins;
%~~vis
\begin{xten}
class Poly {
  public val coeff : Array[Int](1);
  public def this(coeff: Array[Int](1)) { this.coeff = coeff;}
  public static operator (a:Int) as Poly = new Poly([a]);
  public static def main(Array[String](1)):void {
     val three : Poly = 3 as Poly;
  }
}
\end{xten}
%~~siv
%
%~~neg


% TODO

%%TODO%%  You may define a type conversion to a constrained type, like \xcd`Poly` in
%%TODO%%  the previous example.   If you convert to a more specific constraint, X10 will use
%%TODO%%  the conversion, but insert a dynamic check to make sure that you have
%%TODO%%  satisfied the more specific constraint.  
%%TODO%%  For example: 
%%TODO%%  %~x~gen
%%TODO%%  %package Classes.And.Type.Conversions;
%%TODO%%  %~x~vis
%%TODO%%  \begin{xten}
%%TODO%%  class Uni(n:Int) {
%%TODO%%  
%%TODO%%    public def this(n:Int) : Uni{self.n==n} = {property(n);}
%%TODO%%    static operator (String) as Uni{self.n != 9} = new Uni(3);
%%TODO%%    public static def main(Array[String](1)):void {
%%TODO%%      val u = "" as Uni{self.n != 9 && self.n != 3};
%%TODO%%    }
%%TODO%%  }
%%TODO%%  \end{xten}
%%TODO%%  %~x~siv
%%TODO%%  %
%%TODO%%  %~x~neg
%%TODO%%  The string \xcd`""` is converted to \xcd`Uni{self.n != 9}` via the defined
%%TODO%%  conversion operator, and that value is checked against the remaining
%%TODO%%  constraints \xcd`{self.n != 3}` at runtime.  (In this case it will fail.)
%%TODO%%  
%%TODO%%  There may be many conversions from different types to \xcd`T`, but there may
%%TODO%%  be at most one conversion from any given type to \xcd`T`. 
%%TODO%%  
\bard{Syntax}

\subsection{Implicit Type Coercions}

You may also define {\em implicit} type coercions to \xcd`T{c}` as static
operators in class \xcd`T`.  The syntax for this is
\xcd`static operator (x:U) : T{c} = e`.
Implicit coercions are used automatically by the compiler.  
\bard{How does this work?  One coercion, or a chain, and how about ambiguity?}

For example, we can define an implicit coercion from \xcd`Int` to \xcd`Poly`,
and avoid having to define the sum of an integer and a polynomial
as many special cases.  In the following example, we only define \xcd`+` on
two polynomials (using a \xcd`static` operator, so that implicit coercions
will be used -- they would not be for an instance method operator).  The
calculation \xcd`1+x` coerces \xcd`1` to a polynomial and uses polynomial
addition to add it to \xcd`x`.

%~~gen
% package Classes.And.Implicit.Coercions;
% class Poly {
%   public val coeff : Array[Int](1);
%   public def this(coeff: Array[Int](1)) { this.coeff = coeff;}
%   public def degree() = coeff.size()-1;
%   public def  a(i:Int) = (i<0 || i>this.degree()) ? 0 : coeff(i);
%   public final def toString() = {
%      var allZeroSoFar : Boolean = true;
%      var s : String ="";
%      for( [i] in 0..this.degree() ) {
%        val ai = this.a(i);
%        if (ai == 0) continue;
%        if (allZeroSoFar) {
%           allZeroSoFar = false;
%           s = term(ai, i);
%        }
%        else 
%           s += 
%              (ai > 0 ? " + " : " - ")
%             +term(ai, i);
%      }
%      if (allZeroSoFar) s = "0";
%      return s;
%   }
%   private final def term(ai: Int, n:Int) = {
%      val xpow = (n==0) ? "" : (n==1) ? "x" : "x^" + n ;
%      return (ai == 1) ? xpow : "" + Math.abs(ai) + xpow;
%   }

%~~vis
\begin{xten}
  public static operator (c : Int) : Poly = new Poly([c]);

  public static operator (p:Poly) + (q:Poly) = new Poly(
      new Array[Int](
        Math.max(p.coeff.size(), q.coeff.size()),
        (i:Int) => p.a(i) + q.a(i)
     ));

  public static def main(Array[String](1)):void {
     val x = new Poly([0,1]);
     x10.io.Console.OUT.println("1+x=" + (1+x));
  }
\end{xten}
%~~siv
%}
%~~neg

\bard{Syntax}

\subsection{\xcd`set` and \xcd`apply`}
\index{set}
\index{apply}
\index{()}
\index{()=}
\label{set-and-apply}
X10 allows types to implement the subscripting / function application
operator, and indexed assignment.  The \xcd`Array`-like classes take advantage
of both of these in \xcd`a(i) = a(i) + 1`.  Unlike unary and binary operators,
subscripting and indexed assignment are done by methods, \xcd`apply` and
\xcd`set` respectively.

\xcd`a(b,c,d)` is short for the method call \xcd`a.apply(b,c,d)`.  Since it is
possible to overload methods, the application syntax can be overloaded.  For
example, an ordered dictionary structure could allow subscripting by numbers
with \xcd`def apply(i:Int)`, and by string-valued keys with 
\xcd`def apply(s:String)`.  

\xcd`a(i)=b` is short for the method call \xcd`a.set(b,i)`, with one or more
indices \xcd`i`. (This has a
possibly surprising consequence for the order of evaluation: in \xcd`a(i)=b`,
as in \xcd`a.set(b,i)`, \xcd`a` is evaluated first, then \xcd`b`, and finally
\xcd`i`.)  Again, it is possible to overload \xcd`set` to provide a variety of
subscripting operations.  Each \xcd`set` method must have a corresponding
\xcd`apply` method; that is, \xcd`a(i,j)=b` is only defined when \xcd`a(i,j)`
is defined, despite the fact that \xcd`a(i,j)=b` does not evaluate \xcd`a(i,j)`.

The \xcd`Oddvec` class of somewhat peculiar vectors illustrates this.
\xcd`a()` returns a string representation of the oddvec, which probably should
be done by \xcd`toString()` instead.  \xcd`a(i)` picks out one of the three
coordinates of \xcd`a`, which is sensible.  \xcd`a(i)=b` assigns to one of the
coordinates.  \xcd`a(i,j)=b` assigns different values to \xcd`a(i)` and
\xcd`a(j)`, purely for the sake of the example.

%~~gen
% package Classes.Assignments.Are.Not.From.Any.Course.Of.Study;
%~~vis
\begin{xten}
class Oddvec {
  var v : Array[Int](1) = new Array[Int](3, (Int)=>0);
  public def apply() = "(" + v(0) + "," + v(1) + "," + v(2) + ")";
  public def apply(i:Int) = v(i);
  public def apply(i:Int, j:Int) = [v(i),v(j)];
  public def set(newval:Int, i:Int) = {v(i) = newval;}
  public def set(newval:Int, i:Int, j:Int) = {
       v(i) = newval; v(j) = newval+1;} 
  // ... 
\end{xten}
%~~siv
%  public static def main(argv:Rail[String]):void {
%     val a = new Oddvec();
%     x10.io.Console.OUT.println(a() + " ... " + a(0));
%     a(1) = 20;
%     x10.io.Console.OUT.println(a());
%     a(0) = 30;
%     x10.io.Console.OUT.println(a());
%     a(0,1) = 100;
%     x10.io.Console.OUT.println(a());
%   }
% }
%~~neg



\section{Class Guards and Invariants}\label{DepType:ClassGuard}
\index{type invariants}
\index{class invariants}
\index{guards}


Classes (and structs and interfaces) may specify a {\em class guard}, a
constraint which must hold on all values of the class.    In the following
example, a \xcd`Line` is defined by two distinct \xcd`Pt`s\footnote{We use \xcd`Pt`
to avoid any possible confusion with the built-in class \xcd`Point`.}
%~~gen
% package classes.guards.invariants.glurp;
%~~vis
\begin{xten}
class Pt(x:Int, y:Int){}
class Line(a:Pt, b:Pt){a != b} {}
\end{xten}
%~~siv
%
%~~neg

In most cases the class guard could be phrased as a type constraint on a property of
the class instead, if preferred.  Arguably, a symmetric constraint like two
points being different is better expressed as a class guard, rather than
asymmetrically as a constraint on one type: 
%~~gen
% package classes.guards.invariants.glurp2;
% class Pt(x:Int, y:Int){}
%~~vis
\begin{xten}
class Line(a:Pt, b:Pt{a != b}) {}
\end{xten}
%~~siv
%
%~~neg



\label{DepType:TypeInvariant}
\index{Class invariant}
\label{DepType:ClassGuardDef}



With every defined class, struct,  or interface \xcd"T" we associate a {\em type
invariant} $\mathit{inv}($\xcd"T"$)$, which describes the guarantees on the
properties of values of type \xcd`T`.  

Every value of \xcd`T` satisfies $\mathit{inv}($\xcd"T"$)$ at all times.  This
is somewhat stronger than the concept of type invariant in most languages
(which only requires that the invariant holds when no method calls are
active).  X10 invariants only concern properties, which are immutable; thus,
once established, they cannot be falsified.

The type
invariant associated with \xcd"x10.lang.Any"
is 
\xcd"true".

The type invariant associated with any interface or struct \xcd"I" that extends
interfaces \xcdmath"I$_1$, $\dots$, I$_k$" and defines properties
\xcdmath"x$_1$: P$_1$, $\dots$, x$_n$: P$_n$" and
specifies a guard \xcd"c" is given by:

\begin{xtenmath}
$\mathit{inv}$(I$_1$) && $\dots$ && $\mathit{inv}$(I$_k$) 
    && self.x$_1$ instanceof P$_1$ &&  $\dots$ &&  self.x$_n$ instanceof P$_n$ 
    && c  
\end{xtenmath}

Similarly the type invariant associated with any class \xcd"C" that
implements interfaces \xcdmath"I$_1$, $\dots$, I$_k$",
extends class \xcd"D" and defines properties
\xcdmath"x$_1$: P$_1$, $\dots$, x$_n$: P$_n$" and
specifies a guard \xcd"c" is
given by the same thing with the invariant of the superclass \xcd`D` conjoined:
\begin{xtenmath}
$\mathit{inv}$(I$_1$) && $\dots$ && $\mathit{inv}$(I$_k$) 
    && self.x$_1$ instanceof P$_1$ &&  $\dots$ &&  self.x$_n$ instanceof P$_n$ 
    && c  
    && $\mathit{inv}$(D)
\end{xtenmath}


Note that the type invariant associated with a class entails the type
invariants of each interface that it implements (directly or indirectly), and
the type invariant of each ancestor class.
It is guaranteed that for any variable \xcd"v" of
type \xcd"T{c}" (where \xcd"T" is an interface name or a class name) the only
objects \xcd"o" that may be stored in \xcd"v" are such that \xcd"o" satisfies
$\mathit{inv}(\mbox{\xcd"T"}[\mbox{\xcd"o"}/\mbox{\xcd"this"}])
\wedge \mbox{\xcd"c"}[\mbox{\xcd"o"}/\mbox{\xcd"self"}]$.



\subsection{Invariants for \Xcd{implements} and \Xcd{extends} clauses}\label{DepType:Implements}
\label{DepType:Extends}
\index{type-checking!implements clause}
\index{type-checking!extends clause}
\index{implements clause}
\index{extends  clause}
Consider a class definition
\begin{xtenmath}
$\mbox{\emph{ClassModifiers}}^{\mbox{?}}$
class C(x$_1$: P$_1$, $\dots$, x$_n$: P$_n$) extends D{d}
   implements I$_1${c$_1$}, $\dots$, I$_k${c$_k$}
$\mbox{\emph{ClassBody}}$
\end{xtenmath}

Each of the following static semantics rules must be satisfied:

\begin{staticrule}{Int-implements}
The type invariant \xcdmath"$\mathit{inv}$(C)" of \xcd"C" must entail
\xcdmath"c$_i$[this/self]" for each $i$ in $\{1, \dots, k\}$
\end{staticrule}

\begin{staticrule}{Super-extends}
The return type \xcd"c" of each constructor in \grammarrule{ClassBody}
must entail \xcd"d".
\end{staticrule}

\subsection{Invariants and constructor definitions}

A constructor for a class \xcd"C" is guaranteed to return an object of the
class on successful termination. This object must satisfy  \xcdmath"$\mathit{inv}$(C)", the
class invariant associated with \xcd"C" (\Sref{DepType:TypeInvariant}).
However,
often the objects returned by a constructor may satisfy {\em stronger}
properties than the class invariant. \Xten{}'s dependent type system
permits these extra properties to be asserted with the constructor in
the form of a constrained type (the ``return type'' of the constructor):

%%GRAMMAR%% \begin{grammar}
%%GRAMMAR%% ConstructorDeclarator \:
%%GRAMMAR%%   \xcd"def" \xcd"this" TypeParameterList\opt \xcd"(" FormalParameterList\opt \xcd")" \\
%%GRAMMAR%%   && ReturnType\opt Guard\opt Throws\opt Offers\opt \\
%%GRAMMAR%% ReturnType    \: \xcd":" Type \\
%%GRAMMAR%% Guard   \: "\{" DepExpression "\}" \\
%%GRAMMAR%% Throws    \: \xcd"throws" ExceptionType  ( \xcd"," ExceptionType )\star \\
%%GRAMMAR%% Offers    \: \xcd"offers" Type \\
%%GRAMMAR%% ExceptionType \: ClassBaseType Annotation\star \\
%%GRAMMAR%% \end{grammar}

\label{ConstructorGuard}

The parameter list for the constructor
may specify a \emph{guard} that is to be satisfied by the parameters
to the list.

\begin{example}
%%TODO--rewrite this
Here is another example, constructed as a simplified 
version of \Xcd{x10.lang.Region}.  The \xcd`mockUnion` method 
has the type that a true \xcd`union` method would have.

%~~gen
%package Classes.Are.For.Wussies.Wimps.And.People.With.Vowels.In.Their.Names;
%~~vis
\begin{xten}
class MyRegion(rank:Int) {
  static type MyRegion(n:Int)=MyRegion{self.rank==n};
  def this(r:Int):MyRegion(r) {
    property(r);
  }
  def this(diag:Array[Int](1)):MyRegion(diag.size){ 
    property(diag.size);
  }
  def mockUnion(r:MyRegion(rank)):MyRegion(rank) = this;
  def example() {
    val R1 : MyRegion(3) = new MyRegion([4,4,4]); 
    val R2 : MyRegion(3) = new MyRegion([5,4,1]); 
    val R3 = R1.mockUnion(R2); // inferred type MyRegion(3)
  }
}
\end{xten}
%~~siv
%
%~~neg
The first constructor returns the empty region of rank \Xcd{r}.  The
second constructor takes a \Xcd{Array[Int](1)} of arbitrary length
\Xcd{n} and returns a \Xcd{MyRegion(n)} (intended to represent the set
of points in the rectangular parallelopiped between the origin and the
\Xcd{diag}.)

The code in \xcd`example` typechecks, and \xcd`R3`'s type is inferred as
\xcd`MyRegion(3)`.  


\end{example}

\begin{staticrule}{Super-invoke}
   Let \xcd"C" be a class with properties
   \xcdmath"p$_1$: P$_1$, $\dots$, p$_n$: P$_n$", invariant \xcd"c"
   extending the constrained type \xcd"D{d}" (where \xcd"D" is the name of a class).

   For every constructor in \xcd"C" the compiler checks that the call to
   super invokes a constructor for \xcd"D" whose return type is strong enough
   to entail \xcd"d". Specifically, if the call to super is of the form 
     \xcdmath"super(e$_1$, $\dots$, e$_k$)"
   and the static type of each expression \xcdmath"e$_i$" is
   \xcdmath"S$_i$", and the invocation
   is statically resolved to a constructor
\xcdmath"def this(x$_1$: T$_1$, $\dots$, x$_k$: T$_k$){c}: D{d$_1$}"
   then it must be the case that 
\begin{xtenmath}
x$_1$: S$_1$, $\dots$, x$_i$: S$_i$ $\vdash$ x$_i$: T$_i$  (for $i \in \{1, \dots, k\}$)
x$_1$: S$_1$, $\dots$, x$_k$: S$_k$ $\vdash$ c  
d$_1$[a/self], x$_1$: S$_1$, ..., x$_k$: S$_k$ $\vdash$ d[a/self]      
\end{xtenmath}
\noindent where \xcd"a" is a constant that does not appear in 
\xcdmath"x$_1$: S$_1$ $\wedge$ ... $\wedge$ x$_k$: S$_k$".
\end{staticrule}

\begin{staticrule}{Constructor return}
   The compiler checks that every constructor for \xcd"C" ensures that
   the properties \xcdmath"p$_1$,..., p$_n$" are initialized with values which satisfy
   \xcdmath"t(C)", and its own return type \xcd"c'" as follows.  In each constructor, the
   compiler checks that the static types \xcdmath"T$_i$" of the expressions \xcdmath"e$_i$"
   assigned to \xcdmath"p$_i$" are such that the following is
   true:
\begin{xtenmath}
p$_1$: T$_1$, $\dots$, p$_n$: T$_n$ $\vdash$ t(C) $\wedge$ c'     
\end{xtenmath}
\end{staticrule}
(Note that for the assignment of \xcdmath"e$_i$" to \xcdmath"p$_i$"
to be type-correct it must be the
    case that \xcdmath"p$_i$: T$_i$ $\wedge$ p$_i$: P$_i$".) 


\begin{staticrule}{Constructor invocation}
The compiler must check that every invocation \xcdmath"C(e$_1$, $\dots$, e$_n$)" to a
constructor is type correct: each argument \xcdmath"e$_i$" must have a static type
that is a subtype of the declared type \xcdmath"T$_i$" for the $i$th
argument of the
constructor, and the conjunction of static types of the argument must
entail the \grammarrule{Guard} in the parameter list of the constructor.
\end{staticrule}


\subsection{Object Initialization}
\label{ObjectInitialization}
\index{initialization}
\index{constructor}
\index{object!constructor}
\index{struct!constructor}

% \noo{Confirm this chapter with the paper}

X10 does object initialization safely.  It avoids certain bad things which
trouble some other languages:
\begin{enumerate}
\item Use of a field before the field has been initialized.
\item A program reading two different values from a \xcd`val` field of a
      container. 
\item \Xcd{this} escaping from a constructor, which can cause problems as
      noted below. 

\end{enumerate}

It should be unsurprising that fields must not be used before they are
initialized. At best, it is uncertain what value will be in them, as in
\Xcd{x} below. Worse, the value might not even be an allowable value; \Xcd{y},
declared to be nonzero in the following example, might be zero before it is
initialized.
\begin{xten}
// Not correct X10
class ThisIsWrong {
  val x : Int;
  val y : Int{y != 0};
  def this() {
    x10.io.Console.OUT.println("x=" + x + "; y=" + y);
    x = 1; y = 2;
  }
}
\end{xten}

One particularly insidious way to read uninitialized fields is to allow
\Xcd{this} to escape from a constructor. For example, the constructor could
put \Xcd{this} into a data structure before initializing it, and another
activity could read it from the data structure and look at its fields:
\begin{xten}
class Wrong {
  val shouldBe8 : Int;
  static Cell[Wrong] wrongCell = new Cell[Wrong]();
  static def doItWrong() {
     finish {
       async { new Wrong(); } // (A)
       assert( wrongCell().shouldBe8 == 8); // (B)
     }
  }
  def this() {
     wrongCell.set(this); // (C) - ILLEGAL
     this.shouldBe8 = 8; // (D)
  }
}
\end{xten}
\noindent
In this example, the underconstructed \Xcd{Wrong} object is leaked into a
storage cell at line \Xcd{(C)}, and then initialized.  The \Xcd{doItWrong}
method constructs a new \Xcd{Wrong} object, and looks at the \Xcd{Wrong}
object in the storage cell to check on its \Xcd{shouldBe8} field.  One
possible order of events is the following:
\begin{enumerate}
\item \Xcd{doItWrong()} is called.
\item \Xcd{(A)} is started.  Space for a new \Xcd{Wrong} object is allocated.
      Its \Xcd{shouldBe8} field, not yet initialized, contains some garbage
      value.
\item \Xcd{(C)} is executed, as part of the process of constructing a new
      \Xcd{Wrong} object.  The new, uninitialized object is stored in
      \Xcd{wrongCell}.
\item Now, the initialization activity is paused, and execution of the main activity
      proceeds from \Xcd{(B)}.
\item The value in \Xcd{wrongCell} is retrieved, and is \Xcd{shouldBe8} field
      is read.  This field contains garbage, and the assertion fails.
\item Now let the initialization activity proceed with \Xcd{(D)},
      initializing \Xcd{shouldBe8} --- too late.
\end{enumerate}

The \xcd`at` statement (\Sref{AtStatement}) introduces the potential for
escape as well. The following class prints an uninitialized value: 
%~~gen ^^^ ThisEscapingViaAt_MustFailCompile
% package ObjInit_at;
% NOCOMPILE
%~~vis
\begin{xten}
// This code violates this chapter's constraints
// and thus will not compile in X10.
class Example {
  val a: Int;
  def this() { 
    at(here.next()) {
      // Recall that 'this' is a copy of 'this' outside 'at'.
      Console.OUT.println("this.a = " + this.a);
    }
    this.a = 1;
  }
}
\end{xten}
%~~siv
%
%~~neg


X10 must protect against such possibilities.  The rules explaining how
constructors can be written are somewhat intricate; they are designed to allow
as much programming as possible without leading to potential problems.
Ultimately, they simply are elaborations of the fundamental principles that
uninitialized fields must never be read, and \Xcd{this} must never be leaked.

%%RAW%% \subsection{Raw and Cooked Objects}
%%RAW%% \index{raw}
%%RAW%% \index{cooked}
%%RAW%% 
%%RAW%% An object is {\em raw} before its constructor ends, and {\em cooked} after its
%%RAW%% constructor ends. Note that, when an object is cooked, all its subobjects are
%%RAW%% cooked.  
%%RAW%% 



\subsection{Constructors and Non-Escaping Methods}
\index{non-escaping}
\label{sect:nonescaping}

In general, constructors must not be allowed to call methods with \Xcd{this} as
an argument or receiver. Such calls could leak references to \Xcd{this},
either directly from a call to \Xcd{cell.set(this)}, or indirectly because
\Xcd{toString} leaks \Xcd{this}, and the concatenation
\Xcd`"Escaper = "+this` calls \Xcd{toString}.\footnote{This is abominable behavior for
\Xcd{toString}, but it cannot be prevented -- save by a scheme such as we
present in this section.}
%~WRONG~gen
%package ObjectInit.CtorAndNonEscaping.One;
%~WRONG~vis
\begin{xten}
// This code violates this chapter's constraints
// and thus will not compile in X10.
class Escaper {
  static val Cell[Escaper] cell = new Cell[Escaper]();
  def toString() {
    cell.set(this);
    return "Evil!";
  }
  def this() {
    cell.set(this);
    x10.io.Console.OUT.println("Escaper = " + this);
  }
}
\end{xten}
%~WRONG~siv
%
%~WRONG~neg

However, it is convenient to be able to call methods from constructors; {\em
e.g.}, a class might have eleven constructors whose common behavior is best
described by three methods.
Under certain stringent conditions, it {\em is}
safe to call a method: the method called must not leak references to
\Xcd{this}, and must not read \Xcd{val}s or \Xcd{var}s which might not have
been assigned.

So, X10 performs a static dataflow analysis, sufficient to guarantee that
method calls in constructors are safe.  This analysis requires having access
to or guarantees about all the code that could possibly be called.  This can
be accomplished in two ways:
\begin{enumerate}
\item Ensuring that only code from the class itself can be called, by
      forbidding overriding of
      methods called from the constructor: they can be marked \Xcd{final} or
      \Xcd{private}, or the whole class can be \Xcd{final}.
\item Marking the methods called from the constructor by
      \xcd`@NonEscaping`.
\end{enumerate}

\subsubsection{Non-Escaping Methods}
\index{method!non-escaping}
\index{method!implicitly non-escaping}
\index{method!NonEscaping}
\index{implicitly non-escaping}
\index{non-escaping}
\index{non-escaping!implicitly}
\index{NonEscaping}


A method may be annotated with \xcd`@NonEscaping`.  This
imposes several restrictions on the method body, and on all methods overriding
it.  However, it is the only way that a method can be called from
constructors.  The
\Xcd{@NonEscaping} annotation makes explicit all the X10 compiler's needs for
constructor-safety.

A method can, however, be safe to call from constructors without being marked
\Xcd{@NonEscaping}. We call such methods {\em implicitly non-escaping}.
Implicitly non-escaping methods need to obey the same constraints on
\Xcd{this}, \Xcd{super}, and variable usage as \Xcd{@NonEscaping} methods. An
implicitly non-escaping method {\em could} be marked as
\xcd`@NonEscaping`; the compiler, in
effect, infers the annotation. In addition, all non-escaping methods
must be \Xcd{private} or \Xcd{final} or members of a \Xcd{final} class; this
corresponds to the hereditary nature of \Xcd{@NonEscaping} (by forbidding
inheritance of implicitly non-escaping methods).

We say that a method is {\em non-escaping} if it is either implicitly
non-escaping, or annotated \Xcd{@NonEscaping}.

The first requirement on non-escaping methods is that they do not allow
\Xcd{this} to escape. Inside of their bodies, \Xcd{this} and \Xcd{super} may
only be used for field access and assignment, and as the receiver of
non-escaping methods.


The following example uses the possible variations.  \Xcd{aplomb()} 
explicitly forbids reading any field but
\Xcd{a}. \Xcd{boric()} is called after \Xcd{a} and \Xcd{b} are set, but
\Xcd{c} is not.
The \xcd`@NonEscaping` annotation on \xcd`boric()` is optional, but the
compiler will print a warning if it is left out.
\Xcd{cajoled()} is only called after all fields are set, so it
can read anything; its annotation, too, is not required.   \Xcd{SeeAlso} is able to override \Xcd{aplomb()}, because
\Xcd{aplomb()} is \xcd`@NonEscaping`; it cannot override the final method
\Xcd{boric()} or the private one \Xcd{cajoled()}.  
%~~gen ^^^ ObjectInitialization10
%package ObjInit.C2;
%~~vis
\begin{xten}
import x10.compiler.*;

final class C2 {
  protected val a:Int, b:Int, c:Int;
  protected var x:Int, y:Int, z:Int;
  def this() {
    a = 1;
    this.aplomb();
    b = 2;
    this.boric();
    c = 3;
    this.cajoled();
  }
  @NonEscaping def aplomb() {
    x = a;
    // this.boric(); // not allowed; boric reads b.
    // z = b; // not allowed -- only 'a' can be read here
  }
  @NonEscaping final def boric() {
    y = b;
    this.aplomb(); // allowed; 
       // a is definitely set before boric is called
    // z = c; // not allowed; c is not definitely written
  }
  @NonEscaping private def cajoled() {
    z = c;
  }
}

\end{xten}
%~~siv
%
%~~neg



\subsection{Fine Structure of Constructors}
\label{SFineStructCtors}

The code of a constructor consists of four segments, three of them optional
and one of them implicit.
\begin{enumerate}
\item The first segment is an optional call to \Xcd{this(...)} or
      \Xcd{super(...)}.  If this is supplied, it must be the first statement
      of the constructor.  If it is not supplied, the compiler treats it as a
      nullary super-call \Xcd{super()};
\item If the class or struct has properties, there must be a single
      \Xcd{property(...)} command in the constructor, or a \xcd`this(...)`
      constructor call.  Every execution path
      through the constructor must go through this \Xcd{property(...)} command
      precisely once.   The second segment of the constructor is the code
      following the first segment, up to and including the \Xcd{property()}
      statement.

      If the class or struct has no properties, the \Xcd{property()} call must
      be omitted. If it is present, the second segment is defined as before.  If
      it is absent, the second segment is empty.
\item The third segment is automatically generated.  Fields with initializers
      are initialized immediately after the \Xcd{property} statement.
      In the following example, \Xcd{b} is initialized to \Xcd{y*9000} in
      segment three.  The initialization makes sense and does the right
      thing; \Xcd{b} will be \Xcd{y*9000} for every \Xcd{Overdone} object.
      (This would not be possible if field initializers were processed
      earlier, before properties were set.)
\item The fourth segment is the remainder of the constructor body.
\end{enumerate}

The segments in the following code are shown in the comments.
%~~gen ^^^ ObjectInitialization20
% package ObjectInitialization.ShowingSegments;
%~~vis
\begin{xten}
class Overlord(x:Int) {
  def this(x:Int) { property(x); }
}//Overlord
class Overdone(y:Int) extends Overlord  {
  val a : Int;
  val b =  y * 9000;
  def this(r:Int) {
    super(r);                      // (1)
    x10.io.Console.OUT.println(r); // (2)
    val rp1 = r+1;
    property(rp1);                 // (2)
    // field initializations here  // (3)
    a = r + 2 + b;                 // (4)
  }
  def this() {
    this(10);                      // (1), (2), (3)
    val x = a + b;                 // (4)
  }
}//Overdone
\end{xten}
%~~siv
%
%~~neg

The rules of what is allowed in the three segments are different, though
unsurprising.  For example, properties of the current class can only be read
in segment 3 or 4---naturally, because they are set at the end of segment 2.

\subsubsection{Initialization and Inner Classses}
\index{constructor!inner classes in}

Constructors of inner classes are tantamount to method calls on \Xcd{this}.
For example, the constructor for Inner {\bf is} acceptable.  It does not leak
\Xcd{this}.  It leaks \Xcd{Outer.this}, which is an utterly different object.
So, the call to \Xcd{this.new Inner()} in the \Xcd{Outer} constructor {\em
is} illegal.  It would leak \Xcd{this}.  There is no special rule in effect
preventing this; a constructor call of an inner class is no
different from a method as far as leaking is concerned.
%~~gen ^^^ ObjectInitialization30
% package ObjInit.InnerClass; 
% // NOTEST-packaging-issue
%~~vis
\begin{xten}
class Outer {
  static val leak : Cell[Outer] = new Cell[Outer](null);
  class Inner {
     def this() {Outer.leak.set(Outer.this);}
  }
  def /*Outer*/this() {
     //ERROR: val inner = this.new Inner();
  }
}
\end{xten}
%~~siv
%
%~~neg



\subsubsection{Initialization and Closures}
\index{constructor!closure in}

Closures in constructors may not refer to \xcd`this`.  They may not even refer
to fields of \xcd`this` that have been initialized.   For example, the
closure \xcd`bad1` is not allowed because it refers to \xcd`this`; \xcd`bad2`
is not allowed because it mentions \xcd`a` --- which is, of course, identical
to \xcd`this.a`. 

%%-deleted-%% valid if they were invoked (or inlined) at the
%%-deleted-%%place of creation. For example, \Xcd{closure} below is acceptable because it
%%-deleted-%%only refers to fields defined at the point it was written.  \Xcd{badClosure}
%%-deleted-%%would not be acceptable, because it refers to fields that were not defined at
%%-deleted-%%that point, although they were defined later.
%~~gen ^^^ ObjectInitialization40
% package ObjectInitialization.Closures; 
%~~vis
\begin{xten}
class C {
  val a:Int;
  def this() {
    this.a = 1;
    //ERROR: val bad1 = () => this; 
    //ERROR: val bad2 = () => a*10;
  }
}
\end{xten}
%~~siv
%
%~~neg


\subsection{Definite Initialization in Constructors}


An instance field \Xcd{var x:T}, when \Xcd{T} has a default value, need not be
explicitly initialized.  In this case, \Xcd{x} will be initialized to the
default value of type \Xcd{T}.  For example, a \Xcd{Score} object will have
its \Xcd{currently} field initialized to zero, below:
%~~gen ^^^ ObjectInitialization50
% package ObjectInit.DefaultInit;
%~~vis
\begin{xten}
class Score {
  public var currently : Int;
}
\end{xten}
%~~siv
%
%~~neg

All other sorts of instance fields do need to be initialized before they can
be used.  \Xcd{val} fields must be initialized, even if their type has a
default value.  It would be silly to have a field \Xcd{val z : Int} that was
always given default value of \Xcd{0} and, since it is \Xcd{val}, can never be
changed.  \Xcd{var} fields whose type has no default value must be initialized
as well, such as \xcd`var y : Int{y != 0}`, since it cannot be assigned a
sensible initial value.

The fundamental principles are:
\begin{enumerate}
\item \Xcd{val} fields must be assigned precisely once in each constructor on every
possible execution path.
\item \Xcd{var} fields of defaultless type must be
assigned at least once on every possible execution path, but may be assigned
more than once.
\item No variable may be read before it is guaranteed to have been
assigned.
\item Initialization may be by field initialization expressions (\Xcd{val x :
      Int = y+z}), or by uninitialized fields \Xcd{val x : Int;} plus
an initializing assignment \Xcd{x = y+z}.  Recall that field initialization
expressions are performed after the \Xcd{property} statement, in segment 3 in
the terminology of \Sref{SFineStructCtors}.
\end{enumerate}



\subsection{Summary of Restrictions on Classes and Constructors}

The following table tells whether a given feature is (yes), is not (no) or is
with some conditions (note) allowed in a given context.   For example, a
property method is allowed with the type of another property, as long as it
only mentions the preceding properties. The first column of the table gives
examples, by line of the following code body.

\begin{tabular}{||l|l|c|c|c|c|c|c||}
\hline
~
  & {\bf Example}
  & {\bf Prop.}
  & {\bf {\tt \small self==this}(1)}
  & {\bf Prop.Meth.}
  & {\bf {\tt this}}
  & {\bf {fields}}
\\\hline
Type of property
  & (A)
  & %?properties
    yes (2)
  & no %? self==this
  & no %? property methods
  & no %? this may be used
  & no %? fields may be used
\\\hline
Class Invariant
  & (B)
  & yes %?properties
  & yes %? self==this
  & yes %? property methods
  & yes %? this may be used
  & no %? fields may be used
\\\hline
Supertype (3)
  & (C), (D)
  & yes%?properties
  & yes%? self==this
  & yes%? property methods
  & no%? this may be used
  & no%? fields may be used
\\\hline
Property Method Body
  & (E)
  & yes %?properties
  & yes %? self==this
  & yes %? property methods
  & yes %? this may be used
  & no %? fields may be used
\\\hline

Static field (4)
  & (F) (G)
  & no %?properties
  & no %? self==this
  & no %? property methods
  & no %? this may be used
  & no %? fields may be used
\\\hline

Instance field (5)
  & (H), (I)
  & yes %?properties
  & yes %? self==this
  & yes %? property methods
  & yes %? this may be used
  & yes %? fields may be used
\\\hline

Constructor arg. type
  & (J)
  & no %?properties
  & no %? self==this
  & no  %? property methods
  & no %? this may be used
  & no %? fields may be used
\\\hline

Constructor guard
  & (K)
  & no %?properties
  & no %? self==this
  & no %? property methods
  & no %? this may be used
  & no %? fields may be used
\\\hline

Constructor ret. type
  & (L)
  & yes %?properties
  & yes %? self==this
  & yes %? property methods
  & yes %? this may be used
  & yes %? fields may be used
\\\hline

Constructor segment 1
  & (M)
  & no%?properties
  & yes%? self==this
  & no%? property methods
  & no%? this may be used
  & no%? fields may be used
\\\hline


Constructor segment 2
  & (N)
  & no%?properties
  & yes%? self==this
  & no%? property methods
  & no%? this may be used
  & no%? fields may be used
\\\hline

Constructor segment 4
  & (O)
  & yes%?properties
  & yes%? self==this
  & yes%? property methods
  & yes%? this may be used
  & yes%? fields may be used
\\\hline

Methods
  & (P)
  & yes %?properties
  & yes %? self==this
  & yes %? property methods
  & yes %? this may be used
  & yes %? fields may be used
\\\hline



\iffalse
place
  & (pos)
  & %?properties
  & %? self==this
  & %? property methods
  & %? this may be used
  & %? fields may be used
\\\hline
\fi
\end{tabular}

Details:

\begin{itemize}
\item (1) {Top-level {\tt self} only.}
\item (2) {The type of the {$i^{th}$} property may only mention
                 properties {$1$} through {$i$}.}
\item (3) Super-interfaces follow the same rules as supertypes.
\item (4) The same rules apply to types and initializers.
\end{itemize}



The example indices refer to the following code:
%~~gen ^^^ ObjectInitialization60
% package ObjectInit.GrandExample;
% class Supertype[T]{}
% interface SuperInterface[T]{}
%~~vis
\begin{xten}
class Example (
   prop : Int,
   proq : Int{prop != proq},                    // (A)
   pror : Int
   )
   {prop != 0}                                  // (B)
   extends Supertype[Int{self != prop}]         // (C)
   implements SuperInterface[Int{self != prop}] // (D)
{
   property def propmeth() = (prop == pror);    // (E)
   static staticField
      : Cell[Int{self != 0}]                    // (F)
      = new Cell[Int{self != 0}](1);            // (G)
   var instanceField
      : Int {self != prop}                      // (H)
      = (prop + 1) as Int{self != prop};        // (I)
   def this(
      a : Int{a != 0},
      b : Int{b != a}                           // (J)
      )
      {a != b}                                  // (K)
      : Example{self.prop == a && self.proq==b} // (L)
   {
      super();                                  // (M)
      property(a,b,a);                          // (N)
      // fields initialized here
      instanceField = b as Int{self != prop};   // (O)
   }

   def someMethod() =
        prop + staticField() + instanceField;     // (P)
}
\end{xten}
%~~siv
%
%~~neg


\section{Method Resolution}
\index{method!resolution}
\index{method!which one will get called}
\label{sect:MethodResolution}

Method resolution is the problem of determining, statically, which method (or
constructor or operator)
should be invoked, when there are several choices that could be invoked.  For
example, the following class has two overloaded \xcd`zap` methods, one taking
an \Xcd{Object}, and the other a \Xcd{Resolve}.  Method resolution will figure
out that the call \Xcd{zap(1..4)} should call \xcd`zap(Object)`, and
\Xcd{zap(new Resolve())} should call \xcd`zap(Resolve)`.  

\begin{ex}
%~~gen ^^^ MethodResolution10
%package MethodResolution.yousayyouwantaresolution;
% // This depends on https://jira.codehaus.org/browse/XTENLANG-2696
%~~vis
\begin{xten}
class Res {
  public static interface Surface {}
  public static interface Deface {}

  public static class Ace implements Surface {
    public static operator (Boolean) : Ace = new Ace();
    public static operator (Place) : Ace = new Ace();
  }
  public static class Face implements Surface, Deface{}

  public static class A {}
  public static class B extends A {}
  public static class C extends B {}

  def m(x:A) = 0;
  def m(x:Int) = 1;
  def m(x:Boolean) = 2;
  def m(x:Surface) = 3;
  def m(x:Deface) = 4; 

  def example() {
     assert m(100) == 1 : "Int"; 
     assert m(new C()) == 0 : "C";
     // An Ace is a Surface, unambiguous best choice
     assert m(new Ace()) == 3 : "Ace";
     // ERROR: m(new Face());

     // The match must be exact.
     // ERROR: assert m(here) == 3 : "Place";

     // Boolean could be handled directly, or by 
     // implicit coercion Boolean -> Ace.
     // Direct matches always win.
     assert m(true) == 2 : "Boolean"; 
  }
\end{xten}
%~~siv
%  public static def main(argv:Array[String](1)) {(new Res()).example(); Console.OUT.println("That's all!");}
% public def claim() { val ace : Ace = here; assert m(ace)==3; }
% }
% class Hook{ def run(){ (new Res()).example(); return true;} }
%~~neg

In the \xcd`"Int"` line, there is a very close match.  \xcd`100` is an
\xcd`Int`.  In fact, \xcd`100` is an \xcd`Int{self==100}`, so even in this
case the type of the actual parameter is not {\em precisely} equal to the type
of the method.

In the \xcd`"C"` line of the example, \xcd`new C()` is an instance of \xcd`C`,
which is a subtype of \xcd`A`, so the \xcd`A` method applies.  No other method
does, and so the \xcd`A` method will be invoked.

Similarly, in the \xcd`"Ace"` line, the \xcd`Ace` class implements
\xcd`Surface`, and so \xcd`new Ace()` matches the \xcd`Surface` method. 

However, a \xcd`Face` is both a \xcd`Surface` and a \xcd`Deface`, so there is
no unique best match for the invocation \xcd`m(new Face())`.  This invocation
would be forbidden, and a compile-time error issued.


The match must be exact.  There is an implicit coercion 
from \xcd`Place` to \xcd`Ace`, and \xcd`Ace` implements \xcd`Surface`, so the
code
\begin{xten}
val ace : Ace = here;
assert m(ace) == 3;
\end{xten}
works, by using the \xcd`Surface` form of \xcd`m`.  But doing it in one step
requires a deeper search than X10 performs\footnote{In general this search is
unbounded, so X10 can't perform it.}, and is not allowed.


For \xcd`m(true)`, both the \xcd`Boolean` and, with the implicit coercion,
\xcd`Ace` methods could apply.  Since the \xcd`Boolean` method applies
directly, and the \xcd`Ace` method requires an implicit coercion, this call
resolves to the \xcd`Boolean` method, without an error.

\end{ex}


The basic concept of method resolution is:
\begin{enumerate}
\item List all the methods that could possibly be used, inferring generic
      types but not performing implicit coercions.    If the
      \xcd`STATIC_CHECKS` compiler flag is specified, the constraints must
      match exactly at this step; if not, they do not, but run-time tests will
      be generated if necessary.
\item If one possible method is more specific than all the others, that one 
      is the desired method.
\item If there are two or more methods neither of which is more specific than
      the others, then the method invocation is ambiguous.  Method resolution
      fails and reports an error.
\item Otherwise, no possible methods were found without implicit coercions.
      Try the preceding steps again, but with coercions allowed: zero or one
      implicit coercion for each argument.  If a single
      most specific method is found with coercions, it is the desired method.
      If there are several, the invocation is ambiguous and erronious.
\item If no methods were found even with coercions, then the method invocation
      is undetermined.  Method resolution fails and reports an error.
\end{enumerate}
\noindent
In the presence of X10's highly-detailed type system, some subtleties arise. 
One point, at least, is {\em not} subtle. The same procedure is used, {\em
mutatis mutandis} for method, constructor, and operator resolution.  



\subsection{Space of Methods}

X10 allows some constructs, particularly \xcd`operator`s, to be defined in a
number of ways, and invoked in a number of ways. This section specifies which
forms of definition could correspond to a given definiendum.
%%OP%% , and (redundantly)
%%OP%% the syntax for invoking that definition unambiguously.  

Method invocations \xcd`a.m(b)`, where \xcd`a` is an expression, can be either
of the following forms.  There may be any number of arguments.
\begin{itemize}
\item An instance method on \xcd`a`, of the form \xcd`def m(B)`.
%%OP%% , so that the   invocation is \xcd`a.m(b)`;
\item A static method on \xcd`a`'s class, of the form \xcd`static def m(B)`.
%%OP%%       so that the invocation is \xcd`A.m(b)`.
\end{itemize}

Static method invocations, \xcd`A.m(b)`, where \xcd`A` is a container name,
can only be static.  There may be any number of arguments.
\begin{itemize}
\item A static method on \xcd`A`, of the form \xcd`static def m(B)`.
%%OP%%       the invocation is \xcd`A.m(b)`; 
\end{itemize}


Constructor invocations, \xcd`new A(b)`, must invoke constructors. There may
be any number of arguments. 
\begin{itemize}
\item A constructor on \xcd`A`, of the form \xcd`def this(B)`.
%%OP%% , so that the
%%OP%%       invocation is \xcd`new A(b)`.
\end{itemize}


A unary operator \xcdmath"$\star$ a" may be defined as: 
\begin{itemize}
\item An instance operator on \xcd`A`, defined as 
      \xcdmath"operator $\star$ this()".
%%OP%%       so that the invocation is 
%%OP%%       \xcdmath"a.operator $\star$()"; or
\item A static operator on \xcd`A`, defined as 
      \xcdmath"operator $\star$(a:A)".
%%OP%%       so that the invocation is 
%%OP%%       \xcdmath"A.operator $\star$(a)"
\end{itemize}

A binary operator \xcdmath"a $\star$ b" may be defined as: 
\begin{itemize}
\item An instance operator on \xcd`A`, defined as 
      \xcdmath"operator this $\star$(b:B)";
%%OP%%       so that the invocation is \xcdmath"a.operator $\star$(b)", 
or
\item A right-hand operator on \xcd`B`, defined as
      \xcdmath"operator (a:A) $\star$ this"; or
%%OP%%       so that the invocation is \xcdmath"b.operator ()$\star$(b)"

\item A static operator on \xcd`A`, defined as
      \xcdmath"operator (a:A) $\star$ (b:B)", 
%%OP%%       so that the invocation is \xcdmath"A.operator $\star$(a,b)"
; or
\item A static operator on \xcd`B`, if \xcd`A` and \xcd`B` are different
      classes, defined as
      \xcdmath"operator (a:A) $\star$ (b:B)"
%%OP%% , so that the invocation is 
%%OP%%       \xcdmath"B.operator $\star$(a,b)".
\end{itemize}
\noindent
If none of those resolve to a method, then either operand may be cast to the
other. 
The following alternate calls are attempted: 
\begin{itemize}
\item \xcdmath"(a as B)$\star$(b)"
\item \xcdmath"a $\star$ (b as A)"
\end{itemize}

An application \xcd`a(b)`, for any number of arguments, can come from a number
of things. 
\begin{itemize}
\item an application operator on \xcd`a`, defined as \xcd`operator this(b:B)`;
%%OP%% , so that the 
%%OP%% invocation is \xcd`a.operator()(b)`
\item If \xcd`a` is an identifier, \xcd`a(b)` can also be a method invocation
      equivalent to \xcd`this.a(b)`, which  invokes \xcd`a` as
      either an instance or static method on \xcd`this`
\item If \xcd`a` is a qualified identifier, \xcd`a(b)` can also be an
      invocation of a struct constructor.
\end{itemize}


An indexed assignment, \xcd`a(b)=c`, for any number of \xcd`b`'s, can only
come from an indexed assignment definition: 
\begin{itemize}
\item \xcd`operator this(b:B)=(c:C) {...}`
%%OP%%       so that the invocation is \xcd`a.operator()=(b,c)`.
\end{itemize}

An implicit coercion, in 
which a value \xcd`a:A` is used in a context which requires a value of some
other non-subtype \xcd`B`, 
can come from one of two places: an implicit coercion operation defined on
\xcd`B`, or, failing that, one defined on \xcd`A`:
\begin{enumerate}
\item an implicit coercion in \xcd`B`:
      \xcd`static operator (a:A):B`;
%%OP%%       so that the coercion is \xcd`B.operator[B](a)`;
\item or, failing that, an implicit coercion in \xcd`A`:
      \xcd`static operator this():B`.
%%OP%%       so that the coercion is \xcd`A.operator[B](a)`
\end{enumerate}

An explicit conversion \xcd`a as B` can come from an explicit conversion
operator, or an implicit coercion operator.  X10 tries three things, in order,
only checking 2 if 1 fails, and 3 if 2 fails: 
\begin{enumerate}
\item An \xcd`as` operator in \xcd`B`: 
      \xcdmath"static operator (a:A) as ?";
%%OP%%       so that the conversion is \xcd`B.operator as[B](a)`

\item or, failing that, an implicit coercion in \xcd`B`:
      \xcd`static operator (a:A):B`;
%%OP%% , so that the conversion is 
%%OP%%       \xcd`B.operator[B](a)`;
\item or, failing that, an implicit coercion in \xcd`A`:
      \xcd`static operator this():B`.
%%OP%% , so that the conversion is 
%%OP%%       \xcd`A.operator[B](a)`.
\end{enumerate}



\subsection{Possible Methods}

This section describes what it means for a method to be a {\em possible}
resolution of a method invocation.  



Generics introduce several subtleties, especially with the inference of
generic types. 
For the purposes of method resolution, all that matters about a method,
constructor, or operator \xcd`M` --- we use the word ``method'' to include all
three choices for this section --- is its signature, plus which method it is.
So, a typical \xcd`M` might look like 
\xcdmath"def m[G$_1$,$\ldots$, G$_g$](x$_1$:T$_1$,$\ldots$, x$_f$:T$_f$){c} =...".  The code body \xcd`...` is irrelevant for the purpose of whether a
given method call means \xcd`M` or not, so we ignore it for this section.

All that matters about a method definition, for the purposes of method
resolution, is: 
\begin{enumerate}
\item The method name \xcd`m`;
\item The generic type parameters of the method \xcd`m`,  \xcdmath"G$_1$,$\ldots$, G$_g$".  If there
      are no generic type parameters, {$g=0$}.  
\item The types \xcdmath"x$_1$:T$_1$,$\ldots$, x$_f$:T$_f$" of the formal parameters.  If
      there are no formal parameters, {$f=0$}. In the case of an instance
      method, the receiver will be the first formal parameter.\footnote{The
      variable names are relevant because one formal can be mentioned in a
      later type, or even a constraint: {\tt def f(a:Int, b:Point\{rank==a\})=...}.}
\item The constraint \xcd`c` of the method \xcd`M`. If no constraint is specified, \xcd`c` is
      \xcd`true`. 
\item A {\em unique identifier} \xcd`id`, sufficient to tell the compiler
      which method body is intended.  A file name and position in that file
      would suffice.  The details of the identifier are not relevant.
\end{enumerate}

For the purposes of understanding method resolution, we assume that all the
actual parameters of an invocation are simply variables: \xcd`x1.meth(x2,x3)`.
This is done routinely by the compiler in any case; the code 
\xcd`tbl(i).meth(true, a+1)` would be treated roughly as 
\begin{xten}
val x1 = tbl(i);
val x2 = true;
val x3 = a+1;
x1.meth(x2,x3);
\end{xten}

All that matters about an invocation \xcd`I` is: 
\begin{enumerate}
\item The method name \xcdmath"m$'$";
\item The generic type parameters \xcdmath"G$'_1$,$\ldots$, G$'_g$".  If there
      are no generic type parameters, {$g=0$}.  
\item The names and types \xcdmath"x$_1$:T$'_1$,$\ldots$, x$_f$:T$'_f$" of the
      actual parameters.
      If
      there are no actual parameters, {$f=0$}. In the case of an instance
      method, the receiver is the first actual parameter.
\end{enumerate}

The signature of the method resolution procedure is: 
\xcd`resolve(invo : Invocation, context: Set[Method]) : MethodID`.  
Given a particular invocation and the set \xcd`context` of all methods
which could be called at that point of code, method resolution either returns
the unique identifier of the method that should be called, or (conceptually)
throws an exception if the call cannot be resolved.

The procedure for computing \xcd`resolve(invo, context)` is: 
\begin{enumerate}
\item Eliminate from \xcd`context` those methods which are not {\em
      acceptable}; \viz, those whose name, type parameters, formal parameters,
      and constraint do not suitably match \xcd`invo`.  In more detail:
      \begin{itemize}
      \item The method name \xcd`m` must simply equal the invocation name \xcdmath"m$'$";
      \item X10 infers type parameters, by an algorithm given in \Sref{TypeParamInfer}.
      \item The method's type parameters are bound to the invocation's for the
            remainder of the acceptability test.
      \item The actual parameter types must be subtypes of the formal
            parameter types, or be coercible to such subtypes.  Parameter $i$
            is a subtype if \xcdmath"T$'_i$ <: T$_i$".  It is implicitly
            coercible to a subtype if either it is a subtype, or if there is
            an implicit coercion operator 
            defined from \xcdmath"T$'_i$" to some type \xcd`U`, and 
            \xcdmath"U <: T$_i$". \index{method resolution!implicit coercions
            and} \index{implicit coercion}\index{coercion}.  If coercions are
            used to resolve the method, they will be called on the arguments
            before the method is invoked.
            
      \item The formal constraint \xcd`c` must be satisfied in the invoking
            context. 
      \end{itemize}
\item Eliminate from \xcd`context` those methods which are not {\em
      available}; \viz, those which cannot be called due to visibility
      constraints, such as methods from other classes marked \xcd`private`.
      The remaining methods are both acceptable and available; they might be
      the one that is intended.
\item If the method invocation is a \xcd`super` invocation appearing in class
      \xcd`Cl`, methods of \xcd`Cl` and its subclasses are considered
      unavailable as well.
      
\item From the remaining methods, find the unique \xcd`ms` which is more specific than all the
      others, \viz, for which \xcd`specific(ms,mo) = true` for all other
      methods \xcd`mo`.
      The specificity test \xcd`specific` is given next.
      \begin{itemize}
      \item If there is a unique such \xcd`ms`, then
            \xcd`resolve(invo,context)` returns the \xcd`id` of \xcd`ms`.  
      \item If there is not a unique such \xcd`ms`, then \xcd`resolve` reports
            an error.
      \end{itemize}

\end{enumerate}

The subsidiary procedure \xcd`specific(m1, m2)` determines whether method
\xcd`m1` is equally or more specific than \xcd`m2`.  \xcd`specific` is not a
total order: is is possible for each one to be considered more specific than
the other, or either to be more specific.  \xcd`specific` is computed as: 
\begin{enumerate}
\item Construct an invocation \xcd`invo1` based on \xcd`m1`: 
      \begin{itemize}
      \item \xcd`invo1`'s method name is \xcd`m1`'s method name;
      \item \xcd`invo1`'s generic parameters are those of \xcd`m1`--- simply
            some type variables.
      \item \xcd`invo1`'s parameters are those of \xcd`m1`.
      \end{itemize}
\item If \xcd`m2` is acceptable for the invocation \xcd`invo1`,
      \xcd`specific(m1,m2)` returns true; 
\item Construct an invocation \xcd`invo2p`, which is \xcd`invo1` with the
      generic parameters erased.  Let \xcd`invo2` be \xcd`invo2p` with generic
      parameters as inferred by X10's type inference algorithm.  If type
      inference fails, \xcd`specific(m1,m2)` returns false.
\item If \xcd`m2` is acceptable for the invocation \xcd`invo2`,
      \xcd`specific(m1,m2)` returns true; 
\item Otherwise, \xcd`specific(m1,m2)` returns false.
\end{enumerate}

\subsection{Other Disambiguations}
\label{sect:disambiguations}

It is possible to have a field of the same name as a method.
Indeed, it is a common pattern to have private field and a public
method of the same name to access it:
\begin{ex}
%~~gen ^^^ MethodResolution_disamb_a
%package MethodResolution_disamb_a;
%~~vis
\begin{xten}
class Xhaver {
  private var x: Int = 0;
  public def x() = x;
  public def bumpX() { x ++; }
}
\end{xten}
%~~siv
%
%~~neg
\end{ex}

\begin{ex}
However, this can lead to syntactic ambiguity in the case where the field
\Xcd{f} of object \xcd`a` is a
function, array, list, or the like, and where \xcd`a` has a method also named
\xcd`f`.  The term \Xcd{a.f(b)} could either mean ``call method \xcd`f` of \xcd`a` upon
\xcd`b`'', or ``apply the function \xcd`a.f` to argument \xcd`b`''.  

%~~gen  ^^^ MethodResolution_disamb_b
%package MethodResolution_disamb_b;
%NOCOMPILE
%~~vis
\begin{xten}
class Ambig {
  public val f : (Int)=>Int =  (x:Int) => x*x;
  public def f(y:int) = y+1;
  public def example() {
      val v = this.f(10);
      // is v 100, or 11?
  }
}
\end{xten}
%~~siv
%
%~~neg
\end{ex}

In the case where a syntactic form \xcdmath"E.m(F$_1$, $\ldots$, F$_n$)" could
be resolved as either a method call, or the application of a field \xcd`E.m`
to some arguments, it will be treated as a method call.  
The application of \xcd`E.m` to some arguments can be specified by adding
parentheses:  \xcdmath"(E.m)(F$_1$, $\ldots$, F$_n$)".

\begin{ex}

%~~gen ^^^ MethodResolution_disamb_c
%package MethodResolution_disamb_c;
%NOCOMPILE
%~~vis
\begin{xten}
class Disambig {
  public val f : (Int)=>Int =  (x:Int) => x*x;
  public def f(y:int) = y+1;
  public def example() {
      assert(  this.f(10)  == 11  );
      assert( (this.f)(10) == 100 );
  }
}
\end{xten}
%~~siv
%
%~~neg

\end{ex}

Similarly, it is possible to have a method with the same name as a struct, say
\xcd`ambig`, giving an ambiguity as to whether \xcd`ambig()` is a struct
constructor invocation or a method invocation.  This ambiguity is resolved by
treating it as a method invocation.  If the constructor invocation is desired,
it can be achieved by including the optional \xcd`new`.  That is, 
\xcd`new ambig()` is struct constructor invocation; \xcd`ambig()` is a 
method invocation.


\section{Static Nested Classes}
\label{StaticNestedClasses}
\index{class!static nested}
\index{class!nested}
\index{static nested class}

One class (or struct or interface) may be nested within another.  The simplest
way to do this is as a \xcd`static` nested class, written by putting one class
definition at top level inside another, with the inner one having a
\xcd`static` modifier.  
For most purposes, a static nested class behaves like a top-level class.
However, a static nested class has access to private static
fields and methods of its containing class.  

Nested interfaces and static structs are permitted as well.

%~~gen ^^^ InnerClasses10
% package Classes.StaticNested; 
% NOTEST
%~~vis
\begin{xten}
class Outer {
  private static val priv = 1;
  private static def special(n:Int) = n*n;
  public static class StaticNested {
     static def reveal(n:Int) = special(n) + priv;
  }
}
\end{xten}
%~~siv
%
%~~neg

\section{Inner Classes}
\label{InnerClasses}
\index{class!inner}
\index{inner class}


Non-static nested classes are called {\em inner classes}. An inner class
instance can be thought of as a very elaborate member of an object --- one
with a full class structure of its own.   The crucial characteristic of an
inner class instance is that it has an implicit reference to an instance of
its containing class.  

\begin{ex}
This feature is particularly useful when an instance of the inner class makes
no sense without reference to an instance of the outer, and is closely tied to
it.  For example, consider a range class, describing a span of integers {$m$}
to {$n$}, and an iterator over the range.  The iterator might as well have
access to the range object, and there is little point to discussing
iterators-over-ranges without discussing ranges as well.
In the following example, the inner class \xcd`RangeIter` iterates over the
enclosing \xcd`Range`.  

It has its own private cursor field \xcd`n`, telling
where it is in the iteration; different iterations over the same \xcd`Range`
can exist, and will each have their own cursor.
It is perhaps unwise to use the name \xcd`n` for a field of the inner class,
since it is also a field of the outer class, but it is legal.  (It can happen
by accident as well -- \eg, if a programmer were to add a field \xcd`n` to a
superclass of the  outer class, the inner class would still work.)
It does not even
interfere with the inner class's ability to refer to the outer class's \xcd`n`
field: the cursor initialization 
refers to the \xcd`Range`'s lower bound through a fully qualified name
\xcd`Range.this.n`.
The initialization of its \xcd`n` field refers to the outer class's \xcd`m` field, which is
not shadowed and can be referred to directly, as \xcd`m`.


%~~gen ^^^ InnerClasses20
% package Classes.InnerClasses_a; 
% NOTEST
%~~vis
\begin{xten}
class Range(m:Int, n:Int) implements Iterable[Int]{
  public def iterator ()  = new RangeIter();
  private class RangeIter implements Iterator[Int] {
     private var n : Int = m;
     public def hasNext() = n <= Range.this.n;
     public def next() = n++;
  }
  public static def main(argv:Array[String](1)) {
    val r = new Range(3,5);
    for(i in r) Console.OUT.println("i=" + i);
  }
}
\end{xten}
%~~siv
%
%~~neg
\end{ex}

An inner class has full access to the members of its enclosing class, both
static and instance.  In particular, it can access \xcd`private` information,
just as methods of the enclosing class can.  

An inner class can have its own members.  
Inside instance methods of an inner class, \xcd`this` refers to the instance
of the {\em inner} class.  The instance of the outer class can be accessed as
{\em Outer}\xcd`.this` (where {\em Outer} is the name of the outer class).
If, for some dire reason, it is necessary to have an inner class within an inner
class, the innermost class can refer to the \xcd`this` of either outer class
by using its name.

An inner class can inherit from any class in scope,
with no special restrictions. \xcd`super` inside an inner class refers to the
inner class's superclass. If it is necessary to refer to the outer classes's
superclass, use a qualified name of the form {\em Outer}\xcd`.super`.

The only restriction placed on the members of inner classes is that the static
fields of an inner class must be compile-time constant expressions. 

\index{inner class!extending}
Consider
an inner class \xcd`IC1` of some outer class \xcd`OC1`, being extended by 
another class \xcd`IC2`. However, since an \xcd`IC1` only exists as a
dependent of an \xcd`OC1`, each \xcd`IC2` must be associated with an \xcd`OC1`
--- or a subtype thereof --- as well.   So, \xcd`IC2` must be an inner class
of either \xcd`OC1` or some subclass \xcd`OC2 <: OC1`.

\begin{ex}For example, one often extends an
inner class when one extends its outer class: 
%~~gen ^^^ InnerClasses30
% package Classes.Innerclasses.Are.For.Innermasses;
%~~vis
\begin{xten}
class OC1 {
   class IC1 {}
}
class OC2 extends OC1 {
   class IC2 extends IC1 {} 
}
\end{xten}
%~~siv
%
%~~neg
\end{ex}


The hiding of method names has one fine point.  If an inner class defines a
method named \xcd`doit`, then {\em all} methods named \xcd`doit` from the
outer class are hidden --- even if they have different argument types than the
one defined in the inner class.
They are still accessible via
\xcd`Outer.this.doit()`, but not simply via \xcd`doit()`.  The following code
is correct, but would not be correct if the ERROR line were uncommented.

%~~gen ^^^ InnerClasses40
% package Classes.Innerclasses.StupidOverloading; 
% NOTEST
%~~vis
\begin{xten}
class Outer {
  def doit() {}
  def doit(String) {}
  class Inner { 
     def doit(Boolean, Outer) {}
     def example() {
        doit(true, Outer.this);
        Outer.this.doit();
        //ERROR: doit("fails");
     }
  }
}
\end{xten}
%~~siv
%
%~~neg


\subsection{Constructors and Inner Classes}
\label{sect:InnerClassCtor}
\index{inner class!constructor}

If \xcd`IC` is an inner class of \xcd`OC`, then instance code in the body of
\xcd`OC` can create instances of \xcd`IC` simply by calling a constructor
\xcd`new IC(...)`: 
%~~gen ^^^ InnerClasses50
% package Classes.Innerclasses.Constructors.Easy;
%~~vis
\begin{xten}
class OC {
  class IC {}
  def method(){
    val ic = new IC();
  }
}
\end{xten}
%~~siv
%
%~~neg

Instances of \xcd`IC` can be constructed from elsewhere as well.  Since every
instance of \xcd`IC` is associated with an instance of \xcd`OC`, an \xcd`OC`
must be supplied to the \xcd`IC` constructor.  The syntax for doing so is: 
\xcd`oc.new IC()`.  For example: 
%~~gen ^^^ InnerClasses60
% package Classes.Inner_a; 
% NOTEST
% /*NONSTATIC*/
%~~vis
\begin{xten}
class OC {
  class IC {}
  static val oc1 = new OC();
  static val oc2 = new OC();
  static val ic1 = oc1.new IC();
  static val ic2 = oc2.new IC();
}
class Elsewhere{
  def method(oc : OC) {
    val ic = oc.new IC();
  }
}
\end{xten}
%~~siv
%
%~~neg


\section{Local Classes}
\label{sect:LocalClasses}

Classes can be defined and instantiated in the middle of methods and other
code blocks.
A local class in a static method is a static class; a local class in an
instance method is an inner class.
 Local classes are local to the block in which they are defined.
They have access to almost everything defined at that point in the method; the
one exception is that they cannot use \xcd`var` variables. Local classes
cannot be \xcd`public` \xcd`protected`, or \xcd`private`, because they are
only visible from within the block of declaration. They cannot be
\xcd`static`.

\begin{ex}
The following example illustrates the use of a local class \xcd`Local`, 
defined inside the body of method \xcd`m()`. 
%~~gen ^^^ InnerClasses5p9v
% package InnerClasses5p9v;
% NOTEST
%~~vis
\begin{xten}
class Outer {
  val a = 1;
  def m() {
    val a = -2; 
    val b = 2;
    class Local {
      val a = 3;
      def m() = 100*Outer.this.a + 10*b + a; 
    }
    val l : Local = new Local();
    assert l.m() == 123;
  }//end of m()
}
\end{xten}
%~~siv
% class Hook{ def run() {
%   val o <: Outer = new Outer();
%   o.m();
%   return true;
% } }
%~~neg
Note that the middle \xcd`a`,
whose value is \xcd`-2`, is not accessible inside of \xcd`Local`; it is
shadowed by \xcd`Local`'s \xcd`a` field.  \xcd`Outer`'s \xcd`a` is also
shadowed, but the notation \xcd`Outer.this` gives a reference to the enclosing
\xcd`Outer` object.  There is no corresponding notation to access shadowed local
variables from the enclosing block; if you need to get them, rename the fields
of \xcd`Local`.    
\end{ex}





\section{Anonymous Classes}
\index{class!anonymous}
\index{anonymous class}

It is possible to define a new class and instantiate it as part of an
expression.  The new class can extend an extant class or interface.  Its body
can include all of the usual members of a class. It can refer to any
identifiers available at that point in the expression --- except for \xcd`var`
variables.  An anonymous class in a static context is a static inner class.

Anonymous classes are useful when you want to package several pieces of
behavior together (a single piece of behavior can often be expressed as a
function, which is syntactically lighter-weight), or if you want to extend and
vary an extant class without going through the trouble of actually defining a
whole new class.

The syntax for an anonymous class is a constructor call followed immediately
by a braced class body: \xcd`new C(1){def foo()=2;}`.

\begin{ex}In the following minimalist example, the abstract class \xcd`Choice`
encapsulates a decision.   A \xcd`Choice` has a \xcd`yes()` and a \xcd`no()`
method.  The \xcd`choose(b)` method will invoke one of the two.  \xcd`Choice`s
also have names.

The \xcd`main()` method creates a specific \xcd`Choice`.  \xcd`c` is not a
immediate instance of \xcd`Choice` --- as an abstract class, \xcd`Choice` has
no immediate instances. \xcd`c` is an instance of an anonymous class which
inherits from \xcd`Choice`, but supplies \xcd`yes()` and \xcd`no()` methods.
These methods modify the contents of the \xcd`Cell[Int]` \xcd`n`.  (Note that,
as \xcd`n` is a local variable, it would take a few lines more coding to
extract \xcd`c`'s class, name it, and make it an inner class.)  The call to
\xcd`c.choose(true)`  will call \xcd`c.yes()`, incrementing \xcd`n()`, in a
rather roundabout manner.

%~~gen ^^^ InnerClasses70
% package ClassInnnerclassAnonclassOw; 
%~~vis
\begin{xten}
abstract class Choice(name: String) {
  def this(name:String) {property(name);}
  def choose(b:Boolean) { 
     if (b) this.yes(); else this.no(); }
  abstract def yes():void;
  abstract def no():void;
}

class Example {
  static def main(Array[String]) {
    val n = new Cell[Int](0);
    val c = new Choice("Inc Or Dec") {
      def yes() { n() += 1; }
      def no()  { n() -= 1; }
      };
    c.choose(true);
    Console.OUT.println("n=" + n());
  }
}

\end{xten}
%~~siv
%
%~~neg
\end{ex}

Anonymous classes have many of the features of classes in general.  A few
features are unavailable because they don't make sense.

\begin{itemize}

\item Anonymous classes don't have constructors.  Since they don't have names,
      there's no way a constructor could get called in the ordinary way.
      Instead, the \xcd`new C(...)` expression must match a constructor of the
      parent class \xcd`C`, which will be called to initialize the
      newly-created object of the anonymous class.

\item The \xcd`public`,
      \xcd`private`, and \xcd`protected`  modifiers don't make sense for
      anonymous classes:  
      Anonymous classes, being anonymous,
      cannot be referenced at all, so references to them can't be public,
      private, or protected.

\item Anonymous classes cannot be \xcd`abstract`.  Since they only exist in
      combination with a constructor call, they must be constructable.  The
      parent class of the anonymous class may be abstract, or may be an
      interface; in this case, the anonymous class must provide all the
      methods that the parent demands.

\item Anonymous classes cannot have explicit \xcd`extends` or \xcd`implements`
      clauses; there's no place in the syntax for them. They have a single
      parent and that is that. 
\end{itemize}

