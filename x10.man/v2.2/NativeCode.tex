\chapter{Native Code Integration}
\label{NativeCode}
\index{native code}

At times it becomes necessary to call non-X10 code from X10, perhaps to make
use of specialized libraries in other languages or to write more precisely
controlled code than X10 generally makes available. 

The
\xcd`@Native(lang,code) Phrase` annotation from \xcd`x10.compiler.Native` in
X10 can be used to tell the X10 compiler to generate \xcd`code` for certain
kinds of \xcd`Phrase`, instead of what it would normally compile to, when
compiling to the \xcd`lang` back end.

The compiler cannot analyze native code the same way it analyzes X10 code.  In
particular, \xcd`@Native` fields and methods must be explicitly typed; the
compiler will not infer types.

\section{Native {\tt static} Methods}

\xcd`static` methods can be given native implementations.  Note that these
implementations are syntactically {\em expressions}, not statements, in C++ or
Java.   Also, it is possible (and common) to provide native implementations
into both Java and C++ for the same method.
%~~gen ^^^ extern10
% package Extern.or_current_turn;
%~~vis
\begin{xten}
import x10.compiler.Native;
class Son {
  @Native("c++", "printf(\"Hi!\")")
  @Native("java", "System.out.println(\"Hi!\")")
  static def printNatively():void = {};
}
\end{xten}
%~~siv
%
%~~neg

If only some back-end languages are given, the X10 code will be used for the
remaining back ends: 
%~~gen ^^^ extern20
% package Extern.or.burn;
%~~vis
\begin{xten}
import x10.compiler.Native;
class Land {
  @Native("c++", "printf(\"Hi from C++!\")")
  static def example():void = {
    x10.io.Console.OUT.println("Hi from X10!");
  };
}
\end{xten}
%~~siv
%
%~~neg

The \xcd`native` modifier on methods indicates that the method must not have
an X10 code body, and \xcd`@Native` implementations must be given for all back
ends:
%~~gen ^^^ extern30
% package Extern.or_maybe_getting_back_together;
%~~vis
\begin{xten}
import x10.compiler.Native;
class Plants {
  @Native("c++", "printf(\"Hi!\")")
  @Native("java", "System.out.println(\"Hi!\")")
  static native def printNatively():void;
}
\end{xten}
%~~siv
%
%~~neg


Values may be returned from external code to X10.  Scalar types in Java and
C++ correspond directly to the analogous types in X10.  
%~~gen ^^^ extern40
% package Extern.hardy;
%~~vis
\begin{xten}
import x10.compiler.Native;
class Return {
  @Native("c++", "1")
  @Native("java", "1")
  static native def one():Int;
}
\end{xten}
%~~siv
%
%~~neg
Types are {\em not} inferred for methods marked as \xcd`@Native`.

Parameters may be passed to external code.  \xcd`(#1)`  is the first parameter,
\xcd`(#2)` the second, and so forth.  (\xcd`(#0)` is the name of the enclosing
class.)  Be aware that this is macro substitution rather than normal parameter
passing; \eg, if the first actual parameter is \xcd`i++`, and \xcd`(#1)`
appears twice in the external code, \xcd`i` will be incremented twice.
For example, a (ridiculous) way to print the sum of two numbers is: 
%~~gen ^^^ extern50
% package Extern.or_turnabout_is_fair_play;
%~~vis
\begin{xten}
import x10.compiler.Native;
class Species {
  @Native("c++", "printf(\"Sum=%d\", ((#1)+(#2)) )")
  @Native("java", "System.out.println(\"\" + ((#1)+(#2)))")
  static native def printNatively(x:Int, y:Int):void;
}
\end{xten}
%~~siv
%
%~~neg


Static variables in the class are available in the external code. 
\iffalse
%~~gen ^^^ extern60
%package Extern.or.Die;
%~~vis
\begin{xten}
import x10.compiler.Native;
class Ability {
  static val a : Int = 1;
  @Native("java", "a")
  static native def fromStatic():Int;
}
\end{xten}
%~~siv
%
%~~neg
\fi



\section{Native Blocks}

Any block may be annotated with \xcd`@Native(lang,stmt)`, indicating that, in
the given back end, it should be implemented as \xcd`stmt`. All 
variables  from the surrounding context are available inside \xcd`stmt`. For
example, the method call \xcd`born.example(10)`, if compiled to Java, changes
the field \xcd`y` of a \xcd`Born` object to 10. If compiled to C++ (for which
there is no \xcd`@Native`), it sets it to 3. 
%~~gen ^^^ extern70
%package Extern.me.plz; 
%~~vis
\begin{xten}
import x10.compiler.Native;
class Born {
  var y : Int = 1; 
  public def example(x:Int):Int{
    @Native("java", "y=x;") 
    {y = 3;}
    return y;
  }
}
\end{xten}
%~~siv
%
%~~neg

Note that the code being replaced is a statement -- the block \xcd`{y = 3;}`
in this case -- so the replacement should also be a statement. 


Other X10 constructs may or may not be available in Java and/or C++ code.  For
example, type variables do not correspond exactly to type variables in either
language, and may not be available there.  The exact compilation scheme is
{\em not} fully specified.  You may inspect the generated Java or C++ code and
see how to do specific things, but there is no guarantee that fancy extern
coding will continue to work in later versions of X10.



The full facilities of C++ or Java are available in native code blocks.
However, there is no guarantee that advanced features behave sensibly. You
must follow the exact conventions that the code generator does, or you will
get unpredictable results.  Furthermore, the code generator's conventions may
change without notice or documentation from version to version.  In most cases
the  code should either be a very simple expression, or a method
or function call to external code.


\section{External Java Code}

When X10 is compiled to Java, mentioning a Java class name in native code will
cause the Java compiler to find it in the sourcepath or classpath, in the
usual way.  This requires no particular extra work from the programmer.


\section{External C++ Code}

C++ code can be linked to X10 code, either by writing auxiliary C++ files and
adding them with suitable annotations, or by linking libraries.

\subsection{Auxiliary C++ Files}

Auxiliary C++ code can be written in \xcd`.h` and \xcd`.cc` files, which
should be put in the same directory as the the X10 file using them.
Connecting with the library uses the \xcd`@NativeCPPInclude(dot_h_file_name)`
annotation to include the header file, and the 
\xcd`@NativeCPPCompilationUnit(dot_cc_file_name)` annotation to include the
C++ code proper.  For example: 

{\bf MyCppCode.h}: 
\begin{xten}
void foo();
\end{xten}


{\bf MyCppCode.cc}:
\begin{xten}
#include <cstdlib>
#include <cstdio>
void foo() {
    printf("Hello World!\n");
}
\end{xten}

{\bf Test.x10}:
\begin{xten}
import x10.compiler.Native;
import x10.compiler.NativeCPPInclude;
import x10.compiler.NativeCPPCompilationUnit;

@NativeCPPInclude("MyCPPCode.h")
@NativeCPPCompilationUnit("MyCPPCode.cc")
public class Test {
    public static def main (args:Array[String](1)) {
        { @Native("c++","foo();") {} }
    }
}
\end{xten}

\subsection{C++ System Libraries}

If we want to additionally link to more libraries in \xcd`/usr/lib` for
example, it is necessary to adjust the post-compilation directly.  The
post-compilation is the compilation of the C++ which the X10-to-C++ compiler
\xcd`x10c++` produces.  

The mechanism used for this is the \xcd`-post` command line parameter to
\xcd`x10c++`. The following example shows how to compile \xcd`blas` into the
executable via post compiler parameters.

\begin{xten}
x10c++ Test.x10 -post '# # -I /usr/local/blas # -L /usr/local/blas -lblas'
\end{xten}

\begin{itemize}
\item The first \xcd`#` means to use the default compiler for the architecture (from
      x10rt properties file). 
\item The second \xcd`#` is substituted for the .cc files and CXXFLAGS that would
      ordinarily be used. 
\item The third \xcd`#` is substituted for the libraries and LDFLAGS that would
      ordinarily used.
\item For the second and third, if a \xcd`%` is used instead of a \xcd`#` then the
      the substitution does not occur in that position.  The \xcd`%` is
      erased.  The desired parameter value should appear after the \xcd`%` on
      the line. This allows a complete override of the postcompiler behaviour.
\end{itemize}
