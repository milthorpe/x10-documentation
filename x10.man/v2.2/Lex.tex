\chapter{Lexical and Grammatical structure}


Lexically a program consists of a stream of white space, comments,
identifiers, keywords, literals, separators and operators, all of them
composed of Unicode characters in the UTF-8 (or US-ASCII) encoding. 

\section{Whitespace}
\index{white space}
% Whitespace \index{whitespace} follows \java{} rules \cite[Chapter 3.6]{jls2}.
ASCII space, horizontal tab (HT), form feed (FF) and line
terminators constitute white space.

\section{Comments}
\index{comment}
% Comments \index{comments} follows \java{} rules
% \cite[Chapter 3.7]{jls2}. 
All text included within the ASCII characters ``\xcd"/*"'' and
``\xcd"*/"'' is
considered a comment and ignored; nested comments are not
allowed.  All text from the ASCII characters
``\xcd"//"'' to the end of line is considered a comment and is ignored.

\section{Identifiers}
\index{identifier}
\index{variable name}

Identifiers consist of a single letter followed by zero or more letters or
digits. The letters are the ASCII characters \xcd`a` through \xcd`z`, \xcd`A`
through \xcd`Z`, and \xcd`_`. Digits are defined as the ASCII characters
\xcd"0" through \xcd"9". Case is significant; \xcd`a` and \xcd`A` are distinct
identifiers, \xcd`as` is a keyword, but \xcd`As` and \xcd`AS` are identifiers.
(However, case is insignificant in the hexadecimal numbers, exponent markers,
and type-tags of numeric literals -- 0xbabe = 0XBABE.)

In addition, any string of characters may be enclosed in backquotes \xcd"`" to
form an identifier -- though the backquote character itself, and the backslash 
character, must be quoted by a backslash if they are to be included.
This allows,
for example, keywords to be used as identifiers.  The following are
backquoted identifiers: 
%~~gen ^^^ Lex1p9o
% package Lex1p9o;
% 
% class Backyquotey {
% def hokey(e:Int, a:Int, b:Int, d:Int, c:Int) = 0;
% def example() { 
% val `while`=0; val `!`=0; val  `(unbalanced(`=0;  val `\`\\`=0; val `0` = 0;
% hokey(
%~~vis
\begin{xten}
`while`, `!`, `(unbalanced(`,  `\`\\`, `0`
\end{xten}
%~~siv
% ); } } 
%~~neg
\noindent
Certain  back ends and compilation options do not support all choices of
identifier.  

\section{Keywords}
\index{keywords}
\Xten{} uses the following keywords:\\
{\tt 
\begin{tabular}{llllll}
    abstract &
    as&
    assert&
    async&
    at\\
    athome &
    ateach&
    atomic&
    break&
    case\\
    catch &
    class&
    clocked&
    continue&
    def\\
    default &
    do&
    else&
    extends&
    false\\
    final &
    finally&
    finish&
    for&
    goto\\
    haszero &
    here&
    if&
    implements&
    import\\
    in &
    instanceof&
    interface&
    native&
    new\\
    null &
    offer&
    offers&
    operator&
    package\\
    private &
    property&
    protected&
    public&
    return\\
    self &
    static&
    struct&
    super&
    switch\\
    this &
    throw&
    transient&
    true&
    try\\
    type &
    val&
    var&
    void&
    when\\
    while

\end{tabular}
}

Keywords may be used as identifiers by enclosing them in backquotes:
\xcd"`new`" is an identifier, \xcd`new` is a keyword but not an identifier.  

Note that the primitive type names are not considered keywords.

\section{Literals}\label{Literals}\index{literal}

Briefly, \XtenCurrVer{} uses fairly standard syntax for its literals:
integers, unsigned integers, floating point numbers, booleans, 
characters, strings, and \xcd"null".  The most exotic points are (1) unsigned
numbers are marked by a \xcd`u` and cannot have a sign; (2) \xcd`true` and
\xcd`false` are the literals for the booleans; and (3) floating point numbers
are \xcd`Double` unless marked with an \xcd`f` for \xcd`Float`. 

Less briefly, we use the following abbreviations: 
\begin{displaymath}
\begin{array}{rcll}
d &=& \mbox{one or more decimal digits} \mbox{only starting with 0 if it is 0}\\
d_8 &=& \mbox{one or more octal digits}\\
d_{16} &=& \mbox{one or more hexadecimal digits, using \xcd`a`-\xcd`f`
or \xcd`A`-\xcd`F`
for 10-15}\\
i &=& d  
        \mathbin{|} {\tt 0} d_8 
        \mathbin{|} {\tt 0x} d_{16}
        \mathbin{|} {\tt 0X} d_{16}
\\
s &=& \mbox{optional \xcd`+` or \xcd`-`}\\
b &=& d 
          \mathbin{|} d {\tt .}
          \mathbin{|} d {\tt .} d
          \mathbin{|}  {\tt .} d \\
x &=& ({\tt e } \mathbin{|} {\tt E})
         s
         d \\
f &=& b x
\end{array}
\end{displaymath}

\begin{itemize}

\item \xcd`true` and \xcd`false` are the \xcd`Boolean` literals. \index{Boolean!literal}\index{literal!Boolean}

\item \xcd`null` is a literal for the null value.  It has type
      \xcd`Any{self==null}`. \index{null} \index{object!literal}

\item \index{Int!literal}\index{literal!integer}
\xcd`Int` literals have the form {$si$}; \eg, \xcd`123`,
      \xcd`-321` are decimal \xcd`Int`s, \xcd`0123` and \xcd`-0321` are octal
      \xcd`Int`s, and \xcd`0x123`, \xcd`-0X321`,  \xcd`0xBED`, and \xcd`0XEBEC` are
      hexadecimal \xcd`Int`s.  

\item \xcd`Long` literals have the form {$si{\tt l}$} or
      {$si{\tt L}$}. \Eg, \xcd`1234567890L`  and \xcd`0xBABEL` are \xcd`Long` literals. 

\item \xcd`UInt` literals have the form {$i{\tt u}$} or {$i {\tt U}$}.
      \Eg, \xcd`123u`, \xcd`0123u`, and \xcd`0xBEAU` are \xcd`UInt` literals.

\item \xcd`ULong` literals have the form {$i {\tt ul}$} or {$i {\tt
      lu}$}, or capital versions of those.  For example, 
      \xcd`123ul`, \xcd`0124567012ul`,  \xcd`0xFLU`, \xcd`OXba1eful`, and \xcd`0xDecafC0ffeefUL` are \xcd`ULong`
      literals. 

\item \xcd`Short` literals have the form {$si{\tt s}$} or
      {$si{\tt S}$}. \Eg,  414S, \xcd`OxACES` and \xcd`7001s` are short
      literals. 

\item \xcd`UShort` literals  form {$i {\tt us}$} or {$i {\tt
      su}$}, or capital versions of those.  For example, \xcd`609US`, 
      \xcd`107us`, and \xcd`OxBeaus` are unsigned short literals.

\item \xcd`Byte` literals have the form  {$si{\tt y}$} or
      {$si{\tt Y}$}.  (The letter \xcd`B` cannot be used for bytes, as it is
      a hexadecimal digit.)  \xcd`50Y` and \xcd`OxBABY` are byte literals.

\item \xcd`UByte` literals have the form {$i {\tt uy}$} or {$i {\tt yu}$}, or
      capitalized versions of those.  For example, \xcd`9uy` and \xcd`OxBUY`
      are \xcd`UByte` literals.
      


\item \xcd`Float` literals have the form {$s f {\tt f}$} or  {$s
\index{float!literal}
\index{literal!float}
      f {\tt F}$}.  Note that the floating-point marker letter \xcd`f` is
      required: unmarked floating-point-looking literals are \xcd`Double`. 
      \Eg, \xcd`1f`, \xcd`6.023E+32f`, \xcd`6.626068E-34F` are \xcd`Float`
      literals. 

\item \xcd`Double` literals have the form {$s f$}\footnote{Except that
\index{double!literal}
\index{literal!double}
      literals like \xcd`1` 
      which match both {$i$} and {$f$} are counted as
      integers, not \xcd`Double`; \xcd`Double`s require a decimal
      point, an exponent, or the \xcd`d` marker.
      }, {$s f {\tt
      D}$}, and {$s f {\tt d}$}.  
      \Eg, \xcd`0.0`, \xcd`0e100`, \xcd`1.3D`,  \xcd`229792458d`, and \xcd`314159265e-8`
      are \xcd`Double` literals.

\item 
\index{char!literal}
\index{literal!char}
\xcd`Char` literals have one of the following forms: 
      \begin{itemize}
      \item \xcd`'`{\it c}\xcd`'` where {\em c} is any printing ASCII
            character other than 
            \xcd`\` or \xcd`'`, representing the character {\em c} itself; 
            \eg, \xcd`'!'`;
      \item \xcd`'\b'`, representing backspace;
      \item \xcd`'\t'`, representing tab;
      \item \xcd`'\n'`, representing newline;
      \item \xcd`'\f'`, representing form feed;
      \item \xcd`'\r'`, representing return;
      \item \xcd`'\''`, representing single-quote;
      \item \xcd`'\"'`, representing double-quote;
      \item \xcd`'\\'`, representing backslash;
      \item \xcd`'\`{\em dd}\xcd`'`, where {\em dd} is one or more octal
            digits, representing the one-byte character numbered {\em dd}; it
            is an error if {\em dd}{$>0377$}.      
      \end{itemize}

\item
\index{string!literal} 
\index{literal!string}
\xcd`String` literals consist of a double-quote \xcd`"`, followed by
      zero or more of the contents of a \xcd`Char` literal, followed by
      another double quote.  \Eg, \xcd`"hi!"`, \xcd`""`.


\end{itemize}



\section{Separators}
\Xten{} has the following separators and delimiters:
\begin{xten}
( )  { }  [ ]  ;  ,  .
\end{xten}

\section{Operators}
\index{operator}
\Xten{} has the following operator,  type constructor, and miscellaneous symbols.  (\xcd`?` and
\xcd`:` comprise a single ternary operator, but are written separately.)
\begin{xten}
==  !=  <   >   <=  >=
&&  ||  &   |   ^
<<  >>  >>>
+   -   *   /   %
++  --  !   ~
&=  |=  ^=
<<= >>= >>>=
+=  -=  *=  /=  %=
=   ?   :   =>  ->
<:  :>  @   ..
**  !~  -<  >-
\end{xten}

The precedence of the operators is as follows.  Earlier rows of the
table have higher precedence than later rows, binding more tightly.
For example, \xcd`a+b*c<d` parses as \xcd`(a+(b*c))<d`, 
and \xcd`-1 as Byte` parses as \xcd`-(1 as Byte)`. 

\begin{tabular}{l}
postfix \xcd`()`\\
\xcd`as T`, postfix \xcd`++`, postfix \xcd`--`\\
unary \xcd`-`, unary \xcd`+`, prefix \xcd`++`, prefix \xcd`--`\\
unary operators \xcd`!`, \xcd`~`, \xcd`^`, \xcd`*`, \xcd`|`,
\xcd`&`, \xcd`/`, and \xcd`%`
\\
\xcd`..`\\
\xcd`*     `  \xcd`/     `  \xcd`%   ` \xcd`**` \\
\xcd`+` \xcd`     -` \\
\xcd`<<    ` \xcd`>>    ` \xcd`>>>   ` \xcd`->   >-    -<     <-     !` \\
\xcd`>     ` \xcd`>=    ` \xcd`<     ` \xcd`<=  ` 
\xcd`instanceof` \\
\xcd`==    `  \xcd`!=    `  \xcd`!     ` \xcd`!~   `   \\

\xcd`&` \\
\xcd`^` \\
\xcd`|` \\
\xcd`&&` \\
\xcd`||` \\
\xcd`? :` \\
\xcd`=`, \xcd`*=`, \xcd`/=`, \xcd`%=`, 
\xcd`+=`, \xcd`-=`, \xcd`<<=`, \xcd`>>=`, \xcd`>>>=`,
\xcd`&=`, \xcd`^=`, \xcd`|=`
\end{tabular}


\section{Grammatical Notation}

In this manual, ordinary BNF notation is used to specify grammatical
constructions, with a few minor extensions. 
Grammatical rules look like this: \\
\begin{bbgrammar}
Adj \: Adv\opt \mbox{\xcd`happy`}\\
    \| Adv\opt \mbox{\xcd`sad`}\\ 
Adv \: \mbox{\xcd`very`}\\
    \| Adv Adv 
\end{bbgrammar}

Terms in {\em italics} are called {\bf non-terminals}.  They 
represent kinds of phrases; for example, {\em ForStmt}
(\ref{prod:ForStmt})\footnote{Grammar rules are given in \Sref{Grammar},
and referred to by equation number in that section.} describes all \xcd`for` statements.  
Equation numbers refer to the full X10 grammar, in \Sref{Grammar}.  
The small example has two non-terminals, {\em Adv} and {\em Adj}.  

Terms in \xcd`fixed-width font` are {\bf terminals}.  They represent the words
and symbols of the language itself.  In X10, the terminals are the words
described in this chapter.

A single grammatical rule has the form {$ A \mathrel{::=} X_1 X_2 ... X_n $},
where the $X_i$'s are either terminals or nonterminals.  This indicates that
the non-terminal $A$ could be an instance of $X_1$, followed by an instance of
$X_2$, \ldots, followed by an instance of $X_n$.  Multiple rules for the same
$A$ are allowed, giving several possible phrasings of $A$'s.  For brevity,
two rules with the same left-hand side are written with the left-hand side
appearing once, and the right-hand sides 
separated by $|$.  

In the {\em Adj} example, there are two rules for {\it Adv}, 
$Adv \mathrel{::=} \mbox{\xcd`very`}$ and
$Adv \mathrel{::=} Adv~Adv$.  So, an adverb could be \xcd`very`, or (by three
uses of the rule) \xcd`very very`, or, one or more \xcd`very`s.  

The notation $A^{?}$ indicates an optional $A$.  This is an ordinary
non-terminal, defined by the rules:\\ 
\begin{bbgrammar}
A\opt \:  \\
    \| A
\end{bbgrammar}
The first rule says that $A^?$ can amount to nothing; the second, that it can
amount to an $A$.  This concept shows up so often that it is worth having a
separate notation for it.  In the {\em Adj} example, an adjective phrase may be
preceded by an optional adverb. Thus, it may be \xcd`happy`, or 
\xcd`very happy`, or \xcd`very very sad`, etc. 
