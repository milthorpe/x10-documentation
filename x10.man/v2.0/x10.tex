%\documentclass[10pt,twoside,twocolumn,notitlepage]{report}
\documentclass[12pt,twoside,notitlepage]{report}
\usepackage{x10}
\usepackage{tenv}
\def\Hat{{\tt \char`\^}}
\usepackage{url}
\usepackage{times}
\usepackage{txtt}
\usepackage{ifpdf}
\usepackage{tocloft}
\usepackage{bcprules}
\usepackage{xspace}

\newif\ifdraft
%\drafttrue
\draftfalse

\pagestyle{headings}
\showboxdepth=0
\makeindex

\usepackage{commands}

\usepackage[
pdfauthor={Vijay Saraswat, Bard Bloom, Igor Peshansky, Olivier Tardieu, and David Grove},
pdftitle={Report on the Language X10},
pdfcreator={pdftex},
pdfkeywords={X10},
linkcolor=blue,
citecolor=blue,
urlcolor=blue
]{hyperref}

\ifpdf
          \pdfinfo {
              /Author   (Vijay Saraswat, Bard Bloom, Igor Peshansky, Olivier Tardieu, and David Grove)
              /Title    (Report on the Language X10)
              /Keywords (X10)
              /Subject  ()
              /Creator  (TeX)
              /Producer (PDFLaTeX)
          }
\fi

\def\headertitle{The \XtenCurrVer{} Report (Draft) }
\def\integerversion{2.0.6}

% Sizes and dimensions

%\topmargin -.375in       %    Nominal distance from top of page to top of
                         %    box containing running head.
%\headsep 15pt            %    Space between running head and text.

%\textheight 9.0in        % Height of text (including footnotes and figures, 
                         % excluding running head and foot).

%\textwidth 5.5in         % Width of text line.
\columnsep 15pt          % Space between columns 
\columnseprule 0pt       % Width of rule between columns.

\parskip 5pt plus 2pt minus 2pt % Extra vertical space between paragraphs.
\parindent 0pt                  % Width of paragraph indentation.
%\topsep 0pt plus 2pt            % Extra vertical space, in addition to 
                                % \parskip, added above and below list and
                                % paragraphing environments.


\newif\iftwocolumn

\makeatletter
\twocolumnfalse
\if@twocolumn
\twocolumntrue
\fi
\makeatother

\iftwocolumn

\oddsidemargin  0in    % Left margin on odd-numbered pages.
\evensidemargin 0in    % Left margin on even-numbered pages.

\else

\oddsidemargin  .5in    % Left margin on odd-numbered pages.
\evensidemargin .5in    % Left margin on even-numbered pages.

\fi


\newtenv{example}{Example}[section]
\newtenv{planned}{Planned}[section]

\begin{document}

% \parindent 0pt %!! 15pt                    % Width of paragraph indentation.

%\hfil {\bf 7 Feb 2005}
%\hfil \today{}

% First page

\thispagestyle{empty}

% \todo{"another" report?}

\title{ \Xten Language Specification \\
\large Version \integerversion}
%\ifdraft
\author{Vijay Saraswat, Bard Bloom, Igor Peshansky, Olivier Tardieu, and David Grove\\
\\
Please send comments to 
\texttt{vsaraswa@us.ibm.com}}
%\else
%\author{
%Vijay Saraswat \\
%Please send comments to \\
%\texttt{vsaraswa@us.ibm.com}}
%\fi
%\date{\today}

\maketitle

\newcommand\authorsc[1]{#1}
%\newcommand\authorsc[1]{\textsc{#1}}

This report provides a description of the programming
language \Xten. \Xten{} is a class-based object-oriented
programming language designed for high-performance, high-productivity
computing on high-end computers supporting $\approx 10^5$ hardware threads
and $\approx 10^{15}$ operations per second. 

\Xten{} is based on state-of-the-art object-oriented programming
languages and deviates from them only as necessary to support its
design goals. The language is intended to have a simple and clear
semantics and be readily accessible to mainstream OO programmers. It
is intended to support a wide variety of concurrent programming
idioms.
%, incuding data parallelism, task parallelism, pipelining.
%producer/consumer and divide and conquer.

%We expect to revise this document in the light of experience gained in implementing
%and using this language.

The \Xten{} design team consists of
\authorsc{David Cunningham},
\authorsc{David Grove},
\authorsc{Ben Herta},
\authorsc{Vijay Saraswat},
\authorsc{Avraham Shinnar},
\authorsc{Mikio Takeuchi},
\authorsc{Olivier Tardieu}.

Past members include
\authorsc{Shivali Agarwal}, 
\authorsc{Bowen Alpern}, 
\authorsc{David Bacon}, 
\authorsc{Raj Barik}, 
\authorsc{Ganesh Bikshandi}, 
\authorsc{Bob Blainey}, 
\authorsc{Bard Bloom}, 
\authorsc{Philippe Charles}, 
\authorsc{Perry Cheng}, 
\authorsc{Christopher Donawa}, 
\authorsc{Julian Dolby}, 
\authorsc{Kemal Ebcio\u{g}lu},
\authorsc{Stephen Fink},
\authorsc{Robert Fuhrer},
\authorsc{Patrick Gallop}, 
\authorsc{Christian Grothoff}, 
\authorsc{Hiroshi Horii}, 
\authorsc{Kiyokuni Kawachiya}, 
\authorsc{Allan Kielstra}, 
\authorsc{Sreedhar Kodali}, 
\authorsc{Sriram Krishnamoorthy}, 
\authorsc{Yan Li}, 
\authorsc{Bruce Lucas},
\authorsc{Yuki Makino}, 
\authorsc{Nathaniel Nystrom},
\authorsc{Igor Peshansky}, 
\authorsc{Vivek Sarkar},
\authorsc{Armando Solar-Lezama},  
\authorsc{S. Alexander Spoon}, 
\authorsc{Toshio Suganuma}, 
\authorsc{Sayantan Sur}, 
\authorsc{Toyotaro Suzumura}, 
\authorsc{Christoph von Praun},
\authorsc{Leena Unnikrishnan},
\authorsc{Pradeep Varma}, 
\authorsc{Krishna Nandivada Venkata},
\authorsc{Jan Vitek}, 
\authorsc{Hai Chuan Wang}, 
\authorsc{Tong Wen}, 
\authorsc{Salikh Zakirov}, and
\authorsc{Yoav Zibin}.


For extended discussions and support we would like to thank: 
Gheorghe Almasi,
Robert Blackmore,
Rob O'Callahan, 
Calin Cascaval, 
Norman Cohen, 
Elmootaz Elnozahy, 
John Field,
Kevin Gildea,
Chulho Kim,
Orren Krieger, 
Doug Lea, 
John McCalpin, 
Paul McKenney, 
Josh Milthorpe,
Andrew Myers,
Filip Pizlo, 
Ram Rajamony,
R.~K. Shyamasundar, 
V.~T. Rajan, 
Frank Tip,
Mandana Vaziri,
and
Hanhong Xue.


We thank Jonathan Rhees and William Clinger with help in obtaining the
\LaTeX{} style file and macros used in producing the Scheme report,
on which this document is based. We acknowledge the influence of
the $\mbox{\Java}^{\mbox{\textsc{tm}}}$ Language
Specification \cite{jls2}, the Scala language specification
\cite{scala-spec}, and ZPL \cite{zpl}.
%document, as evidenced by the numerous citations in the text.

This document specifies the language corresponding to Version
\integerversion{} of the implementation. The redesign and reimplementation of arrays and rails was  done by Dave Grove and Olivier Tardieu.
 Version 1.7 of the report was co-authored by Nathaniel Nystrom. The design of structs in \Xten{} was led by Olivier Tardieu and Nathaniel Nystrom.

Earlier implementations benefited from significant contributions by
Raj Barik, 
Philippe Charles, 
David Cunningham,
Christopher Donawa, 
Robert Fuhrer,
Christian Grothoff,
Nathaniel Nystrom,  
Igor Peshansky,  
Vijay Saraswat,
Vivek Sarkar, 
Olivier Tardieu,  
Pradeep Varma, 
Krishna Nandivada Venkata, and
Christoph von Praun.
Tong Wen has written many application programs
in \Xten{}. Guojing Cong has helped in the
development of many applications.
The implementation of generics in \Xten{} was influenced by the
implementation of PolyJ~\cite{polyj} by Andrew Myers and Michael Clarkson.
 

\clearpage

{\parskip 0pt
\addtolength{\cftsecnumwidth}{0.5em}
\addtolength{\cftsubsecnumwidth}{0.5em}
%\addtolength{\cftsecindent}{0.5em}
\addtolength{\cftsubsecindent}{0.5em}
\tableofcontents
}

\chapter{Introduction}

\subsection*{Background}



The era of the mighty single-processor computer is over. Now, when more
computing power is needed, one does not buy a faster uniprocessor---one buys
another processor just like those one already has, or another hundred, or
another million, and connects them with a high-speed communication network.
Or, perhaps, one rents them instead, with a cloud computer. This gives one
whatever quantity of computer cycles that one can desire and afford.

Then, one has the problem of how to use those computer cycles effectively.
Programming a multiprocessor is far more agonizing than programming a
uniprocessor.   One can use models of computation which give somewhat of the
illusion of programming a uniprocessor.  Unfortunately, the models which give
the closest imitations of uniprocessing are very expensive to implement,
either increasing the monetary cost of the computer tremendously, or slowing
it down dreadfully. 

One response to this problem has been to move to a {\em fragmented memory
  model}. Multiple processors are programmed largely as if they were
uniprocessors, but are made to interact via a relatively language-neutral
message-passing format such as MPI \cite{mpi}. This model has enjoyed some
success: several high-performance applications have been written in this
style. Unfortunately, this model leads to a {\em loss of programmer
  productivity}: the message-passing format is integrated into the host
language by means of an application-programming interface (API), the
programmer must explicitly represent and manage the interaction between
multiple processes and choreograph their data exchange; large data-structures
(such as distributed arrays, graphs, hash-tables) that are conceptually
unitary must be thought of as fragmented across different nodes; all
processors must generally execute the same code (in an SPMD fashion) etc.

One response to this problem has been the advent of the {\em
partitioned global address space} (PGAS) model underlying languages
such as UPC, Titanium and Co-Array Fortran \cite{pgas,titanium}. These
languages permit the programmer to think of a single computation
running across multiple processors, sharing a common address
space. All data resides at some processor, which is said to have {\em
affinity} to the data.  Each processor may operate directly on the
data it contains but must use some indirect mechanism to access or
update data at other processors. Some kind of global {\em barriers}
are used to ensure that processors remain roughly in lock-step.

\Xten{} is a modern object-oriented programming language
in the PGAS family. The fundamental goal of \Xten{} is to enable
scalable, high-performance, high-productivity transformational
programming for high-end computers---for traditional numerical
computation workloads (such as weather simulation, molecular dynamics,
particle transport problems etc) as well as commercial server
workloads.

\Xten{} is based on state-of-the-art object-oriented
programming ideas primarily to take advantage of their proven
flexibility and ease-of-use for a wide spectrum of programming
problems. \Xten{} takes advantage of several years of research (e.g.,
in the context of the Java Grande forum,
\cite{moreira00java,kava}) on how to adapt such languages to the context of
high-performance numerical computing. Thus \Xten{} provides support
for user-defined {\em struct types} (such as \xcd"Int", \xcd"Float",
\xcd"Complex" etc), supports a very
flexible form of multi-dimensional arrays (based on ideas in ZPL
\cite{zpl}) and supports IEEE-standard floating point arithmetic.
Some capabilities for supporting operator overloading are also provided.

{}\Xten{} introduces a flexible treatment of concurrency, distribution
and locality, within an integrated type system. \Xten{} extends the
PGAS model with {\em asynchrony} (yielding the {\em APGAS} programming
model). {}\Xten{} introduces {\em places} as an abstraction for a
computational context with a locally synchronous view of shared
memory. An \Xten{} computation runs over a large collection of places.
Each place hosts some data and runs one or more {\em
activities}. Activities are extremely lightweight threads of
execution. An activity may synchronously (and {\em atomically}) use
one or more memory locations in the place in which it resides,
leveraging current symmetric multiprocessor (SMP) technology.  
An activity may shift to another place to execute a statement block.
\Xten{} provides weaker ordering guarantees for
inter-place data access, enabling applications to scale.  
Multiple memory locations in multiple places cannot be
accessed atomically.  {\em
Immutable} data needs no consistency management and may be freely
copied by the implementation between places.  One or more {\em clocks}
may be used to order activities running in multiple
places.  \xcd`DistArray`s, distributed arrays,  may be distributed across
multiple 
places and  support parallel collective operations. A novel
exception flow model ensures that exceptions thrown by asynchronous
activities can be caught at a suitable parent activity.  The type
system tracks which memory accesses are local. The programmer may
introduce place casts which verify the access is local at run time.
Linking with native code is supported.

\chapter{Overview of \Xten}

\Xten{} is a statically typed object-oriented language, extending a sequential
core language with \emph{places}, \emph{activities}, \emph{clocks},
(distributed, multi-dimensional) \emph{arrays} and \emph{struct} types. All
these changes are motivated by the desire to use the new language for
high-end, high-performance, high-productivity computing.

\section{Object-oriented features}

The sequential core of \Xten{} is a {\em container-based} object-oriented language
similar to \java{} and C++, and more recent languages such as Scala.  
Programmers write \Xten{} code by defining containers for data and behavior
called 
\emph{classes}
(\Sref{XtenClasses}) and
\emph{structs}
(\Sref{XtenStructs}), 
often abstracted as 
\emph{interfaces}
(\Sref{XtenInterfaces}).
X10 provides inheritance and subtyping in fairly traditional ways. 

\begin{ex}

\xcd`Normed` describes entities with a \xcd`norm()` method. \xcd`Normed` is
intended to be used for entities with a position in some coordinate system,
and \xcd`norm()` gives the distance between the entity and the origin. A
\xcd`Slider` is an object which can be moved around on a line; a
\xcd`PlanePoint` is a fixed position in a plane. Both \xcd`Slider`s and
\xcd`PlanePoint`s have a sensible \xcd`norm()` method, and implement
\xcd`Normed`.

%~~gen ^^^ Overview10
% package Overview;
%~~vis
\begin{xten}
interface Normed {
  def norm():Double;
}
class Slider implements Normed {
  var x : Double = 0;
  public def norm() = Math.abs(x);
  public def move(dx:Double) { x += dx; }
}
struct PlanePoint implements Normed {
  val x : Double, y:Double;
  public def this(x:Double, y:Double) {
    this.x = x; this.y = y;
  }
  public def norm() = Math.sqrt(x*x+y*y);
}
\end{xten}
%~~siv
%
%~~neg
\end{ex}

\paragraph{Interfaces}

An \Xten{} interface specifies a collection of abstract methods; \xcd`Normed`
specifies just \xcd`norm()`. Classes and
structs can be specified to {\em implement} interfaces, as \xcd`Slider` and
\xcd`PlanePoint` implement \xcd`Normed`, and, when they do so, must provide
all the methods that the interface demands.

Interfaces are
purely abstract. Every value of type \xcd`Normed` must be an instance of some
class like \xcd`Slider` or some struct like \xcd`PlanePoint` which implements
\xcd`Normed`; no value can be \xcd`Normed` and nothing else. 


\paragraph{Classes and Structs}



There are two kinds of containers: \emph{classes}
(\Sref{ReferenceClasses}) and \emph{structs} (\Sref{Structs}). Containers hold
data in {\em fields}, and give concrete implementations of 
methods, as \xcd`Slider` and \xcd`PlainPoint` above.

Classes are organized in a single-inheritance tree: a class may have only a
single parent class, though it may implement many interfaces and have many
subclasses. Classes may have mutable fields, as \xcd`Slider` does.

In contrast, structs are headerless values, lacking the internal organs
which give objects their intricate behavior.  This makes them less powerful
than objects (\eg, structs cannot inherit methods, though objects can), but also
cheaper (\eg, they can be inlined, and they require less space than objects).  
Structs are immutable, though their fields may be immutably set to objects
which are themselves mutable.  They behave like objects in all ways consistent
with these limitations; \eg, while they cannot {\em inherit} methods, they can
have them -- as \xcd`PlanePoint` does.

\Xten{} has no primitive classes per se. However, the standard library
\xcd"x10.lang" supplies structs and objects \xcd"Boolean", \xcd"Byte",
\xcd"Short", \xcd"Char", \xcd"Int", \xcd"Long", \xcd"Float", \xcd"Double",
\xcd"Complex" and \xcd"String". The user may defined additional arithmetic
structs using the facilities of the language.



\paragraph{Functions.}

X10 provides functions (\Sref{Closures}) to allow code to be used
as values.  Functions are first-class data: they can be stored in lists,
passed between activities, and so on.  \xcd`square`, below, is a function
which squares an \xcd`Int`.  \xcd`of4` takes an \xcd`Int`-to-\xcd`Int`
function and applies it to the number \xcd`4`.  So, \xcd`fourSquared` computes
\xcd`of4(square)`, which is \xcd`square(4)`, which is 16, in a fairly
complicated way.
%~~gen ^^^ Overview20
% package Overview.of.Functions.one;
% class Whatever{
% def chkplz() {
%~~vis
\begin{xten}
  val square = (i:Int) => i*i;
  val of4 = (f: (Int)=>Int) => f(4);
  val fourSquared = of4(square);
\end{xten}
%~~siv
%}}
%~~neg



Functions are used extensively in X10
programs.  For example, a common way to construct and initialize an \xcd`Array[Int](1)` --
that is, a fixed-length one-dimensional array of numbers, like an \xcd`int[]` in Java -- is to
pass two arguments to a factory method: the first argument being the length of
the array, and the second being a function which computes the initial value of
the \xcd`i`{$^{th}$} element.  The following code constructs a 1-dimensional
array 
initialized to the squares of 0,1,...,9: \xcd`r(0) == 0`, \xcd`r(5)==25`, etc. 
%~~gen ^^^ Overview30
% package Overview.of.Functions.two;
% class Whatevermore {
%  def plzchk(){
%    val square = (i:Int) => i*i;
%~~vis
\begin{xten}
  val r : Array[Int](1) = new Array[Int](10, square);
\end{xten}
%~~siv
%}}
%~~neg








\paragraph{Constrained Types}

X10 containers may declare {\em properties}, which are fields bound immutably
at the creation of the container.  The static analysis system understands
properties, and can work with them logically.   


For example, an implementation of matrices \xcd`Mat` might have the numbers of
rows and columns as properties.  A little bit of care in definitions allows
the definition of a \xcd`+` operation that works on matrices of the same
shape, and \xcd`*` that works on matrices with appropriately matching shapes.
%~~gen ^^^ Overview40
%package Overview.Mat2;
%~~vis
\begin{xten}
abstract class Mat(rows:Int, cols:Int) {
 static type Mat(r:Int, c:Int) = Mat{rows==r&&cols==c};
 abstract operator this + (y:Mat(this.rows,this.cols))
                 :Mat(this.rows, this.cols);
 abstract operator this * (y:Mat) {this.cols == y.rows} 
                 :Mat(this.rows, y.cols);
\end{xten}
%~~siv
%  static def makeMat(r:Int,c:Int) : Mat(r,c) = null;
%  static def example(a:Int, b:Int, c:Int) {
%    val axb1 : Mat(a,b) = makeMat(a,b);
%    val axb2 : Mat(a,b) = makeMat(a,b);
%    val bxc  : Mat(b,c) = makeMat(b,c);
%    val axc  : Mat(a,c) = (axb1 +axb2) * bxc;
%  }
%}
%~~neg



The following code typechecks (assuming that \xcd`makeMat(m,n)` is a function
which creates an \xcdmath"m$\times$n" matrix).
However, an attempt to compute \xcd`axb1 + bxc` or
\xcd`bxc * axb1` would result in a compile-time type error:
%~~gen ^^^ Overview50
%package Overview.Mat1;
% // OPTIONS: -STATIC_CALLS 
%abstract class Mat(rows:Int, cols:Int) {
%  static type Mat(r:Int, c:Int) = Mat{rows==r&&cols==c};
%  public def this(r:Int, c:Int) : Mat(r,c) = {property(r,c);}
%  static def makeMat(r:Int,c:Int) : Mat(r,c) = null;
%  abstract  operator this + (y:Mat(this.rows,this.cols)):Mat(this.rows, this.cols);
%  abstract  operator this * (y:Mat) {this.cols == y.rows} : Mat(this.rows, y.cols);
%~~vis
\begin{xten}
  static def example(a:Int, b:Int, c:Int) {
    val axb1 : Mat(a,b) = makeMat(a,b);
    val axb2 : Mat(a,b) = makeMat(a,b);
    val bxc  : Mat(b,c) = makeMat(b,c);
    val axc  : Mat(a,c) = (axb1 +axb2) * bxc;
    //ERROR: val wrong1 = axb1 + bxc;
    //ERROR: val wrong2 = bxc * axb1;
  }

\end{xten}
%~~siv
%}
%~~neg

The ``little bit of care'' shows off many of the features of constrained
types.    
The \xcd`(rows:Int, cols:Int)` in the class definition declares two
properties, \xcd`rows` and \xcd`cols`.\footnote{The class is officially declared
abstract to allow for multiple implementations, like sparse and band matrices,
but in fact is abstract to avoid having to write the actual definitions of
\xcd`+` and \xcd`*`.}  

A constrained type looks like \xcd`Mat{rows==r && cols==c}`: a type
name, followed by a Boolean expression in braces.  
The \xcd`type` declaration on the second line makes
\xcd`Mat(r,c)` be a synonym for \xcd`Mat{rows==r && cols==c}`,
allowing for compact types in many places.

Functions can return constrained types.  
The \xcd`makeMat(r,c)` method returns a \xcd`Mat(r,c)` -- a matrix whose shape
is given by the arguments to the method.    In
particular, constructors can have constrained return types to provide specific
information about the constructed values.

The arguments of methods can have type constraints as well.  The 
\xcd`operator this +` line lets \xcd`A+B` add two matrices.  The type of the
second argument \xcd`y` is constrained to have the same number of rows and
columns as the first argument \xcd`this`. Attempts to add mismatched matrices
will be flagged as type errors at compilation.

At times it is more convenient to put the constraint on the method as a whole,
as seen in the \xcd`operator this *` line. Unlike for \xcd`+`, there is no
need to constrain both dimensions; we simply need to check that the columns of
the left factor match the rows of the right. This constraint is written in
\xcd`{...}` after the argument list.  The shape of the result is computed from
the shapes of the arguments.

And that is all that is necessary for a user-defined class of matrices to have
shape-checking for matrix addition and multiplication.  The \xcd`example`
method compiles under those definitions.








\paragraph{Generic types}

Containers may have type parameters, permitting the definition of
{\em generic types}.  Type parameters may be instantiated by any X10 type.  It
is thus possible to make a list of integers \xcd`List[Int]`, a list of
non-zero integers \xcd`List[Int{self != 0}]`, or a list of people
\xcd`List[Person]`.  In the definition of \xcd`List`, \xcd`T` is a type
parameter; it can be instantiated with any type.
%~~gen ^^^ Overview60
%~~vis
\begin{xten}
class List[T] {
    var head: T;
    var tail: List[T];
    def this(h: T, t: List[T]) { head = h; tail = t; }
    def add(x: T) {
        if (this.tail == null)
            this.tail = new List[T](x, null);
        else
            this.tail.add(x);
    }
}
\end{xten}
%~~siv
%~~neg
The constructor (\xcd"def this") initializes the fields of the new object.
The \xcd"add" method appends an element to the list.
\xcd"List" is a generic type.  When  instances of \xcd"List" are
allocated, the type \param{} \xcd"T" must be bound to a concrete
type.  \xcd"List[Int]" is the type of lists of element type
\xcd"Int", \xcd"List[List[String]]" is the type of lists whose elements are
themselves lists of string, and so on.

%%BARD-HERE

\section{The sequential core of X10}

The sequential aspects of X10 are mostly familiar from C and its progeny.
\Xten{} enjoys the familiar control flow constructs: \xcd"if" statements,
\xcd"while" loops, \xcd"for" loops, \xcd"switch" statements, \xcd`throw` to
raise exceptions and \xcd`try...catch` to handle them, and so on.

X10 has both implicit coercions and explicit conversions, and both can be
defined on user-defined types.  Explicit conversions are written with the
\xcd`as` operation: \xcd`n as Int`.  The types can be constrained: 
%~~exp~~`~~`~~n:Int~~ ^^^ Overview70
\xcd`n as Int{self != 0}` converts \xcd`n` to a non-zero integer, and throws a
runtime exception if its value as an integer is zero.

\section{Places and activities}

The full power of X10 starts to emerge with concurrency.
An \Xten{} program is intended to run on a wide range of computers,
from uniprocessors to large clusters of parallel processors supporting
millions of concurrent operations. To support this scale, \Xten{}
introduces the central concept of \emph{place} (\Sref{XtenPlaces}).
A place can be thought of as a virtual shared-memory multi-processor:
a computational unit with a finite (though perhaps changing) number of
hardware threads and a bounded amount of shared memory, uniformly
accessible by all threads.



An \Xten{} computation acts on \emph{values}(\Sref{XtenObjects}) through
the execution of lightweight threads called
\emph{activities}(\Sref{XtenActivities}). 
An {\em object}
 has a small, statically fixed set of fields, each of
which has a distinct name. A scalar object is located at a single place and
stays at that place throughout its lifetime. An \emph{aggregate} object has
many fields (the number may be known only when the object is created),
uniformly accessed through an index (\eg, an integer) and may be distributed
across many places. The distribution of an aggregate object remains unchanged
throughout the computation, thought different aggregates may be distributed
differently. Objects are garbage-collected when no longer useable; there are
no operations in the language to allow a programmer to explicitly release
memory.

{}\Xten{} has a \emph{unified} or \emph{global address space}. This means that
an activity can reference objects at other places. However, an activity may
synchronously access data items only in the current place, the place in which
it is running. It may atomically update one or more data items, but only in
the current place.   If it becomes necessary to read or modify an object at
some other place \xcd`q`, the {\em place-shifting} operation \xcd`at(q;F)` can
be used, to move part of the activity to \xcd`q`.  \xcd`F` is a specification
of what information will be sent to \xcd`q` for use by that part of the
computation. 
It is easy to compute
across multiple places, but the expensive operations (\eg, those which require
communication) are readily visible in the code. 

\paragraph{Atomic blocks.}

X10 has a control construct \xcd"atomic S" where \xcd"S" is a statement with
certain restrictions. \xcd`S` will be executed atomically, without
interruption by other activities. This is a common primitive used in
concurrent algorithms, though rarely provided in this degree of generality by
concurrent programming languages.

More powerfully -- and more expensively -- X10 allows conditional atomic
blocks, \xcd`when(B)S`, which are executed atomically at some point when
\xcd`B` is true.  Conditional atomic blocks are one of the strongest
primitives used in concurrent algorithms, and one of the least-often
available. 

\paragraph{Asynchronous activities.}

An asynchronous activity is created by a statement \xcd"async S", which starts
up a new activity running \xcd`S`.  It does not wait for the new activity to
finish; there is a separate statement (\xcd`finish`) to do that.




\section{Clocks}
The MPI style of coordinating the activity of multiple processes with
a single barrier is not suitable for the dynamic network of heterogeneous
activities in an \Xten{} computation.  
X10 allows multiple barriers in a form that supports determinate,
deadlock-free parallel computation, via the \xcd`Clock` type.

A single \xcd`Clock` represents a computation that occurs in phases.
At any given time, an activity is {\em registered} with zero or more clocks.
The X10 statement \xcd`next` tells all of an activity's registered clocks that
the activity has finished the current phase, and causes it to wait for the
next phase.  Other operations allow waiting on a single clock, starting
new clocks or new activities registered on an extant clock, and so on. 

%%INTRO-CLOCK%  Activities may use clocks to repeatedly detect quiescence of arbitrary
%%INTRO-CLOCK%  programmer-specified, data-dependent set of activities. Each activity
%%INTRO-CLOCK%  is spawned with a known set of clocks and may dynamically create new
%%INTRO-CLOCK%  clocks. At any given time an activity is \emph{registered} with zero or
%%INTRO-CLOCK%  more clocks. It may register newly created activities with a clock,
%%INTRO-CLOCK%  un-register itself with a clock, suspend on a clock or require that a
%%INTRO-CLOCK%  statement (possibly involving execution of new async activities) be
%%INTRO-CLOCK%  executed to completion before the clock can advance.  At any given
%%INTRO-CLOCK%  step of the execution a clock is in a given phase. It advances to the
%%INTRO-CLOCK%  next phase only when all its registered activities have \emph{quiesced}
%%INTRO-CLOCK%  (by executing a \xcd"next" operation on the clock).
%%INTRO-CLOCK%  When a clock advances, all its activities may now resume execution.
%%INTRO-CLOCK%  

Clocks act as {barriers} for a dynamically varying collection of activities.
They generalize the barriers found in MPI style program in that an activity
may use multiple clocks simultaneously. Yet programs using clocks properly are
guaranteed not to suffer from deadlock.

%%HERE

\section{Arrays, regions and distributions}

X10 provides \xcd`DistArray`s, {\em distributed arrays}, which spread data
across many places. An underlying \xcd`Dist` object provides the {\em
distribution}, telling which elements of the \xcd`DistArray` go in which
place. \xcd`Dist` uses subsidiary \xcd`Region` objects to abstract over the
shape and even the dimensionality of arrays.
Specialized X10 control statements such as \xcd`ateach` provide efficient
parallel iteration over distributed arrays.


\section{Annotations}

\Xten{} supports annotations on classes and interfaces, methods
and constructors,
variables, types, expressions and statements.
These annotations may be processed by compiler plugins.

\section{Translating MPI programs to \Xten{}}

While \Xten{} permits considerably greater flexibility in writing
distributed programs and data structures than MPI, it is instructive
to examine how to translate MPI programs to \Xten.

Each separate MPI process can be translated into an \Xten{}
place. Async activities may be used to read and write variables
located at different processes. A single clock may be used for barrier
synchronization between multiple MPI processes. \Xten{} collective
operations may be used to implement MPI collective operations.
\Xten{} is more general than MPI in (a)~not requiring synchronization
between two processes in order to enable one to read and write the
other's values, (b)~permitting the use of high-level atomic blocks
within a process to obtain mutual exclusion between multiple
activities running in the same node (c)~permitting the use of multiple
clocks to combine the expression of different physics (e.g.,
computations modeling blood coagulation together with computations
involving the flow of blood), (d)~not requiring an SPMD style of
computation.


%\note{Relaxed exception model}
\section{Summary and future work}
\subsection{Design for scalability}
\Xten{} is designed for scalability, by encouraging working with local data,
and limiting the ability of events at one place to delay those at another. For
example, an activity may atomically access only multiple locations in the
current place. Unconditional atomic blocks are dynamically guaranteed to be
non-blocking, and may be implemented using non-blocking techniques that avoid
mutual exclusion bottlenecks. 
%TODO: yoav says: ``no idea what [the following] means''
Data-flow synchronization permits point-to-point
coordination between reader/writer activities, obviating the need for
barrier-based or lock-based synchronization in many cases.

\subsection{Design for productivity}
\Xten{} is designed for productivity.

\paragraph{Safety and correctness.}

\bard{Confirm some of these claims}

Programs written in \Xten{} are guaranteed to be statically
\emph{type safe}, \emph{memory safe} and \emph{pointer safe}. 

Static type safety guarantees that every location contains only values whose
dynamic type agrees with the location's static type. The compiler allows a
choice of how to handle method calls. In strict mode, method calls are
statically checked to be permitted by the static types of operands. In lax
mode, dynamic checks are inserted when calls may or may not be correct,
providing weaker static correctness guarantees but more programming
convenience. 

Memory safety guarantees that an object may only access memory within its
representation, and other objects it has a reference to. \Xten{} does not
permit 
pointer arithmetic, and bound-checks array accesses dynamically if necessary.
\Xten{} uses garbage collection to collect objects no longer referenced by any
activity. \Xten{} guarantees that no object can retain a reference to an
object whose memory has been reclaimed. Further, \Xten{} guarantees that every
location is initialized at run time before it is read, and every value read
from a word of memory has previously been written into that word.

%XXX
%Pointer safety guarantees that a null pointer exception cannot be
%thrown by an operation on a value of a non-nullable type.

Because places are reflected in the type system, static type safety
also implies \emph{place safety}. All operations that need to be performed
locally are, in fact, performed locally.  All data which is declared to be
stored locally are, in fact, stored locally.

\Xten{} programs that use only clocks and unconditional atomic
blocks are guaranteed not to deadlock. Unconditional atomic blocks
are non-blocking, hence cannot introduce deadlocks.
Many concurrent programs can be shown to be determinate (hence
race-free) statically.

\paragraph{Integration.}
A key issue for any new programming language is how well it can be
integrated with existing (external) languages, system environments,
libraries and tools.

%TODO: Yoav asks ``you mean interop''?
We believe that \Xten{}, like \java{}, will be able to support a large
number of libraries and tools. An area where we expect future versions
of \Xten{} to improve on \java{} like languages is \emph{native
integration} (\Sref{NativeCode}). Specifically, \Xten{} will permit
permit multi-dimensional local arrays to be operated on natively by
native code.

\subsection{Conclusion}
{}\Xten{} is considerably higher-level than thread-based languages in
that it supports dynamically spawning lightweight activities, the
use of atomic operations for mutual exclusion, and the use of clocks
for repeated quiescence detection.

Yet it is much more concrete than languages like HPF in that it forces
the programmer to explicitly deal with distribution of data
objects. In this the language reflects the designers' belief that
issues of locality and distribution cannot be hidden from the
programmer of high-performance code in high-end computing.  A
performance model that distinguishes between computation and
communication must be made explicit and transparent.\footnote{In this
\Xten{} is similar to more modern languages such as ZPL \cite{zpl}.}
At the same time we believe that the place-based type system and
support for generic programming will allow the \Xten{} programmer to
be highly productive; many of the tedious details of
distribution-specific code can be handled in a generic fashion.

\chapter{Lexical structure}


Lexically a program consists of a stream of white space, comments,
identifiers, keywords, literals, separators and operators, all of them
composed of ASCII characters. 

\paragraph{Whitespace}
\index{white space}
% Whitespace \index{whitespace} follows \java{} rules \cite[Chapter 3.6]{jls2}.
ASCII space, horizontal tab (HT), form feed (FF) and line
terminators constitute white space.

\paragraph{Comments}
\index{comment}
% Comments \index{comments} follows \java{} rules
% \cite[Chapter 3.7]{jls2}. 
All text included within the ASCII characters ``\xcd"/*"'' and
``\xcd"*/"'' is
considered a comment and ignored; nested comments are not
allowed.  All text from the ASCII characters
``\xcd"//"'' to the end of line is considered a comment and is ignored.

\paragraph{Identifiers}
\index{identifier}
\index{variable name}

Identifiers consist of a single letter followed by zero or more
letters or digits.
The letters are the ASCII characters \xcd`a` through \xcd`z`, \xcd`A` through
\xcd`Z`, and \xcd`_`.
Digits are defined as the ASCII characters \xcd"0" through \xcd"9". Case is
significant; \xcd`a` and \xcd`A` are distinct identifiers, \xcd`as` is a
keyword, but \xcd`As` and \xcd`AS` are identifiers.

\paragraph{Keywords}
\index{keywords}
\Xten{} reserves the following keywords:
\begin{xten}
abstract       false          offers         transient      
as             final          operator       true           
assert         finally        package        try            
async          finish         private        var            
ateach         for            property       when           
break          goto           protected      while          
case           if             public         at             
catch          implements     return         atomic         
class          import         self           await          
continue       in             static         clocked        
def            instanceof     struct         here           
default        interface      super          next           
do             native         switch         offer          
else           new            this           resume         
extends        null           throw          type           
\end{xten}
Note that the primitive types are not considered keywords.

\paragraph{Literals}\label{Literals}\index{literal}

Briefly, \XtenCurrVer{} uses fairly standard syntax for its literals:
integers, unsigned integers, floating point numbers, booleans, 
characters, strings, and \xcd"null".  The most exotic points are (1) unsigned
numbers are marked by a \xcd`u` and cannot have a sign; (2) \xcd`true` and
\xcd`false` are the literals for the booleans; and (3) floating point numbers
are \xcd`Double` unless marked with an \xcd`f` for \xcd`Float`. 

Less briefly, we use the following abbreviations: 
\begin{displaymath}
\begin{array}{rcll}
d &=& \mbox{one or more decimal digits}\\
d_8 &=& \mbox{one or more octal digits}\\
d_{16} &=& \mbox{one or more hexadecimal digits, using \xcd`a`-\xcd`f`
for 10-15}\\
i &=& d 
        \mathbin{|} {\tt 0} d_8 
        \mathbin{|} {\tt 0x} d_{16}
        \mathbin{|} {\tt 0X} d_{16}
\\
s &=& \mbox{optional \xcd`+` or \xcd`-`}\\
b &=& d 
          \mathbin{|} d {\tt .}
          \mathbin{|} d {\tt .} d
          \mathbin{|}  {\tt .} d \\
x &=& ({\tt e } \mathbin{|} {\tt E})
         s
         d \\
f &=& b x
\end{array}
\end{displaymath}

\begin{itemize}

\item \xcd`true` and \xcd`false` are the \xcd`Boolean` literals. \index{Boolean!literal}\index{literal!Boolean}

\item \xcd`null` is a literal for the null value.  It has type
      \xcd`Any{self==null}`. \index{null} \index{object!literal}

\item \index{Int!literal}\index{literal!integer}
\xcd`Int` literals have the form {$si$}; \eg, \xcd`123`,
      \xcd`-321` are decimal \xcd`Int`s, \xcd`0123` and \xcd`-0321` are octal
      \xcd`Int`s, and \xcd`0x123`, \xcd`-0X321`,  \xcd`0xBED`, and \xcd`0XEBEC` are
      hexadecimal \xcd`Int`s.  

\item \xcd`Long` literals have the form {$si{\tt l}$} or
      {$si{\tt L}$}. \Eg, \xcd`1234567890L`  and \xcd`0xBAGEL` are \xcd`Long` literals. 

\item \xcd`UInt` literals have the form {$i{\tt u}$} or {$i {\tt U}$}.
      \Eg, \xcd`123u`, \xcd`0123u`, and \xcd`0xBEAU` are \xcd`UInt` literals.

\item \xcd`ULong` literals have the form {$i {\tt ul}$} or {$i {\tt
      lu}$}, or capital versions of those.  For example, 
      \xcd`123ul`, \xcd`0124567012ul`,  \xcd`0xFLU`, \xcd`OXba1eful`, and \xcd`0xDecafC0ffeefUL` are \xcd`ULong`
      literals. 

\item \xcd`Short` literals have the form {$si{\tt s}$} or
      {$si{\tt S}$}. \Eg,  414S, \xcd`OxACES` and \xcd`7001s` are short
      literals. 

\item \xcd`UShort` literals  form {$i {\tt us}$} or {$i {\tt
      su}$}, or capital versions of those.  For example, \xcd`609US`, 
      \xcd`107us`, and \xcd`OxBeaus` are unsigned short literals.

\item \xcd`Byte` literals have the form  {$si{\tt y}$} or
      {$si{\tt Y}$}.  (The letter \xcd`B` cannot be used for bytes, as it is
      a hexadecimal digit.)  \xcd`50Y` and \xcd`OxBABY` are byte literals.

\item \xcd`UByte` literals have the form {$i {\tt uy}$} or {$i {\tt yu}$}, or
      capitalized versions of those.  For example, \xcd`9uy` and \xcd`OxBUY`
      are \xcd`UByte` literals.
      


\item \xcd`Float` literals have the form {$s f {\tt f}$} or  {$s
\index{float!literal}
\index{literal!float}
      f {\tt F}$}.  Note that the floating-point marker letter \xcd`f` is
      required: unmarked floating-point-looking literals are \xcd`Double`. 
      \Eg, \xcd`1f`, \xcd`6.023E+32f`, \xcd`6.626068E-34F` are \xcd`Float`
      literals. 

\item \xcd`Double` literals have the form {$s f$}\footnote{Except that
\index{double!literal}
\index{literal!double}
      literals like \xcd`1` 
      which match both {$i$} and {$f$} are counted as
      integers, not \xcd`Double`; \xcd`Double`s require a decimal
      point, an exponent, or the \xcd`d` marker.
      }, {$s f {\tt
      D}$}, and {$s f {\tt d}$}.  
      \Eg, \xcd`0.0`, \xcd`0e100`, \xcd`1.3D`,  \xcd`229792458d`, and \xcd`314159265e-8`
      are \xcd`Double` literals.

\item 
\index{char!literal}
\index{literal!char}
\xcd`Char` literals have one of the following forms: 
      \begin{itemize}
      \item \xcd`'`{\it c}\xcd`'` where {\em c} is any printing ASCII
            character other than 
            \xcd`\` or \xcd`'`, representing the character {\em c} itself; 
            \eg, \xcd`'!'`;
      \item \xcd`'\b'`, representing backspace;
      \item \xcd`'\t'`, representing tab;
      \item \xcd`'\n'`, representing newline;
      \item \xcd`'\f'`, representing form feed;
      \item \xcd`'\r'`, representing return;
      \item \xcd`'\''`, representing single-quote;
      \item \xcd`'\"'`, representing double-quote;
      \item \xcd`'\\'`, representing backslash;
      \item \xcd`'\`{\em dd}\xcd`'`, where {\em dd} is one or more octal
            digits, representing the one-byte character numbered {\em dd}; it
            is an error if {\em dd}{$>0377$}.      
      \end{itemize}

\item
\index{string!literal} 
\index{literal!string}
\xcd`String` literals consist of a double-quote \xcd`"`, followed by
      zero or more of the contents of a \xcd`Char` literal, followed by
      another double quote.  \Eg, \xcd`"hi!"`, \xcd`""`.


\end{itemize}



\paragraph{Separators}
\Xten{} has the following separators and delimiters:
\begin{xten}
( )  { }  [ ]  ;  ,  .
\end{xten}

\paragraph{Operators}
\index{operator}
\Xten{} has the following operator,  type constructor, and miscellaneous symbols.  (\xcd`?` and
\xcd`:` comprise a single ternary operator, but are written separately.)
\begin{xten}
==  !=  <   >   <=  >=
&&  ||  &   |   ^
<<  >>  >>>
+   -   *   /   %
++  --  !   ~
&=  |=  ^=
<<= >>= >>>=
+=  -=  *=  /=  %=
=   ?   :  =>  ->
<:  :>  @   ..
\end{xten}





\chapter{Types}
\label{XtenTypes}\index{types}

{}\Xten{} is a {\em strongly typed} object-oriented language: every
variable and expression has a type that is known at compile-time.
Types limit the values that variables can hold.

{}\Xten{} supports three kinds of runtime entities, {\em objects},
{\em structs}, and {\em functions}. Objects are instances of {\em
  classes} (\Sref{ReferenceClasses}). They may contain zero or
more mutable fields, and a reference to the list of methods defined on them.

An object is represented by some (contiguous) memory chunk on the
heap. Entities (such as variables and fields) contain a {\em
  reference} to this chunk. That is, objects are represented through
an extra level of indirection.  A consequence of this flexibility is
that an entity containing a reference to an object \xcd{o} needs only
one word of memory to represent that reference, regardess of the
number of fields in \xcd{o}. An assignment to this entity simply
overwrites the reference with another reference (thus taking constant
time). Another consequence is that every class type contains the value
\Xcd{null} corresponding to the invalid reference. \Xcd{null} is often
useful as a default value. Further, two objects may be compared for
identity (\Xcd{==}) in constant time by simply comparing references to
the memory used to represent the objects. The default hash code for an
object is based on the value of this reference. A downside of this
flexibility is that the operations of accessing a field and invoking a
method are more expensive than simply reading a register and
invoking a static function.


Structs are instances of {\em struct types} (\Sref{StructClasses}).  A
struct is represented without the extra level of indirection, with a
memory chunk of size $N$ words precisely big enough to store the value
of every field of the struct (modulo alignment), plus whatever padding is needed. Thus structs cannot
be shared. Entities (such as variables and fields) refering to the
struct must allocate $N$ words to directly contain the chunk.  An
assignment to this entity must copy the $N$ words representing the
right hand side into the left hand side. Since there are no references
to structs, \Xcd{null} is not a legal value for a struct
type. Comparison for identity (\Xcd{==}) involves examining $N$
words. Additionally, structs do not have any mutable fields, hence
they can be freely copied. The payoff for these restrictions lies in
that fields can be stored in registers or local variables, and 
and method invocation is implemented by invoking a static function.

Functions, called closures, lambda-expressions, and blocks in other languages, are
instances of {\em function types} (\Sref{Functions}). 
A function has zero or more {\em formal parameters} (or {\em arguments}) and a
{\em body}, which is 
an expression that can reference the formal parameters and also other
variables in the surrounding block. For instance, \xcd`(x:Int)=>x*y`
is a unary integer function which multiplies its argument by the
variable \xcd`y` from the surrounding block.  Functions may be freely
copied from place to place and may be repeatedly applied. 

These runtime entities are classified by {\em types}. Types are used in
variable declarations, coercions and  explicit conversions, object creation,
array creation, static state and method accessors, and
\xcd"instanceof" and \xcd`as` expressions.


The basic relationship between values and types is the {\em is an
element of} relation.  We also often say ``$e$ has type $T$'' to
mean ``$e$ is an element of type $T$''.  For example, \xcd`1` has type
\xcd`Int` (the type of all integers representible in 32 bits). It also
has type \xcd`Any` (since all entitites have type \xcd`Any`), type
\xcd`Int{self != 0}` (the type of nonzero integers), type
\xcd`Int{self == 1}` (the type of integers which are equal to \xcd`1`, which
contains only one element), and many others. 

The basic relationship between types is {\em subtyping}: \xcd`T <: U`
holds if every instance of \xcd`T` is also an instance of \xcd`U`. Two
important kinds of subtyping are {\em subclassing} and {\em
  strengthening}. Subclassing is a familiar notion from
object-oriented programming. Here we use it to refer to the
relationship between a class and another class it extends, and the
relationship between a class and another interface it implements. For
instance, in a class hierarchy with classes \xcd`Animal` and \xcd`Cat`
such that \xcd`Cat` extends \xcd`Mammal` and \xcd`Mammal` extends
\xcd`Animal`, every instance of \xcd`Cat` is by definition an instance
of \xcd`Animal` (and \xcd`Mammal`). We say that \xcd`Cat` is a
subclass of \xcd`Animal`, or \xcd`Cat <: Animal` by subclassing. If
\xcd`Animal` implements \xcd`Thing`, then \xcd`Cat` also implements
\xcd`Thing`, and we say \xcd`Cat <: Thing` by subclassing.
Strengthening is an equally familiar notion from logic.  The instances
of \xcd`Int{self == 1}` are all elements of \xcd`Int{self != 0}` as well,
because \xcd`self == 1` logically implies \xcd`self != 0`; so 
\xcd`Int{self  == 1} <: Int{self !=0}` by strengthening.  X10 uses both notions
of subtyping.  See \Sref{DepType:Equivalence} for the full definition
of subtyping in X10.

\subsection{Type System}
\index{type system}
The types in X10 are as follows.  

These are the {\em elementary} types. Other
syntactic forms for types exist, but they are simply abbreviations for types
in the following system.  For example, \xcd`Array[Int](1)` is the type of
one-dimensional integer-valued arrays; it is an abbreviation for
\xcd`Array[Int]{rank==1}`.\\

% remove \refstepcounter{equation}
% snag the argument of \label{X}
% change the (\arabic{equation}) into (\ref{X})

%##(Type FunctionType ConstrainedType
\begin{bbgrammar}
%(FROM #(prod:Type)#)
                Type \: FunctionType & (\ref{prod:Type}) \\
                     \| ConstrainedType \\
                     \| VoidType \\
%(FROM #(prod:FunctionType)#)
        FunctionType \: TypeParams\opt \xcd"(" FormalList\opt \xcd")" Guard\opt Offers\opt \xcd"=>" Type & (\ref{prod:FunctionType}) \\
%(FROM #(prod:ConstrainedType)#)
     ConstrainedType \: NamedType & (\ref{prod:ConstrainedType}) \\
                     \| AnnotatedType \\
                     \| \xcd"(" Type \xcd")" \\
\end{bbgrammar}
%##)


Types may be given by name. 
For example, 
%~~type~~`~~`~~ ~~ ^^^ Types10
\xcd`Int`
is the type of 32-bit integers.
Given a class declaration 
%~~gen ^^^ Types20
%package Types.Core.TypeName; 
%~~vis
\begin{xten}
class Triple { /* ... */ }
\end{xten}
%~~siv
%
%~~neg
the identifier \xcd`Triple` may be used as a type.

The type {\em TypeName \xcd`[` Types{$^?$} \xcd`]`} is an instance of
a {\em generic} (or {\em parameterized}) type. 
 For example,
\xcd`Array[Int]` is the type of arrays of integers. 
\xcd`HashMap[String,Int]` is the type of hash maps from strings to
integers.

The type {\em Type \xcd`{` Constraint \xcd`}`} refers to a constrained type.
{\em Constraint} is a Boolean expression -- written in a {\em very} limited
subset of X10 -- describing the acceptable values of the constrained type.
%~~stmt~~`~~`~~ ~~ ^^^ Types30
For example, \xcd`var n : Int{self != 0};` guarantees that \xcd`n` is always a
non-zero integer. 
%~~stmt~~`~~`~~ ~~class Triple{} ^^^ Types40
Similarly, \xcd`var x : Triple{x != null};` defines a \xcd`Triple`-valued
variable \xcd`x` whose value is never null.

The qualified type {\em Type \xcd`.` Type} refers to an instance of a {\em
nested} type; that is, a class or struct defined inside of another class or
struct, and holding an implicit reference to the outer.  For example, given
the type declaration 
%~~gen ^^^ Types50
% package Types.Core.Hardcore.Qualified;
%~~vis
\begin{xten}
class Outer {
  class Inner { /* ... */ }
}
\end{xten}
%~~siv
%
%~~neg
then 
%~~exp~~`~~`~~ ~~ NOTEST class Outer {class Inner { /* ... */ }} ^^^ Types60
\xcd`(new Outer()).new Inner()` creates a value of type 
%~~type~~`~~`~~ ~~class Outer {class Inner { /* ... */ }} ^^^ Types70
\xcd`Outer.Inner`.

Type variables, {\em TypeVar}, refer to types that are parameters.  For
example, the following class defines a cell in a linked list.  
%~~gen ^^^ Types80
% package Types.Core.Bore.Lore;
%~~vis
\begin{xten}
class LinkedList[X] {
  val head : X;
  val tail : LinkedList[X];
  def this(head:X, tail:LinkedList[X]) {
     this.head = head; this.tail = tail;
  }
}
\end{xten}
%~~siv
%
%~~neg
It doesn't
matter what type the cell's element is, but it has to have {\em some} type.
\xcd`LinkedList[Int]` is a linked list of integers.
\xcd`LinkedList[LinkedList[Byte]]` is a list of lists of bytes.
Note that \xcd`LinkedList` is {\em not} a usable type -- it is missing a type parameter.



The function type 
{\em \xcd`(` Formals{$^?$} \xcd`) =>`  Type} 
refers to functions taking the
listed formal parameters and returning a result of {\em Type}.  In
\XtenCurrVer, function types may not be generic.
The closely-related void function type 
{\em \xcd`(` Formals{$^?$} \xcd`) =>`  \xcd`void`}  takes the listed
parameters and returns no value.
For example, 
\begin{xtenmath}
\xcd`(x:Int) => Int{self != x}` 
\end{xtenmath}
is the type of integer-valued functions which have no fixed points -- that is,
for which the output is an integer different from the input.
An example of such a function is \xcd`(x:Int) => x+1`.
For fundamental reasons, X10 --- or any other computer program --- cannot
tell in general whether a function has any fixed points or not.  So, X10
programs using such types must prove to X10 that they are correct. Often this
will involve a run-time check, expressed as a cast, such as: 
%~~gen ^^^ Types3x7m
% package Types3x7m;
% class Example {
%~~vis
\begin{xten}
  val plus1 : (x:Int) => Int{self != x} = 
     (x:Int) => (x+1) as Int{self != x}; 
\end{xten}
%~~siv
%}
%~~neg


The names of the formal parameters are bound in the type, and may be changed
consistently in the usual way without modifying the type.  
For example, \\
\xcd`(a:Int, b:Int{self!=a})=>Int{self!=a, self!=b}` 
and \\
\xcd`(c:Int, d:Int{self!=c})=>Int{self!=c, self!=d}` \\
are equivalent types.  



\section{Classes, Structs,  and interfaces}
\label{ReferenceTypes}

\subsection{Class types}

\index{type!class}
\index{class}
\index{class declaration}
\index{declaration!class declaration}
\index{declaration!reference class declaration}

A {\em class declaration} (\Sref{XtenClasses}) declares a {\em class type},
giving its name, behavior, data, and relationships to other classes and
interfaces. 

\begin{ex}
The \xcd`Position` class below could describe the position of a slider
control: 
%~~gen ^^^ Types100
% package Types.By.Cripes.Classes;
%~~vis
\begin{xten}
class Position {
  private var x : Int = 0;
  public def move(dx:Int) { x += dx; }
  public def pos() : Int = x;
}
\end{xten}
%~~siv
%
%~~neg
\end{ex}
Class instances, also called objects, are created by constructor calls, 
such as \xcd`new Position()`
Class
instances have fields and methods, type members, and value properties bound at
construction time. In addition, classes have static members: static \xcd`val` fields,
methods, type definitions, and member classes and member interfaces.

Classes may be {\em generic}, \ie, defined with one or more type
parameters (\Sref{TypeParameters}).  

%~~gen ^^^ Types110
%~~vis
\begin{xten}
class Cell[T] {
  var contents : T;
  public def this(t:T) { contents = t;  }
  public def putIn(t:T) { contents = t; }
  public def get() = contents;
  }
\end{xten}
%~~siv
%~~neg


%TODO: Yoav: ``This reasoning is no longer true in the new object model''
%% Why not?
\Xten{} does not permit mutable static state. A fundamental principle of the
X10 model of computation is that all mutable state be local to some place
(\Sref{XtenPlaces}), and, as static variables are
globally available, they
cannot be mutable. When mutable global state is necessary, programmers should
use singleton classes, putting the state in an object and using place-shifting
commands (\Sref{AtStatement}) and atomicity (\Sref{AtomicBlocks}) as necessary
to mutate it safely.

\index{\Xcd{Object}}
\index{\Xcd{x10.lang.Object}}

Classes are structured in a single-inheritance hierarchy. All classes extend
the class \xcd"x10.lang.Object", directly or indirectly. Each class other than
\xcd`Object` extends a single parent class.  \xcd`Object` provides no behaviors
of its own, beyond those required by \xcd`Any`.

\index{class!reference class}
\index{reference class type}
\index{\Xcd{Object}}
\index{\Xcd{x10.lang.Object}}


\index{null}


The null value, represented by the literal
\xcd"null", is a value of every class type \xcd`C`. The type whose values are
all instances of \xcd`C` except 
\xcd`null` can be defined as \xcd`C{self != null}`.

\subsection{Struct Types}

A {\em struct declaration} \Sref{XtenStructs} introduces a {\em struct type}
containing all instances of the struct.  The \xcd`Coords` struct below gives
an immutable position in 3-space: 
%~~gen ^^^ Types120
% package Types.Structs.Coords;
%~~vis
\begin{xten}
struct Position {
  public val x:Double, y:Double, z:Double; 
  def this(x:Double, y:Double, z:Double) {
     this.x = x; this.y = y; this.z = z;
  }
}
\end{xten}
%~~siv
%
%~~neg

Structs have many capabilities of classes: they can have methods, implement
interfaces, and be generic. However, they have certain restrictions; for
example, they cannot contain mutable (\xcd`var`) fields, or inherit from
superclasses. There is no \xcd`null` value for structs. Due to these
restrictions, structs may be implemented more efficiently than objects.


\subsection{Interface types}
\label{InterfaceTypes}

\index{type!interface}
\index{interface}
\index{interface declaration}
\index{declaration!interface declaration}

An {\em interface declaration} (\Sref{XtenInterfaces}) defines an {\em
interface type}, specifying a set of methods 
%type members, 
and properties which must be provided by any class declared to implement the
interface. 


Interfaces can also have static members: static fields, type
definitions, and member classes, structs and interfaces.  However,
interfaces cannot specify that implementing classes must provide
static members or constructors.

\begin{ex}
In the following interface, \xcd`PI` is a static field, 
\xcd`Vec` a static type definition, 
\xcd`Pair` a static member class.
It can't insist that implementations provide a static method 
like \xcd`meth`, or a nullary constructor.
%~~gen ^^^ Types2y3i
% NOTEST
% package Types2y3i;
%~~vis
\begin{xten}
interface Stat {
  static val PI = 3.14159; 
  static type R = Double;
  static class Pair(x:R, y:R) {}
  // ERROR: static def meth():Int;
  // ERROR: static def this();
}
class Example {
  static def example() {
     val p : Stat.Pair = new Stat.Pair(Stat.PI, Stat.PI);
  }
}
\end{xten}
%~~siv
%
%~~neg

\end{ex}

An interface may extend multiple interfaces.  
%~~gen ^^^ Types130
%package Types.For.Snipes.Interfaces;
%~~vis
\begin{xten}
interface Named {
  def name():String;
}
interface Mobile {
  def move(howFar:Int):void;
}
interface Person extends Named, Mobile {}
interface NamedPoint extends Named, Mobile {} 
\end{xten}
%~~siv
%
%~~neg


Classes and structs may be declared to implement multiple interfaces. Semantically, the
interface type is the set of all objects that are instances of classes
or structs that
implement the interface. A class or struct implements an interface if it is declared to
and if it concretely or abstractly implements all the methods and properties
defined in the interface. For example, \xcd`Kim` implements
\xcd`Person`, and hence \xcd`Named` and \xcd`Mobile`. It would be a static
error if \xcd`Kim` had no \xcd`name` method, unless \xcd`Kim` were also
declared \xcd`abstract`.

%~~gen ^^^ Types140
%interface Named {
%   def name():String;
% }
% interface Mobile {
%   def move(howFar:Int):void;
% }
% interface Person extends Named, Mobile {}
% interface NamedPoint extends Named, Mobile{} 
%~~vis
\begin{xten}
class Kim implements Person {
   var pos : Int = 0;
   public def name() = "Kim (" + pos + ")";
   public def move(dPos:Int) { pos += dPos; }
}
\end{xten}
%~~siv
%
%~~neg


\subsection{Properties}
\index{property}
\label{properties}

Classes, interfaces, and structs may have {\em properties}, specified in
parentheses after the type name. Properties are much like public \xcd`val`
instance fields. They have certain restrictions on their use, however, which
allows the compiler to understand them much better than other public \xcd`val`
fields. In particular, they can be used in types.  \Eg, the number of elements
in an array is a property of the array, and an X10 program can specify that
two arrays have the same number of elements.

\begin{ex}
The
following code declares a class named \xcd"Coords" with properties
\xcd"x" and \xcd"y" and a \xcd"move" method. The properties are bound
using the \xcd"property" statement in the constructor.

%~~gen ^^^ Types150
%package not.x10.lang;
%~~vis
\begin{xten}
class Coords(x: Int, y: Int) { 
  def this(x: Int, y: Int) :
    Coords{self.x==x, self.y==y} = { 
    property(x, y); 
  } 

  def move(dx: Int, dy: Int) = new Coords(x+dx, y+dy); 
}
\end{xten}
%~~siv
%~~neg
\end{ex}
Properties, unlike other public \xcd`val` fields, can be used  
at compile time in {constraints}. This allows us
to specify subtypes based on properties, by appending a boolean expression to
the type. For example, the type \xcd"Coords{x==0}" is the set of all points
whose \xcd"x" property is \xcd"0".  Details of this substantial topic are
found in \Sref{ConstrainedTypes}.



\section{Type Parameters and Generic Types}
\label{TypeParameters}

\index{type!parameter}
\index{method!parametrized}
\index{constructor!parametrized}
\index{closure!parametrized}
\label{Generics}
\index{type!generic}

A class, interface, method, or type definition  may have type
parameters.  Type parameters can be used as types, and will be bound to types
on instantiation.  
For example, a generic stack class may be defined as 
\xcd`Stack[T]{...}`.  Stacks can hold values of any type; \eg, 
%~~type~~`~~`~~ ~~class Stack[T]{} ^^^ Types160
\xcd`Stack[Int]` is a stack of integers, and 
%~~type~~`~~`~~ ~~class Stack[T]{} ^^^ Types170
\xcd`Stack[Point {self!=null}]` is a stack of non-null \xcd`Point`s.
Generics {\em must} be instantiated when they are used: \xcd`Stack`, by
itself, is not a valid type.
Type parameters may be constrained by a guard on the declaration
(\Sref{TypeDefGuard},
\Sref{MethodGuard},\Sref{ClosureGuard}).

\index{type!concrete}
\index{concrete type}
A {\em generic class} (or struct, interface, or type definition) 
is a class (resp. struct, interface, or type definition) 
declared with $k \geq 1$ type parameters. 
A generic class (or struct, interface, or type definition) 
can be used to form a type by supplying $k$ types as type arguments within
\xcd`[` \ldots \xcd`]`.
%%When instantiated,
%%with concrete (\viz, non-generic) types for its parameters, 
%%a generic type becomes a concrete type and can be
%%used like any other type. 
For example,
\xcd`Stack` is a generic class, 
%~~type~~`~~`~~ ~~class Stack[T]{} ^^^ Types180
\xcd`Stack[Int]` is a type, and can be used as one: 
%~~stmt~~`~~`~~ ~~class Stack[T]{} ^^^ Types190
\xcd`var stack : Stack[Int];`

\begin{ex}A \xcd`Cell[T]` is a generic object, capable of holding a value of type
\xcd`T`.  For example, a \xcd`Cell[Int]` can hold an \xcd`Int`, and a
\xcd`Cell[Cell[Int{self!=0}]]` can hold a \xcd`Cell` which in turn can
only hold non-zero numbers. 
%% vj: Dont know what this saying: bound immutably... but mutable?
%% \xcd`Cell`s are actually useful in situations
%%where values must be bound immutably for one reason, but need to be mutable.
%~~gen ^^^ Types200
% package ch4;
%~~vis
\begin{xten}
class Cell[T] {
    var x: T;
    def this(x: T) { this.x = x; }
    def get(): T = x;
    def set(x: T) = { this.x = x; }
}
\end{xten}
%~~siv
%~~neg


\xcd"Cell[Int]" is the type of \xcd`Int`-holding cells.  
The \xcd"get" method on a \xcd`Cell[Int]` returns an \xcd"Int"; the
\xcd"set" method takes an \xcd"Int" as argument.  Note that
\xcd"Cell" alone is not a legal type because the parameter is
not bound.
\end{ex}

A class (whether generic or not) may have generic methods.

\begin{ex}
\xcd`NonGeneric` has a generic method 
\xcd`first[T](x:List[T])`. An invocation of such a method may supply
the type parameters explicitly (\eg, \xcd`first[Int](z)`).
 In certain cases (\eg, \xcd`first(z)`)
type parameters may
be omitted and are inferred by the compiler (\Sref{TypeInference}).

%~~gen ^^^ Types210
% package Types.For.Cripes.Sake.Generic.Methods;
% import x10.util.*;
%~~vis
\begin{xten}
class NonGeneric {
  static def first[T](x:List[T]):T = x(0);
  def m(z:List[Int]) {
    val f = first[Int](z);
    val g = first(z);
    return f == g;
  }
}
\end{xten}
%~~siv
%
%~~neg


\end{ex}


\limitation{ \XtenCurrVer{}'s C++ back end requires generic methods to be
static or final; the Java back end can accomodate generic instance methods as well. }

Unlike other kinds of variables, type parameters may {\em not} be shadowed.  
If name \xcd`X` is in scope as a type, \xcd`X` may not be rebound as a type
variable.  

\begin{ex}
Neither \xcd`class B` nor \xcd`class C[B]` are allowed in the
following code, because they both shadow the type variable \xcd`B`.
%~~gen ^^^ TypesNoShadow
% package TypesNoShadow;
% KNOWNFAIL-https://jira.codehaus.org/browse/XTENLANG-2621
%~~vis
\begin{xten}
class A[B] {
  //ERROR: class B{} 
  //ERROR: class C[B]{} 
}
\end{xten}
%~~siv
%
%~~neg
\end{ex}

\subsection{Use of Generics}

An unconstrained type variable \Xcd{X} can be instantiated any type. Within a
generic struct or class, all the operations of \Xcd{Any} are available on a
variable of type unconstrained \Xcd{X}. Additionally, variables of type
\Xcd{X} may be used with \Xcd{==, !=}, in \Xcd{instanceof}, and casts.  

If a type variable is constrained, the operations implied by its constraint
are available as well.

\begin{ex}
The interface \xcd`Named` describes entities which know their own name.  The
class \xcd`NameMap[T]` is a specialized map which stores and retrieves
\xcd`Named` entities by name.  The call \xcd`t.name()` in \xcd`put()` is only
valid because the constraint \xcd`{T <: Named}` implies that \xcd`T` is a
subtype of \xcd`Named`, and hence provides all the operations of \xcd`Named`. 
%~~gen ^^^ Types6e6x
% package Types6e6x;
% import x10.util.*;
%~~vis
\begin{xten}
interface Named { def name():String; }
class NameMap[T]{T <: Named} {
   val m = new HashMap[String, T]();
   def put(t:T) { m.put(t.name(), t); }
   def get(s:String):T = m.getOrThrow(s);
}
\end{xten}
%~~siv
%
%~~neg


\end{ex}


%%NO-VARIANCE%% \subsection{Variance of Type Parameters}
%%NO-VARIANCE%% \index{covariant}
%%NO-VARIANCE%% \index{contravariant}
%%NO-VARIANCE%% \index{invariant}
%%NO-VARIANCE%% \index{type parameter!covariant}
%%NO-VARIANCE%% \index{type parameter!contravariant}
%%NO-VARIANCE%% \index{type parameter!invariant}
%%NO-VARIANCE%% 
%%NO-VARIANCE%% % Uncomment this when the language implementation properly supports variance.
%%NO-VARIANCE%% %%%\subsection{Variance of Type Parameters}
%%\index{covariant}
%%\index{contravariant}
%%\index{invariant}
%%\index{type parameter!covariant}
%%\index{type parameter!contravariant}
%%\index{type parameter!invariant}

%TODO - examples courtesy of Nate
% 
% class OutputStream[-A] {
%    def write(a: A) = /* implementation left as an exercise for the reader */
% }
% 
% Also:
% 
% interface Comparator[-A] {
%    def compare(A): Int;
% }
% 
% and:
% 
% class HashMap[-K,+V] { ... }
% 
% 

Type parameters of classes (though not of methods) can be {\em variant}.

Consider classes \xcd`Person :> Child`.  Every child is a person, but there
are people who are not children.  What is the relationship between
\xcd`Cell[Person]` and \xcd`Cell[Child]`?  

\subsubsection{Why Variance Is Necessary}

In this case, \xcd`Cell[Person]` and \xcd`Cell[Child]` should be unrelated.  
If we had \xcd`Cell[Person] :> Cell[Child]`, the following code would let us
assign a \xcd`old` (a \xcd`Person` but not a \xcd`Child`) to a
variable \xcd`young` of type \xcd`Child`, thereby breaking the type system: 
\begin{xten}
// INCORRECTLY assuming Cell[Person] :> Cell[Child]
val cc : Cell[Child] = new Cell[Child]();
val cp : Cell[Person] = cc; // legal upcast
cp.set(old);       // legal since old : Person
val young : Child = cc.get(); 
\end{xten}

Similarly, if \xcd`Cell[Person] <: Cell[Child]`: 
\begin{xten}
// INCORRECTLY assuming Cell[Person] <: Cell[Child]
val cp : Cell[Person] = new Cell[Person];
val cc : Cell[Child] = cp; // legal upcast
val cp.set(old); 
val young : Child = cc.get();
\end{xten}

So, there cannot be a subtyping relationship in either direction between the
two. And indeed, neither of these programs passes the X10 typechecker.


\subsubsection{Legitimate Variance}

The \xcd`Cell[Person]`-vs-\xcd`Cell[Child]` problems occur because it is
possible to both store and retrieve values from the same object. However,
entities with only one of the two capabilities {\em can} sensibly have some
subtyping relations. Furthermore, both sorts of entity are useful. An entity
which can store values but not retrieve them can nonetheless summarize them.
An object which can retrieve values but not store values can be constructed
with an initial value, providing a read-only cell.

So, X10 provides {\em variance} to support these options.  Type parameters
may be defined in one of three forms.  
\begin{enumerate}
\item {\em invariant}: Given a definition \xcd`class C[T]{...}`, \xcd`C[Person]` and
      \xcd`C[Child]` are unrelated classes; neither is a subclass of the
      other.
\item {\em covariant}: Given a definition \xcd`class C[+T]{...}` (the \xcd`+` indicates
      covariance), \xcd`C[Person] :> C[Child]`.  This is appropriate when
      \xcd`C` allows retrieving values but not setting them.
\item {\em contravariant}: Given a definition \xcd`class C[-T]{...}` (the \xcd`-` indicates
      contravariance), \xcd`C[Person] <: C[Child]`.  This is appropriate when
      \xcd`C` allows storing values but not retrieving them.
\end{enumerate}


The \xcd"T" parameter of \xcd"Cell" above is
invariant.  

A typical example of covariance is \xcd`Get`.  As the \xcd`example()` method
shows, a \xcd`Get[T]` must be constructed with its value, and will return that
value whenever desired.  \xcd`Get[T]` is only moderately useful as a class; it
is more useful as an interface for providing a limited (read) access to a more
powerful data structure.
%~~gen
% package Variance_gone;
%~~vis
\begin{xten}
class Get[+T] {
  val x: T;
  def this(x: T) { this.x = x; }
  def get(): T = x;
  static def example() {
     val g : Get[Int] = new Get[Int](31);
     val n : Int = g.get();
     x10.io.Console.OUT.print("It's " + n);
     x10.io.Console.OUT.print("It's still " + g.get());
  }
}
\end{xten}
%~~siv
%~~neg

There are few if any {\em classes} with contravariant type parameters.
(Covariant type parameters are only moderately more common.)  
However, it is frequently useful to have {\em interfaces} with contravariant
type parameters.  For example: 
%~~gen
% package Types_contravariance_a;
%~~vis
\begin{xten}
interface OutputStream[-T] {
   def write(T) : void;
}
interface ComparableTo[-T] {
   def compare(T) : Int;
}
\end{xten}
%~~siv
%
%~~neg
Clearly, \xcd`Int <: Any`. 
An \xcd`OutputStream[Int]` is only capable of writing \xcd`Int`s.  
An \xcd`OutputStream[Any]` is capable of writing anything.  In particular, it
can write \xcd`Int`s. Thus, an \xcd`OutputStream[Any]` can be used in place of
an \xcd`OutputStream[Int]`, and hence \xcd`OutputStream[Any] <: OutputStream[Int]`.
Similarly, a \xcd`ComparableTo[Int]` can be compared to an integer. A
\xcd`ComparableTo[Any]` can be compared to anything, and, in particular, to an
integer.  Thus \xcd`ComparableTo[Any] <: ComparableTo[Int]`.
So, both of these interfaces are contravariant.


Given types \xcd"S" and \xcd"T": 
\begin{itemize}
\item
If the parameter of \xcd"Get" is covariant, then
\xcd"Get[S]" is a subtype of \xcd"Get[T]" if
\xcd"S" is a {\em subtype} of \xcd"T".

\item
If the parameter of \xcd"Set" is contravariant, then
\xcd"OutputStream[S]" is a subtype of \xcd"OutputStream[T]" if
\xcd"S" is a {\em supertype} of \xcd"T".

\item
If the parameter of \xcd"Cell" is invariant, then
\xcd"Cell[S]" is a subtype of \xcd"Cell[T]" if
\xcd"S" is a {\em equal} to \xcd"T".
\end{itemize}


In order to make types marked as covariant and contravariant semantically
sound, X10 performs extra checks.  
A covariant type parameter is permitted to appear only in covariant type positions,
and a contravariant type parameter in contravariant positions. 
\begin{itemize}
\item The return type of a method is a covariant position.
\item The argument types of a method are contravariant positions.
\item Whether a type argument position of a generic class, interface or struct type \Xcd{C}
is covariant or contravariant is determined by the \Xcd{+} or \Xcd{-} annotation
at that position in the declaration of \Xcd{C}.
\end{itemize}


There are similar restrictions on use of covariant and contravariant variables.

\limitationx{} Full checking of covariance and contravariance is not yet
implemented.  Covariant and contravariant classes and structs should be used
with great caution.

%TODO: Yoav says ``There are other rules, not implemented or specified,
%involving fields, inheritance, etc.  There are several JIRAs on it. No idea
%what is the work around -- maybe just say ``limitation''?'''




%%NO-VARIANCE%% 
%%NO-VARIANCE%% Class, struct and interface definitions are permitted to specify a {\em
%%NO-VARIANCE%%   variance} 
%%NO-VARIANCE%% for each type parameter. 
%%NO-VARIANCE%% There are three variance specifications: 
%%NO-VARIANCE%% \xcd`+` indicates {\em co-variance},  \xcd`-` indicates {\em
%%NO-VARIANCE%%   contravariance} and the absence of  \xcd`+` and 
%%NO-VARIANCE%%  \xcd`-` indicates {\em invariance}. For a class (or struct or
%%NO-VARIANCE%%  interface) \xcd`S` specifying that a particular parameter position
%%NO-VARIANCE%%  (say, \xcd`i`) is covariant means that 
%%NO-VARIANCE%% if \xcd`Si <: Ti` then
%%NO-VARIANCE%% \xcdmath"S[S1,$\ldots$,Sn] <: S[S1,$\ldots$, Si-1,Ti,Si+1,$\ldots$ Sn]".
%%NO-VARIANCE%% Similarly, saying that position \xcd`i` is is contravariant means
%%NO-VARIANCE%% that 
%%NO-VARIANCE%% if \xcd`Si <: Ti` then
%%NO-VARIANCE%% \xcdmath"S[S1,$\ldots$, Si-1,Ti,Si+1,$\ldots$ Sn] <: S[S1,$\ldots$,Sn]". If the
%%NO-VARIANCE%% position is invariant, then no such relationship is asserted between
%%NO-VARIANCE%% \xcd`Si <: Ti` 
%%NO-VARIANCE%% and
%%NO-VARIANCE%% \xcdmath"S[S1,$\ldots$, Si-1,Ti,Si+1,$\ldots$ Sn]". The compiler must perform
%%NO-VARIANCE%% several checks on the body of the class (or struct or interface) to
%%NO-VARIANCE%% establish that type parameters with a variance are used in a manner
%%NO-VARIANCE%% that is consistent with their semantics.
%%NO-VARIANCE%% 
%%NO-VARIANCE%% \limitation{} The implementation of variance specifications  suffers from
%%NO-VARIANCE%% various limitations in \XtenCurrVer. Users are strongly encouraged not
%%NO-VARIANCE%% to use variance. (Some classes, structs, and interfaces in the standard
%%NO-VARIANCE%% libraries use variance specifications in a careful way that
%%NO-VARIANCE%% circumvents these limitations.)
%%NO-VARIANCE%% 

\section{Type definitions}
\label{TypeDefs}

\index{type!definitions}
\index{declaration!type}
A type definition can be thought of as a type-valued function,
mapping type parameters and value parameters to a concrete type.

%##(TypeDefDecl TypeParams Formals Guard
\begin{bbgrammar}
%(FROM #(prod:TypeDefDecl)#)
         TypeDefDecl \: Mods\opt \xcd"type" Id TypeParams\opt Guard\opt \xcd"=" Type \xcd";" & (\ref{prod:TypeDefDecl}) \\
                     \| Mods\opt \xcd"type" Id TypeParams\opt \xcd"(" FormalList \xcd")" Guard\opt \xcd"=" Type \xcd";" \\
%(FROM #(prod:TypeParams)#)
          TypeParams \: \xcd"[" TypeParamList \xcd"]" & (\ref{prod:TypeParams}) \\
%(FROM #(prod:Formals)#)
             Formals \: \xcd"(" FormalList\opt \xcd")" & (\ref{prod:Formals}) \\
%(FROM #(prod:Guard)#)
               Guard \: DepParams & (\ref{prod:Guard}) \\
\end{bbgrammar}
%##)

\noindent 
During type-checking the compiler replaces the use of such a defined
type with its body, substituting the actual type and value parameters
in the call for the formals. This replacement is performed recursively
until the type no longer contains a defined type or a predetermined
compiler limit is reached (in which case the compiler declares an
error). Thus, recursive type definitions are not permitted.

Thus type definitions are considered applicative and not generative --
they do not define new types, only aliases for existing types.

\label{TypeDefGuard}
Type definitions may have guards: an invocation of a type definition
is illegal unless the guard is satisified when formal types and values
are replaced by the actual parameters.

Type definitions may be overloaded: two type definitions with
the same name are permitted provided that they have a different number
of type parameters or different number or type of value parameters.  The rules
for type definition resolution are identical to those for method resolution.

However, \xcd`T()` is not allowed. If there is an argument list, it must be
nonempty.  This avoids a possible confusion between 
\xcd`type T = ...` and \xcd`type T() = ...`.  

A type definition for a type \xcd`T` must appear: 
\begin{itemize}
\item As a top-level definition in a file named \xcd`T.x10`; or
\item As a static member in a container definition; or
\item In a block statment.
\end{itemize}


\paragraph{Use of type definitions in constructor invocations}
If a type definition has no type parameters and no value
parameters and is an alias for a class type, a \xcd"new"
expression may be used to create an instance of the class using
the type definition's name.
Given the following type definition:
%TODO: Yoav says ``I just opened a jira on it: [1918].  I don't think you
% should be able to have {c} on the typedef A if you want to use it in a 'new'
% expression. If we do allow it, then we should allow: new
% Array[Int]{rank==1}(0..2) and new Array[Int](1)(0..2).
\begin{xtenmath}
type A = C[T$_1$, $\dots$, T$_k$]{c};
\end{xtenmath}
where 
\xcdmath"C[T$_1$, $\dots$, T$_k$]" is a
class type, a constructor of \xcdmath"C" may be invoked with
\xcdmath"new A(e$_1$, $\dots$, e$_n$)", if the
invocation
\xcdmath"new C[T$_1$, $\dots$, T$_k$](e$_1$, $\dots$, e$_n$)" is
legal and if the constructor return type is a subtype of
\xcd"A".

\paragraph{Automatically imported type definitions}
\index{import,type definitions}
\label{X10LangUnderscore}

The collection of type definitions in
\xcdmath"x10.lang._" is automatically imported in every compilation unit.


\subsection{Motivation and use}
The primary purpose of type definitions is to provide a succinct,
meaningful name for complex types
and combinations of types. 
With value arguments, type arguments, and constraints, the syntax for \Xten{}
types can often be verbose. 
For example, a non-null list of non-null strings is \\
%~~type~~`~~`~~ ~~import x10.util.*; ^^^ Types220
\xcd`List[String{self!=null}]{self!=null}`.

We could name that type: 
%~~gen ^^^ Types230
% package TypeDefs.glip.first;
% import x10.util.*;
% class LnSn {
% 
%~~vis
\begin{xten}
static type LnSn = List[String{self!=null}]{self!=null};
\end{xten}
%~~siv
%}
%~~neg
Or, we could abstract it somewhat, defining a type constructor
\xcd`Nonnull[T]` for the type of \xcd`T`'s which are not null:
%~~gen ^^^ Types240
% package TypeDefs.glip.second;
% import x10.util.*;
% 
%~~vis
\begin{xten}
class Example {
  static type Nonnull[T]{T <: Object}  = T{self!=null};
  var example : Nonnull[Example] = new Example();
}
\end{xten}
%~~siv
%
%~~neg

Type definitions can also refer to values, in particular, inside 
constraints.  The type of \xcd`n`-element \xcd`Array[Int](1)`s  is 
%~~type~~`~~`~~n:Int ~~ ^^^ Types250
\xcd`Array[Int]{self.rank==1 && self.size == n}`
but it is often convenient to give a shorter name: 
%~~gen ^^^ Types260
% package TypeDefs.glip.third;
% class Xmpl {
% def example() {
%~~vis
\begin{xten}
type Vec(n:Int) = Array[Int]{self.rank==1, self.size == n}; 
var example : Vec(78); 
\end{xten}
%~~siv
%}}
%~~neg

%
The following examples are legal type definitions, given \xcd`import x10.util.*`:
%~~gen ^^^ Types270
% package Types.TypeDef.Examples;
% import x10.util.*;
%~~vis
\begin{xten}
class TypeExamples {
  static type StringSet = Set[String];
  static type MapToList[K,V] = Map[K,List[V]];
  static type Int(x: Int) = Int{self==x};
  static type Dist(r: Int) = Dist{self.rank==r};
  static type Dist(r: Region) = Dist{self.region==r};
  static type Redund(n:Int, r:Region){r.rank==n} 
      = Dist{rank==n && region==r};
}
\end{xten}
%~~siv
% 
%~~neg

The following code illustrates that type definitions are applicative rather
than generative.  \xcd`B` and \xcd`C` are both aliases for \xcd`String`,
rather than new types, and so are interchangeable with each other and with
\xcd`String`. Similarly, \xcd`A` and \xcd`Int` are equivalent.
%~~gen ^^^ Types280
% package Types.TypeDef.Example.NonGenerative;
% import x10.util.*;
% class TypeDefNonGenerative {
%~~vis
\begin{xten}
def someTypeDefs () {
  type A = Int;
  type B = String;
  type C = String;
  a: A = 3;
  b: B = new C("Hi");
  c: C = b + ", Mom!";
  }
\end{xten}
%~~siv
% }
%~~neg
% An instance of a defined type with no type parameters and no
% value parameters may 


%%MEMBERSHIP%% All type definitions are members of their enclosing package or
%%MEMBERSHIP%% class.  A compilation unit may have one or more type definitions
%%MEMBERSHIP%% or class or interface declarations with the same name, as long
%%MEMBERSHIP%% as the definitions have distinct parameters according to the
%%MEMBERSHIP%% method overloading rules (\Sref{MethodOverload}).


\section{Constrained types}
\label{ConstrainedTypes}
\label{DepType:DepType}
\label{DepTypes}

\index{dependent type}
\index{type!dependent}
\index{constrained type}
\index{generic type}
\index{type!constrained}
\index{type!generic}


Basic types, like \xcd`Int` and \xcd`List[String]`, provide useful
descriptions of data.  

However, one frequently wants to say more.  One might want to know
that a \xcd`String` variable is not \xcd`null`, or that a matrix is
square, or that one matrix has the same number of columns as another
has rows (so they can be multiplied).  In the multicore setting, one
might wish to know that two values are located at the same processor,
or that one is located at the same place as the current computation.

In most languages, there is simply no way to say these things
statically.  Programmers must made do with comments, \xcd`assert`
statements, and dynamic tests.  X10 programs can do better, with {\em
  constraints} on types, and guards on class, method and type
definitions,

A constraint is a boolean expression \xcd`e` attached to a basic type \xcd`T`,
written \xcd`T{e}`.  (Only a limited selection of boolean expressions is
available.)  The values of type \xcd`T{e}` are the values of \xcd`T` for which
\xcd`e` is true.  

\index{self}When constraining a value of type \xcd`T`, \xcd`self` refers to the object of
type \xcd`T` which is being constrained.  For example, \xcd`Int{self == 4}` is
the type of \xcd`Int`s which are equal to 4 -- the best possible description
of \xcd`4`, and a very difficult type to express without using \xcd`self`.  

\begin{ex}

\begin{itemize}
%~~type~~`~~`~~ ~~ ^^^ Types290
\item \xcd`String{self != null}` is the type of non-null strings.  \xcd`self`
      is a special variable available only in constraints; it refers to the
      datum being constrained, and its type is the type to which the
      constraint is attached.
\item Suppose that \xcd`Matrix` is a matrix class with  properties \xcd`rows`
      and \xcd`cols`.  
%~~type~~`~~`~~ ~~class Matrix(rows:Int,cols:Int){} ^^^ Types300
      \xcd`Matrix{self.rows == self.cols}` is the type of square matrices.
\item One way to say that \xcd`a` has the same number of columns that \xcd`b`
      has rows (so that \xcd`a*b` is a valid matrix product), one could say: 
%~~gen ^^^ Types310
% package Types.cripes.whered.you.get.those.gripes;
% class Matrix(rows:Int, cols:Int){
% public static def someMatrix(): Matrix = null;
% public static def example(){
%~~vis
\begin{xten}
  val a : Matrix = someMatrix() ;
  var b : Matrix{b.rows == a.cols} ;
\end{xten}
%~~siv
%}}
%~~neg
\end{itemize}
\end{ex}



\xcd"T{e}" is a {\em dependent type}, that is, a type dependent on values. The
type \xcd"T" is called the {\em base type} and \xcd"e" is called the {\em
  constraint}. If the constraint is omitted, it is \xcd`true`---that is, the
  base type is unconstrained.

Constraints may refer to immutable values in the local environment: 
%~~gen ^^^ Types320
% class ConstraintsMayReferToValues {
% def thoseValues() {
%~~vis
\begin{xten}
     val n = 1;
     var p : Point{rank == n};
\end{xten}
%~~siv
%}}
%~~neg
In a variable declaration, the variable itself is in scope in its
type. For example, \xcd`val nz: Int{nz != 0} = 1;` declares a
non-zero variable \xcd`nz`.
\bard{This will need to be explained further once the language issues are
sorted out.}

%%TYPES-CONSTR-EXP%% We permit variable declarations \xcd"v: T" where \xcd"T" is obtained
%%TYPES-CONSTR-EXP%% from a dependent type \xcd"C{c}" by replacing one or more occurrences
%%TYPES-CONSTR-EXP%% of \xcd"self" in \xcd"c" by \xcd"v". (If such a declaration \xcd"v: T"
%%TYPES-CONSTR-EXP%% is type-correct, it must be the case that the variable \xcd"v" is not
%%TYPES-CONSTR-EXP%% visible at the type \xcd"T". Hence we can always recover the
%%TYPES-CONSTR-EXP%% underlying dependent type \xcd"C{c}" by replacing all occurrences of \xcd"v"
%%TYPES-CONSTR-EXP%% in the constraint of \xcd"T" by \xcd"self".)
%%TYPES-CONSTR-EXP%% 
%%TYPES-CONSTR-EXP%% For instance, \xcd"v: Int{v == 0}" is shorthand for \xcd"v: Int{self == 0}".
%%TYPES-CONSTR-EXP%% 
%%TYPES-CONSTR-EXP%% 
%%TYPES-CONSTR-EXP%% A variable occurring in the constraint \xcd"c" of a dependent type, other than
%%TYPES-CONSTR-EXP%% \xcd"self" or a property of \xcd"self", is said to be a {\em
%%TYPES-CONSTR-EXP%% parameter} of \xcd"c".\label{DepType:Parameter} \index{parameter}

A constrained type may be constrained further: the type \xcd`S{c}{d}`
is the same as the type \xcd`S{c,d}`.  Multiple constraints are equivalent to
conjoined constraints: \xcd`S{c,d}` in turn is the same as \xcd`S{c && d}`.

\subsection{Syntax of constraints}
\index{constraint!permitted}
\index{constraint!syntax}
\label{PermittedConstraints}
\index{constraint}
\index{expression!allowed in constraint}
\index{expression!constraint}

Only a few kinds of expressions can appear in constraints.  For fundamental
reasons of mathematical logic, the more kinds of expressions that can appear
in constraints, the harder it is to compute the essential properties of
constrained type -- in particular, the harder it is to compute 
\xcd`A{c} <: B{d}` or even \xcd`E : T{c}`.  It doesn't take much to make this
basic fact undecidable. 
In order to make sure that it stays decidable, X10 places stringent restrictions on
constraints.  

Only the following forms of expression are allowed in constraints.  

{\bf Value expressions in constraints} may be: 
\begin{enumerate}
\item Literal constants, like \xcd`3` and \xcd`true`;
% \item Expressions computable at compile time, like \Xcd{3*4+5};
\item Accessible, immutable (\xcd`val`) variables and parameters;
% \item Accessible and immutable fields of objects;
% \item Properties of the type being constrained;
\item \xcd`this`, if the constraint is in a place where \xcd`this` is defined;
\item \xcd`here`, if the constraint is in a place where \xcd`here` is defined;
\item \xcd`self`;
\item A field selection expression \xcd`t.f`, where \xcd`t` is a value
      expression allowed in constraints, and \xcd`f` is a field of \xcd`t`'s
      type.    
 \item Invocations of property methods,  \xcd`p(a,b,...,c)` or
      \xcd`a.p(b,c,...d)`, where the receiver and arguments must be
       value expressions acceptable in constraints, as long as the expansion
       (\viz, the expression obtained by taking the body of the definition of
       \xcd`p`, and replacing the formal parameters by the actual parameters)
       of the invocation is allowed as a value expression in constraints.  
\end{enumerate}
For an expression \xcd`self.p` to be legal in a constraint, 
\xcd`p` must be 
a property. However terms \xcd`t.f` may be
used in constraints (where \xcd`t` is a term other than \xcd`self` and
\xcd`f` is an immutable field.)

{\bf Constraints}  may be any of
the following, where 
all value expressions are of the forms which may appear in constraints: 
\begin{enumerate}
\item Equalities \xcd`e == f`;
\item Inequalities of the form \xcd`e != f`;\footnote{Currently inequalities
      of the form \xcd`e < f` are not supported.}
\item Conjunctions of Boolean expressions that may appear in constraints (but
      only in top-level constraints, not in Boolean expressions in constraints);
\item Subtyping and supertyping expressions: \xcd`T <: U` and \xcd`T :> U`; 
\item Type equalities and inequalities: \xcd`T == U` and \xcd`T != U`; 
\item Invocations of a property method, \xcd`p(a,b,...,c)` or
      \xcd`a.p(b,c,...d)`, where the receiver and arguments must be value
      expressions acceptable in constraints, as long as the expansion of the
      invocation is allowed as a constraint.
\item Testing a type for a default: \Xcd{T haszero}.
\end{enumerate}

All variables appearing in a constraint expression must be visible wherever
that expression can used.  \Eg, properties and public fields of an object are
always permitted, but private fields of an object can only constrain private
members.  (Consider a class \xcd`PriVio` with a private field \xcd`p` and a
public method \xcd`m(x: Int{self != p})`, and a call \xcd`ob.m(10)` made
outside of the class. Since \xcd`p` is only visible inside the class, there is
no way to tell if \xcd`10` is of type \xcd`Int{self != p}` at the call site.)

{\bf Note:} Constraints may not contain casts.   In particular, comparisons of
values of incompatible types are not allowed.  If \xcd`i:Int`, then \xcd`i==0`
is allowed as a constraint, but \xcd`i==0L` is an error, and 
\xcd`i as Long==0L` is outside of the constraint language.


\subsubsection{Semantics of constraints}
\index{constraint!semantics}
\label{SemanticsOfConstraints}
An assignment of values to variables is said to be a {\em solution} for a
constraint \xcd`c` if under this assignment \xcd`c` evaluates to
\xcd`true`. For instance, the assignment that maps 
the variables \xcd`a` and \xcd`b` to a value \xcd`t` is a solution for
the constraint \xcd`a==b`. An assignment that maps \xcd`a` to 
\xcd`s` and \xcd`b` to a distinct value \xcd`t` is a solution for 
\xcd`a != b`. 

An instance \xcd"o" of \xcd"C" is said to be of type \xcd"C{c}" (or {\em
belong to} \xcd"C{c}") if the constraint \xcd"c" evaluates to \xcd"true" in
the current lexical environment augmented with the binding \xcd"self"
$\mapsto$ \xcd"o".

A constraint \xcd`c` is said to {\em entail} a
constraint \xcd`d` if every solution for \xcd`c` is also a solution
for \xcd`d`. For instance the constraint
\xcd`x==y && y==z && z !=a` entails \xcd`x != a`.

The constraint solver considers the assignment \xcd`a` to \xcd`null`
to satisfy any constraint of the form \xcd`a.f==t`. 
Thus, the declaration 
\xcd`var x:Tree{self.r==p}=null` does not
produce an error, since \xcd`self==null` satisfies the constraint
\xcd`self.r==p`.  
(This only applies in constraints, not in expression evaluation.  
\xcd`if(a.f==t)...` will throw an exception if \xcd`a==null`.)
If \xcd`null` is not an acceptable value for some class \xcd`Tree`, 
add \xcd`self!=null` as a constraint: 
\xcd`Tree{self!=null}` is the class of non-\xcd`null` \xcd`Tree`s.

To ensure that type-checking is decidable, we require that property graphs be
acyclic.  The property graph, at an instant in an X10 execution, is the graph
whose nodes are all objects in existence at that instance, with an edge from
{$x$} to {$y$} if {$x$} is an object with a property whose value is {$y$}. 
The rules for constructors guarantee this.

Constraints participate in the subtyping relationship in a natural way:
\xcdmath"S[S1,$\ldots$, Sm]{c}" 
is a subtype of 
\xcdmath"T[T1,$\ldots$, Tn]{d}" 
if \xcdmath"S[S1,$\ldots$,Sm]" is a subtype of \xcdmath"T[T1,$\ldots$,Tn]" and
\xcd"c" entails \xcd"d".

For examples of constraints and entailment, see (\Sref{ConstraintExamples})
%%TYPES-CONSTR-EXP%% 
%%TYPES-CONSTR-EXP%% \begin{grammar}
%%TYPES-CONSTR-EXP%% Constraint \: ValueArguments     Guard\opt \\
%%TYPES-CONSTR-EXP%%            \| ValueArguments\opt Guard     \\
%%TYPES-CONSTR-EXP%%            \\
%%TYPES-CONSTR-EXP%% ValueArguments   \:  \xcd"(" ArgumentList\opt \xcd")" \\
%%TYPES-CONSTR-EXP%% ArgumentList     \:  Expression ( \xcd"," Expression )\star \\
%%TYPES-CONSTR-EXP%% Guard            \: \xcd"{" DepExpression \xcd"}" \\
%%TYPES-CONSTR-EXP%% DepExpression    \: ( Formal \xcd";" )\star ArgumentList \\
%%TYPES-CONSTR-EXP%% \end{grammar}
%%TYPES-CONSTR-EXP%% 
%%TYPES-CONSTR-EXP%% In \XtenCurrVer{} value constraints may be equalities (\xcd"=="),
%%TYPES-CONSTR-EXP%% disequalities (\xcd"!=") and conjunctions thereof.  The terms over
%%TYPES-CONSTR-EXP%% which these constraints are specified include literals and
%%TYPES-CONSTR-EXP%% (accessible, immutable) variables and fields, property methods, and the special
%%TYPES-CONSTR-EXP%% constants {\tt here}, {\tt self}, and {\tt this}. Additionally, place
%%TYPES-CONSTR-EXP%% types are permitted (\Sref{PlaceTypes}).
%%TYPES-CONSTR-EXP%% 
%%TYPES-CONSTR-EXP%% \index{self}

%%TYPES-CONSTR-EXP%% Type constraints may be subtyping and supertyping (\xcd"<:" and
%%TYPES-CONSTR-EXP%% \xcd":>") expressions over types.

\subsection{Constraint solver: incompleteness and approximation}
\index{constraint solver!incompleteness}
\index{constraint!entailment}
\index{constraint!subtyping}



The constraint solver is sound in that if it claims that \xcd`c` entails \xcd`d`
then in fact it is the case that every value that satisfies \xcd`c`
satisfies \xcd`d`. 

\limitationx{}
X10's constraint solver is incomplete. There are situations
in which \xcd`c` entails \xcd`d` but the solver cannot establish it. For
instance it cannot establish that \xcd`a != b && a != c && b != c`
entails \xcd`false` if \xcd`a`, \xcd`b`, and \xcd`c` are of type
\xcd`Boolean`.


Certain other constraint entailments are prohibitively expensive to calculate.  The issues
concern constraints that connect different levels of recursively-defined
types, such as the following.  
%~~gen ^^^ Types330
% package Types.Entailment.EntailFail;
%~~vis
\begin{xten}
class Listlike(x:Int) {
  val kid : Listlike{self.x == this.x};
  def this(x:Int, kid:Listlike) { 
     property(x); 
     this.kid = kid as Listlike{self.x == this.x};}
}
\end{xten}
%~~siv
%
%~~neg
There is nothing wrong with \xcd`Listlike` itself, or with most uses of it;
however, a sufficiently complicated use of it could, in principle, cause X10's
typechecker to fail. 
It is hard to give a plausible example of when X10's algorithm fails, as we
have not yet observed such a failure in practice for a correct program.  

The entailment algorithm of X10 imposes a certain limit on the number of
times such types will be unwound.   If this limit is exceeded, the compiler
will print a warning, and type-checking will fail in a situation where it is
semantically allowed.  In this case, insert a dynamic cast at the point where
type-checking failed.  

\limitation{ Support for comparisons of generic type variables is
  limited. This will be fixed in future releases.}
% //, and existential quantification over typed variables.

%%TYPES-CONSTR-EXP%% \emph{
%%TYPES-CONSTR-EXP%% Subsequent implementations are intended to support boolean algebra,
%%TYPES-CONSTR-EXP%% arithmetic, relational algebra, etc., to permit types over regions and
%%TYPES-CONSTR-EXP%% distributions. We envision this as a major step towards removing most,
%%TYPES-CONSTR-EXP%% if not all, dynamic array bounds and place checks from \Xten{}.
%%TYPES-CONSTR-EXP%% }




%%PLACE%%\subsection{Place constraints}
%%PLACE%%\label{PlaceTypes}
%%PLACE%%\label{PlaceType}
%%PLACE%%\index{place types}
%%PLACE%%\label{DepType:PlaceType}\index{placetype}
%%PLACE%%
%%PLACE%%An \Xten{} computation spans multiple places (\Sref{XtenPlaces}). Much data
%%PLACE%%can only be accessed from the proper place, and often it is preferable to
%%PLACE%%determine this statically. So, X10 has special syntax for working with places.
%%PLACE%%\xcd`T!` is a value of type \xcd`T` located at the right place for the current
%%PLACE%%computation, and \xcd`T!p` is one located at place \xcd`p`.
%%PLACE%%
%%PLACE%%\begin{grammar}
%%PLACE%%PlaceConstraint     \: \xcd"!" Place\opt \\
%%PLACE%%Place              \:   Expression \\
%%PLACE%%\end{grammar}
%%PLACE%%
%%PLACE%%More specifically, All \Xten{} classes extend the class \xcd"x10.lang.Object",
%%PLACE%%which defines a property \xcd"home" of type \xcd"Place".  \xcd`T!p`, when
%%PLACE%%\xcd`T` is a class, is \xcd`T{self.home==p}`.  If \xcd`p` is omitted, it
%%PLACE%%defaults to \xcd`here`.   \xcd`T!` is far and away the most common usage of
%%PLACE%%\xcd`!`. 
%%PLACE%%
%%PLACE%%Structs don't have \xcd`home`; they are available everywhere.  For structs, 
%%PLACE%%\xcd`T!` and \xcd`T!p` are synonyms for \xcd`T`. Since \xcd`T` is available
%%PLACE%%everywhere, it is available \xcd`here` and at \xcd`p`. 
%%PLACE%%
%%PLACE%%\xcd`!` may be combined with other constraints.  \xcd`T{c}!` is the type of
%%PLACE%%values of \xcd`T!` which satisfy \xcd`c`; it is \xcd`T{c && self.home==here}`
%%PLACE%%for an object type and \xcd`T{c}` for a struct type.  
%%PLACE%%\xcd`T{c}!p` is the type of
%%PLACE%%values of \xcd`T!p` which satisfy \xcd`c`; it is \xcd`T{c && self.home==p}`
%%PLACE%%for an object type and \xcd`T{c}` for a struct type.  
%%PLACE%%
%%PLACE%%
%%PLACE%%
%%PLACE%%% The place specifier \xcd"any" specifies that the object can be
%%PLACE%%% located anywhere.  Thus, the location is unconstrained; that is,
%%PLACE%%% \xcd"C{c}!any" is equivalent to \xcd"C{c}".
%%PLACE%%
%%PLACE%%% XXX ARRAY
%%PLACE%%%The place specifier \xcd"current" on an array base type
%%PLACE%%%specifies that an object with that type at point \xcd"p"
%%PLACE%%%in the array 
%%PLACE%%%is located at \xcd"dist(p)".  The \xcd"current" specifier can be
%%PLACE%%%used only with array types.
%%PLACE%%
%%PLACE%%

\subsection{Limitation: Runtime Constraint Erasure}
\index{cast!to generic type}

The X10 runtime does not maintain a representation of constraints.
In many cases, it does not need to.
If X10 has an object \xcd`x` of some type \xcd`T` around, it can check at
runtime whether or not \xcd`x` satisfies some constraint \xcd`c`, and hence
tell if \xcd`x` is a member of \xcd`T{c}`. 

\begin{ex}
Although there is no runtime representation of the constrained type 
\xcd`Int{self==1}`, X10 can generate a (correct) test for membership in it,
anyhow: 
%~~gen ^^^ Types3u5w
% package Types3u5w;
% class Example {
%~~vis
\begin{xten}
static def example(n:Int) {
  val b = (n instanceof Int{self == 1});
  assert b == (n == 1); 
}
\end{xten}
%~~siv
% }
% class Hook{ def run() {Example.example(0); Example.example(1);
% Example.example(2); return true; } }
%~~neg
\end{ex}

However, in cases where there is no object of type \xcd`T` around, there's
nothing that can be checked. For example, X10 cannot tell -- and in fact no
computer program can tell --  whether an
instance of a function type 
\begin{xtenmath}
(Int)=>Int
\end{xtenmath}
(unary functions returning
integers) is actually an instance of a more specific type
\begin{xtenmath}
(Int)=>Int{self!=0}
\end{xtenmath}
(unary functions returning non-zero integers).

In other cases, there might or might not be an object of type \xcd`T`, and X10
cannot tell until runtime.  Consider an array \xcd`a:Array[T]`.  If \xcd`a` is
nonempty, there is an instance of \xcd`T` at hand, and testing it for
constraints would be possible though potentially quite expensive. 
But \xcd`a` might be an
empty array, and testing its element type would be impossible. 

Rather than pay the runtime costs for keeping and manipulating constraints
(which can be considerable), X10 omits them.
However, this renders certain type checks uncertain: X10 needs some
information at runtime, but does not have it.  
Specifically, all casts to instances of generic types are forbidden.  

\begin{ex}
The following code  needs to be, and is, statically
rejected.  It constructs an array \xcd`a` of \xcd`Int{self==3}`'s -- integers
which 
are statically known to be 3. 
The only number that can be stored into \xcd`a` is \xcd`3`.  
Then (in the line that is rejected) it attempts to trick the compiler into
thinking that it is an array of \xcd`Int`, without restriction on the
elements, giving it the name \xcd`b` at that type.  
The cast \xcd`aa as Array[Int]` is a cast to an instance of a generic type,
and hence is forbidden. 

But, if that cast were allowed to work, it could store \xcd`1` into the array
under the alias 
\xcd`b`, thereby violating 
the invariant that all the elements of the array are 3.  
This could lead to program failures, as illustrated by the failing assertion.  
\begin{xten}
  val a = new Array[Int{self==3}](0..10, 3);
  // a(0) = 1; would be illegal
  a(0) = 3; // LEGAL
  val aa = a as Any;
  /* THE FOLLOWING IS A STATIC ERROR:
  val b = aa as Array[Int];
  b(0) = 1;
  val x : Int{self==3} = a(0);
  assert x == 3 : "This would fail at runtime.";
  */
\end{xten}
\end{ex}



\subsection{Example of Constraints}
\label{ConstraintExamples}

Example of entailment and subtyping involving constraints.
\begin{itemize}
\item \xcd`Int{self == 3} <: Int{self != 14}`.  The only value of
      \xcd`Int{self ==3}` is $3$.  All integers but $14$ are members of
      \xcd`Int{self != 14}`, and in particular $3$ is.  
\item Suppose we have classes \xcd`Child <: Person`, and \xcd`Person` has a
      \xcd`ssn:Long` property.  If \xcd`rhys : Child{ssn == 123456789}`, then
      \xcd`rhys` is also a \xcd`Person`.  
      \xcd`rhys`'s \xcd`ssn` field is the same, \xcd`123456789`, whether 
      \xcd`rhys` is regarded as a \xcd`Child` or a \xcd`Person`.  
      Thus, 
      \xcd`rhys : Person{ssn==123456789}` as well.  
      So, 
\begin{xtenmath}
Child{ssn == 123456789} <: Person{ssn == 123456789}.
\end{xtenmath}
\item Furthermore, since \xcd`123456789 != 555555555`, 
      is is clear that 
      \xcd`rhys : Person{ssn != 555555555}`.  
      So, 
\begin{xtenmath}
Child{ssn == 123456789} <: Person{ssn != 555555555}.  
\end{xtenmath}
\item \xcd`T{e} <: T` for any type \xcd`T`.  That is, if you have a value
      \xcd`v` of some base type \xcd`T` which satisfied \xcd`e`, then \xcd`v`
      is of that base type \xcd`T` (with the constraint ignored).
\item If \xcd`A <: B`, then \xcd`A{c} <: B{c}` for every constraint \xcd`{c}`
      for which \xcd`A{c}` and \xcd`B{c}` are defined.  That is, if every
      \xcd`A` is also a \xcd`B`, and \xcd`a : A{c}`, then 
      \xcd`a` is an \xcd`A` and \xcd`c` is true of it. So \xcd`a` is also a
      \xcd`B` (and \xcd`c` is still true of 
      it), so \xcd`a : B{c}`.  
\end{itemize}

Constraints can be used to express simple relationships between objects,
enforcing some class invariants statically.  For example, in geometry, a line
is determined by two {\em distinct} points; a \xcd`Line` struct can specify the
distinctness in a type constraint:\footnote{We call them
\xcd`Position` to avoid confusion with the built-in class \xcd`Point`. 
Also, \xcd`Position` is a struct rather than a class so that the non-equality
test \xcd`start != end` compares the coordinates.  If \xcd`Position` were a
class, \xcd`start != end` would check for different \xcd`Position` objects,
which might have the same coordinates.
}


%~~gen ^^^ Types340
% package triangleExample.partOne;
%~~vis
\begin{xten}
struct Position(x: Int, y: Int) {}
struct Line(start: Position, end: Position){start != end}
  {}
\end{xten}

%~~siv
%~~neg

Extending this concept, a \xcd`Triangle` can be defined as a figure with three
line segments which match up end-to-end.  Note that the degenerate case in
which two or three of the triangle's vertices coincide is excluded by the
constraint on \xcd`Line`.  However, not all degenerate cases can be excluded
by the type system; in particular, it is impossible to check that the three
vertices are not collinear. 

%~~gen ^^^ Types350
%package triangleExample.partTwo;
% struct Position(x: Int, y: Int) {
%    def this(x:Int,y:Int){property(x,y);}
%    }
% class Line(start: Position, 
%            end: Position{self != start}) {}
% 
%~~vis
\begin{xten}
struct Triangle 
 (a: Line, 
  b: Line{a.end == b.start}, 
  c: Line{b.end == c.start && c.end == a.start})  
 {}
\end{xten}
%~~siv
%
%~~neg

The \xcd`Triangle` class automatically gets a ternary constructor which takes
suitably constrained \xcd`a`, \xcd`b`, and \xcd`c` and produces a new
triangle. 

\section{Default Values}
\index{default value}
\index{type!default value}
\label{DefaultValues}

Some types have default values, and some do not. Default values are used in
situations where variables can legitimately be used without having been
initialized; types without default values cannot be used in such situations.
For example, a field of an object \xcd`var x:T` can be left uninitialized if
\xcd`T` has a default value; it cannot be if \xcd`T` does not. Similarly, a
transient (\Sref{TransientFields}) field \xcd`transient val x:T` is only
allowed if \xcd`T` has a default value.

Default values, or lack of them, is defined thus:
\begin{itemize}
\item The fundamental numeric types (\xcd`Int`, \xcd`UInt`,
      \xcd`Long`, \xcd`ULong`, 
%%limitation%%       \xcd`Short`, \xcd`UShort`, \xcd`Byte`,
%%limitation%%       \xcd`UByte`, 
      \xcd`Float`, \xcd`Double`) all have default value 0.
\item \xcd`Boolean` has default value \xcd`false`.
\item \xcd`Char` has default value \xcd`'\0'`.
\item Struct types other than those listed above have no default value.
\item A function type has a default value of \xcd`null`.
\item A class type has a default value of \xcd`null`.
\item The constrained type \xcd`T{c}` has the same default value as \xcd`T` if
      that default value satisfies \xcd`c`.  If the default value of \xcd`T`
      doesn't satisfy \xcd`c`, then \xcd`T{c}` has no default value.
\end{itemize}

\begin{ex}
\xcd`var x: Int{x != 4}` has default value 0, which is allowed
because \xcd`0 != 4` satisfies the constraint on \xcd`x`. 
\xcd`var y : Int{y==4}` has no default value, because \xcd`0` does not satisfy \xcd`y==4`.
The fact that \xcd`Int{y==4}` has precisely one value, \viz{} 4, doesn't
matter; the only candidate for its default value, as for any subtype of
\xcd`Int`, is 0. \xcd`y` must be initialized before it is used.
\end{ex}

The predicate \xcd`T haszero` tells if the type \xcd`T` has a default value.
\xcd`haszero` may be used in constraints. 

\begin{ex}
The following code defines a sort of cell holding a single value of type
\xcd`T`. The cell is initially empty -- that is, has \xcd`T`'s zero value --
but may be filled later. 
%~~gen ^^^ TypesHaszero10
% package TypesHaszero10;
%~~vis
\begin{xten}
class Cell0[T]{T haszero} {
  public var contents : T;
  public def put(t:T) { contents = t; }
}
\end{xten}
%~~siv
%
%~~neg
\end{ex}

The built-in type \xcd`Zero` has the method \xcd`get[T]()` which
returns the default value of type \xcd`T`.  

\begin{ex}
As a variation on a theme of \xcd`Cell0`, we define a class \xcd`Cell1[T]` which can be initialized with a value of an arbitrary
type
\xcd`T`, or, if \xcd`T` has a default value, can be created with the default
value.  Note that \xcd`T haszero` is a constraint on one of
the constructors, not the whole type:  
%~~gen ^^^ TypesHaszero20
% package TypesHaszero20;
%~~vis
\begin{xten}
class Cell1[T] {
  public var contents: T;
  def this(t:T) { contents = t; }
  def this(){T haszero} { contents = Zero.get[T](); }
  public def put(t:T) {contents = t;}
}
\end{xten}
%~~siv
%
%~~neg

\end{ex}

\section{Function types}
\label{FunctionTypes}
\label{FunctionType}
\index{function!types}
\index{type!function}

%##(FunctionType
\begin{bbgrammar}
%(FROM #(prod:FunctionType)#)
        FunctionType \: TypeParams\opt \xcd"(" FormalList\opt \xcd")" Guard\opt Offers\opt \xcd"=>" Type & (\ref{prod:FunctionType}) \\
\end{bbgrammar}
%##)


For every sequence of types \xcd"T1,..., Tn,T", and \xcd"n" distinct variables
\xcd"x1,...,xn" and constraint \xcd"c", the expression
\xcd"(x1:T1,...,xn:Tn){c}=>T" is a \emph{function type}. It stands for
 the set of all functions \xcd"f" which can be applied to a
 list of values \xcd"(v1,...,vn)" provided that the constraint
 \xcd"c[v1,...,vn,p/x1,...,xn]" is true, and which returns a value of
 type \xcd"T[v1,...vn/x1,...,xn]". When \xcd"c" is true, the clause \xcd"{c}" can be
 omitted. When \xcd"x1,...,xn" do not occur in \xcd"c" or \xcd"T", they can be
 omitted. Thus the type \xcd"(T1,...,Tn)=>T" is actually shorthand for
 \xcd"(x1:T1,...,xn:Tn){true}=>T", for some variables \xcd"x1,...,xn".

\limitationx{}
Constraints on closures are not supported.  They parse, but are not checked.

X10 functions, like mathematical functions, take some arguments and produce a
result.  X10 functions, like other X10 code, can change mutable state and
throw exceptions.  Closures (\Sref{Closures})  are of function type -- and so
are arrays.


\begin{ex}Typical functions are the reciprocal function: 
%~~gen ^^^ Types360
% package Types.Functions;
% class RecipEx {
% static 
%~~vis
\begin{xten}
val recip = (x : Double) => 1/x;
\end{xten}
%~~siv
%}
%~~neg
and a function which increments  element \xcd`i` of an array \xcd`r`, or throws an exception
if there is no such element, where, for the sake of example, we constrain the
type of \xcd`i` to avoid one of the many integers which are not possible subscripts:  
%~~gen ^^^ Types_constraint_b
% package Types_constraint_b;
% NOTEST
% /*NONSTATIC*/class IncrElEx {
% static def example()  {
%~~vis
\begin{xten}
val inc = (r:Array[Int](1), i: Int{i != r.size}) => {
  if (i < 0 || i >= r.size) throw new DoomExn();
  r(i)++;
};
\end{xten}
%~~siv
%}}
%class DoomExn extends Exception{}
%~~neg
\end{ex}

In general, a function type needs to list the types 
\xcdmath"T$_i$"
of all the formal parameters,
and their distinct names \xcdmath"x$_i$" in case other types refer to them; a
constraint 
\xcd"c" on the
function as a whole; a return type \xcd"T".

\begin{xtenmath}
(x$_1$: T$_1$, $\dots$, x$_n$: T$_n$){c} => T
\end{xtenmath}


The names \xcdmath"x$_i$" of the formal parameters are not relevant.  Types
which differ only in the names of formals (following the usual rules for
renaming of variables, as in {$\alpha$}-renaming in the {$\lambda$} calculus
\bard{cite something}) are considered equal.  \Eg, the two function types
%~~type~~`~~`~~ ~~ ^^^ Types370
\xcd`(a:Int, b:Array[String](1){b.size==a}) => Boolean`
and \\
%~~type~~`~~`~~ ~~ ^^^ Types380
\xcd`(b:Int, a:Array[String](1){a.size==b}) => Boolean`
are equivalent.

\limitation{
Function types differing only in the names of bound variables may wind up being
considered different in X10 v2.2, especially if the variables appear in
constraints.  
}

The formal parameter names are in scope from the point of definition to the
end of the function type---they may be used in the types of other formal parameters
and in the return type. 
Value parameters names may be
omitted if they are not used; the type of the reciprocal function can be
written as
%~~type~~`~~`~~ ~~ ^^^ Types390
\xcd`(Double)=>Double`. 

A function type is covariant in its result type and contravariant in
each of its argument types. That is, let 
\xcd"S1,...,Sn,S,T1,...Tn,T" be any
types satisfying \xcd"Si <: Ti" and \xcd"S <: T". Then
\xcd"(x1:T1,...,xn:Tn){c}=>S" is a subtype of
\xcd"(x1:S1,...,xn:Sn){c}=>T".

A class or struct definition may use a function type 
\begin{xtenmath}
F = (x1:T1,...,xn:Tn){c}=>T
\end{xtenmath}
in its 
implements clause; 
this is equivalent to implementing an interface requiring the single operator
\begin{xtenmath}
public operator this(x1:T1,...,xn:Tn){c}:T
\end{xtenmath}
Similarly, an interface
definition may specify a function type \xcd"F" in its \xcd"extends" clause.
Values of a class or struct implementing \xcd`F` 
can be used as functions of type \xcd`F` in all ways.  
In particular, applying one to suitable arguments calls the \xcd`apply`
method. 

\limitationx{} A class or struct may not implement two different
instantiations of a generic interface. In particular, a class or
struct can implement only one function type.


A function type \xcd"F" is not a class type in that it does not extend any
type or implement any interfaces, or support equality tests. 
\xcd`F` may be implemented, but not extended, by a class or function type. 
Nor is it a struct type, for it has no predefined notion of equality.


\section{Annotated types}
\label{AnnotatedTypes}

\index{type!annotated}
\index{annotations!type annotations}

        Any \Xten{} type may be annotated with zero or more
        user-defined \emph{type annotations}
        (\Sref{XtenAnnotations}).  

        Annotations are defined as (constrained) interface types and are
        processed by compiler plugins, which may interpret the
        annotation symbolically.

        A type \xcd"T" is annotated by interface types
        \xcdmath"A$_1$", \dots,
        \xcdmath"A$_n$"
        using the syntax
        \xcdmath"@A$_1$ $\dots$ @A$_n$ T".

\section{Subtyping and type equivalence}\label{DepType:Equivalence}
\index{type equivalence}
\index{subtyping}

Intuitively, type \xcdmath"T$_1$" is a subtype of type \xcdmath"T$_2$", 
written \xcdmath"T$_1$ <: T$_2$", 
if
every instance of \xcdmath"T$_1$" is also an instance of \xcdmath"T$_2$".  For
example, \xcd`Child` is a subtype of \xcd`Person` (assuming a suitably defined
class hierarchy): every child is a person.  Similarly, \xcd`Int{self != 0}`
is a subtype of \xcd`Int` -- every non-zero integer is an integer.  

This section formalizes the concept of subtyping. Subtyping of types depends
on a {\em type context}, \viz. a set of constraints on type parameters
and variables that occur in the type.
For example: 

%~~gen ^^^ Types400
% package Types.subtyping.cons;
% NOCOMPILE
%~~vis
\begin{xten}
class ConsTy[T,U] {
   def upcast(t:T){T <: U} :U = t;
}
\end{xten}
%~~siv
%
%~~neg
\noindent
Inside \xcd`upcast`, \xcd`T` is constrained to be a subtype of \xcd`U`, and so
\xcd`T <: U` is true, and \xcd`t` can be treated as a value of type \xcd`U`.  
Outside of \xcd`upcast`, there is no reason to expect any relationship between
them, and \xcd`T <: U` may be false.
However, subtyping of types that have no free variables does not depend
on the context.    \xcd`Int{self != 0} <: Int` is always
true.

\limitation{Subtyping of type variables does not work under all circumstances
in the X10 2.2 implementation.}


\begin{itemize}
\item {\bf Reflexivity:} Every type \xcd`T` is a subtype of itself: \xcd`T <: T`.

\item {\bf Transitivity:} If \xcd`T <: U` and \xcd`U <: V`, then \xcd`T <: V`. 

\iffalse
{\bf Class types:}  
Given the definition 
\xcd`class C[$\vec{X}$] extends D[$\vec{Y}$]{d} implements I1, ..., In {...}`
where {$\vec{X}$} is a vector of type variables, and 
{$\vec{Y$} a vector of types possibly involving variables from {$\vec{X}$}, 
and {$\vec{T$} an instantiation of {$\vec{X$} and {$\vec{U$} the corresponding
instantiation of {$\vec{Y$}, 
then 
\xcdmath"C[$\vec{T}$]`"is a subtype of \xcd`D[$\vec{U}$]{d}`, \xcd`I1`, ..., \xcd`In`. 

\item
{\bf Interface types:}  
Given the definition 
\xcdmath"interface I[$\vec{X}$] extends I1, ... In {...}`"
then \xcdmath"I` is a subtype of \xcd`"1`, ..., \xcd`In`.

\item 
{\bf Struct types:} 
Given the definition 
\xcdmath"struct S implements I1, ..., In {...}`"then \xcd`S` is a 
subtype of \xcd`I1`, ..., \xcd`In`. 
\fi

\item {\bf Direct Subclassing:} 
Let {$\vec{X}$} be a (possibly empty) vector of type variables, and
{$\vec{Y}$}, {$\vec{Y_i}$} be vectors of type terms over {$\vec{X}$}.
Let {$\vec{T}$} be an instantiation of {$\vec{X}$}, 
and {$\vec{U}$}, {$\vec{U_i}$} the corresponding instantiation of 
{$\vec{Y}$}, {$\vec{Y_i}$}.  Let \xcd`c` be a constraint, and \xcdmath"c$'$"
be the corresponding instantiation.
We elide properties, and interpret empty vectors as absence of the relevant
clauses. 
Suppose that \xcd`C` is declared by one of the
forms: 
\begin{enumerate}
\item \xcdmath"class C[$\vec{X}$]{c} extends D[$\vec{Y}$]{d}"\\
\xcdmath"implements I$_1[\vec{Y_1}]${i$_1$},...,I$_n[\vec{Y_n}]${i$_n$}{"
\item \xcdmath"interface C[$\vec{X}$]{c} extends I$_1[\vec{Y_1}]${i$_1$},...,I$_n[\vec{Y_n}]${i$_n$}{"
\item \xcdmath"struct C[$\vec{X}$]{c} implements I$_1[\vec{Y_1}]${i$_1$},...,I$_n[\vec{Y_n}]${i$_n$}{"
\end{enumerate}
Then: 
\begin{enumerate}
\item \xcdmath"C[$\vec{T}$] <: D[$\vec{U}$]{d}" for a class
\item \xcdmath"C[$\vec{T}$] <: I$_i$[$\vec{U_i}$]{i$_i$}" for all cases.
\item \xcdmath"C[$\vec{T}$] <: C[$\vec{T}$]{c$'$}" for all cases.
\end{enumerate}


\item
{\bf Function types:}
\begin{xtenmath}
(x$_1$: T$_1$, $\dots$, x$_n$: T$_n$){c} => T
\end{xtenmath}
is a  subtype of 
\begin{xtenmath}
(x$'_1$: T$'_1$, $\dots$, x$'_n$: T$'_n$){c$'$} => T$'$
\end{xtenmath}
if: 
\begin{enumerate}
\item Each \xcdmath"T$_i$ <: T$'_i$";
\item \xcdmath"c[x$'_1$, $\ldots$, x$'_n$ / x$_1$, $\ldots$, x$_n$]" entails \xcdmath"c$'$";
\item \xcdmath"T$'$ <: T";
\end{enumerate}

\item
{\bf Constrained types:}
\xcd`T{c}` is a subtype of \xcd`T{d}` if \xcd`c` entails \xcd`d`. 

\item {\bf Any:} 
Every type \xcd`T` is a subtype of \xcd`x10.lang.Any`.

\item 
{\bf Type Variables:}
Inside the scope of a constraint \xcd`c` which entails \xcd`A <: B`, we have
\xcd`A <: B`.  \eg, \xcd`upcast` above.


%%NO-VARIANCE%% \item 
%%NO-VARIANCE%% {\bf Covariant Generic Types:} 
%%NO-VARIANCE%% If \xcd`C` is a generic type whose {$i$}th type parameter is covariant, 
%%NO-VARIANCE%% and {\xcdmath"T$'_i$ <: T$_i$"}
%%NO-VARIANCE%% and  {\xcdmath"T$'_j$ == T$_j$"} for all {$j \ne i$}, 
%%NO-VARIANCE%% then {\xcdmath"C[T$'_1$, $\ldots$, T$'_n$] <: C[T$'_1$, $\ldots$, T$'_n$]"}.
%%NO-VARIANCE%% \Eg, \xcd`class C[T1, +T2, T3]` with {$i=2$}, and \xcd"U2 <: T2", then
%%NO-VARIANCE%% \xcd`C[T1,U2,T3] <: C[T1,T2,T3]`.
%%NO-VARIANCE%% 
%%NO-VARIANCE%% \item 
%%NO-VARIANCE%% {\bf Contravariant Generic Types:} 
%%NO-VARIANCE%% If \xcd`C` is a generic type whose {$i$}th type parameter is contravariant, 
%%NO-VARIANCE%% and \xcdmath"T$'_i$ <: T$_i$"
%%NO-VARIANCE%% and  \xcdmath"T$'_j$ == T$_j$" for all {$j \ne i$}, 
%%NO-VARIANCE%% then \xcdmath"C[T$'_1$, $\ldots$, T$'_n$] :> C[T$'_1$, $\ldots$, T$'_n$]".
%%NO-VARIANCE%% \Eg, \xcd`class C[T1, -T2, T3]` with {$i=2$}, and \xcdmath"U2 <: T2", then
%%NO-VARIANCE%% \xcd`C[T1,U2,T3] :> C[T1,T2,T3]`.
%%NO-VARIANCE%% 

\end{itemize}


Two types are {\em equivalent}, \xcd`T == U`, if \xcd`T <: U` and \xcd`U <: T`. 


\section{Common ancestors of types}
\label{LCA}

There are several situations where X10 must find a type \xcd`T` that describes
values of two or more different types.  This arises when X10 is trying to find
a good type for: 
\begin{itemize}
%~~exp~~`~~`~~test:Boolean ~~ ^^^ Types410
\item Conditional expressions, like \xcd`test ? 0 : "non-zero"` or even \\
%~~exp~~`~~`~~test:Boolean ~~ ^^^ Types420
      \xcd`test ? 0 : 1`;
%~~exp~~`~~`~~ ~~ ^^^ Types430
\item Array construction, like \xcd`[0, "non-zero"]` and 
%~~exp~~`~~`~~ ~~ ^^^ Types440
      \xcd`[0,1]`;
\item Functions with multiple returns, like
%~~gen ^^^ Types450
% package Types_odd_inferred_return_type;
% class Examplerator {
%~~vis
\begin{xten}
def f(a:Int) {
  if (a == 0) return 0;
  else return "non-zero";
}
\end{xten}
%~~siv
%}
%~~neg
\end{itemize}

In some cases, there is a unique best type describing the expression.  For
example, if \xcd`B` and \xcd`C` are direct subclasses of \xcd`A`, \xcd`pick`
will have return type \xcd`A`: 
%~~gen ^^^ Types_uniq
% package Types.For.Gripes.About.Pipes.Full.Of.Wipes;
%  class A {} class B extends A{} class C extends A{}
% class D {
%~~vis
\begin{xten}
static def pick(t:Boolean, b:B, c:C) = t ? b : c;  
\end{xten}
%~~siv
%}
%~~neg

However, in many common cases, there is no unique best type describing the
expression.  For example, consider the expression {$E$} 
\begin{xtenmath}
b ? 0 : 1   // Call this expression $E$
\end{xtenmath}
The
best type of \xcd`0` 
is \xcd`Int{self==0}`, and the best type of 1 is \xcd`Int{self==1}`.
Certainly {$E$} could be given the type \xcd`Int`, or even \xcd`Any`, and that
would describe all possible results.  However, we actually know more.
\xcd`Int{self != 2}` is a better description of the type of {$E$}---certainly
the result of {$E$} can never be \xcd`2`.   \xcd`Int{self != 2, self != 3}` is
an even better description; {$E$} can't be \xcd`3` either.  We can continue
this process forever, adding integers which {$E$} will definitely not return
and getting better and better approximations. (If the constraint
sublanguage had \xcd`||`, we could give it the type 
\xcd`Int{self == 0 || self == 1`, which would be nearly perfect.  But 
\xcd`||` makes typechecking far more expensive, so it is excluded.)
No X10 type is the best description of {$E$}; there is always a better one.

Similarly, consider two unrelated interfaces: 
%~~gen ^^^ Types460
% package Types.For.Gripes.About.Snipes;
%~~vis
\begin{xten}
interface I1 {}
interface I2 {}
class A implements I1, I2 {}
class B implements I1, I2 {}
class C {
  static def example(t:Boolean, a:A, b:B) = t ? a : b;
}
\end{xten}
%~~siv
%
%~~neg
\xcd`I1` and \xcd`I2` are both perfectly good descriptions of \xcd`t ? a : b`, 
but neither one is better than the other, and there is no single X10 type
which is better than both. (Some languages have {\em conjunctive
    types}, and could say that the return type of \xcd`example` was 
\xcd`I1 && I2`.  This, too, complicates typechecking.)


So, when confronted with expressions like this, X10 computes {\em some}
satisfactory type for the expression, but not necessarily the {\em best} type.  
X10 provides certain guarantees about the common type \xcd`V{v}` computed for 
\xcd`T{t}` and \xcd`U{u}`: 
\begin{itemize}
\item If \xcd`T{t} == U{u}`, then \xcd`V{v} == T{t} == U{u}`.  So, if X10's
      algorithm produces an utterly untenable type for \xcd`a ? b : c`, and
      you want the result to have type \xcd`T{t}`, you can 
      (in the worst case) rewrite it to 
\begin{xtenmath}
a ? b as T{t} : c as T{t}
\end{xtenmath}
\item If \xcd`T == U`, then \xcd`V == T == U`.  For example, 
      X10 will compute the type of \xcd`b ? 0 : 1` as 
      \xcd`Int{c}` for some constraint \xcd`c`---perhaps simply 
      picking \xcd`Int{true}`, \viz, \xcd`Int`. 
\item X10 preserves place information about \xcd`GlobalRef`s, because it is so important. If both
      \xcd`t` and \xcd`u` entail \xcd`self.home==p`, then  
      \xcd`v` will also entail \xcd`self.home==p`.  
\item X10 similarly preserves nullity information.  If \xcd`t` and \xcd`u`
      both entail \xcd`x == null` or \xcd`x != null` for some variable
      \xcd`x`, then \xcd`v` will also entail it as well.

\item The computed upper bound of function types with the {\em same} argument
      types is found by computing the upper bound of the result types.  
      If 
      \xcdmath"T = (T$_1$, $\ldots$, T$_n$) => T'"
      and 
      \xcdmath"U = (T$_1$, $\ldots$, T$_n$) => U'", 
      and \xcd`V'` is the computed upper bound of \xcd`T'` and \xcd`U'`, 
      then the computed upper bound of \xcd`T` and \xcd`U` is 
      \xcdmath"U = (T$_1$, $\ldots$, T$_n$) => V'".
      (But, if the argument types are different, the computed upper bound may
      be \xcd`Any`.)

\end{itemize}

%\subsection{Syntactic abbreviations}\label{DepType:SyntaxAbbrev}

\section{Fundamental types}

Certain types are used in fundamental ways by X10.  

\subsection{The interface {\tt Any}}

It is quite convenient to have a type which all values are instances of; that
is, a supertype of all types.\footnote{Java, for one, suffers a number of
  inconveniences because some built-in types like \xcd`int` and \xcd`char`
  aren't subtypes of anything else.}  X10's universal supertype is the
  interface \xcd`Any`. 

\begin{xten}
package x10.lang;
public interface Any {
  def toString():String;
  def typeName():String;
  def equals(Any):Boolean;
  def hashCode():Int;
}
\end{xten}

\xcd`Any` provides a handful of essential methods that make sense and are
useful for everything. \xcd`a.toString()` produces a
string representation of \xcd`a`, and \xcd`a.typeName()` the string
representation of its type; both are useful for debugging.  \xcd`a.equals(b)`
is the programmer-overridable equality test, and \xcd`a.hashCode()` an integer
useful for hashing.  


\subsection{The class {\tt Object}}
\label{Object}
\index{\Xcd{Object}}
\index{\Xcd{x10.lang.Object}}

The class \xcd"x10.lang.Object" is the supertype of all classes.
A variable of this type can hold a reference to any object.
\xcd`Object` implements \xcd`Any`.



\section{Type inference}
\label{TypeInference}
\index{type!inference}
\index{type inference}

\XtenCurrVer{} supports limited local type inference, permitting
certain variable types and return types to be elided.
It is a static error if an omitted type cannot be inferred or
uniquely determined. Type inference does not consider coercions.

\subsection{Variable declarations}

The type of a \xcd`val` variable declaration can be omitted if the
declaration has an initializer.  The inferred type of the
variable is the computed type of the initializer.
For example, 
%~~stmt~~`~~`~~ ~~ ^^^ Types470
\xcd`val seven = 7;`
is identical to 
\begin{xtenmath}
val seven: Int{self==7} = 7;
\end{xtenmath}
Note that type inference gives the most precise X10 type, which might be more
specific than the type that a programmer would write.



\limitation{At the moment,  \xcd`var` declarations may not have their types
elided in this way.  
}

\subsection{Return types}

The return type of a method can be omitted if the method has a body (\ie, is
not \xcd"abstract" or \xcd"native"). The inferred return type is the computed
type of the body.  In the following example, the return type inferred for
\xcd`isTriangle` is 
%~~type~~`~~`~~ ~~ ^^^ Types490
\xcd`Boolean{self==false}`
%~~gen ^^^ Types500
% package Types.Inferred.Return;
%~~vis
\begin{xten}
class Shape {
  def isTriangle() = false; 
}  
\end{xten}
%~~siv
%
%~~neg
Note that, as with other type inference, methods are given the most specific
type.  In many cases, this interferes with subtyping.  For example, if one
tried to write: 
\begin{xten}
class Triangle extends Shape {
  def isTriangle() = true;
}
\end{xten}
\noindent
the compiler would reject this program for attempting to override
\xcd`isTriangle()` by a method with the wrong type, \viz,
\xcd`Boolean{self==true}`.  In this case, supply the type that is actually
intended for \xcd`isTriangle`: 
\begin{xtenmath}def isTriangle() :Boolean =false;
\end{xtenmath}

The return type of a closure can be omitted.
The inferred return type is the computed type of the body.

The return type of a constructor can be omitted if the
constructor has a body.
The inferred return type is the enclosing class type with
properties bound to the arguments in the constructor's \xcd"property"
statement, if any, or to the unconstrained class type.
For example, the \xcd`Spot` class has two constructors, the first of which has
inferred return type \xcd`Spot{x==0}` and the second of which has 
inferred return type \xcd`Spot{x==xx}`. 
%~~gen ^^^ Types510
% package Types.Inferred.By.Phone;
%~~vis
\begin{xten}
class Spot(x:Int) {
  def this() {property(0);}
  def this(xx: Int) { property(xx); }
}
\end{xten}
%~~siv
%class Confirm{ 
% static val s0 : Spot{x==0} = new Spot();
% static val s1 : Spot{x==1} = new Spot(1);
%}
%~~neg


\index{void}

A method or closure that has expression-free \xcd`return` statements
(\xcd`return;` rather than \xcd`return e;`) is said to return \xcd`void`.
\xcd`void` is not a type; there are no \xcd`void` values, nor can \xcd`void`
be used as the argument of a generic type. However, \xcd`void` takes the
syntactic place of a type in a few contexts. A method returning \xcd`void` can be specified by
\xcd`def m():void`: 

%~~gen ^^^ Types520
% package Types.voidd;
% class voidddd {
% static 
%~~vis
\begin{xten}
val f : () => void = () => {return;};
\end{xten}
%~~siv
%}
%~~neg

By a convenient abuse of language, \xcd`void` is sometimes
lumped in with types; \eg, we may say ``return type of a method'' rather than
the formally correct but rather more awkward ``return type of a method, or
\xcd`void`''.   Despite this informal usage, \xcd`void` is not a type.  For
example, given 
%~~gen ^^^ Types530
% package Types.void_is_not_a_type;
% class EEEEVil {
%~~vis
\begin{xten}
  static def eval[T] (f:()=>T):T = f();
\end{xten}
%~~siv
% }
%~~neg
\noindent
The call \xcd`eval[void](f)` does {\em not} typecheck; \xcd`void` is not a
type and thus cannot be used as a type argument.  There is no way in X10 to
write a generic function which works with both functions which return a value
and functions which do not.  In most cases, functions which have no sensible
return value can be provided with a dummy return value.

\subsection{Inferring Type Arguments}
\label{TypeParamInfer}


A call to a polymorphic method %, closure, or constructor 
may omit the
explicit type arguments.  
X10 will compute a type from the types of the actual arguments. 

(Exception: it is an error if the method call provides no information about
a type parameter that must be inferred.  For example, given the method
definition \xcd`def m[T](){...}`, an invocation \xcd`m()` is considered a
static error.  The compiler has no idea what \xcd`T` the programmer intends.)



\begin{ex}Consider the following method, which chooses one of its arguments.  (A more
sophisticated one might sometimes choose the second argument, but that does
not matter for the sake of this example.)
\begin{xten}
static def choose[T](a: T, b: T): T = a; 
\end{xten}


The type argument \xcd`T` can always be supplied: 
\xcd`choose[Int](1, 2)` picks an integer, 
and \xcd`choose[Any](1, "yes")` picks a value that might be an integer or a
string.  
However, the type argument can be elided.  Suppose that \xcd`Sub <: Super`;
then the following compiles: 

%~~gen ^^^ Types540
% package Types.GenericInference;
% class Exampllll{ 
%~~vis
\begin{xten}
  static def choose[T](a: T, b: T): T = a; 
  static val j : Any = choose("string", 1);
  static val k : Super = choose(new Sub(), new Super());
\end{xten}
%~~siv
%}
% class Super {}
% class Sub extends Super {}
%~~neg
\end{ex}

The type parameter doesn't need to be the type of a variable. It can be found
inside of the type of a variable; X10 can extract it.

\begin{ex}
The \xcd`first` method below returns the first element of a one-dimensional
array.  The type parameter \xcd`T` represents the type of the array's
elements. There is no parameter of type \xcd`T`. There is one of type
\xcd`Array[T]{c}`; X10 strips off the constraint \xcd`{c}` and the
\xcd`Array[...]` type to get at the \xcd`T` inside.
%~~gen ^^^ Types3d5j
% package Types3d5j;
% class Example {
%~~vis
\begin{xten}
static def first[T](x:Array[T](1)) = x(0);
static def example() {
  val ss <: Array[String] = ["X10", "Scala", "Thorn"];
  val s1 = first(ss);
  assert s1.equals("X10");
}
\end{xten}
%~~siv
%}
% class Hook{ def run() {Example.example(); return true;}}
%~~neg

\end{ex}


\subsubsection{Sketch of X10 Type Inference for Method Calls}

When the X10 compiler sees a method call 
\begin{xtenmath}
a.m(b$_1$, $\ldots$,b$_n$)
\end{xtenmath}
and attempts to infer type parameters to
see if it could be a use of a
method 
\begin{xtenmath}
def m[X$_1$, $\ldots$, X$_t$](y$_1$: S$_1$, $\ldots$, y$_n$:S$_n$),
\end{xtenmath}
it reasons as follows. 



Suppose that \xcdmath"b$_i$" has type \xcdmath"T$_i$".  Then, X10 is seeking a
set of type {$B$} bindings 
\begin{xtenmath}
X$_j$ = U$_j$, 
\end{xtenmath}
for $1 \le j \le t$, 
such that 
\xcdmath"T$_i$ <: S$^*_i$" for {$1 \le i \le n$}, where \xcdmath"S$^*$" is
\xcd`S` with each type variable \xcdmath"X$_j$" replaced by the corresponding
\xcdmath"U$_j$".  If it can find such a {$B$}, it has a usable choice of type
arguments and can do the type inference.  If it cannot find {$B$}, then it
cannot do type inference.    (Note that X10's type inference algorithm is
incomplete -- there may {\em be} such a {$B$} that X10 cannot find.  If this
occurs in your program, you will have to write down the type arguments
explicitly.) 

Let $B_0$ be the set {$\{ T_i \subtype S_i | 1 \le i \le n\}$}.  Let
{$B_{n+1}$} be {$B_n$} with one element {$F \subtype G$} or 
{$F \typeeq G$} removed, and
{$C(F \subtype G)$} 
or {$C(F \typeeq G)$} (defined below) added.  Repeat this until 
{$B_n$} consists entirely of comparisons with type variables (\viz, 
\xcdmath"Y$_j$ == U", 
\xcdmath"Y$_j$ <: U", and
\xcdmath"Y$_j$ :> U"), 
or until some {$n$} exceeds a predefined compiler limit. 

The candidate inferred types may be read off of {$B_n$}.  The guessed binding
for \xcdmath"X$_j$" is: 
\begin{itemize}
\item If there is an equality \xcdmath"X$_j$==W" in {$B_n$}, then guess the
      binding \xcdmath"X$_j$=W".  Note that there may be several such
      equalities with different choices of \xcd`W`; pick any one.  If the
      chosen binding does not equal the others, the candidate binding will be
      rejected later. 
\item Otherwise, if there is one or more upper bounds 
\xcdmath"X$_j$ <: V$_k$" in {$B_n$}, guess the binding 
\xcdmath"X$_j$ = V$_+$", where 
\xcdmath"V$_+$" is the computed lower bound of all the \xcdmath"V$_k$"'s.
\item Otherwise, if there is one or more lower bounds 
\xcdmath"R$_k$ <: X$_j$", guess that
\xcdmath"X$_j$ = R$_+$", where 
\xcdmath"R$_+$" is the computed upper bound of all the \xcdmath"R$_k$"'s.
\end{itemize}
If this does not yield a binding for some variable \xcdmath"X$_j$", then type
inference fails.  Furthermore, if every variable \xcdmath"X$_j$" is given a
binding \xcdmath"U$_j$", but the 
bindings do not work --- 
that is, if 
\xcdmath"a.m[U$_1$, $\ldots$, U$_t$](b$_1$, $\ldots$,b$_n$)"
is not a call of 
the original method 
\xcdmath"def m[X$_1$, $\ldots$, X$_t$](y$_1$: S$_1$, $\ldots$, y$_n$:S$_n$)"
--- then type inference also fails.

\paragraph{Computation of the Replacement Elements}

Given a type relation
{$r$} of the form {$F \subtype G$}
or {$F \typeeq G$}, we compute the set {$C(r)$} of
replacement constraints.  There are a number of cases; we present only the
interesting ones. 

\begin{itemize}
\item If $F$ has the form \xcdmath"$F'${c}", then  
\xcdmath"$C(r)$" is defined to be
 \xcdmath"$F'$ == $G$" if $r$ is an equality, or 
 \xcdmath"$F'$ <: $G$" if {$r$} is a subtyping.
That is, we erase type constraints.  
Validity is not an issue at this point in the algorithm, as 
we check at the end that the result is valid.
Note that, if the equation had the form \xcdmath"Z{c} == A", it could be
solved by either \xcd`Z==A` or by \xcd`Z = A{c}`.  By dropping constraints in this
rule, we choose the former solution. 

\item Similarly, we drop constraints on {$G$} as well.

\item If {$F$} has the form \xcdmath"K[F$_1$, $\ldots$, F$_k$]"
and 
{$G$}
has the form \xcdmath"K[G$_1$, $\ldots$, G$_k$]", 
then {$C(r)$} has one type relation comparing each parameter of 
{$F$} with the corresponding one of {$G$}: 
\[C(r) = \{ F_l \typeeq G_l | 1 \le l \le k \} \]

For example, the constraint \xcdmath"List[X] == List[Y]" induces the
constraint \xcd`X==Y`.  
\xcd`List[X] <: List[Y]` also induces the same constraint.  The only way that
\xcd`List[X]` could be a subtype of \xcd`List[Y]` in X10 is if \xcd`X==Y`.
List of different types are incomparable.\footnote{The situation would be more
complex if X10 had covariant and contravariant types.}

\item Other cases are fairly routine.  \Eg, if {$F$} is a \xcd`type`-defined
      abbreviation, it is expanded.

\end{itemize}

\begin{ex}
Consider the program: 
%~~gen ^^^ Types1s4y
% package Types1s4y;
%~~vis
\begin{xten}
import x10.util.*;
class Cl[C1, C2, C3]{}
class Example {
  static def me[X1, X2](Cl[Int, X1, X2]) = 
     new Cl[X1, X2, Point]();
  static def example() {
    val a = new Cl[Int, Boolean, String]();
    val b : Cl[Boolean, String, Point] 
          = me[Boolean, String](a);
    val c : Cl[Boolean, String, Point] 
          = me(a);
  }
}
\end{xten}
%~~siv
%
%~~neg
The method call for \xcd`b` has explicit type parameters.  
The call for \xcd`c` infers the parameters.  The computation 
starts with one equation, saying that the type of the formal parameter of 
\xcd`me` has to be the same as the type of the actual parameter, \viz, the
type of \xcd`a`:
\begin{xtenmath}
Cl[Int, X1, X2] == Cl[Int, Boolean, String]
\end{xtenmath}
Note that both terms are \xcd`Cl` of three things. 
This is broken into three equations: 
\begin{xtenmath}
Int == Int
\end{xtenmath}
which is easy to satisfy,
\begin{xtenmath}
X1 == Boolean
\end{xtenmath}
which suggests a possible value for \xcd`X1`,  and 
\begin{xtenmath}
X2 == String
\end{xtenmath}
which suggests a value for \xcd`X2`.  
All of these equations are simple enough, so the algorithm terminates. 

Then, X10 confirms that the binding \xcd`X1==Boolean`, \xcd`X2==String`
atually generates a correct call, which it does.  
\end{ex}

\section{Type Dependencies}

Type definitions may not be circular, in the sense that no type may be its own
supertype, nor may it be a container for a supertype. This forbids interfaces
like \xcd`interface Loop extends Loop`, and indirect self-references such as
\xcd`interface A extends B.C` where \xcd`interface B extends A`.  
The formal definition of this is based on Java's.  

An {\em entity type} is a class, interface, or struct type.   

Entity type $E$ {\em directly depends on} entity type $F$ if $F$ is mentioned
in the \xcd`extends` or \xcd`implements` clause of $E$, either by itself or as
a qualifier within a super-entity-type name.  

\begin{ex}
In the following, \xcd`A` directly depends on \xcd`B`, \xcd`C`, \xcd`D`, 
\xcd`E`, and \xcd`F`.    It does not directly depend on \xcd`G`.
%~~gen ^^^ Types6a9m
% package Types6a9m;
% NOTEST
% class B{ static class C{}}
% class D{ static interface E{}}
% interface F[X]{}
% class G{}
%~~vis
\begin{xten}
class A extends B.C implements D.E, F[G] {}
\end{xten}
%~~siv
%
%~~neg

It is an ordinary programming idiom to use \xcd`A` as an argument to a generic
interface that \xcd`A` implements.  For example, \xcd`ComparableTo[T]`
describes things which can be compared to a value of type \xcd`T`. Saying that
\xcd`A` implements \xcd`ComparableTo[A]` means that one \xcd`A` can be
compared to another, which is reasonable and useful: 
%~~gen ^^^ Types2x6d
% package Types2x6d;
%~~vis
\begin{xten}
interface ComparableTo[T] {
  def eq(T):Boolean;
}
class A implements ComparableTo[A] {
  public def eq(other:A) = this.equals(other);
}
\end{xten}
%~~siv
%
%~~neg
\end{ex}

Entity type $E$ {\em depends on} entity type $F$ if
either $E$ directly depends on $F$, or $E$ directly depends on an entity type
that depends on $F$.   That is, the relation ``depends on'' is the transitive
closure of the relation ``directly depends on''.  

It is a static error if any entity type $E$ depends on itself.

\section{Limitations of Strict Typing}

X10's type checking provides substantial guarantees.  In most cases, a program
that passes the X10 type checker will not have any runtime type errors.
However, there are a modest number of compromises with practicality in the
type system: places where a program can pass the typechecker and still have a
type error.

\begin{enumerate}


\item The library type \xcd`IndexedMemoryChuck` provides a low-level interface
      to blocks of memory.  A few methods on that class are not type-safe. See
      the API if you must.

\item Custom serialization (\Sref{sect:ser+deser}) allows user code to
      construct new objects in ways that can subvert the type system.

\item Code written to use the underlying Java or C++ (\Sref{NativeCode}) can
      break X10's guarantees.

\end{enumerate}
	

\chapter{Variables}\label{XtenVariables}\index{variable}

%%OLDA variable is a storage location.  \Xten{} supports seven kinds of
%%OLDvariables: constant {\em class variables} (static variables), {\em
%%OLD  instance variables} (the instance fields of a class), {\em array
%%OLD  components}, {\em method parameters}, {\em constructor parameters},
%%OLD{\em exception-handler parameters} and {\em local variables}.

A {\em variable} is an X10 identifier associated with a value within some
context. Variable bindings have these essential properties:
\begin{itemize}
\item {\bf Type:} What sorts of values can be bound to the identifier;
\item {\bf Scope:} The region of code in which the identifier is associated
      with the entity;
\item {\bf Lifetime:} The interval of time in which the identifier is
      associated with the entity.
\item {\bf Visibility:} Which parts of the program can read or manipulate the
      value through the variable.
\end{itemize}



X10 has many varieties of variables, used for a number of purposes. 
\begin{itemize}
\item Class variables, also known as the static fields of a class, which hold
      their values for the lifetime of the class.  
\item Instance variables, which hold their values for the lifetime of an
      object;
\item Array elements, which are not individually named and hold their values
      for the lifetime of an array;
\item Formal parameters to methods, functions, and constructors, which hold
      their values for the duration of method (etc.) invocation;
\item Local variables, which hold their values for the duration of execution
      of a block.
\item Exception-handler parameters, which hold their values for the execution
      of the exception being handled. 
\end{itemize}
A few other kinds of things are called variables for historical reasons; \eg,
type parameters are often called type variables, despite not being variables
in this sense because they do not refer to X10 values.  Other named entities,
such as classes and methods, are not called variables.  However, all
name-to-whatever bindings enjoy similar concepts of scope and visibility.  

\begin{ex}
In the following example, 
\xcd`n` is an instance variable, and \xcd`nxt` is a
local variable defined within the method \xcd`bump`.\footnote{This code is
unnecessarily turgid for the sake of the example.  One would generally write
\xcd`public def bump() = ++n;`.   }
%~~gen ^^^ Vars10
% package Vars.For.Squares;
%~~vis
\begin{xten}
class Counter {
  private var n : Int = 0;
  public def bump() : Int {
    val nxt = n+1;
    n = nxt;
    return nxt;
    }
}
\end{xten}
%~~siv
% class Hook{ def run() { val c = new Counter(); val d = new Counter();
%   assert c.bump() == 1;  
%   assert c.bump() == 2;  
%   assert c.bump() == 3;  
%   assert c.bump() == 4;  
%   assert d.bump() == 1;  
%   assert c.bump() == 5;  
%   return true;
% } }
%~~neg
Both variables have type \xcd`Int` (or
perhaps something more specific).    The scope of \xcd`n` is the body of
\xcd`Counter`; the scope of \xcd`nxt` is the body of \xcd`bump`.  The
lifetime of \xcd`n` is the lifetime of the \xcd`Counter` object holding it;
the lifetime of \xcd`nxt` is the duration of the call to \xcd`bump`. Neither
variable can be seen from outside of its scope.
\end{ex}
\label{exploded-syntax}
\label{VariableDeclarations}
\index{variable declaration}


Variables whose value may not be changed after initialization are said to be
{\em immutable}, or {\em constants} (\Sref{FinalVariables}), or simply
\xcd`val` variables. Variables whose value may change are {\em mutable} or
simply \xcd`var` variables. \xcd`var` variables are declared by the \xcd`var`
keyword. \xcd`val` variables may be declared by the \xcd`val` keyword; when a
variable declaration does not include either \xcd`var` or \xcd`val`, it is
considered \xcd`val`. 

A variable---even a \xcd`val` -- can be declared in one statement, and then
initialized later on.  It must be initialized before it can be used
(\Sref{sect:DefiniteAssignment}).  


\begin{ex}
The following example illustrates many of the variations on variable
declaration: 
%~~gen ^^^ Vars20
%package Vars.For.Bears.In.Chairs;
%class VarExample{
%static def example() {
%~~vis
\begin{xten}
val a : Int = 0;               // Full 'val' syntax
b : Int = 0;                   // 'val' implied
val c = 0;                     // Type inferred
var d : Int = 0;               // Full 'var' syntax
var e : Int;                   // Not initialized
var f : Int{self != 100} = 0;  // Constrained type
val g : Int;                   // Init. deferred
if (a > b) g = 1; else g = 2;  // Init. done here.
\end{xten}
%~~siv
%}}
%~~neg
\end{ex}





\section{Immutable variables}
\label{FinalVariables}
\index{variable!immutable}
\index{immutable variable}
\index{variable!val}
\index{val}

%##(LocalVariableDeclarationStatement LocalVariableDeclaration
\begin{bbgrammar}
%(FROM #(prod:LocVarDeclnStmt)#)
     LocVarDeclnStmt \: LocVarDecln \xcd";" & (\ref{prod:LocVarDeclnStmt}) \\
%(FROM #(prod:LocVarDecln)#)
         LocVarDecln \: Mods\opt VarKeyword VariableDeclrs & (\ref{prod:LocVarDecln}) \\
                     \| Mods\opt VarDeclsWType \\
                     \| Mods\opt VarKeyword FormalDeclrs \\
\end{bbgrammar}
%##)

An immutable (\xcd`val`) variable can be given a value (by initialization or
assignment) at 
most once, and must be given a value before it is used.  Usually this is
achieved by declaring and initializing the variable in a single statement, 
such as \Xcd{val x = 3}, with syntax 
(\ref{prod:LocVarDecln}) using the {\it VariableDeclarators} or {\it
VarDelcsWType} alternatives.

\begin{ex}
After these declarations, \xcd`a` and \xcd`b` cannot be assigned to further,
or even redeclared:  
%~~gen ^^^ Vars30
% package Vars.In.Snares;
% class ABitTedious{
% def example() {
%~~vis
\begin{xten}
val a : Int = 10;
val b = (a+1)*(a-1);
// ERROR: a = 11;  // vals cannot be assigned to.
// ERROR: val a = 11; // no redeclaration.
\end{xten}
%~~siv
%}}
%~~neg

\end{ex}

In two special cases, the declaration and assignment are separate.  One 
case is how constructors give values to \xcd`val` fields of objects.  In this
case, production (\ref{prod:LocVarDecln}) is taken, with the {\it
FormalDeclarators} option, such as  \Xcd{var n:Int;}.  

\begin{ex} The
\xcd`Example` class has an immutable field \xcd`n`, which is given different
values depending on which constructor was called. \xcd`n` can't be given its
value by initialization when it is declared, since it is not knowable which
constructor is called at that point.  
%~~gen ^^^ Vars40
% package Vars.For.Cares;
%~~vis
\begin{xten}
class Example {
  val n : Int; // not initialized here
  def this() { n = 1; }
  def this(dummy:Boolean) { n = 2;}
}
\end{xten}
%~~siv
%
%~~neg
\end{ex}


The other case of separating declaration and assignment is in function
and method call, described in \Sref{sect:formal-parameters}.  The formal
parameters are bound to the corresponding actual 
parameters, but the binding does not happen until the function is called.  

\begin{ex}
In
the code below, \xcd`x` is initialized to \xcd`3` in the first call and
\xcd`4` in the second.
%~~gen ^^^ Vars50
%package Vars.For.Swears;
%class Examplement {
%static def whatever() {
%~~vis
\begin{xten}
val sq = (x:Int) => x*x;
x10.io.Console.OUT.println("3 squared = " + sq(3));
x10.io.Console.OUT.println("4 squared = " + sq(4));
\end{xten}
%~~siv
%}}
%~~neg
\end{ex}




%%IMMUTABLE%% An immutable variable satisfies two conditions: 
%%IMMUTABLE%% \begin{itemize}
%%IMMUTABLE%% \item it can be assigned to at most once, 
%%IMMUTABLE%% \item it must be assigned to before use. 
%%IMMUTABLE%% \end{itemize}
%%IMMUTABLE%% 
%%IMMUTABLE%% \Xten{} follows \java{} language rules in this respect \cite[\S
%%IMMUTABLE%% 4.5.4,8.3.1.2,16]{jls2}. Briefly, the compiler must undertake a
%%IMMUTABLE%% specific analysis to statically guarantee the two properties above.
%%IMMUTABLE%% 
%%IMMUTABLE%% Immutable local variables and fields are defined by the \xcd"val"
%%IMMUTABLE%% keyword.  Elements of value arrays are also immutable.
%%IMMUTABLE%% 
%%IMMUTABLE%% \oldtodo{Check if this analysis needs to be revisited.}

\section{Initial values of variables}
\label{NullaryConstructor}\index{nullary constructor}
\index{initial value}
\index{initialization}


Every assignment, binding, or initialization to a variable of type \xcd`T{c}`
must be an instance of type \xcd`T` satisfying the constraint \xcd`{c}`.
Variables must be given a value before they are used. This may be done by
initialization -- giving a variable a value as part
of its declaration. 

\begin{ex}
These variables are all initialized: 
%~~gen ^^^ Vars60
%package Vars.For.Bears;
%class VarsForBears{
%def check() {
%~~vis
\begin{xten}
  val immut : Int = 3;
  var mutab : Int = immut;
  val use = immut + mutab;
\end{xten}
%~~siv
%}}
%~~neg
\end{ex}

Or, a variable may be given a value by an assignment.  \xcd`var` variables may
be assigned to repeatedly.  \xcd`val` variables may only be assigned once; the
compiler will ensure that they are assigned before they are used.

\begin{ex}
The variables in the following example are given their initial values by
assignment.  Note that they could not be used before those assignments,
nor could \xcd`immu` be assigned repeatedly.
%~~gen ^^^ Vars70
%package Vars.For.Stars;
%abstract class VarsForStars{
% abstract def cointoss(): Boolean;
% abstract def println(Any):void;
%def check() {
%~~vis
\begin{xten}
  var muta : Int;
  // ERROR:  println(muta);
  muta = 4;
  val use2A = muta * 10;
  val immu : Int;
  // ERROR: println(immu);
  if (cointoss())   {immu = 1;}
  else              {immu = use2A;}
  val use2B = immu * 10;
  // ERROR: immu = 5;
\end{xten}
%~~siv
%}}
%~~neg
\end{ex}

Every class variable must be initialized before it is read, through
the execution of an explicit initializer. Every
instance variable must be initialized before it is read, through the
execution of an explicit or implicit initializer or a constructor.
Implicit initializers initialize \xcd`var`s to the default values of their
types (\Sref{DefaultValues}). Variables of types which do not have default
values are not implicitly initialized.



Each method and constructor parameter is initialized to the
corresponding argument value provided by the invoker of the method. An
exception-handling parameter is initialized to the object thrown by
the exception. A local variable must be explicitly given a value by
initialization or assignment, in a way that the compiler can verify
using the rules for definite assignment (\Sref{sect:DefiniteAssignment}).


\section{Destructuring syntax}
\index{variable declarator!destructuring}
\index{destructuring}
\Xten{} permits a \emph{destructuring} syntax for local variable
declarations with explicit initializers,  and for formal parameters, of type
\xcd`Point`, \Sref{point-syntax} and \xcd`Array`, \Sref{XtenArrays}.
A point is a sequence of zero or more \xcd`Int`-valued coordinates; an array
is an indexed collection of data. 
It is often useful to get at the coordinates or elements directly, in
variables.

%##(VariableDeclarator
\begin{bbgrammar}
%(FROM #(prod:VariableDeclr)#)
       VariableDeclr \: Id HasResultType\opt \xcd"=" VariableInitializer & (\ref{prod:VariableDeclr}) \\
                     \| \xcd"[" IdList \xcd"]" HasResultType\opt \xcd"=" VariableInitializer \\
                     \| Id \xcd"[" IdList \xcd"]" HasResultType\opt \xcd"=" VariableInitializer \\
\end{bbgrammar}
%##)

The syntax \xcdmath"val [a$_1$, $\ldots$, a$_n$] = e;", 
where \xcd`e` is a \xcd`Point`,
declares {$n$}
\xcd`Int` variables, bound to the precisely {$n$} components of the \xcd`Point` value of
\xcd`e`; it is an error if \xcd`e` is not a \xcd`Point` with precisely {$n$} components.
The syntax \xcdmath"val p[a$_1$, $\ldots$, a$_n$] = e;"  is similar, but also
declares the variable \xcd`p` to be of type \xcdmath"Point(n)".  


The syntax \xcdmath"val [a$_1$, $\ldots$, a$_n$] = e;", 
where \xcd`e` is an \xcd`Array[T]` for some type \xcd`T`,
declares {$n$}
variables of type \xcd`T`, bound to the precisely {$n$} components of the \xcd`Array[T]` value of
\xcd`e`; it is an error if \xcd`e` is not a \xcd`Array[T]` 
with \xcd`rank==1` and \xcdmath"size==$n$". 
The syntax \xcdmath"val p[a$_1$, $\ldots$, a$_n$] = e;"  is similar, but also
declares the variable \xcd`p` to be of type
\xcdmath"Array[T]{rank==1,size==n}".   


\begin{ex}
The following code makes an anonymous point with one coordinate \xcd`11`, and
binds \xcd`i` to \xcd`11`.  Then it makes a point with coordinates \xcd`22`
and \xcd`33`, binds \xcd`p` to that point, and \xcd`j` and \xcd`k` to \xcd`22`
and \xcd`33` respectively.
%~~gen ^^^ Vars80
% package Vars.For.Glares;
% class Example {
% static def example () {
%~~vis
\begin{xten}
val [i] : Point = Point.make(11);
assert i == 11;
val p[j,k] = Point.make(22,33);
assert j == 22 && k == 33;
val q[l,m] = [44,55] as Point; 
assert l == 44 && m == 55;
//ERROR: val [n] = p;
\end{xten}
%~~siv
%}}
% class Hook{ def run() {Example.example(); return true;}}
%~~neg

Destructuring is allowed wherever a \xcd`Point` or \xcd`Array[T]` variable is
declared, \eg, as the formal parameters of a method.
\begin{ex}
The methods below take a single argument each: a three-element point for
\xcd`example1`, a three-element array for \xcd`example2`.  The argument itself
is bound to \xcd`x` in both cases, and its elements are bound to \xcd`a`,
\xcd`b`, and \xcd`c`.  
%~~gen ^^^ Vars2e6j
% package Vars2e6j;
%class Example {
%~~vis
\begin{xten}
static def example1(x[a,b,c]:Point){}
static def example2(x[a,b,c]:Array[String]{rank==1,size==3}){}
\end{xten}
%~~siv
%}
%~~neg
\end{ex}

\end{ex}


\section{Formal parameters}
\label{sect:formal-parameters}
\index{formal parameter}
\index{parameter}


Formal parameters are the variables which hold values transmitted into a
method or function.  
They are always declared with a type.  (Type inference is not
available, because there is no single expression to deduce a type from.)
The variable name can be omitted if it is not to be used in the
scope of the declaration, as in the type of the method 
\xcd`static def main(Rail[String]):void` executed at the start of a program that
does not use its command-line arguments.

\xcd`var` and \xcd`val` behave just as they do for local
variables, \Sref{local-variables}.  In particular, the following \xcd`inc`
method is allowed, but, unlike some languages, does {\em not} increment its
actual parameter.  \xcd`inc(j)` creates a new local 
variable \xcd`i` for the method call, initializes \xcd`i` with the value of
\xcd`j`, increments \xcd`i`, and then returns.  \xcd`j` is never changed.
%~~gen ^^^ Vars100
% package Vars.For.Squares.Of.Mares;
% class Example {
%~~vis
\begin{xten}
static def inc(var i:Int) { i += 1; }
static def example() {
   var j : Int = 0;
   assert j == 0;
   inc(j);
   assert j == 0;
}
\end{xten}
%~~siv
%}
% class Hook{ def run() {Example.example(); return true;}}
%~~neg


\section{Local variables and Type Inference}
\label{local-variables}
\index{variable!local}
\index{local variable}
Local variables are declared in a limited scope, and, dynamically, keep their
values only for so long as the scope is being executed.  They may be \xcd`var`
or \xcd`val`.  
They may have 
initializer expressions: \xcd`var i:Int = 1;` introduces 
a variable \xcd`i` and initializes it to 1.
If the variable is immutable
(\xcd"val")
the type may be omitted and
inferred from the initializer type (\Sref{TypeInference}).

The variable declaration \xcd`val x:T=e;` confirms that \xcd`e`'s value is of
type \xcd`T`, and then introduces the variable \xcd`x` with type \xcd`T`.  For
example, consider a class \xcd`Tub` with a property \xcd`p`.
%~~gen ^^^ Vars_Tub
% package Vars.Local;
%~~vis
\begin{xten}
class Tub(p:Int){
  def this(pp:Int):Tub{self.p==pp} {property(pp);}
  def example() {
    val t : Tub = new Tub(3);
  }
}
\end{xten}
%~~siv
%
%~~neg
\noindent
produces a variable \xcd`t` of type \xcd`Tub`, even though the expression
\xcd`new Tub(3)` produces a value of type \xcd`Tub{self.p==3}` -- that is, a
\xcd`Tub`  whose \xcd`p` field is 3.  This can be inconvenient when the
constraint information is required.

\index{\Xcd{<:}}
\index{documentation type declaration}
Including type information in variable declarations is generally good
programming practice: it explains to both the compiler and human readers
something of the intent of the variable.  However, including types in 
\xcd`val t:T=e` can obliterate helpful information.  So, X10 allows a {\em
documentation type declaration}, written 
\begin{xtenmath}
val t <: T = e
\end{xtenmath}
This 
has the same effect as \xcd`val t = e`, giving \xcd`t` the full type inferred
from \xcd`e`; but it also confirms statically that that type is at least
\xcd`T`.  

\begin{ex}The following gives \xcd`t` the type \xcd`Tub{self.p==3}` as
desired.  However, a similar declaration with an inappropriate type will fail
to compile.
%~~gen ^^^ Vars_Var_Bounded
% package Vars.Local.not.the.express.plz;
% class Tub(p:Int){
%   def this(pp:Int):Tub{self.p==pp} {property(pp);}
%   def example() {
%     val t : Tub = new Tub(3);
%   }
% }
% class TubBounded{
% def example() {
%~~vis
\begin{xten}
   val t <: Tub = new Tub(3);
   // ERROR: val u <: Int = new Tub(3);
\end{xten}
%~~siv
%}}
%~~neg

\end{ex}



\section{Fields}
\index{field}
\index{object!field}
\index{struct!field}
\index{class!field}

%##(FieldDeclarators FieldDecln FieldDeclarator HasResultType  Mod
\begin{bbgrammar}
%(FROM #(prod:FieldDeclrs)#)
         FieldDeclrs \: FieldDeclr & (\ref{prod:FieldDeclrs}) \\
                     \| FieldDeclrs \xcd"," FieldDeclr \\
%(FROM #(prod:FieldDecln)#)
          FieldDecln \: Mods\opt VarKeyword FieldDeclrs \xcd";" & (\ref{prod:FieldDecln}) \\
                     \| Mods\opt FieldDeclrs \xcd";" \\
%(FROM #(prod:FieldDeclr)#)
          FieldDeclr \: Id HasResultType & (\ref{prod:FieldDeclr}) \\
                     \| Id HasResultType\opt \xcd"=" VariableInitializer \\
%(FROM #(prod:HasResultType)#)
       HasResultType \: ResultType & (\ref{prod:HasResultType}) \\
                     \| \xcd"<:" Type \\
%(FROM #(prod:Mod)#)
                 Mod \: \xcd"abstract" & (\ref{prod:Mod}) \\
                     \| Annotation \\
                     \| \xcd"atomic" \\
                     \| \xcd"final" \\
                     \| \xcd"native" \\
                     \| \xcd"private" \\
                     \| \xcd"protected" \\
                     \| \xcd"public" \\
                     \| \xcd"static" \\
                     \| \xcd"transient" \\
                     \| \xcd"clocked" \\
\end{bbgrammar}
%##)

Like most other kinds of variables in X10, 
the fields of an object can be either \xcd`val` or \xcd`var`. 
\xcd`val` fields can be \xcd`static` (\Sref{FieldDefinitions}).
Field declarations may have optional
initializer expressions, as for local variables, \Sref{local-variables}.
\xcd`var` fields without an initializer are initialized with the default value
of their type. \xcd`val` fields without an initializer must be initialized by
each constructor.


For \xcd`val` fields, as for \xcd`val` local variables, the type may be
omitted and inferred from the initializer type (\Sref{TypeInference}).
\xcd`var` fields, like \xcd`var` local variables, must be declared with a type.



%%GRAM%% \begin{grammar}
%%GRAM%% FieldDeclaration
%%GRAM%%         \: FieldModifier\star \xcd"var" FieldDeclaratorsWithType \\&& ( \xcd"," FieldDeclaratorsWithType )\star \\
%%GRAM%%         \| FieldModifier\star \xcd"val" FieldDeclarators \\&& ( \xcd"," FieldDeclarators )\star \\
%%GRAM%%         \| FieldModifier\star FieldDeclaratorsWithType \\&& ( \xcd"," FieldDeclaratorsWithType )\star \\
%%GRAM%% FieldDeclarators
%%GRAM%%         \: FieldDeclaratorsWithType \\
%%GRAM%%         \: FieldDeclaratorWithInit \\
%%GRAM%% FieldDeclaratorId
%%GRAM%%         \: Identifier  \\
%%GRAM%% FieldDeclaratorWithInit
%%GRAM%%         \: FieldDeclaratorId Init \\
%%GRAM%%         \| FieldDeclaratorId ResultType Init \\
%%GRAM%% FieldDeclaratorsWithType
%%GRAM%%         \: FieldDeclaratorId ( \xcd"," FieldDeclaratorId )\star ResultType \\
%%GRAM%% FieldModifier \: Annotation \\
%%GRAM%%                 \| \xcd"static" \\ \| \xcd`property` \\ \| \xcd`global` \\
%%GRAM%% \end{grammar}
%%GRAM%% 
%%GRAM%% 

%%ACC%%  \section{Accumulator Variables}
%%ACC%%  
%%ACC%%  Accumulator variables allow the accumulation of partial results to produce a
%%ACC%%  final result.  For example, an accumulator variable could compute a running
%%ACC%%  sum, product, maximum, or minimum of a collection of numbers.  In particular,
%%ACC%%  many concurrent activites can accumulate safely into the {\em same} local
%%ACC%%  variable, without need for \Xcd{atomic} blocks or other explicit coordination.  
%%ACC%%  
%%ACC%%  An accumulator variable is associated with a {\em reducer}, which explains how
%%ACC%%  new partial values are accumulated.
%%ACC%%  
%%ACC%%  \subsection{Reducers}
%%ACC%%  
%%ACC%%  A notion of accumulation has two aspects: 
%%ACC%%  \begin{enumerate}
%%ACC%%  \item A {\bf zero} value, which is the initial value of the accumulator,
%%ACC%%        before any partial results have been included.  When accumulating a sum,
%%ACC%%        the zero value is \Xcd{0}; when accumulating a product, it is \Xcd{1}.
%%ACC%%  \item A {\bf combining function}, explaining how to combine two partial
%%ACC%%        accumulations into a whole one.  When accumulating a sum, partial sums
%%ACC%%        should be added together; for a product, they should be multiplied.  
%%ACC%%  \end{enumerate}
%%ACC%%  
%%ACC%%  In X10, this is represented as a value of type
%%ACC%%  \Xcd{x10.lang.Reducer[T]}: 
%%ACC%%  %~acc~gen
%%ACC%%  %package Vars.Notx10lang.Reducerererer;
%%ACC%%  %~acc~vis
%%ACC%%  \begin{xten}
%%ACC%%  struct Reducer[T](zero:T, apply: (T,T)=>T){}
%%ACC%%  \end{xten}
%%ACC%%  %~acc~siv
%%ACC%%  %
%%ACC%%  %~acc~neg
%%ACC%%  \noindent 
%%ACC%%  If \Xcd{r:Reducer[T]}, then \Xcd{r.zero} is the zero element, and
%%ACC%%  \Xcd{r(a,b)} --- which can also be written \Xcd{r.apply(a,b)} --- is the
%%ACC%%  combination of \Xcd{a} and \Xcd{b}.
%%ACC%%  
%%ACC%%  For example, the reducers for adding and multiplying integers are: 
%%ACC%%  %~acc~gen
%%ACC%%  %package Vars.Notx10lang.Reducererererererer;
%%ACC%%  %struct Reducer[T](zero:T, apply: (T,T)=>T){}
%%ACC%%  %class Example{
%%ACC%%  %~acc~vis
%%ACC%%  \begin{xten}
%%ACC%%  val summer = Reducer[Int](0, Int.+);
%%ACC%%  val producter = Reducer[Int](1, Int.*);
%%ACC%%  \end{xten}
%%ACC%%  %~acc~siv
%%ACC%%  %}
%%ACC%%  %~acc~neg
%%ACC%%  
%%ACC%%  
%%ACC%%  Reduction is guaranteed to be deterministic if the reducer is {\em
%%ACC%%  Abelian},\footnote{This term is borrowed from abstract algebra, where such a
%%ACC%%  reducer, together with its type, forms an Abelian monoid.}
%%ACC%%  that is, 
%%ACC%%  \begin{enumerate}
%%ACC%%  \item \Xcd{r.apply} is pure; that is, has no side effects;
%%ACC%%  \item \Xcd{r.apply} is commutative; that is, \Xcd{r(a,b) == r(b,a)} for all
%%ACC%%        inputs \Xcd{a} and \Xcd{b};
%%ACC%%  \item \Xcd{r.apply} is associative; that is, 
%%ACC%%        \Xcd{r(a,r(b,c)) == r(r(a,b),c)} for all \Xcd{a}, \Xcd{b}, and \Xcd{c}.
%%ACC%%  \item \Xcd{r.zero} is the identity element for \Xcd{r.apply}; that is, 
%%ACC%%        \Xcd{r(a, r.zero) == a}
%%ACC%%        for all \Xcd{a}.
%%ACC%%  \end{enumerate}
%%ACC%%  
%%ACC%%  
%%ACC%%  
%%ACC%%  
%%ACC%%  \Xcd{summer} and \Xcd{producter} satisfy all these conditions, and give
%%ACC%%  determinate reductions. The compiler does not require or check these, though.
%%ACC%%  
%%ACC%%  
%%ACC%%  \subsection{Accumulators}
%%ACC%%  
%%ACC%%  If \Xcd{r} is a  value of type \Xcd{Reducer[T]}, then an accumulator of type
%%ACC%%  \Xcd{T} using \Xcd{r} is declared as:
%%ACC%%  %~accTODO~gen
%%ACC%%  % package Vars.Accumulators.Basic.Little.Idea;
%%ACC%%  % class C[T]{
%%ACC%%  % static def example (r:Reducer[T]) {
%%ACC%%  %~accTODO~vis
%%ACC%%  \begin{xten}
%%ACC%%  acc(r) x : T;
%%ACC%%  acc(r) y; 
%%ACC%%  \end{xten}
%%ACC%%  %~accTODO~siv
%%ACC%%  %
%%ACC%%  %~accTODO~neg
%%ACC%%  The type declaration \Xcd{T} is optional; if specified, it must be the same
%%ACC%%  type that the reducer \Xcd{r} uses.
%%ACC%%  
%%ACC%%  \subsection{Sequential Use of Accumulators}
%%ACC%%  
%%ACC%%  The sequential use of accumulator variables is straightforward, and could be
%%ACC%%  done as easily without accumulators.  (The power of accumulators is in their
%%ACC%%  concurrent use, \Sref{ConcurrentUseOfAccumulators}.)
%%ACC%%  
%%ACC%%  A variable declared as \Xcd{acc(r) x:T;} is initialized to \Xcd{r.zero}.  
%%ACC%%  
%%ACC%%  Assignment of values of \Xcd{acc} variables has nonstandard semantics.
%%ACC%%  \Xcd{x = v;} causes the value \Xcd{r(v,x)} to be stored in \Xcd{x} --- in
%%ACC%%  particular, {\em not} the value of \Xcd{v}.
%%ACC%%  
%%ACC%%  Reading a value from an accumulator retrieves the current accumulation.
%%ACC%%  
%%ACC%%  For example, the sum and product of a list \Xcd{L} of integers can be computed
%%ACC%%  by: 
%%ACC%%  %~accTODO~gen
%%ACC%%  %package Vars.Accumulators.Are.For.Bisimulators;
%%ACC%%  % import java.util.*;
%%ACC%%  % class Example{
%%ACC%%  % static def example(L: List[Int]) {
%%ACC%%  %~accTODO~vis
%%ACC%%  \begin{xten}
%%ACC%%  val summer = Reducer[Int](0, Int.+);
%%ACC%%  val producter = Reducer[Int](1, Int.*);
%%ACC%%  acc(summer) sum;
%%ACC%%  acc(producter) prod;
%%ACC%%  for (x in L) {
%%ACC%%    sum = x;
%%ACC%%    prod = x;
%%ACC%%  }
%%ACC%%  x10.io.Console.OUT.println("Sum = " + sum + "; Product = " + prod);
%%ACC%%  \end{xten}
%%ACC%%  %~accTODO~siv
%%ACC%%  %
%%ACC%%  %~accTODO~neg
%%ACC%%  
%%ACC%%  
%%ACC%%  
%%ACC%%  \subsection{Concurrent Use of Accumulators}
%%ACC%%  \label{ConcurrentUseOfAccumulators}
%%ACC%%  \index{accumulator!and activities}
%%ACC%%  
%%ACC%%  Accumulator variables are restricted and synchronized in ways that make them
%%ACC%%  ideally suited for concurrent accumulation of data.   The {\em governing
%%ACC%%  activity} of an accumulator is the activity in which the \Xcd{acc} variable is
%%ACC%%  declared.  
%%ACC%%  
%%ACC%%  \begin{enumerate}
%%ACC%%  \item The governing activity can read the accumulator at any point that it has
%%ACC%%        no running sub-activities.  
%%ACC%%  \item Any activity that has lexical access to the accumulator can write to it.  
%%ACC%%        All
%%ACC%%        writes are performed atomically, without need for \Xcd{atomic} or other
%%ACC%%        concurrency control.
%%ACC%%  \end{enumerate}
%%ACC%%  
%%ACC%%  If the reducer is Abelian, this guarantees that \Xcd{acc} variables cannot
%%ACC%%  cause race conditions; the result of such a computation is determinate,
%%ACC%%  independent of the scheduling of activities. Read-read conflicts are
%%ACC%%  impossible, as only a single activity, the governing activity, can read the
%%ACC%%  \Xcd{acc} variable. Read-write conflicts are impossible, as reads are only
%%ACC%%  allowed at points where the only activity which can refer to the \Xcd{acc}
%%ACC%%  variable is the governing activity. Two activities may try to write the
%%ACC%%  \Xcd{acc} variable at the same time. The writes are performed atomically, so
%%ACC%%  they behave as if they happened in some (arbitrary) order---and, because the
%%ACC%%  reducer is Abelian, the order of writes doesn't matter.
%%ACC%%  
%%ACC%%  If the reducer is not Abelian---\eg, it is accumulating a string result by
%%ACC%%  concatenating a lot of partial strings together---the result is indeterminate.
%%ACC%%  However, because the accumulator operations are atomic, it will be the result
%%ACC%%  of {\em some} combination of the individual elements by the reduction
%%ACC%%  operation, \eg, the concatenation of the partial strings in {\em some} order.  
%%ACC%%  
%%ACC%%  
%%ACC%%  
%%ACC%%  For example, the following code computes triangle numbers {$\sum_{i=1}^{n}i$}
%%ACC%%  concurrently.\footnote{This program is highly inefficient. Even ignoring the
%%ACC%%    constant-time formula {$\sum_{i=1}^{n}i = \frac{n(n+1)}{2}$}, this program
%%ACC%%    incurs the cost of starting {$n$} activities and coordinating {$n$} accesses
%%ACC%%    to the accumulator. Accumulator variables are of most value in multi-place,
%%ACC%%    multi-core computations.}
%%ACC%%  
%%ACC%%  
%%ACC%%  %~accTODO~gen
%%ACC%%  %package Vars.Accumulator.Concurrency.Example;
%%ACC%%  %class Example{
%%ACC%%  %
%%ACC%%  %~accTODO~vis
%%ACC%%  \begin{xten}
%%ACC%%  def triangle(n:Int) {
%%ACC%%    val summer = Reducer[Int](0, Int.+);
%%ACC%%    acc(summer) sum; 
%%ACC%%    finish {
%%ACC%%      for([i] in 1..n) async {
%%ACC%%        sum = i;  // (A)
%%ACC%%      }
%%ACC%%      // (C)
%%ACC%%    }
%%ACC%%    return sum; // (B)
%%ACC%%  }
%%ACC%%  \end{xten}
%%ACC%%  %~accTODO~siv
%%ACC%%  %}
%%ACC%%  %~accTODO~neg
%%ACC%%  
%%ACC%%  The governing activity of the \Xcd{acc} variable \Xcd{sum} is the activity
%%ACC%%  including the body of \Xcd{triangle}.  It starts up \Xcd{n} sub-activities,
%%ACC%%  each of which adds one value to \Xcd{sum} at point \Xcd{(A)}.  Note that these
%%ACC%%  activities cannot {\em read} the value of \Xcd{sum}---only the governing
%%ACC%%  activity can do that---but they can update it.  
%%ACC%%  
%%ACC%%  At point \Xcd{(B)}, \Xcd{triangle} returns the value in \Xcd{sum}. It is
%%ACC%%  clear, from the \Xcd{finish} statement, that all sub-activities started by the
%%ACC%%  governing process have finished at this point. X10 forbids reading of
%%ACC%%  \Xcd{sum}, even by the governing process, at point \Xcd{(C)}, since
%%ACC%%  sub-activities writing into it could still be active when the governing
%%ACC%%  activity reaches this point.  The \Xcd{return sum;} statement could not be
%%ACC%%  moved to \Xcd{(C)}, which is good, because the program would be wrong if it
%%ACC%%  were there.
%%ACC%%  
%%ACC%%  
%%ACC%%  
%%ACC%%  
%%ACC%%  \subsubsection{Accumulators and Places}
%%ACC%%  \index{accumulator!and places} Activity variables can be read and written from
%%ACC%%  any place, without need for \Xcd{GlobalRef}s. We may spread the previous
%%ACC%%  computation out among all the available processors by simply sticking in an
%%ACC%%  \Xcd{at(...)} statement at point \Xcd{(D)}, without need for other
%%ACC%%  restructuring of the program.
%%ACC%%  
%%ACC%%  %~accTODO~gen
%%ACC%%  %package Vars.Accumulator.Concurrency.Example.Multiplacey;
%%ACC%%  %class Example{
%%ACC%%  %~accTODO~vis
%%ACC%%  \begin{xten}
%%ACC%%  def triangle(n:Int) {
%%ACC%%    val summer = Reducer[Int](0, Int.+);
%%ACC%%    acc(summer) sum; 
%%ACC%%    finish {
%%ACC%%      for([i] in 1..n) async 
%%ACC%%        at(Places.place(i % Places.MAX_PLACES) { //(D)
%%ACC%%          sum = i;  // (A)
%%ACC%%      }
%%ACC%%    }
%%ACC%%    return sum; // (B)
%%ACC%%  }
%%ACC%%  \end{xten}
%%ACC%%  %~accTODO~siv
%%ACC%%  %}
%%ACC%%  %~accTODO~neg
%%ACC%%  
%%ACC%%  \subsubsection{Accumulator Parameters}
%%ACC%%  \index{accumulator variables!as parameters}
%%ACC%%  \index{parameters!accumulator}
%%ACC%%  
%%ACC%%  Accumulators can be passed to methods and closures, by giving the keyword 
%%ACC%%  \Xcd{acc} instead of \Xcd{var} or \Xcd{val}.  Reducers are not specified; each
%%ACC%%  accumulator comes with its own reducer.  However, the type \Xcd{T} of the
%%ACC%%  accumulator {\em is} required.
%%ACC%%  
%%ACC%%  For example, the following method takes a list of numbers, and accumulates
%%ACC%%  those that are divisible by 2 in \Xcd{evens}, and those that are divisible by
%%ACC%%  3 in \Xcd{triples}: 
%%ACC%%  %~accTODO~gen
%%ACC%%  %package Vars.accumulators.parameters.oscillators.convulsitors.proximators;
%%ACC%%  %import x10.util.*;
%%ACC%%  %class Whatever {
%%ACC%%  %~accTODO~vis
%%ACC%%  \begin{xten}
%%ACC%%  static def split23(L:List[Int], acc evens:Int, acc triples:Int) {
%%ACC%%    for(n in L) {
%%ACC%%       if (n % 2 == 0) evens = n;
%%ACC%%       if (n % 3 == 0) triples = n;
%%ACC%%    }
%%ACC%%  }
%%ACC%%  static val summer = Reducer[Int](0, Int.+);
%%ACC%%  static val producter = Reducer[Int](1, Int.*);
%%ACC%%  static def sumEvenPlusProdTriple(L:List[Int]) {
%%ACC%%    acc(summer) sumEven;
%%ACC%%    acc(producter) prodTriple;
%%ACC%%    split23(L, sumEven, prodTriple);
%%ACC%%    return sumEven + prodTriple;
%%ACC%%  }
%%ACC%%  \end{xten}
%%ACC%%  %~accTODO~siv
%%ACC%%  %}
%%ACC%%  %~accTODO~neg
%%ACC%%  
%%ACC%%  \subsection{Indexed Accumulators}
%%ACC%%  \index{accumulator!indexed}
%%ACC%%  \index{accumulator!array}
%%ACC%%  
%%ACC%%  
%%ACC%%  \noo{Define this!}
%%ACC%%  
%%ACC%%  %~accTODO~gen
%%ACC%%  % package Vars.Indexed.Accumulators;
%%ACC%%  %~accTODO~vis
%%ACC%%  \begin{xten}
%%ACC%%  class BoolAccum implements SelfAccumulator[Boolean, Int] {
%%ACC%%    var sumTrue = 0, sumFalse = 0;
%%ACC%%    def update(k:Boolean, v:Int) { 
%%ACC%%       if (k) sumTrue += k; else sumFalse += k;
%%ACC%%    }
%%ACC%%    def update(ks:Array[Boolean]{rail}, vs:Array[Int]{ks.size == vs.size}) {
%%ACC%%       for([i] in ks.region) update(ks(i), vs(i));  }
%%ACC%%    
%%ACC%%  }
%%ACC%%  \end{xten}
%%ACC%%  %~accTODO~siv
%%ACC%%  %
%%ACC%%  %~accTODO~neg

\chapter{Names and packages}
\label{packages} \index{names}\index{packages}\index{public}\index{protected}\index{private}

\Xten{} supports mechanisms for names and packages in the style of Java
\cite[\S 6,\S 7]{jls2}, including \xcd"public", \xcd"protected", \xcd"private"
and package-specific access control.

\section{Packages}

A package is a named collection of top-level type declarations, \viz, class,
interface, and struct declarations. Package names are sequences of
identifiers, like \xcd`x10.lang` and \xcd`com.ibm.museum`. The multiple names
are simply a convenience. Packages \xcd`a`, \xcd`a.b`, and \xcd`a.c` have only
a very tenuous relationship, despite the similarity of their names.

Packages and protection modifiers determine which top-level names can be used
where. Only the \xcd`public` members of package \xcd`pack.age` can be accessed
outside of \xcd`pack.age` itself.  
%~~gen~~Stimulus
%
%~~vis
\begin{xten}
package pack.age;
class Deal {
  public def make() {}
}
public class Stimulus {
  private def taxCut() = true;
  protected def benefits() = true;
  public def jobCreation() = true;
  /*package*/ def jumpstart() = true;
}
\end{xten}
%~~siv
%
%~~neg

The class \xcd`Stimulus` can be referred to from anywhere outside of
\xcd`pack.age` by its full name of \xcd`pack.age.Stimulus`, or can be imported
and referred to simply as \xcd`Stimulus`.  The public \xcd`jobCreation()`
method of a \xcd`Stimulus` can be referred to from anywhere as well; the other
methods have smaller visibility.  The non-\xcd`public` class \xcd`Deal` cannot
be used from outside of \xcd`pack.age`.  



\subsection{Name Collisions}

It is a static error for a package to have two members, or apparent members,
with the same name.  For example, package \xcd`pack.age` cannot define two
classes both named \xcd`Crash`, nor a class and an interface with that name.

Furthermore, \xcd`pack.age` cannot define a member \xcd`Crash` if there is
another package named \xcd`pack.age.Crash`, nor vice-versa. (This prohibition
is the only actual relationship between the two packages.)  This prevents the
ambiguity of whether \xcd`pack.age.Crash` refers to the class or the package.  
Note that the naming convention that package names are lower-case and package
members are capitalized prevents such collisions.


\section{\xcd`import` Declarations}

Any public member of a package can be referred to from anywhere through a
fully-qualified name: \xcd`pack.age.Stimulus`.    

Often, this is too awkward.  X10 has two ways to allow code outside of a class
to refer to the class by its short name (\xcd`Stimulus`): single-type imports
and on-demand imports.   

Imports of either kind appear at the start of the file, immediately after the
\xcd`package` directive if there is one; their scope is the whole file.

\subsection{Single-Type Import}

The declaration \xcd`import ` {\em TypeName} \xcd`;` imports a single type
into the current namespace.  The type it imports must be a fully-qualified
name of an extant type, and it must either be in the same package (in which
case the \xcd`import` is redundant) or be declared \xcd`public`.  

Furthermore, when importing \xcd`pack.age.T`, there must not be another type
named \xcd`T` at that point: neither a  \xcd`T` declared in \xcd`pack.age`,
nor a \xcd`inst.ant.T` imported from some other package.

\subsection{Automatic Import}

The automatic import \xcd`import pack.age.*;`, loosely, imports all the public
members of \xcd`pack.age`.  In fact, it does so somewhat carefully, avoiding
certain errors that could occur if it were done naively.  Types defined in the
current package, and those imported by single-type imports, shadow those
imported by automatic imports.  

\subsection{Implicit Imports}

The packages \xcd`x10.lang` and \xcd`x10.array` are imported in all files
without need for further specification.

%%BARD-HERE



\section{Conventions on Type Names}

\begin{grammar}
TypeName   \: Identifier \\
        \| TypeName \xcd"." Identifier \\
        \| PackageName \xcd"." Identifier \\
PackageName   \: Identifier \\
        \| PackageName \xcd"." Identifier \\
\end{grammar}


While not enforced by the compiler, classes and interfaces
in the \Xten{} library follow the following naming conventions.
Names of types---including classes,
type parameters, and types specified by type definitions---are in
CamelCase and begin with an uppercase letter.  (Type variables are often
single capital letters, such as \xcd`T`.)
For backward
compatibility with languages such as C and \java{}, type
definitions are provided to allow primitive types
such as \xcd"int" and \xcd"boolean" to be written in lowercase.
Names of methods, fields, value properties, and packages are in camelCase and
begin with a lowercase letter. 
Names of \xcd"static val" fields are in all uppercase with words
separated by `\xcd"_"''s.



\chapter{Interfaces}
\label{XtenInterfaces}\index{interface}

An interface specifies signatures for zero or more public methods, property
methods,
\xcd`static val`s, 
classes, structs, interfaces, types
and an invariant. 

The following puny example illustrates all these features: 
% TODO Well, it would if there weren't a compiler bug in the way.
%~~gen ^^^Interfaces_static_val
% package Interfaces_static_val;
% 
%~~vis
\begin{xten}
interface Pushable{prio() != 0} {
  def push(): void;
  static val MAX_PRIO = 100;
  abstract class Pushedness{}
  struct Pushy{}
  interface Pushing{}
  static type Shove = Int;
  property text():String;
  property prio():Int;
}
class MessageButton(text:String)
  implements Pushable{self.prio()==Pushable.MAX_PRIO} {
  public def push() { 
    x10.io.Console.OUT.println(text + " pushed");
  }
  public property text() = text;
  public property prio() = Pushable.MAX_PRIO;
}
\end{xten}
%~~siv
%
%~~neg
\noindent
\xcd`Pushable` defines two property methods, one normal method, and a static
value.  It also 
establishes an invariant, that \xcd`prio() != 0`. 
\xcd`MessageButton` implements a constrained version of \xcd`Pushable`,
\viz\ one with maximum priority.  It
defines the \xcd`push()` method given in the interface, as a \xcd`public`
method---interface methods are implicitly \xcd`public`.

\limitation{X10 may not always detect that type invariants of interfaces are
satisfied, even when they obviously are.}
%% TODO - is this a JIRA?  

A container---a class or struct---can {\em implement} an interface,
typically by having all the methods and property methods that the interface
requires, and by providing a suitable \xcd`implements` clause in its definition.

A variable may be declared to be of interface type.  Such a variable has all
the property and normal methods declared (directly or indirectly) by the
interface; 
nothing else is statically available.  Values of any concrete type which
implement the interface may be stored in the variable.  

\begin{ex}
The following code puts two quite different objects into the variable
\xcd`star`, both of which satisfy the interface \xcd`Star`.
%~~gen ^^^ Interfaces6l3f
% package Interfaces6l3f;
%~~vis
\begin{xten}
interface Star { def rise():void; }
class AlphaCentauri implements Star {
   public def rise() {}
}
class ElvisPresley implements Star {
   public def rise() {}
}
class Example {
   static def example() {
      var star : Star;
      star = new AlphaCentauri();
      star.rise();
      star = new ElvisPresley();
      star.rise();
   }
}
\end{xten}
%~~siv
%
%~~neg
\end{ex}
An interface may extend several interfaces, giving
X10 a large fraction of the power of multiple inheritance at a tiny fraction
of the cost.

\begin{ex}
%~~gen ^^^ Interfaces6g4u
% package Interfaces6g4u;
%~~vis
\begin{xten}
interface Star{}
interface Dog{}
class Sirius implements Dog, Star{}
class Lassie implements Dog, Star{}
\end{xten}
%~~siv
%
%~~neg
\end{ex}


\section{Interface Syntax}

\label{DepType:Interface}

%##(NormalInterfaceDecl TypeParamsI TypeParamI Guard ExtendsInterfaces InterfaceBody InterfaceMemberDecl
\begin{bbgrammar}
%(FROM #(prod:NormalInterfaceDecl)#)
 NormalInterfaceDecl \: Mods\opt \xcd"interface" Id TypeParamsI\opt Properties\opt Guard\opt ExtendsInterfaces\opt InterfaceBody & (\ref{prod:NormalInterfaceDecl}) \\
%(FROM #(prod:TypeParamsI)#)
         TypeParamsI \: \xcd"[" TypeParamIList \xcd"]" & (\ref{prod:TypeParamsI}) \\
%(FROM #(prod:TypeParamI)#)
          TypeParamI \: Id & (\ref{prod:TypeParamI}) \\
                     \| \xcd"+" Id \\
                     \| \xcd"-" Id \\
%(FROM #(prod:Guard)#)
               Guard \: DepParams & (\ref{prod:Guard}) \\
%(FROM #(prod:ExtendsInterfaces)#)
   ExtendsInterfaces \: \xcd"extends" Type & (\ref{prod:ExtendsInterfaces}) \\
                     \| ExtendsInterfaces \xcd"," Type \\
%(FROM #(prod:InterfaceBody)#)
       InterfaceBody \: \xcd"{" InterfaceMemberDecls\opt \xcd"}" & (\ref{prod:InterfaceBody}) \\
%(FROM #(prod:InterfaceMemberDecl)#)
 InterfaceMemberDecl \: MethodDecl & (\ref{prod:InterfaceMemberDecl}) \\
                     \| PropertyMethodDecl \\
                     \| FieldDecl \\
                     \| ClassDecl \\
                     \| InterfaceDecl \\
                     \| TypeDefDecl \\
                     \| \xcd";" \\
\end{bbgrammar}
%##)


\noindent
The invariant associated with an interface is the conjunction of the
invariants associated with its superinterfaces and the invariant
defined at the interface. 



A class \xcd"C"  implements an interface \xcd"I" if \xcd`I`, or a subtype of \xcd`I`, appears in the \xcd`implements` list
of \xcd`C`.  
In this case,
 \xcd`C` implicitly gets all the methods and property methods of \xcd`I`,
      as \xcd`abstract` \xcd`public` methods.  If \xcd`C` does not declare
      them explicitly, then they are \xcd`abstract`, and \xcd`C` must be
      \xcd`abstract` as well.   If \xcd`C` does declare them all, \xcd`C` may
      be concrete.



If \xcd`C` implements \xcd`I`, then the class invariant
(\Sref{DepType:ClassGuardDef}) for \xcd`C`,   $\mathit{inv}($\xcd"C"$)$, implies
the class invariant for \xcd`I`, $\mathit{inv}($\xcd"I"$)$.  That is, if the
interface \xcd`I` specifies some requirement, then every class \xcd`C` that
implements it satisfies that requirement.

\section{Access to Members}

All interface members are \xcd`public`, whether or not they are declared
public.  There is little purpose to non-public methods of an interface; they
would specify that implementing classes and structs have methods that cannot
be seen.

\section{Property Methods}

An interface may declare \xcd`property` methods.  All non-\xcd`abstract`
containers implementing such an interface must provide all the \xcd`property`
methods specified.  

\section{Field Definitions}
\index{interface!field definition in}

An interface may declare a \xcd`val` field, with a value.  This field is implicitly
\xcd`public static val`.  In particular, it is {\em not} an instance field. 
%~~gen ^^^ Interfaces10
% package Interface.Field;
%~~vis
\begin{xten}
interface KnowsPi {
  PI = 3.14159265358;
}
\end{xten}
%~~siv
%
%~~neg

Classes and structs implementing such an interface get the interface's fields as
\xcd`public static` fields.  Unlike  methods, there is no need
for the implementing class to declare them. 
%~~gen ^^^ Interfaces20
% package Interface.Field.Two;
% interface KnowsPi {PI = 3.14159265358;}
%~~vis
\begin{xten}
class Circle implements KnowsPi {
  static def area(r:Double) = PI * r * r;
}
class UsesPi {
  def circumf(r:Double) = 2 * r * KnowsPi.PI;
}
\end{xten}
%~~siv
%
%~~neg

\subsection{Fine Points of Fields}

If two parent interfaces give different static fields of the same name, 
those fields must be referred to by qualified names.
%~~gen ^^^ Interface_field_name_collision
% 
%~~vis
\begin{xten}
interface E1 {static val a = 1;}
interface E2 {static val a = 2;}
interface E3 extends E1, E2{}
class Example implements E3 {
  def example() = E1.a + E2.a;
}
\end{xten}
%~~siv
%
%~~neg

If the {\em same} field \xcd`a` is inherited through many paths, there is no need to
disambiguate it:
%~~gen ^^^ Interfaces_multi
% package Interfaces.Mult.Inher.Field;
%~~vis
\begin{xten}
interface I1 { static val a = 1;} 
interface I2 extends I1 {}
interface I3 extends I1 {}
interface I4 extends I2,I3 {}
class Example implements I4 {
  def example() = a;
}
\end{xten}
%~~siv
%
%~~neg

The initializer of a field in an interface may be any expression.  It is
evaluated under the same rules as a \xcd`static` field of a class. 

\begin{ex}
In this example, a class \xcd`TheOne` is defined,
with an inner interface \xcd`WelshOrFrench`, whose field \xcd`UN` (named in
either Welsh or French) has value 1.  Note that \xcd`WelshOrFrench` does not
define any methods, so it can be trivially added to the \xcd`implements`
clause of any class, as it is for \xcd`Onesome`. 
This allows the body of \xcd`Onesome` to use \xcd`UN` through an unqualified
name, as is done in \xcd`example()`.

%~~gen ^^^ Interfaces3l4a
% package Interfaces3l4a;
%~~vis
\begin{xten}
class TheOne {
  static val ONE = 1;
  interface WelshOrFrench {
    val UN = 1;
  }
  static class Onesome implements WelshOrFrench {
    static def example() {
      assert UN == ONE;
    }
  }
}
\end{xten}
%~~siv
% class Hook{ def run() {TheOne.Onesome.example(); return true;}}
%~~neg
\end{ex}

\section{Generic Interfaces}

Interfaces, like classes and structs, can have type parameters.  
The discussion of generics in \Sref{TypeParameters} applies to interfaces,
without modification.

\begin{ex}
%~~gen ^^^ Interfaces7n1z
% package Interfaces7n1z;
%~~vis
\begin{xten}
interface ListOfFuns[T,U] extends x10.util.List[(T)=>U] {}
\end{xten}
%~~siv
%
%~~neg

\end{ex}

\section{Interface Inheritance}

The {\em direct superinterfaces} of a non-generic interface \xcd`I` are the interfaces
(if any) mentioned in the \xcd`extends` clause of \xcd`I`'s definition.
If \xcd`I`  is generic, the direct superinterfaces are of an instantiation of
\xcd`I` are the corresponding instantiations of those interfaces.
A {\em superinterface} of \xcd`I` is either \xcd`I` itself, or a direct
superinterface of a superinterface of \xcd`I`, and similarly for generic
interfaces.    

\xcd`I` inherits the members of all of its superinterfaces. Any class or
struct that has \xcd`I` in its \xcd`implements` clause also implements all of
\xcd`I`'s superinterfaces. 






\section{Members of an Interface}

The members of an interface \xcd`I` are the union of the following sets: 
\begin{enumerate}
\item All of the members appearing in \xcd`I`'s declaration;
\item All the members of its direct super-interfaces, except those which are
      hidden (\Sref{sect:Hiding}) by \xcd`I`
\item The members of \xcd`Any`.
\end{enumerate}

Overriding for instances is defined as for classes, \Sref{MethodOverload}



\chapter{Classes}
\label{XtenClasses}\index{class}
\label{ReferenceClasses}





\section{Principles of X10 Objects}\label{XtenObjects}\index{object}
\index{class}

\subsection{Basic Design}

Objects are instances of classes: the most common and most powerful sort of
value in X10.  The other kinds of values, structs and functions, are more
specialized, better in some circumstances but not in all.

Classes are structured in a forest of single-inheritance code
hierarchies. Like C++, but unlike Java, there is no single root
class (\Xcd{java.lang.Object}) that all classes inherit from.  Classes
may have any or all of these features: 
\begin{itemize}
\item Implementing any number of interfaces;
\item Static and instance \xcd`val` fields; 
\item Instance \xcd`var` fields; 
\item Static and instance methods;
\item Constructors;
\item Properties;
\item Static and instance nested containers.
\item Static type definitions
\end{itemize}


\Xten{} objects (unlike Java objects) do not have locks associated with them.
Programmers may use atomic blocks (\Sref{AtomicBlocks}) for mutual
exclusion and clocks (\Sref{XtenClocks}) for sequencing multiple parallel
operations.

An object exists in a single location: the place that it was created.  One
place cannot use or even directly refer to an object in a different place.   A
special type, \Xcd{GlobalRef[T]}, allows explicit cross-place references. 

The basic operations on objects are:
\begin{itemize}

\item Construction (\Sref{ObjectInitialization}).  Objects are created, 
      their \xcd`var` and \xcd`val` fields initialized, and other invariants
      established.

\item Field access (\Sref{FieldAccess}). 
The static, instance, and property fields of an object can be retrieved; \xcd`var` fields
can be set.  

\item Method invocation (\Sref{MethodInvocation}).  
Static, instance, and property methods of an object can be invoked.

\item Casting (\Sref{ClassCast}) and instance testing with \xcd`instanceof`
(\Sref{instanceOf}) Objects can be cast or type-tested.  

\item The equality operators \xcd"==" and \xcd"!=".  
Objects can be compared for equality with the \Xcd{==} operation.  This checks
object {\em identity}: two objects are \Xcd{==} iff they are the same object.

\end{itemize}

  

\subsection{Class Declaration Syntax}
\label{sect:ClassDeclSyntax}

The {\em class declaration} has a list of type parameters, a list of
properties, a constraint (the {\em class invariant}), zero or one superclass,
zero or more interfaces that it implements, and a class body containing the
the definition of fields, properties, methods, and member types. Each such
declaration introduces a class type (\Sref{ReferenceTypes}).

%##(ClassDecln TypeParamsI TypeParamIList Properties PropertyList Property Guard Super Interfaces InterfaceTypeList ClassBody ClassMemberDeclns ClassMemberDecln
\begin{bbgrammar}
%(FROM #(prod:ClassDecln)#)
          ClassDecln \: Mods\opt \xcd"class" Id TypeParamsI\opt Properties\opt Guard\opt Super\opt Interfaces\opt ClassBody & (\ref{prod:ClassDecln}) \\
%(FROM #(prod:TypeParamsI)#)
         TypeParamsI \: \xcd"[" TypeParamIList \xcd"]" & (\ref{prod:TypeParamsI}) \\
%(FROM #(prod:TypeParamIList)#)
      TypeParamIList \: TypeParam & (\ref{prod:TypeParamIList}) \\
                     \| TypeParamIList \xcd"," TypeParam \\
                     \| TypeParamIList \xcd"," \\
%(FROM #(prod:Properties)#)
          Properties \: \xcd"(" PropList \xcd")" & (\ref{prod:Properties}) \\
%(FROM #(prod:PropList)#)
            PropList \: Prop & (\ref{prod:PropList}) \\
                     \| PropList \xcd"," Prop \\
%(FROM #(prod:Prop)#)
                Prop \: Annotations\opt Id ResultType & (\ref{prod:Prop}) \\
%(FROM #(prod:Guard)#)
               Guard \: DepParams & (\ref{prod:Guard}) \\
%(FROM #(prod:Super)#)
               Super \: \xcd"extends" ClassType & (\ref{prod:Super}) \\
%(FROM #(prod:Interfaces)#)
          Interfaces \: \xcd"implements" InterfaceTypeList & (\ref{prod:Interfaces}) \\
%(FROM #(prod:InterfaceTypeList)#)
   InterfaceTypeList \: Type & (\ref{prod:InterfaceTypeList}) \\
                     \| InterfaceTypeList \xcd"," Type \\
%(FROM #(prod:ClassBody)#)
           ClassBody \: \xcd"{" ClassMemberDeclns\opt \xcd"}" & (\ref{prod:ClassBody}) \\
%(FROM #(prod:ClassMemberDeclns)#)
   ClassMemberDeclns \: ClassMemberDecln & (\ref{prod:ClassMemberDeclns}) \\
                     \| ClassMemberDeclns ClassMemberDecln \\
%(FROM #(prod:ClassMemberDecln)#)
    ClassMemberDecln \: InterfaceMemberDecln & (\ref{prod:ClassMemberDecln}) \\
                     \| CtorDecln \\
\end{bbgrammar}
%##)




\section{Fields}
\label{FieldDefinitions}
\index{object!field}
\index{field}

Objects may have {\em instance fields}, or simply {\em fields} (called
``instance variables'' in C++ and Smalltalk, and ``slots'' in CLOS): places to
store data that is pertinent to the object.  Fields, like variables, may be
mutable (\xcd`var`) or immutable (\xcd`val`).  

A class may have {\em static fields}, which store data pertinent to the
entire class of objects.  See \Sref{StaticInitialization} for more
information. 
Because of its emphasis on safe concurrency, \Xten{} requires static
fields to be immutable (\xcd`val`). 

No two fields of the same class may have the same name.  A field may have the
same name as a method, although for fields of functional type there is a
subtlety (\Sref{sect:disambiguations}).  

\subsection{Field Initialization}
\index{field!initialization}
\index{initialization!of field}

Fields may be given values via {\em field initialization expressions}:
\xcd`val f1 = E;` and \xcd`var f2 : Int = F;`. Other fields of \xcd`this` may
be referenced, but only those that {\em precede} the field being initialized.


\begin{ex}The following is correct, but would not be if the fields were
reversed:

%~~gen ^^^ Classes10
%package Classes_field_init_expr_a;
%~~vis
\begin{xten}
class Fld{
  val a = 1;
  val b = 2+a;
}
\end{xten}
%~~siv
% class Hook{ def run() {
%   val f = new Fld();
%   assert f.a == 1 && f.b == 3;
%   return true;}}
%~~neg
\end{ex}

\subsection{Field hiding}
\label{sect:FieldHiding}
\index{field!hiding}


A subclass that defines a field \xcd"f" hides any field \xcd"f"
declared in a superclass, regardless of their types.  The
superclass field \xcd"f" may be accessed within the body of
the subclass via the reference \xcd"super.f".

With inner classes, it is occasionally necessary to 
write \xcd`Cls.super.f` to get at a hidden field \xcd`f` of an outer class
\xcd`Cls`. 

\begin{ex}
The \xcd`f` field in \xcd`Sub` hides the \xcd`f` field in \xcd`Super`
The \xcd`superf` method provides access to the \xcd`f` field in \xcd`Super`.
%~~gen ^^^ Classes20
% package classes.fields.primus;
%~~vis
\begin{xten}
class Super{ 
  public val f = 1; 
}
class Sub extends Super {
  val f = true;
  def superf() : Int = super.f; // 1
}
\end{xten}
%~~siv
% class Hook { def run() { 
%   val sub = new Sub();
%   assert sub.f == true;
%   assert sub.superf() == 1;
%   return true;} }
%~~neg
\end{ex}

\begin{ex}
Hidden fields of outer classes can be accessed by suitable forms: 
%~~gen ^^^ Classes30
% package classes.fields.secundus; 
% // NOTEST
%~~vis
\begin{xten}
class A {
   val f = 3;
}
class B extends A {
   val f = 4;
   class C extends B {
      // C is both a subclass and inner class of B
      val f = 5;
       def example() {
         assert f         == 5 : "field of C";
         assert super.f   == 4 : "field of superclass";
         assert B.this.f  == 4 : "field of outer instance";
         assert B.super.f == 3 : "super.f of outer instance";
       }
    }
}
\end{xten}
%~~siv
% class Hook { def run() { ((new B()).new C()).example(); return true; } }
%~~neg
\end{ex}

\subsection{Field qualifiers}
\label{FieldQualifier}
\index{qualifier!field}
\index{field!qualifier}

The behavior of a field may be changed by a field qualifier, such as
\xcd`static` or \xcd`transient`.  


\subsubsection{\Xcd{static} qualifier}
\index{field!static}

A \xcd`val` field may be declared to be {\em static}, as described in
\Sref{FieldDefinitions}. 

\subsubsection{\Xcd{transient} Qualifier}
\label{TransientFields}
\index{transient}
\index{field!transient}

A field may be declared to be {\em transient}.  Transient fields are excluded
from the deep copying that happens when information is sent from place to
place in an \Xcd{at} statement.    The value of a transient field of a copied
object is the default value of its type, regardless of the value of the field
in the original.  If the type of a field has no
default value, it cannot be marked \Xcd{transient}.

%%AT-COPY%% %~~gen ^^^ Classes40
%%AT-COPY%% % package Classes.Transient.Example;
%%AT-COPY%% % KNOWNFAIL-at
%%AT-COPY%% %~~vis
%%AT-COPY%% \begin{xten}
%%AT-COPY%% class Trans { 
%%AT-COPY%%    val copied = "copied";
%%AT-COPY%%    transient var transy : String = "a very long string";
%%AT-COPY%%    def example() {
%%AT-COPY%%       at (here; this) { // causes copying of 'this'
%%AT-COPY%%          assert(this.copied.equals("copied"));
%%AT-COPY%%          assert(this.transy == null);
%%AT-COPY%%       }
%%AT-COPY%%    }
%%AT-COPY%% }
%%AT-COPY%% \end{xten}
%%AT-COPY%% %~~siv
%%AT-COPY%% % class Hook{ def run() {(new Example()).example(); return true;}}
%%AT-COPY%% %~~neg
%%AT-COPY%% 

%~~gen ^^^ Classes40
% package Classes.Transient.Example;
% 
%~~vis
\begin{xten}
class Trans { 
   val copied = "copied";
   transient var transy : String = "a very long string";
   def example() {
      at (here) { // causes copying of 'this'
         assert(this.copied.equals("copied"));
         assert(this.transy == null);
      }
   }
}
\end{xten}
%~~siv
% class Hook{ def run() {(new Trans()).example(); return true;}}
%~~neg


\section{Properties}
\label{PropertiesInClasses}
\index{property}

The properties of an object (or struct) are a restricted form of public
\xcd`val` fields.\footnote{In many cases, a 
\xcd`val` field can be upgraded to a \xcd`property`, which 
entails no compile-time or runtime cost.  Some cannot be, \eg, in cases where
cyclic structures of \xcd`val` fields are required.} 
For example,  every array has a \xcd`rank` telling
how many subscripts it takes.  User-defined classes can have whatever
properties are desired. 

Properties differ from public \xcd`val` fields in a few ways: 
\begin{enumerate}
\item Property references are allowed on \xcd`self` in constraints:
      \xcd`self.prop`.  Field references are not.
\item Properties are in scope for all instance initialization expressions.
      \xcd`val` fields are not.
\item The graph of values reachable from a given object by following only
      property links is acyclic.  Conversely, it is possible (and routine) for
      two objects to point to each other with \xcd`val` fields.
\item Properties are declared in the class header; \xcd`val` fields are
      defined in the class body.
\item Properties are set in constructors by a \xcd`property` statement.
      \xcd`val` fields are set by assignment.
\end{enumerate}



Properties are defined in parentheses, after the name of the class.  They are
given values by the \xcd`property` command in constructors.

\begin{ex}
\xcd`Proper` has a single property, \xcd`t`.  \xcd`new Proper(4)` creates a
\xcd`Proper` object with \xcd`t==4`. 
%~~gen ^^^ Classes50
% package Classes.Toss.Freedom.Disk2;
%~~vis
\begin{xten}
class Proper(t:Int) {
  def this(t:Int) {property(t);}
}
\end{xten}
%~~siv
% class Hook{ def run() {
%   val p = new Proper(4);
%   return p.t == 4;
% } } 
%~~neg

\end{ex}


It is a static error for a class
defining a property \xcd"x: T" to have a subclass class that defines
a property or a field with the name \xcd"x".


A property \xcd`x:T` induces a field with the same name and type, 
as if defined with: 
%~~gen ^^^ Classes60
% package Classes.For.Masses.Of.NevermindTheRest;
% class Exampll[T] {
%~~vis
\begin{xten}
public val x : T;
\end{xten} 
%~~siv
% def this(y:T) { x=y; }
% }
%~~neg

\index{property!initialization}
Properties are initialized in a constructor by the invocation of a special \Xcd{property}
statement. The requirement to use the \xcd`property` statement means that all properties
must be given values at the same time: a container either has its properties
or it does not.
\begin{xten}
property(e1,..., en);
\end{xten}
The number and types of arguments to the \Xcd{property} statement must match
the number and types of the properties in the class declaration, in order.  
Every constructor of a class with properties must invoke \xcd`property(...)`
precisely once; it is a static error if X10 cannot prove that this holds.



By construction, the graph whose nodes are values and whose edges are
properties is acyclic.  \Eg, there cannot be values \xcd`a` and \xcd`b` with
properties \xcd`c` and \xcd`d` such that \xcd`a.c == b` and \xcd`b.d == a`.

\begin{ex}
%~~gen ^^^ Classes7h2f
% package Classes7h2f;
%~~vis
\begin{xten}
class Proper(a:Int, b:String) {
  def this(a:Int, b:String) {
      property(a, b);
  }
  def this(z:Int) {
      val theA = z+5;
      val theB = "X"+z;
      property(theA, theB);
  }
  static def example() {
      val p = new Proper(1, "one");
      assert p.a == 1 && p.b.equals("one");
      val q = new Proper(10);
      assert q.a == 15 && q.b.equals("X10");
  }
}
\end{xten}
%~~siv
% class Hook{ def run() {Proper.example(); return true;}}
%~~neg
\end{ex}

\subsection{Properties and Field Initialization}

Fields with explicit initializers are evaluated immediately after the
\xcd`property` command, and all properties are in scope when initializers are
evaluated.  

\begin{ex}
Class \xcd`Init` initializes the field \xcd`a` to be an array of \xcd`n`
elements, where \xcd`n` is a property.    
When \xcd`new Init(4)` is executed, the constructor first sets \xcd`n` to
\xcd`4` via the \xcd`property` statement, and then initializes \xcd`a` to a
4-element array.

However, \xcd`Outit` uses a field rather than a property for \xcd`n`.  
If the \xcd`ERROR` line were present, it would not compile.  \xcd`n` has not
been definitely assigned (\Sref{sect:DefiniteAssignment}) at this point, and
\xcd`n` has not been given its value, so \xcd`a` cannot be computed.  
(If one insisted that \xcd`n` be a property, \xcd`a` would have to be
initialized in the constructor, rather than by an initialization expression.)
%~~gen ^^^ Classes9c9r
% package Classes9c9r;
%~~vis
\begin{xten}
class Init(n:Int) {
  val a = new Rail[String](n, "");
  def this(n:Int) { property(n); }
}
class Outit {
  val n : Int;
  //ERROR: val a = new Rail[String](n, "");
  def this(m:Int) { this.n = m; }
}
\end{xten}
%~~siv
%
%~~neg


\end{ex}

\subsection{Properties and Fields}

A container with a property named \xcd`p`, or a nullary property method named
\xcd`p()`, cannot have a field named \xcd`p` --- either defined in that
container, or inherited from a superclass.

\subsection{Acyclicity of Properties}
\index{properties!acyclic}

X10 has certain restrictions that, ultimately, require that properties are
simpler than their containers.  For example, \xcd`class A(a:A){}` is not
allowed.  
Formally, this requirement is that there is  a total order $\preceq$ 
on all classes and
structs such that, if $A$ extends $B$, then $A \prec B$, and
if $A$ has a property of type $B$, then $A \prec B$, where $A \prec B$ means
$A \preceq B$ and $A \ne B$.   
For example, the preceding class \xcd`A` is ruled out because we would need
\xcd`A`$\prec$\xcd`A`, which violates the definition of $\prec$.
The programmer need not (and cannot) specify
$\preceq$, and rarely need worry about its existence.  

Similarly, 
the type of a property may not simply be a type parameter.  
For example, \xcd`class A[X](x:X){}` is illegal.





\section{Methods}
\label{sect:Methods}
\index{method}
\index{signature}
\index{method!signature}
\index{method!instance}
\index{method!static}

As is common in object-oriented languages, objects can have {\em methods}, of
two sorts.  {\em Static methods} are functions, conceptually associated with a
class and defined in its namespace.  {\em Instance methods} are parameterized
code bodies associated with an instance of the class, which execute with
convenient access to that instance's fields. 

Each method has a {\em signature}, telling what arguments it accepts, what
type it returns, and what precondition it requires. Method definitions may be
overridden by subclasses; the overriding definition may have a declared return
type that is a subtype of the return type of the definition being overridden.
Multiple methods with the same name but different signatures may be provided
\index{overloading}
\index{polymorphism}
on a class (called ``overloading'' or ``ad hoc polymorphism''). Methods may be
declared \Xcd{public}, \Xcd{private}, \Xcd{protected}, or given default package-level access
rights.

%##(MethMods MethodDeclaration TypeParams Formals FormalList HasResultType MethodBody BinOpDecln PrefixOpDecln ApplyOpDecln ConversionOpDecln
\begin{bbgrammar}
%(FROM #(prod:MethMods)#)
            MethMods \: Mods\opt & (\ref{prod:MethMods}) \\
                     \| MethMods \xcd"property"  \\
                     \| MethMods Mod \\
%(FROM #(prod:MethodDecln)#)
         MethodDecln \: MethMods \xcd"def" Id TypeParams\opt Formals Guard\opt Throws\opt HasResultType\opt MethodBody & (\ref{prod:MethodDecln}) \\
                     \| BinOpDecln \\
                     \| PrefixOpDecln \\
                     \| ApplyOpDecln \\
                     \| SetOpDecln \\
                     \| ConversionOpDecln \\
%(FROM #(prod:TypeParams)#)
          TypeParams \: \xcd"[" TypeParamList \xcd"]" & (\ref{prod:TypeParams}) \\
%(FROM #(prod:Formals)#)
             Formals \: \xcd"(" FormalList\opt \xcd")" & (\ref{prod:Formals}) \\
%(FROM #(prod:FormalList)#)
          FormalList \: Formal & (\ref{prod:FormalList}) \\
                     \| FormalList \xcd"," Formal \\
%(FROM #(prod:Throws)#)
             Throws \: \xcd"throws" ThrowList & (\ref{prod:Throws}) \\
%(FROM #(prod:ThrowsList)#)
          ThrowsList \: Type & (\ref{prod:ThrowsList}) \\
                     \| ThrowsList \xcd"," Type \\
%(FROM #(prod:HasResultType)#)
       HasResultType \: ResultType & (\ref{prod:HasResultType}) \\
                     \| \xcd"<:" Type \\
%(FROM #(prod:MethodBody)#)
          MethodBody \: \xcd"=" LastExp \xcd";" & (\ref{prod:MethodBody}) \\
                     \| \xcd"=" Annotations\opt \xcd"{" BlockStmts\opt LastExp \xcd"}" \\
                     \| \xcd"=" Annotations\opt Block \\
                     \| Annotations\opt Block \\
                     \| \xcd";" \\
%(FROM #(prod:BinOpDecln)#)
          BinOpDecln \: MethMods \xcd"operator" TypeParams\opt \xcd"(" Formal  \xcd")" BinOp \xcd"(" Formal  \xcd")" Guard\opt HasResultType\opt MethodBody & (\ref{prod:BinOpDecln}) \\
                     \| MethMods \xcd"operator" TypeParams\opt \xcd"this" BinOp \xcd"(" Formal  \xcd")" Guard\opt HasResultType\opt MethodBody \\
                     \| MethMods \xcd"operator" TypeParams\opt \xcd"(" Formal  \xcd")" BinOp \xcd"this" Guard\opt HasResultType\opt MethodBody \\
%(FROM #(prod:PrefixOpDecln)#)
       PrefixOpDecln \: MethMods \xcd"operator" TypeParams\opt PrefixOp \xcd"(" Formal  \xcd")" Guard\opt HasResultType\opt MethodBody & (\ref{prod:PrefixOpDecln}) \\
                     \| MethMods \xcd"operator" TypeParams\opt PrefixOp \xcd"this" Guard\opt HasResultType\opt MethodBody \\
%(FROM #(prod:ApplyOpDecln)#)
        ApplyOpDecln \: MethMods \xcd"operator" \xcd"this" TypeParams\opt Formals Guard\opt HasResultType\opt MethodBody & (\ref{prod:ApplyOpDecln}) \\
%(FROM #(prod:ConversionOpDecln)#)
   ConversionOpDecln \: ExplConvOpDecln & (\ref{prod:ConversionOpDecln}) \\
                     \| ImplConvOpDecln \\
\end{bbgrammar}
%##)


\index{parameter!var}
\index{parameter!val}
A formal parameter may have a \xcd"val" or \xcd"var"
% , or \Xcd{ref}
modifier; \xcd`val` is the default.
The body of the method is executed in an environment in which 
each formal parameter corresponds to a local variable (\xcd`var` iff the
formal parameter is \xcd`var`)
and is initialized with the value of the actual parameter.

\subsection{Forms of Method Definition}

There are several syntactic forms for definining methods.   The forms that
include a block, such as \xcd`def m(){S}`, allow an arbitrary block.  These
forms can define a \xcd`void` method, which does not return a value. 

The
forms that include an expression, such as \xcd`def m()=E`, require a
syntactically and semantically valid expression.   These forms cannot define a
\xcd`void` method, because expressions cannot be \xcd`void`.  

There are no other semantic differences between the two forms. 

\subsection{Method Return Types}

A method with an explicit return type returns that type.
A method without an
explicit return type is given a return type by type inference.
A {\em call} to a method has type given by substituting information about the
actual \xcd`val` parameters for the formals.

\begin{ex}

In the example below, \xcd`met1` has an explicit return type \xcd`Ret{n==a}`.
\xcd`met2` does not, so its return type is computed, also to be
\xcd`Ret{n==a}`, because that's what the implicitly-defined constructor 
returns.

\xcd`use3` requires that its argument have \xcd`n==3`.  
\xcd`example` shows that both \xcd`met1` and \xcd`met2` can be used to produce
such an object.  In both cases, the actual argument \xcd`3` is substituted for
the formal argument \xcd`a` in the return type expression for the method
\xcd`Ret{n==a}`, giving the type \xcd`Ret{n==3}` as required by \xcd`use3`.

%~~gen ^^^ Classes9q2w
% package Classes9q2w;
%~~vis
\begin{xten}
class Ret(n:Int) {
  static def met1(a:Int) : Ret{n==a} = new Ret(a);
  static def met2(a:Int)             = new Ret(a);
  static def use3(Ret{n==3}) {}
  static def example() {
     use3(met1(3));
     use3(met2(3));
  }  
}
\end{xten}
%~~siv
%
%~~neg


\end{ex}


\subsection{Throws Clause}
The \xcd`throws` clause indicates what checked exceptions may be
raised during the execution of the method and are not handled by
\xcd'catch' blocks within the method.  If a checked exception may
escape from the method, then it must be by a subtype of one of the
types listed in the \xcd`throws` clause of the method.   Checked
exceptions are defined to be any subclass of
\xcd{x10.lang.CheckedThrowable} that are not also subclasses of
either \xcd{x10.lang.Exception} or \xcd{x10.lang.Error}. 

If a method is implementing an interface or overriding a superclass
method the set of types represented by its \xcd'throws' clause must by
a (potentially improper) subset of the types of the \xcd'throws'
clause of the method it is overriding. 

\subsection{Final Methods}
\index{final}
\index{method!final}
An instance method may be given the \xcd`final` qualifier.  \xcd`final`
methods may not be overridden.

\subsection{Generic Instance Methods}
\index{method!generic instance}

\limitationx{}
In X10, an instance method may be generic: 
%~~gen ^^^ Classes1b7z
% package Classes1b7z;
% NOTEST
%~~vis
\begin{xten}
class Example {
  def example[T](t:T) = "I like " + t;
}
\end{xten}
%~~siv
%
%~~neg

However, the C++ back end does not currently support generic virtual instance
methods like \xcd`example`.  It does allow generic instance methods which are
\xcd`final` or \xcd`private`, and it does allow generic static methods.  


\subsection{Method Guards}
\label{MethodGuard}
\index{method!guard}
\index{guard!on method}

Often, a method will only make sense to invoke under certain
statically-determinable conditions.  These conditions may be expressed as a
guard on the method.

\begin{ex}
For example, \xcd`example(x)` is only
well-defined when \xcd`x != null`, as \xcd`null.toString()` throws a null
pointer exception, and returns nothing: 
%~~gen ^^^ Classes80
% package Classes.methodwithconstraintthingie;
% 
%~~vis
\begin{xten}
class Example {
   var f : String = "";
   def setF(x:Any){x != null} : void = {
      this.f = x.toString();
   }
}
\end{xten}
%~~siv
%
%~~neg
\noindent
(We could have used a constrained type \xcd`Any{self!=null}` for \xcd`x`
instead; in
most cases it is a matter of personal preference or convenience of expression
which one to use.) 
\end{ex}


The requirement of having a method guard 
is that callers must demonstrate to
the X10
compiler that the guard is satisfied.  
With the \xcd`STATIC_CHECKS` compiler option in force (\Sref{sect:Callstyle}), this is
checked at compile time. 
As usual with static constraint
checking, there is no runtime cost.  Indeed, this code can be more efficient
than usual, as it is statically provable that \xcd`x != null`.

When \xcd`STATIC_CHECKS` is not in force, dynamic checks are generated as
needed; method guards are checked at runtime. This is potentially more
expensive, but may be more convenient. 

\begin{ex}
The following code fragment contains a line which will not compile 
with \xcd`STATIC_CHECKS` on (assuming the guarded \xcd`example` method above).  (X10's type system does not attempt to propagate 
information from \xcd`if`s.)  It will compile with \xcd`STATIC_CHECKS` off,
but it may insert an extra \xcd`null`-test for \xcd`x`.  
%~~gen ^^^ Classes90
% package Classes.methodguardnadacastthingie;
%//OPTIONS: -STATIC_CHECKS
% class Example {var f : String = ""; def example(x:Any){x != null} = {this.f = x.toString();}}
% class Eyample {
%~~vis
\begin{xten}
  def exam(e:Example, x:Any) {
    if (x != null) 
       e.example(x as Any{x != null});
       // If STATIC_CHECKS is in force: 
       // ERROR: if (x != null) e.example(x); 
  }
\end{xten}
%~~siv
%}
%~~neg
\end{ex}


The guard \xcd`{c}` 
in a guarded method 
\xcd`def m(){c} = E;`
specifies a constraint \xcd"c" on the
properties of the class \xcd"C" on which the method is being defined. The
method, in effect, only exists  for those instances of \xcd"C" which satisfy
\xcd"c".  It is 
illegal for code to invoke the method on objects whose static type is
not a subtype of \xcd"C{c}".

Specifically: 
    the compiler checks that every method invocation
    \xcdmath"o.m(e$_1$, $\dots$, e$_n$)"
    is type correct. Each argument
    \xcdmath"e$_i$" must have a
    static type \xcdmath"S$_i$" that is a subtype of the declared type
    \xcdmath"T$_i$" for the $i$th
    argument of the method, and the conjunction of the constraints on the
    static types 
    of the arguments must entail the guard in the parameter list
    of the method.

    The compiler checks that in every method invocation
    \xcdmath"o.m(e$_1$, $\dots$, e$_n$)"
    the static type of \xcd"o", \xcd"S", is a subtype of \xcd"C{c}", where the method
    is defined in class \xcd"C" and the guard for \xcd"m" is equivalent to
    \xcd"c".

    Finally, if the declared return type of the method is
    \xcd"D{d}", the
    return type computed for the call is
    \xcdmath"D{a: S; x$_1$: S$_1$; $\dots$; x$_n$: S$_n$; d[a/this]}",
    where \xcd"a" is a new
    variable that does not occur in
    \xcdmath"d, S, S$_1$, $\dots$, S$_n$", and
    \xcdmath"x$_1$, $\dots$, x$_n$" are the formal
    parameters of the method.


\limitation{
Using a reference to an outer class, \xcd`Outer.this`, in a constraint, is not supported.
}


\subsection{Property methods}
\index{method!property}
\index{property method}

%##(PropertyMethodDeclaration
\begin{bbgrammar}
%(FROM #(prod:PropMethodDecln)#)
     PropMethodDecln \: MethMods Id TypeParams\opt Formals Guard\opt Throws\opt HasResultType\opt MethodBody & (\ref{prod:PropMethodDecln}) \\
                     \| MethMods Id Guard\opt HasResultType\opt MethodBody \\
\end{bbgrammar}
%##)

Property methods are methods that can be evaluated in constraints, as
properties can.   They provide a means of abstraction over properties; \eg,
interfaces can specify property methods that implementing containers must
provide, but, just as they cannot specify ordinary fields, they cannot specify
property fields.   Property methods are very limited in computing power: they
must obey the same restrictions as constraint expressions.  In particular,
they cannot have side effects, or even much code in their bodies.


\begin{ex}
The \xcd`eq()` method below tells if the \xcd`x` and \xcd`y`
properties are equal; the \xcd`is(z)` method tells if they are both equal to
\xcd`z`.  
The \xcd`eq` and \xcd`is` property methods are used in types in the
\xcd`example` method.
%~~gen ^^^ Classes100
%package Classes.PropertyMethods;
%~~vis
\begin{xten}
class Example(x:Int, y:Int) {
   def this(x:Int, y:Int) { property(x,y); }
   property eq() = (x==y);
   property is(z:Int) = x==z && y==z;
   def example( a : Example{eq()}, b : Example{is(3)} ) {}
}
\end{xten}
%~~siv
%
%~~neg
\end{ex}

A property method declared in a class must have
a body and must not be \xcd"void".  The body of the method must
consist of only a single \xcd"return" statement with an expression,  or a single
expression.  It is a static error if the expression cannot be
represented in the constraint system.   Property methods may be \xcd`abstract`
in \xcd`abstract` classes, and may be specified in interfaces, but are
implicitly \xcd`final` in 
non-\xcd`abstract` classes. 

The expression may contain invocations of other property methods.  The
compiler ensures that there are no circularities in property methods, so
property method evaluations always terminate.

Property methods in classes are implicitly \xcd"final"; they cannot be
overridden.  It is a static error if a superclass has a property method with a
given signature, and a subclass has a method or property method with the same
signature.   It is a static error if a superclass has a property with some
name \xcd`p`, and a subclass has a nullary method of any kind (instance,
static, or property) also named \xcd`p`. 



A nullary property method definition may omit 
the \xcd"def" keyword.  That is, the following are equivalent:

%~~gen ^^^ Classes110
% package classes.waifsome1;
% class Waif(rect:Boolean, onePlace:Place, zeroBased:Boolean) {
%~~vis
\begin{xten}
property def rail(): Boolean = 
   rect && onePlace == here && zeroBased;
\end{xten}
%~~siv
%}
%~~neg
and
%~~gen ^^^ Classes120
% package classes.waifsome2;
% class Waif(rect:Boolean, onePlace:Place, zeroBased:Boolean) {
%~~vis
\begin{xten}
property rail(): Boolean = 
   rect && onePlace == here && zeroBased;
\end{xten}
%~~siv
%}
%~~neg

Similarly, nullary property methods can be inspected in constraints without
\xcd`()`. If \xcd`ob`'s type has a property \xcd`p`, then \xcd`ob.p` is that
property. Otherwise, if it has a nullary property method \xcd`p()`, \xcd`ob.p`
is equivalent to \xcd`ob.p()`. As a consequence, if the type provides both a
property \xcd`p` and a nullary method \xcd`p()`, then the property can be
accessed as \xcd`ob.p` and the method as \xcd`ob.p()`.\footnote{This only
applies to nullary property methods, not nullary instance methods.  Nullary
property methods perform limited computations, have no side effects, and
always return the same value, since
they have to be expressed in the constraint sublanguage.  In this sense, a
nullary property method does not behave hugely different from a property.
Indeed, a compilation scheme which cached the value of the property method
would all but erase the distinction.  Other methods may
have more behavior, \eg, side effects, so we keep the \xcd`()` to make it
clear that a method call is potentially large.
}

%~~longexp~~`~~` ^^^ Classes130
% package classes.not.weasels;
% class Waif(rect:Boolean, onePlace:Place, zeroBased:Boolean) {
%   def this(rect:Boolean, onePlace:Place, zeroBased:Boolean) 
%          :Waif{self.rect==rect, self.onePlace==onePlace, self.zeroBased==zeroBased}
%          = {property(rect, onePlace, zeroBased);}
%   property rail(): Boolean = rect && onePlace == here && zeroBased;
%   static def zoink() {
%      val w : Waif{
%~~vis
\xcd`w.rail`, with either definition above, 
% }= new Waif(true, here, true);
% }}
%~~pxegnol
is equivalent to 
%~~longexp~~`~~` ^^^ Classes140
% package classes.not.ferrets;
% class Waif(rect:Boolean, onePlace:Place, zeroBased:Boolean) {
%   def this(rect:Boolean, onePlace:Place, zeroBased:Boolean) 
%          :Waif{self.rect==rect, self.onePlace==onePlace, self.zeroBased==zeroBased}
%          = {property(rect, onePlace, zeroBased);}
%   property rail(): Boolean = rect && onePlace == here && zeroBased;
%   static def zoink() {
%      val w : Waif{
%~~vis
\xcd`w.rail()`
% }= new Waif(true, here, true);
% }}
%~~pxegnol


\subsubsection{Limitation of Property Methods}

\limitationx{} 
Currently, X10 forbids the use of property methods which have all the
following features: 
\begin{itemize}
\item they are abstract, and
\item they have one or more arguments, and
\item they appear as subterms in constraints.
\end{itemize}
Any two of these features may be combined, but the three together may not be. 

\begin{ex} 
The constraint in \xcd`example1` is concrete, not abstract.  The constraint in
\xcd`example2` is nullary, and has no arguments.  The constraint in
\xcd`example3` appears at the top level, rather than as a subterm ({\em cf.}
the equality expressions \xcd`A==B` in the other examples).    However,
\xcd`example4` combines all three features, and is not allowed.
%~~gen ^^^ Classes7a5j
% package Classes7a5j;
% // If example4() compiles, then the limitation in Classes7a5j's section is
% // gone, so delete the whole subsection from the spec.
%~~vis
\begin{xten}
class Cls {
  property concrete(a:Int) = 7;
}
interface Inf {
  property nullary(): Int;
  property topLevel(z:Int):Boolean;
  property allThree(z:Int):Int;
}
class Example{
  def example1(Cls{self.concrete(3)==7}) = 1;
  def example2(Inf{self.nullary()==7})   = 2;
  def example3(Inf{self.topLevel(3)})    = 3;
  //ERROR: def example4(Inf{self.allThree(3)==7}) = "fails";
}
\end{xten}
%~~siv
%
%~~neg
\end{ex}


\subsection{Method overloading, overriding, hiding, shadowing and obscuring}
\label{MethodOverload}
\index{method!overloading}



The definitions of method overloading, overriding, hiding, shadowing and
obscuring in \Xten{} are familiar from languages such as Java, modulo the
following considerations motivated by type parameters and dependent types.



Two or more methods of a class or interface may have the same
name if they have a different number of type parameters, or
they have formal parameters of different constraint-erased types (in some instantiation of the
generic parameters). 



\begin{ex}
The following overloading of \xcd`m` is unproblematic.
%~~gen ^^^ Classes150
% package Classes.Mful;
%~~vis
\begin{xten}
class Mful{
   def m() = 1;
   def m[T]() = 2;
   def m(x:Int) = 3;
   def m[T](x:Int) = 4;
}
\end{xten}
%~~siv
%
%~~neg
\end{ex}


A class definition may include methods which are ambiguous in {\em some}
generic instantiation. (It is a compile-time error if the methods are
ambiguous in {\em every} generic instantiation, but excluding class
definitions which are are ambiguous in {\em some} instantiation would exclude
useful cases.)  It is a compile-time error to {\em use} an ambiguous method
call. 

\begin{ex}
The following class definition is acceptable.  However, the marked method
calls are ambiguous, and hence not acceptable.
%~~gen ^^^ Classes4d5e
% package Classes4d5e;
%~~vis
\begin{xten}
class Two[T,U]{
  def m(x:T)=1;
  def m(x:Int)=2;
  def m[X](x:X)=3;
  def m(x:U)=4;
  static def example() {
    val t12 = new Two[Int, Any]();
    // ERROR: t12.m(2);
    val t13  = new Two[String, Any]();
    t13.m("ferret");
    val t14 = new Two[Boolean,Boolean]();
    // ERROR: t14.m(true);
  }
}
\end{xten}
%~~siv
%~~neg
\noindent
The call \xcd`t12.m(2)` could refer to either the \xcd`1` or \xcd`2`
definition of \xcd`m`, so it is not allowed.   
The call \xcd`t14.m(true)` could refer to either the \xcd`1` or \xcd`4`
definition, so it, too, is not allowed.

The call \xcd`t13.m("ferret")` refers only to the \xcd`1` definition.  If
the \xcd`1` definition were absent, type argument inference would make it
refer to the \xcd`3` definition.  However, X10 will choose a fully-specified
call if there is one, before trying type inference, so this call unambiguously
refers to \xcd`1`.
\end{ex}


\XtenCurrVer{} does not permit overloading based on constraints. That is, the
following is {\em not} legal, although either method definition individually
is legal:
\begin{xten}
   def n(x:Int){x==1} = "one";
   def n(x:Int){x!=1} = "not";
\end{xten}




The definition of a method declaration \xcdmath"m$_1$" ``having the same signature
as'' a method declaration \xcdmath"m$_2$" involves identity of types. 



The {\em constraint erasure} of a type \xcdmath"T", 
\xcdmath"$ce$(T)",
is obtained by removing all the constraints outside of functions in \xcd`T`,
specificially: 
\begin{eqnarray}
ce({\tt T}) &=& {\tt T} \mbox{ if \xcd`T` is a container or interface}\\
ce({\tt T\{c\}}) &=& ce({\tt T})\\
ce({\tt T[S}_1{\tt,}\ldots{\tt,S}_n{\tt ]})
  &=&
ce({\tt T}){\tt [} ce({\tt S}_1){\tt,}\ldots{\tt,} ce({\tt S}_n){\tt ]}\\
ce({\tt (S}_1{\tt,}\ldots{\tt,S}_n{\tt ) => T})
  &=&
{\tt }{\tt (} ce({\tt S}_1){\tt,}\ldots{\tt,} ce({\tt S}_n){\tt ) => } 
ce({\tt T})
\end{eqnarray}



 Two methods are said to have {\em erasedly equivalent signatures} if (a) they have the
 same number of type parameters, 
(b) they have the same number of formal (value) parameters, and (c)
for each formal parameter the constraint erasure of its types are erasedly equivalent.
It is a 
compile-time error for there to be two methods with the same name and
erasedly equivalent signatures in a class (either defined in that class or in a
superclass), unless the signatures are identical (without erasures) and one of the methods is
defined in a superclass (in which case the superclass's method is overridden
by the subclass's, and the subclass's method's return type must be a subtype of
the superclass's method's return type).  

 



In addition, the guard of an overridden method
must entail
the guard of the overriding method.   This
ensures that any virtual call to the method
satisfies the guard of the callee.

\begin{ex}
In the following example, the call to \xcd`s.recip(3)` in \xcd`example()`
will invoke \xcd`Sub.recip(n)`.  The call is legitimate because
\xcd`Super.recip`'s guard, \xcd`n != 0`, is satisfied by \xcd`3`.  
The guard on \xcd`Sub.recip(n)` is simply
\xcd`true`, which is also satisfied.  However, if we had used the \xcd`ERROR`
line's definition, the guard on \xcd`Sub.recip(n)` would be \xcd`n != 0, n != 3`, which
is not satisfied by \xcd`3`, so -- despite the call statically type-checking
-- at runtime the call would violate its guard and (in this case) throw an exception.
%~~gen ^^^ Classes5l3r
% package Classes5l3r;
% // FOR-ERR-CASE-DELETE: def recip(m:Int){true} = 1.0/m;
%~~vis
\begin{xten}
class Super {
  def recip(n:Int){n != 0} = 1.0/n;
}
class Sub extends Super{
  //ERROR: def recip(n:Int){n != 0, n != 3} = 1.0/(n * (n-3));
  def recip(m:Int){true} = 1.0/m;
}
class Example{
  static def example() {
     val s : Super = new Sub();
     s.recip(3);
  }
}
\end{xten}
%~~siv
%
%~~neg

\end{ex}


  If a class \xcd"C" overrides a method of a class or interface
  \xcd"B", the guard of the method in \xcd"B" must entail
  the guard of the method in \xcd"C".


A class \xcd"C" inherits from its direct superclass and superinterfaces all
their methods visible according to the access modifiers
of the superclass/superinterfaces that are not hidden or overridden. A method \xcdmath"M$_1$" in a class
\xcd"C" overrides
a method \xcdmath"M$_2$" in a superclass \xcd"D" if
\xcdmath"M$_1$" and \xcdmath"M$_2$" have erasedly equivalent signatures.
Methods are overriden on a signature-by-signature basis.  It is a compile-time
error if an instance method overrides a static method.  (But is it permitted
for an instance {\em field} to hide a static {\em field}; that's hiding
(\Sref{sect:FieldHiding}), not 
overriding, and hence totally different.)

\section{Constructors}
\label{sect:constructors}
\index{constructor}

Instances of classes are created by the \xcd`new` expression: \\
%##(ClassInstCreationExp
\begin{bbgrammar}
%(FROM #(prod:ObCreationExp)#)
       ObCreationExp \: \xcd"new" TypeName TypeArgs\opt \xcd"(" ArgumentList\opt \xcd")" ClassBody\opt & (\ref{prod:ObCreationExp}) \\
                     \| Primary \xcd"." \xcd"new" Id TypeArgs\opt \xcd"(" ArgumentList\opt \xcd")" ClassBody\opt \\
                     \| FullyQualifiedName \xcd"." \xcd"new" Id TypeArgs\opt \xcd"(" ArgumentList\opt \xcd")" ClassBody\opt \\
\end{bbgrammar}
%##)

This constructs a new object, and calls some code, called a {\em constructor},
to initialize the newly-created object properly.

Constructors are defined like methods, except that they must be named \xcd`this`
and ordinary methods may not be.    The content of a constructor body has
certain capabilities (\eg, \xcd`val` fields of the object may be initialized)
and certain restrictions (\eg, most methods cannot be called); see
\Sref{ObjectInitialization} for the details.

\begin{ex}

The following class provides two constructors.  The unary constructor 
\xcd`def this(b : Int)` allows initialization of the \xcd`a` field to an 
arbitrary value.  The nullary constructor \xcd`def this()` gives it a default
value of 10.  The \xcd`example` method illustrates both of these calls.


%~~gen ^^^ ClassesCtor10
% package ClassesCtor10;
%~~vis
\begin{xten}
class C {
  public val a : Int;
  def this(b : Int) { a = b; } 
  def this()        { a = 10; }
  static def example() {
     val two = new C(2);
     assert two.a == 2;
     val ten = new C(); 
     assert ten.a == 10;
  }
}
\end{xten}
%~~siv
% class Hook{ def run() {C.example(); return true;}}
%~~neg
\end{ex}


\subsection{Automatic Generation of Constructors}
\index{constructor!generated}

Classes that have no constructors written in the class declaration are
automatically given a constructor which sets the class properties and does
nothing else. If this automatically-generated constructor is not valid (\eg,
if the class has \xcd`val` fields that need to be initialized in a
constructor), the class has no constructor, which is a static error.

\begin{ex}
The following class has no explicit constructor.
Its implicit constructor is 
\xcd`def this(x:Int){property(x);}`
This implicit constructor is valid, and so is the class. 
%~~gen ^^^ ClassesCtor20
% package ClassesCtor20;
%~~vis
\begin{xten}
class C(x:Int) {
  static def example() {
    val c : C = new C(4);
    assert c.x == 4;
  }
}
\end{xten}
%~~siv
% class Hook{ def run() {C.example(); return true;}}
%~~neg
\noindent 


The following class has the same default constructor.  However, that
constructor does not initialize \xcd`d`, and thus is invalid.  This 
class does not compile; it needs an explicit constructor.
%~~gen ^^^ ClassCtor30_MustFailCompile
% NOCOMPILE
%~~vis
\begin{xten}
// THIS CODE DOES NOT COMPILE
class Cfail(x:Int) {
  val d: Int;
  static def example() {
    val wrong = new Cfail(40);
  }
}
\end{xten}
%~~siv
%
%~~neg


\end{ex}

\subsection{Calling Other Constructors}
\label{sect:call-another-ctor}

The {\em first} statement of a constructor body may be a call of the form 
\xcd`this(a,b,c)` or \xcd`super(a,b,c)`.  The former will execute the body of
the matching constructor of the current class; the latter, of the superclass. 
This allows a measure of abstraction in constructor definitions; one may be
defined in terms of another.

\begin{ex}
The following class has two constructors.  \xcd`new Ctors(123)` constructs a
new \xcd`Ctors` object with parameter 123.  \xcd`new Ctors()` constructs one
whose parameter has a default value of 100: 
%~~gen ^^^ Classes5q6q
% package Classes5q6q;
%~~vis
\begin{xten}
class Ctors {
  public val a : Int;
  def this(a:Int) { this.a = a; }
  def this()      { this(100);  }
}
\end{xten}
%~~siv
%class Hook{ def run() {
% val x = new Ctors(10); assert x.a == 10;
% val y = new Ctors(); assert y.a == 100;
% return true;}}
%~~neg
\end{ex}

In the case of a class which implements \xcd`operator ()` 
--- or any other constructor and application with the same signature --- 
this can be ambiguous.  If \xcd`this()` appears as the first statement of a
constructor body, it could, in principle, mean either a constructor call or an
operator evaluation.   This ambiguity is resolved so that \xcd`this()` always
means the constructor invocation.  If, for some reason, it is necessary to
invoke an application operator as the first meaningful statement of a
constructor, write the target of the application as \xcd`(this)`, \eg,
\xcd`(this)(a,b);`. 

\subsection{Return Type of Constructor}

A constructor for class \xcd`C` may have a return type \xcd`C{c}`.  The return
type specifies a constraint on the kind of object returned.  It cannot change
its {\em class} --- a constructor for class \xcd`C` always returns an instance
of class \xcd`C`.  
If no explicit return type is specified, the constructor's return type is
inferred.

\begin{ex}
The constructor \xcd`(A)` below, having no explicit return type, 
has its return type inferred.  
\xcd`n` is set by the \xcd`property` statement to \xcd`1`, so the return type
is inferred as \xcd`Ret{self.n==1}.`
The constructor \xcd`(B)` has \xcd`Ret{n==self.n}` as an explicit return type.
The \xcd`example()` code shows both of these in action.

%~~gen ^^^ Classes1v9a
% package Classes1v9a;
%~~vis
\begin{xten}
class Ret(n:Int) {
   def this()    { property(1); }     // (A)
   def this(n:Int) : Ret{n==self.n} { // (B)
      property(n);
   }
   static def typeIs[T](x:T){}
   static def example() {
     typeIs[Ret{self.n==1}](new Ret());  // uses (A)
     typeIs[Ret{self.n==3}](new Ret(3)); // uses (B)
   }
}
\end{xten}
%~~siv
%
%~~neg


\end{ex}

\section{Static initialization}
\label{StaticInitialization}
\index{initialization!static} 
Static fields in \Xten{} are immutable and are guaranteed to be
initialized before they are accessed. Static fields are initialized on
a per-Place basis; thus an activity that reads a static field in two
different Places may read different values for the content of the
field in each Place.  Static fields are not eagerly initialized, thus
if a particular static field is not accessed in a given Place then the
initializer expression for that field may not be evaluated in that
Place.

When an activity running in a Place \Xcd{P} attempts to read a static
field \Xcd{F} that has not yet been initialized in \Xcd{P}, then the
activity will evaluate the initializer expression for \Xcd{F} and
store the resulting value in \Xcd{F}. It is guaranteed that at most
one activity in each Place will attempt to evaluate the initializer
expression for a given static field.  If a second activity attempts to
read \Xcd{F} while the first activity is still executing the
initializer expression the second activity will be suspended until the
first activity finishes evaluating the initializer and stores the
resulting value in \Xcd{F}.

The initializer expression may directly or indirectly read other
static fields in the program.  If there is a cycle in the field
initialization dependency graph for a set of static fields, then any
activities accessing those fields may deadlock, which in turn may
result in the program deadlocking.\footnote{The current \Xten{}
  runtime does not dynamically detect this situation. Future versions
  of \Xten may be able to detect this and convert such a deadlock into the
  throwing of an \Xcd{ExceptionInInitializer} exception.}.

If an exception is raised during the evaluation of a static field's
initializer expression, then the field is deemed uninitializable in
that Place and any subsequent attempt to access the static field's
value by another activity in the Place will also result in an
exception being raised.\footnote{The implementation will make a best
  effort attempt to present stack trace information about the original
  cause of the exception in all subsequent raised exceptions}.  Failure
to initialize a field in one Place does not impact the initialization
status of the field in other Places.

\subsection{Compatability with Prior Versions of \Xten{}}
Previous versions of \Xten{} eagerly initialized all static fields in
the program at Place 0 and serialized the resulting values to all
other Places before beginning execution of the user main function.  It
is possible to simulate these serialization semantics for specific
static fields under the lazy per-Place initialization semantics
by using the idiom below:

\begin{xten}
// Pre X10 2.3 code
// expr evaluated once in Place 0 and resulting value 
// serialized to all other places
public static val x:T = expr;

// X10 2.3 code when T haszero is false
private static val x_holder:Cell[T] = 
    (here == Place.FIRST_PLACE) ? new Cell[T](expr): null;
public static val x:T = at (Place.FIRST_PLACE) x_holder();

// simpler X10 2.3 code when T haszero is true
private static val x_holder:T = 
    (here == Place.FIRST_PLACE) ? expr : Zero.get[T]();
public static val x:T = at (Place.FIRST_PLACE) x_holder;

\end{xten}

A slightly more complex variant of the above idiom in which the
initializer expression for the public field conditionally does the \xcd{at}
only when not executed at \xcd{Place.FIRST_PLACE} can be used to
obtain exactly the same serialization behavior as the pre \Xten{} v2.3
semantics.  When necessary, eager initialization for specific static fields
can be simulated by reading the static fields in \xcd{main} before
executing the rest of the program.

\section{User-Defined Operators}
\label{sect:operators}
\index{operator}
\index{operator!user-defined}

%##(MethodDeclaration
\begin{bbgrammar}
%(FROM #(prod:MethodDecln)#)
         MethodDecln \: MethMods \xcd"def" Id TypeParams\opt Formals Guard\opt Throws\opt HasResultType\opt MethodBody & (\ref{prod:MethodDecln}) \\
                     \| BinOpDecln \\
                     \| PrefixOpDecln \\
                     \| ApplyOpDecln \\
                     \| SetOpDecln \\
                     \| ConversionOpDecln \\
\end{bbgrammar}
%##)


It is often convenient to have methods named by symbols rather than words.
For example, suppose that we wish to define a \xcd`Poly` class of
polynomials -- for the sake of illustration, single-variable polynomials with
\xcd`Int` coefficients.  It would be very nice to be able to manipulate these
polynomials by the usual operations: \xcd`+` to add, \xcd`*` to multiply,
\xcd`-` to subtract, and \xcd`p(x)` to compute the value of the polynomial at
argument \xcd`x`.  We would like to write code thus: 
%~~gen ^^^ Classes160
% package Classes.In.Poly101;
% // Integer-coefficient polynomials of one variable.
% class Poly {
%   public val coeff : Rail[Int];
%   public def this(coeff: Rail[Int]) { this.coeff = coeff;}
%   public def degree() = (coeff.size-1) as Int;
%   public def a(i:Int) = (i<0 || i>this.degree()) ? 0 : coeff(i);
%
%   public static operator (c : Int) as Poly = new Poly([c as Int]);
%
%   public operator this(x:Long) {
%     val d = this.degree();
%     var s : Long = this.a(d);
%     for( i in 1 .. this.degree() ) {
%        s = x * s + a(d-i);
%     }
%     return s;
%   }
%
%   public operator this + (p:Poly) =  new Poly(
%      new Rail[Int](
%         Math.max(this.coeff.size, p.coeff.size) as Int,
%         (i:Int) => this.a(i) + p.a(i)
%      ));
%   public operator this - (p:Poly) = this + (-1)*p;
%
%   public operator this * (p:Poly) = new Poly(
%      new Rail[Int](
%        this.degree() + p.degree() + 1,
%        (k:Int) => sumDeg(k, this, p)
%        )
%      );
%
%
%   public operator (n : Int) + this = (n as Poly) + this;
%   public operator this + (n : Int) = (n as Poly) + this;
%
%   public operator (n : Int) - this = (n as Poly) + (-1) * this;
%   public operator this - (n : Int) = ((-n) as Poly) + this;
%
%   public operator (n : Int) * this = new Poly(
%      new Rail[Int](
%        this.degree()+1,
%        (k:Int) => n * this.a(k)
%      ));
%   private static def sumDeg(k:Int, a:Poly, b:Poly) {
%      var s : Int = 0;
%      for( i in 0 .. k ) s += a.a(i) * b.a(k-i);
%        // x10.io.Console.OUT.println("sumdeg(" + k + "," + a + "," + b + ")=" + s);
%      return s;
%      };
%   public final def toString() = {
%      var allZeroSoFar : Boolean = true;
%      var s : String ="";
%      for( i in 0..this.degree() ) {
%        val ai = this.a(i);
%        if (ai == 0) continue;
%        if (allZeroSoFar) {
%           allZeroSoFar = false;
%           s = term(ai, i);
%        }
%        else
%           s +=
%              (ai > 0 ? " + " : " - ")
%             +term(ai, i);
%      }
%      if (allZeroSoFar) s = "0";
%      return s;
%   }
%   private final def term(ai: Int, n:Int) = {
%      val xpow = (n==0) ? "" : (n==1) ? "x" : "x^" + n ;
%      return (ai == 1) ? xpow : "" + Math.abs(ai) + xpow;
%   }
%
%   public static def Main(ss:Rail[String]):void {main(ss);};
%


%~~vis
\begin{xten}
  public static def main(Rail[String]):void {
     val X = new Poly([0,1]);
     val t <: Poly = 7 * X + 6 * X * X * X; 
     val u <: Poly = 3 + 5*X - 7*X*X;
     val v <: Poly = t * u - 1;
     for( i in -3 .. 3) {
       x10.io.Console.OUT.println(
         "" + i + "	X:" + X(i) + "	t:" + t(i) 
         + "	u:" + u(i) + "	v:" + v(i)
         );
     }
  }

\end{xten}
%~~siv
%}
%~~neg

Writing the same code with method calls, while possible, is far less elegant: 
%~~gen ^^^ Classes170

%package Classes.In.Remedial.Poly101;
% // Integer-coefficient polynomials of one variable.
% class UglyPoly {
%   public val coeff : Rail[Int];
%   public def this(coeff: Rail[Int]) { this.coeff = coeff;}
%   public def degree() = (coeff.size-1) as Int;
%   public  def  a(i:Int) = (i<0 || i>this.degree()) ? 0 : coeff(i);
%
%   public static operator (c : Int) as UglyPoly = new UglyPoly([c as Int]);
%
%   public def apply(x:Int) {
%     val d = this.degree();
%     var s : Int = this.a(d);
%     for( i in 1 .. this.degree() ) {
%        s = x * s + a(d-i);
%     }
%     return s;
%   }
%
%   public operator this + (p:UglyPoly) =  new UglyPoly(
%      new Rail[Int](
%         Math.max(this.coeff.size, p.coeff.size) as Int,
%         (i:Int) => this.a(i) + p.a(i)
%      ));
%   public operator this - (p:UglyPoly) = this + (-1)*p;
%
%   public operator this * (p:UglyPoly) = new UglyPoly(
%      new Rail[Int](
%        this.degree() + p.degree() + 1,
%        (k:Int) => sumDeg(k, this, p)
%        )
%      );
%
%
%   public operator (n : Int) + this = (n as UglyPoly) + this;
%   public operator this + (n : Int) = (n as UglyPoly) + this;
%
%   public operator (n : Int) - this = (n as UglyPoly) + (-1) * this;
%   public operator this - (n : Int) = ((-n) as UglyPoly) + this;
%
%   public operator (n : Int) * this = new UglyPoly(
%      new Rail[Int](
%        this.degree()+1,
%        (k:Int) => n * this.a(k)
%      ));
%   private static def sumDeg(k:Int, a:UglyPoly, b:UglyPoly) {
%      var s : Int = 0;
%      for( i in 0 .. k ) s += a.a(i) * b.a(k-i);
%        // x10.io.Console.OUT.println("sumdeg(" + k + "," + a + "," + b + ")=" + s);
%      return s;
%      };
%   public final def toString() = {
%      var allZeroSoFar : Boolean = true;
%      var s : String ="";
%      for( i in 0..this.degree() ) {
%        val ai = this.a(i);
%        if (ai == 0) continue;
%        if (allZeroSoFar) {
%           allZeroSoFar = false;
%           s = term(ai, i);
%        }
%        else
%           s +=
%              (ai > 0 ? " + " : " - ")
%             +term(ai, i);
%      }
%      if (allZeroSoFar) s = "0";
%      return s;
%   }
%   private final def term(ai: Int, n:Int) = {
%      val xpow = (n==0) ? "" : (n==1) ? "x" : "x^" + n ;
%      return (ai == 1) ? xpow : "" + Math.abs(ai) + xpow;
%   }
%
%   def mult(p:UglyPoly) : UglyPoly = this * p;
%   def mult(n:Int)      : UglyPoly = n * this;
%   def plus(p:UglyPoly) : UglyPoly = this + p;
%   def plus(n:Int)      : UglyPoly = n + this;
%   def minus(p:UglyPoly): UglyPoly = this - p;
%   def minus(n:Int)     : UglyPoly = this - n;
%   static def const(n:Int): UglyPoly = n as UglyPoly;
%
%~~vis
\begin{xten}
  public static def uglymain() {
     val X = new UglyPoly([0,1]);
     val t <: UglyPoly 
           = X.mult(7).plus(
               X.mult(X).mult(X).mult(6));  
     val u <: UglyPoly 
           = const(3).plus(
               X.mult(5)).minus(X.mult(X).mult(7));
     val v <: UglyPoly = t.mult(u).minus(1);
     for( i in -3 .. 3) {
       x10.io.Console.OUT.println(
         "" + i + "	X:" + X.apply(i) + "	t:" + t.apply(i) 
          + "	u:" + u.apply(i) + "	v:" + v.apply(i)
         );
     }
  }
\end{xten}
%~~siv
%}
%~~neg

The operator-using code can be written in X10, though a few variations are
necessary to handle such exotic cases as \xcd`1+X`.



Most X10 operators can be given definitions.\footnote{Indeed, even for the
standard types, these operators are defined in the library.  Not even as basic
an operation as integer addition is built into the language.  Conversely, if
you define a full-featured numeric type, it will have most of the privileges that
the standard ones enjoy.  The missing priveleges are (1) literals; (2) 
the \xcd`..` operator won't compute the \xcd`zeroBased` and \xcd`rail`
properties as it does for \xcd`Int` ranges; (3) \xcd`*` won't track ranks, as
it does for \xcd`Region`s; 
(4) \xcd`&&` and \xcd`||` won't short-circuit, as they do for \xcd`Boolean`s, 
(5) the built-in notion of equality \xcd`a==b` will only coincide with
the programmible notion \xcd`a.equals(b)`, as they do for most library types,
if coded that way; and (6) it is 
impossible to define an 
operation like \xcd`String.+` which converts both its left and right arguments
from any type.  For example, a \xcd`Polar` type might
have many representations for the origin, as radius 0 and any angle; these
will be \xcd`equals()`, but will not be \xcd`==`; whereas for the standard
\xcd`Complex` type, the two equalities are the same.}  (However, \xcd`&&` and
\xcd`||` 
are only short-circuiting for \xcd`Boolean` expressions; user-defined versions
of these operators have no special execution behavior.)

The user-definable operations are (in order of precedence): \\
\begin{tabular}{l}
implicit type coercions\\
postfix \xcd`()`\\
\xcd`as T`\\
these unary operators:  \xcd`- + ! ~ | & / ^ * %`\\
\xcd`..`\\
\xcd`*      /       %      **`\\
\xcd`+` \xcd`     -` \\
\xcd`<<     >>      >>>    ->     <-     >-      -<      !`\\
\xcd`>      ` \xcd`>=     ` \xcd`<     ` \xcd`<=     ` 
\xcd`~      !~`\\
\xcd`&` \\
\xcd`^` \\
\xcd`|` \\
\xcd`&&` \\
\xcd`||` \\
\end{tabular}

Several of these operators have no standard meaning on any library type, and
are included purely for programmer convenience.  


Many operators may be defined either in \xcd`static` or instance forms.  Those
defined in instance form are dynamically dispatched, just like an instance
method.  Those defined in static form are statically dispatched, just like a
static method.  Operators are scoped like methods; static operators are scoped
like static methods.

\begin{ex}
%~~gen ^^^ Classes6a1j
% package oifClasses6a1j;
% class Whatever {
% 
%~~vis
\begin{xten}
static class Trace(n:Int){
  public static operator !(f:Trace) 
      = new Trace(10 * f.n + 1);
  public operator -this = new Trace (10 * this.n + 2);
}
static class Brace extends Trace{
  def this(n:Int) { super(n); }
  public operator -this = new Brace (10 * this.n + 3);
  static def example() {
     val t = new Trace(1);
     assert (!t).n == 11;
     assert (-t).n == 12 && (-t instanceof Trace);
     val b = new Brace(1);
     assert (!b).n == 11;
     assert (-b).n == 13 && (-b instanceof Brace);
  }
}

\end{xten}
%~~siv
% // And checking the unambiguous syntax while I'm here...
% //static class Glook { def checky(t:Trace) { 
% //   Trace.operator !(t);
% //   t.operator -();
% //} }
% }
%~~neg
\end{ex}

%%OP%% Operators may be invoked by unambiguous syntax, loosely akin to a
%%OP%% fully-qualified name. For example, \xcd`!t` above may be invoked as
%%OP%% \xcd`Trace.operator !(t)`. This unambiguous syntax may be used even if there
%%OP%% are several \xcd`!` operators that could apply to \xcd`t`, rendering the
%%OP%% convenient short form \xcd`!t` unavailable in some context.




\subsection{Binary Operators}

Binary operators, illustrated by \xcd`+`, may be defined statically in a
container \xcd`A` as:
\begin{xten}
static operator (b:B) + (c:C) = ...;
\end{xten}
%%OP%% In this case it may be invoked as \xcd`A.operator +(b,c)`.
Or, it may be defined as  as an instance operator by one of the forms:
\begin{xten}
operator this + (b:B) = ...;
operator (b:B) + this = ...;
\end{xten}
%%OP%% and be invoked as 
%%OP%% \xcd`a.operator +(b)`
%%OP%% and as 
%%OP%% \xcd`a.operator ()+(b)` 
%%OP%% respectively.

\begin{ex}

Defining the sum \xcd`P+Q` of two polynomials looks much like a method
definition.  It uses the \xcd`operator` keyword instead of \xcd`def`, and
\xcd`this` appears in the definition in the place that a \xcd`Poly` would
appear in a use of the operator.  So, 
\xcd`operator this + (p:Poly)` explains how to add \xcd`this` to a
\xcd`Poly` value.
%~~gen ^^^ Classes180
% package Classes.In.Poly102;
%~~vis
\begin{xten}
class Poly {
  public val coeff : Rail[Int];
  public def this(coeff: Rail[Int]) { 
    this.coeff = coeff;}
  public def degree() = coeff.size-1 as Int;
  public def  a(i:Int) 
    = (i<0 || i>this.degree()) ? 0 : coeff(i);
  public operator this + (p:Poly) =  new Poly(
     new Rail[Int](
        Math.max(this.coeff.size, p.coeff.size) as Int,
        (i:Int) => this.a(i) + p.a(i)
     )); 
  // ... 
\end{xten}
%~~siv
%   public operator (n : Int) + this = new Poly([n as Int]) + this;
%   public operator this + (n : Int) = new Poly([n as Int]) + this;
% 
%   def makeSureItWorks() {
%      val x = new Poly([0,1]);
%      val p <: Poly = x+x+x;
%      val q <: Poly = 1+x;
%      val r <: Poly = x+1;
%   }
%     
% }
%~~neg


The sum of a polynomial and an integer, \xcd`P+3`, looks like
an overloaded method definition.  
%~~gen ^^^ Classes190
% package Classes.In.Poly103;
% class Poly {
%   public val coeff : Rail[Int];
%   public def this(coeff: Rail[Int]) { this.coeff = coeff;}
%   public def degree() = coeff.size-1 as Int;
%   public def  a(i:Int) = (i<0 || i>this.degree()) ? 0 : coeff(i);
% 
%   public operator this + (p:Poly) =  new Poly(
%      new Rail[Int](
%         Math.max(this.coeff.size, p.coeff.size) as Int,
%         (i:Int) => this.a(i) + p.a(i)
%      ));
%    public operator (n : Int) + this = new Poly([n as Int]) + this;
%~~vis
\begin{xten}
   public operator this + (n : Int) 
          = new Poly([n as Int]) + this;
\end{xten}
%~~siv
% 
%   def makeSureItWorks() {
%      val x = new Poly([0,1]);
%      val p <: Poly = x+x+x;
%      val q <: Poly = 1+x;
%      val r <: Poly = x+1;
%   }
%     
% }
%~~neg


However, we want to allow the sum of an integer and a polynomial as well:
\xcd`3+P`.  It would be quite inconvenient to have to define this as a method
on \xcd`Int`; changing \xcd`Int` is far outside of normal coding.  So, we
allow it as a method on \xcd`Poly` as well.


%~~gen ^^^ Classes200
% package Classes.In.Poly104o;
% class Poly {
%   public val coeff : Rail[Int];
%   public def this(coeff: Rail[Int]) { this.coeff = coeff;}
%   public def degree() = coeff.size-1 as Int;
%   public def  a(i:Int) = (i<0 || i>this.degree()) ? 0 : coeff(i);
% 
%   public operator this + (p:Poly) =  new Poly(
%      new Rail[Int](
%         Math.max(this.coeff.size, p.coeff.size) as Int,
%         (i:Int) => this.a(i) + p.a(i)
%      ));
%~~vis
\begin{xten}
   public operator (n : Int) + this 
          = new Poly([n as Int]) + this;
\end{xten}
%~~siv
% 
%   public operator this + (n : Int) = new Poly([n as Int]) + this;
%   def makeSureItWorks() {
%      val x = new Poly([0,1]);
%      val p <: Poly = x+x+x;
%      val q <: Poly = 1+x;
%      val r <: Poly = x+1;
%   }
%     
% }
%~~neg

Furthermore, it is sometimes convenient to express a binary operation as a
static method on a class. 
The definition for the sum of two
\xcd`Poly`s could have been written:
%~~gen ^^^ Classes210
% package Classes.In.Poly105;
% class Poly {
%   public val coeff : Rail[Int];
%   public def this(coeff: Rail[Int]) { this.coeff = coeff;}
%   public def degree() = coeff.size-1 as Int;
%   public def  a(i:Int) = (i<0 || i>this.degree()) ? 0 : coeff(i);
%~~vis
\begin{xten}
  public static operator (p:Poly) + (q:Poly) =  new Poly(
     new Rail[Int](
        Math.max(q.coeff.size, p.coeff.size) as Int,
        (i:Int) => q.a(i) + p.a(i)
     ));
\end{xten}
%~~siv
%
%   public operator (n : Int) + this = new Poly([n as Int]) + this;
%   public operator this + (n : Int) = new Poly([n as Int]) + this;
% 
%   def makeSureItWorks() {
%      val x = new Poly([0,1]);
%      val p <: Poly = x+x+x;
%      val q <: Poly = 1+x;
%      val r <: Poly = x+1;
%   }
%     
% }
%~~neg

\end{ex}

When X10 attempts to typecheck a binary operator expression like \xcd`P+Q`, it
first typechecks \xcd`P` and \xcd`Q`. Then, it looks for operator declarations
for \xcd`+` in the types of \xcd`P` and \xcd`Q`. If there are none, it is a
static error. If there is precisely one, that one will be used. If there are
several, X10 looks for a {\em best-matching} operation, \viz{} one which does
not require the operands to be converted to another type. For example,
\xcd`operator this + (n:Long)` and \xcd`operator this + (n:Int)` both apply to
\xcd`p+1`, because \xcd`1` can be converted from an \xcd`Int` to a \xcd`Long`.
However, the \xcd`Int` version will be chosen because it does not require a
conversion. If even the best-matching operation is not uniquely determined,
the compiler will report a static error.


\subsection{Unary Operators}

Unary operators,  illustrated by \xcd`!`, may be defined statically in
container 
\xcd`A` as 
\begin{xten}
static operator !(x:A) = ...;
\end{xten}
or as instance operators by: 
\begin{xten}
operator !this = ...;
\end{xten}

%%OP%% A statically-defined unary operator \xcd`!` may be invoked on \xcd`a:A` as 
%%OP%% \xcd`A.operator !(a)`.  An instance operator may be invoked as
%%OP%% \xcd`a.operator !()`.  

The rules for typechecking a unary operation are the same as for methods; the
complexities of binary operations are not needed.

\begin{ex}
The operator to negate a polynomial is: 

%~~gen ^^^ Classes220
% package Classes.In.Poly106;
% class Poly {
%   public val coeff : Rail[Int];
%   public def this(coeff: Rail[Int]) { this.coeff = coeff;}
%   public def degree() = coeff.size-1 as Int;
%   public def  a(i:Int) = (i<0 || i>this.degree()) ? 0 : coeff(i);
%~~vis
\begin{xten}
  public operator - this = new Poly(
    new Rail[Int](coeff.size as Int, (i:Int) => -coeff(i))
    );
\end{xten}
%~~siv
%   def makeSureItWorks() {
%      val x = new Poly([0,1]);
%      val p <: Poly = -x;
%   }
% }
%~~neg



\end{ex}


\subsection{Type Conversions}
\label{sect:type-conv}
\index{type conversion!user-defined}


Explicit type conversions, \xcd`e as A`, can be defined as operators on
class \xcd`A`, or on the container type of \xcd`e`.  These must be static
operators.  

To define an operator in \xcd`class A` (or \xcd`struct A)` converting values
of type \xcd`B` into type \xcd`A`, use the syntax: 
\begin{xten}
static operator (x:B) as ? {c} = ... 
\end{xten}
The \xcd`?` indicates the containing type \xcd`A`.  
The guard clause \xcd`{c}` may be omitted.



\begin{ex}
%~~gen ^^^ Classes230
% package Classes_explicit_type_conversions_a;
%~~vis
\begin{xten}
class Poly {
  public val coeff : Rail[Int];
  public def this(coeff: Rail[Int]) { this.coeff = coeff;}
  public static operator (a:Int) as ? = new Poly([a as Int]);
  public static def main(Rail[String]):void {
     val three : Poly = 3 as Poly;
  }
}
\end{xten}
%~~siv
%
%~~neg
\end{ex}
The \xcd`?` may be given a bound, such as \xcd`as ? <: Caster`, if desired.
  

There is little difference between an explicit conversion \xcd`e as T` and a
method call \xcd`e.asT()`.  The explicit conversion does say undeniably what
the result type will be.  However, as described in \Sref{sect:ambig-cast},
sometimes the built-in meaning of \xcd`as` as a cast overrides the
user-defined explicit conversion.  

Explicit casts are most suitable for cases
which resemble the use of explicit casts among the arithmetic types, where, 
for example, \xcd`1.0 as Int` is a way to turn a floating-point number into the
corresponding integer.  
While there is nothing in X10 which
requires it, \xcd`e as T` has the connotation that it gives a good
approximation of \xcd`e` in type \xcd`T`, just as \xcd`1` is a good
(indeed, perfect) approximation of \xcd`1.0` in type \xcd`Int`.  

\subsection{Implicit Type Coercions}
\label{sect:ImplicitCoercion}
\index{type conversion!implicit}

An implicit type conversion from \xcd`U`  to \xcd`T` may be specified in
container \xcd`T`.  
The syntax for it is: 
\begin{xten}
static operator (u:U) : T = e;
\end{xten}
%%OP%% which may be invoked by the unambiguous syntax 
%%OP%% \xcd`T.operator[T](u)` or \xcd`U.operator[T](u)`.
%%OP%% 



Implicit coercions are used automatically by the compiler on method calls 
(\Sref{sect:MethodResolution}) and assignments (\Sref{AssignmentStatement}).
Implicit coercions may be used when a value of type \xcd`T` appears in a
context expecting a value of type \xcd`U`.  If \xcd`T <: U`, no implicit
coercion is needed; \eg, a method \xcd`m` expecting an \xcd`Int` argument may 
be called as \xcd`m(3)`, with an argument of type \xcd`Int{self==3}`, which is
a subtype of \xcd`m`'s argument type \xcd`Int`. 
However, if it is not the case that \xcd`T <: U`, but there is an implicit
coercion from \xcd`T` to \xcd`U` defined in container \xcd`U`, then this
implicit coercion will be applied.

\begin{ex}
We can define an implicit coercion from \xcd`Int` to \xcd`Poly`,
and avoid having to define the sum of an integer and a polynomial
as many special cases.  In the following example, we only define \xcd`+` on
two polynomials.  The
calculation \xcd`1+x` coerces \xcd`1` to a polynomial and uses polynomial
addition to add it to \xcd`x`.

%~~gen ^^^ Classes240
% package Classes.And.Implicit.Coercions;
% class Poly {
%   public val coeff : Rail[Int];
%   public def this(coeff: Rail[Int]) { this.coeff = coeff;}
%   public def degree() = (coeff.size-1) as Int;
%   public def  a(i:Int) = (i<0 || i>this.degree()) ? 0 : coeff(i);
%   public final def toString() = {
%      var allZeroSoFar : Boolean = true;
%      var s : String ="";
%      for( i in 0..this.degree() ) {
%        val ai = this.a(i);
%        if (ai == 0) continue;
%        if (allZeroSoFar) {
%           allZeroSoFar = false;
%           s = term(ai, i);
%        }
%        else 
%           s += 
%              (ai > 0 ? " + " : " - ")
%             +term(ai, i);
%      }
%      if (allZeroSoFar) s = "0";
%      return s;
%   }
%   private final def term(ai: Int, n:Int) = {
%      val xpow = (n==0) ? "" : (n==1) ? "x" : "x^" + n ;
%      return (ai == 1) ? xpow : "" + Math.abs(ai) + xpow;
%   }

%~~vis
\begin{xten}
  public static operator (c : Int) : Poly 
     = new Poly([c as Int]);

  public static operator (p:Poly) + (q:Poly) = new Poly(
      new Rail[Int](
        Math.max(p.coeff.size, q.coeff.size) as Int,
        (i:Int) => p.a(i) + q.a(i)
     ));

  public static def main(Rail[String]):void {
     val x = new Poly([0,1]);
     x10.io.Console.OUT.println("1+x=" + (1+x));
  }
\end{xten}
%~~siv
%}
%~~neg
\end{ex}



\subsection{Assignment and Application Operators}
\index{assignment operator}
\index{application operator}
\index{()}
\index{()=}
\label{set-and-apply}
X10 allows types to implement the subscripting / function application
operator, and indexed assignment.  The \xcd`Array`-like classes take advantage
of both of these in \xcd`a(i) = a(i) + 1`.  

\xcd`a(b,c,d)`
is an operator call, to an operator defined with 
\xcd`public operator this(b:B, c:C, d:D)`.  It may be overloaded.
For
example, an ordered dictionary structure could allow subscripting by numbers
with \xcd`public operator this(i:Int)`, and by strings with 
\xcd`public operator this(s:String)`.  


\xcd`a(i,j)=b` is an \xcd`operator` as well, with zero or more indices
\xcd`i,j`.  It may also be overloaded. 

The update operations \xcd`a(i) += b` 
(for all binary operators in place of \xcd`+`)
are defined to be the same as the
corresponding \xcd`a(i) = a(i) + b`. This applies for all arities of
arguments, and all types, and all binary operations. Of course to use this,
the \xcd`+`, application and assignment \xcd`operator`s must be defined.


\begin{ex}

The \xcd`Oddvec` class of somewhat peculiar vectors illustrates this.

\xcd`a()` returns a string representation of the oddvec, which ordinarily
would 
0be done by \xcd`toString()` instead.  
\xcd`a(i)` sensibly picks out one of the three
coordinates of \xcd`a`.
\xcd`a()=b` sets all the coordinates of \xcd`a` to \xcd`b`.
\xcd`a(i)=b` assigns to one of the
coordinates.  \xcd`a(i,j)=b` assigns different values to \xcd`a(i)` and
\xcd`a(j)`.  

%~~gen ^^^ Classes250
% package Classes.Assignments1_oddvec;
%~~vis
\begin{xten}
class Oddvec {
  var v : Rail[Int] = new Rail[Int](3);
  public operator this () = 
      "(" + v(0) + "," + v(1) + "," + v(2) + ")";
  public operator this () = (newval: Int) { 
    for(p in v.range) v(p) = newval;
  }
  public operator this(i:Int) = v(i);
  public operator this(i:Int, j:Int) = [v(i),v(j)];
  public operator this(i:Int) = (newval:Int) 
      = {v(i) = newval;}
  public operator this(i:Int, j:Int) = (newval:Int) 
      = { v(i) = newval; v(j) = newval+1;} 
  public def example() {
    this(1) = 6;   assert this(1) == 6;
    this(1) += 7;  assert this(1) == 13;
  }
\end{xten}
%~~siv
% }
%  class Hook { def run() {
%     val a = new Oddvec();
%     assert a().equals("(0,0,0)");
%     a() = 1;
%     assert a().equals("(1,1,1)");
%     a(1) = 4;
%     assert a().equals("(1,4,1)");
%     a(0,2) = 5;
%     assert a().equals("(5,4,6)");
%     return true;
%   }
% }
%~~neg

\end{ex}

\section{Class Guards and Invariants}\label{DepType:ClassGuard}
\index{type invariants}
\index{class invariants}
\index{invariant!type}
\index{invariant!class}
\index{guard}


Classes (and structs and interfaces) may specify a {\em class guard}, a
constraint which must hold on all values of the class.    In the following
example, a \xcd`Line` is defined by two distinct \xcd`Pt`s\footnote{We use \xcd`Pt`
to avoid any possible confusion with the built-in class \xcd`Point`.}
%~~gen ^^^ Classes260
% package classes.guards.invariants.glurp;
%~~vis
\begin{xten}
class Pt(x:Int, y:Int){}
class Line(a:Pt, b:Pt){a != b} {}
\end{xten}
%~~siv
%
%~~neg

In most cases the class guard could be phrased as a type constraint on a property of
the class instead, if preferred.  Arguably, a symmetric constraint like two
points being different is better expressed as a class guard, rather than
asymmetrically as a constraint on one type: 
%~~gen ^^^ Classes270
% package classes.guards.invariants.glurp2;
% class Pt(x:Int, y:Int){}
%~~vis
\begin{xten}
class Line(a:Pt, b:Pt{a != b}) {}
\end{xten}
%~~siv
%
%~~neg



\label{DepType:TypeInvariant}
\index{class invariant}
\index{invariant!class}
\index{class!invariant}
\label{DepType:ClassGuardDef}



With every container  or interface \xcd"T" we associate a {\em type
invariant} $\mathit{inv}($\xcd"T"$)$, which describes the guarantees on the
properties of values of type \xcd`T`.  

Every value of \xcd`T` satisfies $\mathit{inv}($\xcd"T"$)$ at all times.  This
is somewhat stronger than the concept of type invariant in most languages
(which only requires that the invariant holds when no method calls are
active).  X10 invariants only concern properties, which are immutable; thus,
once established, they cannot be falsified.

The type
invariant associated with \xcd"x10.lang.Any"
is 
\xcd"true".

The type invariant associated with any interface or struct \xcd"I" that extends
interfaces \xcdmath"I$_1$, $\dots$, I$_k$" and defines properties
\xcdmath"x$_1$: P$_1$, $\dots$, x$_n$: P$_n$" and
specifies a guard \xcd"c" is given by:

\begin{xtenmath}
$\mathit{inv}$(I$_1$) && $\dots$ && $\mathit{inv}$(I$_k$) &&
self.x$_1$ instanceof P$_1$ &&  $\dots$ &&  self.x$_n$ instanceof P$_n$ 
&& c  
\end{xtenmath}

Similarly the type invariant associated with any class \xcd"C" that
implements interfaces \xcdmath"I$_1$, $\dots$, I$_k$",
extends class \xcd"D" and defines properties
\xcdmath"x$_1$: P$_1$, $\dots$, x$_n$: P$_n$" and
specifies a guard \xcd"c" is
given by the same thing with the invariant of the superclass \xcd`D` conjoined:
\begin{xtenmath}
$\mathit{inv}$(I$_1$) && $\dots$ && $\mathit{inv}$(I$_k$) 
&& self.x$_1$ instanceof P$_1$ &&  $\dots$ &&  self.x$_n$ instanceof P$_n$ 
&& c  
&& $\mathit{inv}$(D)
\end{xtenmath}


Note that the type invariant associated with a class entails the type
invariants of each interface that it implements (directly or indirectly), and
the type invariant of each ancestor class.
It is guaranteed that for any variable \xcd"v" of
type \xcd"T{c}" (where \xcd"T" is an interface name or a class name) the only
objects \xcd"o" that may be stored in \xcd"v" are such that \xcd"o" satisfies
$\mathit{inv}(\mbox{\xcd"T"}[\mbox{\xcd"o"}/\mbox{\xcd"this"}])
\wedge \mbox{\xcd"c"}[\mbox{\xcd"o"}/\mbox{\xcd"self"}]$.



\subsection{Invariants for {\tt implements} and {\tt extends} clauses}\label{DepType:Implements}
\label{DepType:Extends}
\index{type-checking!implements clause}
\index{type-checking!extends clause}
\index{implements}
\index{extends}
Consider a class definition
\begin{xtenmath}
$\mbox{\emph{ClassModifiers}}^{\mbox{?}}$
class C(x$_1$: P$_1$, $\dots$, x$_n$: P$_n$){c} extends D{d}
   implements I$_1${c$_1$}, $\dots$, I$_k${c$_k$}
$\mbox{\emph{ClassBody}}$
\end{xtenmath}

These two rules must be satisfied:


\begin{itemize}

\item 
The type invariant \xcdmath"$\mathit{inv}$(C)" of \xcd"C" must entail
\xcdmath"c$_i$[this/self]" for each $i$ in $\{1, \dots, k\}$


\item The return type \xcd"c" of each constructor in a class \xcd`C`
must entail the invariant \xcdmath"$\mathit{inv}$(C)".
\end{itemize}

\subsection{Timing of Invariant Checks}

\index{invariant!checked}

The invariants for a container are checked immediately after the
\xcd`property` statement in the container's constructor. 
This is the earliest that the invariant could possibly be checked. 
Recall that an invariant 
can mention the properties of the container (which are set, forever, at that
point in the code), but cannot mention the \xcd`val`
or \xcd`var` fields (which might not be set at that point), or \xcd`this`
(which might not have been fully initialized).  

If X10 can prove that the invariant always holds given the \xcd`property`
statement and other known information, it may omit the actual check.




\subsection{Invariants and constructor definitions}
\index{invariant!and constructor}
\index{constructor!and invariant}

A constructor for a class \xcd"C" is guaranteed to return an object of the
class on successful termination. This object must satisfy  \xcdmath"$\mathit{inv}$(C)", the
class invariant associated with \xcd"C" (\Sref{DepType:TypeInvariant}).
However,
often the objects returned by a constructor may satisfy {\em stronger}
properties than the class invariant. \Xten{}'s dependent type system
permits these extra properties to be asserted with the constructor in
the form of a constrained type (the ``return type'' of the constructor):

%##(CtorDeclaration
\begin{bbgrammar}
%(FROM #(prod:CtorDecln)#)
           CtorDecln \: Mods\opt \xcd"def" \xcd"this" TypeParams\opt Formals Guard\opt HasResultType\opt CtorBody & (\ref{prod:CtorDecln}) \\
\end{bbgrammar}
%##)

\label{ConstructorGuard}

The parameter list for the constructor
may specify a \emph{guard} that is to be satisfied by the parameters
to the list.

\begin{ex}
%%TODO--rewrite this
Here is another example, constructed as a simplified 
version of \Xcd{x10.array.Region}.  The \xcd`mockUnion` method 
has the type, though not the value, that a true \xcd`union` method would have.

%~~gen ^^^ Classes280
%package Classes.SimplifiedRegion;
%~~vis
\begin{xten}
class MyRegion(rank:Int) {
  static type MyRegion(n:Int)=MyRegion{rank==n};
  def this(r:Int):MyRegion(r) {
    property(r);
  }
  def this(diag:Rail[Int]):MyRegion(diag.size){ 
    property(diag.size);
  }
  def mockUnion(r:MyRegion(rank)):MyRegion(rank) = this;
  def example() {
    val R1 : MyRegion(3) = new MyRegion([4,4,4 as Int]); 
    val R2 : MyRegion(3) = new MyRegion([5,4,1]); 
    val R3 = R1.mockUnion(R2); // inferred type MyRegion(3)
  }
}
\end{xten}
%~~siv
%
%~~neg
The first constructor returns the empty region of rank \Xcd{r}.  The
second constructor takes a \Xcd{Array[Int](1)} of arbitrary length
\Xcd{n} and returns a \Xcd{MyRegion(n)} (intended to represent the set
of points in the rectangular parallelopiped between the origin and the
\Xcd{diag}.)

The code in \xcd`example` typechecks, and \xcd`R3`'s type is inferred as
\xcd`MyRegion(3)`.  


\end{ex}

   Let \xcd"C" be a class with properties
   \xcdmath"p$_1$: P$_1$, $\dots$, p$_n$: P$_n$", and invariant \xcd"c"
   extending the constrained type \xcd"D{d}" (where \xcd"D" is the name of a
   class).



   For every constructor in \xcd"C" the compiler checks that the call to
   super invokes a constructor for \xcd"D" whose return type is strong enough
   to entail \xcd"d". Specifically, if the call to super is of the form 
     \xcdmath"super(e$_1$, $\dots$, e$_k$)"
   and the static type of each expression \xcdmath"e$_i$" is
   \xcdmath"S$_i$", and the invocation
   is statically resolved to a constructor
\xcdmath"def this(x$_1$: T$_1$, $\dots$, x$_k$: T$_k$){c}: D{d$_1$}"
   then it must be the case that 
\begin{xtenmath}
x$_1$: S$_1$, $\dots$, x$_i$: S$_i$ entails x$_i$: T$_i$  (for $i \in \{1, \dots, k\}$)
x$_1$: S$_1$, $\dots$, x$_k$: S$_k$ entails c  
d$_1$[a/self], x$_1$: S$_1$, ..., x$_k$: S$_k$ entails d[a/self]      
\end{xtenmath}
\noindent where \xcd"a" is a constant that does not appear in 
\xcdmath"x$_1$: S$_1$ $\wedge$ ... $\wedge$ x$_k$: S$_k$".

   The compiler checks that every constructor for \xcd"C" ensures that
   the properties \xcdmath"p$_1$,..., p$_n$" are initialized with values which satisfy
   $\mathit{inv}($\xcd"T"$)$, and its own return type \xcd"c'" as follows.  In each constructor, the
   compiler checks that the static types \xcdmath"T$_i$" of the expressions \xcdmath"e$_i$"
   assigned to \xcdmath"p$_i$" are such that the following is
   true:
\begin{xtenmath}
p$_1$: T$_1$, $\dots$, p$_n$: T$_n$ entails $\mathit{inv}($T$)$ $\wedge$ c'     
\end{xtenmath}

(Note that for the assignment of \xcdmath"e$_i$" to \xcdmath"p$_i$"
to be type-correct it must be the
    case that \xcdmath"p$_i$: T$_i$ $\wedge$ p$_i$: P$_i$".) 



The compiler must check that every invocation \xcdmath"C(e$_1$, $\dots$, e$_n$)" to a
constructor is type correct: each argument \xcdmath"e$_i$" must have a static type
that is a subtype of the declared type \xcdmath"T$_i$" for the $i$th
argument of the
constructor, and the conjunction of static types of the argument must
entail the constraint in the parameter list of the constructor.

\section{Generic Classes}

Classes, like other units, can be generic.  They can be parameterized by
types.  The parameter types are used just like ordinary types inside the body
of the generic class -- with a few exceptions.  

\begin{ex}
A \xcd`Colorized[T]` holds a thing of type \xcd`T`, and a string which is intended to represent
its color.  Any type can be used for \xcd`T`; the \xcd`example` method shows
\xcd`Int` and \xcd`Boolean`.  The \xcd`thing()` method retrieves the thing;
note that its return type is the generic type variable \xcd`T`.  X10 is aware 
that \xcd`colInt.thing()` is an \xcd`Int` and \xcd`colTrue.thing()` is a
\xcd`Boolean`, and uses those typings in \xcd`example`. 
%~~gen ^^^ Classes6d9u
% package Classes6d9u;
%~~vis
\begin{xten}
class Colorized[T] {
  private var thing:T; 
  private var color:String; 
  def this(thing:T, color:String) {
     this.thing = thing;
     this.color = color;
  }
  public def thing():T = thing;
  public def color():String = color;  
  public static def example() {
    val colInt  : Colorized[Int] 
                = new Colorized[Int](3, "green");
    assert colInt.thing() == 3 
                && colInt.color().equals("green");
    val colTrue : Colorized[Boolean] 
                = new Colorized[Boolean](true, "blue");
    assert colTrue.thing() 
                && colTrue.color().equals("blue");
  }
}
\end{xten}
%~~siv
%class Hook{ def run() {Colorized.example(); return true;}}
%~~neg


\end{ex}



\subsection{Use of Generics}

An unconstrained type variable \Xcd{X} can be instantiated by any type. All
the operations of \Xcd{Any} are available on a 
variable of type \Xcd{X}. Additionally, variables of type
\Xcd{X} may be used with \Xcd{==, !=}, in \Xcd{instanceof}, and casts.  

If a type variable is constrained, the operations implied by its constraint
are available as well.

\begin{ex}
The interface \xcd`Named` describes entities which know their own name.  The
class \xcd`NameMap[T]` is a specialized map which stores and retrieves
\xcd`Named` entities by name.  The call \xcd`t.name()` in \xcd`put()` is only
valid because the constraint \xcd`{T <: Named}` implies that \xcd`T` is a
subtype of \xcd`Named`, and hence provides all the operations of \xcd`Named`. 
%~~gen ^^^ Types6e6x
% package Types6e6x;
% import x10.util.*;
%~~vis
\begin{xten}
interface Named { def name():String; }
class NameMap[T]{T <: Named} {
   val m = new HashMap[String, T]();
   def put(t:T) { m.put(t.name(), t); }
   def get(s:String):T = m.getOrThrow(s);
}
\end{xten}
%~~siv
%
%~~neg


\end{ex}





\section{Object Initialization}
\label{ObjectInitialization}
\index{initialization}
\index{constructor}
\index{object!constructor}
\index{struct!constructor}

% \noo{Confirm this chapter with the paper}

X10 does object initialization safely.  It avoids certain bad things which
trouble some other languages:
\begin{enumerate}
\item Use of a field before the field has been initialized.
\item A program reading two different values from a \xcd`val` field of a
      container. 
\item \Xcd{this} escaping from a constructor, which can cause problems as
      noted below. 

\end{enumerate}

It should be unsurprising that fields must not be used before they are
initialized. At best, it is uncertain what value will be in them, as in
\Xcd{x} below. Worse, the value might not even be an allowable value; \Xcd{y},
declared to be nonzero in the following example, might be zero before it is
initialized.
\begin{xten}
// Not correct X10
class ThisIsWrong {
  val x : Int;
  val y : Int{y != 0};
  def this() {
    x10.io.Console.OUT.println("x=" + x + "; y=" + y);
    x = 1; y = 2;
  }
}
\end{xten}

One particularly insidious way to read uninitialized fields is to allow
\Xcd{this} to escape from a constructor. For example, the constructor could
put \Xcd{this} into a data structure before initializing it, and another
activity could read it from the data structure and look at its fields:
\begin{xten}
class Wrong {
  val shouldBe8 : Int;
  static Cell[Wrong] wrongCell = new Cell[Wrong]();
  static def doItWrong() {
     finish {
       async { new Wrong(); } // (A)
       assert( wrongCell().shouldBe8 == 8); // (B)
     }
  }
  def this() {
     wrongCell.set(this); // (C) - ILLEGAL
     this.shouldBe8 = 8; // (D)
  }
}
\end{xten}
\noindent
In this example, the underconstructed \Xcd{Wrong} object is leaked into a
storage cell at line \Xcd{(C)}, and then initialized.  The \Xcd{doItWrong}
method constructs a new \Xcd{Wrong} object, and looks at the \Xcd{Wrong}
object in the storage cell to check on its \Xcd{shouldBe8} field.  One
possible order of events is the following:
\begin{enumerate}
\item \Xcd{doItWrong()} is called.
\item \Xcd{(A)} is started.  Space for a new \Xcd{Wrong} object is allocated.
      Its \Xcd{shouldBe8} field, not yet initialized, contains some garbage
      value.
\item \Xcd{(C)} is executed, as part of the process of constructing a new
      \Xcd{Wrong} object.  The new, uninitialized object is stored in
      \Xcd{wrongCell}.
\item Now, the initialization activity is paused, and execution of the main activity
      proceeds from \Xcd{(B)}.
\item The value in \Xcd{wrongCell} is retrieved, and is \Xcd{shouldBe8} field
      is read.  This field contains garbage, and the assertion fails.
\item Now let the initialization activity proceed with \Xcd{(D)},
      initializing \Xcd{shouldBe8} --- too late.
\end{enumerate}

The \xcd`at` statement (\Sref{AtStatement}) introduces the potential for
escape as well. The following class prints an uninitialized value: 
%~~gen ^^^ ThisEscapingViaAt_MustFailCompile
% package ObjInit_at;
% NOCOMPILE
%~~vis
\begin{xten}
// This code violates this chapter's constraints
// and thus will not compile in X10.
class Example {
  val a: Int;
  def this() { 
    at(here.next()) {
      // Recall that 'this' is a copy of 'this' outside 'at'.
      Console.OUT.println("this.a = " + this.a);
    }
    this.a = 1;
  }
}
\end{xten}
%~~siv
%
%~~neg


X10 must protect against such possibilities.  The rules explaining how
constructors can be written are somewhat intricate; they are designed to allow
as much programming as possible without leading to potential problems.
Ultimately, they simply are elaborations of the fundamental principles that
uninitialized fields must never be read, and \Xcd{this} must never be leaked.

%%RAW%% \subsection{Raw and Cooked Objects}
%%RAW%% \index{raw}
%%RAW%% \index{cooked}
%%RAW%% 
%%RAW%% An object is {\em raw} before its constructor ends, and {\em cooked} after its
%%RAW%% constructor ends. Note that, when an object is cooked, all its subobjects are
%%RAW%% cooked.  
%%RAW%% 



\subsection{Constructors and Non-Escaping Methods}
\index{non-escaping}
\label{sect:nonescaping}

In general, constructors must not be allowed to call methods with \Xcd{this} as
an argument or receiver. Such calls could leak references to \Xcd{this},
either directly from a call to \Xcd{cell.set(this)}, or indirectly because
\Xcd{toString} leaks \Xcd{this}, and the concatenation
\Xcd`"Escaper = "+this` calls \Xcd{toString}.\footnote{This is abominable behavior for
\Xcd{toString}, but it cannot be prevented -- save by a scheme such as we
present in this section.}
%~WRONG~gen
%package ObjectInit.CtorAndNonEscaping.One;
%~WRONG~vis
\begin{xten}
// This code violates this chapter's constraints
// and thus will not compile in X10.
class Escaper {
  static val Cell[Escaper] cell = new Cell[Escaper]();
  def toString() {
    cell.set(this);
    return "Evil!";
  }
  def this() {
    cell.set(this);
    x10.io.Console.OUT.println("Escaper = " + this);
  }
}
\end{xten}
%~WRONG~siv
%
%~WRONG~neg

However, it is convenient to be able to call methods from constructors; {\em
e.g.}, a class might have eleven constructors whose common behavior is best
described by three methods.
Under certain stringent conditions, it {\em is}
safe to call a method: the method called must not leak references to
\Xcd{this}, and must not read \Xcd{val}s or \Xcd{var}s which might not have
been assigned.

So, X10 performs a static dataflow analysis, sufficient to guarantee that
method calls in constructors are safe.  This analysis requires having access
to or guarantees about all the code that could possibly be called.  This can
be accomplished in two ways:
\begin{enumerate}
\item Ensuring that only code from the class itself can be called, by
      forbidding overriding of
      methods called from the constructor: they can be marked \Xcd{final} or
      \Xcd{private}, or the whole class can be \Xcd{final}.
\item Marking the methods called from the constructor by
      \xcd`@NonEscaping` or \xcd`@NoThisAccess`
\end{enumerate}

\subsubsection{Non-Escaping Methods}
\index{method!non-escaping}
\index{method!implicitly non-escaping}
\index{method!NonEscaping}
\index{implicitly non-escaping}
\index{non-escaping}
\index{non-escaping!implicitly}
\index{NonEscaping}


A method may be annotated with \xcd`@NonEscaping`.  This
imposes several restrictions on the method body, and on all methods overriding
it.  However, it is the only way that a method can be called from
constructors.  The
\Xcd{@NonEscaping} annotation makes explicit all the X10 compiler's needs for
constructor-safety.

A method can, however, be safe to call from constructors without being marked
\Xcd{@NonEscaping}. We call such methods {\em implicitly non-escaping}.
Implicitly non-escaping methods need to obey the same constraints on
\Xcd{this}, \Xcd{super}, and variable usage as \Xcd{@NonEscaping} methods. An
implicitly non-escaping method {\em could} be marked as
\xcd`@NonEscaping`; the compiler, in
effect, infers the annotation. In addition, all non-escaping methods
must be \Xcd{private} or \Xcd{final} or members of a \Xcd{final} class; this
corresponds to the hereditary nature of \Xcd{@NonEscaping} (by forbidding
inheritance of implicitly non-escaping methods).

We say that a method is {\em non-escaping} if it is either implicitly
non-escaping, or annotated \Xcd{@NonEscaping}.

The first requirement on non-escaping methods is that they do not allow
\Xcd{this} to escape. Inside of their bodies, \Xcd{this} and \Xcd{super} may
only be used for field access and assignment, and as the receiver of
non-escaping methods.


The following example uses the possible variations.  \Xcd{aplomb()} 
explicitly forbids reading any field but
\Xcd{a}. \Xcd{boric()} is called after \Xcd{a} and \Xcd{b} are set, but
\Xcd{c} is not.
The \xcd`@NonEscaping` annotation on \xcd`boric()` is optional, but the
compiler will print a warning if it is left out.
\Xcd{cajoled()} is only called after all fields are set, so it
can read anything; its annotation, too, is not required.   \Xcd{SeeAlso} is able to override \Xcd{aplomb()}, because
\Xcd{aplomb()} is \xcd`@NonEscaping`; it cannot override the final method
\Xcd{boric()} or the private one \Xcd{cajoled()}.  
%~~gen ^^^ ObjectInitialization10
%package ObjInit.C2;
%~~vis
\begin{xten}
import x10.compiler.*;

final class C2 {
  protected val a:Int; protected val b:Int; protected val c:Int;
  protected var x:Int; protected var y:Int; protected var z:Int;
  def this() {
    a = 1;
    this.aplomb();
    b = 2;
    this.boric();
    c = 3;
    this.cajoled();
  }
  @NonEscaping def aplomb() {
    x = a;
    // this.boric(); // not allowed; boric reads b.
    // z = b; // not allowed -- only 'a' can be read here
  }
  @NonEscaping final def boric() {
    y = b;
    this.aplomb(); // allowed; 
       // a is definitely set before boric is called
    // z = c; // not allowed; c is not definitely written
  }
  @NonEscaping private def cajoled() {
    z = c;
  }
}

\end{xten}
%~~siv
%
%~~neg

\subsubsection{NoThisAccess Methods}

A method may be annotated \xcd`@NoThisAccess`.  \xcd`@NoThisAccess` methods
may be called from constructors, and they may be overridden in subclasses.
However, they may not refer to \xcd`this` in any way -- in particular, they
cannot refer to fields of \xcd`this`, nor to \xcd`super`.

\begin{ex}

The class \xcd`IDed` has an \xcd`Float`-valued \xcd`id` field.  The method
\xcd`count()` is used to initialize the \xcd`id`.  For \xcd`IDed` objects,
the \xcd`id` is the count of \xcd`IDed`s created with the same parity of its
\xcd`kind`.   Note that \xcd`count()` does not refer to \xcd`this`, though
it does refer to a \xcd`static` field \xcd`counts`. 

The subclass \xcd`SubIDed` has \xcd`id`s that depend on \xcd`kind%3`
as well as the parity of \xcd`kind`.  It overrides the \xcd`count()`
method.  The body of \xcd`count()` still cannot refer to \xcd`this`.
Nor can it refer to \xcd`super` (which is \xcd`self` under another name).
This precludes the use of a \xcd`super` call.  This is why we have separated
the body of \xcd`count` out as the static method \xcd`kind2count` -- without
that, we would have had to duplicate its body, as we could not call 
\xcd`super.count(kind)` in a \xcd`NoThisAccess` method, as is shown by 
the \xcd`ERROR` line \xcd`(A)`. 

Note that \xcd`NoThisAccess` is in \xcd`x10.compiler` and must be imported,
and that the overriding method \xcd`SubIDed.count` must be declared
\xcd`@NoThisAccess` as well as the overridden method.
Line \xcd`(B)` is not allowed because \xcd`code` is a field of \xcd`this`, 
and field accesses are forbidden.   Line \xcd`(C)` references \xcd`this`
directly, which, of course, is forbidden by \xcd`@NoThisAccess`.  


%~~gen ^^^ ObjectInitialization7p2v
% package ObjectInitialization7p2v;
%~~vis
\begin{xten}
import x10.compiler.*;
class UseNoThisAccess {
  static class IDed {
    protected static val counts = [0 as Int,0];
    protected var code : Int;
    val id: Float;
    public def this(kind:Int) { 
      code = kind;
      this.id = this.count(kind); 
    }
    protected static def kind2count(kind:Int) = ++counts(kind % 2);
    @NoThisAccess def count(kind:Int) : Float = kind2count(kind);
  }
  static class SubIDed extends IDed {
    protected static val subcounts = [0 as Int, 0, 0];
    public static val all = new x10.util.ArrayList[SubIDed]();
    public def this(kind:Int) { 
       super(kind); 
    }
    @NoThisAccess
    def count(kind:Int) : Float {
       val subcount <: Int = ++subcounts(kind % 3);
       val supercount <: Float = kind2count(kind);
       //ERROR: val badSuperCount = super.count(kind); //(A)
       //ERROR: code = kind;                           //(B)
       //ERROR: all.add(this);                         //(C)
       return  supercount + 1.0f / subcount;
    }
  }
}
\end{xten}
%~~siv
%
%~~neg


\end{ex}

\subsection{Fine Structure of Constructors}
\label{SFineStructCtors}

The code of a constructor consists of four segments, three of them optional
and one of them implicit.
\begin{enumerate}
\item The first segment is an optional call to \Xcd{this(...)} or
      \Xcd{super(...)}.  If this is supplied, it must be the first statement
      of the constructor.  If it is not supplied, the compiler treats it as a
      nullary super-call \Xcd{super()};
\item If the class or struct has properties, there must be a single
      \Xcd{property(...)} command in the constructor, or a \xcd`this(...)`
      constructor call.  Every execution path
      through the constructor must go through this \Xcd{property(...)} command
      precisely once.   The second segment of the constructor is the code
      following the first segment, up to and including the \Xcd{property()}
      statement.

      If the class or struct has no properties, the \Xcd{property()} call must
      be omitted. If it is present, the second segment is defined as before.  If
      it is absent, the second segment is empty.
\item The third segment is automatically generated.  Fields with initializers
      are initialized immediately after the \Xcd{property} statement.
      In the following example, \Xcd{b} is initialized to \Xcd{y*9000} in
      segment three.  The initialization makes sense and does the right
      thing; \Xcd{b} will be \Xcd{y*9000} for every \Xcd{Overdone} object.
      (This would not be possible if field initializers were processed
      earlier, before properties were set.)
\item The fourth segment is the remainder of the constructor body.
\end{enumerate}

The segments in the following code are shown in the comments.
%~~gen ^^^ ObjectInitialization20
% package ObjectInitialization.ShowingSegments;
%~~vis
\begin{xten}
class Overlord(x:Int) {
  def this(x:Int) { property(x); }
}//Overlord
class Overdone(y:Int) extends Overlord  {
  val a : Int;
  val b =  y * 9000;
  def this(r:Int) {
    super(r);                      // (1)
    x10.io.Console.OUT.println(r); // (2)
    val rp1 = r+1;
    property(rp1);                 // (2)
    // field initializations here  // (3)
    a = r + 2 + b;                 // (4)
  }
  def this() {
    this(10);                      // (1), (2), (3)
    val x = a + b;                 // (4)
  }
}//Overdone
\end{xten}
%~~siv
%
%~~neg

The rules of what is allowed in the three segments are different, though
unsurprising.  For example, properties of the current class can only be read
in segment 3 or 4---naturally, because they are set at the end of segment 2.

\subsubsection{Initialization and Inner Classses}
\index{constructor!inner classes in}

Constructors of inner classes are tantamount to method calls on \Xcd{this}.
For example, the constructor for Inner {\bf is} acceptable.  It does not leak
\Xcd{this}.  It leaks \Xcd{Outer.this}, which is an utterly different object.
So, the call to \Xcd{this.new Inner()} in the \Xcd{Outer} constructor {\em
is} illegal.  It would leak \Xcd{this}.  There is no special rule in effect
preventing this; a constructor call of an inner class is no
different from a method as far as leaking is concerned.
%~~gen ^^^ ObjectInitialization30
% package ObjInit.InnerClass; 
% // NOTEST-packaging-issue
%~~vis
\begin{xten}
class Outer {
  static val leak : Cell[Outer] = new Cell[Outer](null);
  class Inner {
     def this() {Outer.leak.set(Outer.this);}
  }
  def /*Outer*/this() {
     //ERROR: val inner = this.new Inner();
  }
}
\end{xten}
%~~siv
%
%~~neg



\subsubsection{Initialization and Closures}
\index{constructor!closure in}

Closures in constructors may not refer to \xcd`this`.  They may not even refer
to fields of \xcd`this` that have been initialized.   For example, the
closure \xcd`bad1` is not allowed because it refers to \xcd`this`; \xcd`bad2`
is not allowed because it mentions \xcd`a` --- which is, of course, identical
to \xcd`this.a`. 

%%-deleted-%% valid if they were invoked (or inlined) at the
%%-deleted-%%place of creation. For example, \Xcd{closure} below is acceptable because it
%%-deleted-%%only refers to fields defined at the point it was written.  \Xcd{badClosure}
%%-deleted-%%would not be acceptable, because it refers to fields that were not defined at
%%-deleted-%%that point, although they were defined later.
%~~gen ^^^ ObjectInitialization40
% package ObjectInitialization.Closures; 
%~~vis
\begin{xten}
class C {
  val a:Int;
  def this() {
    this.a = 1;
    //ERROR: val bad1 = () => this; 
    //ERROR: val bad2 = () => a*10;
  }
}
\end{xten}
%~~siv
%
%~~neg


\subsection{Definite Initialization in Constructors}


An instance field \Xcd{var x:T}, when \Xcd{T} has a default value, need not be
explicitly initialized.  In this case, \Xcd{x} will be initialized to the
default value of type \Xcd{T}.  For example, a \Xcd{Score} object will have
its \Xcd{currently} field initialized to zero, below:
%~~gen ^^^ ObjectInitialization50
% package ObjectInit.DefaultInit;
%~~vis
\begin{xten}
class Score {
  public var currently : Int;
}
\end{xten}
%~~siv
%
%~~neg

All other sorts of instance fields do need to be initialized before they can
be used.  \Xcd{val} fields must be initialized in the constructor, even if
their type has a 
default value.  It would be silly to have a field \Xcd{val z : Int} that was
always given default value of \Xcd{0} and, since it is \Xcd{val}, can never be
changed.  \Xcd{var} fields whose type has no default value must be initialized
as well, such as \xcd`var y : Int{y != 0}`, since it cannot be assigned a
sensible initial value.

The fundamental principles are:
\begin{enumerate}
\item \Xcd{val} fields must be assigned precisely once in each constructor on every
possible execution path.
\item \Xcd{var} fields of defaultless type must be
assigned at least once on every possible execution path, but may be assigned
more than once.
\item No variable may be read before it is guaranteed to have been
assigned.
\item Initialization may be by field initialization expressions (\Xcd{val x :
      Int = y+z}), or by uninitialized fields \Xcd{val x : Int;} plus
an initializing assignment \Xcd{x = y+z}.  Recall that field initialization
expressions are performed after the \Xcd{property} statement, in segment 3 in
the terminology of \Sref{SFineStructCtors}.
\end{enumerate}



\subsection{Summary of Restrictions on Classes and Constructors}

The following table tells whether a given feature is (yes), is not (no) or is
with some conditions (note) allowed in a given context.   For example, a
property method is allowed with the type of another property, as long as it
only mentions the preceding properties. The first column of the table gives
examples, by line of the following code body.

\begin{tabular}{||l|l|c|c|c|c|c|c||}
\hline
~
  & {\bf Example}
  & {\bf Prop.}
  & {\bf {\tt \small self==this}(1)}
  & {\bf Prop.Meth.}
  & {\bf {\tt this}}
  & {\bf {fields}}
\\\hline
Type of property
  & (A)
  & %?properties
    yes (2)
  & no %? self==this
  & no %? property methods
  & no %? this may be used
  & no %? fields may be used
\\\hline
Class Invariant
  & (B)
  & yes %?properties
  & yes %? self==this
  & yes %? property methods
  & yes %? this may be used
  & no %? fields may be used
\\\hline
Supertype (3)
  & (C), (D)
  & yes%?properties
  & yes%? self==this
  & yes%? property methods
  & no%? this may be used
  & no%? fields may be used
\\\hline
Property Method Body
  & (E)
  & yes %?properties
  & yes %? self==this
  & yes %? property methods
  & yes %? this may be used
  & no %? fields may be used
\\\hline

Static field (4)
  & (F) (G)
  & no %?properties
  & no %? self==this
  & no %? property methods
  & no %? this may be used
  & no %? fields may be used
\\\hline

Instance field (5)
  & (H), (I)
  & yes %?properties
  & yes %? self==this
  & yes %? property methods
  & yes %? this may be used
  & yes %? fields may be used
\\\hline

Constructor arg. type
  & (J)
  & no %?properties
  & no %? self==this
  & no  %? property methods
  & no %? this may be used
  & no %? fields may be used
\\\hline

Constructor guard
  & (K)
  & no %?properties
  & no %? self==this
  & no %? property methods
  & no %? this may be used
  & no %? fields may be used
\\\hline

Constructor ret. type
  & (L)
  & yes %?properties
  & yes %? self==this
  & yes %? property methods
  & yes %? this may be used
  & yes %? fields may be used
\\\hline

Constructor segment 1
  & (M)
  & no%?properties
  & yes%? self==this
  & no%? property methods
  & no%? this may be used
  & no%? fields may be used
\\\hline


Constructor segment 2
  & (N)
  & no%?properties
  & yes%? self==this
  & no%? property methods
  & no%? this may be used
  & no%? fields may be used
\\\hline

Constructor segment 4
  & (O)
  & yes%?properties
  & yes%? self==this
  & yes%? property methods
  & yes%? this may be used
  & yes%? fields may be used
\\\hline

Methods
  & (P)
  & yes %?properties
  & yes %? self==this
  & yes %? property methods
  & yes %? this may be used
  & yes %? fields may be used
\\\hline



\iffalse
place
  & (pos)
  & %?properties
  & %? self==this
  & %? property methods
  & %? this may be used
  & %? fields may be used
\\\hline
\fi
\end{tabular}

Details:

\begin{itemize}
\item (1) {Top-level {\tt self} only.}
\item (2) {The type of the {$i^{th}$} property may only mention
                 properties {$1$} through {$i$}.}
\item (3) Super-interfaces follow the same rules as supertypes.
\item (4) The same rules apply to types and initializers.
\end{itemize}



The example indices refer to the following code:
%~~gen ^^^ ObjectInitialization60
% package ObjectInit.GrandExample;
% class Supertype[T]{}
% interface SuperInterface[T]{}
%~~vis
\begin{xten}
class Example (
   prop : Int,
   proq : Int{prop != proq},                    // (A)
   pror : Int
   )
   {prop != 0}                                  // (B)
   extends Supertype[Int{self != prop}]         // (C)
   implements SuperInterface[Int{self != prop}] // (D)
{
   property def propmeth() = (prop == pror);    // (E)
   static staticField
      : Cell[Int{self != 0}]                    // (F)
      = new Cell[Int{self != 0}](1);            // (G)
   var instanceField
      : Int {self != prop}                      // (H)
      = (prop + 1) as Int{self != prop};        // (I)
   def this(
      a : Int{a != 0},
      b : Int{b != a}                           // (J)
      )
      {a != b}                                  // (K)
      : Example{self.prop == a && self.proq==b} // (L)
   {
      super();                                  // (M)
      property(a,b,a);                          // (N)
      // fields initialized here
      instanceField = b as Int{self != prop};   // (O)
   }

   def someMethod() =
        prop + staticField() + instanceField;   // (P)
}
\end{xten}
%~~siv
%
%~~neg

\section{Method Resolution}
\index{method!resolution}
\index{method!which one will get called}
\label{sect:MethodResolution}

Method resolution is the problem of determining, statically, which method (or
constructor or operator)
should be invoked, when there are several choices that could be invoked.  For
example, the following class has two overloaded \xcd`zap` methods, one taking
an \Xcd{Any}, and the other a \Xcd{Resolve}.  Method resolution will figure
out that the call \Xcd{zap(1..4)} should call \xcd`zap(Any)`, and
\Xcd{zap(new Resolve())} should call \xcd`zap(Resolve)`.  

\begin{ex}
%~~gen ^^^ MethodResolution10
%package MethodResolution.yousayyouwantaresolution;
% // This depends on https://jira.codehaus.org/browse/XTENLANG-2696
%~~vis
\begin{xten}
class Res {
  public static interface Surface {}
  public static interface Deface {}

  public static class Ace implements Surface {
    public static operator (Boolean) : Ace = new Ace();
    public static operator (Place) : Ace = new Ace();
  }
  public static class Face implements Surface, Deface{}

  public static class A {}
  public static class B extends A {}
  public static class C extends B {}

  def m(x:A) = 0;
  def m(x:Int) = 1;
  def m(x:Boolean) = 2;
  def m(x:Surface) = 3;
  def m(x:Deface) = 4; 

  def example() {
     assert m(100) == 1 : "Int"; 
     assert m(new C()) == 0 : "C";
     // An Ace is a Surface, unambiguous best choice
     assert m(new Ace()) == 3 : "Ace";
     // ERROR: m(new Face());

     // The match must be exact.
     // ERROR: assert m(here) == 3 : "Place";

     // Boolean could be handled directly, or by 
     // implicit coercion Boolean -> Ace.
     // Direct matches always win.
     assert m(true) == 2 : "Boolean"; 
  }
\end{xten}
%~~siv
%  public static def main(argv:Rail[String]) {(new Res()).example(); Console.OUT.println("That's all!");}
% public def claim() { val ace : Ace = here; assert m(ace)==3; }
% }
% class Hook{ def run(){ (new Res()).example(); return true;} }
%~~neg

In the \xcd`"Int"` line, there is a very close match.  \xcd`100` is an
\xcd`Int`.  In fact, \xcd`100` is an \xcd`Int{self==100}`, so even in this
case the type of the actual parameter is not {\em precisely} equal to the type
of the method.

In the \xcd`"C"` line of the example, \xcd`new C()` is an instance of \xcd`C`,
which is a subtype of \xcd`A`, so the \xcd`A` method applies.  No other method
does, and so the \xcd`A` method will be invoked.

Similarly, in the \xcd`"Ace"` line, the \xcd`Ace` class implements
\xcd`Surface`, and so \xcd`new Ace()` matches the \xcd`Surface` method. 

However, a \xcd`Face` is both a \xcd`Surface` and a \xcd`Deface`, so there is
no unique best match for the invocation \xcd`m(new Face())`.  This invocation
would be forbidden, and a compile-time error issued.


The match must be exact.  There is an implicit coercion 
from \xcd`Place` to \xcd`Ace`, and \xcd`Ace` implements \xcd`Surface`, so the
code
\begin{xten}
val ace : Ace = here;
assert m(ace) == 3;
\end{xten}
works, by using the \xcd`Surface` form of \xcd`m`.  But doing it in one step
requires a deeper search than X10 performs\footnote{In general this search is
unbounded, so X10 can't perform it.}, and is not allowed.


For \xcd`m(true)`, both the \xcd`Boolean` and, with the implicit coercion,
\xcd`Ace` methods could apply.  Since the \xcd`Boolean` method applies
directly, and the \xcd`Ace` method requires an implicit coercion, this call
resolves to the \xcd`Boolean` method, without an error.

\end{ex}


The basic concept of method resolution is:
\begin{enumerate}
\item List all the methods that could possibly be used, inferring generic
      types but not performing implicit coercions.    
\item If one possible method is more specific than all the others, that one 
      is the desired method.
\item If there are two or more methods neither of which is more specific than
      the others, then the method invocation is ambiguous.  Method resolution
      fails and reports an error.
\item Otherwise, no possible methods were found without implicit coercions.
      Try the preceding steps again, but with coercions allowed: zero or one
      implicit coercion for each argument.  If a single
      most specific method is found with coercions, it is the desired method.
      If there are several, the invocation is ambiguous and erronious.
\item If no methods were found even with coercions, then the method invocation
      is undetermined.  Method resolution fails and reports an error.
\end{enumerate}

After method resolution is done, there is a validation phase that checks the
legality of the call, based on the \xcd`STATIC_CHECKS` compiler flag.  
With \xcd`STATIC_CHECKS`, the method's constraints must be satisfied; that is,
they must be entailed (\Sref{SemanticsOfConstraints}) by the information in
force at the point of the call.  With \xcd`DYNAMIC_CHECKS`, if the constraint
is not entailed at that point, a dynamic check is inserted to make sure that
it is true at runtime.

\noindent
In the presence of X10's highly-detailed type system, some subtleties arise. 
One point, at least, is {\em not} subtle. The same procedure is used, {\em
mutatis mutandis} for method, constructor, and operator resolution.  



\subsection{Space of Methods}

X10 allows some constructs, particularly \xcd`operator`s, to be defined in a
number of ways, and invoked in a number of ways. This section specifies which
forms of definition could correspond to a given definiendum.
%%OP%% , and (redundantly)
%%OP%% the syntax for invoking that definition unambiguously.  

Method invocations \xcd`a.m(b)`, where \xcd`a` is an expression, can be either
of the following forms.  There may be any number of arguments.
\begin{itemize}
\item An instance method on \xcd`a`, of the form \xcd`def m(B)`.
%%OP%% , so that the   invocation is \xcd`a.m(b)`;
\item A static method on \xcd`a`'s class, of the form \xcd`static def m(B)`.
%%OP%%       so that the invocation is \xcd`A.m(b)`.
\end{itemize}

The meaning of an invocation of the form \xcd`m(b)`, with any number of
arguments, depends slightly on its context.  Inside of a constraint, it might
mean \xcd`self.m(b)`.  Outside of a constraint, there is no \xcd`self`
defined, so it can't mean that.  The first of these that applies will be
chosen. 
\begin{enumerate}
\item Invoke a method on \xcd`this`, \viz{} \xcd`this.m(b)`.  Inside a
      constraint, it may also invoke a property method on \xcd`self`, \viz.
      \xcd`self.m(b)`.  It is an error if both \xcd`this.m(b)` and
      \xcd`self.m(b)` are possible.
\item Invoke a function named \xcd`m` in a local or field.
\item Construct a structure named \xcd`m`.
\end{enumerate}

Static method invocations, \xcd`A.m(b)`, where \xcd`A` is a container name,
can only be static.  There may be any number of arguments.
\begin{itemize}
\item A static method on \xcd`A`, of the form \xcd`static def m(B)`.
%%OP%%       the invocation is \xcd`A.m(b)`; 
\end{itemize}


Constructor invocations, \xcd`new A(b)`, must invoke constructors. There may
be any number of arguments. 
\begin{itemize}
\item A constructor on \xcd`A`, of the form \xcd`def this(B)`.
%%OP%% , so that the
%%OP%%       invocation is \xcd`new A(b)`.
\end{itemize}


A unary operator \xcdmath"$\star$ a" may be defined as: 
\begin{itemize}
\item An instance operator on \xcd`A`, defined as 
      \xcdmath"operator $\star$ this()".
%%OP%%       so that the invocation is 
%%OP%%       \xcdmath"a.operator $\star$()"; or
\item A static operator on \xcd`A`, defined as 
      \xcdmath"operator $\star$(a:A)".
%%OP%%       so that the invocation is 
%%OP%%       \xcdmath"A.operator $\star$(a)"
\end{itemize}

A binary operator \xcdmath"a $\star$ b" may be defined as: 
\begin{itemize}
\item An instance operator on \xcd`A`, defined as 
      \xcdmath"operator this $\star$(b:B)";
%%OP%%       so that the invocation is \xcdmath"a.operator $\star$(b)", 
or
\item A right-hand operator on \xcd`B`, defined as
      \xcdmath"operator (a:A) $\star$ this"; or
%%OP%%       so that the invocation is \xcdmath"b.operator ()$\star$(b)"

\item A static operator on \xcd`A`, defined as
      \xcdmath"operator (a:A) $\star$ (b:B)", 
%%OP%%       so that the invocation is \xcdmath"A.operator $\star$(a,b)"
; or
\item A static operator on \xcd`B`, if \xcd`A` and \xcd`B` are different
      classes, defined as
      \xcdmath"operator (a:A) $\star$ (b:B)"
%%OP%% , so that the invocation is 
%%OP%%       \xcdmath"B.operator $\star$(a,b)".
\end{itemize}
\noindent
If none of those resolve to a method, then either operand may be implicitly
coerced to the
other.  If one of the following two situations obtains, it will be done; if
both, the expression causes a static error.
\begin{itemize}
\item An implicit coercion from \xcd`A` to \xcd`B`, and 
      an operator \xcdmath"B $\star$ B" can be used, by 
      coercing \xcd`a` to be of type \xcd`B`, and then using \xcd`B`'s
      $\star$.  
\item An implicit coercion from \xcd`B` to \xcd`A`, and 
      an operator \xcdmath"A $\star$ A" can be used,
      coercing \xcd`b` to be of type \xcd`A`, and then using \xcd`A`'s
      $\star$.  
\end{itemize}

An application \xcd`a(b)`, for any number of arguments, can come from a number
of things. 
\begin{itemize}
\item an application operator on \xcd`a`, defined as \xcd`operator this(b:B)`;
%%OP%% , so that the 
%%OP%% invocation is \xcd`a.operator()(b)`
\item If \xcd`a` is an identifier, \xcd`a(b)` can also be a method invocation
      equivalent to \xcd`this.a(b)`, which  invokes \xcd`a` as
      either an instance or static method on \xcd`this`
\item If \xcd`a` is a qualified identifier, \xcd`a(b)` can also be an
      invocation of a struct constructor.
\end{itemize}


An indexed assignment, \xcd`a(b)=c`, for any number of \xcd`b`'s, can only
come from an indexed assignment definition: 
\begin{itemize}
\item \xcd`operator this(b:B)=(c:C) {...}`
%%OP%%       so that the invocation is \xcd`a.operator()=(b,c)`.
\end{itemize}

An implicit coercion, in 
which a value \xcd`a:A` is used in a context which requires a value of some
other non-subtype \xcd`B`, 
can only come from implicit coercion operation defined on
\xcd`B`: 
\begin{itemize}
\item an implicit coercion in \xcd`B`:
      \xcd`static operator (a:A):B`;
%%OP%%       so that the coercion is \xcd`B.operator[B](a)`;
\end{itemize}

An explicit conversion \xcd`a as B` can come from an explicit conversion
operator, or an implicit coercion operator.  X10 tries two things, in order,
only checking 2 if 1 fails: 
\begin{enumerate}
\item An \xcd`as` operator in \xcd`B`: 
      \xcdmath"static operator (a:A) as ?";
%%OP%%       so that the conversion is \xcd`B.operator as[B](a)`

\item or, failing that, an implicit coercion in \xcd`B`:
      \xcd`static operator (a:A):B`.
%%OP%% , so that the conversion is 
%%OP%%       \xcd`B.operator[B](a)`;

\end{enumerate}



\subsection{Possible Methods}

This section describes what it means for a method to be a {\em possible}
resolution of a method invocation.  



Generics introduce several subtleties, especially with the inference of
generic types. 
For the purposes of method resolution, all that matters about a method,
constructor, or operator \xcd`M` --- we use the word ``method'' to include all
three choices for this section --- is its signature, plus which method it is.
So, a typical \xcd`M` might look like 
\xcdmath"def m[G$_1$,$\ldots$, G$_g$](x$_1$:T$_1$,$\ldots$, x$_f$:T$_f$){c} =...".  The code body \xcd`...` is irrelevant for the purpose of whether a
given method call means \xcd`M` or not, so we ignore it for this section.

All that matters about a method definition, for the purposes of method
resolution, is: 
\begin{enumerate}
\item The method name \xcd`m`;
\item The generic type parameters of the method \xcd`m`,  \xcdmath"G$_1$,$\ldots$, G$_g$".  If there
      are no generic type parameters, {$g=0$}.  
\item The types \xcdmath"x$_1$:T$_1$,$\ldots$, x$_f$:T$_f$" of the formal parameters.  If
      there are no formal parameters, {$f=0$}. In the case of an instance
      method, the receiver will be the first formal parameter.\footnote{The
      variable names are relevant because one formal can be mentioned in a
      later type, or even a constraint: {\tt def f(a:Int, b:Point\{rank==a\})=...}.}
\item A {\em unique identifier} \xcd`id`, sufficient to tell the compiler
      which method body is intended.  A file name and position in that file
      would suffice.  The details of the identifier are not relevant.
\end{enumerate}

For the purposes of understanding method resolution, we assume that all the
actual parameters of an invocation are simply variables: \xcd`x1.meth(x2,x3)`.
This is done routinely by the compiler in any case; the code 
\xcd`tbl(i).meth(true, a+1)` would be treated roughly as 
\begin{xten}
val x1 = tbl(i);
val x2 = true;
val x3 = a+1;
x1.meth(x2,x3);
\end{xten}

All that matters about an invocation \xcd`I` is: 
\begin{enumerate}
\item The method name \xcdmath"m$'$";
\item The generic type parameters \xcdmath"G$'_1$,$\ldots$, G$'_g$".  If there
      are no generic type parameters, {$g=0$}.  
\item The names and types \xcdmath"x$_1$:T$'_1$,$\ldots$, x$_f$:T$'_f$" of the
      actual parameters.
      If
      there are no actual parameters, {$f=0$}. In the case of an instance
      method, the receiver is the first actual parameter.
\end{enumerate}

The signature of the method resolution procedure is: 
\xcd`resolve(invo : Invocation, context: Set[Method]) : MethodID`.  
Given a particular invocation and the set \xcd`context` of all methods
which could be called at that point of code, method resolution either returns
the unique identifier of the method that should be called, or (conceptually)
throws an exception if the call cannot be resolved.

The procedure for computing \xcd`resolve(invo, context)` is: 
\begin{enumerate}
\item Eliminate from \xcd`context` those methods which are not {\em
      acceptable}; \viz, those whose name, type parameters, and formal parameters
      do not suitably match \xcd`invo`.  In more detail:
      \begin{itemize}
      \item The method name \xcd`m` must simply equal the invocation name \xcdmath"m$'$";
      \item X10 infers type parameters, by an algorithm given in \Sref{TypeParamInfer}.
      \item The method's type parameters are bound to the invocation's for the
            remainder of the acceptability test.
      \item The actual parameter types must be subtypes of the formal
            parameter types, or be coercible to such subtypes.  Parameter $i$
            is a subtype if \xcdmath"T$'_i$ <: T$_i$".  It is implicitly
            coercible to a subtype if either it is a subtype, or if there is
            an implicit coercion operator 
            defined from \xcdmath"T$'_i$" to some type \xcd`U`, and 
            \xcdmath"U <: T$_i$". \index{method resolution!implicit coercions
            and} \index{implicit coercion}\index{coercion}.  If coercions are
            used to resolve the method, they will be called on the arguments
            before the method is invoked.
            
      \end{itemize}
\item Eliminate from \xcd`context` those methods which are not {\em
      available}; \viz, those which cannot be called due to visibility
      constraints, such as methods from other classes marked \xcd`private`.
      The remaining methods are both acceptable and available; they might be
      the one that is intended.
\item If the method invocation is a \xcd`super` invocation appearing in class
      \xcd`Cl`, methods of \xcd`Cl` and its subclasses are considered
      unavailable as well.
      
\item From the remaining methods, find the unique \xcd`ms` which is more specific than all the
      others, \viz, for which \xcd`specific(ms,mo) = true` for all other
      methods \xcd`mo`.
      The specificity test \xcd`specific` is given next.
      \begin{itemize}
      \item If there is a unique such \xcd`ms`, then
            \xcd`resolve(invo,context)` returns the \xcd`id` of \xcd`ms`.  
      \item If there is not a unique such \xcd`ms`, then \xcd`resolve` reports
            an error.
      \end{itemize}

\end{enumerate}

The subsidiary procedure \xcd`specific(m1, m2)` determines whether method
\xcd`m1` is equally or more specific than \xcd`m2`.  \xcd`specific` is not a
total order: is is possible for each one to be considered more specific than
the other, or either to be more specific.  \xcd`specific` is computed as: 
\begin{enumerate}
\item Construct an invocation \xcd`invo1` based on \xcd`m1`: 
      \begin{itemize}
      \item \xcd`invo1`'s method name is \xcd`m1`'s method name;
      \item \xcd`invo1`'s generic parameters are those of \xcd`m1`--- simply
            some type variables.
      \item \xcd`invo1`'s parameters are those of \xcd`m1`.
      \end{itemize}
\item If \xcd`m2` is acceptable for the invocation \xcd`invo1`,
      \xcd`specific(m1,m2)` returns true; 
\item Construct an invocation \xcd`invo2p`, which is \xcd`invo1` with the
      generic parameters erased.  Let \xcd`invo2` be \xcd`invo2p` with generic
      parameters as inferred by X10's type inference algorithm.  If type
      inference fails, \xcd`specific(m1,m2)` returns false.
\item If \xcd`m2` is acceptable for the invocation \xcd`invo2`,
      \xcd`specific(m1,m2)` returns true; 
\item Otherwise, \xcd`specific(m1,m2)` returns false.
\end{enumerate}

\subsection{Field Resolution}

An identifier \xcd`p` can refer to a number of things.  The rules are somewhat
different inside and outside of a constraint.

Outside of a constraint, the compiler chooses
the first one from the following list which applies: 
\begin{enumerate}
\item A local variable named \xcd`p`.
\item A field of \xcd`this`, \viz{} \xcd`this.p`.
\item A nullary property method, \xcd`this.p()`
\item A member type named \xcd`p`.
\item A package named \xcd`p`.
\end{enumerate}

Inside of a constraint, the rules are slightly different, because \xcd`self`
is available, and packages cannot be used per se.
\begin{enumerate}
\item A local variable named \xcd`p`.
\item A property of \xcd`this` or of \xcd`self`, \viz{} \xcd`this.p` or
      \xcd`self.p`.  If both are available, report an error.
\item A nullary property method, \xcd`this.p()`
\item A member type named \xcd`p`.
\end{enumerate}

\subsection{Other Disambiguations}
\label{sect:disambiguations}

It is possible to have a field of the same name as a method.
Indeed, it is a common pattern to have private field and a public
method of the same name to access it:
\begin{ex}
%~~gen ^^^ MethodResolution_disamb_a
%package MethodResolution_disamb_a;
%~~vis
\begin{xten}
class Xhaver {
  private var x: Int = 0;
  public def x() = x;
  public def bumpX() { x ++; }
}
\end{xten}
%~~siv
%
%~~neg
\end{ex}

\begin{ex}
However, this can lead to syntactic ambiguity in the case where the field
\Xcd{f} of object \xcd`a` is a
function, array, list, or the like, and where \xcd`a` has a method also named
\xcd`f`.  The term \Xcd{a.f(b)} could either mean ``call method \xcd`f` of \xcd`a` upon
\xcd`b`'', or ``apply the function \xcd`a.f` to argument \xcd`b`''.  

%~~gen  ^^^ MethodResolution_disamb_b
%package MethodResolution_disamb_b;
%NOCOMPILE
%~~vis
\begin{xten}
class Ambig {
  public val f : (Int)=>Int =  (x:Int) => x*x;
  public def f(y:int) = y+1;
  public def example() {
      val v = this.f(10);
      // is v 100, or 11?
  }
}
\end{xten}
%~~siv
%
%~~neg
\end{ex}

In the case where a syntactic form \xcdmath"E.m(F$_1$, $\ldots$, F$_n$)" could
be resolved as either a method call, or the application of a field \xcd`E.m`
to some arguments, it will be treated as a method call.  
The application of \xcd`E.m` to some arguments can be specified by adding
parentheses:  \xcdmath"(E.m)(F$_1$, $\ldots$, F$_n$)".

\begin{ex}

%~~gen ^^^ MethodResolution_disamb_c
%package MethodResolution_disamb_c;
%NOCOMPILE
%~~vis
\begin{xten}
class Disambig {
  public val f : (Int)=>Int =  (x:Int) => x*x;
  public def f(y:int) = y+1;
  public def example() {
      assert(  this.f(10)  == 11  );
      assert( (this.f)(10) == 100 );
  }
}
\end{xten}
%~~siv
%
%~~neg

\end{ex}

Similarly, it is possible to have a method with the same name as a struct, say
\xcd`ambig`, giving an ambiguity as to whether \xcd`ambig()` is a struct
constructor invocation or a method invocation.  This ambiguity is resolved by
treating it as a method invocation.  If the constructor invocation is desired,
it can be achieved by including the optional \xcd`new`.  That is, 
\xcd`new ambig()` is struct constructor invocation; \xcd`ambig()` is a 
method invocation.

\section{Static Nested Classes}
\label{StaticNestedClasses}
\index{class!static nested}
\index{class!nested}
\index{static nested class}

One class (or struct or interface) may be nested within another.  The simplest
way to do this is as a \xcd`static` nested class, written by putting one class
definition at top level inside another, with the inner one having a
\xcd`static` modifier.  
For most purposes, a static nested class behaves like a top-level class.
However, a static nested class has access to private static
fields and methods of its containing class.  

Nested interfaces and static structs are permitted as well.

%~~gen ^^^ InnerClasses10
% package Classes.StaticNested; 
% NOTEST
%~~vis
\begin{xten}
class Outer {
  private static val priv = 1;
  private static def special(n:Int) = n*n;
  public static class StaticNested {
     static def reveal(n:Int) = special(n) + priv;
  }
}
\end{xten}
%~~siv
%
%~~neg

\section{Inner Classes}
\label{InnerClasses}
\index{class!inner}
\index{inner class}


Non-static nested classes are called {\em inner classes}. An inner class
instance can be thought of as a very elaborate member of an object --- one
with a full class structure of its own.   The crucial characteristic of an
inner class instance is that it has an implicit reference to an instance of
its containing class.  

\begin{ex}
This feature is particularly useful when an instance of the inner class makes
no sense without reference to an instance of the outer, and is closely tied to
it.  For example, consider a range class, describing a span of integers {$m$}
to {$n$}, and an iterator over the range.  The iterator might as well have
access to the range object, and there is little point to discussing
iterators-over-ranges without discussing ranges as well.
In the following example, the inner class \xcd`RangeIter` iterates over the
enclosing \xcd`Range`.  

It has its own private cursor field \xcd`n`, telling
where it is in the iteration; different iterations over the same \xcd`Range`
can exist, and will each have their own cursor.
It is perhaps unwise to use the name \xcd`n` for a field of the inner class,
since it is also a field of the outer class, but it is legal.  (It can happen
by accident as well -- \eg, if a programmer were to add a field \xcd`n` to a
superclass of the  outer class, the inner class would still work.)
It does not even
interfere with the inner class's ability to refer to the outer class's \xcd`n`
field: the cursor initialization 
refers to the \xcd`Range`'s lower bound through a fully qualified name
\xcd`Range.this.n`.
The initialization of its \xcd`n` field refers to the outer class's \xcd`m` field, which is
not shadowed and can be referred to directly, as \xcd`m`.


%~~gen ^^^ InnerClasses20
% package Classes.InnerClasses_a; 
% NOTEST
%~~vis
\begin{xten}
class Range(m:Int, n:Int) implements Iterable[Int]{
  public def iterator ()  = new RangeIter();
  private class RangeIter implements Iterator[Int] {
     private var n : Int = m;
     public def hasNext() = n <= Range.this.n;
     public def next() = n++;
  }
  public static def main(argv:Rail[String]) {
    val r = new Range(3,5);
    for(i in r) Console.OUT.println("i=" + i);
  }
}
\end{xten}
%~~siv
%
%~~neg
\end{ex}

An inner class has full access to the members of its enclosing class, both
static and instance.  In particular, it can access \xcd`private` information,
just as methods of the enclosing class can.  

An inner class can have its own members.  
Inside instance methods of an inner class, \xcd`this` refers to the instance
of the {\em inner} class.  The instance of the outer class can be accessed as
{\em Outer}\xcd`.this` (where {\em Outer} is the name of the outer class).
If, for some dire reason, it is necessary to have an inner class within an inner
class, the innermost class can refer to the \xcd`this` of either outer class
by using its name.

An inner class can inherit from any class in scope,
with no special restrictions. \xcd`super` inside an inner class refers to the
inner class's superclass. If it is necessary to refer to the outer classes's
superclass, use a qualified name of the form {\em Outer}\xcd`.super`.

The members of inner classes must be instance members.  They cannot be static
members.  Classes, interfaces, static methods, static fields, and typedefs are
not allowed as members of inner classes. 
The same restriction applies to local classes (\Sref{sect:LocalClasses}).

\index{inner class!extending}
Consider
an inner class \xcd`IC1` of some outer class \xcd`OC1`, being extended by 
another class \xcd`IC2`. However, since an \xcd`IC1` only exists as a
dependent of an \xcd`OC1`, each \xcd`IC2` must be associated with an \xcd`OC1`
--- or a subtype thereof --- as well.   So, \xcd`IC2` must be an inner class
of either \xcd`OC1` or some subclass \xcd`OC2 <: OC1`.

\begin{ex}For example, one often extends an
inner class when one extends its outer class: 
%~~gen ^^^ InnerClasses30
% package Classes.Innerclasses.Are.For.Innermasses;
%~~vis
\begin{xten}
class OC1 {
   class IC1 {}
}
class OC2 extends OC1 {
   class IC2 extends IC1 {} 
}
\end{xten}
%~~siv
%
%~~neg
\end{ex}


The hiding of method names has one fine point.  If an inner class defines a
method named \xcd`doit`, then {\em all} methods named \xcd`doit` from the
outer class are hidden --- even if they have different argument types than the
one defined in the inner class.
They are still accessible via
\xcd`Outer.this.doit()`, but not simply via \xcd`doit()`.  The following code
is correct, but would not be correct if the ERROR line were uncommented.

%~~gen ^^^ InnerClasses40
% package Classes.Innerclasses.StupidOverloading; 
% NOTEST
%~~vis
\begin{xten}
class Outer {
  def doit() {}
  def doit(String) {}
  class Inner { 
     def doit(Boolean, Outer) {}
     def example() {
        doit(true, Outer.this);
        Outer.this.doit();
        //ERROR: doit("fails");
     }
  }
}
\end{xten}
%~~siv
%
%~~neg


\subsection{Constructors and Inner Classes}
\label{sect:InnerClassCtor}
\index{inner class!constructor}

If \xcd`IC` is an inner class of \xcd`OC`, then instance code in the body of
\xcd`OC` can create instances of \xcd`IC` simply by calling a constructor
\xcd`new IC(...)`: 
%~~gen ^^^ InnerClasses50
% package Classes.Innerclasses.Constructors.Easy;
%~~vis
\begin{xten}
class OC {
  class IC {}
  def method(){
    val ic = new IC();
  }
}
\end{xten}
%~~siv
%
%~~neg

Instances of \xcd`IC` can be constructed from elsewhere as well.  Since every
instance of \xcd`IC` is associated with an instance of \xcd`OC`, an \xcd`OC`
must be supplied to the \xcd`IC` constructor.  The syntax for doing so is: 
\xcd`oc.new IC()`.  For example: 
%~~gen ^^^ InnerClasses60
% package Classes.Inner_a; 
% NOTEST
% /*NONSTATIC*/
%~~vis
\begin{xten}
class OC {
  class IC {}
  static val oc1 = new OC();
  static val oc2 = new OC();
  static val ic1 = oc1.new IC();
  static val ic2 = oc2.new IC();
}
class Elsewhere{
  def method(oc : OC) {
    val ic = oc.new IC();
  }
}
\end{xten}
%~~siv
%
%~~neg


\section{Local Classes}
\label{sect:LocalClasses}

Classes can be defined and instantiated in the middle of methods and other
code blocks.
A local class in a static method is a static class; a local class in an
instance method is an inner class.
 Local classes are local to the block in which they are defined.
They have access to almost everything defined at that point in the method; the
one exception is that they cannot use \xcd`var` variables. Local classes
cannot be \xcd`public`, \xcd`protected`, or \xcd`private`, because they are
only visible from within the block of declaration. They cannot be
\xcd`static`.

\begin{ex}
The following example illustrates the use of a local class \xcd`Local`, 
defined inside the body of method \xcd`m()`. 
%~~gen ^^^ InnerClasses5p9v
% package InnerClasses5p9v;
% NOTEST
%~~vis
\begin{xten}
class Outer {
  val a = 1;
  def m() {
    val a = -2; 
    val b = 2;
    class Local {
      val a = 3;
      def m() = 100*Outer.this.a + 10*b + a; 
    }
    val l : Local = new Local();
    assert l.m() == 123;
  }//end of m()
}
\end{xten}
%~~siv
% class Hook{ def run() {
%   val o <: Outer = new Outer();
%   o.m();
%   return true;
% } }
%~~neg
Note that the middle \xcd`a`,
whose value is \xcd`-2`, is not accessible inside of \xcd`Local`; it is
shadowed by \xcd`Local`'s \xcd`a` field.  \xcd`Outer`'s \xcd`a` is also
shadowed, but the notation \xcd`Outer.this` gives a reference to the enclosing
\xcd`Outer` object.  There is no corresponding notation to access shadowed local
variables from the enclosing block; if you need to get them, rename the fields
of \xcd`Local`.    
\end{ex}


The members of inner classes must be instance members.  They cannot be static
members.  Classes, interfaces, static methods, static fields, and typedefs are
not allowed as members of local classes. 
The same restriction applies to inner classes (\Sref{InnerClasses}). 





\section{Anonymous Classes}
\index{class!anonymous}
\index{anonymous class}

It is possible to define a new local class and instantiate it as part of an
expression.  The new class can extend an existing class or interface.  Its body
can include all of the usual members of a local class. It can refer to any
identifiers available at that point in the expression --- except for \xcd`var`
variables.  An anonymous class in a static context is a static inner class.

Anonymous classes are useful when you want to package several pieces of
behavior together (a single piece of behavior can often be expressed as a
function, which is syntactically lighter-weight), or if you want to extend and
vary an extant class without going through the trouble of actually defining a
whole new class.

The syntax for an anonymous class is a constructor call followed immediately
by a braced class body: \xcd`new C(1){def foo()=2;}`.

\begin{ex}In the following minimalist example, the abstract class \xcd`Choice`
encapsulates a decision.   A \xcd`Choice` has a \xcd`yes()` and a \xcd`no()`
method.  The \xcd`choose(b)` method will invoke one of the two.  \xcd`Choice`s
also have names.

The \xcd`main()` method creates a specific \xcd`Choice`.  \xcd`c` is not a
immediate instance of \xcd`Choice` --- as an abstract class, \xcd`Choice` has
no immediate instances. \xcd`c` is an instance of an anonymous class which
inherits from \xcd`Choice`, but supplies \xcd`yes()` and \xcd`no()` methods.
These methods modify the contents of the \xcd`Cell[Int]` \xcd`n`.  (Note that,
as \xcd`n` is a local variable, it would take a few lines more coding to
extract \xcd`c`'s class, name it, and make it an inner class.)  The call to
\xcd`c.choose(true)`  will call \xcd`c.yes()`, incrementing \xcd`n()`, in a
rather roundabout manner.

%~~gen ^^^ InnerClasses70
% package ClassInnnerclassAnonclassOw; 
%~~vis
\begin{xten}
abstract class Choice(name: String) {
  def this(name:String) {property(name);}
  def choose(b:Boolean) { 
     if (b) this.yes(); else this.no(); }
  abstract def yes():void;
  abstract def no():void;
}

class Example {
  static def main(Rail[String]) {
    val n = new Cell[Int](0);
    val c = new Choice("Inc Or Dec") {
      def yes() { n() += 1; }
      def no()  { n() -= 1; }
      };
    c.choose(true);
    Console.OUT.println("n=" + n());
  }
}

\end{xten}
%~~siv
%
%~~neg
\end{ex}

Anonymous classes have many of the features of classes in general.  A few
features are unavailable because they don't make sense.

\begin{itemize}

\item Anonymous classes don't have constructors.  Since they don't have names,
      there's no way a constructor could get called in the ordinary way.
      Instead, the \xcd`new C(...)` expression must match a constructor of the
      parent class \xcd`C`, which will be called to initialize the
      newly-created object of the anonymous class.

\item The \xcd`public`,
      \xcd`private`, and \xcd`protected`  modifiers don't make sense for
      anonymous classes:  
      Anonymous classes, being anonymous,
      cannot be referenced at all, so references to them can't be public,
      private, or protected.

\item Anonymous classes cannot be \xcd`abstract`.  Since they only exist in
      combination with a constructor call, they must be constructable.  The
      parent class of the anonymous class may be abstract, or may be an
      interface; in this case, the anonymous class must provide all the
      methods that the parent demands.

\item Anonymous classes cannot have explicit \xcd`extends` or \xcd`implements`
      clauses; there's no place in the syntax for them. They have a single
      parent and that is that. 
\end{itemize}

\chapter{Structs}
\label{XtenStructs}
\label{StructClasses}
\label{Structs}
\index{struct}

X10 objects are a powerful general-purpose programming tool. However, the
power must be paid for in space and time. In space, a typical object
implementation requires some extra memory for run-time class information, as
well as a pointer for each reference to the object.  In time, a typical object
requires an extra indirection to read or write data, and some computation to
figure out which method body to call.  

For high-performance computing, this overhead may not be acceptable for all
objects. X10 provides structs, which are stripped-down objects. They are less
powerful than objects; in particular they lack inheritance and mutable fields.
Without inheritance, method calls do not need to do any lookup; they can be
implemented directly. Accordingly, structs can be implemented and used more
cheaply than objects, potentially avoiding the space and time overhead.
(Currently, the C++ back end avoids the overhead, but the Java back end
implements structs as Java objects and does not avoid it.)

Structs and classes are interoperable. Both can implement interfaces (in
particular, like all X10 values they implement \xcd`Any`), and subprocedures
whose arguments are defined by interfaces can take both structs and classes.
(Some caution is necessary here: referring to a struct through an interface
requires overhead similar to that required for an object.)

They are also interconvertable, within the constraints of structs. If you
start off defining a struct and decide you need a class instead, the code
change required is simply changing the keyword \xcd`struct` to \xcd`class`. If
you have a class that does not use inheritance or mutable fields, it can be
converted to a struct by changing its keyword. Client code using the struct
that was a class will need certain changes: the \xcd`new` keyword must be
added in constructor calls, and structs (unlike classes) do not have default values.



\section{Struct declaration}
\index{struct!declaration}
A struct declaration has the structure: 
\begin{xtenmath}
$\mbox{\emph{StructModifiers}}^{\mbox{?}}$
struct C[X$_1$, $\ldots$, X$_n$](p$_1$:T$_1$, $\ldots$, p$_n$:T$_n$){c} 
   implements I$_1$, $\ldots$, I$_k$ {
$\mbox{\emph{StructBody}}$
}
\end{xtenmath}

All fields of a struct must be \xcd`val`.

A struct \Xcd{S} cannot contain a field of type \Xcd{S}, or a field of struct
type \Xcd{T} which, recursively, contains a field of type \Xcd{S}.  This
restriction is necessary to permit \xcd`S` to be implemented as a contiguous
block of memory of size equal to the sum of the sizes of its fields.  


Values of a struct \Xcd{C} type can be created by invoking a constructor
defined in \Xcd{C}, but without prefixing it with \Xcd{new}: 
%~~gen
% package Structs.For.Muckts;
%~~vis
\begin{xten}
struct Polar(r:Double, theta:Double){
  def this(r:Double, theta:Double) {property(r,theta);}
  static val Origin = Polar(0,0);
  static val x0y1 = Polar(1, 3.14159/2);
}
\end{xten}
%~~siv
%
%~~neg

Structs support the same notions of generics, properties, and constrained
types that classes do.  For example, the \xcd`Pair` type below provides pairs
of values; the \xcd`diag()` method applies only when the two elements of the
pair are equal, and returns that common value: 
%~~gen
% package Structs.For.Muckts;
%~~vis
\begin{xten}
struct Pair[T,U](t:T, u:U) {
  def this(t:T, u:U) { property(t,u); }
  def diag(){T==U && t==u} = t;
}
\end{xten}
%~~siv
%
%~~neg


\section{Boxing of structs}
\index{auto-boxing!struct to interface}
\index{struct!auto-boxing}
\index{struct!casting to interface}
\label{auto-boxing} 
If a struct \Xcd{S} implements an interface \Xcd{I} (\eg, \Xcd{Any}),
a value \Xcd{v} of type \Xcd{S} can be assigned to a variable of type
\Xcd{I}. The implementation creates an object \Xcd{o} that is an
instance of an anonymous class implementing \Xcd{I} and containing
\Xcd{v}.  The result of invoking a method of \Xcd{I} on \Xcd{o} is the
same as invoking it on \Xcd{v}. This operation is termed {\em auto-boxing}.
It allows full interoperability of structs and objects---at the cost of losing
the extra efficiency of the structs when they are boxed.

In a generic class or struct obtained by instantiating a type parameter
\Xcd{T} with a struct \Xcd{S}, variables declared at type \Xcd{T} in the body
of the class are not boxed. They are implemented as if they were declared at
type \Xcd{S}.

\section{Optional Implementation of \Xcd{Any} methods}
\label{StructAnyMethods}
\index{Any!structs}

Two
structs are equal (\Xcd{==}) if and only if their corresponding fields
are equal (\Xcd{==}). 

All structs implement \Xcd{x10.lang.Any}. 
Structs are required to implement the following methods from \xcd`Any`.  
Programmers need not provide them; X10 will produce them automatically if 
the program does not include them. 
\begin{xten}
  public def equals(Any):Boolean;
  public def hashCode():Int;
  public def typeName():String;
  public def toString():String;  
\end{xten}


A programmer who provides an explicit implementation
of \Xcd{equals(Any)} for a struct \Xcd{S} should also consider
supplying a definition for \Xcd{equals(S):Boolean}. This will often
yield better performance since the cost of an upcast to \Xcd{Any} and
then a downcast to \Xcd{S} can be avoided.

\section{Primitive Types}
\index{types!primitive}
\index{primitive types}
\index{Int}
\index{UInt}
\index{Long}
\index{ULong}
\index{Char}
\index{Byte}
\index{UByte}
\index{Boolean}
\index{Short}
\index{UShort}
\index{Float}
\index{Double}

Certain types that might be built in to other languages are in fact
implemented as structs in package \xcd`x10.lang` in X10. Their methods and
operations are often provided with \xcd`@Native` (\Sref{NativeCode}) rather
than X10 code, however. These types are:
\begin{xten}
Boolean, Char, Byte, Short, Int, Long
Float, Double, UByte, UShort, UInt, ULong
\end{xten}
 
\section{Generic programming with structs}
\section{struct!generic}
\section{generics!struct}

An unconstrained type variable \Xcd{X} can be instantiated with \Xcd{Object} or
its subclasses or structs or functions.

Within a generic struct, all the operations of \Xcd{Any} are available
on a variable of type \Xcd{X}. Additionally, variables of type \Xcd{X} may
be used with \Xcd{==, !=}, in \Xcd{instanceof}, and casts.

\bard{The rest of this section is under discussion.  The example is wrong; it
ignores the fact that values can be functions.}
The programmer must be aware of the different interpretations of
equality for structs and classes and ensure that the code is correctly
written for both cases. If necessary the programmer can write code
that distinguishes between the two cases (a type parameter \Xcd{X} is
instantiated to a struct or not) as follows:

\begin{xten}
val x:X = ...;
if (x instanceof Object) { // x is a real object
   val x2 = x as Object; // this cast will always succeed.
   ...
} else { // x is a struct
   ...
}
\end{xten}
 
  
\section{Example structs}

\xcd`x10.lang.Complex` provides a detailed example of a practical struct,
suitable for use in a library.  For a shorter example, we define the
\xcd`Pair` struct---available in \xcd`x10.util.Pair`.  A \xcd`Pair` packages
two values of possibly unrelated type together in a single value, \eg, to
return two values from a function.

%~~gen
% package Structs.Pairs.Are.For.Squares;
%~~vis
\begin{xten}
struct Pair[T,U] {
    public val first:T;
    public val second:U;
    public def this(first:T, second:U):Pair[T,U] {
        this.first = first;
        this.second = second;
    }
    public def toString():String {
        return "(" + first + ", " + second + ")";
    }
}
\end{xten}
%~~siv
%
%~~neg

\section{Nested Structs}
\index{struct!static nested}
\index{static nested struct}

Static nested structs may be defined, essentially as static nested classes
except for making them structs
(\Sref{StaticNestedClasses}).  Inner structs may be defined, essentially as
inner classes except making them structs (\Sref{InnerClasses}).



\chapter{Functions}
\label{Functions}
\label{functions}
\index{functions}
\label{Closures}

\section{Overview}
Functions, the last of the three kinds of values in X10, encapsulate pieces of
code which can be applied to a vector of arguments to produce a value.
Functions, when applied, can do nearly anything that any other code could do:
fail to terminate, throw an exception, offer values, modify variables, spawn activities,
execute in several places, and so on. X10 functions are not mathematical
functions: the \xcd`f(1)` may return \xcd`true` on one call and \xcd`false` on
an immediately following call.

It is a limitation of \XtenCurrVer{} that functions do not support
type arguments. This limitation may be removed in future versions of
the language.

A \emph{function literal} \xcd"(x1:T1,..,xn:Tn){c}:T=>e" creates a function of
type\\ \xcd"(x1:T1,...,xn:Tn){c}=>T" (\Sref{FunctionType}).  For example, 
\xcd`(x:Int) => x*x` is a function literal describing the squaring function on
integers.   
\xcd`null` is also a function value.

Function application is written \xcd`f(a,b,c)`, following common mathematical
usage. 
\index{Exception!unchecked}
Function invocation may throw unchecked exceptions. 

The function body may be a block.  To compute integer squares by repeated
addition (inefficiently), one may write: 
%~~gen
% package Functions.Are.For.Spunctions;
% class Examplllll {
% static 
%~~vis
\begin{xten}
val sq: (Int) => Int 
      = (n:Int) => {
           var s : Int = 0;
           val abs_n = n < 0 ? -n : n;
           for ((i) in 1..abs_n) s += abs_n;
           s
        };
\end{xten}
%~~siv
%}
%~~neg




A function literal evaluates to a function entity {$\phi$}. When {$\phi$} is
applied to a suitable list of actual parameters \xcd`a1`-\xcd`an`, it
evaluates \xcd`e` with the formal parameters bound to the actual parameters.
So, the following are equivalent, where \xcd`e` is an expression involving
\xcd`x1` and \xcd`x2`\footnote{Strictly, there are a few other requirements;
  \eg, \xcd`result` must be a \xcd`var` of type \xcd`T` defined outside the
  outer block, the variables \xcd`a1` and \xcd`a2` had better not appear in
  \xcd`e`, and everything in sight had better typecheck properly.}

%~~gen
% package functions2.why.is.there.a.two;
% abstract class FunctionsTooManyFlippingFunctions[T, T1, T2]{
% abstract def arg1():T1;
% abstract def arg2():T2;
% def thing1(e:T) {var result:T;
%~~vis
\begin{xten}
{
  val f = (x1:T1,x2:T2){true}:T => e;
  val a1 : T1 = arg1();
  val a2 : T2 = arg2();
  result = f(a1,a2);
}
\end{xten}
%~~siv
%}}
%~~neg
and 
%~~gen
% package functions2.why.is.there.a.two.but.here.is.the.other.one;
% abstract class FunctionsTooManyFlippingFunctions[T, T1, T2]{
% abstract def arg1():T1;
% abstract def arg2():T2;
% def thing1(e:T) {var result:T;
%~~vis
\begin{xten}
{
  val a1 : T1 = arg1();
  val a2 : T2 = arg2();
  {
     val x1 : T1 = a1;
     val x2 : T2 = a2;
     result = e;
  }  
}
\end{xten}
%~~siv
%}}
%~~neg
\noindent
This doesn't quite work if the body is a statement rather than an expression.
A few language features are forbidden (\xcd`break` or \xcd`continue` of a loop
that surrounds the function literal) or mean something different (\xcd`return`
inside a function returns from the function). 





The \emph{method selector expression} \Xcd{e.m.(x1:T1,...,xn:Tn)} (\Sref{MethodSelectors})
permits the specification of the function underlying
the method \Xcd{m}, which takes arguments of type \Xcd{(x1:T1,..., xn:Tn)}.
Within this function, \Xcd{this} is bound to the result of evaluating \Xcd{e}.

Function types may be used in \Xcd{implements} clauses of class
definitions. Instances of such classes may be used as functions of the
given type.  Indeed, an object may behave like any (fixed) number of
functions, since the class it is an instance of may implement any
(fixed) number of function types.

%\section{Implementation Notes}
%\begin{itemize}
%
%\item Note that e.m.(T1,...,Tn) will evaluate e to create a
%  function. This function will be applied later to given
%  arguments. Thus this syntax can be used to evaluate the receiver of
%  a method call ahead of the actual invocation. The resulting function
%  can be used multiple times, of course.
%\end{itemize}


\section{Function Literals}
\index{literal!function}
\label{FunctionLiteral}

\Xten{} provides first-class, typed functions, including
\emph{closures}, \emph{operator functions}, and \emph{method
  selectors}.

\begin{grammar}
ClosureExpression \:
        \xcd"("
        Formals\opt
        \xcd")"
\\ &&
        Guard\opt
        ReturnType\opt
        Throws\opt
        Offers\opt
        \xcd"=>" ClosureBody \\
ClosureBody \:
        Expression \\
        \| \xcd"{" Statement\star \xcd"}" \\
        \| \xcd"{" Statement\star Expression \xcd"}" \\
\end{grammar}

Functions have zero or more formal parameters, an optional return type, an
optional set of exceptions throws by the body, and an optional type offered by
the body.  The body has the
same syntax as a method body; it may be either an expression, a block
of statements, or a block terminated by an expression to return. In
particular, a value may be returned from the body of the function
using a return statement (\Sref{ReturnStatement}). 

The type of a
function is a function type (\Sref{FunctionType}).  In some cases the
return type \Xcd{T} is also optional and defaults to the type of the
body. If a formal \Xcd{xi} does not occur in any
\Xcd{Tj}, \Xcd{c}, \Xcd{T} or \Xcd{e}, the declaration \Xcd{xi:Ti} may
be replaced by just \Xcd{Ti}: \xcd`(Int)=>7` is the integer function returning
7 for all inputs.

\label{ClosureGuard}

As with methods, a function may declare a guard to
constrain the actual parameters with which it may be invoked.
The guard may refer to the type parameters, formal parameters,
and any \xcd`val`s in scope at the function expression.

The body of the function is evaluated when the function is
invoked by a call expression (\Sref{Call}), not at the function's
place in the program text.

As with methods, a function with return type \xcd"Void" cannot
have a terminating expression. 
If the return type is omitted, it is inferred, as described in
\Sref{TypeInference}.
It is a static error if the return type cannot be inferred.  \Eg,
\xcd`(Int)=>null` is not well-defined; X10 does not know which type of
\xcd`null` is intended.  
%~~exp~~`~~`~~ ~~
But \xcd`(Int):Rail[Double] => null` is legal.


\begin{example}
The following method takes a function parameter and uses it to
test each element of the list, returning the first matching
element.  It returns \xcd`otherwise` if no element matches.

%~~gen
% package functions2.oh.no;
% import x10.util.*;
% class Finder {
% static 
%~~vis
\begin{xten}

def find[T](f: (T) => Boolean, xs: List[T]!, absent:T): T = {
  for (x: T in xs)
    if (f(x)) return x;
  absent
  }
\end{xten}
%~~siv
% }
%~~neg

The method may be invoked thus:
%~~gen
% package functions2.oh.no.my.ears;
% import x10.util.*;
% class Finderator {
% static def find[T](f: (T) => Boolean, xs: x10.util.List[T]!, absent:T): T = {
%  for (x: T in xs)
%    if (f(x)) return x;
%  absent
%}
% static def checkery() {
%~~vis
\begin{xten}
xs: List[Int]! = new ArrayList[Int]();
x: Int = find((x: Int) => x>0, xs, 0);
\end{xten}
%~~siv
%}}
%~~neg

\end{example}

As with a normal method, the function may have a \xcd"throws"
clause. It is a static error if the body of the function throws a
checked exception that is not declared in the function's \xcd"throws"
clause.

Similarly, it may have an \Xcd{offers T} clause; it is a static error if the
body offers any value not of type \Xcd{T}.

\subsection{Outer variable access}

In a function
\xcdmath"(x$_1$: T$_1$, $\dots$, x$_n$: T$_n$){c} => { s }"
the types \xcdmath"T$_i$", the guard \xcd"c" and the body \xcd"s"
may access many, though not all, sorts of variables from outer scopes.  
Specifically, they can access: 
\begin{itemize}
\item All fields of the enclosing object and class;
\item All type parameters;
\item All \xcd`val` variables;
\item \xcd`var` variables with the \xcd`shared` annotation. 
\end{itemize}


\limitation{\xcd`shared` is not currently supported.}

The function body may refer to instances of enclosing classes using
the syntax \xcd"C.this", where \xcd"C" is the name of the
enclosing class.  \xcd`this` refers to the instance of the immediately
enclosing class, as usual.

For example, the following is legal.  However, it would not be legal to add
\xcd`e` or \xcd`h` to the sum; they are non-\xcd`shared` \xcd`var`s from the
surrounding scope.

%%TODO -- this example uses 'shared', which is not currently available.
\begin{xten}
class Lambda {
   var a : Int = 0;
   val b = 0;
   def m(var c : Int, shared var d : Int,  val e : Int) {
      var f : Int = 0;
      shared var g : Int = 0;
      val h : Int = 0;
      val closure = (var i: Int, val j: Int) => {
    	  return a + b + d + g + i + j + this.a + Lambda.this.a;
      };
      return closure;
   }
}
\end{xten}


{\bf Rationale:} Non-\xcd`shared` \xcd`var`s like \xcd`e` and \xcd`h` are
excluded in X10, as in many other languages, for practical implementation
reasons. They are allocated on the stack, which is desirable for efficiency.
However, the closure may exist for long after the stack frame containing
\xcd`e` and \xcd`h` has been freed, so those storage locations are no longer
valid for those variables. \xcd`shared var`s are heap-allocated, which is less
efficient but allows them to exist after \xcd`m` returns. 


\xcd`shared` does not guarantee {\bf atomic} access to the shared variable. As
with any code that might mutate shared data concurrently, be sure to protect
references to mutable shared state with \xcd`atomic`. For example, the
following code returns a pair of closures which operate on the same shared
variable \xcd`a`, which are concurrency-safe---even if invoked many times
simultaneously. Without \xcd`atomic`, it would no longer be concurrency-safe.


%~fails~gen
% package Functions2.Are.All.Too.Much;
% class Fun2Frivols {
%~fails~vis
\begin{xten}
  def counters() {
      shared var a : Int = 0;
       return [
          () => {atomic a ++;},
          () => {atomic return a;}
          ];
   }
\end{xten}
%~fails~siv
%}
%
%~fails~neg


%SHARED% \begin{note}
%SHARED% The main activity may run in parallel with any
%SHARED% functions it creates. Hence even the read of an outer variable by the
%SHARED% body of a function may result in a race condition. Since functions are
%SHARED% first-class, the analysis of whether a function may execute in parallel
%SHARED% with the activity that created it may be difficult.
%SHARED% \end{note}

%% vj: This should be verified.
%\begin{note}
%The rule for accessing outer variables from function bodies
%should be the same as the rule for accessing outer variables from local
%or anonymous classes.
%\end{note}

\section{Method selectors}
\label{MethodSelectors}
\index{function!method selector}
\index{method!underlying function}

A method selector expression allows a method to be used as a
first-class function, without writing a function expression for it.
For example, consider a class \xcd`Span` defining ranges of integers.  

%~~gen
% package Functions2.Span;
%~~vis
\begin{xten}
class Span(low:Int, high:Int) {
  def this(low:Int, high:Int) {property(low,high);}
  def between(n:Int) = low <= n && n <= high;
  def example() {
    val digit = new Span(0,9);
    val isDigit : (Int) => Boolean = digit.between.(Int);
    if (isDigit(8)) x10.io.Console.OUT.println("8 is!");
  }
}
\end{xten}
%~~siv
%
%~~neg
\noindent


In \xcd`example()`, 
%~~exp~~`~~`~~ digit:Span!~~class Span(low:Int, high:Int) {def this(low:Int, high:Int) {property(low,high);} def between(n:Int) = low <= n && n <= high;}
\xcd`digit.between.(Int)` 
is a unary function testing whether its argument is between zero
and nine.  It could also be written 
%~~exp~~`~~`~~ digit:Span!~~class Span(low:Int, high:Int) {def this(low:Int, high:Int) {property(low,high);} def between(n:Int) = low <= n && n <= high;}
\xcd`(n:Int) => digit.between(n)`.

This is formalized thus:

\begin{grammar}
MethodSelector \:
        Primary \xcd"."
        MethodName \xcd"."
                TypeParameters\opt \xcd"(" Formals\opt \xcd")" \\
      \|
        TypeName \xcd"."
        MethodName \xcd"."
                TypeParameters\opt \xcd"(" Formals\opt \xcd")" \\
\end{grammar}

The \emph{method selector expression} \Xcd{e.m.(T1,...,Tn)} is type
correct only if  the static type of \Xcd{e} is a
class or struct or interface \xcd`V` with a method
\Xcd{m(x1:T1,...xn:Tn)\{c\}:T} defined on it (for some
\Xcd{x1,...,xn,c,T)}. At runtime the evaluation of this expression
evaluates \Xcd{e} to a value \Xcd{v} and creates a function \Xcd{f}
which, when applied to an argument list \Xcd{(a1,...,an)} (of the right
type) yields the value obtained by evaluating \Xcd{v.m(a1,...,an)}.

Thus, the method selector

\begin{xtenmath}
e.m.[X$_1$, $\dots$, X$_m$](T$_1$, $\dots$, T$_n$)
\end{xtenmath}
\noindent behaves as if it were the function
\begin{xtenmath}
((v:V)=>
  [X$_1$, $\dots$, X$_m$](x$_1$: T$_1$, $\dots$, x$_n$: T$_n$){c} 
  => v.m[X$_1$, $\dots$, X$_m$](x$_1$, $\dots$, x$_n$))
(e)
\end{xtenmath}


\limitation{X10 functions, including method selectors, do not currently accept
generic arguments.}

Because of overloading, a method name is not sufficient to
uniquely identify a function for a given class (in Java-like languages).
One needs the argument type information as well.
The selector syntax (dot) is used to distinguish \xcd"e.m()" (a
method invocation on \xcd"e" of method named \xcd"m" with no arguments)
from \xcd"e.m.()"
(the function bound to the method). 

A static method provides a binding from a name to a function that is
independent of any instance of a class; rather it is associated with the
class itself. The static function selector
\xcdmath"T.m.(T$_1$, $\dots$, T$_n$)" denotes the
function bound to the static method named \xcd"m", with argument types
\xcdmath"(T$_1$, $\dots$, T$_n$)" for the type \xcd"T". The return type
of the function is specified by the declaration of \xcd"T.m".

Users of a function type do not care whether a function was defined
directly (using the function syntax), or obtained via (static or
instance) function selectors.


\section{Operator functions}
\label{OperatorFunction}
\index{function!operator}
Every operator (e.g.,
\xcd"+",
\xcd"-",
\xcd"*",
\xcd"/",
\dots) has a family of functions, one for
each type on which the operator is defined. The function can be
selected using the "." syntax:

\begin{grammar}
OperatorFunction
        \: TypeName \xcd"." Operator \xcd"(" Formals\opt \xcd")" \\
        \| TypeName \xcd"." Operator \\
\end{grammar}

If an operator has more than one arity (\eg, unary and binary
\xcd"-"), the unary version may be selected by giving the
formal parameter types.  The binary version is selected by
default, or the types may be specified for clarity.
For example, the following equivalences hold:

\begin{xtenmath}
String.+             $\equiv$ (x: String, y: String): String => x + y
Long.-               $\equiv$ (x: Long, y: Long): Long => x - y
Float.-(Float,Float) $\equiv$ (x: Float, y: Float): Float => x - y
Int.-(Int)           $\equiv$ (x: Int): Int => -x
Boolean.&            $\equiv$ (x: Boolean, y: Boolean): Boolean => x & y
Boolean.!            $\equiv$ (x: Boolean): Boolean => !x
Int.<(Int,Int)       $\equiv$ (x: Int, y: Int): Boolean => x < y
Dist.|(Place)        $\equiv$ (d: Dist, p: Place): Dist => d | p
\end{xtenmath}


%%TODO -- fix commented-out lines!

%~~gen
% package Functions2.For.The.Lose;
% class TypecheckThatSillyExample {
%   def checker() {
%    val l1 : (String, String) => String = String.+;
%    val r1 : (String, String) => String = (x: String, y: String): String => x + y;
%    val l2 : (Long,Long) => Long = Long.-;
%    val r2 : (Long,Long) => Long = (x: Long, y: Long): Long => x - y;
%//var v1 : (Float,Float) => Float = Float.-(Float,Float) ;
%var v2 : (Float,Float) => Float = (x: Float, y: Float): Float => x - y;
%//var v3 : (Int) => Int =  Int.-(Int)     ;      ;
%var v4  : (Int) => Int  =  (x: Int): Int => -x;
%var v5 : (Boolean,Boolean) => Boolean = Boolean.&            ;
%var v6 : (Boolean,Boolean) => Boolean =  (x: Boolean, y: Boolean): Boolean => x & y;
%//var v7 : (Boolean) => Boolean = Boolean.!            ;
%var v8 : (Boolean) => Boolean =  (x: Boolean): Boolean => !x;
%//var v9 : (Int,Int) => Boolean = Int.<(Int,Int)       ;
%var v10: (Int,Int) => Boolean =  (x: Int, y: Int): Boolean => x < y;
%//var v11 : (Dist,Place)=>Dist = Dist.|(Place)        ;
%var v12 : (Dist,Place)=>Dist=  (d: Dist, p: Place): Dist => d | p;
%}
% }
%~~vis
%~~siv
%
%~~neg

Unary and binary promotion (\Sref{XtenPromotions}) is not performed
when invoking these
operations; instead, the operands are coerced individually via implicit
coercions (\Sref{XtenConversions}), as appropriate.


\begin{planned}

{\bf The following is not implemented in version 2.0.3:}

Additionally, for every expression \xcd"e" of a type \xcd"T" at which a binary
operator \xcd"OP" is defined, the expression \xcd"e.OP" or
\xcd"e.OP(T)" represents the function
defined by:

\begin{xten}
(x: T): T => { e OP x }
\end{xten}

\begin{grammar}
Primary \: Expr \xcd"." Operator \xcd"(" Formals\opt \xcd")" \\
        \| Expr \xcd"." Operator \\
\end{grammar}

%% For every expression \xcd"e" of a type \xcd"T" at which a unary
%%operator \xcd"OP" is defined, the expression \xcd"e.OP()"
%% represents the function defined by:

%% \begin{xten}
%% (): T => { OP e }
%% \end{xten}

For example,
one may write an expression that adds one to each member of a
list \xcd"xs" by:

%%TODO -- when this topic works, make the example wwork too.
%~x~gen
% package Functions2.Wants.A.Dinner.Reservation;
% import x10.util.*;
% class Reservation {
% def smerp() {
%   val xs = new ArrayList[Int]();
%~x~vis
\begin{xten}
xs.map(1.+);
\end{xten}
%~x~siv
% }
% }
%
%~x~neg
\end{planned}


\section{Functions as objects of type \Xcd{Any}}
\label{FunctionAnyMethods}

\label{FunctionEquality}
\index{function!equality} \index{equality!function} Two functions \Xcd{f} and
\Xcd{g} are equal (``\Xcd{==}'') if both are instances of classes and the same
object, or if both were obtained by the same evaluation of a function
literal.\footnote{A literal may occur in program text within a loop, and hence
  may be evaluated multiple times.} Further, it is guaranteed that if two
functions are equal then they refer to the same locations in the environment
and represent the same code, so their executions in an identical situation are
indistinguishable. (Specifically, if \xcd`f == g`, then \xcd`f(1)` can be
substituted for \xcd`g(1)` and the result will be identical. However, there is
no guarantee that \xcd`f(1)==g(1)` will evaluate to true. Indeed, there is no
guarantee that \xcd`f(1)==f(1)` will evaluate to true either, as \xcd`f` might
be a function which returns {$n$} on its {$n^{th}$} invocation. However,
\xcd`f(1)==f(1)` and \xcd`f(1)==g(1)` are interchangeable.)
\index{function!==}


Every function type implements all the methods of \Xcd{Any}.
b\xcd`f.equals(g)` is equivalent to \xcd`f==g`.  \xcd`f.hashCode()`, 
\xcd`f.toString()`, and \xcd`f.typeName()` are implementation-dependent, but
respect \xcd`equals` and the basic contracts of \xcd`Any`. 
\xcd`f.home()` returns \xcd`here` and \xcd`f.at(x)`
always returns true, as for structs.

\index{function!equals}
\index{function!hashCode}
\index{function!toString}
\index{function!typeName}
\index{function!home}
\index{function!at(Place)}
\index{function!at(Object)}



\chapter{Expressions}\label{XtenExpressions}\index{expression}

\Xten{} has a rich expression language.
Evaluating an expression produces a value, or, in a few cases, no value. 
Expression evaluation may have side effects, such as change of the value of a 
\xcd`var` variable or a data structure, allocation of new values, or throwing
an exception. 



\section{Literals}
\index{literal}

Literals denote fixed values of built-in types. 
The syntax for literals is given in \Sref{Literals}. 

The type that \Xten{} gives a literal often includes its value. \Eg, \xcd`1`
is of type \xcd`Int{self==1}`, and \xcd`true` is of type
\xcd`Boolean{self==true}`.

\section{{\tt this}}
\index{this}
\index{\Xcd{this}}

\begin{bbgrammar}
%(FROM #(prod:Primary)#)
             Primary \: \xcd"this" (\ref{prod:Primary}) \\
                    \| \xcd"this" \\
                    \| ClassName \xcd"." \xcd"this" \\
\end{bbgrammar}


The expression \xcd"this" is a  local \xcd`val` containing a reference
to an instance of the lexically enclosing class.
It may be used only within the body of an instance method, a
constructor, or in the initializer of a instance field -- that is, the places
where there is an instance of the class under consideration.

Within an inner class, \xcd"this" may be qualified with the
name of a lexically enclosing class.  In this case, it
represents an instance of that enclosing class.  


\begin{ex}
\xcd`Outer` is a class containing \xcd`Inner`.  Each instance of
\xcd`Inner` has a reference \xcd`Outer.this` to the \xcd`Outer` involved in its
creation.  \xcd`Inner` has access to the fields of \xcd`Outer.this`, as seen
in the \xcd`outerThree` and \xcd`alwaysTrue` methods.  Note that \xcd`Inner`
has its own \xcd`three` field, which is different from and not even the same
type as \xcd`Outer.this.three`. 
%~~gen ^^^ Expressions10
% package exp.vexp.pexp.lexp.shexp; 
% NOTEST
%~~vis 
\begin{xten}
class Outer {
  val three = 3;
  class Inner {
     val three = "THREE";
     def outerThree() = Outer.this.three;
     def alwaysTrue() = outerThree() == 3;
  }
}
\end{xten}
%~~siv
%
%~~neg
\end{ex}

The type of a \xcd"this" expression is the
innermost enclosing class, or the qualifying class,
constrained by the class invariant and the
method guard, if any.

The \xcd"this" expression may also be used within constraints in
a class or interface header (the class invariant and
\xcd"extends" and \xcd"implements" clauses).  Here, the type of
\xcd"this" is restricted so that only properties declared in the
class header itself, and specifically not any members declared in the class
body or in supertypes, are accessible through \xcd"this".

\section{Local variables}

%##(Id
\begin{bbgrammar}
%(FROM #(prod:Id)#)
                  Id \: identifier & (\ref{prod:Id}) \\
\end{bbgrammar}
%##)

A local variable expression consists simply of the name of the local variable,
field of the current object, formal parameter in scope, etc. It evaluates to
the value of the local variable. 


\begin{ex}
\xcd`n` in the second line below is a local
variable expression.  The \xcd`n` in the first line is not; it is part of a
local variable declaration.
%~~gen  ^^^ Expressions20
% package exp.loc.al.varia.ble; 
% class Example {
% def example() { 
%~~vis
\begin{xten}
val n = 22;
val m = n + 56;
\end{xten}
%~~siv
%} }
%~~neg

\end{ex}

\section{Field access}
\label{FieldAccess}
\index{field!access to}

%##(FieldAccess
\begin{bbgrammar}
%(FROM #(prod:FieldAccess)#)
         FieldAccess \: Primary \xcd"." Id & (\ref{prod:FieldAccess}) \\
                    \| \xcd"super" \xcd"." Id \\
                    \| ClassName \xcd"." \xcd"super"  \xcd"." Id \\
                    \| Primary \xcd"." \xcd"class"  \\
                    \| \xcd"super" \xcd"." \xcd"class"  \\
                    \| ClassName \xcd"." \xcd"super"  \xcd"." \xcd"class"  \\
\end{bbgrammar}
%##)

A field of an object instance may be  accessed
with a field access expression.

The type of the access is the declared type of the field with the
actual target substituted for \xcd"this" in the type. 

\begin{ex}
The declaration of \xcd`b` below has a constraint involving \xcd`this`.  
The use of an instance of it, \xcd`f.b`, has the same constraint involving
\xcd`f` instead of \xcd`this`, as required.
%~~gen ^^^ Expressions5s7v
% package Expressions5s7v;
%~~vis
\begin{xten}
class Fielded {
  public val a : Int = 1;
  public val b : Int{this.a == b} = this.a;
  static def example() {
    val f : Fielded = new Fielded();
    val fb : Int{fb == f.a} = f.b;
  }
}
\end{xten}
%~~siv
%
%~~neg

\end{ex}
% If the actual
%target is not a final access path (\Sref{FinalAccessPath}),
%an anonymous path is substituted for \xcd"this".

The field accessed is selected from the fields and value properties
of the static type of the target and its superclasses.

If the field target is given by the keyword \xcd"super", the target's type is
the superclass of the enclosing class.  This form is used to access fields of
the parent class shadowed by same-named fields of the current class.

If the field target is \xcd`Cls.super`, then the target's type is \xcd`Cls`,
which must be an  enclosing class.  This (admittedly
obscure) form is used to access fields of an ancestor class which are shadowed
by same-named fields of some more recent ancestor.  

\begin{ex}
This illustrates all four cases of field access.
%~~gen ^^^ Expressions30
% package exp.re.ssio.ns.fiel.dacc.ess;
% NOTEST
%~~vis
\begin{xten}
class Uncle {
  public static val f = 1;
}
class Parent {
  public val f = 2;
}
class Ego extends Parent {
  public val f = 3;
  class Child extends Ego {
     public val f = 4;
     def classNameDotId() =  Uncle.f;     // 1
     def cnDotSuperDotId() = Ego.super.f; // 2
     def superDotId() =      super.f;     // 3
     def expDotId() =        this.f;      // 4
  }
}
\end{xten}
%~~siv
%
%~~neg
\end{ex}

If the field target is \xcd"null", a \xcd"NullPointerException"
is thrown.
If the field target is a class name, a static field is selected.
It is illegal to access  a field that is not visible from
the current context.
It is illegal to access a non-static field
through a static field access expression.  However, it is legal to access a
static field through a non-static reference.

\section{Function Literals}
Function literals are described in \Sref{Functions}.

\section{Calls}
\label{Call}
\label{MethodInvocation}
\label{MethodInvocationSubstitution}
\index{invocation}
\index{call}
\index{invocation!method}
\index{call!method}
\index{invocation!function}
\index{call!function}
\index{method!calling}
\index{method!invoking}


%##(MethodInvocation ArgumentList
\begin{bbgrammar}
%(FROM #(prod:MethodInvocation)#)
    MethodInvocation \: MethodPrimaryPrefix \xcd"(" ArgumentList\opt \xcd")" & (\ref{prod:MethodInvocation}) \\
                    \| MethodSuperPrefix \xcd"(" ArgumentList\opt \xcd")" \\
                    \| MethodClassNameSuperPrefix \xcd"(" ArgumentList\opt \xcd")" \\
                    \| MethodName TypeArguments\opt \xcd"(" ArgumentList\opt \xcd")" \\
                    \| Primary \xcd"." Id TypeArguments\opt \xcd"(" ArgumentList\opt \xcd")" \\
                    \| \xcd"super" \xcd"." Id TypeArguments\opt \xcd"(" ArgumentList\opt \xcd")" \\
                    \| ClassName \xcd"." \xcd"super"  \xcd"." Id TypeArguments\opt \xcd"(" ArgumentList\opt \xcd")" \\
                    \| Primary TypeArguments\opt \xcd"(" ArgumentList\opt \xcd")" \\
%(FROM #(prod:ArgumentList)#)
        ArgumentList \: Exp & (\ref{prod:ArgumentList}) \\
                    \| ArgumentList \xcd"," Exp \\
\end{bbgrammar}
%##)


A \grammarrule{MethodInvocation} may be to either a \xcd"static" method, an
instance method, or a closure.

The syntax for method invocations is ambiguous. \xcd`ob.m()` could either be
the invocation of a method named \xcd`m` on object \xcd`ob`, or the
application of a function held in a field \xcd`ob.m`. The target \xcd`ob` must
be type-checked to determine which of these it is.  It is a static error if
both cases are possible after type checking.

\begin{ex}
%~~gen ^^^ Expressions40
% package expres.sio.nsca.lls;
%~~vis
\begin{xten}
class Callsome {
  static val closure : () => Int = () => 1;
  static def method()            = 2;
  static def example() {
     assert Callsome.closure() == 1;
     assert Callsome.method()  == 2;
  } 
}
\end{xten}
%~~siv
% class Hook{ def run() { Callsome.example(); return true; } }
%~~neg
However, adding a static method [mis]named \xcd`closure` makes
\xcd`Callsome.closure()` 
ambiguous: it could be a call to the closure, or to the static method: 
%~~gen ^^^ Expressions50
% package expres.sio.nsca.lls.twoooo;
% class Callsome {static val closure = () => 1; static def method () = 2; static val methodEvaluated = Callsome.method();
%~~vis
\begin{xten}
  static def closure () = 3;
  // ERROR: static errory = Callsome.closure();
\end{xten}
%~~siv
% }
%~~neg
\end{ex}

The application form \xcd`e(f,g)`, when \xcd`e` evaluates to an object or
struct, invokes the application \xcd`operator`, 
defined in the form 
%~~gen ^^^ Expressions2x1f
% package Expressions2x1f;
% class Example[F,G] {
%~~vis
\begin{xten}
public operator this(f:F, g:G) = "value";
\end{xten}
%~~siv
%  }
%~~neg


Method selection rules are given in \Sref{sect:MethodResolution}.

It is a static error if a method's \grammarrule{Guard} is not statically
satisfied by the 
caller.  

\begin{ex}
In this example, a \xcd`DivideBy` object provides the service of dividing
numbers by \xcd`denom` --- so long as \xcd`denom` is not zero. 
In the \xcd`example` method, \xcd`this.div(100)`  is not allowed; there is no
guarantee that \xcd`denom != 0`.  Casting \xcd`this` to a type 
whose constraint implies \xcd`denom != 0` permits the method call.
%~~gen ^^^ Expressions60
%package Expressions.Calls.Guarded.By.Walls;
% KNOWNFAIL
%~~vis
\begin{xten}
class DivideBy(denom:Int) {
  def div(numer:Int){denom != 0} = numer / denom;
  def example() {
     val thisCast = (this as DivideBy{self.denom != 0});
     thisCast.div(100);
     //ERROR: this.div(100); 
  }
}
\end{xten}
%~~siv
% class Hook{ def run() { (new DivideBy(1)).example(); return true; } }
%~~neg
\end{ex}

\section{Assignment}\index{assignment}\label{AssignmentStatement}

%##(Assignment LeftHandSide AssignmentOperator
\begin{bbgrammar}
%(FROM #(prod:Assignment)#)
          Assignment \: LeftHandSide AssignmentOperator AssignmentExp & (\ref{prod:Assignment}) \\
                    \| ExpName  \xcd"(" ArgumentList\opt \xcd")" AssignmentOperator AssignmentExp \\
                    \| Primary  \xcd"(" ArgumentList\opt \xcd")" AssignmentOperator AssignmentExp \\
%(FROM #(prod:LeftHandSide)#)
        LeftHandSide \: ExpName & (\ref{prod:LeftHandSide}) \\
                    \| FieldAccess \\
%(FROM #(prod:AssignmentOperator)#)
  AssignmentOperator \: \xcd"=" & (\ref{prod:AssignmentOperator}) \\
                    \| \xcd"*=" \\
                    \| \xcd"/=" \\
                    \| \xcd"%=" \\
                    \| \xcd"+=" \\
                    \| \xcd"-=" \\
                    \| \xcd"<<=" \\
                    \| \xcd">>=" \\
                    \| \xcd">>>=" \\
                    \| \xcd"&=" \\
                    \| \xcd"^=" \\
                    \| \xcd"|=" \\
\end{bbgrammar}
%##)



The assignment expression \xcd"x = e" assigns a value given by
expression \xcd"e"
to a variable \xcd"x".  
Most often, \xcd`x` is mutable, a \xcd`var` variable.  The same syntax is
used for delayed initialization of a \xcd`val`, but \xcd`val`s can only be
initialized once.
%~~gen ^^^ Expressions70
% package express.ions.ass.ignment;
% class Example {
% static def exasmple() {
%~~vis
\begin{xten}
  var x : Int;
  val y : Int;
  x = 1;
  y = 2; // Correct; initializes y
  x = 3; 
  // ERROR: y = 4;
\end{xten}
%~~siv
% } } 
%~~neg


There are three syntactic forms of
assignment: 
\begin{enumerate}
\item \xcd`x = e;`, assigning to a local variable, formal parameter, field of
      \xcd`this`, etc. 
\item \xcd`x.f = e;`, assigning to a field of an object.
\item \xcdmath`a(i$_1$,$\ldots$,i$_n$) = v;`, where {$n \ge 0$}, assigning to
      an element of an array or some other such structure. This is an operator
      call (\Sref{sect:operators}).  For well-behaved classes it works like
      array assignment, mutatis mutandis, but there is no actual guarantee,
      and the compiler makes no assumptions about how this works for arbitrary \xcd`a`.
      Naturally, it is a static error if no suitable assignment operator
      for \xcd`a`.
\end{enumerate}

For a binary operator $\diamond$, the $\diamond$-assignment expression
\xcdmath"x $\diamond$= e" combines the current value of \xcd`x` with the value
of \xcd`e` by {$\diamond$}, and stores the result back into \xcd`x`.  
\xcd`i += 2`, for example, adds 2 to \xcd`i`. For variables and fields, 
\xcdmath"x $\diamond$= e" behaves just like \xcdmath"x = x $\diamond$ e". 

The subscripting forms of \xcdmath"a(i) $\diamond$= b" are slightly subtle.
Subexpressions of \xcd`a` and \xcd`i` are only evaluated once.  However,
\xcd`a(i)` and \xcd`a(i)=c` are each executed once---in particular, there is
one call to the application operator, and one to the assignment operator.
If subscripting is implemented strangely for
the class of \xcd`a`, the behavior is {\em not} necessarily updating a single
storage location. Specifically, \xcd`A()(I()) += B()` is tantamount to: 
%~~gen ^^^ Expressions80
% package expressions.stupid.addab;
% class Example {
% def example(A:()=>Rail[Int], I: () => Int, B: () => Int ) {
%~~vis
\begin{xten}
{
  val aa = A();  // Evaluate A() once
  val ii = I();  // Evaluate I() once
  val bb = B();  // Evaluate B() once
  val tmp = aa(ii) + bb; // read aa(ii)
  aa(ii) = tmp;  // write sum back to aa(ii)
}
\end{xten}
%~~siv
%}}
%~~neg

\limitation{+= does not currently meet this specification.}




\section{Increment and decrement}
\index{increment}
\index{decrement}
\index{\Xcd{++}}
\index{\Xcd{--}}


The operators \xcd"++" and \xcd"--" increment and decrement
a variable, respectively.  
\xcd`x++` and \xcd`++x` both increment \xcd`x`, just as the statement 
\xcd`x += 1` would, and similarly for \xcd`--`.  

The difference between the two is the return value.  
\xcd`++x` and \xcd`--x` return the {\em new} value of \xcd`x`, after
incrementing or decrementing.
\xcd`x++` and \xcd`x--` return the {\em old} value of \xcd`x`, before
incrementing or decrementing.


\limitation{This currently only works for numeric types.}

\section{Numeric Operations}
\label{XtenPromotions}
\index{promotion}
\index{numeric promotion}
\index{numeric operations}
\index{operation!numeric}

Numeric types (\xcd`Byte`, \xcd`Short`, \xcd`Int`, \xcd`Long`, \xcd`Float`,
\xcd`Double`, \xcd`Complex`, and unsigned variants of fixed-point types) are normal X10
structs, though most of their methods are implemented via native code. They
obey the same general rules as other X10 structs. For example, numeric
operations, coercions, and conversions are defined by \xcd`operator` definitions, the same way you could
for any struct.

Promoting a numeric value to a longer numeric type preserves the sign of the
value.  For example, \xcd`(255 as UByte) as UInt` is 255. 

Most of these operations can be defined on user-defined types as well.  While
it is good practice to keep such operations consistent with the numeric
operations whenever possible, the compiler neither enforces nor assumes any
particular semantics of user-defined operations. 

\subsection{Conversions and coercions}

Specifically, each numeric type can be converted or coerced into each other
numeric type, perhaps with loss of accuracy.
%~~gen ^^^ Expressions90
% package exp.ress.io.ns.numeric.conversions;
% class ExampleOfConversionAndStuff {
% def example() {
%~~vis
\begin{xten}
val n : Byte = 123 as Byte; // explicit 
val f : (Int)=>Boolean = (Int) => true; 
val ok = f(n); // implicit
\end{xten}
%~~siv
% } }
%~~neg



\subsection{Unary plus and unary minus}

The unary \xcd`+` operation on numbers is an identity function.
The unary \xcd`-` operation on numbers is a negation function.
On unsigned numbers, these are two's-complement.  For example, 
\xcd`-(0x0F as UByte)` is 
\xcd`(0xF1 as UByte)`.
\bard{UInts and such are closed under negation -- the negative of a UInt is
done binarily.  }



\section{Bitwise complement}

The unary \xcd"~" operator, only defined on integral types, complements each
bit in its operand.  

\section{Binary arithmetic operations} 

The binary arithmetic operators perform the familiar binary arithmetic
operations: \xcd`+` adds, \xcd`-` subtracts, \xcd`*` multiplies, 
\xcd`/` divides, and \xcd`%`
computes remainder.

On integers, the operands are coerced to the longer of their two types, and
then operated upon.  
Floating point operations are determined by the IEEE 754
standard. 
The integer \xcd"/" and \xcd"%" throw an exception 
if the right operand is zero.



\section{Binary shift operations}

The operands of the binary shift operations must be of integral type.
The type of the result is the type of the left operand.
The right operand, describing a number of bits, must be unsigned: 
%~~exp~~`~~`~~ x:Int ~~ ^^^Expressions1l4m
\xcd`x << 1U`.  


If the promoted type of the left operand is \xcd"Int",
the right operand is masked with \xcd"0x1f" using the bitwise
AND (\xcd"&") operator, giving a number at most the number of bits in an
\xcd`Int`. 
If the promoted type of the left operand is \xcd"Long",
the right operand is masked with \xcd"0x3f" using the bitwise
AND (\xcd"&") operator, giving a number at most the number of bits in a
\xcd`Long`. 

The \xcd"<<" operator left-shifts the left operand by the number of
bits given by the right operand.
The \xcd">>" operator right-shifts the left operand by the number of
bits given by the right operand.  The result is sign extended;
that is, if the right operand is $k$,
the most significant $k$ bits of the result are set to the most
significant bit of the operand.

The \xcd">>>" operator right-shifts the left operand by the number of
bits given by the right operand.  The result is not sign extended;
that is, if the right operand is $k$,
the most significant $k$ bits of the result are set to \xcd"0".
This operation is deprecated, and may be removed in a later version of the
language. 


\section{Binary bitwise operations}

The binary bitwise operations operate on integral types, which are promoted to
the longer of the two types.
The \xcd"&" operator  performs the bitwise AND of the promoted operands.
The \xcd"|" operator  performs the bitwise inclusive OR of the promoted operands.
The \xcd"^" operator  performs the bitwise exclusive OR of the promoted operands.

\section{String concatenation}
\index{string!concatenation}

The \xcd"+"  operator is used for string concatenation 
 as well as addition.
If either operand is of static type \xcd"x10.lang.String",
 the other operand is converted to a \xcd"String" , if needed,
  and  the two strings  are concatenated.
 String conversion of a non-\xcd"null" value is  performed by invoking the
 \xcd"toString()" method of the value.
  If the value is \xcd"null", the value is converted to 
  \xcd'"null"'.

The type of the result is \xcd"String".

 For example, 
%~~exp~~`~~`~~ ~~ ^^^ Expressions100
      \xcd`"one " + 2 + here` 
      evaluates to  \xcd`one 2(Place 0)`.  

\section{Logical negation}

The operand of the  unary \xcd"!" operator 
must be of type \xcd"x10.lang.Boolean".
The type of the result is \xcd"Boolean".
If the value of the operand is \xcd"true", the result is \xcd"false"; if
if the value of the operand  is \xcd"false", the result is \xcd"true".

\section{Boolean logical operations}

Operands of the binary boolean logical operators must be of type \xcd"Boolean".
The type of the result is \xcd"Boolean"

The \xcd"&" operator  evaluates to \xcd"true" if both of its
operands evaluate to \xcd"true"; otherwise, the operator
evaluates to \xcd"false".

The \xcd"|" operator  evaluates to \xcd"false" if both of its
operands evaluate to \xcd"false"; otherwise, the operator
evaluates to \xcd"true".

\section{Boolean conditional operations}

Operands of the binary boolean conditional operators must be of type
\xcd"Boolean". 
The type of the result is \xcd"Boolean"

The \xcd"&&" operator  evaluates to \xcd"true" if both of its
operands evaluate to \xcd"true"; otherwise, the operator
evaluates to \xcd"false".
Unlike the logical operator \xcd"&",
if the first operand is \xcd"false",
the second operand is not evaluated.

The \xcd"||" operator  evaluates to \xcd"false" if both of its
operands evaluate to \xcd"false"; otherwise, the operator
evaluates to \xcd"true".
Unlike the logical operator \xcd"||",
if the first operand is \xcd"true",
the second operand is not evaluated.

\section{Relational operations} 

The relational operations on numeric types compare numbers, producing
\xcd`Boolean` results.

The \xcd"<" operator evaluates to \xcd"true" if the left operand is
less than the right.
The \xcd"<=" operator evaluates to \xcd"true" if the left operand is
less than or equal to the right.
The \xcd">" operator evaluates to \xcd"true" if the left operand is
greater than the right.
The \xcd">=" operator evaluates to \xcd"true" if the left operand is
greater than or equal to the right.

Floating point comparison is determined by the IEEE 754
standard.  Thus,
if either operand is NaN, the result is \xcd"false".
Negative zero and positive zero are considered to be equal.
All finite values are less than positive infinity and greater
than negative infinity.



\section{Conditional expressions}
\index{\Xcd{? :}}
\index{conditional expression}
\index{expression!conditional}
\label{Conditional}

%##(ConditionalExp
\begin{bbgrammar}
%(FROM #(prod:ConditionalExp)#)
      ConditionalExp \: ConditionalOrExp & (\ref{prod:ConditionalExp}) \\
                    \| ClosureExp \\
                    \| AtExp \\
                    \| FinishExp \\
                    \| ConditionalOrExp \xcd"?" Exp \xcd":" ConditionalExp \\
\end{bbgrammar}
%##)

A conditional expression evaluates its first subexpression (the
condition); if \xcd"true"
the second subexpression (the consequent) is evaluated; otherwise,
the third subexpression (the alternative) is evaluated.

The type of the condition must be \xcd"Boolean".
The type of the conditional expression is some common 
ancestor (as constrained by \Sref{LCA}) of the types of the consequent and the
alternative. 

\begin{ex}
%~~exp~~`~~`~~a:Int,b:Int ~~ ^^^ Expressions110
\xcd`a == b ? 1 : 2`
evaluates to \xcd`1` if \xcd`a` and \xcd`b` are the same, and \xcd`2` if they
are different.   As the type of \xcd`1` is \xcd`Int{self==1}` and of \xcd`2`
is \xcd`Int{self==2}`, the type of the conditional expression has the form
\xcd`Int{c}`, where \xcd`self==1` and \xcd`self==2` both imply \xcd`c`.  For
example, it might be \xcd`Int{true}` -- or perhaps it might be 
\xcd`Int{self != 8}`. Note that this term has no most accurate type in the X10
type system.
\end{ex}

The subexpression not selected is not evaluated.

\begin{ex}
The following use of the conditional expression prevents division by zero.  If
\xcd`den==0`, the division is not performed at all.
%~~gen ^^^ Expressions4t3m
% package Expressions4t3m;
% class Hook {
% static def example(num:Int, den:Int ) =
%~~vis
\begin{xten}
(den == 0) ? 0 : num/den
\end{xten}
%~~siv
%; 
% def run() { 
%   return example(1,0) == 0 && example(6,3) == 2;
% } }
%~~neg

Similarly, the following code performs a method call if \xcd`op` is non-null,
and avoids the null pointer error if it is null.  Defensive coding like this
is quite common when working with possibly-null objects.
%~~gen ^^^ Expressions6o2b
% package Expressions6o2b;
% class Hook { 
% static def example(ob:Object) = 
%~~vis
\begin{xten}
(ob == null) ? null : ob.toString();
\end{xten}
%~~siv
%def run() {
%  return example(null) == null && example("yes").equals("yes"); 
% } } 
%~~neg



\end{ex}

\section{Stable equality}
\label{StableEquality}
\index{\Xcd{==}}
\index{equality}

\begin{bbgrammar}
 EqualityExp    \: RelationalExp & (\ref{prod:EqualityExp})\\
%<FROM #(prod:EqualityExp)#
    \| EqualityExp \xcd"==" RelationalExp\\
    \| EqualityExp \xcd"!=" RelationalExp\\
    \| Type  \xcd"==" Type \\
\end{bbgrammar}


The \xcd"==" and \xcd"!=" operators provide a fundamental, though
non-abstract, notion of equality.  \xcd`a==b` is true if the values of \xcd`a`
and \xcd`b` are extremely identical.

\begin{itemize}
\item If \xcd`a` and \xcd`b` are values of object type, then \xcd`a==b` holds
      if \xcd`a` and \xcd`b` are the same object.
\item If one operand is \xcd`null`, then \xcd`a==b` holds iff the other is
      also \xcd`null`.
\item If the operands both have struct type, then they must be structurally equal;
that is, they must be instances of the same struct
and all their fields or components must be \xcd"==". 
\item The definition of equality for function types is specified in
      \Sref{FunctionEquality}.
\item No implicit coercions are performed by \xcd`==`.  
\item It is a static error to have an expression \xcd`a == b` if the types of
      \xcd`a` and \xcd`b` are disjoint.  
\end{itemize}

\xcd`a != b`
is true iff \xcd`a==b` is false.

The predicates \xcd"==" and \xcd"!=" may not be overridden by the programmer.

\xcd`==` provides a {\em stable} notion of equality.  If two values are
\xcd`==` at any time, they remain \xcd`==` forevermore, regardless of what
happens to the mutable state of the program. 

\begin{ex}
Regardless of the values and types of \xcd`a` and \xcd`b`, 
or the behavior of \xcd`any_code_at_all` (which may, indeed, be
any code at all---not just a method call), the value of 
\xcd`a==b` does not change: 
%~~gen ^^^ Expressions1i5k
% package Expressions1i5k;
% class Example{ 
% def example( something: ()=>Int, something_else: ()=>Int,
%   any_code_at_all: () => Int) {
%~~vis
\begin{xten}
val a = something();
val b = something_else();
val eq1 = (a == b);
any_code_at_all();
val eq2 = (a == b);
assert eq1 == eq2;
\end{xten}
%~~siv
%} } 
%~~neg
\end{ex}

\subsection{No Implicit Coercions}
\label{sect:eqeq-no-coerce}

\xcd`==` is a primitive operation in X10 -- one of very few. Most operations,
like \xcd`+` and \xcd`<=`, are defined as \xcd`operator`s. \xcd`==` and
\xcd`!=` are not. As non-\xcd`operator`s, they need not and do not follow the
general method resolution procedure of \Sref{sect:MethodResolution}. In
particular, while \xcd`operator`s perform implicit conversions on their
arguments, \xcd`==` and \xcd`!=` do not.

The advantage of this restriction is that \xcd`==`'s behavior is as simple and
efficient as possible.  It never runs user-defined code, and the compiler can
analyze and understand it in detail -- and guarantee that it is efficient.

The disadvantage is that certain straightforward-looking idioms do not work.
One may not, for example, write
\begin{xten}
// NOT ALLOWED
for(var i : Long = 0; i != 100; i++) 
\end{xten}
A \xcd`Long` like \xcd`i` can never \xcd`==` an \xcd`Int` like \xcd`100`.

We can write \xcd`i = i + 1;`, adding an \xcd`Int` to \xcd`i`. This works 
because the expression uses \xcd`+`,  an ordinary \xcd`operator`.
There is an implicit coercion from \xcd`Int` to \xcd`Long`, so the
\xcd`1` can be converted to \xcd`1L`, which can be added to \xcd`i`.  

However, \xcd`==` does not permit implicit coercions, and so the \xcd`100`
stays an \xcd`Int`.  The loop must be written with a comparison of two
\xcd`Long`s: 
\begin{xten}
for(var i : Long = 0; i != 100L; i++) 
\end{xten}

Incidentally, it could also be written 
\begin{xten}
for(var i : Long = 0; i <= 100; i++) 
\end{xten}
The operation \xcd`<=` is a regular operator, and thus uses coercions in its
arguments, so \xcd`100` gets coerced to \xcd`100L`.  

\subsection{Non-Disjointness Requirement}

It is a static error to have an expression \xcd`a==b` where \xcd`a` and
\xcd`b` could not possibly be equal, based on their types.  This is a
practical codicil to \Sref{sect:eqeq-no-coerce}.  Consider the illegal code 
\begin{xten}
// NOT ALLOWED
for(var i : Long = 0; i != 100; i++) 
\end{xten}

\xcd`100` and \xcd`100L` are different values; they are not \xcd`==`. A
coercion could make them equal, but \xcd`==` does not allow coercions. So, if
\xcd`100 == 100L` were going to return anything, it would have to return
\xcd`false`. This would have the unfortunate effect of making the \xcd`for`
loop diverge.

Since this and related idioms are so common, and since so many programmers are
used to languages which are less precise about their numeric types, X10 avoids
the mistake by declaring it a static error in most cases.  Specifically,
\xcd`a==b` is not allowed if, by inspection of the types, \xcd`a` and \xcd`b`
could not possibly be equal.


\begin{itemize}

\item Numbers of different base types cannot be compared for equality.  
\xcd`100==100L` is a static error.  To compare numbers, explicitly cast them
%~~exp~~`~~`~~ ~~ ^^^Expressions2g6f
to the same type: \xcd`100 as Long == 100L`.

\item Indeed, structs of different types cannot be equal, and so they cannot be
compared for equality.  

\item For objects, the story is different. Unconstrained object types can
      always be compared for equality. Given objects \xcd`a:Person` and
      \xcd`b:Theory`, \xcd`a==b` could be true if \xcd`a==null` and
      \xcd`b==null`. 

\item Constrained object types may or may not be comparable.  For example,  
      if \xcd`Person` and \xcd`Theory` are both direct subclasses of
      \xcd`Object`, and \xcd`a:Person{self!=null}` and \xcd`b:Theory`, then
      \xcd`a==b` is not allowed, since the two could not possibly be equal.

\item Explicit casts erase type information.  If you wanted
      to have a comparison \xcd`a==b` for \xcd`a:Person{self!=null}` and
      \xcd`b:Theory`, you could write it as \xcd`a as Object == b as Object`.
      It would, of course, return \xcd`false`, but it would not be a compiler
      error.\footnote{Code generators often find this trick too be useful.}
      A struct and an object may both be cast to \xcd`Any` and compared for
      equality, though they, too, will always be different.

\end{itemize}





\section{Allocation}
\label{ClassCreation}
\index{new}
\index{allocation}
\index{class!instantation}
\index{class!construction}
\index{struct!instantation}
\index{struct!construction}
\index{instantation}

%##(ClassInstCreationExp
\begin{bbgrammar}
%(FROM #(prod:ClassInstCreationExp)#)
ClassInstCreationExp \: \xcd"new" TypeName TypeArguments\opt \xcd"(" ArgumentList\opt \xcd")" ClassBody\opt & (\ref{prod:ClassInstCreationExp}) \\
                    \| \xcd"new" TypeName \xcd"[" Type \xcd"]" \xcd"[" ArgumentList\opt \xcd"]" \\
                    \| Primary \xcd"." \xcd"new" Id TypeArguments\opt \xcd"(" ArgumentList\opt \xcd")" ClassBody\opt \\
                    \| AmbiguousName \xcd"." \xcd"new" Id TypeArguments\opt \xcd"(" ArgumentList\opt \xcd")" ClassBody\opt \\
\end{bbgrammar}
%##)

An allocation expression creates a new instance of a class and
invokes a constructor of the class.
The expression designates the class name and passes
type and value arguments to the constructor.

The allocation expression may have an optional class body.
In this case, an anonymous subclass of the given class is
allocated.   An anonymous class allocation may also specify a
single super-interface rather than a superclass; the superclass
of the anonymous class is \xcd"x10.lang.Object".

If the class is anonymous---that is, if a class body is
provided---then the constructor is selected from the superclass.
The constructor to invoke is selected using the same rules as
for method invocation (\Sref{MethodInvocation}).

The type of an allocation expression
is the return type of the constructor invoked, with appropriate
substitutions  of actual arguments for formal parameters, as
specified in \Sref{MethodInvocationSubstitution}.

It is illegal to allocate an instance of an \xcd"abstract" class.
The usual visibility rules apply to allocations: 
it is illegal to allocate an instance of a class or to invoke a
constructor that is not visible at
the allocation expression.

Note that instantiating a struct type can use function application syntax; 
\xcd`new` is optional.  As structs do not have subclassing, there is no need or
possibility of a {\em ClassBody}.


\section{Casts}\label{ClassCast}\index{cast}
\index{type conversion}

The cast operation may be used to cast an expression to a given type:

%##(CastExp
\begin{bbgrammar}
%(FROM #(prod:CastExp)#)
             CastExp \: Primary & (\ref{prod:CastExp}) \\
                    \| ExpName \\
                    \| CastExp \xcd"as" Type \\
\end{bbgrammar}
%##)

The result of this operation is a value of the given type if the cast
is permissible at run time, and either a compile-time error or a runtime
exception 
(\xcd`x10.lang.TypeCastException`) if it is not.  

When evaluating \xcd`E as T{c}`, first the value of \xcd`E` is converted to
type \xcd`T` (which may fail), and then the constraint \xcd`{c}` is checked. 



\begin{itemize}
\item If \xcd`T` is a primitive type, then \xcd`E`'s value is converted to type
      \xcd`T` according to the rules of
      \Sref{sec:effects-of-explicit-numeric-coercions}. 
      
\item If \xcd`T` is a class, then the first half of the cast succeeds if the
      run-time value of \xcd`E` is an instance of class \xcd`T`, or of a
      subclass. 

\item If \xcd`T` is an interface, then the first half of the cast succeeds if
      the run-time value of \xcd`E` is an instance of a class implementing
      \xcd`T`. 

\item If \xcd`T` is a struct type, then the first half of the cast succeeds if
      the run-time value of \xcd`E` is an instance of \xcd`T`.  

\item If \xcd`T` is a function type, then the first half of the cast succeeds
      if the run-time value of \xcd`X` is a function of that type, or a
      subtype of it.
\end{itemize}

If the first half of the cast succeeds, the second half -- the constraint
\xcd`{c}` -- must be checked.  In general this will be done at runtime, though
in special cases it can be checked at compile time.   For example, 
\xcd`n as Int{self != w}` succeeds if \xcd`n != w` --- even if \xcd`w` is a value
read from input, and thus not determined at compile time.

The compiler may forbid casts that it knows cannot possibly work. If there is
no way for the value of \xcd`E` to be of type \xcd`T{c}`, then 
\xcd`E as T{c}` can result in a static error, rather than a runtime error.  
For example, \xcd`1 as Int{self==2}` may fail to compile, because the compiler
knows that \xcd`1`, which has type \xcd`Int{self==1}`, cannot possibly be of
type \xcd`Int{self==2}`. 


%BB% \bard{This section need serious whomping.  The Java mention needs to go.  The
%BB% rules for coercions are given in \Sref{sec:effects-of-explicit-numeric-coercions}.
%BB% If the \xcd`Type` has a constraint, the constraint will be checked at runtime. 
%BB% We need to give examples. 
%BB% }
%BB% 
%BB% Type conversion is checked according to the
%BB% rules of the \java{} language (e.g., \cite[\S 5.5]{jls2}).
%BB% For constrained types, both the base
%BB% type and the constraint are checked.
%BB% If the
%BB% value cannot be cast to the appropriate type, a
%BB% \xcd"ClassCastException"
%BB% is thrown. 



% {\bf Conversions of numeric values}
% {\bf Can't do (a as T) if a can't be a T.}


%If the value cannot be cast to the
%appropriate place type a \xcd"BadPlaceException" is thrown. 

% Any attempt to cast an expression of a reference type to a value type
% (or vice versa) results in a compile-time error. Some casts---such as
% those that seek to cast a value of a subtype to a supertype---are
% known to succeed at compile-time. Such casts should not cause extra
% computational overhead at run time.

\section{\Xcd{instanceof}}
\label{instanceOf}
\index{\Xcd{instanceof}}
\index{instanceof}

\Xten{} permits types to be used in an in instanceof expression
to determine whether an object is an instance of the given type:

%##(RelationalExp
\begin{bbgrammar}
%(FROM #(prod:RelationalExp)#)
       RelationalExp \: RangeExp & (\ref{prod:RelationalExp}) \\
                    \| SubtypeConstraint \\
                    \| RelationalExp \xcd"<" RangeExp \\
                    \| RelationalExp \xcd">" RangeExp \\
                    \| RelationalExp \xcd"<=" RangeExp \\
                    \| RelationalExp \xcd">=" RangeExp \\
                    \| RelationalExp \xcd"instanceof" Type \\
                    \| RelationalExp \xcd"in" ShiftExp \\
\end{bbgrammar}
%##)

In the above expression, \grammarrule{Type} is any type. At run time, the
result of \xcd`e instanceof T`
is \xcd"true" if the
value of \xcd`e` is an instance of type \xcd`T`.
Otherwise the result is \xcd"false". This determination may involve checking
that the constraint, if any, associated with the type is true for the given
expression.

%~~exp~~`~~`~~x:Int~~ ^^^ Expressions120
For example, \xcd`3 instanceof Int{self==x}` is an overly-complicated way of
saying \xcd`3==x`.


However, it is a static error if \xcd`e` cannot possibly be an instance of
\xcd`C{c}`; the compiler will reject \xcd`1 instanceof Int{self == 2}` because
\xcd`1` can never satisfy \xcd`Int{self == 2}`. Similarly, \Xcd{1 instanceof
String} is a static error, rather than an expression always returning false. 

\limitationx
X10 does not currently handle \xcd`instanceof` of generics in the way you
%~NO~exp~~`~~`~~r:Array[Int](1) ~~
might expect.  For example, \xcd`r instanceof Array[Int{self != 0}]` does
not test that every element of \xcd`r` is non-zero; instead, the compiler
rejects it.


\section{Subtyping expressions}
\index{\Xcd{<:}}
\index{\Xcd{:>}}
\index{subtype!test}


%##(SubtypeConstraint
\begin{bbgrammar}
%(FROM #(prod:SubtypeConstraint)#)
   SubtypeConstraint \: Type  \xcd"<:" Type  & (\ref{prod:SubtypeConstraint}) \\
                    \| Type  \xcd":>" Type  \\
\end{bbgrammar}
%##)

The subtyping expression \xcdmath"T$_1$ <: T$_2$" evaluates to \xcd"true" if
\xcdmath"T$_1$" is a subtype of \xcdmath"T$_2$".

The expression \xcdmath"T$_1$ :> T$_2$" evaluates to \xcd"true" if
\xcdmath"T$_2$" is a subtype of \xcdmath"T$_1$".

The expression \xcdmath"T$_1$ == T$_2$"
evaluates to  \xcd"true" if 
\xcdmath"T$_1$" is a subtype of \xcdmath"T$_2$" and
if \xcdmath"T$_2$" is a subtype of \xcdmath"T$_1$".

\begin{ex}
Subtyping expressions are particularly useful in giving constraints on generic
types.  \xcd`x10.util.Ordered[T]` is an interface whose values can be compared
with values of type \xcd`T`. 
In particular, \xcd`T <: x10.util.Ordered[T]` is
true if values of type \xcd`T` can be compared to other values of type
\xcd`T`.  So, if we wish to define a generic class \xcd`OrderedList[T]`, of
lists whose elements are kept in the right order, we need the elements to be
ordered.  This is phrased as a constraint on \xcd`T`: 
%~~gen ^^^ Expressions130
% package expre.ssi.onsfgua.rde.dq.uantification;
%~~vis
\begin{xten}
class OrderedList[T]{T <: x10.util.Ordered[T]} {
  // ...
}
\end{xten}
%~~siv
%
%~~neg
\end{ex}


\section{Contains expressions}
\index{in}

\begin{bbgrammar}
       RelationalExp \:RelationalExp \xcd"in" ShiftExp & (\ref{prod:RelationalExp}) \\
\end{bbgrammar}

\xcd`in` is a binary operator, definable in \Sref{sect:operators}.  It is
conventionally used for checking containment.

\begin{ex}
The built-in type \xcd`Region` provides \xcd`in`, testing whether a
\xcd`Point` is in the region: 
%~~gen ^^^ Expressions6d2z
% package Expressions6d2z;
% class Hook { def run() {
%~~vis
\begin{xten}
assert 3 in 1..10;
assert !(10 in 1..3);
\end{xten}
%~~siv
% return true;
%}}
%~~neg

Other types can provide them as well:
%~~gen ^^^ Expressions3c4m
% package Expressions3c4m;
%~~vis
\begin{xten}
class Cont {
   operator this in (Int) = true;
   operator (String) in this = false;
   static operator (Cont) in (b:Boolean) = b;
   static def example() {
      val c:Cont = new Cont();
      assert c in 4 && !("odd" in c) && (c in true);
   }
}
\end{xten}
%~~siv
%class Hook{ def run() { Cont.example(); return true; } }
%~~neg


\end{ex}

\section{Array Constructors}
\label{sect:ArrayCtors}
\index{array!construction}
\index{array!literal}


\begin{bbgrammar}
             Primary \: 
                    \xcd"[" ArgumentList\opt \xcd"]" 
\end{bbgrammar}
%##)

X10 includes short syntactic forms for constructing one-dimensional arrays.
The shortest form is to enclose some expressions in brackets: 
%~~gen ^^^ Expressions140
% package Expressions.ArrayCtor.Primo;
% class Example {
% def example() {
%~~vis
\begin{xten}
val ints <: Array[Int](1) = [1,3,7,21];
\end{xten}
%~~siv
%}}
%~~neg

The expression \xcdmath"[e$_1$, $\ldots$, e$_n$]" produces an \Xcd{n}-element
\xcd`Array[T](1)`, where \xcd`T` is the computed common supertype (\Sref{LCA}) of the {\bf
base types} of the expressions  \xcdmath"e$_i$". 

\begin{ex}
The type of
\xcd`[0,1,2]` is \Xcd{Array[Int](1)}.    
More importantly, the type of 
\xcd`[0]` is also \xcd`Array[Int](1)`.  It is {\em not} 
\xcd`Array[Int{self==0}](1)`, even though all the elements are all 
of type \xcd`Int{self==0}`.  This is subtle but important. There are many
functions that take \xcd`Array[Int](1)`s, such as conversions to \xcd`Point`.
These functions do {\em not} take
\xcd`Array[Int{self==0}]`'s.

(Suppose that the function took \xcd`a:Array[Int](1)` and did 
the operation \xcd`a(i)=100`.   This operation is perfectly fine for
an \xcd`Array[Int](1)`, which is all the compiler knows about \xcd`a`.  
However, it is invalid for an \xcd`Array[Int{self==0}](1)`, because it assigns
a non-zero value to an element of the array, violating the type constraint
which says that all the elements are zero.  So, \xcd`Array[Int{self==0}](1)`
is not and must not be a subtype of \xcd`Array[Int](1)` --- the two types are simply unrelated.
%~~type~~`~~`~~ ~~ ^^^ Expressions150
Since there are far more uses for \xcd`Array[Int](1)` than
%~~type~~`~~`~~ ~~ ^^^ Expressions160
\xcd`Array[Int{self==0}](1)`, the compiler produces the former.)
\end{ex}



\begin{ex}
Occasionally one does actually need \xcd`Array[Int{self==0}](1)`, 
or, say, \xcd`Array[Eel{self != null}](1)`, an array of non-null \xcd`Eel`s.  
For these cases, cast one of the elements of the array to the desired type,
and the array constructor will do the right thing.  
%~~gen ^^^ Expressions170
%package Expressions.ArrayCtor.Details;
%class Eel{}
%class Example{
%def example(){
%~~vis
\begin{xten}
val zero <: Array[Int{self == 0}](1) 
          = [0];
val non1 <: Array[Int{self != 1}](1) 
          = [0 as Int{self != 1}];
val eels <: Array[Eel{self != null}](1) 
          = [ new Eel() as Eel{self != null}, new Eel(), new Eel()];
\end{xten}
%~~siv
%}}
%~~neg
\end{ex}


\section{Coercions and conversions}
\label{XtenConversions}
\label{User-definedCoercions}
\index{conversion}\index{coercion}
\index{type!conversion}\index{type!coercion}

\XtenCurrVer{} supports the following coercions and conversions.

\subsection{Coercions}

%##(CastExp
\begin{bbgrammar}
%(FROM #(prod:CastExp)#)
             CastExp \: CastExp \xcd"as" Type \\
\end{bbgrammar}
%##)


A {\em coercion} does not change object identity; a coerced object may
be explicitly coerced back to its original type through a cast. A {\em
  conversion} may change object identity if the type being converted
to is not the same as the type converted from. \Xten{} permits
user-defined conversions (\Sref{sec:user-defined-conversions}).

\paragraph{Subsumption coercion.}
A value of a subtype may be implicitly coerced to any supertype.  
\index{coercion!subsumption}

\begin{ex}
If \xcd`Child <: Person` and \xcd`val rhys:Child`, then \xcd`rhys` may be used
in any context that expects a \xcd`Person`.  For example, 
%~~gen ^^^ Expressions7f1h
% package Expressions7f1h;
% class Person{}
% class Child extends Person {}
%~~vis
\begin{xten}
class Example {
  def greet(Person) = "Hi!";
  def example(rhys: Child) {
     greet(rhys);
  }
}
\end{xten}
%~~siv
%
%~~neg

Similarly, \xcd`2` (whose innate type is \xcd`Int{self==2}`)
is usable in a context requiring a non-zero integer
(\xcd`Int{self != 0}`).  
\end{ex}

\paragraph{Explicit Coercion (Casting with \xcd"as")}

All classes and interfaces allow the use of the \xcd`as` operator for explicit
type coercion.  
Any class or
interface may be cast to any interface.  
Any interface may be cast to
any class.  Also, any interface can be cast to a struct that implements
(directly or indirectly) that interface.

\begin{ex}
In the following code, a \xcd`Person` is cast to \xcd`Childlike`.  There is
nothing in the class definition of \xcd`Person` that suggests that a
\xcd`Person` can be \xcd`Childlike`.  However, the \xcd`Person` in question,
\xcd`p`, is actually a \xcd`HappyChild` --- a subclass of \xcd`Person` --- and
is, in fact, \xcd`Childlike`.  

Similarly, the \xcd`Childlike` value \xcd`cl` is cast to \xcd`Happy`.  Though
these two interfaces are unrelated, the value of \xcd`cl` is, in fact,
\xcd`Happy`.  And the \xcd`Happy` value \xcd`hc` is cast to the class
\xcd`Child`, though there is no relationship between the two, but the actual
value is a \xcd`HappyChild`, and thus the cast is correct at runtime.

\xcd`Cyborg` is a struct rather than a class.  So, it cannot have substructs,
and all the interfaces of all \xcd`Cyborg`s are known: a \xcd`Cyborg` is
\xcd`Personable`, but not \xcd`Childlike` or \xcd`Happy`.  So, it is correct
and meaningful to cast \xcd`r` to \xcd`Personable`.  There is no way that a
cast to \xcd`Childlike` could succeed, so \xcd`r as Childlike` is a static error.

%~~gen ^^^ Expressions180
% package Types.Coercions;
%~~vis
\begin{xten}
interface Personable {}
class Person implements Personable {}
interface Childlike extends Personable {}
class Child extends Person implements Childlike {}
struct Cyborg implements Personable {}
interface Happy {}
class HappyChild extends Child implements Happy {}
class Example {
  static def example() {
    var p : Person = new HappyChild();
    val cl : Childlike = p as Childlike; // class -> interface
    val hc : Happy = cl as Happy; //        interface -> interface
    val ch : Child = hc as Child; //        interface -> class
    var r : Cyborg = Cyborg();
    val rl : Personable = r as Personable; 
    // ERROR: r as Childlike
  }
}
\end{xten}
%~~siv
% class Hook{ def run(){ Example.example(); return true; } }
%~~neg




\end{ex}


If the value coerced is not an instance of the target type,
and no coercion operators that can convert it to that type are defined, 
a \xcd"ClassCastException" is thrown.  Casting to a constrained
type may require a run-time check that the constraint is
satisfied.
\index{coercion!explicit}
\index{cast}
\index{\Xcd{as}}

\limitation{It is currently a static error, rather than the specified
\xcd`ClassCastException`, when the cast is statically determinable to be
impossible.}



\paragraph{Effects of explicit numeric coercion}
\label{sec:effects-of-explicit-numeric-coercions}

Coercing a number of one type to another type gives the best approximation of
the number in the result type, or a suitable disaster value if no
approximation is good enough.  

\begin{itemize}
\item Casting a number to a {\em wider} numeric type is safe and effective,
      and can be done by an implicit conversion as well as an explicit
%~~exp~~`~~`~~ ~~ ^^^ Expressions190
      coercion.  For example, \xcd`4 as Long` produces the \xcd`Long` value of
      4. 
\item Casting a floating-point value to an integer value truncates the digits
      after the decimal point, thereby rounding the number towards zero.  
%~~exp~~`~~`~~ ^^^ Expressions200
      \xcd`54.321 as Int` is \xcd`54`, and 
%~~exp~~`~~`~~ ~~ ^^^ Expressions210
      \xcd`-54.321 as Int` is \xcd`-54`.
      If the floating-point value is too large to represent as that kind of
      integer, the coercion returns the largest or smallest value of that type
      instead: \xcd`1e110 as Int` is 
      \xcd`Int.MAX_VALUE`, \viz{} \xcd`2147483647`. 

\item Casting a \xcd`Double` to a \xcd`Float` normally truncates binary digits: 
%~~exp~~`~~`~~ ~~ ^^^ Expressions220
      \xcd`0.12345678901234567890 as Float` is approximately \xcd`0.12345679f`.  This can
      turn a nonzero \xcd`Double` into \xcd`0.0f`, the zero of type
      \xcd`Float`: 
%~~exp~~`~~`~~ ~~ ^^^ Expressions230
      \xcd`1e-100 as Float` is \xcd`0.0f`.  Since 
      \xcd`Double`s can be as large as about \xcd`1.79E308` and \xcd`Float`s
      can only be as large as about \xcd`3.4E38f`, a large \xcd`Double` will
      be converted to the special \xcd`Float` value of \xcd`Infinity`: 
%~~exp~~`~~`~~ ~~ ^^^ Expressions240
      \xcd`1e100 as Float` is \xcd`Infinity`.
\item Integers are coerced to smaller integer types by truncating the
      high-order bits. If the value of the large integer fits into the smaller
      integer's range, this gives the same number in the smaller type: 
%~~exp~~`~~`~~ ~~ ^^^ Expressions250
      \xcd`12 as Byte` is the \xcd`Byte`-sized 12, 
%~~exp~~`~~`~~ ~~ ^^^ Expressions260
      \xcd`-12 as Byte` is -12. 
      However, if the larger integer {\em doesn't} fit in the smaller type,
%~~exp~~`~~`~~ ~~ ^^^ Expressions270
      the numeric value and even the sign can change: \xcd`254 as Byte` is
      the \xcd`Byte`sized \xcd`-2y`.  

\item Casting an unsigned integer type to a signed integer type of the same
      size (\eg, \xcd`UInt` to \xcd`Int`) preserves 2's-complement bit pattern
      (\eg,  
      \xcd`UInt.MAX_VALUE as Int == -1`.   Casting an unsigned integer type to
      a signed integer type of a different size is equivalent to first casting
      to an unsigned integer type of the target size, and then casting to a
      signed integer type.

\item Casting a signed integer type to an unsigned one is similar.  

\end{itemize}

\subsubsection{User-defined Coercions}
\index{coercion!user-defined}

Users may define coercions from arbitrary types into the container type
\xcd`B`, and coercions from \xcd`B` to arbitrary types, by providing
\xcd`static operator` definitions for the \xcd`as` operator in the definition of
\xcd`B`.  

\begin{ex}

%~~gen ^^^ Expressions2j7z
% package Expressions2j7z;
% KNOWNFAIL
%~~vis
\begin{xten}
class Bee {
  public static operator (x:Bee) as Int = 1;
  public static operator (x:Int) as Bee = new Bee();
  def example() {
    val b:Bee = 2 as Bee; 
    assert (b as Int) == 1;
  }
}
\end{xten}
%~~siv
%
%~~neg


\end{ex}



\subsection{Conversions}
\index{conversion}
\index{type!conversion}

\paragraph{Widening numeric conversion.}
\label{WideningConversions}
A numeric type may be implicitly converted to a wider numeric type. In
particular, an implicit conversion may be performed between a numeric
type and a type to its right, below:

\begin{xten}
Byte < Short < Int < Long < Float < Double
UByte < UShort < UInt < ULong
\end{xten}

Furthermore, an unsigned integer type may be implicitly coerced a signed type
large 
enough to hold any value of the type: \xcd`UByte` to \xcd`Short`, \xcd`UShort`
to \xcd`Int`, \xcd`UInt` to \xcd`Long`.  There are no implicit conversions
from signed to unsigned numbers, since they cannot treat negatives properly.

There are no implicit conversions in cases when overflow is possible.  For
example, there is no implicit conversion between \xcd`Int` and \xcd`UInt`.  If
it is necessary to convert between these types, use \xcd`n as Int` or 
\xcd`n as UInt`, generally with a test to ensure that the value will fit and
code to handle the case in which it does not.  


\index{conversion!widening}
\index{conversion!numeric}

\paragraph{String conversion.}
Any value that is an operand of the binary
\xcd"+" operator may
be converted to \xcd"String" if the other operand is a \xcd"String".
A conversion to \xcd"String" is performed by invoking the \xcd"toString()"
method.

\index{conversion!string}

\paragraph{User defined conversions.}\label{sec:user-defined-conversions}
\index{conversion!user-defined}

The user may define implicit conversion operators from type \Xcd{A} {\em to} a
container type \Xcd{B} by specifying an operator in \Xcd{B}'s definition of the form:

\begin{xten}
  public static operator (r: A): T = ... 
\end{xten}

The return type \Xcd{T} should be a subtype of \Xcd{B}. The return
type need not be specified explicitly; it will be computed in the
usual fashion if it is not. However, it is good practice for the
programmer to specify the return type for such operators explicitly.
The return type can be more specific than simply \xcd`B`, for cases when there
is more information available.


\begin{ex}
The code for \Xcd{x10.lang.Point} contains a conversion from 
one-dimensional \xcd`Array`s of integers to \xcd`Point`s of the same length: 
\begin{xten}
  public operator (r: Array[Int](1)): Point(r.length) = make(r);
\end{xten}
This conversion is used whenever an array of integers appears in a 
context that requires a \xcd`Point`, such as subscripting. Note 
that \xcd`a` requires a \xcd`Point` of rank 2 as a subscript, and that 
a two-element \xcd`Array` (like \xcd`[2,4]`) is converted to a 
\xcd`Point(2)`.
%~~gen ^^^ Expressions4f4y
% package Expressions4f4y;
% class Example { def example() {
%~~vis
\begin{xten}
val a = new Array[String]((2..3) * (4..5), "hi!");
a([2,4]) = "converted!";
\end{xten}
%~~siv
%} } 
%~~neg


\end{ex}

\section{Parenthesized Expressions}

If \xcd`E` is any expression, \xcd`(E)` is an expression which, when
evaluated, produces the same result as \xcd`E`.   

\begin{ex}
The main use of parentheses is to write complex expressions for which the 
standard precedence order of operations is not appropriate: \xcd`1+2*3` is 7,
but \xcd`(1+2)*3` is 9.  

Similarly, but perhaps less familiarly, 
parentheses can disambiguate other expressions.  In the following code, 
\xcd`funny.f` is a field-selection expression, and so \xcd`(funny.f)()` means
``select the \xcd`f` field from \xcd`funny`, and evaluate it''.  However, 
\xcd`funny.f()` means ``evaluate the \xcd`f` method on object \xcd`funny`.''  
%~~gen ^^^ Expressions3f6f
% package Expressions3f6f;
%~~vis
\begin{xten}
class Funny {
  def f () = 1;
  val f = () => 2;
  static def example() {
    val funny = new Funny();
    assert funny.f() == 1;
    assert (funny.f)() == 2;
  }
}
\end{xten}
%~~siv
% class Hook{ def run() { Funny.example(); return true; }}
%~~neg


\end{ex}

Note that this does {\em
not} mean that \xcd`E` and \xcd`(E)` are identical in all respects; for
example, if \xcd`i` is an \xcd`Int` variable, \xcd`i++` increments \xcd`i`,
but \xcd`(i)++` is not allowed.    \xcd`++` is an assignment; it operates on
variables, not merely values, and \xcd`(i)` is simply an expression whose {\em
value} is the same as that of \xcd`i`. 
	
\chapter{Statements}\label{XtenStatements}\index{statement}

This chapter describes the statements in the sequential core of
\Xten{}.  Statements involving concurrency and distribution
are described in \Sref{XtenActivities}.

\section{Empty statement}

The empty statement \xcd";" does nothing.  

\begin{ex}
Sometimes, the syntax of X10 requires a statement in some position, but you do
not actually want to do any computation there.   
The following code searches the array \xcd`a` for the value \xcd`v`, assumed
to appear somewhere in \xcd`a`, and returns the index at which it was found.  
There is no computation to do in the loop body, so we use an empty statement
there. 
%~~gen ^^^ Statements10
% package statements.emptystatement;
% class EmptyStatementExample {
%~~vis
\begin{xten}
static def search[T](a: Array[T](1), v: T):Int {
  var i : Int;
  for(i = a.region.min(0); a(i) != v; i++)
     ;
  return i;
}
\end{xten}
%~~siv
%}
%~~neg

\end{ex}

\section{Local variable declaration}
\label{sect:LocalVarDecl}
\index{variable!declaration}
\index{var}
\index{val}

%##(LocVarDecl LocVarDeclStmt VarDeclWType VarDeclsWType VariableDeclarators VariableInitializer FormalDeclarators
\begin{bbgrammar}
%(FROM #(prod:LocVarDecl)#)
          LocVarDecl \: Mods\opt VarKeyword VariableDeclarators & (\ref{prod:LocVarDecl}) \\
                     \| Mods\opt VarDeclsWType \\
                     \| Mods\opt VarKeyword FormalDeclarators \\
%(FROM #(prod:LocVarDeclStmt)#)
      LocVarDeclStmt \: LocVarDecl \xcd";" & (\ref{prod:LocVarDeclStmt}) \\
%(FROM #(prod:VarDeclWType)#)
        VarDeclWType \: Id HasResultType \xcd"=" VariableInitializer & (\ref{prod:VarDeclWType}) \\
                     \| \xcd"[" IdList \xcd"]" HasResultType \xcd"=" VariableInitializer \\
                     \| Id \xcd"[" IdList \xcd"]" HasResultType \xcd"=" VariableInitializer \\
%(FROM #(prod:VarDeclsWType)#)
       VarDeclsWType \: VarDeclWType & (\ref{prod:VarDeclsWType}) \\
                     \| VarDeclsWType \xcd"," VarDeclWType \\
%(FROM #(prod:VariableDeclarators)#)
 VariableDeclarators \: VariableDeclarator & (\ref{prod:VariableDeclarators}) \\
                     \| VariableDeclarators \xcd"," VariableDeclarator \\
%(FROM #(prod:VariableInitializer)#)
 VariableInitializer \: Exp & (\ref{prod:VariableInitializer}) \\
%(FROM #(prod:FormalDeclarators)#)
   FormalDeclarators \: FormalDeclarator & (\ref{prod:FormalDeclarators}) \\
                     \| FormalDeclarators \xcd"," FormalDeclarator \\
\end{bbgrammar}
%##)

Short-lived variables are introduced by local variables declarations, as
described in \Sref{VariableDeclarations}. Local variables may be declared only
within a block statement (\Sref{Blocks}). The scope of a local variable
declaration is the subsequent statements in the
block.   
%~~gen ^^^ Statements20
% package statements.should.have.locals;
% class LocalExample {
% def example(a:Int) {
%~~vis
\begin{xten}
  if (a > 1) {
     val b = a/2;
     var c : Int = 0;
     // b and c are defined here
  }
  // b and c are not defined here.
\end{xten}
%~~siv
%} }
%~~neg

Variables declared in such statements shadow variables of the same
name declared elsewhere.
A local variable of a given name, say \xcd`x`, cannot shadow another local
variable or parameter named \xcd`x` unless there is an intervening method,
constructor, initializer, or
closure declaration.
%%, or unless the inner \xcd`x` is declared inside an
%%\xcd`async` or \xcd`at` statement and the outer variable is declared outside
%% of that.   Strictly, \xcd`at` introduces a new scope that does not share the
%%variables of the external scope, so its variables do not actually shadow those
%%outside.  

\begin{ex}
The following code illustrates both legal and illegal uses of shadowing.
Note that a shadowed {\em field} name \xcd`x` can still be accessed 
as \xcd`this.x`. 

%%AT-COPY%% %~~gen ^^^ Statements4h6p
%%AT-COPY%% % package Statements4h6p;
%%AT-COPY%% %~~vis
%%AT-COPY%% \begin{xten}
%%AT-COPY%% class Shadow{
%%AT-COPY%%   var x : Int; 
%%AT-COPY%%   def this(x:Int) { 
%%AT-COPY%%      // Parameter can shadow field
%%AT-COPY%%      this.x = x; 
%%AT-COPY%%   }
%%AT-COPY%%   def example(y:Int) {
%%AT-COPY%%      val x = "shadows a field";
%%AT-COPY%%      // ERROR: val y = "shadows a param";
%%AT-COPY%%      val z = "local";
%%AT-COPY%%      for (a in [1,2,3]) {
%%AT-COPY%%         // ERROR: val x = "can't shadow local var";
%%AT-COPY%%      }
%%AT-COPY%%      async {
%%AT-COPY%%         val x = "can shadow through async";
%%AT-COPY%%      }        
%%AT-COPY%%      at (here;) {
%%AT-COPY%%         val x = "at gives a whole new namespace";
%%AT-COPY%%      }        
%%AT-COPY%%      val f = () => { 
%%AT-COPY%%        val x = "can shadow through closure";
%%AT-COPY%%        x
%%AT-COPY%%      };
%%AT-COPY%%   }
%%AT-COPY%% }
%%AT-COPY%% \end{xten}
%%AT-COPY%% %~~siv
%%AT-COPY%% %
%%AT-COPY%% %~~neg
%%AT-COPY%% 
%~~gen ^^^ Statements4h6p
% package Statements4h6p;
% // NOTEST-stupid-packaging-issue
%~~vis
\begin{xten}
class Shadow{
  var x : Int; 
  def this(x:Int) { 
     // Parameter can shadow field
     this.x = x; 
  }
  def example(y:Int) {
     val x = "shadows a field";
     // ERROR: val y = "shadows a param";
     val z = "local";
     for (a in [1,2,3]) {
        // ERROR: val x = "can't shadow local var";
     }
     async {
        // ERROR: val x = "can't shadow through async";
     }        
     val f = () => { 
       val x = "can shadow through closure";
       x
     };
     class Local {
        val f = at(here.next()){ val x = "can here"; x };
        def this() { val x = "can here, too"; }
     }
  }
}
\end{xten}
%~~siv
%
%~~neg



\end{ex}

\begin{ex}
Note that recursive definitions of local variables is not allowed.  There are
few useful recursive declarations of objects and structs; \xcd`x`, in the
following example, has no meaningful definition.  Recursive declarations of
local functions is forbidden, even though (like \xcd`f` below) there are
meaningful uses of it.  
\begin{xten}
val x : Int = x + 1; // ERROR: recursive local declaration
val f : (Int)=>Int 
      = (n:Int) => (n <= 2) ? 1 : f(n-1) + f(n-2);
      // ERROR: recursive local declaration
\end{xten}

\end{ex}



\section{Block statement}
\index{block}
\label{Blocks}

%##(Block BlockStatements BlockStatement
\begin{bbgrammar}
%(FROM #(prod:Block)#)
               Block \: \xcd"{" BlockStatements\opt \xcd"}" & (\ref{prod:Block}) \\
%(FROM #(prod:BlockStatements)#)
     BlockStatements \: BlockStatement & (\ref{prod:BlockStatements}) \\
                     \| BlockStatements BlockStatement \\
%(FROM #(prod:BlockStatement)#)
      BlockStatement \: LocVarDeclStmt & (\ref{prod:BlockStatement}) \\
                     \| ClassDecl \\
                     \| TypeDefDecl \\
                     \| Statement \\
\end{bbgrammar}
%##)


A block statement consists of a sequence of statements delimited by
``\xcd"{"'' and ``\xcd"}"''. When a block is evaluated, the statements inside
of it are evaluated in order.  Blocks are useful for putting several
statements in a place where X10 asks for a single one, such as the consequent
of an \xcd`if`, and for limiting the scope of local variables.
%~~gen ^^^ Statements30
% package statements.FOR.block.heads;
% class Example {
% def example(b:Boolean, S1:(Int)=>void, S2:(Int)=>void ) {
%~~vis
\begin{xten}
if (b) {
  // This is a block
  val v = 1;
  S1(v); 
  S2(v);
}
\end{xten}
%~~siv
%  } } 
%~~neg



\section{Expression statement}

Any expression may be used as a statement.

%##(ExpStatement StatementExp
\begin{bbgrammar}
%(FROM #(prod:ExpStatement)#)
        ExpStatement \: StatementExp \xcd";" & (\ref{prod:ExpStatement}) \\
%(FROM #(prod:StatementExp)#)
        StatementExp \: Assignment & (\ref{prod:StatementExp}) \\
                     \| PreIncrementExp \\
                     \| PreDecrementExp \\
                     \| PostIncrementExp \\
                     \| PostDecrementExp \\
                     \| MethodInvocation \\
                     \| ClassInstCreationExp \\
\end{bbgrammar}
%##)

The expression statement evaluates an expression. The value of the expression
is not used. Side effects of the expression occur, and may produce results
used by following statements. Indeed, statement expressions which terminate
without side effects cannot have any visible effect on the results of the
computation. 


\begin{ex}
%~~gen ^^^ Statements40
% package Sta.tem.ent.s.expressions;
% import x10.util.*;
%~~vis
\begin{xten}
class StmtEx {
  def this() { 
     x10.io.Console.OUT.println("New StmtEx made");  }
  static def call() { 
     x10.io.Console.OUT.println("call!");}
  def example() {
     var a : Int = 0;
     a = 1; // assignment
     new StmtEx(); // allocation
     call(); // call
  }
}
\end{xten}
%~~siv
%
%~~neg
\end{ex}



\section{Labeled statement}
\index{label}
\index{statement label}


\begin{bbgrammar}
    LabeledStatement \: Id \xcd":" Statement 
\end{bbgrammar}


Statements may be labeled. The label may be used to describe the target of a
\xcd`break` statement appearing within a substatement (which, when executed,
ends the labeled statement), or, in the case of a loop, a \xcd`continue` as
well (which, when executed, proceeds to the next iteration of the loop). The
scope of a label is the statement labeled.

\begin{ex}
The label on the outer \xcd`for` statement allows \xcd`continue` and
\xcd`break` statements to continue or break it.  Without the label,
\xcd`continue` or \xcd`break` would only continue or break the inner \xcd`for`
loop. 
%~~gen ^^^ Statements50
% package state.meant.labe.L;
% class Example {
% def example(a:(Int,Int) => Int, do_things_to:(Int)=>Int) {
%~~vis
\begin{xten}
lbl : for (i in 1..10) {
   for (j in i..10) {  
      if (a(i,j) == 0) break lbl;
      if (a(i,j) == 1) continue lbl;
      if (a(i,j) == a(j,i)) break lbl;
   }
}
\end{xten}
%~~siv
%} } 
%~~neg
\end{ex}

In particular, a block statement may be labeled: \xcd` L:{S}`.  This allows
the use of \xcd`break L` within \xcd`S` to leave \xcd`S`, which can, if
carefully used, avoid deeply-nested \xcd`if`s. 

\begin{ex}
%~~gen ^^^ Statements51
% package statements51;
% abstract class Example {
% abstract def phase1(String):void;
% abstract def phase2(String):void;
% abstract def phase3(String):void;
% abstract def suitable_for_phase_2(String):Boolean;
% abstract def suitable_for_phase_3(String):Boolean;
% def example(filename: String) {
% KNOWNFAIL-labelled-blocks
%~~vis
\begin{xten}
multiphase: {
  if (!exists(filename)) break multiphase;
  phase1(filename);
  if (!suitable_for_phase_2(filename)) break multiphase;
  phase2(filename);
  if (!suitable_for_phase_3(filename)) break multiphase;
  phase3(filename);
}
// Now the file has been phased as much as possible
\end{xten}
%~~siv
%}
%~~neg
\end{ex}

\limitation{Blocks cannot currently be labeled.}

\section{Break statement}
\index{break}

%##(BreakStatement
\begin{bbgrammar}
%(FROM #(prod:BreakStatement)#)
      BreakStatement \: \xcd"break" Id\opt \xcd";" & (\ref{prod:BreakStatement}) \\
\end{bbgrammar}
%##)


An unlabeled break statement exits the currently enclosing loop or switch
statement. A labeled break statement exits the enclosing 
statement with the given label.
It is illegal to break out of a statement not defined in the current
method, constructor, initializer, or closure.  
\xcd`break` is only allowed in sequential code.

\begin{ex}
The following code searches for an element of a two-dimensional
array and breaks out of the loop when it is found:
%~~gen ^^^ Statements60
% package statements.come.from.banks.and.cranks;
% class LabelledBreakeyBreakyHeart {
% def findy(a:Array[Array[Int](1)](1), v:Int): Boolean {
%~~vis
\begin{xten}
var found: Boolean = false;
outer: for (var i: Int = 0; i < a.size; i++)
    for (var j: Int = 0; j < a(i).size; j++)
        if (a(i)(j) == v) {
            found = true;
            break outer;
        }
\end{xten}
%~~siv
% return found;
%}}
%~~neg
\end{ex}

\section{Continue statement}
\index{continue}

%##(ContinueStatement
\begin{bbgrammar}
%(FROM #(prod:ContinueStatement)#)
   ContinueStatement \: \xcd"continue" Id\opt \xcd";" & (\ref{prod:ContinueStatement}) \\
\end{bbgrammar}
%##)
An unlabeled \xcd`continue` skips the rest of the current iteration of the
innermost enclosing loop, and proceeds on to the next.  A labeled
\xcd`continue` does the same to the enclosing loop with that label.
It is illegal to continue a loop not defined in the current
method, constructor, initializer, or closure.
\xcd`continue` is only allowed in sequential code.



\section{If statement}
\index{if}

%##(IfThenStatement IfThenElseStatement
\begin{bbgrammar}
%(FROM #(prod:IfThenStatement)#)
     IfThenStatement \: \xcd"if" \xcd"(" Exp \xcd")" Statement & (\ref{prod:IfThenStatement}) \\
%(FROM #(prod:IfThenElseStatement)#)
 IfThenElseStatement \: \xcd"if" \xcd"(" Exp \xcd")" Statement  \xcd"else" Statement  & (\ref{prod:IfThenElseStatement}) \\
\end{bbgrammar}
%##)

An if statement comes in two forms: with and without an else
clause.

The if-then statement evaluates a condition expression, which must be of type
\xcd`Boolean`. If the condition is \xcd`true`, it evaluates the then-clause.
If the condition is \xcd"false", the if-then statement completes normally.

The if-then-else statement evaluates a \xcd`Boolean` expression and 
evaluates the then-clause if the condition is
\xcd"true"; otherwise, the \xcd`else`-clause is evaluated.

As is traditional in languages derived from Algol, the if-statement is syntactically
ambiguous.  That is, 
\begin{xten}
if (B1) if (B2) S1 else S2
\end{xten}
could be intended to mean either 
\begin{xten}
if (B1) { if (B2) S1 else S2 }
\end{xten} 
or 
\begin{xten}
if (B1) {if (B2) S1} else S2
\end{xten}
X10, as is traditional, attaches an \xcd`else` clause to the most recent
\xcd`if` that doesn't have one.
This example is interpreted as 
\xcd`if (B1) { if (B2) S1 else S2 }`. 



\section{Switch statement}
\index{switch}

%##(SwitchStatement SwitchBlock SwitchBlockGroups SwitchBlockGroup SwitchLabels SwitchLabel
\begin{bbgrammar}
%(FROM #(prod:SwitchStatement)#)
     SwitchStatement \: \xcd"switch" \xcd"(" Exp \xcd")" SwitchBlock & (\ref{prod:SwitchStatement}) \\
%(FROM #(prod:SwitchBlock)#)
         SwitchBlock \: \xcd"{" SwitchBlockGroups\opt SwitchLabels\opt \xcd"}" & (\ref{prod:SwitchBlock}) \\
%(FROM #(prod:SwitchBlockGroups)#)
   SwitchBlockGroups \: SwitchBlockGroup & (\ref{prod:SwitchBlockGroups}) \\
                     \| SwitchBlockGroups SwitchBlockGroup \\
%(FROM #(prod:SwitchBlockGroup)#)
    SwitchBlockGroup \: SwitchLabels BlockStatements & (\ref{prod:SwitchBlockGroup}) \\
%(FROM #(prod:SwitchLabels)#)
        SwitchLabels \: SwitchLabel & (\ref{prod:SwitchLabels}) \\
                     \| SwitchLabels SwitchLabel \\
%(FROM #(prod:SwitchLabel)#)
         SwitchLabel \: \xcd"case" ConstantExp \xcd":" & (\ref{prod:SwitchLabel}) \\
                     \| \xcd"default" \xcd":" \\
\end{bbgrammar}
%##)

A switch statement evaluates an index expression and then branches to
a case whose value is equal to the value of the index expression.
If no such case exists, the switch branches to the 
\xcd"default" case, if any.

Statements in each case branch are evaluated in sequence.  At the
end of the branch, normal control-flow falls through to the next case, if
any.  To prevent fall-through, a case branch may be exited using
a \xcd"break" statement.

The index expression must be of type \xcd"Int".
Case labels must be of type \xcd"Int", \xcd`Byte`, or \xcd`Short`, 
and must be compile-time 
constants.  Case labels cannot be duplicated within the
\xcd"switch" statement.

\begin{ex}
In this \xcd`switch`, case \xcd`1` falls through to case \xcd`2`.  The
other cases are separated by \xcd`break`s.
%~~gen ^^^ Statements70
% package Statement.Case;
% class Example {
% def example(i : Int, println: (String)=>void) {
%~~vis
\begin{xten}
switch (i) {
  case 1: println("one, and ");
  case 2: println("two"); 
          break;
  case 3: println("three");
          break;
  default: println("Something else");
           break;
}
\end{xten}
%~~siv
% } } 
%~~neg
\end{ex}

\section{While statement}
\index{while}

%##(WhileStatement
\begin{bbgrammar}
%(FROM #(prod:WhileStatement)#)
      WhileStatement \: \xcd"while" \xcd"(" Exp \xcd")" Statement & (\ref{prod:WhileStatement}) \\
\end{bbgrammar}
%##)

A while statement evaluates a \xcd`Boolean`-valued condition and executes a
loop body if \xcd"true". If the loop body completes normally (either by
reaching the end or via a \xcd"continue" statement with the loop header as
target), the condition is reevaluated and the loop repeats if \xcd"true". If
the condition is \xcd"false", the loop exits.

\begin{ex}
A loop to execute the process in the Collatz conjecture (a.k.a. 3n+1 problem,
Ulam conjecture, Kakutani's problem, Thwaites conjecture, Hasse's algorithm,
and Syracuse problem) can be written as follows:
%~~gen ^^^ Statements80
% package Statements.AreFor.Flatements;
% class Example {
% def example(var n:Int) {
%~~vis
\begin{xten}
  while (n > 1) {
     n = (n % 2 == 1) ? 3*n+1 : n/2;
  }
\end{xten}
%~~siv
% } } 
%~~neg
\end{ex}
\section{Do--while statement}
\index{do}

%##(DoStatement
\begin{bbgrammar}
%(FROM #(prod:DoStatement)#)
         DoStatement \: \xcd"do" Statement \xcd"while" \xcd"(" Exp \xcd")" \xcd";" & (\ref{prod:DoStatement}) \\
\end{bbgrammar}
%##)


A \Xcd{do-while} statement executes the loop body, and then evaluates a
\xcd`Boolean`-valued condition expression. If \xcd"true", the loop repeats.
Otherwise, the loop exits.


\section{For statement}
\index{for}

%##(ForStatement BasicForStatement ForInit ForUpdate StatementExpList EnhancedForStatement
\begin{bbgrammar}
%(FROM #(prod:ForStatement)#)
        ForStatement \: BasicForStatement & (\ref{prod:ForStatement}) \\
                     \| EnhancedForStatement \\
%(FROM #(prod:BasicForStatement)#)
   BasicForStatement \: \xcd"for" \xcd"(" ForInit\opt \xcd";" Exp\opt \xcd";" ForUpdate\opt \xcd")" Statement & (\ref{prod:BasicForStatement}) \\
%(FROM #(prod:ForInit)#)
             ForInit \: StatementExpList & (\ref{prod:ForInit}) \\
                     \| LocVarDecl \\
%(FROM #(prod:ForUpdate)#)
           ForUpdate \: StatementExpList & (\ref{prod:ForUpdate}) \\
%(FROM #(prod:StatementExpList)#)
    StatementExpList \: StatementExp & (\ref{prod:StatementExpList}) \\
                     \| StatementExpList \xcd"," StatementExp \\
%(FROM #(prod:EnhancedForStatement)#)
EnhancedForStatement \: \xcd"for" \xcd"(" LoopIndex \xcd"in" Exp \xcd")" Statement & (\ref{prod:EnhancedForStatement}) \\
                     \| \xcd"for" \xcd"(" Exp \xcd")" Statement \\
\end{bbgrammar}
%##)

\xcd`for` statements provide bounded iteration, such as looping over a list.
It has two forms: a basic form allowing near-arbitrary iteration, {\em a la}
C, and an enhanced form designed to iterate over a collection.

A basic \xcd`for` statement provides for arbitrary iteration in a somewhat
more organized fashion than a \xcd`while`.  The loop 
\xcd`for(init; test; step)body` is
similar to: 
\begin{xten}
{
   init;
   while(test) {
      body;
      step;
   }
}
\end{xten}
\noindent
except that \xcd`continue` statements which continue the \xcd`for` loop will
perform the \xcd`step`, which, in the \xcd`while` loop, they will not do. 

\xcd`init` is performed before the loop, and is traditionally used to declare
and/or initialize the loop variables. It may be a single variable binding
statement, such as \xcd`var i:Int = 0` or \xcd`var i:Int=0, j:Int=100`. (Note
that a single variable binding statement may bind multiple variables.)
Variables introduced by \xcd`init` may appear anywhere in the \xcd`for`
statement, but not outside of it.  Or, it may be a sequence of expression
statements, such as \xcd`i=0, j=100`, operating on already-defined variables.
If omitted, \xcd`init` does nothing.

\xcd`test` is a Boolean-valued expression; an iteration of the loop will only
proceed if \xcd`test` is true at the beginning of the loop, after \xcd`init`
on the first iteration or after \xcd`step` on later ones. If omitted, \xcd`test`
defaults to \xcd`true`, giving a loop that will run until stopped by some
other means such as \xcd`break`, \xcd`return`, or \xcd`throw`.

\xcd`step` is performed after the loop body, between one iteration and the
next. It traditionally updates the loop variables from one iteration to the
next: \eg, \xcd`i++` and \xcd`i++,j--`.  If omitted, \xcd`step` does nothing.

\xcd`body` is a statement, often a code block, which is performed whenever
\xcd`test` is true.  If omitted, \xcd`body` does nothing.




\label{ForAllLoop}


An enhanced for statement is used to iterate over a collection, or other
structure designed to support iteration by implementing the interface
\xcd`Iterable[T]`.    The loop variable must be of type \xcd`T`, 
or destructurable from a value of type \xcd`T`
(\Sref{exploded-syntax}).  
Each iteration of the loop
binds the iteration variable to another element of the collection.
The loop \xcd`for(x in c)S` behaves like: 
%~~gen ^^^ Statements5e4u
% package Statements5e4u;
% class ForAll {
% def forall[T](c:Iterable[T], S: () => void) {
%~~vis
\begin{xten}
val iterator: Iterator[T] = c.iterator();
while (iterator.hasNext()) {
  val x : T = iterator.next();
  S();
}
\end{xten}
%~~siv
%} }
%~~neg

A number of library classes implement \xcd`Iterable`, and thus can be iterated
over.  For example, iterating over a \xcd`Region` iterates the \xcd`Point`s in
the region, and iterating over an \xcd`Array` iterates over the
\xcd`Point`s at which the  array is defined.

The type of the loop variable may be supplied as \xcd`x <: T`.  In this case
the iterable \xcd`c` must have type \xcd`Iterable[U}` for some \xcd`U <: T`,
and \xcd`x` will be given the type \xcd`U`.

\begin{ex}
This loop adds up the elements of a \xcd`List[Int]`.
Note that iterating over a list yields the elements of the list, as specified
in the \xcd`List` API. 
%~~gen ^^^ Statements3d9l
% package Statements3d9l;
% class Example {
%~~vis
\begin{xten}
static def sum(a:x10.util.List[Int]):Int {
  var s : Int = 0;
  for(x in a) s += x;
  return s;
}
\end{xten}
%~~siv
%}
%~~neg

The following code sums the elements of an integer array.  Note that the
\xcd`for` loop iterates over the indices of the array, not the elements, as
specified in the \xcd`Array` API.  
%~~gen ^^^ Statements2d4h
% package Statements2d4h;
% class Example { 
%~~vis
\begin{xten}
static def sum(a: Array[Int]): Int {
  var s : Int = 0;
  for(p in a) s += a(p);
  return s;
}
\end{xten}
%~~siv
%}
%~~neg

Iteration over an \xcd`IntRange` (\Sref{sect:intrange}) is quite common. This
allows looping while varying an integer index: 
%~~gen ^^^ Statements3o9s
% package Statements3o9s;
% class Example { static def example() {
%~~vis
\begin{xten}
var sum : Int = 0;
for(i in 1..10) sum += i;
assert sum == 55;
\end{xten}
%~~siv
%} } 
% class Hook { def run() { Example.example(); return true; } }
%~~neg


\end{ex}

Iteration variables have the \xcd`for` statement as scope.  They shadow other
variables of the same names.


\section{Return statement}
\label{ReturnStatement}
\index{return}

%##(ReturnStatement
\begin{bbgrammar}
%(FROM #(prod:ReturnStatement)#)
     ReturnStatement \: \xcd"return" Exp\opt \xcd";" & (\ref{prod:ReturnStatement}) \\
\end{bbgrammar}
%##)

Methods and closures may return values using a return statement.
If the method's return type is explicitely declared \xcd"void",
the method must return without a value; otherwise, it must return
a value of the appropriate type.

\begin{ex}
The following code illustrates returning values from a closure and a method.
The \xcd`return` inside of \xcd`closure` returns from \xcd`closure`, not from
\xcd`method`.  
%~~gen ^^^ Statements2j1d
% package Statements2j1d;
% class Example {
%~~vis
\begin{xten}
def method(x:Int) {
  val closure = (y:Int) => {return x+y;}; 
  val res = closure(0);
  assert res == x;
  return res == x;
}
\end{xten}
%~~siv
%}
%~~neg


\end{ex}


\section{Assert statement} 
\index{assert}

%##(AssertStatement
\begin{bbgrammar}
%(FROM #(prod:AssertStatement)#)
     AssertStatement \: \xcd"assert" Exp \xcd";" & (\ref{prod:AssertStatement}) \\
                     \| \xcd"assert" Exp  \xcd":" Exp  \xcd";" \\
\end{bbgrammar}
%##)

The statement \xcd`assert E` checks that the Boolean expression \xcd`E`
evaluates to true, and, if not, throws an \xcd`x10.lang.Error`  exception.  
The annotated assertion statement \xcd`assert E : F;` checks \xcd`E`, and, if
it is 
false, throws an \xcd`x10.lang.Error` exception with \xcd`F`'s value attached
to it. 

\begin{ex}
The following code compiles properly.  
%~~gen ^^^ Statements100
% package Statements.Assert;
% 
%~~vis
\begin{xten}
class Example {
  public static def main(argv:Array[String](1)) {
    val a = 1;
    assert a != 1 : "Changed my mind about a.";
  }
}
\end{xten}
%~~siv
%~~neg
\noindent
However, when run, it 
prints a stack trace starting with 
\begin{xten}
x10.lang.Error: Changed my mind about a.
\end{xten}
\end{ex}

\section{Exceptions in X10}
\index{exception}
\index{termination!abrupt}
\index{termination!normal}

X10 programs can throw {\em Exceptions} to indicate unusual or problematic
situations; this is {\em abrupt termination}.  Exceptions, as data values, are
objects which which inherit from 
\xcd`x10.lang.Throwable`.    Exceptions may be thrown intentionally with the
\xcd`throw` statement. Many primitives and library functions throw exceptions
if they encounter problems; \eg, dividing by zero throws an instance of
\xcd`x10.lang.ArithmeticException`. 

When an exception is thrown, statically and dynamically enclosing
\xcd`try`-\xcd`catch` blocks in the same activity can attempt to handle it.   If the throwing
statement in inside some \xcd`try` clause, and some matching \xcd`catch`
clause catches that type of exception, the corresponding \xcd`catch` body will
be executed, and the process of throwing is finished.  
If no statically-enclosing \xcd`try`-\xcd`catch` block can handle the
exception, the current method call returns (abnormally), throwing the same
exception from the point at which the method was called.  

This process continues until the exception is handled or there are no more
calling methods in the activity. In the latter case, the activity will
terminate abnormally, and the exception will propagate to the activity's root;
see \Sref{ExceptionModel} for details.

Unlike some statically-typed languages with exceptions, X10's exceptions are
all {\em unchecked}. Methods do not declare which exceptions they might throw;
any method can, potentially, throw any exception.


\section{Throw statement}
\index{throw}

%##(ThrowStatement
\begin{bbgrammar}
%(FROM #(prod:ThrowStatement)#)
      ThrowStatement \: \xcd"throw" Exp \xcd";" & (\ref{prod:ThrowStatement}) \\
\end{bbgrammar}
%##)

\index{Exception}
\xcd"throw E" throws an exception whose value is \xcd`E`, which must be an
instance of a subtype of \xcd`x10.lang.Throwable`. 

\begin{ex}
The following code checks if an index is in range and
throws an exception if not.

%~~gen ^^^ Statements110
% package Statements_index_check;
% class ThrowStatementExample {
% def thingie(i:Int, x:Array[Boolean](1))  {
%~~vis
\begin{xten}
if (i < 0 || i >= x.size)
    throw new MyIndexOutOfBoundsException();
\end{xten}
%~~siv
%} }
% class MyIndexOutOfBoundsException extends Exception {}
%~~neg
\end{ex}

\section{Try--catch statement}
\index{try}
\index{catch}
\index{finally}
\index{exception}

%##(TryStatement Catches CatchClause Finally
\begin{bbgrammar}
%(FROM #(prod:TryStatement)#)
        TryStatement \: \xcd"try" Block Catches & (\ref{prod:TryStatement}) \\
                     \| \xcd"try" Block Catches\opt Finally \\
%(FROM #(prod:Catches)#)
             Catches \: CatchClause & (\ref{prod:Catches}) \\
                     \| Catches CatchClause \\
%(FROM #(prod:CatchClause)#)
         CatchClause \: \xcd"catch" \xcd"(" Formal \xcd")" Block & (\ref{prod:CatchClause}) \\
%(FROM #(prod:Finally)#)
             Finally \: \xcd"finally" Block & (\ref{prod:Finally}) \\
\end{bbgrammar}
%##)

Exceptions are handled with a \xcd"try" statement.
A \xcd"try" statement consists of a \xcd"try" block, zero or more
\xcd"catch" blocks, and an optional \xcd"finally" block.

First, the \xcd"try" block is evaluated.  If the block throws an
exception, control transfers to the first matching \xcd"catch"
block, if any.  A \xcd"catch" matches if the value of the
exception thrown is a subclass of the \xcd"catch" block's formal
parameter type.

The \xcd"finally" block, if present, is evaluated on all normal
and exceptional control-flow paths from the \xcd"try" block.
If the \xcd"try" block completes normally
or via a \xcd"return", a \xcd"break", or a
\xcd"continue" statement, 
the \xcd"finally"
block is evaluated, and then control resumes at
the statement following the \xcd"try" statement, at the branch target, or at
the caller as appropriate.
If the \xcd"try" block completes
exceptionally, the \xcd"finally" block is evaluated after the
matching \xcd"catch" block, if any, and when and if the \xcd`finally` block
finishs normally, the
exception is rethrown.


The parameter of a \xcd`catch` block has the block as scope.  It shadows other
variables of the same name.

\begin{ex}
The \xcd`example()` method below executes without any assertion errors
%~~gen ^^^ Statements9x3m
% package Statements9x3m;
% 
%~~vis
\begin{xten}
class Example {
  class SeriousExn extends Throwable {}
  class SillyExn   extends Throwable {}
  var didFinally : Boolean = false;
  def example() {
    try {
       throw new SillyExn();
       assert false; // This cannot happen
    }
    catch(SillyExn)   {return true;}
    catch(SeriousExn) {return false;}
    finally {
       this.didFinally = true;
    }
    return false;
  }
  static def doExample() {
    val e = new Example();
    assert e.example();
    assert e.didFinally == true;
  }
}
\end{xten}
%~~siv
% 
% class Hook { def run() { Example.doExample(); return true; } }
%~~neg

\end{ex}

\limitation{Constraints on exception types in \xcd`catch` blocks are not
currently supported. 
}

\section{Assert}

The \xcd`assert` statement 
%~~stmt~~`~~`~~B:Boolean ~~
\xcd`assert B;` 
checks that the Boolean expression \xcd`B` evaluates to true.  If so,
computation proceeds.  If not, it throws \xcd`x10.lang.AssertionError`.

The extended form 
%~~stmt~~`~~`~~B:Boolean, A:Any ~~ 
\xcd`assert B:A;`
is similar, but provides more debugging information.  The value of the
expression \xcd`A` is available as part of the \xcd`AssertionError`, \eg, to
be printed on the console.

\begin{ex}
\xcd`assert` is useful for confirming properties that you believe to be true
and wish to rely on.  In particular, well-chosen \xcd`assert`s make a program
robust in the face of code changes and unexpected uses of methods.
For example, the following method compute percent differences, but asserts
that it is not dividing by zero.  If the mean is zero, it throws an exception,
including the values of the numbers as potentially useful debugging
information. 
%~~gen ^^^ StmtAssert10
%package StmtAssert10;
% class Example {
%~~vis
\begin{xten}
static def percentDiff(x:Double, y:Double) {
  val diff = x-y;
  val mean = (x+y)/2;
  assert mean != 0.0  : [x,y]; 
  return Math.abs(100 * (diff / mean));
}
\end{xten}
%~~siv
% }
%~~neg

\end{ex}


At times it may be considered important not to check \xcd`assert` statements;
\eg, if the test is expensive and the code is sufficiently well-tested.  The
\xcd`-noassert` command line option causes the compiler to ignore all
\xcd`assert` statements. 
	

\chapter{Places}
\label{XtenPlaces}
\index{place}

An \Xten{} place is a repository for data and activities, corresponding
loosely to a process or a processor. Places induce a concept of ``local''. The
activities running in a place may access data items located at that place with
the efficiency of on-chip access. Accesses to remote places may take orders of
magnitude longer. X10's system of places is designed to make this obvious.
Programmers are aware of the places of their data, and know when they are
incurring communication costs, but the actual operation to do so is easy. It's
not hard to use non-local data; it's simply hard to to do so accidentally.

The set of places available to a computation is determined at the time that
the program is started, and remains fixed through the run of the program. See
the {\tt README} documentation on how to set command line and configuration
options to set the number of places.

Places are first-class values in X10, as instances 
\xcd"x10.lang.Place".   \xcd`Place` provides a number of useful ways to
query places, such as \xcd`Place.places`, which is a  \xcd`Sequence[Place]` of 
the places
available to the current run of the program.

Objects and structs (with one exception) are created in a single place -- the
place that the constructor call was running in. They cannot change places.
They can be {\em copied} to other places, and the special library struct
\Xcd{GlobalRef} allows values at one place to point to values at another.  

\section{The Structure of Places}
\index{place!MAX\_PLACES}
\index{place!FIRST\_PLACES}
\index{MAX\_PLACES}
\index{FIRST\_PLACE}

%~~exp~~`~~`~~ ~~ ^^^ Places10
Places are numbered 0 through \xcd`Place.MAX_PLACES-1`; the number is stored
in the field 
\xcd`pl.id`.  The \xcd`Sequence[Place]` \xcd`Place.places()` contains the places of the
program, in numeric order. 
The program starts by executing a \xcd`main` method at
%~~exp~~`~~`~~ ~~ ^^^ Places20
\xcd`Place.FIRST_PLACE`, which is 
%~~exp~~`~~`~~ ~~ ^^^ Placesoik
\xcd`Place.places()(0)`; see
\Sref{initial-computation}. 

Operations on places include \xcd`pl.next()`, which gives the next entry
(looping around) in \xcd`Place.places` and its opposite \xcd`pl.prev()`. 
In multi-place executions, 
\xcd`here.next()` is a convenient way to express ``a place other than \xcd`here`''.
There are also tests, like  
%~~exp~~`~~`~~pl:Place ~~ ^^^ Placesoid
\xcd`pl.isCUDA()`, which test for particular kinds of processors.




\section{{\tt here}}\index{here}\label{Here}

The variable \xcd"here" is always bound to the place at which the current
computation is running, in the same way that \xcd`this` is always bound to the
instance of the current class (for non-static code), or \xcd`self` is bound to
the instance of the type currently being constrained.  
\xcd`here` may denote different places in the same method body or even the
same expression, due to
place-shifting operations.


This is not unusual for automatic variables:  \Xcd{self} denotes 
two different values (one \xcd`List`, one \xcd`Int`) 
when one describes a non-null list of non-zero numbers as
\xcd`List[Int{self!=0}]{self!=null}`. In the following 
code, \xcd`here` has one value at 
\xcd`h0`, and a different one at \xcd`h1` (unless there is only one place).
%~~gen ^^^ Placesoijo
% package places.are.For.Graces;
% class Example {
% def example() {
%~~vis
\begin{xten}
val h0 = here;
at (here.next()) {
  val h1 = here; 
  assert (h0 != h1);
}
\end{xten}
%~~siv
%} } 
% 
%~~neg
\noindent
(Similar examples show that \xcd`self` and \xcd`this` have the same behavior:
\xcd`self` can be shadowed by constrained types appearing inside of type
constraints, and \xcd`this` by inner classes.)



The following example looks through a list of references to \Xcd{Thing}s.  
It finds those references to things that are \Xcd{here}, and deals with them.  
%~~gen ^^^ Places70
%package Places.Are.For.Graces.2;
%import x10.util.*;
%abstract class Thing {}
%class DoMine {
%  static def dealWith(Thing) {}	
%~~vis
\begin{xten}
  public static def deal(things: List[GlobalRef[Thing]]) {
     for(gr in things) {
        if (gr.home == here) {
           val grHere = 
               gr as GlobalRef[Thing]{gr.home == here};
           val thing <: Thing = grHere();
           dealWith(thing);
        }
     }
  }
\end{xten}
%~~siv
%}
% 
%~~neg

\section{ {\tt at}: Place Changing}\label{AtStatement}
\index{at}
\index{place!changing}

An activity may change place synchronously using the \xcd"at" statement or
\xcd"at" expression. Like any distributed operation, it is 
potentially expensive, as it requires, at a minimum, two messages
and the copying of all data used in the operation, and must be used with care
-- but it provides the basis for multicore programming in X10.

%##(AtStatement AtExp
\begin{bbgrammar}
%(FROM #(prod:AtStmt)#)
              AtStmt \: \xcd"at" \xcd"(" Exp \xcd")" Stmt & (\ref{prod:AtStmt}) \\
%(FROM #(prod:AtExp)#)
               AtExp \: \xcd"at" \xcd"(" Exp \xcd")" ClosureBody & (\ref{prod:AtExp}) \\
\end{bbgrammar}
%##)

The {\it PlaceExp} must be an expression of type \xcd`Place` or some
subtype. For programming convenience, if {\it PlaceExp} is of type
\xcd`GlobalRef[T]` then the \xcd'home' property of \xcd'GlobalRef' is
used as the value of {\it PlaceExp}.

%%AT-COPY%% The \xcd`at`-statment \xcd`at(p;F)S` first evaluates \xcd`p` to a place, then
%%AT-COPY%% copies information to that place as determined by \xcd`F`, and then executes
%%AT-COPY%% \xcd`S` using the resulting copies.  The \xcd`at`-{\em expression}
%%AT-COPY%% \xcd`at(p;F)E` is similar, but it copies the result of the expression \xcd`E`
%%AT-COPY%% and returns the copy as its result.
%%AT-COPY%% 
%%AT-COPY%% The clause \xcd`F` in \xcd`at(p;F)S` is a list of zero or more {\em copy
%%AT-COPY%% specifiers}, explaining what values are to be copied to the place \xcd`p`, and
%%AT-COPY%% how they are to be referred to at \xcd`p`.  
%%AT-COPY%% 

%%AT-COPY%% \begin{ex}
%%AT-COPY%% The following example creates a rail \xcd`a` located \xcd`here`, and copies
%%AT-COPY%% it to another place, giving the copy the name \xcd`a2` there.  The copy is
%%AT-COPY%% modified and examined.  After the \xcd`at` finishes, the original is also
%%AT-COPY%% examined, and (since only the copy, not the original, was modified) is observed
%%AT-COPY%% to be unchanged. 
%%AT-COPY%% %~x~gen ^^^ Places6e1o
%%AT-COPY%% % package Places6e1o;
%%AT-COPY%% % KNOWNFAIL-at
%%AT-COPY%% % class Example { static def example() { 
%%AT-COPY%% %~x~vis
%%AT-COPY%% \begin{xten}
%%AT-COPY%% val a = [1,2,3];
%%AT-COPY%% at(here.next(); a2 = a) {
%%AT-COPY%%   a2(1) = 4;
%%AT-COPY%%   assert a2(0)==1 && a2(1)==4 && a2(2)==3; 
%%AT-COPY%%   // 'a' is not accessible here
%%AT-COPY%% }
%%AT-COPY%% assert  a(0)==1 && a(1)==2 && a(2)==3; 
%%AT-COPY%% \end{xten}
%%AT-COPY%% %~x~siv
%%AT-COPY%% %} } 
%%AT-COPY%% % class Hook { def run() { Example.example(); return true; }}
%%AT-COPY%% %~x~neg
%%AT-COPY%% \end{ex}
%%AT-COPY%% 

\begin{ex}
The following example creates a rail \xcd`a` located \xcd`here`, and copies
it to another place.  \xcd`a` in the second place (\xcd`here.next()`) refers
to the copy.  The copy is
modified and examined.  After the \xcd`at` finishes, the original is also
examined, and (since only the copy, not the original, was modified) is observed
to be unchanged. 
%~~gen ^^^ Places6e1o
% package Places6e1o;
% KNOWNFAIL-at
% class Example { static def example() { 
%~~vis
\begin{xten}
val a = [1,2,3];
at(here.next()) {
  a(1) = 4;
  assert a(0)==1 && a(1)==4 && a(2)==3; 
}
assert  a(0)==1 && a(1)==2 && a(2)==3; 
\end{xten}
%~~siv
%} } 
% class Hook { def run() { Example.example(); return true; }}
%~~neg
\end{ex}

%%AT-COPY%% \subsection{Copy Specifiers}
%%AT-COPY%% \label{sect:copy-spec}
%%AT-COPY%% \index{copy specifier}
%%AT-COPY%% \index{at!copy specifier}
%%AT-COPY%% 
%%AT-COPY%% A single copy specifier can be one of the following forms.   
%%AT-COPY%% Each copy specifier determines an {\em original-expression}, saying what value
%%AT-COPY%% will be copied, and a {\em target variable}, saying what it will be called.
%%AT-COPY%% 
%%AT-COPY%% \begin{itemize}
%%AT-COPY%% 
%%AT-COPY%% \item \xcd`val x = E`, and its usual variants \xcd`val x:T = E`, 
%%AT-COPY%%       \xcd`x : T = E`, and 
%%AT-COPY%%       \xcd`val x <: T = E`, evaluate the expression \xcd`E` at the initial
%%AT-COPY%%       place, copy it to \xcd`p`, and bind \xcd`x` to the copy, as normal for a
%%AT-COPY%%       local \xcd`val` binding.  If a type is supplied, it is checked
%%AT-COPY%%       statically in the usual way.  
%%AT-COPY%%       The original-expression is \xcd`E`, and the target variable is \xcd`x`.
%%AT-COPY%% 
%%AT-COPY%% \begin{ex}
%%AT-COPY%% The following code copies a variable \xcd`a` located \xcd`here` to a variable
%%AT-COPY%% \xcd`d` located \xcd`there`.  
%%AT-COPY%% Note that, while the copy \xcd`d` is available \xcd`there` inside of the \xcd`at`-block,
%%AT-COPY%% the original \xcd`a` is not.  (\xcd`a` could not be available in the block in
%%AT-COPY%% any case; it is not located \xcd`there`.)
%%AT-COPY%% %~~gen ^^^ Places9v2e1
%%AT-COPY%% % package Places9v2e1;
%%AT-COPY%% % KNOWNFAIL-at
%%AT-COPY%% % class Example{ 
%%AT-COPY%% % static def use(Any) = 1;
%%AT-COPY%% % static def example() { 
%%AT-COPY%% %  val there = here.next();
%%AT-COPY%% %~~vis
%%AT-COPY%% \begin{xten}
%%AT-COPY%% var a : Int = 1;
%%AT-COPY%% at(there; val d = a) {
%%AT-COPY%%    assert d == 1;
%%AT-COPY%%    // ERROR: assert a == 1;
%%AT-COPY%% }
%%AT-COPY%% \end{xten}
%%AT-COPY%% %~~siv
%%AT-COPY%% % } } 
%%AT-COPY%% % class Hook{ def run() {Example.example(); return true;}}
%%AT-COPY%% %~~neg
%%AT-COPY%% \end{ex}
%%AT-COPY%% 
%%AT-COPY%% \item \xcd`var x : T = E` evaluates \xcd`E` at the initial place, copies it to
%%AT-COPY%%       \xcd`p`, and binds \xcd`x` to a new \xcd`var` whose initial value is the
%%AT-COPY%%       copy, as normal for a local \xcd`var` binding.
%%AT-COPY%%       If a type is supplied, it is checked
%%AT-COPY%%       statically in the usual way.
%%AT-COPY%%       The original-expression is \xcd`E`, and the target variable is \xcd`x`.
%%AT-COPY%%       Note that, like a \xcd`var` parameter to a method, \xcd`x` is a local
%%AT-COPY%%       variable.  Changes to \xcd`x` will not change anything else. In
%%AT-COPY%%       particular, even if \xcd`x` has the same name as a \xcd`var` variable
%%AT-COPY%%       outside, the two \xcd`var`s are unconnected.  
%%AT-COPY%%       See \Sref{sect:athome} for the way to modify a variable from the
%%AT-COPY%%       surrounding scope.
%%AT-COPY%% 
%%AT-COPY%% \begin{ex}
%%AT-COPY%% The following code copies \xcd`a` to a \xcd`var` named \xcd`e`.  Changing
%%AT-COPY%% \xcd`e` does not change \xcd`a`; the two \xcd`var`s have no ongoing relationship.
%%AT-COPY%% %~~gen ^^^ Places9v2e2
%%AT-COPY%% % package Places9v2e2;
%%AT-COPY%% % KNOWNFAIL-at
%%AT-COPY%% % class Example{ 
%%AT-COPY%% % static def use(Any) = 1;
%%AT-COPY%% % static def example() { 
%%AT-COPY%% %  val there = here.next();
%%AT-COPY%% %~~vis
%%AT-COPY%% \begin{xten}
%%AT-COPY%% var a : Int = 1;
%%AT-COPY%% assert a == 1;
%%AT-COPY%% at(there; var e = a) { 
%%AT-COPY%%    assert e == 1;
%%AT-COPY%%    e += 1;
%%AT-COPY%%    assert e == 2;
%%AT-COPY%% }
%%AT-COPY%% assert a == 1; 
%%AT-COPY%% \end{xten}
%%AT-COPY%% %~~siv
%%AT-COPY%% % 
%%AT-COPY%% % }  } 
%%AT-COPY%% % class Hook{ def run() {Example.example(); return true;}}
%%AT-COPY%% %~~neg
%%AT-COPY%% \end{ex}
%%AT-COPY%% 
%%AT-COPY%% \item \xcd`x = E`, as a copy specifier, is equivalent to \xcd`val x = E`.
%%AT-COPY%%       Note that this abbreviated form is not available as a local variable
%%AT-COPY%%       definition, (because it is used as an assignment statement), but in a
%%AT-COPY%%       copy specifier there are no assignment statements and so the
%%AT-COPY%%       abbreviation is allowed.
%%AT-COPY%%       The original-expression is \xcd`E`, and the target variable is \xcd`x`.
%%AT-COPY%% 
%%AT-COPY%% \begin{ex}
%%AT-COPY%% The following code evaluates an expression \xcd`a+b(0)`.  The result of this
%%AT-COPY%% expression is stored \xcd`there`, in the \xcd`val` variable \xcd`f`, but is
%%AT-COPY%% not stored \xcd`here`. 
%%AT-COPY%% %~~gen ^^^ Places9v2e3
%%AT-COPY%% % package Places9v2e3;
%%AT-COPY%% % KNOWNFAIL-at
%%AT-COPY%% % class Example{ 
%%AT-COPY%% % static def use(Any) = 1;
%%AT-COPY%% % static def example() { 
%%AT-COPY%% %  val there = here.next();
%%AT-COPY%% %~~vis
%%AT-COPY%% \begin{xten}
%%AT-COPY%% var a : Int = 1;
%%AT-COPY%% var b : Rail[Int] = [2,3,4];
%%AT-COPY%% at(there; f = a + b(0)) {
%%AT-COPY%%    assert f == 3;
%%AT-COPY%% }
%%AT-COPY%% \end{xten}
%%AT-COPY%% %~~siv
%%AT-COPY%% % }  } 
%%AT-COPY%% % class Hook{ def run() {Example.example(); return true;}}
%%AT-COPY%% % 
%%AT-COPY%% %~~neg
%%AT-COPY%% 
%%AT-COPY%% 
%%AT-COPY%% \end{ex}
%%AT-COPY%% 
%%AT-COPY%% \item \xcd`x` alone, as a copy specifier, is equivalent to \xcd`val x = x`.
%%AT-COPY%%       It says that the variable \xcd`x` will be copied, and the copy will also
%%AT-COPY%%       be named \xcd`x`.  
%%AT-COPY%%       The original-expression is \xcd`x`, and the target variable is \xcd`x`.
%%AT-COPY%% 
%%AT-COPY%% \begin{ex}
%%AT-COPY%% The following code copies \xcd`b` to \xcd`there`.  The copy is also called
%%AT-COPY%% \xcd`b`.  The two \xcd`b`'s are not connected; \eg, changing one does not
%%AT-COPY%% change the other.
%%AT-COPY%% %~~gen ^^^ Places9v2e4
%%AT-COPY%% % package Places9v2e4;
%%AT-COPY%% % KNOWNFAIL-at
%%AT-COPY%% % class Example{ 
%%AT-COPY%% % static def use(Any) = 1;
%%AT-COPY%% % static def example() { 
%%AT-COPY%% %  val there = here.next();
%%AT-COPY%% %~~vis
%%AT-COPY%% \begin{xten}
%%AT-COPY%% var b : Rail[Int] = [2,3,4];
%%AT-COPY%% assert b(0) == 2;
%%AT-COPY%% at(there; b) {
%%AT-COPY%%   b(0) = 200;  // Modify copy of b.
%%AT-COPY%%   assert b(0) == 200;
%%AT-COPY%% }
%%AT-COPY%% assert b(0) == 2; 
%%AT-COPY%% \end{xten}
%%AT-COPY%% %~~siv
%%AT-COPY%% % 
%%AT-COPY%% % }  } 
%%AT-COPY%% % class Hook{ def run() {Example.example(); return true;}}
%%AT-COPY%% %~~neg
%%AT-COPY%% \end{ex}
%%AT-COPY%% 
%%AT-COPY%% \item A field assignment statements \xcdmath"a.fld = $E_2$", evaluates 
%%AT-COPY%%       \xcd`a` and $E_2$ on the sending side to values $v_1$ and {$v_2$}.  
%%AT-COPY%%       {$v_1$} must be an object with a mutable field \xcd`fld`.  {$v_1$} and
%%AT-COPY%%       {$v_2$} are sent to place \xcd`p`, and the field assignment is performed
%%AT-COPY%%       there.  The modified version of {$v_1$} is available as a \xcd`val`
%%AT-COPY%%       variable \xcd`a`.   The compiler may optimize this, \eg, by neglecting to
%%AT-COPY%%       deserialize \xcdmath"$v_1$.fld", and deserializing {$v_2$} directly into
%%AT-COPY%%       that field rather than into a separate buffer.
%%AT-COPY%% 
%%AT-COPY%% \begin{ex}
%%AT-COPY%% %~~gen ^^^ Places9v2e5
%%AT-COPY%% % package Places9v2e5;
%%AT-COPY%% % KNOWNFAIL
%%AT-COPY%% % class Example {
%%AT-COPY%% % static def use(Any) = 1;
%%AT-COPY%% % static def example() { 
%%AT-COPY%% %  val there = here.next();
%%AT-COPY%% %~~vis
%%AT-COPY%% \begin{xten}
%%AT-COPY%% class Example{ 
%%AT-COPY%%    var f : Int = 1;
%%AT-COPY%%    var g : Int = 2;
%%AT-COPY%%    static def example() { 
%%AT-COPY%%       val there = here.next();
%%AT-COPY%%       val e : Example = new Example();
%%AT-COPY%%       assert e.f == 1 && e.g == 2;
%%AT-COPY%%       at(there; e.f = 3) {
%%AT-COPY%%           assert e.f == 3; && e.g == 2;
%%AT-COPY%%       }
%%AT-COPY%%       assert e.f == 1 && e.g == 2;
%%AT-COPY%%    }
%%AT-COPY%% }
%%AT-COPY%% \end{xten}
%%AT-COPY%% %~~siv
%%AT-COPY%% % class Hook{ def run() {Example.example(); return true;}}
%%AT-COPY%% %~~neg
%%AT-COPY%% %
%%AT-COPY%% \end{ex}
%%AT-COPY%% 
%%AT-COPY%% \item A rail-element assignment 
%%AT-COPY%%       \xcdmath"a($E_1$, $\ldots$, $E_n$) = $E_+$".
%%AT-COPY%%       This copies and transmits \xcd`a` as normal for a rail.  In addition,
%%AT-COPY%%       and 
%%AT-COPY%%       much like a field assignment, it also evaluates all the expressions $E_i$
%%AT-COPY%%       at the sending side to values $v_i$, and transmits them.  \xcd`a`'s value must
%%AT-COPY%%       admit a suitably-typed $n$-ary subscripting operation.  That operation
%%AT-COPY%%       is applied after the values are deserialized at \xcd`p`.  The compiler
%%AT-COPY%%       may optimize this, \eg, by neglecting to deserialize one element of the
%%AT-COPY%%       rail $v_0$, and deserializing $v_+$ directly into that location.  
%%AT-COPY%% 
%%AT-COPY%% 
%%AT-COPY%% \begin{ex}
%%AT-COPY%% The following code sends a modified \xcd`b` to \xcd`there`, while (as always)
%%AT-COPY%% keeping an unmodified version \xcd`here`.   X10 may perform optimizations to
%%AT-COPY%% avoid transmitting the original value of \xcd`b(1)`, since it will be
%%AT-COPY%% overwritten immediately in any case.
%%AT-COPY%% %~~gen ^^^ Places9v2e6
%%AT-COPY%% % package Places9v2e6;
%%AT-COPY%% % KNOWNFAIL
%%AT-COPY%% % class Example{ 
%%AT-COPY%% % static def use(Any) = 1;
%%AT-COPY%% % static def example() { 
%%AT-COPY%% %  val there = here.next();
%%AT-COPY%% %~~vis
%%AT-COPY%% \begin{xten}
%%AT-COPY%% var b = [2,3,4];
%%AT-COPY%% assert b(0) == 2 && b(1) == 3;
%%AT-COPY%% at(there; b(1) = 300) {
%%AT-COPY%%   assert b(0) == 2 && b(1) == 300;
%%AT-COPY%% }
%%AT-COPY%% assert b(0) == 2 && b(1) == 3;
%%AT-COPY%% \end{xten}
%%AT-COPY%% %~~siv
%%AT-COPY%% % 
%%AT-COPY%% %~~neg
%%AT-COPY%% % }  }
%%AT-COPY%% % class Hook{ def run() {Example.example(); return true;}}
%%AT-COPY%% \end{ex}
%%AT-COPY%% 
%%AT-COPY%% \item \xcd`*` may appear as the last copy specifier in the list, indicating
%%AT-COPY%%       that all \xcd`val` variables from outside \xcd`S` which are used in
%%AT-COPY%%       \xcd`S` should be copied. Specifically, let 
%%AT-COPY%%       \xcdmath"x$_1, \ldots, $x$_n$" be all the \xcd`val` variables defined
%%AT-COPY%%       outside of \xcd`S` 
%%AT-COPY%%       mentioned in \xcd`S`. The \xcd`*` copy specifier is equivalent to 
%%AT-COPY%%       the list of variables 
%%AT-COPY%%       \xcdmath"x$_1, \ldots, $x$_n$".
%%AT-COPY%% 
%%AT-COPY%% \begin{ex}
%%AT-COPY%% %~~gen ^^^ Places9v2e7
%%AT-COPY%% % package Places9v2e7;
%%AT-COPY%% % KNOWNFAIL-at
%%AT-COPY%% % class Example{ 
%%AT-COPY%% % static def use(Any) = 1;
%%AT-COPY%% % static def example() { 
%%AT-COPY%% %  val there = here.next();
%%AT-COPY%% %~~vis
%%AT-COPY%% \begin{xten}
%%AT-COPY%% var a : Int = 1;
%%AT-COPY%% val b = [2,3,4];
%%AT-COPY%% at(there; *) {
%%AT-COPY%%   assert a + b(0) == b(1);
%%AT-COPY%% }
%%AT-COPY%% \end{xten}
%%AT-COPY%% %~~siv
%%AT-COPY%% % }  }
%%AT-COPY%% % class Hook{ def run() {Example.example(); return true;}}
%%AT-COPY%% %~~neg
%%AT-COPY%% 
%%AT-COPY%% \end{ex}
%%AT-COPY%% 
%%AT-COPY%% \end{itemize}
%%AT-COPY%% 
%%AT-COPY%% As an important special case, \xcd`at(p;)S` copies {\em nothing} to \xcd`S`.
%%AT-COPY%% This must not be confused with \xcd`at(p)S`, which copies {\em everything}.
%%AT-COPY%% 
%%AT-COPY%% 
%%AT-COPY%% 
%%AT-COPY%% Note that \xcd`at(p;x,*)use(x,y);` is equivalent to \xcd`at(p;*)use(x,y);`.
%%AT-COPY%% In both statements, the \xcd`*` indicates that all variables used in the body
%%AT-COPY%% are to be copied in.  The former makes clear that \xcd`x` is one of the things
%%AT-COPY%% being copied, but, from the \xcd`*`, there may be others. 
%%AT-COPY%% 
%%AT-COPY%% However, other copy specifiers may be used to compute
%%AT-COPY%% values in \xcd`S` which are not available (and thus need not be stored)
%%AT-COPY%% outside of it.  
%%AT-COPY%% 
%%AT-COPY%% \begin{ex}The following code may end up with a large object \xcd`c` in
%%AT-COPY%% memory at \xcd`p` but not at the initial place: 
%%AT-COPY%% %~~gen ^^^ Places3q9u
%%AT-COPY%% % package Places3q9u;
%%AT-COPY%% % KNOWNFAIL-at
%%AT-COPY%% % class Example { 
%%AT-COPY%% % def use(Example, Example, Example) = 1;
%%AT-COPY%% % def Elephant(Example) = 1;
%%AT-COPY%% % static def example(a: Example, b:Example, p:Place) { 
%%AT-COPY%% %~~vis
%%AT-COPY%% \begin{xten}
%%AT-COPY%% at(p; c = a.Elephant(b), *) {
%%AT-COPY%%   use(a,b,c);
%%AT-COPY%% }
%%AT-COPY%% \end{xten}
%%AT-COPY%% %~~siv
%%AT-COPY%% %} } 
%%AT-COPY%% %~~neg
%%AT-COPY%% \end{ex}
%%AT-COPY%% 
%%AT-COPY%% The blanket \xcd`at`-statement \xcd`at(p)S` copies everything.  It is an
%%AT-COPY%% abbreviation for \xcd`at(p;*)S`.  
%%AT-COPY%% When this manual refers to a generic \xcd`at`-statement as \xcd`at(p;F)S`, it
%%AT-COPY%% should be understood as including the blanket \xcd`at` statement \xcd`at(p)S`
%%AT-COPY%% with this interpretation.
%%AT-COPY%% 


\subsection{Copying Values}
%%AT-COPY%% An activity executing statement \xcd"at (q;F) S" at a place \xcd`p`
%%AT-COPY%% evaluates \xcd`q` at \xcd`p` and then moves to \xcd`q` to execute
%%AT-COPY%% \xcd`S`.  
%%AT-COPY%% The original-expressions of \xcd`F` are evaluated at \xcd`p`.
%%AT-COPY%% Their values are copied (\Sref{sect:at-init-val}) to \xcd`q`, and bound to 
%%AT-COPY%% names there, as specified by \xcd`F`.  
%%AT-COPY%% \xcd`S` is evaluated in an environment containing the target variables of
%%AT-COPY%% \xcd`F`, and \xcd`here` and {\em no} other variables.  (In particular, if this
%%AT-COPY%% statement appears in an instance method body and \xcd`this` is not copied,
%%AT-COPY%% \xcd`this` is not accessible.  This fact is important: it allows the
%%AT-COPY%% programmer to control when \xcd`this` is copied, which may be expensive for
%%AT-COPY%% large containers.)

An activity executing \xcd`at(q)S` at a place \xcd`p` evaluates \xcd`q` at
place \xcd`p`, which should be a \xcd`Place`.  It then moves to place \xcd`q`
to execute \xcd`S`.  The values variables that \xcd`S` refers to are copied
(\Sref{sect:at-init-val}) to \xcd`q`, and bound to the variables of the same
name.   If the \xcd`at` is inside of an instance method and \xcd`S` uses
\xcd`this`, \xcd`this` is copied as well.  Note that a field reference
\xcd`this.fld` or a method call \xcd`this.meth()` will cause \xcd`this` to be
copied --- as will their abbreviated forms \xcd`fld` and \xcd`meth()`, despite
the lack of a visible \xcd`this`.  


Note that the value obtained by evaluating \xcd`q`
is not necessarily distinct from \xcd`p` (\eg, \xcd`q` may be
\xcd`here`). 
This does not alter the behavior of \xcd`at`.  
%%AT-COPY%%  \xcd`at(here;F)S` will copy all the values specified by \xcd`F`, 
%%AT-COPY%% even though there is no actual change of place, and even though the original
%%AT-COPY%% values already exist there.
\xcd`at(here)S` will copy all the values mentioned in \xcd`S`, even though
there is no actual change of place, and even though the original values
already exist there. 

On normal termination of \xcd`S` control returns to \xcd`p` and
execution is continued with the statement following 
%%AT-COPY%% \xcd`at (q;F) S`. 
\xcd`at (q) S`. 
If
\xcd`S` terminates abruptly with exception \xcd`E`, \xcd`E` is
serialized into a buffer, the buffer is communicated to \xcd`p` where
it is deserialized into an exception \xcd`E1` and \xcd`at (p) S`
throws \xcd`E1`.

Since 
%%AT-COPY%% \xcd`at(p;F) S` 
\xcd`at(p) S` 
is a synchronous construct, usual control-flow
constructs such as \xcd`break`, \xcd`continue`, \xcd`return` and 
\xcd`throw` are permitted in \xcd`S`.  All concurrency related
constructs -- \xcd`async`, \xcd`finish`, \xcd`atomic`, \xcd`when` are
also permitted.

The \xcd`at`-expression 
%%AT-COPY%% \xcd`at(p;F)E` 
\xcd`at(p)E` 
is similar, except that, in the case of
normal termination of \xcd`E`, the value that \xcd`E` produces is serialized
into a buffer, transported to the starting place, and deserialized, and the
value of the \xcd`at`-expression is the result of deserialization.

\limitation{
X10 does not currently allow {\tt break}, {\tt continue}, or {\tt return}
to exit from an {\tt at}.
}



\subsection{How {\tt at} Copies Values}
\label{sect:at-init-val}

%%AT-COPY%% The values of the original-expressions  specified by \xcd`F` in 
%%AT-COPY%% \xcd`at (p;F)S` are copied to \xcd`p`, as follows.

The values mentioned in \xcd`S` are copied to place \xcd`p` by \xcd`at(p)S` as follows.

First, the original-expressions are evaluated to give a vector of X10 values.
Consider the graph of all values reachable from these values (except for 
\xcd`transient` fields 
(\Sref{sect:transient}, \xcd`GlobalRef`s (\Sref{GlobalRef}); also custom
serialization (\Sref{sect:ser+deser} may alter this behavior)). 

Second this graph is {\em
serialized} into a buffer and transmitted to place \xcd`q`.  Third,
the vector of X10 values is 
re-created at \xcd`q` 
by deserializing the buffer at
\xcd`q`. Fourth, \xcd`S` is executed at \xcd`q`, in an environment in
which each variable \xcd`v` declared in \xcd`F` 
refers to the corresponding deserialized value.  

Note that since values accessed across an \xcd`at` boundary are
copied, the programmer may wish to adopt the discipline that either
variables accessed across an \xcd`at` boundary  contain only structs 
or stateless objects, or the methods invoked on them do not access any
mutable state on the objects. Otherwise the programmer has to ensure
that side effects are made to the correct copy of the object. For this
the struct \xcd`x10.lang.GlobalRef[T]` is often useful.


\subsubsection{Serialization and deserialization.}
\label{sect:ser+deser}
\index{transient}
\index{field!transient}
The X10 runtime provides a default mechanism for
serializing/deserializing an object graph with a given set of roots.
This mechanism may be overridden by the programmer on a per class or
struct basis as described in the API documentation for
\xcd`x10.io.CustomSerialization`.  
The default mechanism performs a
deep copy of the object graph (that is, it copies the object or struct
and, recursively, the values contained in its fields), but does not
traverse or copy \xcd`transient` fields. \xcd`transient` fields are omitted from the
serialized data.   On deserialization, \xcd`transient` fields are initialized
with their default values (\Sref{DefaultValues}).    The types of
\xcd`transient` fields must therefore have default values.



A struct \xcd`s` of type \xcd`x10.lang.GlobalRef[T]` \ref{GlobalRef}
is serialized as a unique global reference to its contained object
\xcd`o` (of type \xcd`T`).  Please see the documentation
of \xcd`x10.lang.GlobalRef[T]` for more details.



\subsection{{\tt at} and Activities}
%%AT-COPY%% \xcd`at(p;F)S` 
\xcd`at(p)S` 
does {\em not} start a new activity.  It should be thought of as
transporting the current activity to \xcd`p`, running \xcd`S` there, and then
transporting it back.  \xcd`async` is the only construct in the
language that starts a new activity. In different contexts, each one
of the following makes sense:
%%AT-COPY%% (1)~\xcd`async at(p;F) S` 
(1)~\xcd`async at(p) S` 
(spawn an activity locally to execute \xcd`S` at
\xcd`p`; here \xcd`p` is evaluated by the spawned activity) , 
%%AT-COPY%% (2)~\xcd`at(p;F) async S` 
(2)~\xcd`at(p) async S` 
(evaluate \xcd`p` and then at \xcd`p` spawn an
activity to execute \xcd`S`), and,
%%AT-COPY%% (3)~\xcd`async at(p;F) async S`. 
(3)~\xcd`async at(p) async S`. 
%%AT-COPY%% In most cases, \xcd`at(p;F) async S` is preferred to
%%\xcd`async at(p;F)`, since In most cases, \xcd`at(p) async S` is
preferred to \xcd`async at(p) S`, since the former form enables a more
efficient runtime implementation.  In the first case, the expression
\xcd`p` is evaluated synchronously by the current activity and then a
single remote async is spawned.  In the second case, \xcd`p` is
semantically required to be evaluated asynchronously with the parent
async as it is contained in the body of an async.  Therefore, if the
compiler cannot prove that "async at (p)" can be safely rewritten into
"at (p) async", a first local async is spawned to evaluate \xcd`p`
then a remote async is spawned to evaluate \xcd`S`.

Since 
%%AT-COPY%% \Xcd{at(p;F) S} 
\Xcd{at(p) S} 
does not start a new activity, 
\xcd`S` may contain constructs which only make sense
within a single activity.  
For example, 
\begin{xten}
    for(x in globalRefsToThings) 
      if (at(x.home) x().isNice()) 
        return x();
\end{xten}
returns the first nice thing in a collection.   If we had used 
\xcd`async at(x.home)`, this would not be allowed; 
you can't \xcd`return` from an
\xcd`async`. 

\limitation{
X10 does not currently allow {\tt break}, {\tt continue}, or {\tt return}
to exit from an {\tt at}.
}



\subsection{Copying from {\tt at} }
\index{at!copying}

%%AT-COPY%% \xcd`at(p;F)S` copies data as specified by \xcd`F`, and sends it
\xcd`at(p)S` copies data required in \xcd`S`, and sends it
to place \xcd`p`, before executing \xcd`S` there. The only things that are not
copied are values only reachable through \xcd`GlobalRef`s and \xcd`transient`
fields, and data omitted by custom serialization.    
%%AT-COPY%% Several choices of copy specifier use the same identifier for the original
%%AT-COPY%% variable outside of 
%%AT-COPY%% \xcd`at(p)S` 
%%AT-COPY%% and its copy inside of \xcd`S`.  
%%AT-COPY%% 

\begin{ex}
%%AT-COPY%% 
%%AT-COPY%% %~~gen ^^^ Places_implicit_copy_from_at_example_1
%%AT-COPY%% % package Places.implicitcopyfromat;
%%AT-COPY%% % class Example {
%%AT-COPY%% % static def example() {
%%AT-COPY%% % 
%%AT-COPY%% %~~vis
%%AT-COPY%% \begin{xten}
%%AT-COPY%% val c = new Cell[Int](9); // (1)
%%AT-COPY%% at (here;c) {             // (2)
%%AT-COPY%%    assert(c() == 9);      // (3)
%%AT-COPY%%    c.set(8);              // (4)
%%AT-COPY%%    assert(c() == 8);      // (5)
%%AT-COPY%% }
%%AT-COPY%% assert(c() == 9);         // (6)
%%AT-COPY%% \end{xten}
%%AT-COPY%% %~~siv
%%AT-COPY%% %}}
%%AT-COPY%% % class Hook{ def run() { Example.example(); return true; } }
%%AT-COPY%% %~~neg
%%AT-COPY%% 

%~~gen ^^^ Places_implicit_copy_from_at_example_1
% package Places.implicitcopyfromat;
% class Example {
% static def example() {
% 
%~~vis
\begin{xten}
val c = new Cell[Int](9); // (1)
at (here) {               // (2) 
   assert(c() == 9);      // (3)
   c.set(8);              // (4)
   assert(c() == 8);      // (5)
}
assert(c() == 9);         // (6)
\end{xten}
%~~siv
%}}
% class Hook{ def run() { Example.example(); return true; } }
%~~neg


The \xcd`at` statement copies the \xcd`Cell` and its contents.  
After \xcd`(1)`, \xcd`c` is a \xcd`Cell` containing 9; call that cell {$c_1$}
At \xcd`(2)`, that cell is copied, resulting in another cell {$c_2$} whose
contents are also 9, as tested at \xcd`(3)`.
(Note that the copying behavior of \xcd`at` happens {\em even when the
destination place is the same as the starting place}--- even with
\xcd`at(here)`.)
At \xcd`(4)`, the contents of {$c_2$} are changed to 8, as confirmed at \xcd`(5)`; the contents of
{$c_1$} are of course untouched.    Finally, at \xcd`(c)`, outside the scope
of the \xcd`at` started at line \xcd`(2)`, \xcd`c` refers to its original
value {$c_1$} rather than the copy {$c_2$}.  
\end{ex}

The \xcd`at` statement induces a {\em deep copy}.  Not only does it copy the
values of variables, it copies values that they refer to through zero or more
levels of reference.  Structures are preserved as well: if two fields
\xcd`x.f` and \xcd`x.g` refer to the same object {$o_1$} in the original, then
\xcd`x.f` and \xcd`x.g` will both refer to the same object {$o_2$} in the
copy.  

\begin{ex}
In the following variation of the preceding example,
\xcd`a`'s original value {$a_1$} is a rail with two references to the same
\xcd`Cell[Int]` {$c_1$}.  The fact that {$a_1(0)$} and {$a_1(1)$} are both
identical to {$c_1$} is demonstrated in \xcd`(A)`-\xcd`(C)`, as {$a_1(0)$} is modified
and {$a_1(1)$} is observed to change.  In \xcd`(D)`-\xcd`(F)`, the copy
{$a_2$} is tested in the same way, showing that {$a_2(0)$} and {$a_2(1)$} both
refer to the same \xcd`Cell[Int]` {$c_2$}.  However, the test at \xcd`(G)`
shows that {$c_2$} is a different cell from {$c_1$}, because changes to
{$c_2$} did not propagate to {$c_1$}.  

%%AT-COPY%% %~~gen ^^^ PlacesAtCopy
%%AT-COPY%% %package Places.AtCopy2;
%%AT-COPY%% %class example {
%%AT-COPY%% %static def Example() {
%%AT-COPY%% %
%%AT-COPY%% %~~vis
%%AT-COPY%% \begin{xten}
%%AT-COPY%% val c = new Cell[Int](5);
%%AT-COPY%% val a : Rail[Cell[Int]] = [c,c as Cell[Int]];
%%AT-COPY%% assert(a(0)() == 5 && a(1)() == 5);     // (A)
%%AT-COPY%% c.set(6);                               // (B)
%%AT-COPY%% assert(a(0)() == 6 && a(1)() == 6);     // (C)
%%AT-COPY%% at(here;a) {
%%AT-COPY%%   assert(a(0)() == 6 && a(1)() == 6);   // (D)
%%AT-COPY%%   c.set(7);                             // (E)
%%AT-COPY%%   assert(a(0)() == 7 && a(1)() == 7);   // (F)
%%AT-COPY%% }
%%AT-COPY%% assert(a(0)() == 6 && a(1)() == 6);     // (G)
%%AT-COPY%% \end{xten}
%%AT-COPY%% %~~siv
%%AT-COPY%% %}}
%%AT-COPY%% %class Hook{ def run() { example.Example(); return true; } }
%%AT-COPY%% %~~neg

%~~gen ^^^ PlacesAtCopy
%package Places.AtCopy2;
%class example {
%static def Example() {
%
%~~vis
\begin{xten}
val c = new Cell[Int](5);
val a : Rail[Cell[Int]] = [c,c as Cell[Int]];
assert(a(0)() == 5 && a(1)() == 5);     // (A)
c.set(6);                               // (B)
assert(a(0)() == 6 && a(1)() == 6);     // (C)
at(here) {
  assert(a(0)() == 6 && a(1)() == 6);   // (D)
  c.set(7);                             // (E)
  assert(a(0)() == 7 && a(1)() == 7);   // (F)
}
assert(a(0)() == 6 && a(1)() == 6);     // (G)
\end{xten}
%~~siv
%}}
%class Hook{ def run() { example.Example(); return true; } }
%~~neg


\end{ex}

\subsection{Copying and Transient Fields}
\label{sect:transient}
\index{at!transient fields and}
\index{transient}
\index{field!transient}

Recall that fields of classes and structs marked \xcd`transient` are not copied by
\xcd`at`.  Instead, they are set to the default values for their types. Types
that do not have default values cannot be used in \xcd`transient` fields.

\begin{ex}
Every \xcd`Trans` object has an \xcd`a`-field equal
to 1.  However, despite the initializer on the \xcd`b` field, it is not the
case that every \xcd`Trans` has \xcd`b==2`.  Since \xcd`b` is \xcd`transient`,
when the \xcd`Trans` value \xcd`this` is copied at \xcd`at(here){...}` in
\xcd`example()`, its \xcd`b` field is not copied, and the default value for an
\xcd`Int`, 0, is used instead.  
Note that we could not make a transient field \xcd`c : Int{c != 0}`, since the
type has no default value, and copying would in fact set it to zero.

%%AT-COPY%% %~~gen ^^^ Places40
%%AT-COPY%% %package Places_transient_a;
%%AT-COPY%% % 
%%AT-COPY%% %~~vis
%%AT-COPY%% \begin{xten}
%%AT-COPY%% class Trans {
%%AT-COPY%%    val a : Int = 1;
%%AT-COPY%%    transient val b : Int = 2;
%%AT-COPY%%    //ERROR transient val c : Int{c != 0} = 3;
%%AT-COPY%%    def example() {
%%AT-COPY%%      assert(a == 1 && b == 2);
%%AT-COPY%%      at(here;a) {
%%AT-COPY%%         assert(a == 1 && b == 0);
%%AT-COPY%%      }
%%AT-COPY%%    }
%%AT-COPY%% }
%%AT-COPY%% \end{xten}
%%AT-COPY%% %~~siv
%%AT-COPY%% %class Hook{ def run() { (new Trans()).example(); return true; } }
%%AT-COPY%% %~~neg

%~~gen ^^^ Places40
%package Places_transient_a;
% 
%~~vis
\begin{xten}
class Trans {
   val a : Int = 1;
   transient val b : Int = 2;
   //ERROR: transient val c : Int{c != 0} = 3;
   def example() {
     assert(a == 1 && b == 2);
     at(here) {
        assert(a == 1 && b == 0);
     }
   }
}
\end{xten}
%~~siv
%class Hook{ def run() { (new Trans()).example(); return true; } }
%~~neg



\end{ex}

\subsection{Copying and GlobalRef}
\label{GlobalRef}
\index{at!GlobalRef}
\index{at!blocking copying}

%%The other barrier to the potentially copious copying behavior of \xcd`at`
%%is the \xcd`GlobalRef` struct.  
A \xcd`GlobalRef[T]` (say \xcd`g`) contains a reference to
a value \xcd`v` of type \xcd`T`, in a form which can be transmitted, and a \xcd`Place`
\xcd`g.home` indicating where the value lives. When a 
\xcd`GlobalRef` is serialized an opaque, globally unique handle to
\xcd`v` is created.  

\begin{ex}The following example does not copy the value \xcd`huge`.  However, \xcd`huge`
would have been copied if it had been put into a \xcd`Cell`, or simply used
directly. 

%%AT-COPY%% %~~gen ^^^ Places50
%%AT-COPY%% %package Places.copyingblockingwithglobref;
%%AT-COPY%% % class GR {
%%AT-COPY%% %  static def use(Any){}
%%AT-COPY%% %  static def example() {
%%AT-COPY%% % 
%%AT-COPY%% %~~vis
%%AT-COPY%% \begin{xten}
%%AT-COPY%% val huge = "A potentially big thing";
%%AT-COPY%% val href = GlobalRef(huge);
%%AT-COPY%% at (here;href) {
%%AT-COPY%%    use(href);
%%AT-COPY%%   }
%%AT-COPY%% }
%%AT-COPY%% \end{xten}
%%AT-COPY%% %~~siv
%%AT-COPY%% %}
%%AT-COPY%% % class Hook{ def run() { GR.example(); return true; } }
%%AT-COPY%% %~~neg

%~~gen ^^^ Places50
%package Places.copyingblockingwithglobref;
% class GR {
%  static def use(Any){}
%  static def example() {
% 
%~~vis
\begin{xten}
val huge = "A potentially big thing";
val href = GlobalRef(huge);
at (here) {
   use(href);
  }
}
\end{xten}
%~~siv
%}
% class Hook{ def run() { GR.example(); return true; } }
%~~neg


\end{ex}

Values protected in \xcd`GlobalRef`s can be retrieved by the application
%~~exp~~`~~`~~ g:GlobalRef[Any]{here == g.home}~~ ^^^Places4e7q
operation \xcd`g()`.  \xcd`g()` is guarded; it can 
only be called when \xcd`g.home == here`.  If you  want to do anything other
than pass a global reference around or compare two of them for equality, you
need to placeshift back to the home place of the reference, often with
\xcd`at(g.home)`.   

\begin{ex}The following program, for reasons best known to the programmer,
modifies the 
command-line argument array.

%%AT-COPY%% 
%%AT-COPY%% %~~gen ^^^ Places60
%%AT-COPY%% % package Places.Atsome.Globref2;
%%AT-COPY%% % class GR2 {
%%AT-COPY%% % 
%%AT-COPY%% %~~vis
%%AT-COPY%% \begin{xten}
%%AT-COPY%%   public static def main(argv:Rail[String]) {
%%AT-COPY%%     val argref = GlobalRef[Rail[String]](argv);
%%AT-COPY%%     at(here.next(); argref) 
%%AT-COPY%%         use(argref);
%%AT-COPY%%   }
%%AT-COPY%%   static def use(argref : GlobalRef[Rail[String]]) {
%%AT-COPY%%     at(argref.home; argref) {
%%AT-COPY%%       val argv = argref();
%%AT-COPY%%       argv(0) = "Hi!";
%%AT-COPY%%     }
%%AT-COPY%%   }
%%AT-COPY%% \end{xten}
%%AT-COPY%% %~~siv
%%AT-COPY%% %} 
%%AT-COPY%% % class Hook{ def run() { GR2.main(["what, me weasel?" as String]); return true; }}
%%AT-COPY%% %~~neg
%%AT-COPY%% 

%~~gen ^^^ Places60
% package Places.Atsome.Globref2;
% class GR2 {
% 
%~~vis
\begin{xten}
  public static def main(argv:Rail[String]) {
    val argref = GlobalRef[Rail[String]](argv);
    at(here.next()) 
        use(argref);
  }
  static def use(argref : GlobalRef[Rail[String]]) {
    at(argref) {
      val argv = argref();
      argv(0) = "Hi!";
    }
  }
\end{xten}
%~~siv
%} 
% class Hook{ def run() { GR2.main(["what, me weasel?" as String]); return true; }}
%~~neg

\end{ex}

There is an implicit coercion from \xcd`GlobalRef[T]` to \xcd`Place`, so
\xcd`at(argref)S` goes to \xcd`argref.home`.  


\subsection{Warnings about \xcd`at`}
There are two dangers involved with \xcd`at`: 
\begin{itemize}
\item Careless use of \xcd`at` can result in copying and transmission
of very large data structures.  
%%AT-COPY%% This is particularly an issue with the blanket
%%AT-COPY%% \xcd`at` statement, \xcd`at(p)S`, where everything used in \xcd`S` is copied.  
In particular, it is very easy to capture
\xcd`this` -- a field reference will do it -- and accidentally copy everything
that \xcd`this` refers to, which can be very large.  A disciplined use of copy
specifiers to make explicit just what gets copied can ameliorate this issue.

\item As seen in the examples above, a local variable reference
  \xcd`x` may refer to different objects in different nested \xcd`at`
  scopes. The programmer must either ensure that a variable accessed
  across an \xcd`at` boundary has no mutable state or be prepared to
  reason about which copy gets modified.   A disciplined use of copy specifiers to give
  different names to variables can ameliorate this concern.
\end{itemize}


%%AT-COPY%% \section{{\tt athome}: Returning Values from {\tt at}-Blocks}
%%AT-COPY%% \label{sect:athome}
%%AT-COPY%% \index{athome}
%%AT-COPY%% 
%%AT-COPY%% The 
%%AT-COPY%% \xcd`at(p;F)S` 
%%AT-COPY%% construct renders external variables unavailable within
%%AT-COPY%% \xcd`S`.  However, it is often useful to transmit values back from \xcd`S`,
%%AT-COPY%% and store them in external variables. 
%%AT-COPY%% 
%%AT-COPY%% The \xcd`athome(V;F)S` construct provides
%%AT-COPY%% this ability.  \xcd`V` is a list of variables, which must all be defined at
%%AT-COPY%% the same place.  \xcd`athome(V;F)S` goes to the place where the variables are
%%AT-COPY%% defined, copying \xcd`F` as for \xcd`at(p;F)S`, and executes \xcd`S` ---
%%AT-COPY%% allowing reading, assignment and initialization of the listed variables in
%%AT-COPY%% \xcd`V`. 
%%AT-COPY%% 
%%AT-COPY%% \xcd`V`, the list of variables, may include one or more variables.  It is a
%%AT-COPY%% static error if X10 cannot determine that all the variables in the list are
%%AT-COPY%% defined at the same place.
%%AT-COPY%% 
%%AT-COPY%% 
%%AT-COPY%% 
%%AT-COPY%% 
%%AT-COPY%% \begin{ex}
%%AT-COPY%% \xcd`athome` allows returning multiple pieces of information from an
%%AT-COPY%% \xcd`at`-statement.  In the following example, we return two data: 
%%AT-COPY%% one as a \xcd`val` named \xcd`square`, and the other as an addition in to a
%%AT-COPY%% partially-computed polynomial named \xcd`poly`.  
%%AT-COPY%% %~~gen ^^^ Places5f9g
%%AT-COPY%% % package Places5f9g;
%%AT-COPY%% % % KNOWNFAIL-at
%%AT-COPY%% % class Example { 
%%AT-COPY%% %~~vis
%%AT-COPY%% \begin{xten}
%%AT-COPY%% static def example(a: Int, mathProc: Place) { 
%%AT-COPY%%   val square : Int;
%%AT-COPY%%   var poly : Int = 1 + a; // will be 1+a+a*a
%%AT-COPY%%   at(mathProc; a) {
%%AT-COPY%%     val sq = a*a; 
%%AT-COPY%%     athome(square, poly; sq) {
%%AT-COPY%%        square = sq;  // initialization
%%AT-COPY%%        poly += sq;   // read and update
%%AT-COPY%%     }
%%AT-COPY%%   return [square, poly];
%%AT-COPY%%   }
%%AT-COPY%% \end{xten}
%%AT-COPY%% %~~siv
%%AT-COPY%% %}}
%%AT-COPY%% % class Hook { def run() { 
%%AT-COPY%% %   val e = example(2, here);
%%AT-COPY%% %   assert e(0) == 4 && e(1) == 7;
%%AT-COPY%% %   return true;
%%AT-COPY%% % }} 
%%AT-COPY%% %~~neg
%%AT-COPY%% \end{ex}
%%AT-COPY%% 
%%AT-COPY%% The abbreviated forms 
%%AT-COPY%% \xcd`athome (*) S` and 
%%AT-COPY%% \xcd`athome S` 
%%AT-COPY%% allow a block of assignments without specifying the variables being assigned
%%AT-COPY%% to, which is convenient for a small set of assignments. 
%%AT-COPY%% They 
%%AT-COPY%% are both equivalent to \xcd`athome(V;F)S`,
%%AT-COPY%% where: 
%%AT-COPY%% \begin{itemize}
%%AT-COPY%% \item \xcd`V` is the list of all variables appearing on the left-hand side of
%%AT-COPY%%       an assignment or update statement in \xcd`S`, excluding those which
%%AT-COPY%%       appear inside the body of an \xcd`at` or \xcd`athome` statement in \xcd`S`;
%%AT-COPY%% \item \xcd`F` is the same as for \xcd`at(p)S` (\Sref{sect:copy-spec})
%%AT-COPY%% \end{itemize}
%%AT-COPY%% 
%%AT-COPY%% 
%%AT-COPY%% \begin{ex}
%%AT-COPY%% 
%%AT-COPY%% Much as the blanket \xcd`at` construct \xcd`at(p)S` is convenient for
%%AT-COPY%% executing a small code body at another place, the blanket \xcd`athome`
%%AT-COPY%% construct \xcd`athome(*) S` 
%%AT-COPY%% (which may be written as simply \xcd`athome S`)
%%AT-COPY%% is convenient for returning a result or two.   The
%%AT-COPY%% preceding example could have been written using blanket statements.
%%AT-COPY%% 
%%AT-COPY%% %~~gen ^^^ Places5f9gblanket
%%AT-COPY%% % package Places5f9gblanket;
%%AT-COPY%% % class Example { 
%%AT-COPY%% % KNOWNFAIL-at
%%AT-COPY%% %~~vis
%%AT-COPY%% \begin{xten}
%%AT-COPY%% static def example(a: Int, mathProc: Place) { 
%%AT-COPY%%   val square : Int;
%%AT-COPY%%   var poly : Int = 1 + a; // will be 1+a+a*a
%%AT-COPY%%   at(mathProc) {
%%AT-COPY%%     val sq = a*a; 
%%AT-COPY%%     athome {
%%AT-COPY%%        square = sq;  // initialization
%%AT-COPY%%        poly += sq;   // read and update
%%AT-COPY%%     }
%%AT-COPY%%   return [square, poly];
%%AT-COPY%%   }
%%AT-COPY%% \end{xten}
%%AT-COPY%% %~~siv
%%AT-COPY%% %}}
%%AT-COPY%% % class Hook { def run() { 
%%AT-COPY%% %   val e = example(2, here);
%%AT-COPY%% %   assert e(0) == 4 && e(1) == 7;
%%AT-COPY%% %   return true;
%%AT-COPY%% % }} 
%%AT-COPY%% %~~neg
%%AT-COPY%% \end{ex}
%%AT-COPY%% 
%%AT-COPY%% {\bf Design:} It is not fundamentally essential to distinguish \xcd`at` from
%%AT-COPY%% \xcd`athome`.  \xcd`at(p;F)S` could allow writing to variables whose homes are
%%AT-COPY%% known at compile-time to be equal to \xcd`p`.  Indeed, in earlier versions of
%%AT-COPY%% X10, it did so.    This required an idiom in which programmers had to manage
%%AT-COPY%% the home locations of variables directly, and keep track of which home
%%AT-COPY%% location corresponded to which variable.  The \xcd`athome` construct makes
%%AT-COPY%% this idiom more convenient. 
	
\chapter{Activities}\label{XtenActivities}

An {\em activity} is a statement being executed, independently, with its own
local variables; it may be thought of as a very light-weight thread. An
\Xten{} computation may have many concurrent {\em activities} executing at any
give time.  All X10 code runs as part of an activity; when an X10 program is
started, the \xcd`main` method is invoked in an activity, called the {\em root
activity}.\index{root
activity}


Activities coordinate their execution by various control and data structures.
For example, `
%~~stmt~~`~~`~~x:Int, var y:Int ~~
\xcd`when(x==0);` blocks the current activity until some other activity
sets \xcd`x` to zero.  However, activities determine the places at which they
may be blocked and resumed, by \xcd`when` and similar constructs.  There are
no means by which one activity can arbitrarily interrupt, block, or resume
another, no method  \xcd`activity.interrupt()`.

An activity may be {\em running}, {\em blocked} on some condition or {\em
terminated}. If terminated, it is terminated in the same way that its
statement is: in particular, if the statement terminates abruptly, the
activity terminates abruptly for the same reason.
(\Sref{ExceptionModel}).

Activities can be long-running entities with a good deal of local state.  In
particular they can involve recursive method calls (and therefore have runtime
stacks).  However, activities can also be short-running light-weight entities,
\eg, it is reasonable to have an activity that simply increments a variable.

An activity may asynchronously and in parallel launch activities at
other places.  Every activity save the initial \xcd`main` activity is spawned
by another.  Thus, at any instant, the activities in a program form a tree.

X10 uses this tree in crucial ways.  
First is the distinction 
between {\em local} termination and {\em global}
termination of a statement. The execution of a statement by an
activity is said to terminate locally when the activity has finished
all its computation. (For instance the
creation of an asynchronous activity terminates locally when the
activity has been created.)  It is said to terminate globally when it
has terminated locally and all activities that it may have spawned at
any place have, recursively, terminated globally.
For example, consider: 
%~~gen
% package Activites.Are.For.Whacktivities;
% class Example {
% def example( s1:() => Void, s2 : () => Void ) {
%~~vis
\begin{xten}
async {s1();}
async {s2();}
\end{xten}
%~~siv
% } } 
%~~neg
The primary activity spawns two child activities and then terminates locally,
very quickly.  The child activities may take arbitrary amounts of time to
terminate (and may spawn grandchildren).  When \xcd`s1()`, \xcd`s2()`, and
all their descendants terminate locally, then the primary activity terminates
globally. 

The program as a whole terminates when the root activity terminates globally.
In particular, X10 does not permit the creation of 
daemon threads---threads that outlive the lifetime of the root
activity.  We say that an \Xten{} computation is {\em rooted}
(\Sref{initial-computation}).

\futureext{ We may permit the initial activity to be a daemon activity
to permit reactive computations, such as webservers, that may not
terminate.}

\section{The \Xten{} rooted exception model}
\label{ExceptionModel}
\index{Exception!model}

The rooted nature of \Xten{} computations permits the definition of a
{\em rooted exception model.} In multi-threaded programming languages
there is a natural parent-child relationship between a thread and a
thread that it spawns. Typically the parent thread continues execution
in parallel with the child thread. Therefore the parent thread cannot
serve to catch any exceptions thrown by the child thread. 

The presence of a root activity and the concept of global termination permits
\Xten{} to adopt a more powerful exception model. In any state of the
computation, say that an activity $A$ is {\em a root of} an activity $B$ if
$A$ is an ancestor of $B$ and $A$ is blocked at a statement (such as the
\xcd"finish" statement \Sref{finish}) awaiting the termination of $B$ (and
possibly other activities). For every \Xten{} computation, the \emph{root-of}
relation is guaranteed to be a tree. The root of the tree is the root activity
of the entire computation. If $A$ is the nearest root of $B$, the path from
$A$ to $B$ is called the {\em activation path} for the activity.\footnote{Note
  that depending on the state of the computation the activation path may
  traverse activities that are running, blocked or terminated.}

We may now state the exception model for \Xten.  An uncaught exception
propagates up the activation path to its nearest root activity, where
it may be handled locally or propagated up the \emph{root-of} tree when
the activity terminates (based on the semantics of the statement being
executed by the activity).\footnote{In \XtenCurrVer{} the \xcd"finish"
statement is the only statement that marks its activity as a root
activity. Future versions of the language may introduce more such
statements.}  In Java, exceptions may be overlooked because there is no good
place to put a \xcd`try`-\xcd`catch` block; this is never the case in X10.

\section{\xcd`at`: Place changing}\label{AtStatement}

An activity may change place using the \xcd"at" statement or
\xcd"at" expression:

\begin{grammar}
Statement \: AtStatement \\
AtStatement \: \xcd"at" PlaceExpressionSingleList Statement \\
Expression \: AtExpression \\
AtExpression \: \xcd"at" PlaceExpressionSingleList ClosureBody 
\end{grammar}

The statement \xcd"at (p) S" executes the statement \xcd"S"
synchronously at a place described by \xcd"p".
The expression \xcd"at (p) E" executes the statement \xcd"E"
synchronously at place \xcd"p", returning the result to the
originating place.  




\xcd`p` may be an expression of type \xcd`Place`, in which case its value is
used as the place to execute the body: 
%~~gen
% package Activities.At.A.Standstill;
% class Example {
% def example(ob: Object, S: ()=>Void) {
%~~vis
\begin{xten}
   at (here.next()) S();
\end{xten}
%~~siv
% } } 
%~~neg
\noindent



\xcd`at(p)S` does {\em not} start a new activity.  It should be thought of as
transporting the current activity to \xcd`p`, running \xcd`S` there, and then
transporting it back.    If you want to start a new activity, use \xcd`async`;
if you want to start a new activity at \xcd`p`, use 
\xcd`at(p) async S`.  

As a consequence of this, \xcd`S` may contain constructs which only make sense
within a single activity.  
For example, 
\begin{xten}
    for(x in globalRefsToThings) 
    if (at(x.home) x().nice()) 
        return x();
\end{xten}
returns the first nice thing in a collection.   If we had used 
\xcd`async at(x.home)`, this would not be allowed; 
you can't \xcd`return` from an
\xcd`async`. 



\section{\xcd`async`: Spawning an activity}\label{AsynchronousActivity}\label{AsyncActivity}

Asynchronous activities serve as a single abstraction for supporting a
wide range of concurrency constructs such as message passing, threads,
DMA, streaming, data prefetching. (In general, asynchronous operations
are better suited for supporting scalability than synchronous
operations.)

An activity is created by executing the \xcd`async` statement: 

\begin{grammar}
Statement \: AsyncStatement \\
AsyncStatement \: \xcd"async"  Statement \\
PlaceExpressionSingleList \: \xcd"(" PlaceExpression \xcd")" \\
PlaceExpression \: Expression 
\end{grammar} 


The basic form of \xcd`async` is \xcd`async S`, which starts a new activity
located \xcd`here` executing \xcd`S`.   


\bard{The followingin para is under investigation:}
In many cases the compiler may infer the unique place at which the
statement is to be executed by an analysis of the types of the
variables occurring in the statement. (The place must be such that the
statement can be executed safely, without generating a
\xcd"BadPlaceException".) In such cases the programmer may omit the
place designator; the compiler will throw an error if it cannot
determine the unique designated place.\footnote{\XtenCurrVer{} does
not specify a particular algorithm; this will be fixed in future
versions.}

An activity $A$ executes the statement \xcd"async (P) S" by launching
a new activity $B$ at place \xcd`P` (or \xcd`P.home` if \xcd`P` is of an
object type), to execute \xcd`S`. The statement terminates locally as soon as $B$ is
launched.  The activation path for $B$ is that of $A$ augmented by the
information that {$A$} is the parent of {$B$}. 
$B$
terminates normally when $S$ terminates normally.  It terminates
abruptly if $S$ throws an uncaught exception. The exception is
propagated to $A$ if $A$ is a root activity (see \Sref{finish}),
otherwise it is propagated through $A$ to $A$'s root activity. Note that while
{$A$} is running, exceptions thrown by activities it has already
spawned may propagate through it up to its root activity, without {$A$} noticing.

Multiple activities launched by a single activity at another place are not
ordered in any way. They are added to the set of activities at the target
place and will be executed based on the local scheduler's decisions.
If some particular sequencing of events is needed, \xcd`when`, \xcd`atomic`,
\xcd`finish`, clocks, and other X10 constructs can be used.
\Xten{} implementations are not required to have fair schedulers,
though every implementation should make a best faith effort to ensure
that every activity eventually gets a chance to make forward progress.

\begin{staticrule*}
The statement in the body of an \xcd"async" is subject to the
restriction that it must be acceptable as the body of a \xcd"void"
method for an anonymous inner class declared at that point in the code,
which throws no checked exceptions. As such, it may reference
variables in lexically enclosing scopes (including \xcd"clock"
variables, \Sref{XtenClocks}) provided that such variables are
(implicitly or explicitly) \xcd"val".
\end{staticrule*}

\section{Finish}\index{finish}\label{finish}
The statement \xcd"finish S" converts global termination to local
termination and introduces a root activity.   It executes \xcd`S`, and then
waits for all activities spawned by \xcd`S`, directly or indirectly, to
finish. It also collects exceptions thrown by those activities.

\begin{grammar}
Statement \: FinishStatement \\
FinishStatement \: \xcd"finish" Statement 
\end{grammar}

An activity $A$ executes \xcd"finish S" by executing \xcd"S".  The
execution of \xcd"S" may spawn other asynchronous activities (here or
at other places).  Uncaught exceptions thrown or propagated by any
activity spawned by \xcd"S" are accumulated at \xcd"finish S".
\xcd"finish S" terminates locally when all activities spawned by
\xcd"S" terminate globally (either abruptly or normally). If \xcd"S"
terminates normally, then \xcd"finish S" terminates normally and $A$
continues execution with the next statement after \xcd"finish S".  If
\xcd"S" terminates abruptly, then \xcd"finish S" terminates abruptly
and throws a single exception, \Xcd{x10.lang.MultipleExceptions}
formed from the collection of exceptions accumulated at \xcd"finish S".

Thus a \xcd"finish S" statement serves as a collection point for
uncaught exceptions generated during the execution of \xcd"S".

Note that repeatedly \xcd"finish"ing a statement has little effect after
the first \xcd"finish": \xcd"finish finish S" is indistinguishable
from \xcd"finish S" if \xcd`S` throws no exceptions.  (If \xcd`S` throws
exceptions, \xcd`finish S` wraps them in one layer of 
\xcd`MultipleExceptions` and \xcd`finish finish S` in two layers.)

%%OLIVIER-DENIES%% \paragraph{Interaction with clocks.}\label{sec:finish:clock-rule}
%%OLIVIER-DENIES%% 
%%OLIVIER-DENIES%% \xcd"finish S" interacts with clocks (\Sref{XtenClocks}). 
%%OLIVIER-DENIES%% While executing \xcd"S", an activity must not spawn any \xcd"clocked"
%%OLIVIER-DENIES%% asyncs. (Asyncs spawned during the execution of \xcd"S" may spawn
%%OLIVIER-DENIES%% clocked asyncs.) A
%%OLIVIER-DENIES%% \xcd"ClockUseException"\index{clock!ClockUseException} is thrown
%%OLIVIER-DENIES%% if (and when) this condition is violated.
%%OLIVIER-DENIES%% 
%%OLIVIER-DENIES%% This is necessary to prevent deadlocks.  In the following invalid code 
%%OLIVIER-DENIES%% %~s~gen
%%OLIVIER-DENIES%% % package Activities.Finish.Hates.Clocks;
%%OLIVIER-DENIES%% % class Example{
%%OLIVIER-DENIES%% % def example() {
%%OLIVIER-DENIES%% %~s~vis
%%OLIVIER-DENIES%% \begin{xten}
%%OLIVIER-DENIES%% val c:Clock = Clock.make();
%%OLIVIER-DENIES%% async clocked(c) {                // (A) 
%%OLIVIER-DENIES%%       finish async clocked(c) {   // (B) INVALID
%%OLIVIER-DENIES%%             next;                 // (Bnext)
%%OLIVIER-DENIES%%       }
%%OLIVIER-DENIES%%       next;                       // (Anext)
%%OLIVIER-DENIES%% }
%%OLIVIER-DENIES%% \end{xten}
%%OLIVIER-DENIES%% %~s~siv
%%OLIVIER-DENIES%% % } } 
%%OLIVIER-DENIES%% %~s~neg
%%OLIVIER-DENIES%% \xcd`(A)`, first of all, waits for the \xcd`finish` containing \xcd`(B)` to
%%OLIVIER-DENIES%% finish.  
%%OLIVIER-DENIES%% \xcd`(B)` will execute its \xcd`next` at \xcd`(Bnext)`, and then wait for all
%%OLIVIER-DENIES%% other activities registered on \xcd`c` to execute their \xcd`next`s.
%%OLIVIER-DENIES%% However, \xcd`(A)` is registered on \xcd`c`.  So, \xcd`(B)` cannot finish
%%OLIVIER-DENIES%% until \xcd`(A)` has proceeded to \xcd`(Anext)`, and \xcd`(A)` cannot proceed
%%OLIVIER-DENIES%% until \xcd`(B)` finishes. Thus, this causes deadlock.
%%OLIVIER-DENIES%% 
%%OLIVIER-DENIES%% 
%%OLIVIER-DENIES%% 
%%OLIVIER-DENIES%% In \XtenCurrVer{} this condition is checked dynamically; future
%%OLIVIER-DENIES%% versions of the language will introduce type qualifiers which permit
%%OLIVIER-DENIES%% this condition to be checked statically.
%%OLIVIER-DENIES%% 
%%OLIVIER-DENIES%% \futureext{
%%OLIVIER-DENIES%% The semantics of \xcd"finish S" is conjunctive; it terminates when all
%%OLIVIER-DENIES%% the activities created during the execution of \xcd"S" (recursively)
%%OLIVIER-DENIES%% terminate. In many situations (e.g., nondeterministic search) it is
%%OLIVIER-DENIES%% natural to require a statement to terminate when any {\em one} of the
%%OLIVIER-DENIES%% activities it has spawned succeeds. The other activities may then be
%%OLIVIER-DENIES%% safely aborted. Future versions of the language may introduce a
%%OLIVIER-DENIES%% \xcd"finishone S" construct to support such speculative or nondeterministic
%%OLIVIER-DENIES%% computation.
%%OLIVIER-DENIES%% }
%%OLIVIER-DENIES%% 



\section{Initial activity}\label{initial-computation}\index{initial activity}

An \Xten{} computation is initiated from the command line on the
presentation of a classname \xcd"C". The class must have a
\xcd"public static def main(a: Rail[String]):Void" method, otherwise an
exception is thrown
and the computation terminates.  The single statement
\begin{xten}
finish async (Place.FIRST_PLACE) {
  C.main(s);
}
\end{xten} 
\noindent is executed where \xcd"s" is an Rail of strings created
from the command line arguments. This single activity is the root activity
for the entire computation. (See \Sref{XtenPlaces} for a discussion of
places.)

%% Say something about configuration information? 




\section{Ateach statements}\index{\Xcd{ateach}}\label{ateach-section}

\begin{grammar}
Statement \: AtEachStatement \\
AtEachStatement \:
      \xcd"ateach" \xcd"(" Formal \xcd"in" Expression \xcd")"
         Statement \\
AtEachStatement \:
      \xcd"ateach" \xcd"(" Expression \xcd")"
         Statement 
\end{grammar}

The \xcd"ateach" statement \xcd`ateach (p in D) S`
spawns an activity \xcd`S` at each place \xcd`p` of a distribution \xcd`D`. 
In \xcd`ateach(p in D) S`, 
\xcd`D` must be either of type \xcd"Dist" 
(see \Sref{XtenDistributions})
or of type
\xcd`DistArray[T]` (see \Sref{XtenArrays}), 
and \xcd`p` will be of type \xcd"Point" (see \Sref{point-syntax}).

\xcd`ateach(p in D)S` is equivalent to 
\xcd`for(p in D) at(D(p)) async S`.  That is, the elements of \xcd`D` are all
points \xcd`p`.  \xcd`D(p)` is a \xcd`Place`.  \xcd`ateach(p in D)S` executes
the body \xcd`S` at the place \xcd`D(p)` (and may use the point \xcd`p`
there). 


However, the compiler may implement it more efficiently to avoid extraneous
communications.  In particular, it is recommended that \xcd`ateach(p in D)S`
be implemented as the following code, which coordinates with each place of
\xcd`D` just once, rather than once per element of \xcd`D` at that place: 

%~~gen
% package Activities.Activities.Activities;
% class EquivCode {
% static def S(pt:Point) {}
% static def example(D:Dist) {
%~~vis
\begin{xten}
for (p in D.places()) async at (p) {
    for (pt in D|here) async {
        S(pt);
    }
}
\end{xten}
%~~siv
%}} 
%~~neg

If \xcd`e` is an \xcd`DistArray[T]`, then \xcd`ateach (p in e)S` is identical to
\xcd`ateach(p in e.dist)S`; the iteration is over the array's underlying
distribution.   
The code below is a common and generally efficient way to work with the
elements of a distributed array:
%~~gen
%package Activities.For.Fnu.And.Pforit;
%class Example[T]{
%  def dealWith(T):Void = {}
% def idiom(A:DistArray[T]){
%~~vis
\begin{xten}
ateach(p in A) 
  dealWith(A(p));
\end{xten}
%~~siv
%}}
%~~neg








\section{At expressions}

\begin{grammar}
Expression \: \xcd"at" \xcd"(" Expression \xcd")" Expression
\end{grammar}

An \Xcd{at} expression evaluates an expression synchronously at the
given place and returns its value.  For instance a copy of the
value pointed to by a \Xcd{GlobalRef} may be obtained using
the \Xcd{fetch} method:
%~~gen
% package Activities.AtExpressions.Fetching;
% class Example[T] {
%~~vis
\begin{xten}
  def fetch(g:GlobalRef[T]):T = at (g) g();
\end{xten}
%~~siv
% } 
%~~neg

The expression evaluation may spawn asynchronous activities. The \Xcd{at}
expression will return without waiting for those activities to terminate. That
is, \Xcd{at} does not have built-in \Xcd{finish} semantics.

\section{Atomic blocks}\label{AtomicBlocks}\index{atomic blocks}
Languages such as \java{} use low-level synchronization locks to allow
multiple interacting threads to coordinate the mutation of shared
data. \Xten{} eschews locks in favor of a very simple high-level
construct, the {\em atomic block}.

A programmer may use atomic blocks to guarantee that invariants of
shared data-structures are maintained even as they are being accessed
simultaneously by multiple activities running in the same place.  

For example, consider a class \xcd`Redund[T]`, which encapsulates a list
\xcd`list` and, (redundantly) keeps the size of the list in a second field
\xcd`size`.  Then \xcd`r:Redund[T]` has the invariant 
\xcd`r.list.size() == r.size`, which must be true at any point that there are
no method calls on \xcd`r` active.

If the \xcd`add` method on \xcd`Redund` (which adds an element to the list) 
were defined as: 
%~~gen
% package Activities.Atomic.Redund.One;
% import x10.util.*;
% class Redund[T] {
%   val list = new ArrayList[T]();
%   var size : Int = 0;
%~~vis
\begin{xten}
def add(x:T) { // Incorrect
  this.list.add(x);
  this.size = this.size + 1;
}
\end{xten}
%~~siv
%}
%~~neg
Then two activities simultaneously adding elements to the same \xcd`r` could break the
invariant.  Suppose that \xcd`r` starts out empty.  Let the first activity
perform the \xcd`list.add`, and compute \xcd`this.size+1`, which is 1, but not store it
back into \xcd`this.size` yet.  
(At this point, \xcd`r.list.size()==1` and \xcd`r.size==0`; the invariant
expression is false, but, as the first call to \xcd`r.add()` is active, the
invariant does not need to be true -- it only needs to be true when the
call finishes.)
Now, let the second activity do its call to
\xcd`add` to completion, which finishes with \xcd`r.size==1`.  
(As before, the invariant expression is false, but a call to \xcd`r.add()` is
still active, so the invariant need not be true.)
Finally, let
the first activity finish, which assigns the \xcd`1` computed before back into
\xcd`this.size`.  At the end, there are two elements in \xcd`r.list`, but
\xcd`r.size==1`. Since there are no calls to \xcd`r.add()` active, the
invariant must be true, but it is not.

In this case, the invariant can be maintained by making the increment atomic.
Doing so forbids that sequence of events; the \xcd`atomic` block cannot be
stopped partway.  
%~~gen
% package Activities.Atomic.Redund.Two;
% import x10.util.*;
% class Redund[T] {
%   val list = new ArrayList[T]();
%   var size : Int = 0;
%~~vis
\begin{xten}
def add(x:T) { 
  this.list.add(x);
  atomic { this.size = this.size + 1; }
}
\end{xten}
%~~siv
%}
%~~neg



\subsection{Unconditional atomic blocks}
The simplest form of an atomic block is the {\em unconditional
atomic block}:

\begin{grammar}
Statement \: AtomicStatement \\
AtomicStatement \: \xcd"atomic"  Statement \\
MethodModifier \: \xcd"atomic" \\
\end{grammar}

For the sake of efficient implementation \XtenCurrVer{} requires
that the atomic block be {\em analyzable}, that is, the set of
locations that are read and written by the \grammarrule{BlockStatement} are
bounded and determined statically.\footnote{A static bound is a constant
that depends only on the program text, and is independent 
of any runtime parameters. }
The exact algorithm to be used by
the compiler to perform this analysis will be specified in future
versions of the language.
\tbd{}

Such a statement is executed by an activity as if in a single step
during which all other concurrent activities in the same place are
blocked. If execution of the statement may throw an exception, it is
the programmer's responsibility to wrap the atomic block within a
\xcd"try"/\xcd"finally" clause and include undo code in the finally
clause. Thus the \xcd"atomic" statement only guarantees atomicity on
successful execution, not on a faulty execution.


We allow methods of an object to be annotated with \xcd"atomic". Such
a method is taken to stand for a method whose body is wrapped within an
\xcd"atomic" statement.

Atomic blocks are closely related to non-blocking synchronization
constructs \cite{herlihy91waitfree}, and can be used to implement 
non-blocking concurrent algorithms.

\begin{staticrule*}
In \xcd"atomic S", \xcd"S" may include calls to \xcd`safe` methods, and use of
sequential control structures.

It may {\em not} include an \xcd"async" activity (such as creation
of a \Xcd{future}).

It may {\em not} include any statement that may potentially block at
runtime (\eg, \xcd"when", \xcd"force" operations, \xcd"next"
operations on clocks, \xcd"finish"). 

It may {\em not} include any \xcd`at` expressions or
statements. (Hence all locations accessed in the atomic block must
belong to the current place.)
\index{locality condition}\label{LocalityCondition} 

\end{staticrule*}

The compiler checks for this condition by checking whether the statement
could be the body of a \xcd"void" method annotated with \xcd"safe" at
that point in the code (\Sref{SafeAnnotation}).

\paragraph{Consequences.}
Note an important property of an (unconditional) atomic block:

\begin{eqnarray}
 \mbox{\xcd"atomic \{s1; atomic s2\}"} &=& \mbox{\xcd"atomic \{s1; s2\}"}
\end{eqnarray}

Atomic blocks do not introduce deadlocks.    They may exhibit all the bad
behavior of sequential programs, including throwing exceptions and running
forever, but they are guaranteed not to deadlock.


\subsubsection{Example}

The following class method implements a (generic) compare and swap (CAS) operation:


%~~gen
% package Activities.And.Protein;
% class CASSizer{
%~~vis
\begin{xten}
var target:Object = null;
public atomic def CAS(old1: Object, new1: Object): Boolean {
   if (target.equals(old1)) {
     target = new1;
     return true;
   }
   return false;
}
\end{xten}
%~~siv
%}
%~~neg

\subsection{Conditional atomic blocks}

Conditional atomic blocks allow the activity to wait for some condition to be
satisfied before executing an atomic block. For example, consider a
\xcd`Redund` class holding a list \xcd`r.list` and, redundantly, its length
\xcd`r.size`.  A \xcd`pop` operation will delay until the \xcd`Redund` is
nonempty, and then remove an element and update the length.  
%~~gen
% package Activities.Condato.Example.Not.A.Tree;
% import x10.util.*;
% class Redund[T] {
% val list = new ArrayList[T]();
% var size : Int = 0;
%~~vis
\begin{xten}
def pop():T {
  var ret : T;
  when(size>0) {
    ret = list.removeAt(0);
    size --;
    }
  return ret;
}
\end{xten}
%~~siv
% }
%~~neg


The execution of the test is atomic with the execution of the block.  This is
important; it means that no other activity can sneak in and make the condition
be false before the block is executed.  In this example, two \xcd`pop`s
executing on a list with one element would work properly. Without the
conditional atomic block -- even doing the decrement atomically -- one call to
\xcd`pop` could pass the \xcd`size>0` guard; then the other call could run to
completion (removing the only element of the list); then, when the first call
proceeds, its \xcd`removeAt` will fail.  

Note that \xcd`if` would not work here.  
\xcd`if(size>0) atomic{size--; return list.removeAt(0);}` allows another
activity to act between the test and the atomic block.  
And 
\xcd`atomic{ if(size>0) {size--; ret = list.removeAt(0);}}` 
does not wait for \xcd`size>0` to become true.


Conditional atomic blocks are of the form:

\begin{grammar}
Statement \:  WhenStatement \\
WhenStatement \:  \xcd"when" \xcd"(" Expression \xcd")" Statement \\
            \| WhenStatement \xcd"or" \xcd"(" Expression \xcd")" Statement 
\end{grammar}

In such a statement the one or more expressions are called {\em
guards} and must be \xcd"Boolean" expressions. The statements are the
corresponding {\em guarded statements}.  

An activity executing such a statement suspends until such time as any
one of the guards is true in the current state. In that state, the
statement corresponding to the first guard that is true is executed.
The checking of the guards and the execution of the corresponding
guarded statement is done atomically. 

\Xten{} does not guarantee that a conditional atomic block
will execute if its condition holds only intermittently. For, based on
the vagaries of the scheduler, the precise instant at which a
condition holds may be missed. Therefore the programmer is advised to
ensure that conditions being tested by conditional atomic blocks are
eventually stable, \ie, they will continue to hold until the block
executes (the action in the body of the block may cause the condition
to not hold any more).

%%Fourth, \Xten{} does not guarantees only {\em weak fairness} when executing
%%conditional atomic blocks. Let $c$ be the guard of some conditional
%%atomic block $A$. $A$ is required to make forward progress only if
%%$c$ is {\em eventually stable}. That is, any execution $s_1, s_2,
%%\ldots$ of the program is considered illegal only if there is a $j$
%%such that $c$ holds in all states $s_k$ for $k > j$ and in which $A$
%%does not execute. Specifically, if the system executes in such a way
%%that $c$ holds only intermmitently (that is, for some state in which
%%$c$ holds there is always a later state in which $c$ does not hold),
%%$A$ is not required to be executed (though it may be executed).

\begin{rationale}
The guarantee provided by \xcd"wait"/\xcd"notify" in \java{} is no
stronger. Indeed conditional atomic blocks may be thought of as a
replacement for \java's wait/notify functionality.
\end{rationale} 


The statement \xcd"when (true) S" is
behaviorally identical to \xcd"atomic S": it never suspends.

\begin{staticrule*}
For the sake of efficient implementation certain restrictions are
placed on the guards and statements in a conditional atomic
block. 
\end{staticrule*}

Guards are statically required not to have side-effects, not to spawn
asynchronous activities (as for the \xcd`sequential` qualifier on methods) and
to have a statically determinable upper bound on their execution (as for the
\xcd`nonblocking` qualifier on methods).

The body of a \xcd"when" statement must satisfy the conditions
for the body of an \xcd"atomic" block.

Note that this implies that guarded statements are required to be {\em
flat}, that is, they may not contain conditional atomic blocks. (The
implementation of nested conditional atomic blocks may require
sophisticated operational techniques such as rollbacks.)


\begin{example}
The following class shows how to implement a bounded buffer of size
$1$ in \Xten{} for repeated communication between a sender and a
receiver.  The call \xcd`buf.send(ob)` waits until the buffer has space, and
then puts \xcd`ob` into it.  Dually, \xcd`buf.receive()` waits until the
buffer has something in it, and then returns that thing.


%~~gen
% package Activities;
%~~vis
\begin{xten}
class OneBuffer[T] {
  var datum: T;
  def this(t:T) { this.datum = t; this.filled = true; }
  var filled: Boolean;
  public def send(v: T) {
    when (!filled) {
      this.datum = v;
      this.filled = true;
    }
  }
  public def receive(): T {
    when (filled) {
      v: T = datum;
      filled = false;
      return v;
    }
  }
}
\end{xten}
%~~siv
%
%~~neg
\end{example}

	
\chapter{Clocks}\label{XtenClocks}\index{clocks}

Many concurrent algorithms proceed in phases: in phase {$k$}, several
activities work independently, but synchronize together before proceeding on
to phase {$k+1$}. X10 supports this communication structure (and many
variations on it) with a generalization of barriers 
called {\em clocks}. Clocks are designed so that programs which follow a
simple syntactic discipline will not have either deadlocks or race conditions.


The following minimalist example of clocked code has two worker activities A
and B, and three phases. In the first phase, each worker activity says its
name followed by 1; in the second phase, by a 2, and in the third, by a 3.  
So, if \xcd`say` prints its argument, 
\xcd`A-1 B-1 A-2 B-2 B-3 A-3`
would be a legitimate run of the program, but
\xcd`A-1 A-2 B-1 B-2 A-3 B-3`
(with \xcd`A-2` before \xcd`B-1`) would not.

The program creates a clock \xcd`cl` to manage the phases.  Each participating
activity does
the work of its first phase, and then executes \xcd`next;` to signal that it
is finished with that work. \xcd`next;` is blocking, and causes the participant to
wait until all participant have finished with the phase -- as measured by the
clock \xcd`cl` to which they are both registered.  
Then they do the second phase, and another \xcd`next;` to make sure that
neither proceeds to the third phase until both are ready.  This example uses
\xcd`finish` to wait for both particiants to finish.  The parent thread is also
registered on the clock just as the particiants are, and executes \xcd`next;next;`
to run through the phases.


%%TODO -- put the 'atomic' back in when that's legal.

%~~gen
%package Clocks.For.Spock;
%class ClockEx {
%  static def say(s:String) = 
% { /*atomic{x10.io.Console.OUT.println(s);}*/ }
%  public static def main(argv:Rail[String]) {
%~~vis
\begin{xten}
    finish async{
      val cl = Clock.make();
      async clocked(cl) {// Activity A
        say("A-1");
        next;
        say("A-2");
        next;
        say("A-3"); 
      }// Activity A

      async clocked(cl) {// Activity B
        say("B-1");
        next;
        say("B-2");
        next;
        say("B-3"); 
      }// Activity B
    }
\end{xten}
%~~siv
%  }
% }
%~~neg

This chapter describes the syntax and semantics of clocks and
statements in the language that have parameters of type \xcd"Clock". 

The key invariants associated with clocks are as follows.  At any
stage of the computation, a clock has zero or more {\em registered}
activities. An activity may perform operations only on those clocks it
is registered with (these clocks constitute its {\em clock set}). 
An attempt by an activity to operate on a clock it is not registered with
will cause a 
\xcd"ClockUseException"\index{clock!ClockUseException}. 
to be thrown.  
An activity is registered with zero or more clocks when it is created.
During its lifetime the only additional clocks it is registered with
are exactly those that it creates. In particular it is not possible
for an activity to register itself with a clock it discovers by
reading a data structure.

The primary operations that an activity \xcd`a` may perform on a clock \xcd`c`
that it is registered upon are: 
\begin{itemize}
\item It may spawn and simultaneously  {\em register} a new activity on
      \xcd`c`, with the statement       \xcd`async clocked(c){S}`.
\item It may {\em unregister} itself from \xcd`c`, with \xcd`c.drop()`.  After
      doing so, it can no longer use most primary operations on \xcd`c`.
\item It may {\em resume} the clock, with \xcd`c.resume()`, indicating that it
      has finished with the current phase associated with \xcd`c` and is ready
      to move on to the next one.
\item It may {\em wait} on the clock, with \xcd`c.next()`.  This first does
      \xcd`c.resume()`, and then blocks the current activity until the start
      of the next phase, \viz, until all other activities registered on that
      clock have called \xcd`c.resume()`.
\item It may {\em block} on all the clocks it is registered with
      simultaneously, by the command \xcd`next;`.  This, in effect, calls
      \xcd`c.next()` simultaneously 
      on all clocks \xcd`c` that the current activity is registered with.
\item Other miscellaneous operations are available as well; see the
      \xcd`Clock` API.
\end{itemize}

%%CLOCK%% An activity may perform the following operations on a clock \xcd"c".
%%CLOCK%% It may {\em unregister} with \xcd"c" by executing \xcd"c.drop();".
%%CLOCK%% After this, it may perform no further actions on \xcd"c"
%%CLOCK%% for its lifetime. It may {\em check} to see if it is unregistered on a
%%CLOCK%% clock. It may {\em register} a newly forked activity with \xcd"c".
%%CLOCK%% %% It may {\em post} a statement \xcd"S" for completion in the current phase
%%CLOCK%% %% of \xcd"c" by executing the statement \xcd"now(c) S". 
%%CLOCK%% Once registered and "active" (see below), it may also perform the following operations.
%%CLOCK%% It may {\em resume} the clock by executing \xcd"c.resume();". This
%%CLOCK%% indicates to \xcd"c" that it has finished posting all statements it
%%CLOCK%% wishes to perform in the current phase. Finally, it may {\em block}
%%CLOCK%% (by executing \xcd"next;") on all the clocks that it is registered
%%CLOCK%% with. (This operation implicitly \xcd"resume"'s all clocks for the
%%CLOCK%% activity.) It will resume from this statement only when all these
%%CLOCK%% clocks are ready to advance to the next phase.

%%CLOCK%% A clock becomes ready to advance to the next phase when every activity
%%CLOCK%% registered with the clock has executed at least one \xcd"resume"
%%CLOCK%% operation on that clock and all statements posted for completion in
%%CLOCK%% the current phase have been completed.

%%OLIVIER-DENIES%% Though clocks introduce a blocking statement (\xcd"next") an important
%%OLIVIER-DENIES%% property of \Xten{} is that clocks -- when used with the \xcd`next;` {\em
%%OLIVIER-DENIES%%   statement} only, without the \xcd`c.next()` method call -- cannot introduce
%%OLIVIER-DENIES%% deadlocks. That is, the system cannot reach a quiescent state (in which no
%%OLIVIER-DENIES%% activity is progressing) from which it is unable to progress. For, before
%%OLIVIER-DENIES%% blocking each activity resumes all clocks it is registered with. Thus if a
%%OLIVIER-DENIES%% configuration were to be stuck (that is, no activity can progress) all clocks
%%OLIVIER-DENIES%% will have been resumed. But this implies that all activities blocked on
%%OLIVIER-DENIES%% \xcd"next" may continue and the configuration is not stuck. The only other
%%OLIVIER-DENIES%% possibility is that an activity may be stuck on \xcd"finish". But the
%%OLIVIER-DENIES%% interaction rule between \xcd"finish" and clocks
%%OLIVIER-DENIES%% (\Sref{sec:finish:clock-rule}) guarantees that this cannot cause a cycle in
%%OLIVIER-DENIES%% the wait-for graph. A more rigorous proof may be found in \cite{X10-concur05}.

\section{Clock operations}\label{sec:clock}
There are two language constructs for working with clocks. 
\xcd`async clocked(cl) S` starts a new activity registered on one or more
clocks.  \xcd`next;` blocks the current activity until all the activities
sharing clocks with it are ready to proceed to the next clock phase. 
Clocks are objects, and have a number of useful methods on them as well.

\subsection{Creating new clocks}\index{clock!creation}\label{sec:clock:create}

Clocks are created using a factory method on \xcd"x10.lang.Clock":


%~~gen
% package Clocks.For.Spocks;
%class Clockuser {
% def example() {
%~~vis
\begin{xten}
val c: Clock = Clock.make();
\end{xten}
%~~siv
%}}
%~~neg

%%CLOCKVAR%% \eat{All clocked variables are implicitly \xcd`val`. The initializer for a
%%CLOCKVAR%% local variable declaration of type \xcd"Clock" must be a new clock
%%CLOCKVAR%% expression. Thus \Xten{} does not permit aliasing of clocks.
%%CLOCKVAR%% Clocks are created in the place global heap and hence outlive the
%%CLOCKVAR%% lifetime of the creating activity.  Clocks are structs, hence may be freely
%%CLOCKVAR%% copied from place to 
%%CLOCKVAR%% place. (Clock instances typically contain references to mutable state
%%CLOCKVAR%% that maintains the current state of the clock.)
%%CLOCKVAR%% }

The current activity is automatically registered with the newly
created clock.  It may deregister using the \xcd"drop" method on
clocks (see the documentation of \xcd"x10.lang.Clock"). All activities
are automatically deregistered from all clocks they are registered
with on termination (normal or abrupt).

\subsection{Registering new activities on clocks}
\index{clock!clocked statements}\label{sec:clock:register}

The statement 

%~~gen
%package Clocks.For.Jocks;
%class Qlocked{
%static def S():void{}
%static def flock() { 
% val c1 = Clock.make(), c2 = Clock.make(), c3 = Clock.make();
%~~vis
\begin{xten}
  async clocked (c1, c2, c3) S
\end{xten}
%~~siv
%();
%}}
%~~neg
starts a new activity, initially registered with
clocks \xcd`c1`, \xcd`c2`, and \xcd`c3`, and  running \xcd`S`. The activity running this code must
be registered on those clocks. 
Violations of these conditions are punished by the throwing of a
\xcd"ClockUseException"\index{clock!ClockUseException}. 

% An activity may transmit only those clocks that are registered with and
% has not quiesced on (\Sref{resume}). 
% A \xcd"ClockUseException"\index{clock!ClockUseException} is
%thrown if (and when) this condition is violated.

If an activity {$a$} that has executed \xcd`c.resume()` then starts a
new activity {$b$} also registered on \xcd`c` (\eg, via \Xcd{async
clocked(c) S}), the new activity {$b$} starts out having also resumed
\xcd`c`, as if it too had executed \xcd`c.resume()`.  
%~~gen
% package Clocks.For.Jocks.In.Clicky.Smocks;
%class Example{
%static def S():void{}
%static def a_phase_two():void{}
%static def b_phase_two():void{}
%static def example() {
%~~vis
\begin{xten}
//a
val c = Clock.make();
c.resume();
async clocked(c) {
  // b
  c.next();
  b_phase_two();
}
c.next();
a_phase_two();
\end{xten}
%~~siv
%} }
%~~neg
In the proper execution, {$a$} and {$b$} both perform
\xcd`c.next()` and then their phase-2 actions.  
However, if {$b$} were not
initially in the resume state for \xcd`c`, there would be a race condition;
{$b$} could perform \xcd`c.next()` and proceed to \xcd`b_phase_two`
before {$a$} performed \xcd`c.next()`.


An activity may check that it is registered on a clock \xcd"c" by
%~~exp~~`~~`~~c:Clock ~~
the predicate \xcd`c.registered()`


\begin{note}
\Xten{} does not contain a ``register'' operation that would allow an activity
to discover a clock in a datastructure and register itself on it. Therefore,
while a clock \xcd`c` may be stored in a data structure by one activity
\xcd`a` and read from it by another activity \xcd`b`, \xcd`b` cannot do much
with \xcd`c` unless it is already registered with it.  In particular, it
cannot register itself on \xcd`c`, and, lacking that registration, cannot
register a sub-activity on it with \xcd`async clocked(c) S`.
\end{note}


\subsection{Resuming clocks}\index{clock!resume}\label{resume}\label{sec:clock:resume}
\Xten{} permits {\em split phase} clocks. An activity may wish
to indicate that it has completed whatever work it wishes to perform
in the current phase of a  clock \xcd"c" it is registered with, without
suspending altogether. It may do so  by executing 
%~~exp~~`~~`~~c:Clock ~~
\xcd`c.resume()`.



An activity may invoke \xcd`resume()` only on a clock it is registered with,
and has not yet dropped (\Sref{sec:clock:drop}). A
\xcd"ClockUseException"\index{clock!ClockUseException} is thrown if this
condition is violated. Nothing happens if the activity has already invoked a
\xcd"resume" on this clock in the current phase.
%%OLIVIER-DENIES%%  Otherwise, \xcd`c.resume()`
%%OLIVIER-DENIES%% indicates that the activity will not transmit \xcd"c" to an 
%%OLIVIER-DENIES%% \xcd"async" (through a \xcd"clocked" clause), 
%%OLIVIER-DENIES%% until it terminates, drops \xcd"c" or executes a \xcd"next".

%%OLIVIER-DENIES%% \bard{The following is under investigation}
%%OLIVIER-DENIES%% \begin{staticrule*}
%%OLIVIER-DENIES%% It is a static error if any activity has a potentially
%%OLIVIER-DENIES%% live execution path from a \xcd"resume" statement on a clock \xcd"c"
%%OLIVIER-DENIES%% to a
%%OLIVIER-DENIES%% %\xcd"now" or
%%OLIVIER-DENIES%% async spawn statement (which registers the new
%%OLIVIER-DENIES%% activity on \xcd"c") unless the path goes through a \xcd"next"
%%OLIVIER-DENIES%% statement. (A \xcd"c.drop()" following a \xcd"c.resume()" is legal,
%%OLIVIER-DENIES%% as is \xcd"c.resume()" following a \xcd"c.resume()".)
%%OLIVIER-DENIES%% \end{staticrule*}

\subsection{Advancing clocks}\index{clock!next}\label{sec:clock:next}
An activity may execute the statement
\begin{xten}
next;
\end{xten}

\noindent 
Execution of this statement blocks until all the clocks that the
activity is registered with (if any) have advanced. (The activity
implicitly issues a \xcd"resume" on all clocks it is registered
with before suspending.)

\xcd`next;` may be thought of as calling \xcd`c.next()` in parallel for all
clocks that the current activity is registered with.  (The parallelism is
conceptually important: if activities {$a$} and {$b$} are both
registered on clocks \xcd`c` and \xcd`d`, and {$a$} executes
\xcd`c.wait(); d.wait()` while {$b$} executes \xcd`d.wait(); c.wait()`,
then the two will deadlock.  However, if the two clocks are waited on in
parallel, as \xcd`next;` does, {$a$} and {$b$} will not deadlock.)

Equivalently, \xcd`next;` sequentially calls \xcd`c.resume()` for each
registered clock \xcd`c`, in arbitrary order, and then \xcd`c.wait()` for each
clock, again in arbitrary order.  


%%OLIVIER-DENIES%% An \Xten{} computation is said to be {\em quiescent} on a clock
%%OLIVIER-DENIES%% \xcd"c" if each activity registered with \xcd"c" has resumed \xcd"c".
%%OLIVIER-DENIES%% Note that once a computation is quiescent on \xcd"c", it will remain
%%OLIVIER-DENIES%% quiescent on \xcd"c" forever (unless the system takes some action),
%%OLIVIER-DENIES%% since no other activity can become registered with \xcd"c".  That is,
%%OLIVIER-DENIES%% quiescence on a clock is a {\em stable property}.

%%OLIVIER-DENIES%% Once the implementation has detected quiescence on \xcd"c", the system
%%OLIVIER-DENIES%% marks all activities registered with \xcd"c" as being able to progress
%%OLIVIER-DENIES%% on \xcd"c". 
%%OLIVIER-DENIES%% 
An activity blocked on \xcd"next" resumes execution once
it is marked for progress by all the clocks it is registered with.

\subsection{Dropping clocks}\index{clock!drop}\label{sec:clock:drop}
%~~exp~~`~~`~~ c:Clock~~
An activity may drop a clock by executing \xcd`c.drop()`.



\noindent{} The activity is no longer considered registered with this
clock.  A \xcd"ClockUseException" is thrown if the activity has
already dropped \xcd"c".

\section{Deadlock Freedom}

In general, programs using clocks can deadlock, just as programs using loops
can fail to terminate.  However, programs written with a particular syntactic
discipline {\em are} guaranteed to be deadlock-free, just as programs which
use only bounded loops are guaranteed to terminate.  The syntactic discipline
is: 
\begin{itemize}
\item The \xcd`next()` {\bf method} may not be called on any clock. (The
      \xcd`next;` statement is allowed.)
\item Inside of \xcd`finish{S}`, all clocked \xcd`async`s must be in the scope
      an unclocked \xcd`async`.  
\end{itemize}


The second clause prevents the following deadlock.  
%~~gen
% package Clocks.Finish.Hates.Clocks;
% class Example{
% def example() {
%~~vis
\begin{xten}
val c:Clock = Clock.make();
async clocked(c) {                // (A) 
      finish async clocked(c) {   // (B) Violates clause 2
            next;                 // (Bnext)
      }
      next;                       // (Anext)
}
\end{xten}
%~~siv
% } } 
%~~neg
\xcd`(A)`, first of all, waits for the \xcd`finish` containing \xcd`(B)` to
finish.  
\xcd`(B)` will execute its \xcd`next` at \xcd`(Bnext)`, and then wait for all
other activities registered on \xcd`c` to execute their \xcd`next`s.
However, \xcd`(A)` is registered on \xcd`c`.  So, \xcd`(B)` cannot finish
until \xcd`(A)` has proceeded to \xcd`(Anext)`, and \xcd`(A)` cannot proceed
until \xcd`(B)` finishes. Thus, this causes deadlock.


\section{Program equivalences}
From the discussion above it should be clear that the following
equivalences hold:

\begin{eqnarray}
 \mbox{\xcd"c.resume(); next;"}       &=& \mbox{\xcd"next;"}\\
 \mbox{\xcd"c.resume(); d.resume();"} &=& \mbox{\xcd"d.resume(); c.resume();"}\\
 \mbox{\xcd"c.resume(); c.resume();"} &=& \mbox{\xcd"c.resume();"}
\end{eqnarray}

Note that \xcd"next; next;" is not the same as \xcd"next;". The
first will wait for clocks to advance twice, and the second
once.  

%\notinfouro{\subsection{Implementation Notes}
Clocks may be implemented efficiently with message passing, e.g.{} by
using short-circuit ideas in \cite{SaraswatPODC88}.  Recall that every
activity is spawned with references to a fixed number of clocks. Each
reference should be thought of as a global pointer to a location in
some place representing the clock. (We shall discuss a further
optimization below.) Each clock keeps two counters: the total number
of outstanding references to the clock, and the number of activities
that are currently suspended on the clock.

When an activity $A$ spawns another activity $B$ that will reference a
clock $c$ referenced by $A$, $A$ {\em splits} the reference by sending
a message to the clock. Whenever an activity drops a reference to a
clock, or suspends on it, it sends a message to the clock. 

The optimization is that the clock can be represented in a distributed
fashion. Each place keeps a local counter for each clock that is
referenced by an activity in that place. The global location for the
clock simply keeps track of the places that have references and that
are quiescent. This can reduce the inter-place message traffic
significantly.
}
%\notinfouro{\section{Clocked types}\index{types!clocked}

We allow types to specify clocks, via a {\cf clocked(c)} modifier,
e.g.{}

\begin{x10}
  clocked(c) int r;
\end{x10}

This declaration asserts that {\cf r} is accessible
(readable/writable) only by those statements that are clocked on {\cf
c}. Thus propagation of the clock provides some control over the
``visibility'' of {\cf r}.

The declaration 

\begin{x10}
  clocked(c) final int l = r;
\end{x10}

\noindent asserts additionally that in each clock instant {\cf l} is final, 
i.e.{} the value of {\cf l} may be reset at the beginning of each phase
of {\tt c} but stays constant during the phase.

This statement terminates when the computation of {\tt r} has
terminated and the assignment has been performed.

\todo{Generalize the syntax so that aggregate variables can be clocked with an aggregate clock of the same shape.}

\subsection{Clocked assignment}\index{assignment!clocked}
We expect that most arrays containing application data will be
declared to be {\cf clocked final}. We support this very powerful type
declaration by providing a new statement:
{\footnotesize
\begin{verbatim}
  next(c) l = r; 
\end{verbatim}}


\noindent 
for a variable $l$ declared to be clocked on $c$. The statement
assigns $r$ to the {\em next} value of $l$. There may be multiple such
assignments before the clock advances. The last such assignment
specifies the value of the variable that will be visible after the
clock has advanced.  If the variable is {\cf clocked final} it is
guaranteed that {\em all} readers of the variable throughout this
phase will see the value $r$.

The expression {\tt r} is implicitly treated as {\tt now(c) r}. That
is, the clock {\tt c} will not advance until the computation of {\tt r} has
terminated.

}
%\notinfouro{\section{Examples}
\todo{Bring in other examples from Concur paper.}
Consider the core of the ASCI Benchmark Sweep3D program for computing
solutions to mass transport problems.

In a nutshell the core computation is a triply nested sequential loop
in which the value of a variable in the current iteration is dependent
on the values of neighboring variables in a past iteration. Such a
problem can be parallelized through pipelining. One visualizes a
diagonal wavefront sweeping through the array. An MPI version of the
program may be described as follows. There is a two dimensional grid
of processors which performs the following computation
repeatedly. Each processor synchronously receives a value from the
processor to its west, then to its north, then computes some function
of these values and computes a new value to be sent to the processor
to its east and then to its south.  Ignoring the behavior of the
boundary processors for the moment such a computation may be described
by the following \Xten{} program:

\begin{x10}
region R = [1..n0,1..m0];
clock[R] W,N;
clock(W) final double [cyclic(R)] A; 
for (int t : 1..TMax) \{
  ateach( i,j:A) 
    clock (W[i-1,j],N[i,j-1],W[i,j],N[i,j]) \{
      double west = now (W[i-1,j]) future\{A[i-1,j]\}; 
      W[i-1,j].continue();           
      double north = now (N[i,j-1]) future\{A[i,j-1]\}; 
      N[i,j-1].continue();
      next(W[i,j]) A[i,j] = compute(west, north);
      next W[i-1,j],N[i,j-1],W[i,j],N[i,j];
  \}
\}
\end{x10}
}

\section{Clocked Finish}
\index{finish!clocked}
\index{async!clocked}
\index{clocked!finish}
\index{clocked!async}
\label{ClockedFinish}

In the most common case of a single clock coordinating a few behaviors, X10
allows coding with an implicit clock.  \xcd`finish` and \xcd`async` statements
may be qualified with \xcd`clocked`.  

A \xcd`clocked finish` introduces a new clock.  It executes its body in the
usual way that a \xcd`finish` does--- except that, when its body completes,
the activity executing the \xcd`clocked finish` drops the clock, while it
waits for asynchronous spawned \xcd`async`s to terminate.  

A \xcd`clocked async` registers its async with the implicit clock of
the surrounding \xcd`clocked finish`.   

Both the \xcd`clocked finish` and \xcd`clocked async` may use the \xcd`next`
statement to advance implicit clock.  Since the implicit clock is not
available in a variable, it cannot be manipulated directly. (If you want to
manipulate the clock directly, use an explicit clock.)

The following code starts two activities, each of which perform their first
phase, wait for the other to finish phase 1, and then perform their second
phase.  
%~~gen
%package Clocks.ClockedFinish;
%class Example{
%static def phase(String, Int) {}
%def example() {
%~~vis
\begin{xten}
clocked finish {
  clocked async {
     phase("A", 1);
     next;
     phase("A", 2);
  }
  clocked async {
     phase("B", 1);
     next;
     phase("B", 2);
  }
}
\end{xten}
%~~siv
%}}
%~~neg


\index{finish!nested clocked}
\index{clocked finish!nested}

Clocked finishes may be nested.  The inner \xcd`clocked finish` operates in a
single phase of the outer one.  
	
\chapter{Local and Distributed Arrays}\label{XtenArrays}\index{array}

\Xcd{Array}s provide indexed access to data at a single \Xcd{Place}, {\em via}
\Xcd{Point}s---indices of any dimensionality. \Xcd{DistArray}s is similar, but
spreads the data across multiple \xcd`Place`s, {\em via} \Xcd{Dist}s.  
We refer to arrays either sort as ``general arrays''.  


This chapter provides an overview of the \Xcd{x10.array} classes \Xcd{Array}
and \Xcd{DistArray}, and their supporting classes \Xcd{Point}, \Xcd{Region}
and \Xcd{Dist}.  


\section{Points}\label{point-syntax}
\index{point}
\index{point!syntax}


General arrays are indexed by \xcd`Point`s, which are $n$-dimensional tuples of
integers.  The \xcd`rank`
property of a point gives its dimensionality.  Points can be constructed from
integers or \xcd`Array[Int](1)`s by
the \xcd`Point.make` factory methods:
%~~gen
% package Arrays.Points.Example1;
% class Example1 {
% def example1() {
%~~vis
\begin{xten}
val origin_1 : Point{rank==1} = Point.make(0);
val origin_2 : Point{rank==2} = Point.make(0,0);
val origin_5 : Point{rank==5} = Point.make([0,0,0,0,0]);
\end{xten}
%~~siv
% } } 
%~~neg

%~~type~~`~~`~~ ~~
There is an implicit conversion from \xcd`Array[Int](1)` to 
%~~type~~`~~`~~ ~~
\xcd`Point`, giving
a convenient syntax for constructing points: 

%~~gen
% package Arrays.Points.Example2;
% class Example{
% def example() {
%~~vis
\begin{xten}
val p : Point = [1,2,3];
val q : Point{rank==5} = [1,2,3,4,5];
val r : Point(3) = [11,22,33];
\end{xten}
%~~siv
% } } 
%~~neg

The coordinates of a point are available by subscripting; \xcd`p(i)` is the
\xcd`i`th coordinate of the point \xcd`p`. 
\xcdmath`Point($n$)` is a \Xcd{type}-defined shorthand  for 
\xcdmath`Point{rank==$n$}`.


\section{Regions}\label{XtenRegions}\index{region}
\index{region!syntax}

A region is a set of points of the same rank.  {}\Xten{}
provides a built-in class, \xcd`x10.array.Region`, to allow the
creation of new regions and to perform operations on regions. 
Each region \xcd`R` has a property \xcd`R.rank`, giving the dimensionality of
all the points in it.

%~~gen
% package Arrays.Some.Examples.Fidget.Fidget;
% class Example {
% static def example() {
%~~vis
\begin{xten}
val MAX_HEIGHT=20;
val Null = Region.makeUnit();  // Empty 0-dimensional region
val R1 = 1..100; // 1-dim region with extent 1..100
val R2 = (1..100) as Region(1); // same as R1
val R3 = (0..99) * (-1..MAX_HEIGHT);
val R4 = Region.makeUpperTriangular(10);
val R5 = R4 && R3; // intersection of two regions
\end{xten}
%~~siv
% } } 
%~~neg

The expression \xcdmath`m..n`, for integer expressions \Xcd{m} and \Xcd{n},
evaluates to the rectangular, rank-1 region consisting of the points
$\{$\xcdmath`[m]`, \dots, \xcdmath`[n]`$\}$. If \xcdmath`m` is greater than
\xcdmath`n`, the region \Xcd{m..n} is empty.

%%MAYBE%% A region may be constructed by converting from a rail of
%%MAYBE%% regions (\eg, \xcd`R4` above).
%%MAYBE%% The region constructed from a rail of regions represents the Cartesian product
%%MAYBE%% of the arguments. \Eg, \Xcd{R8} is a region of {$100 \times 16 \times 78$}
%%MAYBE%% points, in
%%MAYBE%% %~s~gen
%%MAYBE%% %package Arrays.Region.RailOfRegions;
%%MAYBE%% %class Example{
%%MAYBE%% %def example(){
%%MAYBE%% %~s~vis
%%MAYBE%% \begin{xten}
%%MAYBE%%   val R8 = [1..100, 3..18, 1..78] as Region(3);
%%MAYBE%% \end{xten}
%%MAYBE%% %~s~siv
%%MAYBE%% %}}
%%MAYBE%% %~s~neg


\index{region!upperTriangular}
\index{region!lowerTriangular}\index{region!banded}

Various built-in regions are provided through  factory
methods on \xcd`Region`.  
\begin{itemize}
\item \Xcd{Region.makeEmpty(n)} returns an empty region of rank \Xcd{n}.
\item \Xcd{Region.makeFull(n)} returns the region containing all points of
      rank \Xcd{n}.  
\item \Xcd{Region.makeUnit()} returns the region of rank 0 containing the
      unique point of rank 0.  It is useful as the identity for Cartesian
      product of regions.
\item \Xcd{Region.makeHalfspace(normal:Point, k:Int)} returns the unbounded
      half-space of rank \Xcd{normal.rank}, consisting of all points \Xcd{p}
      satisfying \xcdmath`p$\cdot$normal $\le$ k`.
\item \Xcd{Region.makeRectangular(min, max)}, where \Xcd{min} and \Xcd{max}
      are \Xcd{Int} rails or valrails of length \Xcd{n}, returns a
      \Xcd{Region(n)} equal to: 
      \xcdmath`[min(0) .. max(0), $\ldots$, min(n-1)..max(n-1)]`.
\item \Xcd{Region.make(regions)} constructs the Cartesian product of the
      \Xcd{Region(1)}s in \Xcd{regions}.
\item \Xcd{Region.makeBanded(size, upper, lower)} constructs the
      banded \Xcd{Region(2)} of size \Xcd{size}, with \Xcd{upper} bands above
      and \Xcd{lower} bands below the diagonal.
\item \Xcd{Region.makeBanded(size)} constructs the banded \Xcd{Region(2)} with
      just the main diagonal.
\item \xcd`Region.makeUpperTriangular(N)` returns a region corresponding
to the non-zero indices in an upper-triangular \xcd`N x N` matrix.
\item \xcd`Region.makeLowerTriangular(N)` returns a region corresponding
to the non-zero indices in a lower-triangular \xcd`N x N` matrix.
\item 
  If \xcd`R` is a region, and \xcd`p` a Point of the same rank, then 
%~~exp~~`~~`~~R:Region, p:Point(R.rank) ~~
  \xcd`R+p` is \xcd`R` translated forwards by 
  \xcd`p` -- the region whose
%~~exp~~`~~`~~r:Point, p:Point(r.rank) ~~
  points are \xcd`r+p` 
  for each \xcd`r` in \xcd`R`.
\item 
  If \xcd`R` is a region, and \xcd`p` a Point of the same rank, then 
%~~exp~~`~~`~~R:Region, p:Point(R.rank) ~~
  \xcd`R-p` is \xcd`R` translated backwards by 
  \xcd`p` -- the region whose
%~~exp~~`~~`~~r:Point, p:Point(r.rank) ~~
  points are \xcd`r-p` 
  for each \xcd`r` in \xcd`R`.

\end{itemize}

All the points in a region are ordered canonically by the
lexicographic total order. Thus the points of the region \xcd`(1..2)*(1..2)`
are ordered as 
\begin{xten}
(1,1), (1,2), (2,1), (2,2)
\end{xten}
Sequential iteration statements such as \xcd`for` (\Sref{ForAllLoop})
iterate over the points in a region in the canonical order.

A region is said to be {\em rectangular}\index{region!convex} if it is of
the form \xcdmath`(T$_1$ * $\cdots$ * T$_k$)` for some set of intervals
\xcdmath`T$_i = $ l$_i$ .. h$_i$ `. Such a
region satisfies the property that if two points $p_1$ and $p_3$ are
in the region, then so is every point $p_2$ between them (that is, it is {\em convex}). 
(Banded and triangular regions are not rectangular.)
The operation
%~~exp~~`~~`~~R:Region ~~
\xcd`R.boundingBox()` gives the smallest rectangular region containing
\xcd`R`.

\subsection{Operations on regions}
\index{region!operations}

Let \xcd`R` be a region. A {\em sub-region} is a subset of \Xcd{R}.
\index{region!sub-region}

Let \xcdmath`R1` and \xcdmath`R2` be two regions whose types establish that
they are of the same rank. Let \xcdmath`S` be another region; its rank is
irrelevant. 

\xcdmath`R1 && R2` is the intersection of \xcdmath`R1` and
\xcdmath`R2`, \viz, the region containing all points which are in both
\Xcd{R1} and \Xcd{R2}.  \index{region!intersection}
%~~exp~~`~~`~~ ~~
For example, \xcd`1..10 && 2..20` is \Xcd{2..10}.


%%NO-DIFF%% \xcdmath`R1 - R2` is the set difference of \xcdmath`R1` and
%%NO-DIFF%% \xcdmath`R2`; \viz, the points in \xcdmath`R1` which are not in
%%NO-DIFF%% \xcdmath`R2`.\index{region!set difference}
%%NO-DIFF%% For example, 
%%NO-DIFF%% ~~exp~~`~~`~~ ~~
%%NO-DIFF%% \xcd`(1..10) - (1..3)` 
%%NO-DIFF%% is 
%%NO-DIFF%% \Xcd{4..10}.

\xcdmath`R1 * S` is the Cartesian product of \xcdmath`R1` and
\xcdmath`S`,  formed by pairing each point in \xcdmath`R1` with every  point in \xcdmath`S`.
\index{region!product}
%~~exp~~`~~`~~ ~~
Thus, \xcd`(1..2)*(3..4)*(5..6)`
is the region of rank \Xcd{3} containing the eight points with coordinates
\xcd`[1,3,5]`, \xcd`[1,3,6]`, \xcd`[1,4,5]`, \xcd`[1,4,6]`,
\xcd`[2,3,5]`, \xcd`[2,3,6]`, \xcd`[2,4,5]`, \xcd`[2,4,6]`.


For a region \xcdmath`R` and point \xcdmath`p` of the same rank,
%~~exp~~`~~`~~R:Region, p:Point{p.rank==R.rank} ~~
\xcd`R+p` 
and
%~~exp~~`~~`~~R:Region, p:Point{p.rank==R.rank} ~~
\xcd`R-p` 
represent the translation of the region
forward 
and backward 
by \xcdmath`p`. That is, \Xcd{R+p} is the set of points
\Xcd{p+q} for all \Xcd{q} in \Xcd{R}, and \Xcd{R-p} is the set of \Xcd{q-p}.

More \Xcd{Region} methods are described in the API documentation.

\section{Arrays}
\index{array}

Arrays are organized data, arranged so that it can be accessed by subscript.
An \xcd`Array[T]` \Xcd{A} has a \Xcd{Region} \Xcd{A.region}, telling which
\Xcd{Point}s are in \Xcd{A}.  For each point \Xcd{p} in \Xcd{A.region},
\Xcd{A(p)} is the datum of type \Xcd{T} associated with \Xcd{p}.  X10
implementations should 
attempt to store \xcd`Array`s efficiently, and to make array element accesses
quick---\eg, avoiding constructing \Xcd{Point}s when unnecessary.

This generalizes the concepts of arrays appearing in many other programming
languages.  A \Xcd{Point} may have any number of coordinates, so an
\xcd`Array` can have, in effect, any number of integer subscripts.  

Indeed, it is possible to write code that works on \Xcd{Array}s regardless 
of dimension.  For example, to add one \Xcd{Array[Int]} \Xcd{src} into another
\Xcd{dest}, 
%~~gen
%package Arrays.Arrays.Arrays.Example;
% class Example{
%~~vis
\begin{xten}
static def addInto(src: Array[Int], dest:Array[Int])
  {src.region == dest.region}
  = {
    for (p in src.region) 
       dest(p) += src(p);
  }
\end{xten}
%~~siv
%}
%~~neg
\noindent
Since \Xcd{p} is a \Xcd{Point}, it can hold as many coordinates as are
necessary for the arrays \Xcd{src} and \Xcd{dest}.

The basic operation on arrays is subscripting: if \Xcd{A} is an \Xcd{Array[T]}
and \Xcd{p} a point with the same rank as \xcd`A.region`, then
%~~exp~~`~~`~~A:Array[Int], p:Point{self.rank == A.region.rank} ~~
\xcd`A(p)`
is the value of type \Xcd{T} associated with point \Xcd{p}.

Array elements can be changed by assignment. If \Xcd{t:T}, 
%~~gen
%package Arrays.Arrays.Subscripting.Is.From.Mars;
%class Example{
%def example[T](A:Array[T], p: Point{rank == A.region.rank}, t:T){
%~~vis
\begin{xten}
A(p) = t;
\end{xten}
%~~siv
%} } 
%~~neg
modifies the value associated with \Xcd{p} to be \Xcd{t}, and leaves all other
values in \Xcd{A} unchanged.

An \Xcd{Array[T]} \Xcd{A} has: 
\begin{itemize}
%~~exp~~`~~`~~A:Array[Int] ~~
\item \xcd`A.region`: the \Xcd{Region} upon which \Xcd{A} is defined.
%~~exp~~`~~`~~A:Array[Int] ~~
\item \xcd`A.size`: the number of elements in \Xcd{A}.
%~~exp~~`~~`~~A:Array[Int] ~~
\item \xcd`A.rank`, the rank of the points usable to subscript \Xcd{A}.
      Identical to \Xcd{A.region.rank}.
\end{itemize}

\subsection{Array Constructors}
\index{array!constructor}

To construct an array whose elements all have the same value \Xcd{init}, call
\Xcd{new Array[T](R, init)}. 
For example, an array of a thousand \xcd`"oh!"`s can be made by:
%~~exp~~`~~`~~ ~~
\xcd`new Array[String](1..1000, "oh!")`.


To construct and initialize an array, call the two-argument constructor. 
\Xcd{new Array[T](R, f)} constructs an array of elements of type \Xcd{T} on
region \Xcd{R}, with \Xcd{A(p)} initialized to \Xcd{f(p)} for each point
\Xcd{p} in \Xcd{R}.  \Xcd{f} must be a function taking a point of rank
\Xcd{R.rank} to a value of type \Xcd{T}.  \Eg, to construct an array of a
hundred zero values, call
%~~exp~~`~~`~~ ~~
\xcd`new Array[Int](1..100, (Point(1))=>0)`. 
To construct a multiplication table, call
%~~exp~~`~~`~~ ~~
\xcd`new Array[Int]((0..9)*(0..9), (p:Point(2)) => p(0)*p(1))`.

Other constructors are available; see the API documentation and
\Sref{sect:ArrayCtors}. 

\subsection{Array Operations}
\index{array!operations on}

The basic operation on \Xcd{Array}s is subscripting.  If \Xcd{A:Array[T]} and 
\xcd`p:Point{rank == A.rank}`, then \Xcd{a(p)} is the value of type \Xcd{T}
appearing at position \Xcd{p} in \Xcd{A}.    The syntax is identical to
function application, and, indeed, arrays may be used as functions.
\Xcd{A(p)} may be assigned to, as well, by the usual assignment syntax
%~~exp~~`~~`~~A:Array[Int], p:Point{rank == A.rank}, t:Int ~~
\xcd`A(p)=t`.
(This uses the application and setting syntactic sugar, as given in \Sref{set-and-apply}.)

Sometimes it is more convenient to subscript by integers.  Arrays of rank 1-4
can, in fact, be accessed by integers: 
%~~gen
%package Arrays.Arrays.wombatsfromlemuria;
%class Example{
%def example(){
%~~vis
\begin{xten}
val A1 = new Array[Int](1..10, 0);
A1(4) = A1(4) + 1;
val A4 = new Array[Int]((1..2)*(1..3)*(1..4)*(1..5), 0);
A4(2,3,4,5) = A4(1,1,1,1)+1;
\end{xten}
%~~siv
%}}
%~~neg



Iteration over an \Xcd{Array} is defined, and produces the \Xcd{Point}s of the
array's region.  If you want to use the values in the array, you have to
subscript it.  For example, you could double every element of an
\Xcd{Array[Int]} by: 
%~~gen
%package Arrays.Arrays.mostly_dire_dreams_tonight;
%class Example{
%def example(A:Array[Int]) {
%~~vis
\begin{xten}
for (p in A) A(p) = 2*A(p);
\end{xten}
%~~siv
%}}
%~~neg



\section{Distributions}\label{XtenDistributions}
\index{distribution}

Distributed arrays are spread across multiple \xcd`Place`s.  
A {\em distribution}, a mapping from a region to a set of places, 
describes where each element of a distributed array is kept.
Distributions are embodied by the class \Xcd{x10.array.Dist}.
This class is \xcd`final` in
{}\XtenCurrVer; future versions of the language may permit
user-definable distributions. 
The {\em rank} of a distribution is the rank of the underlying region, and
thus the rank of every point that the distribution applies to.



%~~gen
%package Arrays.Dists.Examples.Examples.EXAMPLESDAMMIT;
% class Example{
% def example() {
%~~vis
\begin{xten}
val R  <: Region = 1..100;
val D1 <: Dist = Dist.makeBlock(R);
val D2 <: Dist = R -> here;
\end{xten}
%~~siv
% } } 
%~~neg

Let \xcd`D` be a distribution. 
%~~exp~~`~~`~~D:Dist ~~
\xcd`D.region` 
denotes the underlying
region. 
Given a point \xcd`p`, the expression
%~~exp~~`~~`~~ D:Dist, p:Point{p.rank == D.rank}~~
\xcd`D(p)` represents the application of \xcd`D` to \xcd`p`, that is,
the place that \xcd`p` is mapped to by \xcd`D`. The evaluation of the
expression \xcd`D(p)` throws an \xcd`ArrayIndexOutofBoundsException`
if \xcd`p` does not lie in the underlying region.
%%NO-R2D2-CONV%% 
%%NO-R2D2-CONV%% When operated on as a distribution, a region \xcd`R` implicitly
%%NO-R2D2-CONV%% behaves as the distribution mapping each item in \xcd`R` to \xcd`here`
%%NO-R2D2-CONV%% (\ie, \xcd`R->here`, see below). Conversely, when used in a context
%%NO-R2D2-CONV%% expecting a region, a distribution \xcd`D` should be thought of as
%%NO-R2D2-CONV%% standing for \xcd`D.region`.


\subsection{Operations returning distributions}
\index{distribution!operations}

Let \xcd`R` be a region, \xcd`Q` a Sequence of places \{\xcd`p1`, \dots,
\xcd`pk`\} (enumerated in canonical order), and \xcd`P` a place.

\paragraph{Unique distribution} \index{distribution!unique}
%~~exp~~`~~`~~Q:Sequence[Place] ~~
The distribution \xcd`Dist.makeUnique(Q)` is the unique distribution from the
region \xcd`1..k` to \xcd`Q` mapping each point \xcd`i` to \xcd`pi`.

\paragraph{Constant distributions.} \index{distribution!constant}
%~~exp~~`~~`~~R:Region, P:Place ~~
The distribution \xcd`R->P` maps every point in region \xcd`R` to place \xcd`P`, as does
%~~exp~~`~~`~~R:Region, P:Place ~~
\xcd`Dist.makeConstant(R,P)`. 

\paragraph{Block distributions.}\index{distribution!block}
%~~exp~~`~~`~~R:Region ~~
The distribution \xcd`Dist.makeBlock(R)` distributes the elements of \xcd`R`,
in order, over all the places available to the program. 
Let $p$ equal \xcd`|R| div N` and $q$ equal \xcd`|R| mod N`,
where \xcd`N` is the size of \xcd`Q`, and 
\xcd`|R|` is the size of \xcd`R`.  The first $q$ places get
successive blocks of size $(p+1)$ and the remaining places get blocks of
size $p$.

There are other \xcd`Dist.makeBlock` methods capable of controlling the
distribution and the set of places used; see the API documentation.


%%NO-CYCLIC-DIST%%  \paragraph{Cyclic distributions.} \index{distribution!cyclic}
%%NO-CYCLIC-DIST%%  The distribution \xcd`Dist.makeCyclic(R, Q)` distributes the points in \xcd`R`
%%NO-CYCLIC-DIST%%  cyclically across places in \xcd`Q` in order.
%%NO-CYCLIC-DIST%%  
%%NO-CYCLIC-DIST%%  The distribution \xcd`Dist.makeCyclic(R)` is the same distribution as
%%NO-CYCLIC-DIST%%  \xcd`Dist.makeCyclic(R, Place.places)`. 
%%NO-CYCLIC-DIST%%  
%%NO-CYCLIC-DIST%%  Thus the distribution \xcd`Dist.makeCyclic(Place.MAX_PLACES)` provides a 1--1
%%NO-CYCLIC-DIST%%  mapping from the region \xcd`Place.MAX_PLACES` to the set of all
%%NO-CYCLIC-DIST%%  places and is the same as the distribution \xcd`Dist.makeCyclic(Place.places)`.
%%NO-CYCLIC-DIST%%  
%%NO-CYCLIC-DIST%%  \paragraph{Block cyclic distributions.}\index{distribution!block cyclic}
%%NO-CYCLIC-DIST%%  The distribution \xcd`Dist.makeBlockCyclic(R, N, Q)` distributes the elements
%%NO-CYCLIC-DIST%%  of \xcd`R` cyclically over the set of places \xcd`Q` in blocks of size
%%NO-CYCLIC-DIST%%  \xcd`N`.
%%NO-ARB-DIST%%  
%%NO-ARB-DIST%%  \paragraph{Arbitrary distributions.} \index{distribution!arbitrary}
%%NO-ARB-DIST%%  The distribution \xcd`Dist.makeArbitrary(R,Q)` arbitrarily allocates points in {\cf
%%NO-ARB-DIST%%  R} to \xcd`Q`. As above, \xcd`Dist.makeArbitrary(R)` is the same distribution as
%%NO-ARB-DIST%%  \xcd`Dist.makeArbitrary(R, Place.places)`.
%%NO-ARB-DIST%%  
%%NO-ARB-DIST%%  \oldtodo{Determine which other built-in distributions to provide.}
%%NO-ARB-DIST%%  
\paragraph{Domain Restriction.} \index{distribution!restriction!region}

If \xcd`D` is a distribution and \xcd`R` is a sub-region of {\cf
%~~exp~~`~~`~~D:Dist,R :Region{R.rank==D.rank} ~~
D.region}, then \xcd`D | R` represents the restriction of \xcd`D` to
\xcd`R`---that is, the distribution that takes each point \xcd`p` in \xcd`R`
to 
%~~exp~~`~~`~~D:Dist, p:Point{p.rank==D.rank} ~~
\xcd`D(p)`, 
but doesn't apply to any points but those in \xcd`R`.

\paragraph{Range Restriction.}\index{distribution!restriction!range}

If \xcd`D` is a distribution and \xcd`P` a place expression, the term
%~~exp~~`~~`~~ D:Dist, P:Place~~
\xcd`D | P` 
denotes the sub-distribution of \xcd`D` defined over all the
points in the region of \xcd`D` mapped to \xcd`P`.

Note that \xcd`D | here` does not necessarily contain adjacent points
in \xcd`D.region`. For instance, if \xcd`D` is a cyclic distribution,
\xcd`D | here` will typically contain points that differ by the number of
places. 
An implementation may find a
way to still represent them in contiguous memory, \eg, using a
complex arithmetic function to map from the region index to an index
into the array.

%%NO-USER-DIST%%  \subsection{User-defined distributions}\index{distribution!user-defined}
%%NO-USER-DIST%%  
%%NO-USER-DIST%%  Future versions of \Xten{} may provide user-defined distributions, in
%%NO-USER-DIST%%  a way that supports static reasoning.

%%NO-flinking-operations-on-DIST%%  \subsection{Operations on distributions}
%%NO-flinking-operations-on-DIST%%  
%%NO-flinking-operations-on-DIST%%  A {\em sub-distribution}\index{sub-distribution} of \xcd`D` is
%%NO-flinking-operations-on-DIST%%  any distribution \xcd`E` defined on some subset of the region of
%%NO-flinking-operations-on-DIST%%  \xcd`D`, which agrees with \xcd`D` on all points in its region.
%%NO-flinking-operations-on-DIST%%  We also say that \xcd`D` is a {\em super-distribution} of
%%NO-flinking-operations-on-DIST%%  \xcd`E`. A distribution \xcdmath`D1` {\em is larger than}
%%NO-flinking-operations-on-DIST%%  \xcdmath`D2` if \xcdmath`D1` is a super-distribution of
%%NO-flinking-operations-on-DIST%%  \xcdmath`D2`.
%%NO-flinking-operations-on-DIST%%  
%%NO-flinking-operations-on-DIST%%  Let \xcdmath`D1` and \xcdmath`D2` be two distributions with the same rank.  
%%NO-flinking-operations-on-DIST%%  

%%NO-&&-DIST%%  \paragraph{Intersection of distributions.}\index{distribution!intersection}
%%NO-&&-DIST%%  ~~exp~~`~~`~~D1:Dist, D2:Dist{D1.rank==D2.rank} ~~
%%NO-&&-DIST%%  \xcdmath`D1 && D2`, the intersection 
%%NO-&&-DIST%%  of \xcdmath`D1`
%%NO-&&-DIST%%  and \xcdmath`D2`, is the largest common sub-distribution of
%%NO-&&-DIST%%  \xcdmath`D1` and \xcdmath`D2`.

%%NO-overlay-DIST%%  \paragraph{Asymmetric union of distributions.}\index{distribution!union!asymmetric}
%%NO-overlay-DIST%%  ~~exp~~`~~`~~D1:Dist, D2:Dist{D1.rank==D2.rank} ~~
%%NO-overlay-DIST%%  \xcdmath`D1.overlay(D2)`, the asymmetric union of
%%NO-overlay-DIST%%  \xcdmath`D1` and \xcdmath`D2`, is the distribution whose
%%NO-overlay-DIST%%  region is the union of the regions of \xcdmath`D1` and
%%NO-overlay-DIST%%  \xcdmath`D2`, and whose value at each point \xcd`p` in its
%%NO-overlay-DIST%%  region is \xcdmath`D2(p)` if \xcdmath`p` lies in
%%NO-overlay-DIST%%  \xcdmath`D2.region` otherwise it is \xcdmath`D1(p)`.
%%NO-overlay-DIST%%  (\xcdmath`D1` provides the defaults.)

%%NO-flinking-operations-on-DIST%%  \paragraph{Disjoint union of distributions.}\index{distribution!union!disjoint}
%%NO-flinking-operations-on-DIST%%  ~~exp~~`~~`~~D1:Dist, D2:Dist{D1.rank==D2.rank} ~~
%%NO-flinking-operations-on-DIST%%  \xcdmath`D1 || D2`, the disjoint union of
%%NO-flinking-operations-on-DIST%%  \xcdmath`D1`
%%NO-flinking-operations-on-DIST%%  and \xcdmath`D2`, is defined only if the regions of
%%NO-flinking-operations-on-DIST%%  \xcdmath`D1` and \xcdmath`D2` are disjoint. Its value is
%%NO-flinking-operations-on-DIST%%  \xcdmath`D1.overlay(D2)` (or equivalently
%%NO-flinking-operations-on-DIST%%  \xcdmath`D2.overlay(D1)`.  (It is the least
%%NO-flinking-operations-on-DIST%%  super-distribution of \xcdmath`D1` and \xcdmath`D2`.)
%%NO-flinking-operations-on-DIST%%  
%%NO-flinking-operations-on-DIST%%  \paragraph{Difference of distributions.}\index{distribution!difference}
%%NO-flinking-operations-on-DIST%%  \xcdmath`D1 - D2` is the largest sub-distribution of
%%NO-flinking-operations-on-DIST%%  \xcdmath`D1` whose region is disjoint from that of
%%NO-flinking-operations-on-DIST%%  \xcdmath`D2`.
%%NO-flinking-operations-on-DIST%%  
%%NO-flinking-operations-on-DIST%%  
%%What-Is-This-Example%% \subsection{Example}
%%What-Is-This-Example%% \begin{xten}
%%What-Is-This-Example%% def dotProduct(a: Array[T](D), b: Array[T](D)): Array[Double](D) =
%%What-Is-This-Example%%   (new Array[T]([1:D.places],
%%What-Is-This-Example%%       (Point) => (new Array[T](D | here,
%%What-Is-This-Example%%                     (i): Point) => a(i)*b(i)).sum())).sum();
%%What-Is-This-Example%% \end{xten}
%%What-Is-This-Example%% 
%%What-Is-This-Example%% This code returns the inner product of two \xcd`T` vectors defined
%%What-Is-This-Example%% over the same (otherwise unknown) distribution. The result is the sum
%%What-Is-This-Example%% reduction of an array of \xcd`T` with one element at each place in the
%%What-Is-This-Example%% range of \xcd`D`. The value of this array at each point is the sum
%%What-Is-This-Example%% reduction of the array formed by multiplying the corresponding
%%What-Is-This-Example%% elements of \xcd`a` and \xcd`b` in the local sub-array at the current
%%What-Is-This-Example%% place.
%%What-Is-This-Example%% 

\section{Distributed Arrays}
\index{array!distributed}
\index{distributed array}
\index{\Xcd{DistArray}}
\index{DistArray}

Distributed arrays, instances of \xcd`DistArray[T]`, are very much like
\xcd`Array`s, except that they distribute information among multiple
\xcd`Place`s according to a \xcd`Dist` value passed in as a constructor
argument.  For example, the following code creates a distributed array holding
a thousand cells, each initialized to 0.0, distributed via a block
distribution over all places.
%~~gen
% package Arrays.Distarrays.basic.example;
% class Example {
% def example() {
%~~vis
\begin{xten}
val R <: Region = 1..1000;
val D <: Dist = Dist.makeBlock(R);
val da <: DistArray[Float] = DistArray.make[Float](D, (Point(1))=>0.0f);
\end{xten}
%~~siv
%}}
%~~neg




\section{Distributed Array Construction}\label{ArrayInitializer}
\index{distributed array!creation}
\index{\Xcd{DistArray}!creation}
\index{DistArray!creation}

\xcd`DistArray`s are instantiated by invoking one of the \xcd`make` factory
methods of the \xcd`DistArray` class.
A \xcd`DistArray` creation 
must take either an \xcd`Int` as an argument or a \xcd`Dist`. In the first
case,  a distributed array is created over the distribution \xcd`[0:N-1]->here`;
in the second over the given distribution. 

A distributed array creation operation may also specify an initializer
function.
The function is applied in parallel
at all points in the domain of the distribution. The
construction operation terminates locally only when the \xcd`DistArray` has been
fully created and initialized (at all places in the range of the
distribution).

For instance:
%~~gen
% package Arrays.DistArray.Construction.FeralWolf;
% class Example {
% def example() {
%~~vis
\begin{xten}
val data : DistArray[Int]
    = DistArray.make[Int](1..1000->here, ([i]:Point(1)) => i);
val blocked = Dist.makeBlock((1..1000)*(1..1000));
val data2 : DistArray[Int]
    = DistArray.make[Int](blocked, ([i,j]:Point(2)) => i*j);
\end{xten}
%~~siv
% }  }
%~~neg


{}\noindent 
The first declaration stores in \xcd`data` a reference to a mutable
distributed array with \xcd`1000` elements each of which is located in the
same place as the array. The element at \Xcd{[i]} is initialized to its index
\xcd`i`. 

The second declaration stores in \xcd`data2` a reference to a mutable
two-dimensional distributed array, whose coordinates both range from 1 to
1000, distributed in blocks over all \xcd`Place`s, 
initialized with \xcd`i*j`
at point \xcd`[i,j]`.

%%WHY-THIS-EXAMPLE%% In the following 
%%WHY-THIS-EXAMPLE%% %~x~gen
%%WHY-THIS-EXAMPLE%% % package Arrays.DistArrays.FlistArrays.GlistArrays;
%%WHY-THIS-EXAMPLE%% %~x~vis
%%WHY-THIS-EXAMPLE%% \begin{xten}
%%WHY-THIS-EXAMPLE%% val D1:Dist(1) = Dist.makeBlock(1..100);
%%WHY-THIS-EXAMPLE%% val D2:Dist(2) = Dist.makeBlock((1..100)*(-1..1));
%%WHY-THIS-EXAMPLE%% val ints : Array[Int]
%%WHY-THIS-EXAMPLE%%     = Array.make[Int](1000, ((i):Point) => i*i);
%%WHY-THIS-EXAMPLE%% val floats1 : Array[Float]
%%WHY-THIS-EXAMPLE%%     = Array.make[Float](D1, ((i):Point) => i*i as Float);
%%WHY-THIS-EXAMPLE%% val floats2 : Array[Float]
%%WHY-THIS-EXAMPLE%%    = Array.make[Float](D2, ((i,j):Point) => i+j as Float);;
%%WHY-THIS-EXAMPLE%% \end{xten}
%%WHY-THIS-EXAMPLE%% %~x~siv
%%WHY-THIS-EXAMPLE%% %
%%WHY-THIS-EXAMPLE%% %~x~neg


\section{Operations on Arrays and Distributed Arrays}

Arrays and distributed arrays share many operations.
In the following, let \xcd`a` be an array with base type T, and \xcd`da` be an
array with distribution \xcd`D` and base type \xcd`T`.




\subsection{Element operations}\index{array!access}
The value of \xcd`a` at a point \xcd`p` in its region of definition is
%~~exp~~`~~`~~a:Array[Int](3), p:Point(3) ~~
obtained by using the indexing operation \xcd`a(p)`. 
The value of \xcd`da` at \xcd`p` is similarly
%~~exp~~`~~`~~da:DistArray[Int](3), p:Point(3) ~~
\xcd`da(p)`
This operation
may be used on the left hand side of an assignment operation to update
the value: 
%~~stmt~~`~~`~~a:Array[Int](3), p:Point(3), t:Int ~~
\xcd`a(p)=t;`
and 
%~~stmt~~`~~`~~da:DistArray[Int](3), p:Point(3), t:Int ~~
\xcd`da(p)=t;`
The operator assignments, \xcd`a(i) += e` and so on,  are also
available. 

It is a runtime error to use either \xcd`da(p)` or \xcd`da(p)=v` at a place
other than \xcd`da.dist(p)`, \viz{} at the place that the element exists. 

%%HUH%%  For distributed array variables, the right-hand-side of an assignment must
%%HUH%%  have the same distribution \xcd`D` as an array being assigned. This
%%HUH%%  assignment involves
%%HUH%%  control communication between the sites hosting \xcd`D`. Each
%%HUH%%  site performs the assignment(s) of array components locally. The
%%HUH%%  assignment terminates when assignment has terminated at all
%%HUH%%  sites hosting \xcd`D`.

\subsection{Constant promotion}\label{ConstantArray}
\index{array!constant promotion}

For a region \xcd`R` and a value \xcd`v` of type \xcd`T`, the expression 
%~~genexp~~`~~`~~T~~R:Region, v:T ~~
\xcd`new Array[T](R, v)` 
produces an array on region \xcd`R` initialized with value \xcd`v`
Similarly, 
for a distribution \xcd`D` and a value \xcd`v` of
type \xcd`T` the expression 
%~~genexp~~`~~`~~T ~~D:Dist, v:T ~~
\xcd`DistArray.make[T](D, (Point(D.rank))=>v)`
constructs a distributed array with
distribution \xcd`D` and base type \xcd`T` initialized with \xcd`v`
at every point.

Note that \xcd`Array`s are constructed by constructor calls, but
\xcd`DistArrays` are constructed by calls to the factory methods
\xcd`DistArray.make`. This is because \xcd`Array`s are fairly simple objects,
but \xcd`DistArray`s may be implemented by different classes for different
distributions. The use of the factory method gives the library writer the
freedom to select appropriate implementations.


\subsection{Restriction of an array}\index{array!restriction}

Let \xcd`R` be a sub-region of \xcd`da.region`. Then 
%~~exp~~`~~`~~da:DistArray[Int](3), p:Point(3), R: Region(da.rank) ~~
\xcd`da | R`
represents the sub-\xcd`DistArray` of \xcd`da` on the region \xcd`R`.
That is, \xcd`da | R` has the same values as \xcd`da` when subscripted by a
%~~exp~~`~~`~~R:Region, da: DistArray[Int]{da.region.rank == R.rank} ~~
point in region \xcd`R && da.region`, and is undefined elsewhere.
`
Recall that a rich set of operators are available on distributions
(\Sref{XtenDistributions}) to obtain sub-distributions
(e.g. restricting to a sub-region, to a specific place etc).

%%GONE-AWAY%%  \subsection{Assembling an array}
%%GONE-AWAY%%  Let \xcd`da1,da2` be distributed arrays of the same base type \xcd`T` defined over
%%GONE-AWAY%%  distributions \xcd`D1` and \xcd`D2` respectively. 
%%GONE-AWAY%%  \paragraph{Assembling arrays over disjoint regions}\index{array!union!disjoint}
%%GONE-AWAY%%  
%%GONE-AWAY%%  
%%GONE-AWAY%%  If \xcd`D1` and \xcd`D2` are disjoint then the expression 
%%GONE-AWAY%%  %~x~genexp~~`~~`~~T ~~ da1: Array[T], da2: Array[T](da1.rank)~~
%%GONE-AWAY%%  \xcd`da1 || da2` denotes the unique array of base type \xcd`T` defined over the
%%GONE-AWAY%%  distribution \xcd`D1 || D2` such that its value at point \xcd`p` is
%%GONE-AWAY%%  \xcd`a1(p)` if \xcd`p` lies in \xcd`D1` and \xcd`a2(p)`
%%GONE-AWAY%%  otherwise. This array is a reference (value) array if \xcd`a1` is.
%%GONE-AWAY%%  
%%GONE-AWAY%%  \paragraph{Overlaying an array on another}\index{array!union!asymmetric}
%%GONE-AWAY%%  The expression
%%GONE-AWAY%%  \xcd`a1.overlay(a2)` (read: the array \xcd`a1` {\em overlaid with} \xcd`a2`)
%%GONE-AWAY%%  represents an array whose underlying region is the union of that of
%%GONE-AWAY%%  \xcd`a1` and \xcd`a2` and whose distribution maps each point \xcd`p`
%%GONE-AWAY%%  in this region to \xcd`D2(p)` if that is defined and to \xcd`D1(p)`
%%GONE-AWAY%%  otherwise. The value \xcd`a1.overlay(a2)(p)` is \xcd`a2(p)` if it is defined and \xcd`a1(p)` otherwise.
%%GONE-AWAY%%  
%%GONE-AWAY%%  This array is a reference (value) array if \xcd`a1` is.
%%GONE-AWAY%%  
%%GONE-AWAY%%  The expression \xcd`a1.update(a2)` updates the array \xcd`a1` in place
%%GONE-AWAY%%  with the result of \xcd`a1.overlay(a2)`.
%%GONE-AWAY%%  
%%GONE-AWAY%%  \oldtodo{Define Flooding of arrays}
%%GONE-AWAY%%  
%%GONE-AWAY%%  \oldtodo{Wrapping an array}
%%GONE-AWAY%%  
%%GONE-AWAY%%  \oldtodo{Extending an array in a given direction.}
%%GONE-AWAY%%  
\subsection{Operations on Whole Arrays}

\paragraph{Pointwise operations}\label{ArrayPointwise}\index{array!pointwise operations}
The unary \xcd`map` operation applies a function to each element of
a distributed or non-distributed array, returning a new distributed array with
the same distribution, or a non-distributed array with the same region.
For example, the following produces an array of cubes: 
%~~gen
%package Arrays.arrays.ginungagap.bakery.treats;
%class Example{
%def example() {
%~~vis
\begin{xten}
val A = new Array[Int](1..10, (p:Point(1))=>p(0) );
// A = 1,2,3,4,5,6,7,8,9,10
val cube = (i:Int) => i*i*i;
val B = A.map(cube);
// B = 1,8,27,64,216,343,512,729,1000
\end{xten}
%~~siv
%} } 
%~~neg

A variant operation lets you specify the array \Xcd{B} into which the result
will be stored.   
%~~gen
%package Arrays.arrays.ginungagap.bakery.treats.doomed;
%class Example{
%def example() {
%~~vis
\begin{xten}
val A = new Array[Int](1..10, (p:Point(1))=>p(0) );
// A = 1,2,3,4,5,6,7,8,9,10
val cube = (i:Int) => i*i*i;
val B = new Array[Int](A.region); // B = 0,0,0,0,0,0,0,0,0,0
A.map(B, cube);
// B = 1,8,27,64,216,343,512,729,1000
\end{xten}
%~~siv
%} } 
%~~neg
\noindent
This is convenient if you have an already-allocated array lying around unused.
In particular, it can be used if you don't need \Xcd{A} afterwards and want to
reuse its space:
%~~gen
%package Arrays.arrays.ginungagap.bakery.treats.doomed.spackled;
%class Example{
%def example() {
%~~vis
\begin{xten}
val A = new Array[Int](1..10, (p:Point(1))=>p(0) );
// A = 1,2,3,4,5,6,7,8,9,10
val cube = (i:Int) => i*i*i;
A.map(A, cube);
// A = 1,8,27,64,216,343,512,729,1000
\end{xten}
%~~siv
%} } 
%~~neg


The binary \xcd`map` operation takes a binary function and
another
array over the same region or distributed array over the same  distribution,
and applies the function 
pointwise to corresponding elements of the two arrays, returning
a new array or distributed array of the same shape.
The following code adds two distributed arrays: 
%~~gen
% package Arrays.Pointwise.Pointless.Map2;
% class Example{
%~~vis
\begin{xten}
static def add(da:DistArray[Int], db: DistArray[Int]{da.dist==db.dist})
    = da.map(db, Int.+);
\end{xten}
%~~siv
%}
%~~neg



\paragraph{Reductions}\label{ArrayReductions}\index{array!reductions}

Let \xcd`f` be a function of type \xcd`(T,T)=>T`.  Let
\xcd`a` be an array over base type \xcd`T`.
Let \xcd`unit` be a value of type \xcd`T`.
Then the
%~~genexp~~`~~`~~ T ~~ f:(T,T)=>T, a : Array[T], unit:T ~~
operation \xcd`a.reduce(f, unit)` returns a value of type \xcd`T` obtained
by combining all the elements of \xcd`a` by use of  \xcd`f` in some unspecified order
(perhaps in parallel).   
The following code gives one method which 
meets the definition of \Xcd{reduce},
having a running total \Xcd{r}, and accumulating each value \xcd`a(p)` into it
using \Xcd{f} in turn.  (This code is simply given as an example; \Xcd{Array}
has this operation defined already.)
%~~gen
%package Arrays.Reductions.And.Eliminations;
% class Example {
%~~vis
\begin{xten}
def oneWayToReduce[T](a:Array[T], f:(T,T)=>T, unit:T):T {
  var r : T = unit;
  for(p in a.region) r = f(r, a(p));
  return r;
}
\end{xten}
%~~siv
%}
%~~neg


For example,  the following sums an array of integers.  \Xcd{f} is addition,
and \Xcd{unit} is zero.  
%~~gen
% package Arrays.Reductions.And.Emulsions;
% class Example {
% def example() {
%~~vis
\begin{xten}
val a = [1,2,3,4];
val sum = a.reduce(Int.+, 0); 
assert(sum == 10); // 10 == 1+2+3+4
\end{xten}
%~~siv
%}}
%~~neg

Other orders of evaluation, degrees of parallelism, and applications of
\Xcd{f(x,unit)} and \xcd`f(unit,x)`are also correct.
In order to guarantee that the result is precisely
determined, the  function \xcd`f` should be associative and
commutative, and the value \xcd`unit` should satisfy
\xcd`f(unit,x)` \xcd`==` \xcd`x` \xcd`==` \xcd`f(x,unit)`
for all \Xcd{x:T}.  




\xcd`DistArray`s have the same operation.
This operation involves communication between the places over which
the \xcd`DistArray` is distributed. The \Xten{} implementation guarantees that
only one value of type \xcd`T` is communicated from a place as part of
this reduction process.

\paragraph{Scans}\label{ArrayScans}\index{array!scans}


Let \xcd`f:(T,T)=>T`, \xcd`unit:T`, and \xcd`a` be an \xcd`Array[T]` or
\xcd`DistArray[T]`.  Then \xcd`a.scan(f,unit)` is the array or distributed
array of type \xcd`T` whose {$i$}th element in canonical order is the
reduction by \xcd`f` with unit \xcd`unit` of the first {$i$} elements of
\xcd`a`. 


This operation involves communication between the places over which the
distributed array is distributed. The \Xten{} implementation will endeavour to
minimize the communication between places to implement this operation.

Other operations on arrays, distributed arrays, and the related classes may be
found in the \xcd`x10.array` package.
	
\chapter{Annotations}\label{XtenAnnotations}\index{annotations}


\Xten{} provides an 
an annotation system  system for to allow the
compiler to be extended with new static analyses and new
transformations.

Annotations are constraint-free interface types that decorate the abstract syntax tree
of an \Xten{} program.  The \Xten{} type-checker ensures that an annotation
is a legal interface type.
In \Xten{}, interfaces may declare
both methods and properties.  Therefore, like any interface type, an
annotation may instantiate
one or more of its interface's properties.
%%PLUGINNERY%%  Unlike with Java
%%PLUGINNERY%%  annotations,
%%PLUGINNERY%%  property initializers need not be
%%PLUGINNERY%%  compile-time constants;
%%PLUGINNERY%%  however, a given compiler plugin
%%PLUGINNERY%%  may do additional checks to constrain the allowable
%%PLUGINNERY%%  initializer expressions.
%%PLUGINNERY%%  The \Xten{} type-checker does not check that
%%PLUGINNERY%%  all properties of an annotation are initialized,
%%PLUGINNERY%%  although this could be enforced by
%%PLUGINNERY%%  a compiler plugin.

\section{Annotation syntax}

The annotation syntax consists of an ``\texttt{@}'' followed by an interface type.

%##(Annotations Annotation
\begin{bbgrammar}
%(FROM #(prod:Annotations)#)
         Annotations \: Annotation & (\ref{prod:Annotations}) \\
                     \| Annotations Annotation \\
%(FROM #(prod:Annotation)#)
          Annotation \: \xcd"@" NamedTypeNoConstraints & (\ref{prod:Annotation}) \\
\end{bbgrammar}
%##)

Annotations can be applied to most syntactic constructs in the language
including class declarations, constructors, methods, field declarations,
local variable declarations and formal parameters, statements,
expressions, and types.
Multiple occurrences of the same annotation (i.e., multiple
annotations with the same interface type) on the same entity are permitted.

%%OBSOLETE%% \begin{grammar}
%%OBSOLETE%% ClassModifier \: Annotation \\
%%OBSOLETE%% InterfaceModifier \: Annotation \\
%%OBSOLETE%% FieldModifier \: Annotation \\
%%OBSOLETE%% MethodModifier \: Annotation \\
%%OBSOLETE%% VariableModifier \: Annotation \\
%%OBSOLETE%% ConstructorModifier \: Annotation \\
%%OBSOLETE%% AbstractMethodModifier \: Annotation \\
%%OBSOLETE%% ConstantModifier \: Annotation \\
%%OBSOLETE%% Type \: AnnotatedType \\
%%OBSOLETE%% AnnotatedType \: Annotation\plus Type \\
%%OBSOLETE%% Statement \: AnnotatedStatement \\
%%OBSOLETE%% AnnotatedStatement \: Annotation\plus Statement \\
%%OBSOLETE%% Expression \: AnnotatedExpression \\
%%OBSOLETE%% AnnotatedExpression \: Annotation\plus Expression \\
%%OBSOLETE%% \end{grammar}

\noindent
Recall that interface types may have dependent parameters.

\noindent
The following examples illustrate the syntax:

\begin{itemize}
\item Declaration annotations:
\begin{xtennoindent}
  // class annotation
  @Value
  class Cons { ... }

  // method annotation
  @PreCondition(0 <= i && i < this.size)
  public def get(i: Int): Object { ... }

  // constructor annotation
  @Where(x != null)
  def this(x: T) { ... }

  // constructor return type annotation
  def this(x: T): C@Initialized { ... }

  // variable annotation
  @Unique x: A;
\end{xtennoindent}
\item Type annotations:
\begin{xtennoindent}
  List@Nonempty

  Int@Range(1,4)

  Array[Array[Double]]@Size(n * n)
\end{xtennoindent}
\item Expression annotations:
\begin{xtennoindent}
  m()  @RemoteCall
\end{xtennoindent}
\item Statement annotations:
\begin{xtennoindent}
  @Atomic { ... }

  @MinIterations(0)
  @MaxIterations(n)
  for (var i: Int = 0; i < n; i++) { ... }

  // An annotated empty statement ;
  @Assert(x < y);
\end{xtennoindent}
\end{itemize}

\section{Annotation declarations}

Annotations are declared as interfaces.  They must be
subtypes of the interface \texttt{x10.lang.annotation.Annotation}.
Annotations on particular static entities must extend the corresponding
\xcd`Annotation` subclasses, as follows: 
\begin{itemize}
\item Expressions---\xcd`ExpressionAnnotation`
\item Statements---\xcd`StatementAnnotation`
\item Classes---\xcd`ClassAnnotation`
\item Fields---\xcd`FieldAnnotation`
\item Methods---\xcd`MethodAnnotation`
\item Imports---\xcd`ImportAnnotation`
\item Packages---\xcd`PackageAnnotation`
\end{itemize}

\section{Linking with native code}\label{extern}\index{extern}
\XtenCurrVer{} supports a simple facility to permit the efficient
intra-thread communication of an array of primitive type to code
written in the language {\tt C}.  The array must be a ``local''
array. The primary intent of this design is to permit the reuse of
native code that efficiently implements some numeric array/matrix
calculation.

Future language releases are expected to support similar bindings to
{\sc Fortran}, and to support parallel native processing of
distributed \Xten{} arrays. 

The interface consists of two parts. First, an array intended to be
communicated to native code must be created as an {\tt unsafe} array:
\begin{x10}
450 ArrayCreationExpression ::= 
      new ArrayBaseType Unsafeopt [ ] 
        ArrayInitializer
451   | new ArrayBaseType Unsafeopt [ Expression ]
452   | new ArrayBaseType Unsafeopt 
          [ Expression ] Expression
453   | new ArrayBaseType Unsafeopt [ Expression ] 
          ( FormalParameter ) MethodBody
454   | new ArrayBaseType value 
           Unsafeopt [ Expression ]
455   | new ArrayBaseType value 
           Unsafeopt [ Expression ] Expression
456   | new ArrayBaseType value 
        Unsafeopt [ Expression ] 
          ( FormalParameter ) MethodBody
530   Unsafeopt ::=
531     | unsafe
\end{x10}
Unsafe arrays can be of any dimension. However, \XtenCurrVer{}
requires that unsafe arrays be of a primitive type, and local (i.e.{}
with an underlying distribution that maps all elements in its region
to {\tt here}).

Unsafe arrays are allocated in a special array of memory that permits
their efficient transmission to natively linked code.
%% Comment about when this memory is freed.

Second, the \Xten{} programmer may specify that certain methods are to
be implemented natively by using the keyword {\tt extern}:
\begin{x10}
446   MethodModifier ::= extern
\end{x10}
Such a method must have the statement ``{\tt ;}'' as its body.
\XtenCurrVer{} requires that the method be {\tt static}; this
restriction is likely to be lifted in the future.  Primitive types in
the method argument are translated to their corresponding JNI type
(e.g.{} {\tt float} is translated to {\tt jfloat}, {\tt double} to
{\tt jdouble} etc).  The only non-primitive type permitted in an {\tt
extern} method is an (unsafe) array. This is passed at type {\tt
jlong} as an eight byte address into the unsafe region which contains
the data for the array. ({\tt jlong} is not the same as {\tt long} on
32-bit machines.)


Since only the starting address of an array is passed, if the array is
multidimensional, the user must explicitly communicate (or have a
guarantee of) the rank of the passed array, and must either typecast
or explicitly code the address calculation.  Note that all \Xten{}
arrays are created in row-major order, and so any native routine must
also access them in the same order.

For each class {\tt C} that contains an {\tt extern} method, the
\Xten{} compiler generates a text file {\tt C\_x10stub.c}.  This file
contains generated {\tt C} stub functions which are called from the
{\tt extern} routines.  The name of the stub function is derived from
the name of the {\tt extern} method. If the method is {\tt
C.process()}, the stub function will be {\tt
Java\_C\_C\_process()}. The name is suffixed with the signature of the
method if the method is overloaded.

The programmer must write {\tt C} code to implement the native method,
using the methods in the {\tt C} stub file to call the actual native
method.  The programmer must compile these files and link them into a
dynamically linked library (DLL).  Note that the {\tt jni.h} header file
must be in the include path.  The programmer must ensure this library
is loaded by the program before the method is called e.g.{} add a {\tt
System.loadlibrary} call (in a static initializer of the
\Xten{} class).

\paragraph{Example.}
The following class illustrates the use of {\tt unsafe} and native
linking. 
\begin{x10}
public class IntArrayExternUnsafe \{
  public static extern 
      void process(int [.] yy, int size);
  static {System.loadLibrary("IntArrayExternUnsafe");}
  public static void main(String args[]) \{
     boolean b= (new IntArrayExternUnsafe()).run();
     System.out.println("++++++ Test "
                         +(b?"succeeded.":"failed."));
     System.exit(b?0:1);
  \}
  public boolean run()\{
    int high = 10;
    boolean verified=false;
    distribution d= (0:high) -> here;
    int [.] y = new int unsafe[d]; 
    for( int j=0;j < 10;++j)
        y[j] = j;
    process(y,high);
    for(int j=0;j < 10;++j)\{
      int expected = j+100;
      if(y[j] != expected)\{
        System.out.println("y["+j+"]="
                           +y[j]+" != "+expected);
        return false;
       \}
    \}
    return true;
  \}
\}
\end{x10}

The programmer may then write the {\tt C} code thus:
\begin{x10}
void IntArrayExternUnsafe\_process(jlong yy, 
                                signed int size)\{
  int i;
  int* array = (int *)(long)yy;
  for(i = 0;i < size;++i)\{
    array[i] += 100;
  \}
\}
/* automatically generated in \_x10stub.c*/
void 
 Java\_IntArrayExternUnsafe\_IntArrayExternUnsafe\_process
 (JNIEnv *env,  jobject obj,jlong yy,jint size)\{
   IntArrayExternUnsafe\_process(yy,size);
\}
\end{x10}

This code may be linked with the stub file (or textually placed in
it). The programmer must then compile and link the {\tt C} code and
ensure that the DLL is on the appropriate classpath. 


\chapter{Lost Bits}

\bard{This chapter should not exist.  The material in it was misplaced.  It 
is stored here temporarily
so it doesn't get lost}

\section{Visibility of Local Variables and Formals}

In general, variables (\ie, local variables, parameters,
properties, fields) are visible at
a point in the c
if they are defined before \xcd"T" in the program. This rule applies to
types in property lists as well as parameter lists (for methods and
constructors).
A formal parameter is visible in the types of all other formal
parameters of the same method, constructor, or type definition,
as well as in the method or constructor body itself.
Properties are accessible via their containing object--\xcd"this"
within the body of their class declaration.  The special
variable \xcd"this" is in scope at each property
declaration, constructor signatures and bodies, instance method signatures
and bodies,
and instance field signatures and initializers, but not in scope
at \xcd"static" method or field declarations or \xcd"static"
initializers.  


%\chapter{Grammar}


\begin{bbgrammar}

 MethodInvocation  \refstepcounter{equation}\label{prod:MethodInvocation}  \: MethodPrimaryPrefix \xcd"(" ArgumentList\opt \xcd")" & (\arabic{equation})\\
    \| MethodSuperPrefix \xcd"(" ArgumentList\opt \xcd")"\\
    \| MethodClassNameSuperPrefix \xcd"(" ArgumentList\opt \xcd")"\\
 Mod  \refstepcounter{equation}\label{prod:Mod}  \: \xcd"abstract" & (\arabic{equation})\\
    \| Annotation\\
    \| atomic\\
    \| \xcd"final"\\
    \| \xcd"native"\\
    \| \xcd"private"\\
    \| \xcd"protected"\\
    \| \xcd"public"\\
    \| \xcd"static"\\
    \| \xcd"transient"\\
    \| clocked\\
 TypeDefDecl  \refstepcounter{equation}\label{prod:TypeDefDecl}  \: Mods\opt type Id TypeParams\opt FormalParams\opt WhereClause\opt \xcd"=" Type \xcd";" & (\arabic{equation})\\
 Properties  \refstepcounter{equation}\label{prod:Properties}  \: \xcd"(" PropertyList \xcd")" & (\arabic{equation})\\
\end{bbgrammar}

\begin{bbgrammar}

 PropertyList  \refstepcounter{equation}\label{prod:PropertyList}  \: Property & (\arabic{equation})\\
    \| PropertyList \xcd"," Property\\
 Property  \refstepcounter{equation}\label{prod:Property}  \: Annotations\opt Id ResultType & (\arabic{equation})\\
 MethodDecl  \refstepcounter{equation}\label{prod:MethodDecl}  \: MethodMods\opt \xcd"def" Id TypeParams\opt FormalParams WhereClause\opt HasResultType\opt Offers\opt MethodBody & (\arabic{equation})\\
    \| MethodMods\opt \xcd"operator" TypeParams\opt \xcd"(" FormalParam  \xcd")" BinOp \xcd"(" FormalParam  \xcd")" WhereClause\opt HasResultType\opt Offers\opt MethodBody\\
    \| MethodMods\opt \xcd"operator" TypeParams\opt PrefixOp \xcd"(" FormalParam  \xcd")" WhereClause\opt HasResultType\opt Offers\opt MethodBody\\
    \| MethodMods\opt \xcd"operator" TypeParams\opt \xcd"this" BinOp \xcd"(" FormalParam  \xcd")" WhereClause\opt HasResultType\opt Offers\opt MethodBody\\
    \| MethodMods\opt \xcd"operator" TypeParams\opt \xcd"(" FormalParam  \xcd")" BinOp \xcd"this" WhereClause\opt HasResultType\opt Offers\opt MethodBody\\
    \| MethodMods\opt \xcd"operator" TypeParams\opt PrefixOp \xcd"this" WhereClause\opt HasResultType\opt Offers\opt MethodBody\\
    \| MethodMods\opt \xcd"operator" \xcd"this" TypeParams\opt FormalParams WhereClause\opt HasResultType\opt Offers\opt MethodBody\\
    \| MethodMods\opt \xcd"operator" \xcd"this" TypeParams\opt FormalParams \xcd"=" \xcd"(" FormalParam  \xcd")" WhereClause\opt HasResultType\opt Offers\opt MethodBody\\
    \| MethodMods\opt \xcd"operator" TypeParams\opt \xcd"(" FormalParam  \xcd")" \xcd"as" Type WhereClause\opt Offers\opt MethodBody\\
    \| MethodMods\opt \xcd"operator" TypeParams\opt \xcd"(" FormalParam  \xcd")" \xcd"as" \xcd"?" WhereClause\opt HasResultType\opt Offers\opt MethodBody\\
    \| MethodMods\opt \xcd"operator" TypeParams\opt \xcd"(" FormalParam  \xcd")" WhereClause\opt HasResultType\opt Offers\opt MethodBody\\
 PropertyMethodDecl  \refstepcounter{equation}\label{prod:PropertyMethodDecl}  \: MethodMods\opt Id TypeParams\opt FormalParams WhereClause\opt HasResultType\opt MethodBody & (\arabic{equation})\\
    \| MethodMods\opt Id WhereClause\opt HasResultType\opt MethodBody\\
 ExplicitCtorInvocation  \refstepcounter{equation}\label{prod:ExplicitCtorInvocation}  \: \xcd"this" TypeArguments\opt \xcd"(" ArgumentList\opt \xcd")" \xcd";" & (\arabic{equation})\\
    \| \xcd"super" TypeArguments\opt \xcd"(" ArgumentList\opt \xcd")" \xcd";"\\
    \| Primary \xcd"." \xcd"this" TypeArguments\opt \xcd"(" ArgumentList\opt \xcd")" \xcd";"\\
    \| Primary \xcd"." \xcd"super" TypeArguments\opt \xcd"(" ArgumentList\opt \xcd")" \xcd";"\\
 NormalInterfaceDecl  \refstepcounter{equation}\label{prod:NormalInterfaceDecl}  \: Mods\opt \xcd"interface" Id TypeParamsWithVariance\opt Properties\opt WhereClause\opt ExtendsInterfaces\opt InterfaceBody & (\arabic{equation})\\
 ClassInstCreationExp  \refstepcounter{equation}\label{prod:ClassInstCreationExp}  \: \xcd"new" TypeName TypeArguments\opt \xcd"(" ArgumentList\opt \xcd")" ClassBody\opt & (\arabic{equation})\\
    \| \xcd"new" TypeName \xcd"[" Type \xcd"]" \xcd"[" ArgumentList\opt \xcd"]"\\
    \| Primary \xcd"." \xcd"new" Id TypeArguments\opt \xcd"(" ArgumentList\opt \xcd")" ClassBody\opt\\
    \| AmbiguousName \xcd"." \xcd"new" Id TypeArguments\opt \xcd"(" ArgumentList\opt \xcd")" ClassBody\opt\\
 AssignPropertyCall  \refstepcounter{equation}\label{prod:AssignPropertyCall}  \: \xcd"property" TypeArguments\opt \xcd"(" ArgumentList\opt \xcd")" \xcd";" & (\arabic{equation})\\
 Type  \refstepcounter{equation}\label{prod:Type}  \: FunctionType & (\arabic{equation})\\
    \| ConstrainedType\\
 FunctionType  \refstepcounter{equation}\label{prod:FunctionType}  \: TypeParams\opt \xcd"(" FormalParamList\opt \xcd")" WhereClause\opt Offers\opt \xcd"=>" Type & (\arabic{equation})\\
 ClassType  \refstepcounter{equation}\label{prod:ClassType}  \: NamedType & (\arabic{equation})\\
 AnnotatedType  \refstepcounter{equation}\label{prod:AnnotatedType}  \: Type Annotations & (\arabic{equation})\\
 ConstrainedType  \refstepcounter{equation}\label{prod:ConstrainedType}  \: NamedType & (\arabic{equation})\\
    \| AnnotatedType\\
    \| \xcd"(" Type \xcd")"\\
 PlaceType  \refstepcounter{equation}\label{prod:PlaceType}  \: PlaceExp & (\arabic{equation})\\
 SimpleNamedType  \refstepcounter{equation}\label{prod:SimpleNamedType}  \: TypeName & (\arabic{equation})\\
    \| Primary \xcd"." Id\\
    \| DepNamedType \xcd"." Id\\
 DepNamedType  \refstepcounter{equation}\label{prod:DepNamedType}  \: SimpleNamedType DepParams & (\arabic{equation})\\
    \| SimpleNamedType Arguments\\
    \| SimpleNamedType Arguments DepParams\\
\end{bbgrammar}

\begin{bbgrammar}

    \| SimpleNamedType TypeArguments\\
    \| SimpleNamedType TypeArguments DepParams\\
    \| SimpleNamedType TypeArguments Arguments\\
    \| SimpleNamedType TypeArguments Arguments DepParams\\
 NamedType  \refstepcounter{equation}\label{prod:NamedType}  \: SimpleNamedType & (\arabic{equation})\\
    \| DepNamedType\\
 DepParams  \refstepcounter{equation}\label{prod:DepParams}  \: \xcd"{" ExistentialList\opt Conjunction\opt \xcd"}" & (\arabic{equation})\\
 TypeParamsWithVariance  \refstepcounter{equation}\label{prod:TypeParamsWithVariance}  \: \xcd"[" TypeParamWithVarianceList \xcd"]" & (\arabic{equation})\\
 TypeParams  \refstepcounter{equation}\label{prod:TypeParams}  \: \xcd"[" TypeParamList \xcd"]" & (\arabic{equation})\\
 FormalParams  \refstepcounter{equation}\label{prod:FormalParams}  \: \xcd"(" FormalParamList\opt \xcd")" & (\arabic{equation})\\
 Conjunction  \refstepcounter{equation}\label{prod:Conjunction}  \: Exp & (\arabic{equation})\\
    \| Conjunction \xcd"," Exp\\
 SubtypeConstraint  \refstepcounter{equation}\label{prod:SubtypeConstraint}  \: Type  \xcd"<:" Type  & (\arabic{equation})\\
    \| Type  \xcd":>" Type \\
 WhereClause  \refstepcounter{equation}\label{prod:WhereClause}  \: DepParams & (\arabic{equation})\\
 ExistentialList  \refstepcounter{equation}\label{prod:ExistentialList}  \: FormalParam & (\arabic{equation})\\
    \| ExistentialList \xcd";" FormalParam\\
 ClassDecl  \refstepcounter{equation}\label{prod:ClassDecl}  \: StructDecl & (\arabic{equation})\\
    \| NormalClassDecl\\
 NormalClassDecl  \refstepcounter{equation}\label{prod:NormalClassDecl}  \: Mods\opt \xcd"class" Id TypeParamsWithVariance\opt Properties\opt WhereClause\opt Super\opt Interfaces\opt ClassBody & (\arabic{equation})\\
 StructDecl  \refstepcounter{equation}\label{prod:StructDecl}  \: Mods\opt \xcd"struct" Id TypeParamsWithVariance\opt Properties\opt WhereClause\opt Interfaces\opt ClassBody & (\arabic{equation})\\
 CtorDecl  \refstepcounter{equation}\label{prod:CtorDecl}  \: Mods\opt \xcd"def" \xcd"this" TypeParams\opt FormalParams WhereClause\opt HasResultType\opt Offers\opt CtorBody & (\arabic{equation})\\
 Super  \refstepcounter{equation}\label{prod:Super}  \: \xcd"extends" ClassType & (\arabic{equation})\\
 FieldKeyword  \refstepcounter{equation}\label{prod:FieldKeyword}  \: val & (\arabic{equation})\\
    \| \xcd"var"\\
 VarKeyword  \refstepcounter{equation}\label{prod:VarKeyword}  \: val & (\arabic{equation})\\
    \| \xcd"var"\\
 FieldDecl  \refstepcounter{equation}\label{prod:FieldDecl}  \: Mods\opt FieldKeyword FieldDeclarators \xcd";" & (\arabic{equation})\\
    \| Mods\opt FieldDeclarators \xcd";"\\
 Statement  \refstepcounter{equation}\label{prod:Statement}  \: AnnotationStatement & (\arabic{equation})\\
    \| ExpStatement\\
 AnnotationStatement  \refstepcounter{equation}\label{prod:AnnotationStatement}  \: Annotations\opt NonExpStatement & (\arabic{equation})\\
 NonExpStatement  \refstepcounter{equation}\label{prod:NonExpStatement}  \: Block & (\arabic{equation})\\
    \| EmptyStatement\\
    \| AssertStatement\\
    \| SwitchStatement\\
    \| DoStatement\\
    \| BreakStatement\\
    \| ContinueStatement\\
    \| ReturnStatement\\
    \| ThrowStatement\\
\end{bbgrammar}

\begin{bbgrammar}

    \| TryStatement\\
    \| LabeledStatement\\
    \| IfThenStatement\\
    \| IfThenElseStatement\\
    \| WhileStatement\\
    \| ForStatement\\
    \| AsyncStatement\\
    \| AtStatement\\
    \| AtomicStatement\\
    \| WhenStatement\\
    \| AtEachStatement\\
    \| FinishStatement\\
    \| NextStatement\\
    \| ResumeStatement\\
    \| AssignPropertyCall\\
    \| OfferStatement\\
 OfferStatement  \refstepcounter{equation}\label{prod:OfferStatement}  \: offer Exp \xcd";" & (\arabic{equation})\\
 IfThenStatement  \refstepcounter{equation}\label{prod:IfThenStatement}  \: \xcd"if" \xcd"(" Exp \xcd")" Statement & (\arabic{equation})\\
 IfThenElseStatement  \refstepcounter{equation}\label{prod:IfThenElseStatement}  \: \xcd"if" \xcd"(" Exp \xcd")" Statement  \xcd"else" Statement  & (\arabic{equation})\\
 EmptyStatement  \refstepcounter{equation}\label{prod:EmptyStatement}  \: \xcd";" & (\arabic{equation})\\
 LabeledStatement  \refstepcounter{equation}\label{prod:LabeledStatement}  \: Id \xcd":" LoopStatement & (\arabic{equation})\\
 LoopStatement  \refstepcounter{equation}\label{prod:LoopStatement}  \: ForStatement & (\arabic{equation})\\
    \| WhileStatement\\
    \| DoStatement\\
    \| AtEachStatement\\
 ExpStatement  \refstepcounter{equation}\label{prod:ExpStatement}  \: StatementExp \xcd";" & (\arabic{equation})\\
 StatementExp  \refstepcounter{equation}\label{prod:StatementExp}  \: Assignment & (\arabic{equation})\\
    \| PreIncrementExp\\
    \| PreDecrementExp\\
    \| PostIncrementExp\\
    \| PostDecrementExp\\
    \| MethodInvocation\\
    \| ClassInstCreationExp\\
 AssertStatement  \refstepcounter{equation}\label{prod:AssertStatement}  \: \xcd"assert" Exp \xcd";" & (\arabic{equation})\\
    \| \xcd"assert" Exp  \xcd":" Exp  \xcd";"\\
 SwitchStatement  \refstepcounter{equation}\label{prod:SwitchStatement}  \: \xcd"switch" \xcd"(" Exp \xcd")" SwitchBlock & (\arabic{equation})\\
 SwitchBlock  \refstepcounter{equation}\label{prod:SwitchBlock}  \: \xcd"{" SwitchBlockStatementGroups\opt SwitchLabels\opt \xcd"}" & (\arabic{equation})\\
 SwitchBlockStatementGroups  \refstepcounter{equation}\label{prod:SwitchBlockStatementGroups}  \: SwitchBlockStatementGroup & (\arabic{equation})\\
    \| SwitchBlockStatementGroups SwitchBlockStatementGroup\\
 SwitchBlockStatementGroup  \refstepcounter{equation}\label{prod:SwitchBlockStatementGroup}  \: SwitchLabels BlockStatements & (\arabic{equation})\\
 SwitchLabels  \refstepcounter{equation}\label{prod:SwitchLabels}  \: SwitchLabel & (\arabic{equation})\\
\end{bbgrammar}

\begin{bbgrammar}

    \| SwitchLabels SwitchLabel\\
 SwitchLabel  \refstepcounter{equation}\label{prod:SwitchLabel}  \: \xcd"case" ConstantExp \xcd":" & (\arabic{equation})\\
    \| \xcd"default" \xcd":"\\
 WhileStatement  \refstepcounter{equation}\label{prod:WhileStatement}  \: \xcd"while" \xcd"(" Exp \xcd")" Statement & (\arabic{equation})\\
 DoStatement  \refstepcounter{equation}\label{prod:DoStatement}  \: \xcd"do" Statement \xcd"while" \xcd"(" Exp \xcd")" \xcd";" & (\arabic{equation})\\
 ForStatement  \refstepcounter{equation}\label{prod:ForStatement}  \: BasicForStatement & (\arabic{equation})\\
    \| EnhancedForStatement\\
 BasicForStatement  \refstepcounter{equation}\label{prod:BasicForStatement}  \: \xcd"for" \xcd"(" ForInit\opt \xcd";" Exp\opt \xcd";" ForUpdate\opt \xcd")" Statement & (\arabic{equation})\\
 ForInit  \refstepcounter{equation}\label{prod:ForInit}  \: StatementExpList & (\arabic{equation})\\
    \| LocalVariableDecl\\
 ForUpdate  \refstepcounter{equation}\label{prod:ForUpdate}  \: StatementExpList & (\arabic{equation})\\
 StatementExpList  \refstepcounter{equation}\label{prod:StatementExpList}  \: StatementExp & (\arabic{equation})\\
    \| StatementExpList \xcd"," StatementExp\\
 BreakStatement  \refstepcounter{equation}\label{prod:BreakStatement}  \: \xcd"break" Id\opt \xcd";" & (\arabic{equation})\\
 ContinueStatement  \refstepcounter{equation}\label{prod:ContinueStatement}  \: \xcd"continue" Id\opt \xcd";" & (\arabic{equation})\\
 ReturnStatement  \refstepcounter{equation}\label{prod:ReturnStatement}  \: \xcd"return" Exp\opt \xcd";" & (\arabic{equation})\\
 ThrowStatement  \refstepcounter{equation}\label{prod:ThrowStatement}  \: \xcd"throw" Exp \xcd";" & (\arabic{equation})\\
 TryStatement  \refstepcounter{equation}\label{prod:TryStatement}  \: \xcd"try" Block Catches & (\arabic{equation})\\
    \| \xcd"try" Block Catches\opt Finally\\
 Catches  \refstepcounter{equation}\label{prod:Catches}  \: CatchClause & (\arabic{equation})\\
    \| Catches CatchClause\\
 CatchClause  \refstepcounter{equation}\label{prod:CatchClause}  \: \xcd"catch" \xcd"(" FormalParam \xcd")" Block & (\arabic{equation})\\
 Finally  \refstepcounter{equation}\label{prod:Finally}  \: \xcd"finally" Block & (\arabic{equation})\\
 ClockedClause  \refstepcounter{equation}\label{prod:ClockedClause}  \: clocked \xcd"(" ClockList \xcd")" & (\arabic{equation})\\
 AsyncStatement  \refstepcounter{equation}\label{prod:AsyncStatement}  \: \xcd"async" ClockedClause\opt Statement & (\arabic{equation})\\
    \| clocked \xcd"async" Statement\\
 AtStatement  \refstepcounter{equation}\label{prod:AtStatement}  \: at PlaceExpSingleList Statement & (\arabic{equation})\\
 AtomicStatement  \refstepcounter{equation}\label{prod:AtomicStatement}  \: atomic Statement & (\arabic{equation})\\
 WhenStatement  \refstepcounter{equation}\label{prod:WhenStatement}  \: \xcd"when" \xcd"(" Exp \xcd")" Statement & (\arabic{equation})\\
 AtEachStatement  \refstepcounter{equation}\label{prod:AtEachStatement}  \: \xcd"ateach" \xcd"(" LoopIndex \xcd"in" Exp \xcd")" ClockedClause\opt Statement & (\arabic{equation})\\
    \| \xcd"ateach" \xcd"(" Exp \xcd")" Statement\\
 EnhancedForStatement  \refstepcounter{equation}\label{prod:EnhancedForStatement}  \: \xcd"for" \xcd"(" LoopIndex \xcd"in" Exp \xcd")" Statement & (\arabic{equation})\\
    \| \xcd"for" \xcd"(" Exp \xcd")" Statement\\
 FinishStatement  \refstepcounter{equation}\label{prod:FinishStatement}  \: \xcd"finish" Statement & (\arabic{equation})\\
    \| clocked \xcd"finish" Statement\\
 PlaceExpSingleList  \refstepcounter{equation}\label{prod:PlaceExpSingleList}  \: \xcd"(" PlaceExp \xcd")" & (\arabic{equation})\\
 PlaceExp  \refstepcounter{equation}\label{prod:PlaceExp}  \: Exp & (\arabic{equation})\\
 NextStatement  \refstepcounter{equation}\label{prod:NextStatement}  \: next \xcd";" & (\arabic{equation})\\
 ResumeStatement  \refstepcounter{equation}\label{prod:ResumeStatement}  \: resume \xcd";" & (\arabic{equation})\\
 ClockList  \refstepcounter{equation}\label{prod:ClockList}  \: Clock & (\arabic{equation})\\
    \| ClockList \xcd"," Clock\\
\end{bbgrammar}

\begin{bbgrammar}

 Clock  \refstepcounter{equation}\label{prod:Clock}  \: Exp & (\arabic{equation})\\
 CastExp  \refstepcounter{equation}\label{prod:CastExp}  \: Primary & (\arabic{equation})\\
    \| ExpName\\
    \| CastExp \xcd"as" Type\\
 TypeParamWithVarianceList  \refstepcounter{equation}\label{prod:TypeParamWithVarianceList}  \: TypeParamWithVariance & (\arabic{equation})\\
    \| TypeParamWithVarianceList \xcd"," TypeParamWithVariance\\
 TypeParamList  \refstepcounter{equation}\label{prod:TypeParamList}  \: TypeParam & (\arabic{equation})\\
    \| TypeParamList \xcd"," TypeParam\\
 TypeParamWithVariance  \refstepcounter{equation}\label{prod:TypeParamWithVariance}  \: Id & (\arabic{equation})\\
    \| \xcd"+" Id\\
    \| \xcd"-" Id\\
 TypeParam  \refstepcounter{equation}\label{prod:TypeParam}  \: Id & (\arabic{equation})\\
 AssignmentExp  \refstepcounter{equation}\label{prod:AssignmentExp}  \: Exp  \xcd"->" Exp  & (\arabic{equation})\\
 ClosureExp  \refstepcounter{equation}\label{prod:ClosureExp}  \: FormalParams WhereClause\opt HasResultType\opt Offers\opt \xcd"=>" ClosureBody & (\arabic{equation})\\
 LastExp  \refstepcounter{equation}\label{prod:LastExp}  \: Exp & (\arabic{equation})\\
 ClosureBody  \refstepcounter{equation}\label{prod:ClosureBody}  \: ConditionalExp & (\arabic{equation})\\
    \| Annotations\opt \xcd"{" BlockStatements\opt LastExp \xcd"}"\\
    \| Annotations\opt Block\\
 AtExp  \refstepcounter{equation}\label{prod:AtExp}  \: at PlaceExpSingleList ClosureBody & (\arabic{equation})\\
 FinishExp  \refstepcounter{equation}\label{prod:FinishExp}  \: \xcd"finish" \xcd"(" Exp \xcd")" Block & (\arabic{equation})\\
 identifier  \refstepcounter{equation}\label{prod:identifier}  \: \xcd"IDENTIFIER"  & (\arabic{equation})\\
 TypeName  \refstepcounter{equation}\label{prod:TypeName}  \: Id & (\arabic{equation})\\
    \| TypeName \xcd"." Id\\
 ClassName  \refstepcounter{equation}\label{prod:ClassName}  \: TypeName & (\arabic{equation})\\
 TypeArguments  \refstepcounter{equation}\label{prod:TypeArguments}  \: \xcd"[" TypeArgumentList \xcd"]" & (\arabic{equation})\\
 TypeArgumentList  \refstepcounter{equation}\label{prod:TypeArgumentList}  \: Type & (\arabic{equation})\\
    \| TypeArgumentList \xcd"," Type\\
 PackageName  \refstepcounter{equation}\label{prod:PackageName}  \: Id & (\arabic{equation})\\
    \| PackageName \xcd"." Id\\
 ExpName  \refstepcounter{equation}\label{prod:ExpName}  \: Id & (\arabic{equation})\\
    \| AmbiguousName \xcd"." Id\\
 MethodName  \refstepcounter{equation}\label{prod:MethodName}  \: Id & (\arabic{equation})\\
    \| AmbiguousName \xcd"." Id\\
 PackageOrTypeName  \refstepcounter{equation}\label{prod:PackageOrTypeName}  \: Id & (\arabic{equation})\\
    \| PackageOrTypeName \xcd"." Id\\
 AmbiguousName  \refstepcounter{equation}\label{prod:AmbiguousName}  \: Id & (\arabic{equation})\\
    \| AmbiguousName \xcd"." Id\\
 CompilationUnit  \refstepcounter{equation}\label{prod:CompilationUnit}  \: PackageDecl\opt TypeDecls\opt & (\arabic{equation})\\
    \| PackageDecl\opt ImportDecls TypeDecls\opt\\
    \| ImportDecls PackageDecl  ImportDecls\opt  TypeDecls\opt\\
    \| PackageDecl ImportDecls PackageDecl  ImportDecls\opt  TypeDecls\opt\\
\end{bbgrammar}

\begin{bbgrammar}

 ImportDecls  \refstepcounter{equation}\label{prod:ImportDecls}  \: ImportDecl & (\arabic{equation})\\
    \| ImportDecls ImportDecl\\
 TypeDecls  \refstepcounter{equation}\label{prod:TypeDecls}  \: TypeDecl & (\arabic{equation})\\
    \| TypeDecls TypeDecl\\
 PackageDecl  \refstepcounter{equation}\label{prod:PackageDecl}  \: Annotations\opt \xcd"package" PackageName \xcd";" & (\arabic{equation})\\
 ImportDecl  \refstepcounter{equation}\label{prod:ImportDecl}  \: SingleTypeImportDecl & (\arabic{equation})\\
    \| TypeImportOnDemandDecl\\
 SingleTypeImportDecl  \refstepcounter{equation}\label{prod:SingleTypeImportDecl}  \: \xcd"import" TypeName \xcd";" & (\arabic{equation})\\
 TypeImportOnDemandDecl  \refstepcounter{equation}\label{prod:TypeImportOnDemandDecl}  \: \xcd"import" PackageOrTypeName \xcd"." \xcd"*" \xcd";" & (\arabic{equation})\\
 TypeDecl  \refstepcounter{equation}\label{prod:TypeDecl}  \: ClassDecl & (\arabic{equation})\\
    \| InterfaceDecl\\
    \| TypeDefDecl\\
    \| \xcd";"\\
 Interfaces  \refstepcounter{equation}\label{prod:Interfaces}  \: \xcd"implements" InterfaceTypeList & (\arabic{equation})\\
 InterfaceTypeList  \refstepcounter{equation}\label{prod:InterfaceTypeList}  \: Type & (\arabic{equation})\\
    \| InterfaceTypeList \xcd"," Type\\
 ClassBody  \refstepcounter{equation}\label{prod:ClassBody}  \: \xcd"{" ClassBodyDecls\opt \xcd"}" & (\arabic{equation})\\
 ClassBodyDecls  \refstepcounter{equation}\label{prod:ClassBodyDecls}  \: ClassBodyDecl & (\arabic{equation})\\
    \| ClassBodyDecls ClassBodyDecl\\
 ClassBodyDecl  \refstepcounter{equation}\label{prod:ClassBodyDecl}  \: ClassMemberDecl & (\arabic{equation})\\
    \| CtorDecl\\
 ClassMemberDecl  \refstepcounter{equation}\label{prod:ClassMemberDecl}  \: FieldDecl & (\arabic{equation})\\
    \| MethodDecl\\
    \| PropertyMethodDecl\\
    \| TypeDefDecl\\
    \| ClassDecl\\
    \| InterfaceDecl\\
    \| \xcd";"\\
 FormalDeclarators  \refstepcounter{equation}\label{prod:FormalDeclarators}  \: FormalDeclarator & (\arabic{equation})\\
    \| FormalDeclarators \xcd"," FormalDeclarator\\
 FieldDeclarators  \refstepcounter{equation}\label{prod:FieldDeclarators}  \: FieldDeclarator & (\arabic{equation})\\
    \| FieldDeclarators \xcd"," FieldDeclarator\\
 VariableDeclaratorsWithType  \refstepcounter{equation}\label{prod:VariableDeclaratorsWithType}  \: VariableDeclaratorWithType & (\arabic{equation})\\
    \| VariableDeclaratorsWithType \xcd"," VariableDeclaratorWithType\\
 VariableDeclarators  \refstepcounter{equation}\label{prod:VariableDeclarators}  \: VariableDeclarator & (\arabic{equation})\\
    \| VariableDeclarators \xcd"," VariableDeclarator\\
 VariableInitializer  \refstepcounter{equation}\label{prod:VariableInitializer}  \: Exp & (\arabic{equation})\\
 ResultType  \refstepcounter{equation}\label{prod:ResultType}  \: \xcd":" Type & (\arabic{equation})\\
 HasResultType  \refstepcounter{equation}\label{prod:HasResultType}  \: \xcd":" Type & (\arabic{equation})\\
    \| \xcd"<:" Type\\
 FormalParamList  \refstepcounter{equation}\label{prod:FormalParamList}  \: FormalParam & (\arabic{equation})\\
\end{bbgrammar}

\begin{bbgrammar}

    \| FormalParamList \xcd"," FormalParam\\
 LoopIndexDeclarator  \refstepcounter{equation}\label{prod:LoopIndexDeclarator}  \: Id HasResultType\opt & (\arabic{equation})\\
    \| \xcd"[" IdList \xcd"]" HasResultType\opt\\
    \| Id \xcd"[" IdList \xcd"]" HasResultType\opt\\
 LoopIndex  \refstepcounter{equation}\label{prod:LoopIndex}  \: Mods\opt LoopIndexDeclarator & (\arabic{equation})\\
    \| Mods\opt VarKeyword LoopIndexDeclarator\\
 FormalParam  \refstepcounter{equation}\label{prod:FormalParam}  \: Mods\opt FormalDeclarator & (\arabic{equation})\\
    \| Mods\opt VarKeyword FormalDeclarator\\
    \| Type\\
 Offers  \refstepcounter{equation}\label{prod:Offers}  \: \xcd"offers" Type & (\arabic{equation})\\
 ExceptionTypeList  \refstepcounter{equation}\label{prod:ExceptionTypeList}  \: ExceptionType & (\arabic{equation})\\
    \| ExceptionTypeList \xcd"," ExceptionType\\
 ExceptionType  \refstepcounter{equation}\label{prod:ExceptionType}  \: ClassType & (\arabic{equation})\\
 MethodBody  \refstepcounter{equation}\label{prod:MethodBody}  \: \xcd"=" LastExp \xcd";" & (\arabic{equation})\\
    \| \xcd"=" Annotations\opt \xcd"{" BlockStatements\opt LastExp \xcd"}"\\
    \| \xcd"=" Annotations\opt Block\\
    \| Annotations\opt Block\\
    \| \xcd";"\\
 CtorBody  \refstepcounter{equation}\label{prod:CtorBody}  \: \xcd"=" CtorBlock & (\arabic{equation})\\
    \| CtorBlock\\
    \| \xcd"=" ExplicitCtorInvocation\\
    \| \xcd"=" AssignPropertyCall\\
    \| \xcd";"\\
 CtorBlock  \refstepcounter{equation}\label{prod:CtorBlock}  \: \xcd"{" ExplicitCtorInvocation\opt BlockStatements\opt \xcd"}" & (\arabic{equation})\\
 Arguments  \refstepcounter{equation}\label{prod:Arguments}  \: \xcd"(" ArgumentList\opt \xcd")" & (\arabic{equation})\\
 InterfaceDecl  \refstepcounter{equation}\label{prod:InterfaceDecl}  \: NormalInterfaceDecl & (\arabic{equation})\\
 ExtendsInterfaces  \refstepcounter{equation}\label{prod:ExtendsInterfaces}  \: \xcd"extends" Type & (\arabic{equation})\\
    \| ExtendsInterfaces \xcd"," Type\\
 InterfaceBody  \refstepcounter{equation}\label{prod:InterfaceBody}  \: \xcd"{" InterfaceMemberDecls\opt \xcd"}" & (\arabic{equation})\\
 InterfaceMemberDecls  \refstepcounter{equation}\label{prod:InterfaceMemberDecls}  \: InterfaceMemberDecl & (\arabic{equation})\\
    \| InterfaceMemberDecls InterfaceMemberDecl\\
 InterfaceMemberDecl  \refstepcounter{equation}\label{prod:InterfaceMemberDecl}  \: MethodDecl & (\arabic{equation})\\
    \| PropertyMethodDecl\\
    \| FieldDecl\\
    \| ClassDecl\\
    \| InterfaceDecl\\
    \| TypeDefDecl\\
    \| \xcd";"\\
 Annotations  \refstepcounter{equation}\label{prod:Annotations}  \: Annotation & (\arabic{equation})\\
    \| Annotations Annotation\\
 Annotation  \refstepcounter{equation}\label{prod:Annotation}  \: \xcd"@" NamedType & (\arabic{equation})\\
\end{bbgrammar}

\begin{bbgrammar}

 Id  \refstepcounter{equation}\label{prod:Id}  \: identifier & (\arabic{equation})\\
 Block  \refstepcounter{equation}\label{prod:Block}  \: \xcd"{" BlockStatements\opt \xcd"}" & (\arabic{equation})\\
 BlockStatements  \refstepcounter{equation}\label{prod:BlockStatements}  \: BlockStatement & (\arabic{equation})\\
    \| BlockStatements BlockStatement\\
 BlockStatement  \refstepcounter{equation}\label{prod:BlockStatement}  \: LocalVariableDeclStatement & (\arabic{equation})\\
    \| ClassDecl\\
    \| TypeDefDecl\\
    \| Statement\\
 IdList  \refstepcounter{equation}\label{prod:IdList}  \: Id & (\arabic{equation})\\
    \| IdList \xcd"," Id\\
 FormalDeclarator  \refstepcounter{equation}\label{prod:FormalDeclarator}  \: Id ResultType & (\arabic{equation})\\
    \| \xcd"[" IdList \xcd"]" ResultType\\
    \| Id \xcd"[" IdList \xcd"]" ResultType\\
 FieldDeclarator  \refstepcounter{equation}\label{prod:FieldDeclarator}  \: Id HasResultType & (\arabic{equation})\\
    \| Id HasResultType\opt \xcd"=" VariableInitializer\\
 VariableDeclarator  \refstepcounter{equation}\label{prod:VariableDeclarator}  \: Id HasResultType\opt \xcd"=" VariableInitializer & (\arabic{equation})\\
    \| \xcd"[" IdList \xcd"]" HasResultType\opt \xcd"=" VariableInitializer\\
    \| Id \xcd"[" IdList \xcd"]" HasResultType\opt \xcd"=" VariableInitializer\\
 VariableDeclaratorWithType  \refstepcounter{equation}\label{prod:VariableDeclaratorWithType}  \: Id HasResultType \xcd"=" VariableInitializer & (\arabic{equation})\\
    \| \xcd"[" IdList \xcd"]" HasResultType \xcd"=" VariableInitializer\\
    \| Id \xcd"[" IdList \xcd"]" HasResultType \xcd"=" VariableInitializer\\
 LocalVariableDeclStatement  \refstepcounter{equation}\label{prod:LocalVariableDeclStatement}  \: LocalVariableDecl \xcd";" & (\arabic{equation})\\
 LocalVariableDecl  \refstepcounter{equation}\label{prod:LocalVariableDecl}  \: Mods\opt VarKeyword VariableDeclarators & (\arabic{equation})\\
    \| Mods\opt VariableDeclaratorsWithType\\
    \| Mods\opt VarKeyword FormalDeclarators\\
 Primary  \refstepcounter{equation}\label{prod:Primary}  \: here & (\arabic{equation})\\
    \| \xcd"[" ArgumentList\opt \xcd"]"\\
    \| Literal\\
    \| \xcd"self"\\
    \| \xcd"this"\\
    \| ClassName \xcd"." \xcd"this"\\
    \| \xcd"(" Exp \xcd")"\\
    \| ClassInstCreationExp\\
    \| FieldAccess\\
    \| MethodInvocation\\
    \| MethodSelection\\
    \| OperatorFunction\\
 OperatorFunction  \refstepcounter{equation}\label{prod:OperatorFunction}  \: TypeName \xcd"." \xcd"+" & (\arabic{equation})\\
    \| TypeName \xcd"." \xcd"-"\\
    \| TypeName \xcd"." \xcd"*"\\
    \| TypeName \xcd"." \xcd"/"\\
\end{bbgrammar}

\begin{bbgrammar}

    \| TypeName \xcd"." \xcd"%"\\
    \| TypeName \xcd"." \xcd"&"\\
    \| TypeName \xcd"." \xcd"|"\\
    \| TypeName \xcd"." \xcd"^"\\
    \| TypeName \xcd"." \xcd"<<"\\
    \| TypeName \xcd"." \xcd">>"\\
    \| TypeName \xcd"." \xcd">>>"\\
    \| TypeName \xcd"." \xcd"<"\\
    \| TypeName \xcd"." \xcd"<="\\
    \| TypeName \xcd"." \xcd">="\\
    \| TypeName \xcd"." \xcd">"\\
    \| TypeName \xcd"." \xcd"=="\\
    \| TypeName \xcd"." \xcd"!="\\
 Literal  \refstepcounter{equation}\label{prod:Literal}  \: \xcd"IntegerLiteral"  & (\arabic{equation})\\
    \| \xcd"LongLiteral" \\
    \| \xcd"UnsignedIntegerLiteral" \\
    \| \xcd"UnsignedLongLiteral" \\
    \| \xcd"FloatingPointLiteral" \\
    \| \xcd"DoubleLiteral" \\
    \| BooleanLiteral\\
    \| \xcd"CharacterLiteral" \\
    \| \xcd"StringLiteral" \\
    \| \xcd"null"\\
 BooleanLiteral  \refstepcounter{equation}\label{prod:BooleanLiteral}  \: \xcd"true"  & (\arabic{equation})\\
    \| \xcd"false" \\
 ArgumentList  \refstepcounter{equation}\label{prod:ArgumentList}  \: Exp & (\arabic{equation})\\
    \| ArgumentList \xcd"," Exp\\
 FieldAccess  \refstepcounter{equation}\label{prod:FieldAccess}  \: Primary \xcd"." Id & (\arabic{equation})\\
    \| \xcd"super" \xcd"." Id\\
    \| ClassName \xcd"." \xcd"super"  \xcd"." Id\\
    \| Primary \xcd"." \xcd"class" \\
    \| \xcd"super" \xcd"." \xcd"class" \\
    \| ClassName \xcd"." \xcd"super"  \xcd"." \xcd"class" \\
 MethodInvocation  \refstepcounter{equation}\label{prod:MethodInvocation}  \: MethodName TypeArguments\opt \xcd"(" ArgumentList\opt \xcd")" & (\arabic{equation})\\
    \| Primary \xcd"." Id TypeArguments\opt \xcd"(" ArgumentList\opt \xcd")"\\
    \| \xcd"super" \xcd"." Id TypeArguments\opt \xcd"(" ArgumentList\opt \xcd")"\\
    \| ClassName \xcd"." \xcd"super"  \xcd"." Id TypeArguments\opt \xcd"(" ArgumentList\opt \xcd")"\\
    \| Primary TypeArguments\opt \xcd"(" ArgumentList\opt \xcd")"\\
 MethodSelection  \refstepcounter{equation}\label{prod:MethodSelection}  \: MethodName \xcd"." \xcd"(" FormalParamList\opt \xcd")" & (\arabic{equation})\\
    \| Primary \xcd"." Id \xcd"." \xcd"(" FormalParamList\opt \xcd")"\\
    \| \xcd"super" \xcd"." Id \xcd"." \xcd"(" FormalParamList\opt \xcd")"\\
\end{bbgrammar}

\begin{bbgrammar}

    \| ClassName \xcd"." \xcd"super"  \xcd"." Id \xcd"." \xcd"(" FormalParamList\opt \xcd")"\\
 PostfixExp  \refstepcounter{equation}\label{prod:PostfixExp}  \: CastExp & (\arabic{equation})\\
    \| PostIncrementExp\\
    \| PostDecrementExp\\
 PostIncrementExp  \refstepcounter{equation}\label{prod:PostIncrementExp}  \: PostfixExp \xcd"++" & (\arabic{equation})\\
 PostDecrementExp  \refstepcounter{equation}\label{prod:PostDecrementExp}  \: PostfixExp \xcd"--" & (\arabic{equation})\\
 UnannotatedUnaryExp  \refstepcounter{equation}\label{prod:UnannotatedUnaryExp}  \: PreIncrementExp & (\arabic{equation})\\
    \| PreDecrementExp\\
    \| \xcd"+" UnaryExpNotPlusMinus\\
    \| \xcd"-" UnaryExpNotPlusMinus\\
    \| UnaryExpNotPlusMinus\\
 UnaryExp  \refstepcounter{equation}\label{prod:UnaryExp}  \: UnannotatedUnaryExp & (\arabic{equation})\\
    \| Annotations UnannotatedUnaryExp\\
 PreIncrementExp  \refstepcounter{equation}\label{prod:PreIncrementExp}  \: \xcd"++" UnaryExpNotPlusMinus & (\arabic{equation})\\
 PreDecrementExp  \refstepcounter{equation}\label{prod:PreDecrementExp}  \: \xcd"--" UnaryExpNotPlusMinus & (\arabic{equation})\\
 UnaryExpNotPlusMinus  \refstepcounter{equation}\label{prod:UnaryExpNotPlusMinus}  \: PostfixExp & (\arabic{equation})\\
    \| \xcd"~" UnaryExp\\
    \| \xcd"!" UnaryExp\\
 MultiplicativeExp  \refstepcounter{equation}\label{prod:MultiplicativeExp}  \: UnaryExp & (\arabic{equation})\\
    \| MultiplicativeExp \xcd"*" UnaryExp\\
    \| MultiplicativeExp \xcd"/" UnaryExp\\
    \| MultiplicativeExp \xcd"%" UnaryExp\\
 AdditiveExp  \refstepcounter{equation}\label{prod:AdditiveExp}  \: MultiplicativeExp & (\arabic{equation})\\
    \| AdditiveExp \xcd"+" MultiplicativeExp\\
    \| AdditiveExp \xcd"-" MultiplicativeExp\\
 ShiftExp  \refstepcounter{equation}\label{prod:ShiftExp}  \: AdditiveExp & (\arabic{equation})\\
    \| ShiftExp \xcd"<<" AdditiveExp\\
    \| ShiftExp \xcd">>" AdditiveExp\\
    \| ShiftExp \xcd">>>" AdditiveExp\\
 RangeExp  \refstepcounter{equation}\label{prod:RangeExp}  \: ShiftExp & (\arabic{equation})\\
    \| ShiftExp  \xcd".." ShiftExp \\
 RelationalExp  \refstepcounter{equation}\label{prod:RelationalExp}  \: RangeExp & (\arabic{equation})\\
    \| SubtypeConstraint\\
    \| RelationalExp \xcd"<" RangeExp\\
    \| RelationalExp \xcd">" RangeExp\\
    \| RelationalExp \xcd"<=" RangeExp\\
    \| RelationalExp \xcd">=" RangeExp\\
    \| RelationalExp \xcd"instanceof" Type\\
    \| RelationalExp \xcd"in" ShiftExp\\
 EqualityExp  \refstepcounter{equation}\label{prod:EqualityExp}  \: RelationalExp & (\arabic{equation})\\
    \| EqualityExp \xcd"==" RelationalExp\\
\end{bbgrammar}

\begin{bbgrammar}

    \| EqualityExp \xcd"!=" RelationalExp\\
    \| Type  \xcd"==" Type \\
 AndExp  \refstepcounter{equation}\label{prod:AndExp}  \: EqualityExp & (\arabic{equation})\\
    \| AndExp \xcd"&" EqualityExp\\
 ExclusiveOrExp  \refstepcounter{equation}\label{prod:ExclusiveOrExp}  \: AndExp & (\arabic{equation})\\
    \| ExclusiveOrExp \xcd"^" AndExp\\
 InclusiveOrExp  \refstepcounter{equation}\label{prod:InclusiveOrExp}  \: ExclusiveOrExp & (\arabic{equation})\\
    \| InclusiveOrExp \xcd"|" ExclusiveOrExp\\
 ConditionalAndExp  \refstepcounter{equation}\label{prod:ConditionalAndExp}  \: InclusiveOrExp & (\arabic{equation})\\
    \| ConditionalAndExp \xcd"&&" InclusiveOrExp\\
 ConditionalOrExp  \refstepcounter{equation}\label{prod:ConditionalOrExp}  \: ConditionalAndExp & (\arabic{equation})\\
    \| ConditionalOrExp \xcd"||" ConditionalAndExp\\
 ConditionalExp  \refstepcounter{equation}\label{prod:ConditionalExp}  \: ConditionalOrExp & (\arabic{equation})\\
    \| ClosureExp\\
    \| AtExp\\
    \| FinishExp\\
    \| ConditionalOrExp \xcd"?" Exp \xcd":" ConditionalExp\\
 AssignmentExp  \refstepcounter{equation}\label{prod:AssignmentExp}  \: Assignment & (\arabic{equation})\\
    \| ConditionalExp\\
 Assignment  \refstepcounter{equation}\label{prod:Assignment}  \: LeftHandSide AssignmentOperator AssignmentExp & (\arabic{equation})\\
    \| ExpName  \xcd"(" ArgumentList\opt \xcd")" AssignmentOperator AssignmentExp\\
    \| Primary  \xcd"(" ArgumentList\opt \xcd")" AssignmentOperator AssignmentExp\\
 LeftHandSide  \refstepcounter{equation}\label{prod:LeftHandSide}  \: ExpName & (\arabic{equation})\\
    \| FieldAccess\\
 AssignmentOperator  \refstepcounter{equation}\label{prod:AssignmentOperator}  \: \xcd"=" & (\arabic{equation})\\
    \| \xcd"*="\\
    \| \xcd"/="\\
    \| \xcd"%="\\
    \| \xcd"+="\\
    \| \xcd"-="\\
    \| \xcd"<<="\\
    \| \xcd">>="\\
    \| \xcd">>>="\\
    \| \xcd"&="\\
    \| \xcd"^="\\
    \| \xcd"|="\\
 Exp  \refstepcounter{equation}\label{prod:Exp}  \: AssignmentExp & (\arabic{equation})\\
 ConstantExp  \refstepcounter{equation}\label{prod:ConstantExp}  \: Exp & (\arabic{equation})\\
 PrefixOp  \refstepcounter{equation}\label{prod:PrefixOp}  \: \xcd"+" & (\arabic{equation})\\
    \| \xcd"-"\\
    \| \xcd"!"\\
\end{bbgrammar}

\begin{bbgrammar}

    \| \xcd"~"\\
 BinOp  \refstepcounter{equation}\label{prod:BinOp}  \: \xcd"+" & (\arabic{equation})\\
    \| \xcd"-"\\
    \| \xcd"*"\\
    \| \xcd"/"\\
    \| \xcd"%"\\
    \| \xcd"&"\\
    \| \xcd"|"\\
    \| \xcd"^"\\
    \| \xcd"&&"\\
    \| \xcd"||"\\
    \| \xcd"<<"\\
    \| \xcd">>"\\
    \| \xcd">>>"\\
    \| \xcd">="\\
    \| \xcd"<="\\
    \| \xcd">"\\
    \| \xcd"<"\\
    \| \xcd"=="\\
    \| \xcd"!="\\
\end{bbgrammar}


\renewcommand{\bibname}{References}
\bibliographystyle{plain}
\bibliography{master}

%%\extrapart{Bibliography and references}

% My reference for proper reference format is:
%    Mary-Claire van Leunen.
%    {\em A Handbook for Scholars.}
%    Knopf, 1978.
% I think the references list would look better in ``open'' format,
% i.e. with the three blocks for each entry appearing on separate
% lines.  I used the compressed format for SIGPLAN in the interest of
% space.  In open format, when a block runs over one line,
% continuation lines should be indented; this could probably be done
% using some flavor of latex list environment.  Maybe the right thing
% to do in the long run would be to convert to Bibtex, which probably
% does the right thing, since it was implemented by one of van
% Leunen's colleagues at DEC SRC.
%  -- Jonathan

% This is just a personal remark on your question on the RRRS:
% The language CUCH (Curry-Church) was implemented by 1964 and 
% is a practical version of the lambda-calculus (call-by-name).
% One reference you may find in Formal Language Description Languages
% for Computer Programming T.~B.~Steele, 1965 (or so).
%  -- Matthias Felleisen


\begin{thebibliography}{99}

\bibitem{SICP}
Harold Abelson and Gerald Jay Sussman with Julie Sussman.
{\em Structure and Interpretation of Computer Programs.}
MIT Press, Cambridge, 1985.

\bibitem{readfloat}
William Clinger.
How to read floating point numbers accurately.
In {\em Proceedings of the 1990 ACM SIGPLAN Conference on Programming
  Language Design and Implementation}.  Forthcoming.

University of Oregon Technical Report CIS-TR-90-01.

\bibitem{RRRS}
William Clinger, editor.
The revised revised report on Scheme, or an uncommon Lisp.
MIT Artificial Intelligence Memo 848, August 1985.
Also published as Computer Science Department Technical Report 174,
  Indiana University, June 1985.

\bibitem{R4RS}
William Clinger and Jonathan Rees, editors.
The revised$^4$ report on the algorithmic language Scheme.
University of Oregon Technical Report CIS-TR-90-02.

\bibitem{Scheme311}
Carol Fessenden, William Clinger, Daniel P.~Friedman, and Christopher Haynes.
Scheme 311 version 4 reference manual.
Indiana University Computer Science Technical Report 137, February 1983.
Superceded by~\cite{Scheme84}.

\bibitem{Scheme84}
D.~Friedman, C.~Haynes, E.~Kohlbecker, and M.~Wand.
Scheme 84 interim reference manual.
Indiana University Computer Science Technical Report 153, January 1985.

\bibitem{CFractions}
G.~H.~Hardy and E.~M.~Wright.
{\em An Introduction to the Theory of Numbers.} 5th ed.
Oxford University Press, New York NY, 1979.

\bibitem{Haskell}
Paul Hudak and Philip Wadler, editors.
Report on the Functional Programming Language Haskell.
Yale University Research Report YALEU/DCS/RR-666, December 1988.

\bibitem{IEEE}
{\em IEEE Standard 754-1985.  IEEE Standard for Binary Floating-Point
Arithmetic.}  IEEE, New York, 1985.

\bibitem{Knuth}
Donald E. Knuth.
The Art of Computer Programming, volume 2: Seminumerical Algorithms.
Addison-Wesley, Reading MA, 1969.

\bibitem{Landin65}
Peter Landin.
A correspondence between Algol 60 and Church's lambda notation: Part I.
{\em Communications of the ACM} 8(2):89--101, February 1965.

\bibitem{Matula68}
David W. Matula.
In-and-Out Conversions.
{\em Communications of the ACM} 11(1):47--50, January 1968.

\bibitem{Matula70}
David W. Matula.
A Formalization of Floating-Point Numeric Base Conversion.
{\em IEEE Transactions on Computers} C-19, 8:681-692, August 1970.

\bibitem{MITScheme}
MIT Department of Electrical Engineering and Computer Science.
Scheme manual, seventh edition.
September 1984.

\bibitem{Penfield81}
Paul Penfield, Jr.
Principal values and branch cuts in complex APL.
In {\em APL '81 Conference Proceedings,} pages 248--256.
ACM SIGAPL, San Francisco, September 1981.
Proceedings published as {\em APL Quote Quad} 12(1), ACM, September 1981.

\bibitem{Pitman83}
Kent M.~Pitman.
The revised MacLisp manual (Saturday evening edition).
MIT Laboratory for Computer Science Technical Report 295, May 1983.

\bibitem{Rees82}
Jonathan A.~Rees and Norman I.~Adams IV.
T: A dialect of Lisp or, lambda: The ultimate software tool.
In {\em Conference Record of the 1982 ACM Symposium on Lisp and
  Functional Programming}, pages 114--122.

\bibitem{R3RS}
Jonathan Rees and William Clinger, editors.
The revised$^3$ report on the algorithmic language Scheme.
In {\em ACM SIGPLAN Notices} 21(12), ACM, December 1986.

\bibitem{Reynolds72}
John Reynolds.
Definitional interpreters for higher order programming languages.
In {\em ACM Conference Proceedings}, pages 717--740.
ACM, \todo{month?}~1972.

\bibitem{Rabbit}
Guy Lewis Steele Jr.
Rabbit: a compiler for Scheme.
MIT Artificial Intelligence Laboratory Technical Report 474, May 1978.

\bibitem{CLtL}
Guy Lewis Steele Jr.
{\em Common Lisp: The Language.}
Digital Press, Burlington MA, 1984.

\bibitem{CLtL2}
Guy Lewis Steele Jr.
{\em Common Lisp: The Language.} 2d ed.
Digital Press, Bedford MA, 1990.

\bibitem{Scheme78}
Guy Lewis Steele Jr.~and Gerald Jay Sussman.
The revised report on Scheme, a dialect of Lisp.
MIT Artificial Intelligence Memo 452, January 1978.

\bibitem{Heuristic}
Guy Lewis Steele Jr.~and Jon L White.
How to Print Floating-Point Numbers Accurately.
In {\em Proceedings of the 1990 ACM SIGPLAN Conference on Programming
  Language Design and Implementation}.  Forthcoming.

\bibitem{Stoy77}
Joseph E.~Stoy.
{\em Denotational Semantics: The Scott-Strachey Approach to
  Programming Language Theory.}
MIT Press, Cambridge, 1977.

\bibitem{Scheme75}
Gerald Jay Sussman and Guy Lewis Steele Jr.
Scheme: an interpreter for extended lambda calculus.
MIT Artificial Intelligence Memo 349, December 1975.

\bibitem{Vuillemin}
Jean Vuillemin.
Exact real computer arithmetic with continued fractions.
In {\em Proceedings of the 1988 ACM Conference on Lisp and
  Functional Programming}, pages 14--27.

\end{thebibliography}
	

% Adjustment to avoid having the last index entry on a page by itself.
%\addtolength{\baselineskip}{-0.1pt}

\clearpage
%\documentclass[10pt,twoside,twocolumn,notitlepage]{report}
%\documentclass[12pt,twoside,notitlepage]{report}
\documentclass[10pt,twoside,notitlepage]{report}
\usepackage{tex/x10}
\usepackage{tex/tenv}
\def\Hat{{\tt \char`\^}}
\usepackage{url}
\usepackage{times}
\usepackage{tex/txtt}
\usepackage{ifpdf}
\usepackage{tocloft}
\usepackage{tex/bcprules}
\usepackage{xspace}
\usepackage{makeidx}

\newif\ifdraft
%\drafttrue
\draftfalse

\pagestyle{headings}
\showboxdepth=0
\makeindex

\usepackage{tex/commands}

\usepackage[
pdfauthor={Vijay Saraswat, Bard Bloom, Igor Peshansky, Olivier Tardieu, and David Grove},
pdftitle={X10 Language Specification},
pdfcreator={pdftex},
pdfkeywords={X10},
linkcolor=blue,
citecolor=blue,
urlcolor=blue
]{hyperref}

\ifpdf
          \pdfinfo {
              /Author   (Vijay Saraswat, Bard Bloom, Igor Peshansky, Olivier Tardieu, and David Grove)
              /Title    (X10 Language Specification)
              /Keywords (X10)
              /Subject  ()
              /Creator  (TeX)
              /Producer (PDFLaTeX)
          }
\fi

\def\headertitle{The \XtenCurrVer{} Report}
\def\integerversion{2.4}

% Sizes and dimensions

%\topmargin -.375in       %    Nominal distance from top of page to top of
                         %    box containing running head.
%\headsep 15pt            %    Space between running head and text.

%\textheight 9.0in        % Height of text (including footnotes and figures, 
                         % excluding running head and foot).

%\textwidth 5.5in         % Width of text line.
\columnsep 15pt          % Space between columns 
\columnseprule 0pt       % Width of rule between columns.

\parskip 5pt plus 2pt minus 2pt % Extra vertical space between paragraphs.
\parindent 0pt                  % Width of paragraph indentation.
%\topsep 0pt plus 2pt            % Extra vertical space, in addition to 
                                % \parskip, added above and below list and
                                % paragraphing environments.


\newif\iftwocolumn

\makeatletter
\twocolumnfalse
\if@twocolumn
\twocolumntrue
\fi
\makeatother

\iftwocolumn

\oddsidemargin  0in    % Left margin on odd-numbered pages.
\evensidemargin 0in    % Left margin on even-numbered pages.

\else

\oddsidemargin  .5in    % Left margin on odd-numbered pages.
\evensidemargin .5in    % Left margin on even-numbered pages.

\fi


\newtenv{example}{Example}[section]
\newtenv{planned}{Planned}[section]

\begin{document}

% \section{Work In Progress}
% \begin{itemize}
%     \item Rewrite first chapter
%     \item Describe library classes, including such fundamentals as Any and String
%     \item Examples for covariant/contravariant generics are wrong -- use Nate's examples
%     \item Describe local classes.
%     \item Reduce the use of \xcd`self` in constraints.
%     \item Copy sections of grammar to relevant sections of text.
%     \item Do something about 4.12.3
% \end{itemize}
% 
% {\bf Feedback:} 
% To help us the most, we would appreciate comments in one of these formats: 
% \begin{itemize}
% \item An annotated copy of the PDF document, if it's convenient.  Acrobat
%       Writer can produce helpful highlighting and sticky notes.  If you don't
%       use Acrobat Writer, don't fuss.
% \item Text comments.  Since the document is still being edited, page numbers
%       are going to be useless as pointers to the text.  If possible, we'd like
%       pointers to sections by number and title: {\em In 12.1, ``Empty
%       Statement'', please discuss side effects and performance implications
%       for this construct''}  If it's a long section, giving us a couple words
%       we can grep for would help too.
% \end{itemize}
% 
% Thank you very much!




% \parindent 0pt %!! 15pt                    % Width of paragraph indentation.

%\hfil {\bf 7 Feb 2005}
%\hfil \today{}

% First page

\thispagestyle{empty}

% \todo{"another" report?}

\title{ \Xten Language Specification \\
\large Version \integerversion}
%\ifdraft
\author{Vijay Saraswat, Bard Bloom, Igor Peshansky, Olivier Tardieu, and David Grove\\
\\
Please send comments to 
\texttt{vsaraswa@us.ibm.com}}
%\else
%\author{
%Vijay Saraswat \\
%Please send comments to \\
%\texttt{vsaraswa@us.ibm.com}}
%\fi
%\date{\today}

\maketitle

\newcommand\authorsc[1]{#1}
%\newcommand\authorsc[1]{\textsc{#1}}

This report provides a description of the programming
language \Xten. \Xten{} is a class-based object-oriented
programming language designed for high-performance, high-productivity
computing on high-end computers supporting $\approx 10^5$ hardware threads
and $\approx 10^{15}$ operations per second. 

\Xten{} is based on state-of-the-art object-oriented programming
languages and deviates from them only as necessary to support its
design goals. The language is intended to have a simple and clear
semantics and be readily accessible to mainstream OO programmers. It
is intended to support a wide variety of concurrent programming
idioms.
%, incuding data parallelism, task parallelism, pipelining.
%producer/consumer and divide and conquer.

%We expect to revise this document in the light of experience gained in implementing
%and using this language.

The \Xten{} design team consists of
\authorsc{David Cunningham},
\authorsc{David Grove},
\authorsc{Ben Herta},
\authorsc{Vijay Saraswat},
\authorsc{Avraham Shinnar},
\authorsc{Mikio Takeuchi},
\authorsc{Olivier Tardieu}.

Past members include
\authorsc{Shivali Agarwal}, 
\authorsc{Bowen Alpern}, 
\authorsc{David Bacon}, 
\authorsc{Raj Barik}, 
\authorsc{Ganesh Bikshandi}, 
\authorsc{Bob Blainey}, 
\authorsc{Bard Bloom}, 
\authorsc{Philippe Charles}, 
\authorsc{Perry Cheng}, 
\authorsc{Christopher Donawa}, 
\authorsc{Julian Dolby}, 
\authorsc{Kemal Ebcio\u{g}lu},
\authorsc{Stephen Fink},
\authorsc{Robert Fuhrer},
\authorsc{Patrick Gallop}, 
\authorsc{Christian Grothoff}, 
\authorsc{Hiroshi Horii}, 
\authorsc{Kiyokuni Kawachiya}, 
\authorsc{Allan Kielstra}, 
\authorsc{Sreedhar Kodali}, 
\authorsc{Sriram Krishnamoorthy}, 
\authorsc{Yan Li}, 
\authorsc{Bruce Lucas},
\authorsc{Yuki Makino}, 
\authorsc{Nathaniel Nystrom},
\authorsc{Igor Peshansky}, 
\authorsc{Vivek Sarkar},
\authorsc{Armando Solar-Lezama},  
\authorsc{S. Alexander Spoon}, 
\authorsc{Toshio Suganuma}, 
\authorsc{Sayantan Sur}, 
\authorsc{Toyotaro Suzumura}, 
\authorsc{Christoph von Praun},
\authorsc{Leena Unnikrishnan},
\authorsc{Pradeep Varma}, 
\authorsc{Krishna Nandivada Venkata},
\authorsc{Jan Vitek}, 
\authorsc{Hai Chuan Wang}, 
\authorsc{Tong Wen}, 
\authorsc{Salikh Zakirov}, and
\authorsc{Yoav Zibin}.


For extended discussions and support we would like to thank: 
Gheorghe Almasi,
Robert Blackmore,
Rob O'Callahan, 
Calin Cascaval, 
Norman Cohen, 
Elmootaz Elnozahy, 
John Field,
Kevin Gildea,
Chulho Kim,
Orren Krieger, 
Doug Lea, 
John McCalpin, 
Paul McKenney, 
Josh Milthorpe,
Andrew Myers,
Filip Pizlo, 
Ram Rajamony,
R.~K. Shyamasundar, 
V.~T. Rajan, 
Frank Tip,
Mandana Vaziri,
and
Hanhong Xue.


We thank Jonathan Rhees and William Clinger with help in obtaining the
\LaTeX{} style file and macros used in producing the Scheme report,
on which this document is based. We acknowledge the influence of
the $\mbox{\Java}^{\mbox{\textsc{tm}}}$ Language
Specification \cite{jls2}, the Scala language specification
\cite{scala-spec}, and ZPL \cite{zpl}.
%document, as evidenced by the numerous citations in the text.

This document specifies the language corresponding to Version
\integerversion{} of the implementation. The redesign and reimplementation of arrays and rails was  done by Dave Grove and Olivier Tardieu.
 Version 1.7 of the report was co-authored by Nathaniel Nystrom. The design of structs in \Xten{} was led by Olivier Tardieu and Nathaniel Nystrom.

Earlier implementations benefited from significant contributions by
Raj Barik, 
Philippe Charles, 
David Cunningham,
Christopher Donawa, 
Robert Fuhrer,
Christian Grothoff,
Nathaniel Nystrom,  
Igor Peshansky,  
Vijay Saraswat,
Vivek Sarkar, 
Olivier Tardieu,  
Pradeep Varma, 
Krishna Nandivada Venkata, and
Christoph von Praun.
Tong Wen has written many application programs
in \Xten{}. Guojing Cong has helped in the
development of many applications.
The implementation of generics in \Xten{} was influenced by the
implementation of PolyJ~\cite{polyj} by Andrew Myers and Michael Clarkson.
 

\clearpage

{\parskip 0pt
\addtolength{\cftsecnumwidth}{0.5em}
\addtolength{\cftsubsecnumwidth}{0.5em}
%\addtolength{\cftsecindent}{0.5em}
\addtolength{\cftsubsecindent}{0.5em}
\tableofcontents
}


\chapter{Introduction}

\subsection*{Background}



The era of the mighty single-processor computer is over. Now, when more
computing power is needed, one does not buy a faster uniprocessor---one buys
another processor just like those one already has, or another hundred, or
another million, and connects them with a high-speed communication network.
Or, perhaps, one rents them instead, with a cloud computer. This gives one
whatever quantity of computer cycles that one can desire and afford.

Then, one has the problem of how to use those computer cycles effectively.
Programming a multiprocessor is far more agonizing than programming a
uniprocessor.   One can use models of computation which give somewhat of the
illusion of programming a uniprocessor.  Unfortunately, the models which give
the closest imitations of uniprocessing are very expensive to implement,
either increasing the monetary cost of the computer tremendously, or slowing
it down dreadfully. 

One response to this problem has been to move to a {\em fragmented memory
  model}. Multiple processors are programmed largely as if they were
uniprocessors, but are made to interact via a relatively language-neutral
message-passing format such as MPI \cite{mpi}. This model has enjoyed some
success: several high-performance applications have been written in this
style. Unfortunately, this model leads to a {\em loss of programmer
  productivity}: the message-passing format is integrated into the host
language by means of an application-programming interface (API), the
programmer must explicitly represent and manage the interaction between
multiple processes and choreograph their data exchange; large data-structures
(such as distributed arrays, graphs, hash-tables) that are conceptually
unitary must be thought of as fragmented across different nodes; all
processors must generally execute the same code (in an SPMD fashion) etc.

One response to this problem has been the advent of the {\em
partitioned global address space} (PGAS) model underlying languages
such as UPC, Titanium and Co-Array Fortran \cite{pgas,titanium}. These
languages permit the programmer to think of a single computation
running across multiple processors, sharing a common address
space. All data resides at some processor, which is said to have {\em
affinity} to the data.  Each processor may operate directly on the
data it contains but must use some indirect mechanism to access or
update data at other processors. Some kind of global {\em barriers}
are used to ensure that processors remain roughly in lock-step.

\Xten{} is a modern object-oriented programming language
in the PGAS family. The fundamental goal of \Xten{} is to enable
scalable, high-performance, high-productivity transformational
programming for high-end computers---for traditional numerical
computation workloads (such as weather simulation, molecular dynamics,
particle transport problems etc) as well as commercial server
workloads.

\Xten{} is based on state-of-the-art object-oriented
programming ideas primarily to take advantage of their proven
flexibility and ease-of-use for a wide spectrum of programming
problems. \Xten{} takes advantage of several years of research (e.g.,
in the context of the Java Grande forum,
\cite{moreira00java,kava}) on how to adapt such languages to the context of
high-performance numerical computing. Thus \Xten{} provides support
for user-defined {\em struct types} (such as \xcd"Int", \xcd"Float",
\xcd"Complex" etc), supports a very
flexible form of multi-dimensional arrays (based on ideas in ZPL
\cite{zpl}) and supports IEEE-standard floating point arithmetic.
Some capabilities for supporting operator overloading are also provided.

{}\Xten{} introduces a flexible treatment of concurrency, distribution
and locality, within an integrated type system. \Xten{} extends the
PGAS model with {\em asynchrony} (yielding the {\em APGAS} programming
model). {}\Xten{} introduces {\em places} as an abstraction for a
computational context with a locally synchronous view of shared
memory. An \Xten{} computation runs over a large collection of places.
Each place hosts some data and runs one or more {\em
activities}. Activities are extremely lightweight threads of
execution. An activity may synchronously (and {\em atomically}) use
one or more memory locations in the place in which it resides,
leveraging current symmetric multiprocessor (SMP) technology.  
An activity may shift to another place to execute a statement block.
\Xten{} provides weaker ordering guarantees for
inter-place data access, enabling applications to scale.  
Multiple memory locations in multiple places cannot be
accessed atomically.  {\em
Immutable} data needs no consistency management and may be freely
copied by the implementation between places.  One or more {\em clocks}
may be used to order activities running in multiple
places.  \xcd`DistArray`s, distributed arrays,  may be distributed across
multiple 
places and  support parallel collective operations. A novel
exception flow model ensures that exceptions thrown by asynchronous
activities can be caught at a suitable parent activity.  The type
system tracks which memory accesses are local. The programmer may
introduce place casts which verify the access is local at run time.
Linking with native code is supported.

\chapter{Overview of \Xten}

\Xten{} is a statically typed object-oriented language, extending a sequential
core language with \emph{places}, \emph{activities}, \emph{clocks},
(distributed, multi-dimensional) \emph{arrays} and \emph{struct} types. All
these changes are motivated by the desire to use the new language for
high-end, high-performance, high-productivity computing.

\section{Object-oriented features}

The sequential core of \Xten{} is a {\em container-based} object-oriented language
similar to \java{} and C++, and more recent languages such as Scala.  
Programmers write \Xten{} code by defining containers for data and behavior
called 
\emph{classes}
(\Sref{XtenClasses}) and
\emph{structs}
(\Sref{XtenStructs}), 
often abstracted as 
\emph{interfaces}
(\Sref{XtenInterfaces}).
X10 provides inheritance and subtyping in fairly traditional ways. 

\begin{ex}

\xcd`Normed` describes entities with a \xcd`norm()` method. \xcd`Normed` is
intended to be used for entities with a position in some coordinate system,
and \xcd`norm()` gives the distance between the entity and the origin. A
\xcd`Slider` is an object which can be moved around on a line; a
\xcd`PlanePoint` is a fixed position in a plane. Both \xcd`Slider`s and
\xcd`PlanePoint`s have a sensible \xcd`norm()` method, and implement
\xcd`Normed`.

%~~gen ^^^ Overview10
% package Overview;
%~~vis
\begin{xten}
interface Normed {
  def norm():Double;
}
class Slider implements Normed {
  var x : Double = 0;
  public def norm() = Math.abs(x);
  public def move(dx:Double) { x += dx; }
}
struct PlanePoint implements Normed {
  val x : Double, y:Double;
  public def this(x:Double, y:Double) {
    this.x = x; this.y = y;
  }
  public def norm() = Math.sqrt(x*x+y*y);
}
\end{xten}
%~~siv
%
%~~neg
\end{ex}

\paragraph{Interfaces}

An \Xten{} interface specifies a collection of abstract methods; \xcd`Normed`
specifies just \xcd`norm()`. Classes and
structs can be specified to {\em implement} interfaces, as \xcd`Slider` and
\xcd`PlanePoint` implement \xcd`Normed`, and, when they do so, must provide
all the methods that the interface demands.

Interfaces are
purely abstract. Every value of type \xcd`Normed` must be an instance of some
class like \xcd`Slider` or some struct like \xcd`PlanePoint` which implements
\xcd`Normed`; no value can be \xcd`Normed` and nothing else. 


\paragraph{Classes and Structs}



There are two kinds of containers: \emph{classes}
(\Sref{ReferenceClasses}) and \emph{structs} (\Sref{Structs}). Containers hold
data in {\em fields}, and give concrete implementations of 
methods, as \xcd`Slider` and \xcd`PlainPoint` above.

Classes are organized in a single-inheritance tree: a class may have only a
single parent class, though it may implement many interfaces and have many
subclasses. Classes may have mutable fields, as \xcd`Slider` does.

In contrast, structs are headerless values, lacking the internal organs
which give objects their intricate behavior.  This makes them less powerful
than objects (\eg, structs cannot inherit methods, though objects can), but also
cheaper (\eg, they can be inlined, and they require less space than objects).  
Structs are immutable, though their fields may be immutably set to objects
which are themselves mutable.  They behave like objects in all ways consistent
with these limitations; \eg, while they cannot {\em inherit} methods, they can
have them -- as \xcd`PlanePoint` does.

\Xten{} has no primitive classes per se. However, the standard library
\xcd"x10.lang" supplies structs and objects \xcd"Boolean", \xcd"Byte",
\xcd"Short", \xcd"Char", \xcd"Int", \xcd"Long", \xcd"Float", \xcd"Double",
\xcd"Complex" and \xcd"String". The user may defined additional arithmetic
structs using the facilities of the language.



\paragraph{Functions.}

X10 provides functions (\Sref{Closures}) to allow code to be used
as values.  Functions are first-class data: they can be stored in lists,
passed between activities, and so on.  \xcd`square`, below, is a function
which squares an \xcd`Int`.  \xcd`of4` takes an \xcd`Int`-to-\xcd`Int`
function and applies it to the number \xcd`4`.  So, \xcd`fourSquared` computes
\xcd`of4(square)`, which is \xcd`square(4)`, which is 16, in a fairly
complicated way.
%~~gen ^^^ Overview20
% package Overview.of.Functions.one;
% class Whatever{
% def chkplz() {
%~~vis
\begin{xten}
  val square = (i:Int) => i*i;
  val of4 = (f: (Int)=>Int) => f(4);
  val fourSquared = of4(square);
\end{xten}
%~~siv
%}}
%~~neg



Functions are used extensively in X10
programs.  For example, a common way to construct and initialize an \xcd`Array[Int](1)` --
that is, a fixed-length one-dimensional array of numbers, like an \xcd`int[]` in Java -- is to
pass two arguments to a factory method: the first argument being the length of
the array, and the second being a function which computes the initial value of
the \xcd`i`{$^{th}$} element.  The following code constructs a 1-dimensional
array 
initialized to the squares of 0,1,...,9: \xcd`r(0) == 0`, \xcd`r(5)==25`, etc. 
%~~gen ^^^ Overview30
% package Overview.of.Functions.two;
% class Whatevermore {
%  def plzchk(){
%    val square = (i:Int) => i*i;
%~~vis
\begin{xten}
  val r : Array[Int](1) = new Array[Int](10, square);
\end{xten}
%~~siv
%}}
%~~neg








\paragraph{Constrained Types}

X10 containers may declare {\em properties}, which are fields bound immutably
at the creation of the container.  The static analysis system understands
properties, and can work with them logically.   


For example, an implementation of matrices \xcd`Mat` might have the numbers of
rows and columns as properties.  A little bit of care in definitions allows
the definition of a \xcd`+` operation that works on matrices of the same
shape, and \xcd`*` that works on matrices with appropriately matching shapes.
%~~gen ^^^ Overview40
%package Overview.Mat2;
%~~vis
\begin{xten}
abstract class Mat(rows:Int, cols:Int) {
 static type Mat(r:Int, c:Int) = Mat{rows==r&&cols==c};
 abstract operator this + (y:Mat(this.rows,this.cols))
                 :Mat(this.rows, this.cols);
 abstract operator this * (y:Mat) {this.cols == y.rows} 
                 :Mat(this.rows, y.cols);
\end{xten}
%~~siv
%  static def makeMat(r:Int,c:Int) : Mat(r,c) = null;
%  static def example(a:Int, b:Int, c:Int) {
%    val axb1 : Mat(a,b) = makeMat(a,b);
%    val axb2 : Mat(a,b) = makeMat(a,b);
%    val bxc  : Mat(b,c) = makeMat(b,c);
%    val axc  : Mat(a,c) = (axb1 +axb2) * bxc;
%  }
%}
%~~neg



The following code typechecks (assuming that \xcd`makeMat(m,n)` is a function
which creates an \xcdmath"m$\times$n" matrix).
However, an attempt to compute \xcd`axb1 + bxc` or
\xcd`bxc * axb1` would result in a compile-time type error:
%~~gen ^^^ Overview50
%package Overview.Mat1;
% // OPTIONS: -STATIC_CALLS 
%abstract class Mat(rows:Int, cols:Int) {
%  static type Mat(r:Int, c:Int) = Mat{rows==r&&cols==c};
%  public def this(r:Int, c:Int) : Mat(r,c) = {property(r,c);}
%  static def makeMat(r:Int,c:Int) : Mat(r,c) = null;
%  abstract  operator this + (y:Mat(this.rows,this.cols)):Mat(this.rows, this.cols);
%  abstract  operator this * (y:Mat) {this.cols == y.rows} : Mat(this.rows, y.cols);
%~~vis
\begin{xten}
  static def example(a:Int, b:Int, c:Int) {
    val axb1 : Mat(a,b) = makeMat(a,b);
    val axb2 : Mat(a,b) = makeMat(a,b);
    val bxc  : Mat(b,c) = makeMat(b,c);
    val axc  : Mat(a,c) = (axb1 +axb2) * bxc;
    //ERROR: val wrong1 = axb1 + bxc;
    //ERROR: val wrong2 = bxc * axb1;
  }

\end{xten}
%~~siv
%}
%~~neg

The ``little bit of care'' shows off many of the features of constrained
types.    
The \xcd`(rows:Int, cols:Int)` in the class definition declares two
properties, \xcd`rows` and \xcd`cols`.\footnote{The class is officially declared
abstract to allow for multiple implementations, like sparse and band matrices,
but in fact is abstract to avoid having to write the actual definitions of
\xcd`+` and \xcd`*`.}  

A constrained type looks like \xcd`Mat{rows==r && cols==c}`: a type
name, followed by a Boolean expression in braces.  
The \xcd`type` declaration on the second line makes
\xcd`Mat(r,c)` be a synonym for \xcd`Mat{rows==r && cols==c}`,
allowing for compact types in many places.

Functions can return constrained types.  
The \xcd`makeMat(r,c)` method returns a \xcd`Mat(r,c)` -- a matrix whose shape
is given by the arguments to the method.    In
particular, constructors can have constrained return types to provide specific
information about the constructed values.

The arguments of methods can have type constraints as well.  The 
\xcd`operator this +` line lets \xcd`A+B` add two matrices.  The type of the
second argument \xcd`y` is constrained to have the same number of rows and
columns as the first argument \xcd`this`. Attempts to add mismatched matrices
will be flagged as type errors at compilation.

At times it is more convenient to put the constraint on the method as a whole,
as seen in the \xcd`operator this *` line. Unlike for \xcd`+`, there is no
need to constrain both dimensions; we simply need to check that the columns of
the left factor match the rows of the right. This constraint is written in
\xcd`{...}` after the argument list.  The shape of the result is computed from
the shapes of the arguments.

And that is all that is necessary for a user-defined class of matrices to have
shape-checking for matrix addition and multiplication.  The \xcd`example`
method compiles under those definitions.








\paragraph{Generic types}

Containers may have type parameters, permitting the definition of
{\em generic types}.  Type parameters may be instantiated by any X10 type.  It
is thus possible to make a list of integers \xcd`List[Int]`, a list of
non-zero integers \xcd`List[Int{self != 0}]`, or a list of people
\xcd`List[Person]`.  In the definition of \xcd`List`, \xcd`T` is a type
parameter; it can be instantiated with any type.
%~~gen ^^^ Overview60
%~~vis
\begin{xten}
class List[T] {
    var head: T;
    var tail: List[T];
    def this(h: T, t: List[T]) { head = h; tail = t; }
    def add(x: T) {
        if (this.tail == null)
            this.tail = new List[T](x, null);
        else
            this.tail.add(x);
    }
}
\end{xten}
%~~siv
%~~neg
The constructor (\xcd"def this") initializes the fields of the new object.
The \xcd"add" method appends an element to the list.
\xcd"List" is a generic type.  When  instances of \xcd"List" are
allocated, the type \param{} \xcd"T" must be bound to a concrete
type.  \xcd"List[Int]" is the type of lists of element type
\xcd"Int", \xcd"List[List[String]]" is the type of lists whose elements are
themselves lists of string, and so on.

%%BARD-HERE

\section{The sequential core of X10}

The sequential aspects of X10 are mostly familiar from C and its progeny.
\Xten{} enjoys the familiar control flow constructs: \xcd"if" statements,
\xcd"while" loops, \xcd"for" loops, \xcd"switch" statements, \xcd`throw` to
raise exceptions and \xcd`try...catch` to handle them, and so on.

X10 has both implicit coercions and explicit conversions, and both can be
defined on user-defined types.  Explicit conversions are written with the
\xcd`as` operation: \xcd`n as Int`.  The types can be constrained: 
%~~exp~~`~~`~~n:Int~~ ^^^ Overview70
\xcd`n as Int{self != 0}` converts \xcd`n` to a non-zero integer, and throws a
runtime exception if its value as an integer is zero.

\section{Places and activities}

The full power of X10 starts to emerge with concurrency.
An \Xten{} program is intended to run on a wide range of computers,
from uniprocessors to large clusters of parallel processors supporting
millions of concurrent operations. To support this scale, \Xten{}
introduces the central concept of \emph{place} (\Sref{XtenPlaces}).
A place can be thought of as a virtual shared-memory multi-processor:
a computational unit with a finite (though perhaps changing) number of
hardware threads and a bounded amount of shared memory, uniformly
accessible by all threads.



An \Xten{} computation acts on \emph{values}(\Sref{XtenObjects}) through
the execution of lightweight threads called
\emph{activities}(\Sref{XtenActivities}). 
An {\em object}
 has a small, statically fixed set of fields, each of
which has a distinct name. A scalar object is located at a single place and
stays at that place throughout its lifetime. An \emph{aggregate} object has
many fields (the number may be known only when the object is created),
uniformly accessed through an index (\eg, an integer) and may be distributed
across many places. The distribution of an aggregate object remains unchanged
throughout the computation, thought different aggregates may be distributed
differently. Objects are garbage-collected when no longer useable; there are
no operations in the language to allow a programmer to explicitly release
memory.

{}\Xten{} has a \emph{unified} or \emph{global address space}. This means that
an activity can reference objects at other places. However, an activity may
synchronously access data items only in the current place, the place in which
it is running. It may atomically update one or more data items, but only in
the current place.   If it becomes necessary to read or modify an object at
some other place \xcd`q`, the {\em place-shifting} operation \xcd`at(q;F)` can
be used, to move part of the activity to \xcd`q`.  \xcd`F` is a specification
of what information will be sent to \xcd`q` for use by that part of the
computation. 
It is easy to compute
across multiple places, but the expensive operations (\eg, those which require
communication) are readily visible in the code. 

\paragraph{Atomic blocks.}

X10 has a control construct \xcd"atomic S" where \xcd"S" is a statement with
certain restrictions. \xcd`S` will be executed atomically, without
interruption by other activities. This is a common primitive used in
concurrent algorithms, though rarely provided in this degree of generality by
concurrent programming languages.

More powerfully -- and more expensively -- X10 allows conditional atomic
blocks, \xcd`when(B)S`, which are executed atomically at some point when
\xcd`B` is true.  Conditional atomic blocks are one of the strongest
primitives used in concurrent algorithms, and one of the least-often
available. 

\paragraph{Asynchronous activities.}

An asynchronous activity is created by a statement \xcd"async S", which starts
up a new activity running \xcd`S`.  It does not wait for the new activity to
finish; there is a separate statement (\xcd`finish`) to do that.




\section{Clocks}
The MPI style of coordinating the activity of multiple processes with
a single barrier is not suitable for the dynamic network of heterogeneous
activities in an \Xten{} computation.  
X10 allows multiple barriers in a form that supports determinate,
deadlock-free parallel computation, via the \xcd`Clock` type.

A single \xcd`Clock` represents a computation that occurs in phases.
At any given time, an activity is {\em registered} with zero or more clocks.
The X10 statement \xcd`next` tells all of an activity's registered clocks that
the activity has finished the current phase, and causes it to wait for the
next phase.  Other operations allow waiting on a single clock, starting
new clocks or new activities registered on an extant clock, and so on. 

%%INTRO-CLOCK%  Activities may use clocks to repeatedly detect quiescence of arbitrary
%%INTRO-CLOCK%  programmer-specified, data-dependent set of activities. Each activity
%%INTRO-CLOCK%  is spawned with a known set of clocks and may dynamically create new
%%INTRO-CLOCK%  clocks. At any given time an activity is \emph{registered} with zero or
%%INTRO-CLOCK%  more clocks. It may register newly created activities with a clock,
%%INTRO-CLOCK%  un-register itself with a clock, suspend on a clock or require that a
%%INTRO-CLOCK%  statement (possibly involving execution of new async activities) be
%%INTRO-CLOCK%  executed to completion before the clock can advance.  At any given
%%INTRO-CLOCK%  step of the execution a clock is in a given phase. It advances to the
%%INTRO-CLOCK%  next phase only when all its registered activities have \emph{quiesced}
%%INTRO-CLOCK%  (by executing a \xcd"next" operation on the clock).
%%INTRO-CLOCK%  When a clock advances, all its activities may now resume execution.
%%INTRO-CLOCK%  

Clocks act as {barriers} for a dynamically varying collection of activities.
They generalize the barriers found in MPI style program in that an activity
may use multiple clocks simultaneously. Yet programs using clocks properly are
guaranteed not to suffer from deadlock.

%%HERE

\section{Arrays, regions and distributions}

X10 provides \xcd`DistArray`s, {\em distributed arrays}, which spread data
across many places. An underlying \xcd`Dist` object provides the {\em
distribution}, telling which elements of the \xcd`DistArray` go in which
place. \xcd`Dist` uses subsidiary \xcd`Region` objects to abstract over the
shape and even the dimensionality of arrays.
Specialized X10 control statements such as \xcd`ateach` provide efficient
parallel iteration over distributed arrays.


\section{Annotations}

\Xten{} supports annotations on classes and interfaces, methods
and constructors,
variables, types, expressions and statements.
These annotations may be processed by compiler plugins.

\section{Translating MPI programs to \Xten{}}

While \Xten{} permits considerably greater flexibility in writing
distributed programs and data structures than MPI, it is instructive
to examine how to translate MPI programs to \Xten.

Each separate MPI process can be translated into an \Xten{}
place. Async activities may be used to read and write variables
located at different processes. A single clock may be used for barrier
synchronization between multiple MPI processes. \Xten{} collective
operations may be used to implement MPI collective operations.
\Xten{} is more general than MPI in (a)~not requiring synchronization
between two processes in order to enable one to read and write the
other's values, (b)~permitting the use of high-level atomic blocks
within a process to obtain mutual exclusion between multiple
activities running in the same node (c)~permitting the use of multiple
clocks to combine the expression of different physics (e.g.,
computations modeling blood coagulation together with computations
involving the flow of blood), (d)~not requiring an SPMD style of
computation.


%\note{Relaxed exception model}
\section{Summary and future work}
\subsection{Design for scalability}
\Xten{} is designed for scalability, by encouraging working with local data,
and limiting the ability of events at one place to delay those at another. For
example, an activity may atomically access only multiple locations in the
current place. Unconditional atomic blocks are dynamically guaranteed to be
non-blocking, and may be implemented using non-blocking techniques that avoid
mutual exclusion bottlenecks. 
%TODO: yoav says: ``no idea what [the following] means''
Data-flow synchronization permits point-to-point
coordination between reader/writer activities, obviating the need for
barrier-based or lock-based synchronization in many cases.

\subsection{Design for productivity}
\Xten{} is designed for productivity.

\paragraph{Safety and correctness.}

\bard{Confirm some of these claims}

Programs written in \Xten{} are guaranteed to be statically
\emph{type safe}, \emph{memory safe} and \emph{pointer safe}. 

Static type safety guarantees that every location contains only values whose
dynamic type agrees with the location's static type. The compiler allows a
choice of how to handle method calls. In strict mode, method calls are
statically checked to be permitted by the static types of operands. In lax
mode, dynamic checks are inserted when calls may or may not be correct,
providing weaker static correctness guarantees but more programming
convenience. 

Memory safety guarantees that an object may only access memory within its
representation, and other objects it has a reference to. \Xten{} does not
permit 
pointer arithmetic, and bound-checks array accesses dynamically if necessary.
\Xten{} uses garbage collection to collect objects no longer referenced by any
activity. \Xten{} guarantees that no object can retain a reference to an
object whose memory has been reclaimed. Further, \Xten{} guarantees that every
location is initialized at run time before it is read, and every value read
from a word of memory has previously been written into that word.

%XXX
%Pointer safety guarantees that a null pointer exception cannot be
%thrown by an operation on a value of a non-nullable type.

Because places are reflected in the type system, static type safety
also implies \emph{place safety}. All operations that need to be performed
locally are, in fact, performed locally.  All data which is declared to be
stored locally are, in fact, stored locally.

\Xten{} programs that use only clocks and unconditional atomic
blocks are guaranteed not to deadlock. Unconditional atomic blocks
are non-blocking, hence cannot introduce deadlocks.
Many concurrent programs can be shown to be determinate (hence
race-free) statically.

\paragraph{Integration.}
A key issue for any new programming language is how well it can be
integrated with existing (external) languages, system environments,
libraries and tools.

%TODO: Yoav asks ``you mean interop''?
We believe that \Xten{}, like \java{}, will be able to support a large
number of libraries and tools. An area where we expect future versions
of \Xten{} to improve on \java{} like languages is \emph{native
integration} (\Sref{NativeCode}). Specifically, \Xten{} will permit
permit multi-dimensional local arrays to be operated on natively by
native code.

\subsection{Conclusion}
{}\Xten{} is considerably higher-level than thread-based languages in
that it supports dynamically spawning lightweight activities, the
use of atomic operations for mutual exclusion, and the use of clocks
for repeated quiescence detection.

Yet it is much more concrete than languages like HPF in that it forces
the programmer to explicitly deal with distribution of data
objects. In this the language reflects the designers' belief that
issues of locality and distribution cannot be hidden from the
programmer of high-performance code in high-end computing.  A
performance model that distinguishes between computation and
communication must be made explicit and transparent.\footnote{In this
\Xten{} is similar to more modern languages such as ZPL \cite{zpl}.}
At the same time we believe that the place-based type system and
support for generic programming will allow the \Xten{} programmer to
be highly productive; many of the tedious details of
distribution-specific code can be handled in a generic fashion.

\chapter{Lexical structure}


Lexically a program consists of a stream of white space, comments,
identifiers, keywords, literals, separators and operators, all of them
composed of ASCII characters. 

\paragraph{Whitespace}
\index{white space}
% Whitespace \index{whitespace} follows \java{} rules \cite[Chapter 3.6]{jls2}.
ASCII space, horizontal tab (HT), form feed (FF) and line
terminators constitute white space.

\paragraph{Comments}
\index{comment}
% Comments \index{comments} follows \java{} rules
% \cite[Chapter 3.7]{jls2}. 
All text included within the ASCII characters ``\xcd"/*"'' and
``\xcd"*/"'' is
considered a comment and ignored; nested comments are not
allowed.  All text from the ASCII characters
``\xcd"//"'' to the end of line is considered a comment and is ignored.

\paragraph{Identifiers}
\index{identifier}
\index{variable name}

Identifiers consist of a single letter followed by zero or more
letters or digits.
The letters are the ASCII characters \xcd`a` through \xcd`z`, \xcd`A` through
\xcd`Z`, and \xcd`_`.
Digits are defined as the ASCII characters \xcd"0" through \xcd"9". Case is
significant; \xcd`a` and \xcd`A` are distinct identifiers, \xcd`as` is a
keyword, but \xcd`As` and \xcd`AS` are identifiers.

\paragraph{Keywords}
\index{keywords}
\Xten{} reserves the following keywords:
\begin{xten}
abstract       false          offers         transient      
as             final          operator       true           
assert         finally        package        try            
async          finish         private        var            
ateach         for            property       when           
break          goto           protected      while          
case           if             public         at             
catch          implements     return         atomic         
class          import         self           await          
continue       in             static         clocked        
def            instanceof     struct         here           
default        interface      super          next           
do             native         switch         offer          
else           new            this           resume         
extends        null           throw          type           
\end{xten}
Note that the primitive types are not considered keywords.

\paragraph{Literals}\label{Literals}\index{literal}

Briefly, \XtenCurrVer{} uses fairly standard syntax for its literals:
integers, unsigned integers, floating point numbers, booleans, 
characters, strings, and \xcd"null".  The most exotic points are (1) unsigned
numbers are marked by a \xcd`u` and cannot have a sign; (2) \xcd`true` and
\xcd`false` are the literals for the booleans; and (3) floating point numbers
are \xcd`Double` unless marked with an \xcd`f` for \xcd`Float`. 

Less briefly, we use the following abbreviations: 
\begin{displaymath}
\begin{array}{rcll}
d &=& \mbox{one or more decimal digits}\\
d_8 &=& \mbox{one or more octal digits}\\
d_{16} &=& \mbox{one or more hexadecimal digits, using \xcd`a`-\xcd`f`
for 10-15}\\
i &=& d 
        \mathbin{|} {\tt 0} d_8 
        \mathbin{|} {\tt 0x} d_{16}
        \mathbin{|} {\tt 0X} d_{16}
\\
s &=& \mbox{optional \xcd`+` or \xcd`-`}\\
b &=& d 
          \mathbin{|} d {\tt .}
          \mathbin{|} d {\tt .} d
          \mathbin{|}  {\tt .} d \\
x &=& ({\tt e } \mathbin{|} {\tt E})
         s
         d \\
f &=& b x
\end{array}
\end{displaymath}

\begin{itemize}

\item \xcd`true` and \xcd`false` are the \xcd`Boolean` literals. \index{Boolean!literal}\index{literal!Boolean}

\item \xcd`null` is a literal for the null value.  It has type
      \xcd`Any{self==null}`. \index{null} \index{object!literal}

\item \index{Int!literal}\index{literal!integer}
\xcd`Int` literals have the form {$si$}; \eg, \xcd`123`,
      \xcd`-321` are decimal \xcd`Int`s, \xcd`0123` and \xcd`-0321` are octal
      \xcd`Int`s, and \xcd`0x123`, \xcd`-0X321`,  \xcd`0xBED`, and \xcd`0XEBEC` are
      hexadecimal \xcd`Int`s.  

\item \xcd`Long` literals have the form {$si{\tt l}$} or
      {$si{\tt L}$}. \Eg, \xcd`1234567890L`  and \xcd`0xBAGEL` are \xcd`Long` literals. 

\item \xcd`UInt` literals have the form {$i{\tt u}$} or {$i {\tt U}$}.
      \Eg, \xcd`123u`, \xcd`0123u`, and \xcd`0xBEAU` are \xcd`UInt` literals.

\item \xcd`ULong` literals have the form {$i {\tt ul}$} or {$i {\tt
      lu}$}, or capital versions of those.  For example, 
      \xcd`123ul`, \xcd`0124567012ul`,  \xcd`0xFLU`, \xcd`OXba1eful`, and \xcd`0xDecafC0ffeefUL` are \xcd`ULong`
      literals. 

\item \xcd`Short` literals have the form {$si{\tt s}$} or
      {$si{\tt S}$}. \Eg,  414S, \xcd`OxACES` and \xcd`7001s` are short
      literals. 

\item \xcd`UShort` literals  form {$i {\tt us}$} or {$i {\tt
      su}$}, or capital versions of those.  For example, \xcd`609US`, 
      \xcd`107us`, and \xcd`OxBeaus` are unsigned short literals.

\item \xcd`Byte` literals have the form  {$si{\tt y}$} or
      {$si{\tt Y}$}.  (The letter \xcd`B` cannot be used for bytes, as it is
      a hexadecimal digit.)  \xcd`50Y` and \xcd`OxBABY` are byte literals.

\item \xcd`UByte` literals have the form {$i {\tt uy}$} or {$i {\tt yu}$}, or
      capitalized versions of those.  For example, \xcd`9uy` and \xcd`OxBUY`
      are \xcd`UByte` literals.
      


\item \xcd`Float` literals have the form {$s f {\tt f}$} or  {$s
\index{float!literal}
\index{literal!float}
      f {\tt F}$}.  Note that the floating-point marker letter \xcd`f` is
      required: unmarked floating-point-looking literals are \xcd`Double`. 
      \Eg, \xcd`1f`, \xcd`6.023E+32f`, \xcd`6.626068E-34F` are \xcd`Float`
      literals. 

\item \xcd`Double` literals have the form {$s f$}\footnote{Except that
\index{double!literal}
\index{literal!double}
      literals like \xcd`1` 
      which match both {$i$} and {$f$} are counted as
      integers, not \xcd`Double`; \xcd`Double`s require a decimal
      point, an exponent, or the \xcd`d` marker.
      }, {$s f {\tt
      D}$}, and {$s f {\tt d}$}.  
      \Eg, \xcd`0.0`, \xcd`0e100`, \xcd`1.3D`,  \xcd`229792458d`, and \xcd`314159265e-8`
      are \xcd`Double` literals.

\item 
\index{char!literal}
\index{literal!char}
\xcd`Char` literals have one of the following forms: 
      \begin{itemize}
      \item \xcd`'`{\it c}\xcd`'` where {\em c} is any printing ASCII
            character other than 
            \xcd`\` or \xcd`'`, representing the character {\em c} itself; 
            \eg, \xcd`'!'`;
      \item \xcd`'\b'`, representing backspace;
      \item \xcd`'\t'`, representing tab;
      \item \xcd`'\n'`, representing newline;
      \item \xcd`'\f'`, representing form feed;
      \item \xcd`'\r'`, representing return;
      \item \xcd`'\''`, representing single-quote;
      \item \xcd`'\"'`, representing double-quote;
      \item \xcd`'\\'`, representing backslash;
      \item \xcd`'\`{\em dd}\xcd`'`, where {\em dd} is one or more octal
            digits, representing the one-byte character numbered {\em dd}; it
            is an error if {\em dd}{$>0377$}.      
      \end{itemize}

\item
\index{string!literal} 
\index{literal!string}
\xcd`String` literals consist of a double-quote \xcd`"`, followed by
      zero or more of the contents of a \xcd`Char` literal, followed by
      another double quote.  \Eg, \xcd`"hi!"`, \xcd`""`.


\end{itemize}



\paragraph{Separators}
\Xten{} has the following separators and delimiters:
\begin{xten}
( )  { }  [ ]  ;  ,  .
\end{xten}

\paragraph{Operators}
\index{operator}
\Xten{} has the following operator,  type constructor, and miscellaneous symbols.  (\xcd`?` and
\xcd`:` comprise a single ternary operator, but are written separately.)
\begin{xten}
==  !=  <   >   <=  >=
&&  ||  &   |   ^
<<  >>  >>>
+   -   *   /   %
++  --  !   ~
&=  |=  ^=
<<= >>= >>>=
+=  -=  *=  /=  %=
=   ?   :  =>  ->
<:  :>  @   ..
\end{xten}





\chapter{Types}
\label{XtenTypes}\index{types}

{}\Xten{} is a {\em strongly typed} object-oriented language: every
variable and expression has a type that is known at compile-time.
Types limit the values that variables can hold.

{}\Xten{} supports three kinds of runtime entities, {\em objects},
{\em structs}, and {\em functions}. Objects are instances of {\em
  classes} (\Sref{ReferenceClasses}). They may contain zero or
more mutable fields, and a reference to the list of methods defined on them.

An object is represented by some (contiguous) memory chunk on the
heap. Entities (such as variables and fields) contain a {\em
  reference} to this chunk. That is, objects are represented through
an extra level of indirection.  A consequence of this flexibility is
that an entity containing a reference to an object \xcd{o} needs only
one word of memory to represent that reference, regardess of the
number of fields in \xcd{o}. An assignment to this entity simply
overwrites the reference with another reference (thus taking constant
time). Another consequence is that every class type contains the value
\Xcd{null} corresponding to the invalid reference. \Xcd{null} is often
useful as a default value. Further, two objects may be compared for
identity (\Xcd{==}) in constant time by simply comparing references to
the memory used to represent the objects. The default hash code for an
object is based on the value of this reference. A downside of this
flexibility is that the operations of accessing a field and invoking a
method are more expensive than simply reading a register and
invoking a static function.


Structs are instances of {\em struct types} (\Sref{StructClasses}).  A
struct is represented without the extra level of indirection, with a
memory chunk of size $N$ words precisely big enough to store the value
of every field of the struct (modulo alignment), plus whatever padding is needed. Thus structs cannot
be shared. Entities (such as variables and fields) refering to the
struct must allocate $N$ words to directly contain the chunk.  An
assignment to this entity must copy the $N$ words representing the
right hand side into the left hand side. Since there are no references
to structs, \Xcd{null} is not a legal value for a struct
type. Comparison for identity (\Xcd{==}) involves examining $N$
words. Additionally, structs do not have any mutable fields, hence
they can be freely copied. The payoff for these restrictions lies in
that fields can be stored in registers or local variables, and 
and method invocation is implemented by invoking a static function.

Functions, called closures, lambda-expressions, and blocks in other languages, are
instances of {\em function types} (\Sref{Functions}). 
A function has zero or more {\em formal parameters} (or {\em arguments}) and a
{\em body}, which is 
an expression that can reference the formal parameters and also other
variables in the surrounding block. For instance, \xcd`(x:Int)=>x*y`
is a unary integer function which multiplies its argument by the
variable \xcd`y` from the surrounding block.  Functions may be freely
copied from place to place and may be repeatedly applied. 

These runtime entities are classified by {\em types}. Types are used in
variable declarations, coercions and  explicit conversions, object creation,
array creation, static state and method accessors, and
\xcd"instanceof" and \xcd`as` expressions.


The basic relationship between values and types is the {\em is an
element of} relation.  We also often say ``$e$ has type $T$'' to
mean ``$e$ is an element of type $T$''.  For example, \xcd`1` has type
\xcd`Int` (the type of all integers representible in 32 bits). It also
has type \xcd`Any` (since all entitites have type \xcd`Any`), type
\xcd`Int{self != 0}` (the type of nonzero integers), type
\xcd`Int{self == 1}` (the type of integers which are equal to \xcd`1`, which
contains only one element), and many others. 

The basic relationship between types is {\em subtyping}: \xcd`T <: U`
holds if every instance of \xcd`T` is also an instance of \xcd`U`. Two
important kinds of subtyping are {\em subclassing} and {\em
  strengthening}. Subclassing is a familiar notion from
object-oriented programming. Here we use it to refer to the
relationship between a class and another class it extends, and the
relationship between a class and another interface it implements. For
instance, in a class hierarchy with classes \xcd`Animal` and \xcd`Cat`
such that \xcd`Cat` extends \xcd`Mammal` and \xcd`Mammal` extends
\xcd`Animal`, every instance of \xcd`Cat` is by definition an instance
of \xcd`Animal` (and \xcd`Mammal`). We say that \xcd`Cat` is a
subclass of \xcd`Animal`, or \xcd`Cat <: Animal` by subclassing. If
\xcd`Animal` implements \xcd`Thing`, then \xcd`Cat` also implements
\xcd`Thing`, and we say \xcd`Cat <: Thing` by subclassing.
Strengthening is an equally familiar notion from logic.  The instances
of \xcd`Int{self == 1}` are all elements of \xcd`Int{self != 0}` as well,
because \xcd`self == 1` logically implies \xcd`self != 0`; so 
\xcd`Int{self  == 1} <: Int{self !=0}` by strengthening.  X10 uses both notions
of subtyping.  See \Sref{DepType:Equivalence} for the full definition
of subtyping in X10.

\subsection{Type System}
\index{type system}
The types in X10 are as follows.  

These are the {\em elementary} types. Other
syntactic forms for types exist, but they are simply abbreviations for types
in the following system.  For example, \xcd`Array[Int](1)` is the type of
one-dimensional integer-valued arrays; it is an abbreviation for
\xcd`Array[Int]{rank==1}`.\\

% remove \refstepcounter{equation}
% snag the argument of \label{X}
% change the (\arabic{equation}) into (\ref{X})

%##(Type FunctionType ConstrainedType
\begin{bbgrammar}
%(FROM #(prod:Type)#)
                Type \: FunctionType & (\ref{prod:Type}) \\
                     \| ConstrainedType \\
                     \| VoidType \\
%(FROM #(prod:FunctionType)#)
        FunctionType \: TypeParams\opt \xcd"(" FormalList\opt \xcd")" Guard\opt Offers\opt \xcd"=>" Type & (\ref{prod:FunctionType}) \\
%(FROM #(prod:ConstrainedType)#)
     ConstrainedType \: NamedType & (\ref{prod:ConstrainedType}) \\
                     \| AnnotatedType \\
                     \| \xcd"(" Type \xcd")" \\
\end{bbgrammar}
%##)


Types may be given by name. 
For example, 
%~~type~~`~~`~~ ~~ ^^^ Types10
\xcd`Int`
is the type of 32-bit integers.
Given a class declaration 
%~~gen ^^^ Types20
%package Types.Core.TypeName; 
%~~vis
\begin{xten}
class Triple { /* ... */ }
\end{xten}
%~~siv
%
%~~neg
the identifier \xcd`Triple` may be used as a type.

The type {\em TypeName \xcd`[` Types{$^?$} \xcd`]`} is an instance of
a {\em generic} (or {\em parameterized}) type. 
 For example,
\xcd`Array[Int]` is the type of arrays of integers. 
\xcd`HashMap[String,Int]` is the type of hash maps from strings to
integers.

The type {\em Type \xcd`{` Constraint \xcd`}`} refers to a constrained type.
{\em Constraint} is a Boolean expression -- written in a {\em very} limited
subset of X10 -- describing the acceptable values of the constrained type.
%~~stmt~~`~~`~~ ~~ ^^^ Types30
For example, \xcd`var n : Int{self != 0};` guarantees that \xcd`n` is always a
non-zero integer. 
%~~stmt~~`~~`~~ ~~class Triple{} ^^^ Types40
Similarly, \xcd`var x : Triple{x != null};` defines a \xcd`Triple`-valued
variable \xcd`x` whose value is never null.

The qualified type {\em Type \xcd`.` Type} refers to an instance of a {\em
nested} type; that is, a class or struct defined inside of another class or
struct, and holding an implicit reference to the outer.  For example, given
the type declaration 
%~~gen ^^^ Types50
% package Types.Core.Hardcore.Qualified;
%~~vis
\begin{xten}
class Outer {
  class Inner { /* ... */ }
}
\end{xten}
%~~siv
%
%~~neg
then 
%~~exp~~`~~`~~ ~~ NOTEST class Outer {class Inner { /* ... */ }} ^^^ Types60
\xcd`(new Outer()).new Inner()` creates a value of type 
%~~type~~`~~`~~ ~~class Outer {class Inner { /* ... */ }} ^^^ Types70
\xcd`Outer.Inner`.

Type variables, {\em TypeVar}, refer to types that are parameters.  For
example, the following class defines a cell in a linked list.  
%~~gen ^^^ Types80
% package Types.Core.Bore.Lore;
%~~vis
\begin{xten}
class LinkedList[X] {
  val head : X;
  val tail : LinkedList[X];
  def this(head:X, tail:LinkedList[X]) {
     this.head = head; this.tail = tail;
  }
}
\end{xten}
%~~siv
%
%~~neg
It doesn't
matter what type the cell's element is, but it has to have {\em some} type.
\xcd`LinkedList[Int]` is a linked list of integers.
\xcd`LinkedList[LinkedList[Byte]]` is a list of lists of bytes.
Note that \xcd`LinkedList` is {\em not} a usable type -- it is missing a type parameter.



The function type 
{\em \xcd`(` Formals{$^?$} \xcd`) =>`  Type} 
refers to functions taking the
listed formal parameters and returning a result of {\em Type}.  In
\XtenCurrVer, function types may not be generic.
The closely-related void function type 
{\em \xcd`(` Formals{$^?$} \xcd`) =>`  \xcd`void`}  takes the listed
parameters and returns no value.
For example, 
\begin{xtenmath}
\xcd`(x:Int) => Int{self != x}` 
\end{xtenmath}
is the type of integer-valued functions which have no fixed points -- that is,
for which the output is an integer different from the input.
An example of such a function is \xcd`(x:Int) => x+1`.
For fundamental reasons, X10 --- or any other computer program --- cannot
tell in general whether a function has any fixed points or not.  So, X10
programs using such types must prove to X10 that they are correct. Often this
will involve a run-time check, expressed as a cast, such as: 
%~~gen ^^^ Types3x7m
% package Types3x7m;
% class Example {
%~~vis
\begin{xten}
  val plus1 : (x:Int) => Int{self != x} = 
     (x:Int) => (x+1) as Int{self != x}; 
\end{xten}
%~~siv
%}
%~~neg


The names of the formal parameters are bound in the type, and may be changed
consistently in the usual way without modifying the type.  
For example, \\
\xcd`(a:Int, b:Int{self!=a})=>Int{self!=a, self!=b}` 
and \\
\xcd`(c:Int, d:Int{self!=c})=>Int{self!=c, self!=d}` \\
are equivalent types.  



\section{Classes, Structs,  and interfaces}
\label{ReferenceTypes}

\subsection{Class types}

\index{type!class}
\index{class}
\index{class declaration}
\index{declaration!class declaration}
\index{declaration!reference class declaration}

A {\em class declaration} (\Sref{XtenClasses}) declares a {\em class type},
giving its name, behavior, data, and relationships to other classes and
interfaces. 

\begin{ex}
The \xcd`Position` class below could describe the position of a slider
control: 
%~~gen ^^^ Types100
% package Types.By.Cripes.Classes;
%~~vis
\begin{xten}
class Position {
  private var x : Int = 0;
  public def move(dx:Int) { x += dx; }
  public def pos() : Int = x;
}
\end{xten}
%~~siv
%
%~~neg
\end{ex}
Class instances, also called objects, are created by constructor calls, 
such as \xcd`new Position()`
Class
instances have fields and methods, type members, and value properties bound at
construction time. In addition, classes have static members: static \xcd`val` fields,
methods, type definitions, and member classes and member interfaces.

Classes may be {\em generic}, \ie, defined with one or more type
parameters (\Sref{TypeParameters}).  

%~~gen ^^^ Types110
%~~vis
\begin{xten}
class Cell[T] {
  var contents : T;
  public def this(t:T) { contents = t;  }
  public def putIn(t:T) { contents = t; }
  public def get() = contents;
  }
\end{xten}
%~~siv
%~~neg


%TODO: Yoav: ``This reasoning is no longer true in the new object model''
%% Why not?
\Xten{} does not permit mutable static state. A fundamental principle of the
X10 model of computation is that all mutable state be local to some place
(\Sref{XtenPlaces}), and, as static variables are
globally available, they
cannot be mutable. When mutable global state is necessary, programmers should
use singleton classes, putting the state in an object and using place-shifting
commands (\Sref{AtStatement}) and atomicity (\Sref{AtomicBlocks}) as necessary
to mutate it safely.

\index{\Xcd{Object}}
\index{\Xcd{x10.lang.Object}}

Classes are structured in a single-inheritance hierarchy. All classes extend
the class \xcd"x10.lang.Object", directly or indirectly. Each class other than
\xcd`Object` extends a single parent class.  \xcd`Object` provides no behaviors
of its own, beyond those required by \xcd`Any`.

\index{class!reference class}
\index{reference class type}
\index{\Xcd{Object}}
\index{\Xcd{x10.lang.Object}}


\index{null}


The null value, represented by the literal
\xcd"null", is a value of every class type \xcd`C`. The type whose values are
all instances of \xcd`C` except 
\xcd`null` can be defined as \xcd`C{self != null}`.

\subsection{Struct Types}

A {\em struct declaration} \Sref{XtenStructs} introduces a {\em struct type}
containing all instances of the struct.  The \xcd`Coords` struct below gives
an immutable position in 3-space: 
%~~gen ^^^ Types120
% package Types.Structs.Coords;
%~~vis
\begin{xten}
struct Position {
  public val x:Double, y:Double, z:Double; 
  def this(x:Double, y:Double, z:Double) {
     this.x = x; this.y = y; this.z = z;
  }
}
\end{xten}
%~~siv
%
%~~neg

Structs have many capabilities of classes: they can have methods, implement
interfaces, and be generic. However, they have certain restrictions; for
example, they cannot contain mutable (\xcd`var`) fields, or inherit from
superclasses. There is no \xcd`null` value for structs. Due to these
restrictions, structs may be implemented more efficiently than objects.


\subsection{Interface types}
\label{InterfaceTypes}

\index{type!interface}
\index{interface}
\index{interface declaration}
\index{declaration!interface declaration}

An {\em interface declaration} (\Sref{XtenInterfaces}) defines an {\em
interface type}, specifying a set of methods 
%type members, 
and properties which must be provided by any class declared to implement the
interface. 


Interfaces can also have static members: static fields, type
definitions, and member classes, structs and interfaces.  However,
interfaces cannot specify that implementing classes must provide
static members or constructors.

\begin{ex}
In the following interface, \xcd`PI` is a static field, 
\xcd`Vec` a static type definition, 
\xcd`Pair` a static member class.
It can't insist that implementations provide a static method 
like \xcd`meth`, or a nullary constructor.
%~~gen ^^^ Types2y3i
% NOTEST
% package Types2y3i;
%~~vis
\begin{xten}
interface Stat {
  static val PI = 3.14159; 
  static type R = Double;
  static class Pair(x:R, y:R) {}
  // ERROR: static def meth():Int;
  // ERROR: static def this();
}
class Example {
  static def example() {
     val p : Stat.Pair = new Stat.Pair(Stat.PI, Stat.PI);
  }
}
\end{xten}
%~~siv
%
%~~neg

\end{ex}

An interface may extend multiple interfaces.  
%~~gen ^^^ Types130
%package Types.For.Snipes.Interfaces;
%~~vis
\begin{xten}
interface Named {
  def name():String;
}
interface Mobile {
  def move(howFar:Int):void;
}
interface Person extends Named, Mobile {}
interface NamedPoint extends Named, Mobile {} 
\end{xten}
%~~siv
%
%~~neg


Classes and structs may be declared to implement multiple interfaces. Semantically, the
interface type is the set of all objects that are instances of classes
or structs that
implement the interface. A class or struct implements an interface if it is declared to
and if it concretely or abstractly implements all the methods and properties
defined in the interface. For example, \xcd`Kim` implements
\xcd`Person`, and hence \xcd`Named` and \xcd`Mobile`. It would be a static
error if \xcd`Kim` had no \xcd`name` method, unless \xcd`Kim` were also
declared \xcd`abstract`.

%~~gen ^^^ Types140
%interface Named {
%   def name():String;
% }
% interface Mobile {
%   def move(howFar:Int):void;
% }
% interface Person extends Named, Mobile {}
% interface NamedPoint extends Named, Mobile{} 
%~~vis
\begin{xten}
class Kim implements Person {
   var pos : Int = 0;
   public def name() = "Kim (" + pos + ")";
   public def move(dPos:Int) { pos += dPos; }
}
\end{xten}
%~~siv
%
%~~neg


\subsection{Properties}
\index{property}
\label{properties}

Classes, interfaces, and structs may have {\em properties}, specified in
parentheses after the type name. Properties are much like public \xcd`val`
instance fields. They have certain restrictions on their use, however, which
allows the compiler to understand them much better than other public \xcd`val`
fields. In particular, they can be used in types.  \Eg, the number of elements
in an array is a property of the array, and an X10 program can specify that
two arrays have the same number of elements.

\begin{ex}
The
following code declares a class named \xcd"Coords" with properties
\xcd"x" and \xcd"y" and a \xcd"move" method. The properties are bound
using the \xcd"property" statement in the constructor.

%~~gen ^^^ Types150
%package not.x10.lang;
%~~vis
\begin{xten}
class Coords(x: Int, y: Int) { 
  def this(x: Int, y: Int) :
    Coords{self.x==x, self.y==y} = { 
    property(x, y); 
  } 

  def move(dx: Int, dy: Int) = new Coords(x+dx, y+dy); 
}
\end{xten}
%~~siv
%~~neg
\end{ex}
Properties, unlike other public \xcd`val` fields, can be used  
at compile time in {constraints}. This allows us
to specify subtypes based on properties, by appending a boolean expression to
the type. For example, the type \xcd"Coords{x==0}" is the set of all points
whose \xcd"x" property is \xcd"0".  Details of this substantial topic are
found in \Sref{ConstrainedTypes}.



\section{Type Parameters and Generic Types}
\label{TypeParameters}

\index{type!parameter}
\index{method!parametrized}
\index{constructor!parametrized}
\index{closure!parametrized}
\label{Generics}
\index{type!generic}

A class, interface, method, or type definition  may have type
parameters.  Type parameters can be used as types, and will be bound to types
on instantiation.  
For example, a generic stack class may be defined as 
\xcd`Stack[T]{...}`.  Stacks can hold values of any type; \eg, 
%~~type~~`~~`~~ ~~class Stack[T]{} ^^^ Types160
\xcd`Stack[Int]` is a stack of integers, and 
%~~type~~`~~`~~ ~~class Stack[T]{} ^^^ Types170
\xcd`Stack[Point {self!=null}]` is a stack of non-null \xcd`Point`s.
Generics {\em must} be instantiated when they are used: \xcd`Stack`, by
itself, is not a valid type.
Type parameters may be constrained by a guard on the declaration
(\Sref{TypeDefGuard},
\Sref{MethodGuard},\Sref{ClosureGuard}).

\index{type!concrete}
\index{concrete type}
A {\em generic class} (or struct, interface, or type definition) 
is a class (resp. struct, interface, or type definition) 
declared with $k \geq 1$ type parameters. 
A generic class (or struct, interface, or type definition) 
can be used to form a type by supplying $k$ types as type arguments within
\xcd`[` \ldots \xcd`]`.
%%When instantiated,
%%with concrete (\viz, non-generic) types for its parameters, 
%%a generic type becomes a concrete type and can be
%%used like any other type. 
For example,
\xcd`Stack` is a generic class, 
%~~type~~`~~`~~ ~~class Stack[T]{} ^^^ Types180
\xcd`Stack[Int]` is a type, and can be used as one: 
%~~stmt~~`~~`~~ ~~class Stack[T]{} ^^^ Types190
\xcd`var stack : Stack[Int];`

\begin{ex}A \xcd`Cell[T]` is a generic object, capable of holding a value of type
\xcd`T`.  For example, a \xcd`Cell[Int]` can hold an \xcd`Int`, and a
\xcd`Cell[Cell[Int{self!=0}]]` can hold a \xcd`Cell` which in turn can
only hold non-zero numbers. 
%% vj: Dont know what this saying: bound immutably... but mutable?
%% \xcd`Cell`s are actually useful in situations
%%where values must be bound immutably for one reason, but need to be mutable.
%~~gen ^^^ Types200
% package ch4;
%~~vis
\begin{xten}
class Cell[T] {
    var x: T;
    def this(x: T) { this.x = x; }
    def get(): T = x;
    def set(x: T) = { this.x = x; }
}
\end{xten}
%~~siv
%~~neg


\xcd"Cell[Int]" is the type of \xcd`Int`-holding cells.  
The \xcd"get" method on a \xcd`Cell[Int]` returns an \xcd"Int"; the
\xcd"set" method takes an \xcd"Int" as argument.  Note that
\xcd"Cell" alone is not a legal type because the parameter is
not bound.
\end{ex}

A class (whether generic or not) may have generic methods.

\begin{ex}
\xcd`NonGeneric` has a generic method 
\xcd`first[T](x:List[T])`. An invocation of such a method may supply
the type parameters explicitly (\eg, \xcd`first[Int](z)`).
 In certain cases (\eg, \xcd`first(z)`)
type parameters may
be omitted and are inferred by the compiler (\Sref{TypeInference}).

%~~gen ^^^ Types210
% package Types.For.Cripes.Sake.Generic.Methods;
% import x10.util.*;
%~~vis
\begin{xten}
class NonGeneric {
  static def first[T](x:List[T]):T = x(0);
  def m(z:List[Int]) {
    val f = first[Int](z);
    val g = first(z);
    return f == g;
  }
}
\end{xten}
%~~siv
%
%~~neg


\end{ex}


\limitation{ \XtenCurrVer{}'s C++ back end requires generic methods to be
static or final; the Java back end can accomodate generic instance methods as well. }

Unlike other kinds of variables, type parameters may {\em not} be shadowed.  
If name \xcd`X` is in scope as a type, \xcd`X` may not be rebound as a type
variable.  

\begin{ex}
Neither \xcd`class B` nor \xcd`class C[B]` are allowed in the
following code, because they both shadow the type variable \xcd`B`.
%~~gen ^^^ TypesNoShadow
% package TypesNoShadow;
% KNOWNFAIL-https://jira.codehaus.org/browse/XTENLANG-2621
%~~vis
\begin{xten}
class A[B] {
  //ERROR: class B{} 
  //ERROR: class C[B]{} 
}
\end{xten}
%~~siv
%
%~~neg
\end{ex}

\subsection{Use of Generics}

An unconstrained type variable \Xcd{X} can be instantiated any type. Within a
generic struct or class, all the operations of \Xcd{Any} are available on a
variable of type unconstrained \Xcd{X}. Additionally, variables of type
\Xcd{X} may be used with \Xcd{==, !=}, in \Xcd{instanceof}, and casts.  

If a type variable is constrained, the operations implied by its constraint
are available as well.

\begin{ex}
The interface \xcd`Named` describes entities which know their own name.  The
class \xcd`NameMap[T]` is a specialized map which stores and retrieves
\xcd`Named` entities by name.  The call \xcd`t.name()` in \xcd`put()` is only
valid because the constraint \xcd`{T <: Named}` implies that \xcd`T` is a
subtype of \xcd`Named`, and hence provides all the operations of \xcd`Named`. 
%~~gen ^^^ Types6e6x
% package Types6e6x;
% import x10.util.*;
%~~vis
\begin{xten}
interface Named { def name():String; }
class NameMap[T]{T <: Named} {
   val m = new HashMap[String, T]();
   def put(t:T) { m.put(t.name(), t); }
   def get(s:String):T = m.getOrThrow(s);
}
\end{xten}
%~~siv
%
%~~neg


\end{ex}


%%NO-VARIANCE%% \subsection{Variance of Type Parameters}
%%NO-VARIANCE%% \index{covariant}
%%NO-VARIANCE%% \index{contravariant}
%%NO-VARIANCE%% \index{invariant}
%%NO-VARIANCE%% \index{type parameter!covariant}
%%NO-VARIANCE%% \index{type parameter!contravariant}
%%NO-VARIANCE%% \index{type parameter!invariant}
%%NO-VARIANCE%% 
%%NO-VARIANCE%% % Uncomment this when the language implementation properly supports variance.
%%NO-VARIANCE%% %%%\subsection{Variance of Type Parameters}
%%\index{covariant}
%%\index{contravariant}
%%\index{invariant}
%%\index{type parameter!covariant}
%%\index{type parameter!contravariant}
%%\index{type parameter!invariant}

%TODO - examples courtesy of Nate
% 
% class OutputStream[-A] {
%    def write(a: A) = /* implementation left as an exercise for the reader */
% }
% 
% Also:
% 
% interface Comparator[-A] {
%    def compare(A): Int;
% }
% 
% and:
% 
% class HashMap[-K,+V] { ... }
% 
% 

Type parameters of classes (though not of methods) can be {\em variant}.

Consider classes \xcd`Person :> Child`.  Every child is a person, but there
are people who are not children.  What is the relationship between
\xcd`Cell[Person]` and \xcd`Cell[Child]`?  

\subsubsection{Why Variance Is Necessary}

In this case, \xcd`Cell[Person]` and \xcd`Cell[Child]` should be unrelated.  
If we had \xcd`Cell[Person] :> Cell[Child]`, the following code would let us
assign a \xcd`old` (a \xcd`Person` but not a \xcd`Child`) to a
variable \xcd`young` of type \xcd`Child`, thereby breaking the type system: 
\begin{xten}
// INCORRECTLY assuming Cell[Person] :> Cell[Child]
val cc : Cell[Child] = new Cell[Child]();
val cp : Cell[Person] = cc; // legal upcast
cp.set(old);       // legal since old : Person
val young : Child = cc.get(); 
\end{xten}

Similarly, if \xcd`Cell[Person] <: Cell[Child]`: 
\begin{xten}
// INCORRECTLY assuming Cell[Person] <: Cell[Child]
val cp : Cell[Person] = new Cell[Person];
val cc : Cell[Child] = cp; // legal upcast
val cp.set(old); 
val young : Child = cc.get();
\end{xten}

So, there cannot be a subtyping relationship in either direction between the
two. And indeed, neither of these programs passes the X10 typechecker.


\subsubsection{Legitimate Variance}

The \xcd`Cell[Person]`-vs-\xcd`Cell[Child]` problems occur because it is
possible to both store and retrieve values from the same object. However,
entities with only one of the two capabilities {\em can} sensibly have some
subtyping relations. Furthermore, both sorts of entity are useful. An entity
which can store values but not retrieve them can nonetheless summarize them.
An object which can retrieve values but not store values can be constructed
with an initial value, providing a read-only cell.

So, X10 provides {\em variance} to support these options.  Type parameters
may be defined in one of three forms.  
\begin{enumerate}
\item {\em invariant}: Given a definition \xcd`class C[T]{...}`, \xcd`C[Person]` and
      \xcd`C[Child]` are unrelated classes; neither is a subclass of the
      other.
\item {\em covariant}: Given a definition \xcd`class C[+T]{...}` (the \xcd`+` indicates
      covariance), \xcd`C[Person] :> C[Child]`.  This is appropriate when
      \xcd`C` allows retrieving values but not setting them.
\item {\em contravariant}: Given a definition \xcd`class C[-T]{...}` (the \xcd`-` indicates
      contravariance), \xcd`C[Person] <: C[Child]`.  This is appropriate when
      \xcd`C` allows storing values but not retrieving them.
\end{enumerate}


The \xcd"T" parameter of \xcd"Cell" above is
invariant.  

A typical example of covariance is \xcd`Get`.  As the \xcd`example()` method
shows, a \xcd`Get[T]` must be constructed with its value, and will return that
value whenever desired.  \xcd`Get[T]` is only moderately useful as a class; it
is more useful as an interface for providing a limited (read) access to a more
powerful data structure.
%~~gen
% package Variance_gone;
%~~vis
\begin{xten}
class Get[+T] {
  val x: T;
  def this(x: T) { this.x = x; }
  def get(): T = x;
  static def example() {
     val g : Get[Int] = new Get[Int](31);
     val n : Int = g.get();
     x10.io.Console.OUT.print("It's " + n);
     x10.io.Console.OUT.print("It's still " + g.get());
  }
}
\end{xten}
%~~siv
%~~neg

There are few if any {\em classes} with contravariant type parameters.
(Covariant type parameters are only moderately more common.)  
However, it is frequently useful to have {\em interfaces} with contravariant
type parameters.  For example: 
%~~gen
% package Types_contravariance_a;
%~~vis
\begin{xten}
interface OutputStream[-T] {
   def write(T) : void;
}
interface ComparableTo[-T] {
   def compare(T) : Int;
}
\end{xten}
%~~siv
%
%~~neg
Clearly, \xcd`Int <: Any`. 
An \xcd`OutputStream[Int]` is only capable of writing \xcd`Int`s.  
An \xcd`OutputStream[Any]` is capable of writing anything.  In particular, it
can write \xcd`Int`s. Thus, an \xcd`OutputStream[Any]` can be used in place of
an \xcd`OutputStream[Int]`, and hence \xcd`OutputStream[Any] <: OutputStream[Int]`.
Similarly, a \xcd`ComparableTo[Int]` can be compared to an integer. A
\xcd`ComparableTo[Any]` can be compared to anything, and, in particular, to an
integer.  Thus \xcd`ComparableTo[Any] <: ComparableTo[Int]`.
So, both of these interfaces are contravariant.


Given types \xcd"S" and \xcd"T": 
\begin{itemize}
\item
If the parameter of \xcd"Get" is covariant, then
\xcd"Get[S]" is a subtype of \xcd"Get[T]" if
\xcd"S" is a {\em subtype} of \xcd"T".

\item
If the parameter of \xcd"Set" is contravariant, then
\xcd"OutputStream[S]" is a subtype of \xcd"OutputStream[T]" if
\xcd"S" is a {\em supertype} of \xcd"T".

\item
If the parameter of \xcd"Cell" is invariant, then
\xcd"Cell[S]" is a subtype of \xcd"Cell[T]" if
\xcd"S" is a {\em equal} to \xcd"T".
\end{itemize}


In order to make types marked as covariant and contravariant semantically
sound, X10 performs extra checks.  
A covariant type parameter is permitted to appear only in covariant type positions,
and a contravariant type parameter in contravariant positions. 
\begin{itemize}
\item The return type of a method is a covariant position.
\item The argument types of a method are contravariant positions.
\item Whether a type argument position of a generic class, interface or struct type \Xcd{C}
is covariant or contravariant is determined by the \Xcd{+} or \Xcd{-} annotation
at that position in the declaration of \Xcd{C}.
\end{itemize}


There are similar restrictions on use of covariant and contravariant variables.

\limitationx{} Full checking of covariance and contravariance is not yet
implemented.  Covariant and contravariant classes and structs should be used
with great caution.

%TODO: Yoav says ``There are other rules, not implemented or specified,
%involving fields, inheritance, etc.  There are several JIRAs on it. No idea
%what is the work around -- maybe just say ``limitation''?'''




%%NO-VARIANCE%% 
%%NO-VARIANCE%% Class, struct and interface definitions are permitted to specify a {\em
%%NO-VARIANCE%%   variance} 
%%NO-VARIANCE%% for each type parameter. 
%%NO-VARIANCE%% There are three variance specifications: 
%%NO-VARIANCE%% \xcd`+` indicates {\em co-variance},  \xcd`-` indicates {\em
%%NO-VARIANCE%%   contravariance} and the absence of  \xcd`+` and 
%%NO-VARIANCE%%  \xcd`-` indicates {\em invariance}. For a class (or struct or
%%NO-VARIANCE%%  interface) \xcd`S` specifying that a particular parameter position
%%NO-VARIANCE%%  (say, \xcd`i`) is covariant means that 
%%NO-VARIANCE%% if \xcd`Si <: Ti` then
%%NO-VARIANCE%% \xcdmath"S[S1,$\ldots$,Sn] <: S[S1,$\ldots$, Si-1,Ti,Si+1,$\ldots$ Sn]".
%%NO-VARIANCE%% Similarly, saying that position \xcd`i` is is contravariant means
%%NO-VARIANCE%% that 
%%NO-VARIANCE%% if \xcd`Si <: Ti` then
%%NO-VARIANCE%% \xcdmath"S[S1,$\ldots$, Si-1,Ti,Si+1,$\ldots$ Sn] <: S[S1,$\ldots$,Sn]". If the
%%NO-VARIANCE%% position is invariant, then no such relationship is asserted between
%%NO-VARIANCE%% \xcd`Si <: Ti` 
%%NO-VARIANCE%% and
%%NO-VARIANCE%% \xcdmath"S[S1,$\ldots$, Si-1,Ti,Si+1,$\ldots$ Sn]". The compiler must perform
%%NO-VARIANCE%% several checks on the body of the class (or struct or interface) to
%%NO-VARIANCE%% establish that type parameters with a variance are used in a manner
%%NO-VARIANCE%% that is consistent with their semantics.
%%NO-VARIANCE%% 
%%NO-VARIANCE%% \limitation{} The implementation of variance specifications  suffers from
%%NO-VARIANCE%% various limitations in \XtenCurrVer. Users are strongly encouraged not
%%NO-VARIANCE%% to use variance. (Some classes, structs, and interfaces in the standard
%%NO-VARIANCE%% libraries use variance specifications in a careful way that
%%NO-VARIANCE%% circumvents these limitations.)
%%NO-VARIANCE%% 

\section{Type definitions}
\label{TypeDefs}

\index{type!definitions}
\index{declaration!type}
A type definition can be thought of as a type-valued function,
mapping type parameters and value parameters to a concrete type.

%##(TypeDefDecl TypeParams Formals Guard
\begin{bbgrammar}
%(FROM #(prod:TypeDefDecl)#)
         TypeDefDecl \: Mods\opt \xcd"type" Id TypeParams\opt Guard\opt \xcd"=" Type \xcd";" & (\ref{prod:TypeDefDecl}) \\
                     \| Mods\opt \xcd"type" Id TypeParams\opt \xcd"(" FormalList \xcd")" Guard\opt \xcd"=" Type \xcd";" \\
%(FROM #(prod:TypeParams)#)
          TypeParams \: \xcd"[" TypeParamList \xcd"]" & (\ref{prod:TypeParams}) \\
%(FROM #(prod:Formals)#)
             Formals \: \xcd"(" FormalList\opt \xcd")" & (\ref{prod:Formals}) \\
%(FROM #(prod:Guard)#)
               Guard \: DepParams & (\ref{prod:Guard}) \\
\end{bbgrammar}
%##)

\noindent 
During type-checking the compiler replaces the use of such a defined
type with its body, substituting the actual type and value parameters
in the call for the formals. This replacement is performed recursively
until the type no longer contains a defined type or a predetermined
compiler limit is reached (in which case the compiler declares an
error). Thus, recursive type definitions are not permitted.

Thus type definitions are considered applicative and not generative --
they do not define new types, only aliases for existing types.

\label{TypeDefGuard}
Type definitions may have guards: an invocation of a type definition
is illegal unless the guard is satisified when formal types and values
are replaced by the actual parameters.

Type definitions may be overloaded: two type definitions with
the same name are permitted provided that they have a different number
of type parameters or different number or type of value parameters.  The rules
for type definition resolution are identical to those for method resolution.

However, \xcd`T()` is not allowed. If there is an argument list, it must be
nonempty.  This avoids a possible confusion between 
\xcd`type T = ...` and \xcd`type T() = ...`.  

A type definition for a type \xcd`T` must appear: 
\begin{itemize}
\item As a top-level definition in a file named \xcd`T.x10`; or
\item As a static member in a container definition; or
\item In a block statment.
\end{itemize}


\paragraph{Use of type definitions in constructor invocations}
If a type definition has no type parameters and no value
parameters and is an alias for a class type, a \xcd"new"
expression may be used to create an instance of the class using
the type definition's name.
Given the following type definition:
%TODO: Yoav says ``I just opened a jira on it: [1918].  I don't think you
% should be able to have {c} on the typedef A if you want to use it in a 'new'
% expression. If we do allow it, then we should allow: new
% Array[Int]{rank==1}(0..2) and new Array[Int](1)(0..2).
\begin{xtenmath}
type A = C[T$_1$, $\dots$, T$_k$]{c};
\end{xtenmath}
where 
\xcdmath"C[T$_1$, $\dots$, T$_k$]" is a
class type, a constructor of \xcdmath"C" may be invoked with
\xcdmath"new A(e$_1$, $\dots$, e$_n$)", if the
invocation
\xcdmath"new C[T$_1$, $\dots$, T$_k$](e$_1$, $\dots$, e$_n$)" is
legal and if the constructor return type is a subtype of
\xcd"A".

\paragraph{Automatically imported type definitions}
\index{import,type definitions}
\label{X10LangUnderscore}

The collection of type definitions in
\xcdmath"x10.lang._" is automatically imported in every compilation unit.


\subsection{Motivation and use}
The primary purpose of type definitions is to provide a succinct,
meaningful name for complex types
and combinations of types. 
With value arguments, type arguments, and constraints, the syntax for \Xten{}
types can often be verbose. 
For example, a non-null list of non-null strings is \\
%~~type~~`~~`~~ ~~import x10.util.*; ^^^ Types220
\xcd`List[String{self!=null}]{self!=null}`.

We could name that type: 
%~~gen ^^^ Types230
% package TypeDefs.glip.first;
% import x10.util.*;
% class LnSn {
% 
%~~vis
\begin{xten}
static type LnSn = List[String{self!=null}]{self!=null};
\end{xten}
%~~siv
%}
%~~neg
Or, we could abstract it somewhat, defining a type constructor
\xcd`Nonnull[T]` for the type of \xcd`T`'s which are not null:
%~~gen ^^^ Types240
% package TypeDefs.glip.second;
% import x10.util.*;
% 
%~~vis
\begin{xten}
class Example {
  static type Nonnull[T]{T <: Object}  = T{self!=null};
  var example : Nonnull[Example] = new Example();
}
\end{xten}
%~~siv
%
%~~neg

Type definitions can also refer to values, in particular, inside 
constraints.  The type of \xcd`n`-element \xcd`Array[Int](1)`s  is 
%~~type~~`~~`~~n:Int ~~ ^^^ Types250
\xcd`Array[Int]{self.rank==1 && self.size == n}`
but it is often convenient to give a shorter name: 
%~~gen ^^^ Types260
% package TypeDefs.glip.third;
% class Xmpl {
% def example() {
%~~vis
\begin{xten}
type Vec(n:Int) = Array[Int]{self.rank==1, self.size == n}; 
var example : Vec(78); 
\end{xten}
%~~siv
%}}
%~~neg

%
The following examples are legal type definitions, given \xcd`import x10.util.*`:
%~~gen ^^^ Types270
% package Types.TypeDef.Examples;
% import x10.util.*;
%~~vis
\begin{xten}
class TypeExamples {
  static type StringSet = Set[String];
  static type MapToList[K,V] = Map[K,List[V]];
  static type Int(x: Int) = Int{self==x};
  static type Dist(r: Int) = Dist{self.rank==r};
  static type Dist(r: Region) = Dist{self.region==r};
  static type Redund(n:Int, r:Region){r.rank==n} 
      = Dist{rank==n && region==r};
}
\end{xten}
%~~siv
% 
%~~neg

The following code illustrates that type definitions are applicative rather
than generative.  \xcd`B` and \xcd`C` are both aliases for \xcd`String`,
rather than new types, and so are interchangeable with each other and with
\xcd`String`. Similarly, \xcd`A` and \xcd`Int` are equivalent.
%~~gen ^^^ Types280
% package Types.TypeDef.Example.NonGenerative;
% import x10.util.*;
% class TypeDefNonGenerative {
%~~vis
\begin{xten}
def someTypeDefs () {
  type A = Int;
  type B = String;
  type C = String;
  a: A = 3;
  b: B = new C("Hi");
  c: C = b + ", Mom!";
  }
\end{xten}
%~~siv
% }
%~~neg
% An instance of a defined type with no type parameters and no
% value parameters may 


%%MEMBERSHIP%% All type definitions are members of their enclosing package or
%%MEMBERSHIP%% class.  A compilation unit may have one or more type definitions
%%MEMBERSHIP%% or class or interface declarations with the same name, as long
%%MEMBERSHIP%% as the definitions have distinct parameters according to the
%%MEMBERSHIP%% method overloading rules (\Sref{MethodOverload}).


\section{Constrained types}
\label{ConstrainedTypes}
\label{DepType:DepType}
\label{DepTypes}

\index{dependent type}
\index{type!dependent}
\index{constrained type}
\index{generic type}
\index{type!constrained}
\index{type!generic}


Basic types, like \xcd`Int` and \xcd`List[String]`, provide useful
descriptions of data.  

However, one frequently wants to say more.  One might want to know
that a \xcd`String` variable is not \xcd`null`, or that a matrix is
square, or that one matrix has the same number of columns as another
has rows (so they can be multiplied).  In the multicore setting, one
might wish to know that two values are located at the same processor,
or that one is located at the same place as the current computation.

In most languages, there is simply no way to say these things
statically.  Programmers must made do with comments, \xcd`assert`
statements, and dynamic tests.  X10 programs can do better, with {\em
  constraints} on types, and guards on class, method and type
definitions,

A constraint is a boolean expression \xcd`e` attached to a basic type \xcd`T`,
written \xcd`T{e}`.  (Only a limited selection of boolean expressions is
available.)  The values of type \xcd`T{e}` are the values of \xcd`T` for which
\xcd`e` is true.  

\index{self}When constraining a value of type \xcd`T`, \xcd`self` refers to the object of
type \xcd`T` which is being constrained.  For example, \xcd`Int{self == 4}` is
the type of \xcd`Int`s which are equal to 4 -- the best possible description
of \xcd`4`, and a very difficult type to express without using \xcd`self`.  

\begin{ex}

\begin{itemize}
%~~type~~`~~`~~ ~~ ^^^ Types290
\item \xcd`String{self != null}` is the type of non-null strings.  \xcd`self`
      is a special variable available only in constraints; it refers to the
      datum being constrained, and its type is the type to which the
      constraint is attached.
\item Suppose that \xcd`Matrix` is a matrix class with  properties \xcd`rows`
      and \xcd`cols`.  
%~~type~~`~~`~~ ~~class Matrix(rows:Int,cols:Int){} ^^^ Types300
      \xcd`Matrix{self.rows == self.cols}` is the type of square matrices.
\item One way to say that \xcd`a` has the same number of columns that \xcd`b`
      has rows (so that \xcd`a*b` is a valid matrix product), one could say: 
%~~gen ^^^ Types310
% package Types.cripes.whered.you.get.those.gripes;
% class Matrix(rows:Int, cols:Int){
% public static def someMatrix(): Matrix = null;
% public static def example(){
%~~vis
\begin{xten}
  val a : Matrix = someMatrix() ;
  var b : Matrix{b.rows == a.cols} ;
\end{xten}
%~~siv
%}}
%~~neg
\end{itemize}
\end{ex}



\xcd"T{e}" is a {\em dependent type}, that is, a type dependent on values. The
type \xcd"T" is called the {\em base type} and \xcd"e" is called the {\em
  constraint}. If the constraint is omitted, it is \xcd`true`---that is, the
  base type is unconstrained.

Constraints may refer to immutable values in the local environment: 
%~~gen ^^^ Types320
% class ConstraintsMayReferToValues {
% def thoseValues() {
%~~vis
\begin{xten}
     val n = 1;
     var p : Point{rank == n};
\end{xten}
%~~siv
%}}
%~~neg
In a variable declaration, the variable itself is in scope in its
type. For example, \xcd`val nz: Int{nz != 0} = 1;` declares a
non-zero variable \xcd`nz`.
\bard{This will need to be explained further once the language issues are
sorted out.}

%%TYPES-CONSTR-EXP%% We permit variable declarations \xcd"v: T" where \xcd"T" is obtained
%%TYPES-CONSTR-EXP%% from a dependent type \xcd"C{c}" by replacing one or more occurrences
%%TYPES-CONSTR-EXP%% of \xcd"self" in \xcd"c" by \xcd"v". (If such a declaration \xcd"v: T"
%%TYPES-CONSTR-EXP%% is type-correct, it must be the case that the variable \xcd"v" is not
%%TYPES-CONSTR-EXP%% visible at the type \xcd"T". Hence we can always recover the
%%TYPES-CONSTR-EXP%% underlying dependent type \xcd"C{c}" by replacing all occurrences of \xcd"v"
%%TYPES-CONSTR-EXP%% in the constraint of \xcd"T" by \xcd"self".)
%%TYPES-CONSTR-EXP%% 
%%TYPES-CONSTR-EXP%% For instance, \xcd"v: Int{v == 0}" is shorthand for \xcd"v: Int{self == 0}".
%%TYPES-CONSTR-EXP%% 
%%TYPES-CONSTR-EXP%% 
%%TYPES-CONSTR-EXP%% A variable occurring in the constraint \xcd"c" of a dependent type, other than
%%TYPES-CONSTR-EXP%% \xcd"self" or a property of \xcd"self", is said to be a {\em
%%TYPES-CONSTR-EXP%% parameter} of \xcd"c".\label{DepType:Parameter} \index{parameter}

A constrained type may be constrained further: the type \xcd`S{c}{d}`
is the same as the type \xcd`S{c,d}`.  Multiple constraints are equivalent to
conjoined constraints: \xcd`S{c,d}` in turn is the same as \xcd`S{c && d}`.

\subsection{Syntax of constraints}
\index{constraint!permitted}
\index{constraint!syntax}
\label{PermittedConstraints}
\index{constraint}
\index{expression!allowed in constraint}
\index{expression!constraint}

Only a few kinds of expressions can appear in constraints.  For fundamental
reasons of mathematical logic, the more kinds of expressions that can appear
in constraints, the harder it is to compute the essential properties of
constrained type -- in particular, the harder it is to compute 
\xcd`A{c} <: B{d}` or even \xcd`E : T{c}`.  It doesn't take much to make this
basic fact undecidable. 
In order to make sure that it stays decidable, X10 places stringent restrictions on
constraints.  

Only the following forms of expression are allowed in constraints.  

{\bf Value expressions in constraints} may be: 
\begin{enumerate}
\item Literal constants, like \xcd`3` and \xcd`true`;
% \item Expressions computable at compile time, like \Xcd{3*4+5};
\item Accessible, immutable (\xcd`val`) variables and parameters;
% \item Accessible and immutable fields of objects;
% \item Properties of the type being constrained;
\item \xcd`this`, if the constraint is in a place where \xcd`this` is defined;
\item \xcd`here`, if the constraint is in a place where \xcd`here` is defined;
\item \xcd`self`;
\item A field selection expression \xcd`t.f`, where \xcd`t` is a value
      expression allowed in constraints, and \xcd`f` is a field of \xcd`t`'s
      type.    
 \item Invocations of property methods,  \xcd`p(a,b,...,c)` or
      \xcd`a.p(b,c,...d)`, where the receiver and arguments must be
       value expressions acceptable in constraints, as long as the expansion
       (\viz, the expression obtained by taking the body of the definition of
       \xcd`p`, and replacing the formal parameters by the actual parameters)
       of the invocation is allowed as a value expression in constraints.  
\end{enumerate}
For an expression \xcd`self.p` to be legal in a constraint, 
\xcd`p` must be 
a property. However terms \xcd`t.f` may be
used in constraints (where \xcd`t` is a term other than \xcd`self` and
\xcd`f` is an immutable field.)

{\bf Constraints}  may be any of
the following, where 
all value expressions are of the forms which may appear in constraints: 
\begin{enumerate}
\item Equalities \xcd`e == f`;
\item Inequalities of the form \xcd`e != f`;\footnote{Currently inequalities
      of the form \xcd`e < f` are not supported.}
\item Conjunctions of Boolean expressions that may appear in constraints (but
      only in top-level constraints, not in Boolean expressions in constraints);
\item Subtyping and supertyping expressions: \xcd`T <: U` and \xcd`T :> U`; 
\item Type equalities and inequalities: \xcd`T == U` and \xcd`T != U`; 
\item Invocations of a property method, \xcd`p(a,b,...,c)` or
      \xcd`a.p(b,c,...d)`, where the receiver and arguments must be value
      expressions acceptable in constraints, as long as the expansion of the
      invocation is allowed as a constraint.
\item Testing a type for a default: \Xcd{T haszero}.
\end{enumerate}

All variables appearing in a constraint expression must be visible wherever
that expression can used.  \Eg, properties and public fields of an object are
always permitted, but private fields of an object can only constrain private
members.  (Consider a class \xcd`PriVio` with a private field \xcd`p` and a
public method \xcd`m(x: Int{self != p})`, and a call \xcd`ob.m(10)` made
outside of the class. Since \xcd`p` is only visible inside the class, there is
no way to tell if \xcd`10` is of type \xcd`Int{self != p}` at the call site.)

{\bf Note:} Constraints may not contain casts.   In particular, comparisons of
values of incompatible types are not allowed.  If \xcd`i:Int`, then \xcd`i==0`
is allowed as a constraint, but \xcd`i==0L` is an error, and 
\xcd`i as Long==0L` is outside of the constraint language.


\subsubsection{Semantics of constraints}
\index{constraint!semantics}
\label{SemanticsOfConstraints}
An assignment of values to variables is said to be a {\em solution} for a
constraint \xcd`c` if under this assignment \xcd`c` evaluates to
\xcd`true`. For instance, the assignment that maps 
the variables \xcd`a` and \xcd`b` to a value \xcd`t` is a solution for
the constraint \xcd`a==b`. An assignment that maps \xcd`a` to 
\xcd`s` and \xcd`b` to a distinct value \xcd`t` is a solution for 
\xcd`a != b`. 

An instance \xcd"o" of \xcd"C" is said to be of type \xcd"C{c}" (or {\em
belong to} \xcd"C{c}") if the constraint \xcd"c" evaluates to \xcd"true" in
the current lexical environment augmented with the binding \xcd"self"
$\mapsto$ \xcd"o".

A constraint \xcd`c` is said to {\em entail} a
constraint \xcd`d` if every solution for \xcd`c` is also a solution
for \xcd`d`. For instance the constraint
\xcd`x==y && y==z && z !=a` entails \xcd`x != a`.

The constraint solver considers the assignment \xcd`a` to \xcd`null`
to satisfy any constraint of the form \xcd`a.f==t`. 
Thus, the declaration 
\xcd`var x:Tree{self.r==p}=null` does not
produce an error, since \xcd`self==null` satisfies the constraint
\xcd`self.r==p`.  
(This only applies in constraints, not in expression evaluation.  
\xcd`if(a.f==t)...` will throw an exception if \xcd`a==null`.)
If \xcd`null` is not an acceptable value for some class \xcd`Tree`, 
add \xcd`self!=null` as a constraint: 
\xcd`Tree{self!=null}` is the class of non-\xcd`null` \xcd`Tree`s.

To ensure that type-checking is decidable, we require that property graphs be
acyclic.  The property graph, at an instant in an X10 execution, is the graph
whose nodes are all objects in existence at that instance, with an edge from
{$x$} to {$y$} if {$x$} is an object with a property whose value is {$y$}. 
The rules for constructors guarantee this.

Constraints participate in the subtyping relationship in a natural way:
\xcdmath"S[S1,$\ldots$, Sm]{c}" 
is a subtype of 
\xcdmath"T[T1,$\ldots$, Tn]{d}" 
if \xcdmath"S[S1,$\ldots$,Sm]" is a subtype of \xcdmath"T[T1,$\ldots$,Tn]" and
\xcd"c" entails \xcd"d".

For examples of constraints and entailment, see (\Sref{ConstraintExamples})
%%TYPES-CONSTR-EXP%% 
%%TYPES-CONSTR-EXP%% \begin{grammar}
%%TYPES-CONSTR-EXP%% Constraint \: ValueArguments     Guard\opt \\
%%TYPES-CONSTR-EXP%%            \| ValueArguments\opt Guard     \\
%%TYPES-CONSTR-EXP%%            \\
%%TYPES-CONSTR-EXP%% ValueArguments   \:  \xcd"(" ArgumentList\opt \xcd")" \\
%%TYPES-CONSTR-EXP%% ArgumentList     \:  Expression ( \xcd"," Expression )\star \\
%%TYPES-CONSTR-EXP%% Guard            \: \xcd"{" DepExpression \xcd"}" \\
%%TYPES-CONSTR-EXP%% DepExpression    \: ( Formal \xcd";" )\star ArgumentList \\
%%TYPES-CONSTR-EXP%% \end{grammar}
%%TYPES-CONSTR-EXP%% 
%%TYPES-CONSTR-EXP%% In \XtenCurrVer{} value constraints may be equalities (\xcd"=="),
%%TYPES-CONSTR-EXP%% disequalities (\xcd"!=") and conjunctions thereof.  The terms over
%%TYPES-CONSTR-EXP%% which these constraints are specified include literals and
%%TYPES-CONSTR-EXP%% (accessible, immutable) variables and fields, property methods, and the special
%%TYPES-CONSTR-EXP%% constants {\tt here}, {\tt self}, and {\tt this}. Additionally, place
%%TYPES-CONSTR-EXP%% types are permitted (\Sref{PlaceTypes}).
%%TYPES-CONSTR-EXP%% 
%%TYPES-CONSTR-EXP%% \index{self}

%%TYPES-CONSTR-EXP%% Type constraints may be subtyping and supertyping (\xcd"<:" and
%%TYPES-CONSTR-EXP%% \xcd":>") expressions over types.

\subsection{Constraint solver: incompleteness and approximation}
\index{constraint solver!incompleteness}
\index{constraint!entailment}
\index{constraint!subtyping}



The constraint solver is sound in that if it claims that \xcd`c` entails \xcd`d`
then in fact it is the case that every value that satisfies \xcd`c`
satisfies \xcd`d`. 

\limitationx{}
X10's constraint solver is incomplete. There are situations
in which \xcd`c` entails \xcd`d` but the solver cannot establish it. For
instance it cannot establish that \xcd`a != b && a != c && b != c`
entails \xcd`false` if \xcd`a`, \xcd`b`, and \xcd`c` are of type
\xcd`Boolean`.


Certain other constraint entailments are prohibitively expensive to calculate.  The issues
concern constraints that connect different levels of recursively-defined
types, such as the following.  
%~~gen ^^^ Types330
% package Types.Entailment.EntailFail;
%~~vis
\begin{xten}
class Listlike(x:Int) {
  val kid : Listlike{self.x == this.x};
  def this(x:Int, kid:Listlike) { 
     property(x); 
     this.kid = kid as Listlike{self.x == this.x};}
}
\end{xten}
%~~siv
%
%~~neg
There is nothing wrong with \xcd`Listlike` itself, or with most uses of it;
however, a sufficiently complicated use of it could, in principle, cause X10's
typechecker to fail. 
It is hard to give a plausible example of when X10's algorithm fails, as we
have not yet observed such a failure in practice for a correct program.  

The entailment algorithm of X10 imposes a certain limit on the number of
times such types will be unwound.   If this limit is exceeded, the compiler
will print a warning, and type-checking will fail in a situation where it is
semantically allowed.  In this case, insert a dynamic cast at the point where
type-checking failed.  

\limitation{ Support for comparisons of generic type variables is
  limited. This will be fixed in future releases.}
% //, and existential quantification over typed variables.

%%TYPES-CONSTR-EXP%% \emph{
%%TYPES-CONSTR-EXP%% Subsequent implementations are intended to support boolean algebra,
%%TYPES-CONSTR-EXP%% arithmetic, relational algebra, etc., to permit types over regions and
%%TYPES-CONSTR-EXP%% distributions. We envision this as a major step towards removing most,
%%TYPES-CONSTR-EXP%% if not all, dynamic array bounds and place checks from \Xten{}.
%%TYPES-CONSTR-EXP%% }




%%PLACE%%\subsection{Place constraints}
%%PLACE%%\label{PlaceTypes}
%%PLACE%%\label{PlaceType}
%%PLACE%%\index{place types}
%%PLACE%%\label{DepType:PlaceType}\index{placetype}
%%PLACE%%
%%PLACE%%An \Xten{} computation spans multiple places (\Sref{XtenPlaces}). Much data
%%PLACE%%can only be accessed from the proper place, and often it is preferable to
%%PLACE%%determine this statically. So, X10 has special syntax for working with places.
%%PLACE%%\xcd`T!` is a value of type \xcd`T` located at the right place for the current
%%PLACE%%computation, and \xcd`T!p` is one located at place \xcd`p`.
%%PLACE%%
%%PLACE%%\begin{grammar}
%%PLACE%%PlaceConstraint     \: \xcd"!" Place\opt \\
%%PLACE%%Place              \:   Expression \\
%%PLACE%%\end{grammar}
%%PLACE%%
%%PLACE%%More specifically, All \Xten{} classes extend the class \xcd"x10.lang.Object",
%%PLACE%%which defines a property \xcd"home" of type \xcd"Place".  \xcd`T!p`, when
%%PLACE%%\xcd`T` is a class, is \xcd`T{self.home==p}`.  If \xcd`p` is omitted, it
%%PLACE%%defaults to \xcd`here`.   \xcd`T!` is far and away the most common usage of
%%PLACE%%\xcd`!`. 
%%PLACE%%
%%PLACE%%Structs don't have \xcd`home`; they are available everywhere.  For structs, 
%%PLACE%%\xcd`T!` and \xcd`T!p` are synonyms for \xcd`T`. Since \xcd`T` is available
%%PLACE%%everywhere, it is available \xcd`here` and at \xcd`p`. 
%%PLACE%%
%%PLACE%%\xcd`!` may be combined with other constraints.  \xcd`T{c}!` is the type of
%%PLACE%%values of \xcd`T!` which satisfy \xcd`c`; it is \xcd`T{c && self.home==here}`
%%PLACE%%for an object type and \xcd`T{c}` for a struct type.  
%%PLACE%%\xcd`T{c}!p` is the type of
%%PLACE%%values of \xcd`T!p` which satisfy \xcd`c`; it is \xcd`T{c && self.home==p}`
%%PLACE%%for an object type and \xcd`T{c}` for a struct type.  
%%PLACE%%
%%PLACE%%
%%PLACE%%
%%PLACE%%% The place specifier \xcd"any" specifies that the object can be
%%PLACE%%% located anywhere.  Thus, the location is unconstrained; that is,
%%PLACE%%% \xcd"C{c}!any" is equivalent to \xcd"C{c}".
%%PLACE%%
%%PLACE%%% XXX ARRAY
%%PLACE%%%The place specifier \xcd"current" on an array base type
%%PLACE%%%specifies that an object with that type at point \xcd"p"
%%PLACE%%%in the array 
%%PLACE%%%is located at \xcd"dist(p)".  The \xcd"current" specifier can be
%%PLACE%%%used only with array types.
%%PLACE%%
%%PLACE%%

\subsection{Limitation: Runtime Constraint Erasure}
\index{cast!to generic type}

The X10 runtime does not maintain a representation of constraints.
In many cases, it does not need to.
If X10 has an object \xcd`x` of some type \xcd`T` around, it can check at
runtime whether or not \xcd`x` satisfies some constraint \xcd`c`, and hence
tell if \xcd`x` is a member of \xcd`T{c}`. 

\begin{ex}
Although there is no runtime representation of the constrained type 
\xcd`Int{self==1}`, X10 can generate a (correct) test for membership in it,
anyhow: 
%~~gen ^^^ Types3u5w
% package Types3u5w;
% class Example {
%~~vis
\begin{xten}
static def example(n:Int) {
  val b = (n instanceof Int{self == 1});
  assert b == (n == 1); 
}
\end{xten}
%~~siv
% }
% class Hook{ def run() {Example.example(0); Example.example(1);
% Example.example(2); return true; } }
%~~neg
\end{ex}

However, in cases where there is no object of type \xcd`T` around, there's
nothing that can be checked. For example, X10 cannot tell -- and in fact no
computer program can tell --  whether an
instance of a function type 
\begin{xtenmath}
(Int)=>Int
\end{xtenmath}
(unary functions returning
integers) is actually an instance of a more specific type
\begin{xtenmath}
(Int)=>Int{self!=0}
\end{xtenmath}
(unary functions returning non-zero integers).

In other cases, there might or might not be an object of type \xcd`T`, and X10
cannot tell until runtime.  Consider an array \xcd`a:Array[T]`.  If \xcd`a` is
nonempty, there is an instance of \xcd`T` at hand, and testing it for
constraints would be possible though potentially quite expensive. 
But \xcd`a` might be an
empty array, and testing its element type would be impossible. 

Rather than pay the runtime costs for keeping and manipulating constraints
(which can be considerable), X10 omits them.
However, this renders certain type checks uncertain: X10 needs some
information at runtime, but does not have it.  
Specifically, all casts to instances of generic types are forbidden.  

\begin{ex}
The following code  needs to be, and is, statically
rejected.  It constructs an array \xcd`a` of \xcd`Int{self==3}`'s -- integers
which 
are statically known to be 3. 
The only number that can be stored into \xcd`a` is \xcd`3`.  
Then (in the line that is rejected) it attempts to trick the compiler into
thinking that it is an array of \xcd`Int`, without restriction on the
elements, giving it the name \xcd`b` at that type.  
The cast \xcd`aa as Array[Int]` is a cast to an instance of a generic type,
and hence is forbidden. 

But, if that cast were allowed to work, it could store \xcd`1` into the array
under the alias 
\xcd`b`, thereby violating 
the invariant that all the elements of the array are 3.  
This could lead to program failures, as illustrated by the failing assertion.  
\begin{xten}
  val a = new Array[Int{self==3}](0..10, 3);
  // a(0) = 1; would be illegal
  a(0) = 3; // LEGAL
  val aa = a as Any;
  /* THE FOLLOWING IS A STATIC ERROR:
  val b = aa as Array[Int];
  b(0) = 1;
  val x : Int{self==3} = a(0);
  assert x == 3 : "This would fail at runtime.";
  */
\end{xten}
\end{ex}



\subsection{Example of Constraints}
\label{ConstraintExamples}

Example of entailment and subtyping involving constraints.
\begin{itemize}
\item \xcd`Int{self == 3} <: Int{self != 14}`.  The only value of
      \xcd`Int{self ==3}` is $3$.  All integers but $14$ are members of
      \xcd`Int{self != 14}`, and in particular $3$ is.  
\item Suppose we have classes \xcd`Child <: Person`, and \xcd`Person` has a
      \xcd`ssn:Long` property.  If \xcd`rhys : Child{ssn == 123456789}`, then
      \xcd`rhys` is also a \xcd`Person`.  
      \xcd`rhys`'s \xcd`ssn` field is the same, \xcd`123456789`, whether 
      \xcd`rhys` is regarded as a \xcd`Child` or a \xcd`Person`.  
      Thus, 
      \xcd`rhys : Person{ssn==123456789}` as well.  
      So, 
\begin{xtenmath}
Child{ssn == 123456789} <: Person{ssn == 123456789}.
\end{xtenmath}
\item Furthermore, since \xcd`123456789 != 555555555`, 
      is is clear that 
      \xcd`rhys : Person{ssn != 555555555}`.  
      So, 
\begin{xtenmath}
Child{ssn == 123456789} <: Person{ssn != 555555555}.  
\end{xtenmath}
\item \xcd`T{e} <: T` for any type \xcd`T`.  That is, if you have a value
      \xcd`v` of some base type \xcd`T` which satisfied \xcd`e`, then \xcd`v`
      is of that base type \xcd`T` (with the constraint ignored).
\item If \xcd`A <: B`, then \xcd`A{c} <: B{c}` for every constraint \xcd`{c}`
      for which \xcd`A{c}` and \xcd`B{c}` are defined.  That is, if every
      \xcd`A` is also a \xcd`B`, and \xcd`a : A{c}`, then 
      \xcd`a` is an \xcd`A` and \xcd`c` is true of it. So \xcd`a` is also a
      \xcd`B` (and \xcd`c` is still true of 
      it), so \xcd`a : B{c}`.  
\end{itemize}

Constraints can be used to express simple relationships between objects,
enforcing some class invariants statically.  For example, in geometry, a line
is determined by two {\em distinct} points; a \xcd`Line` struct can specify the
distinctness in a type constraint:\footnote{We call them
\xcd`Position` to avoid confusion with the built-in class \xcd`Point`. 
Also, \xcd`Position` is a struct rather than a class so that the non-equality
test \xcd`start != end` compares the coordinates.  If \xcd`Position` were a
class, \xcd`start != end` would check for different \xcd`Position` objects,
which might have the same coordinates.
}


%~~gen ^^^ Types340
% package triangleExample.partOne;
%~~vis
\begin{xten}
struct Position(x: Int, y: Int) {}
struct Line(start: Position, end: Position){start != end}
  {}
\end{xten}

%~~siv
%~~neg

Extending this concept, a \xcd`Triangle` can be defined as a figure with three
line segments which match up end-to-end.  Note that the degenerate case in
which two or three of the triangle's vertices coincide is excluded by the
constraint on \xcd`Line`.  However, not all degenerate cases can be excluded
by the type system; in particular, it is impossible to check that the three
vertices are not collinear. 

%~~gen ^^^ Types350
%package triangleExample.partTwo;
% struct Position(x: Int, y: Int) {
%    def this(x:Int,y:Int){property(x,y);}
%    }
% class Line(start: Position, 
%            end: Position{self != start}) {}
% 
%~~vis
\begin{xten}
struct Triangle 
 (a: Line, 
  b: Line{a.end == b.start}, 
  c: Line{b.end == c.start && c.end == a.start})  
 {}
\end{xten}
%~~siv
%
%~~neg

The \xcd`Triangle` class automatically gets a ternary constructor which takes
suitably constrained \xcd`a`, \xcd`b`, and \xcd`c` and produces a new
triangle. 

\section{Default Values}
\index{default value}
\index{type!default value}
\label{DefaultValues}

Some types have default values, and some do not. Default values are used in
situations where variables can legitimately be used without having been
initialized; types without default values cannot be used in such situations.
For example, a field of an object \xcd`var x:T` can be left uninitialized if
\xcd`T` has a default value; it cannot be if \xcd`T` does not. Similarly, a
transient (\Sref{TransientFields}) field \xcd`transient val x:T` is only
allowed if \xcd`T` has a default value.

Default values, or lack of them, is defined thus:
\begin{itemize}
\item The fundamental numeric types (\xcd`Int`, \xcd`UInt`,
      \xcd`Long`, \xcd`ULong`, 
%%limitation%%       \xcd`Short`, \xcd`UShort`, \xcd`Byte`,
%%limitation%%       \xcd`UByte`, 
      \xcd`Float`, \xcd`Double`) all have default value 0.
\item \xcd`Boolean` has default value \xcd`false`.
\item \xcd`Char` has default value \xcd`'\0'`.
\item Struct types other than those listed above have no default value.
\item A function type has a default value of \xcd`null`.
\item A class type has a default value of \xcd`null`.
\item The constrained type \xcd`T{c}` has the same default value as \xcd`T` if
      that default value satisfies \xcd`c`.  If the default value of \xcd`T`
      doesn't satisfy \xcd`c`, then \xcd`T{c}` has no default value.
\end{itemize}

\begin{ex}
\xcd`var x: Int{x != 4}` has default value 0, which is allowed
because \xcd`0 != 4` satisfies the constraint on \xcd`x`. 
\xcd`var y : Int{y==4}` has no default value, because \xcd`0` does not satisfy \xcd`y==4`.
The fact that \xcd`Int{y==4}` has precisely one value, \viz{} 4, doesn't
matter; the only candidate for its default value, as for any subtype of
\xcd`Int`, is 0. \xcd`y` must be initialized before it is used.
\end{ex}

The predicate \xcd`T haszero` tells if the type \xcd`T` has a default value.
\xcd`haszero` may be used in constraints. 

\begin{ex}
The following code defines a sort of cell holding a single value of type
\xcd`T`. The cell is initially empty -- that is, has \xcd`T`'s zero value --
but may be filled later. 
%~~gen ^^^ TypesHaszero10
% package TypesHaszero10;
%~~vis
\begin{xten}
class Cell0[T]{T haszero} {
  public var contents : T;
  public def put(t:T) { contents = t; }
}
\end{xten}
%~~siv
%
%~~neg
\end{ex}

The built-in type \xcd`Zero` has the method \xcd`get[T]()` which
returns the default value of type \xcd`T`.  

\begin{ex}
As a variation on a theme of \xcd`Cell0`, we define a class \xcd`Cell1[T]` which can be initialized with a value of an arbitrary
type
\xcd`T`, or, if \xcd`T` has a default value, can be created with the default
value.  Note that \xcd`T haszero` is a constraint on one of
the constructors, not the whole type:  
%~~gen ^^^ TypesHaszero20
% package TypesHaszero20;
%~~vis
\begin{xten}
class Cell1[T] {
  public var contents: T;
  def this(t:T) { contents = t; }
  def this(){T haszero} { contents = Zero.get[T](); }
  public def put(t:T) {contents = t;}
}
\end{xten}
%~~siv
%
%~~neg

\end{ex}

\section{Function types}
\label{FunctionTypes}
\label{FunctionType}
\index{function!types}
\index{type!function}

%##(FunctionType
\begin{bbgrammar}
%(FROM #(prod:FunctionType)#)
        FunctionType \: TypeParams\opt \xcd"(" FormalList\opt \xcd")" Guard\opt Offers\opt \xcd"=>" Type & (\ref{prod:FunctionType}) \\
\end{bbgrammar}
%##)


For every sequence of types \xcd"T1,..., Tn,T", and \xcd"n" distinct variables
\xcd"x1,...,xn" and constraint \xcd"c", the expression
\xcd"(x1:T1,...,xn:Tn){c}=>T" is a \emph{function type}. It stands for
 the set of all functions \xcd"f" which can be applied to a
 list of values \xcd"(v1,...,vn)" provided that the constraint
 \xcd"c[v1,...,vn,p/x1,...,xn]" is true, and which returns a value of
 type \xcd"T[v1,...vn/x1,...,xn]". When \xcd"c" is true, the clause \xcd"{c}" can be
 omitted. When \xcd"x1,...,xn" do not occur in \xcd"c" or \xcd"T", they can be
 omitted. Thus the type \xcd"(T1,...,Tn)=>T" is actually shorthand for
 \xcd"(x1:T1,...,xn:Tn){true}=>T", for some variables \xcd"x1,...,xn".

\limitationx{}
Constraints on closures are not supported.  They parse, but are not checked.

X10 functions, like mathematical functions, take some arguments and produce a
result.  X10 functions, like other X10 code, can change mutable state and
throw exceptions.  Closures (\Sref{Closures})  are of function type -- and so
are arrays.


\begin{ex}Typical functions are the reciprocal function: 
%~~gen ^^^ Types360
% package Types.Functions;
% class RecipEx {
% static 
%~~vis
\begin{xten}
val recip = (x : Double) => 1/x;
\end{xten}
%~~siv
%}
%~~neg
and a function which increments  element \xcd`i` of an array \xcd`r`, or throws an exception
if there is no such element, where, for the sake of example, we constrain the
type of \xcd`i` to avoid one of the many integers which are not possible subscripts:  
%~~gen ^^^ Types_constraint_b
% package Types_constraint_b;
% NOTEST
% /*NONSTATIC*/class IncrElEx {
% static def example()  {
%~~vis
\begin{xten}
val inc = (r:Array[Int](1), i: Int{i != r.size}) => {
  if (i < 0 || i >= r.size) throw new DoomExn();
  r(i)++;
};
\end{xten}
%~~siv
%}}
%class DoomExn extends Exception{}
%~~neg
\end{ex}

In general, a function type needs to list the types 
\xcdmath"T$_i$"
of all the formal parameters,
and their distinct names \xcdmath"x$_i$" in case other types refer to them; a
constraint 
\xcd"c" on the
function as a whole; a return type \xcd"T".

\begin{xtenmath}
(x$_1$: T$_1$, $\dots$, x$_n$: T$_n$){c} => T
\end{xtenmath}


The names \xcdmath"x$_i$" of the formal parameters are not relevant.  Types
which differ only in the names of formals (following the usual rules for
renaming of variables, as in {$\alpha$}-renaming in the {$\lambda$} calculus
\bard{cite something}) are considered equal.  \Eg, the two function types
%~~type~~`~~`~~ ~~ ^^^ Types370
\xcd`(a:Int, b:Array[String](1){b.size==a}) => Boolean`
and \\
%~~type~~`~~`~~ ~~ ^^^ Types380
\xcd`(b:Int, a:Array[String](1){a.size==b}) => Boolean`
are equivalent.

\limitation{
Function types differing only in the names of bound variables may wind up being
considered different in X10 v2.2, especially if the variables appear in
constraints.  
}

The formal parameter names are in scope from the point of definition to the
end of the function type---they may be used in the types of other formal parameters
and in the return type. 
Value parameters names may be
omitted if they are not used; the type of the reciprocal function can be
written as
%~~type~~`~~`~~ ~~ ^^^ Types390
\xcd`(Double)=>Double`. 

A function type is covariant in its result type and contravariant in
each of its argument types. That is, let 
\xcd"S1,...,Sn,S,T1,...Tn,T" be any
types satisfying \xcd"Si <: Ti" and \xcd"S <: T". Then
\xcd"(x1:T1,...,xn:Tn){c}=>S" is a subtype of
\xcd"(x1:S1,...,xn:Sn){c}=>T".

A class or struct definition may use a function type 
\begin{xtenmath}
F = (x1:T1,...,xn:Tn){c}=>T
\end{xtenmath}
in its 
implements clause; 
this is equivalent to implementing an interface requiring the single operator
\begin{xtenmath}
public operator this(x1:T1,...,xn:Tn){c}:T
\end{xtenmath}
Similarly, an interface
definition may specify a function type \xcd"F" in its \xcd"extends" clause.
Values of a class or struct implementing \xcd`F` 
can be used as functions of type \xcd`F` in all ways.  
In particular, applying one to suitable arguments calls the \xcd`apply`
method. 

\limitationx{} A class or struct may not implement two different
instantiations of a generic interface. In particular, a class or
struct can implement only one function type.


A function type \xcd"F" is not a class type in that it does not extend any
type or implement any interfaces, or support equality tests. 
\xcd`F` may be implemented, but not extended, by a class or function type. 
Nor is it a struct type, for it has no predefined notion of equality.


\section{Annotated types}
\label{AnnotatedTypes}

\index{type!annotated}
\index{annotations!type annotations}

        Any \Xten{} type may be annotated with zero or more
        user-defined \emph{type annotations}
        (\Sref{XtenAnnotations}).  

        Annotations are defined as (constrained) interface types and are
        processed by compiler plugins, which may interpret the
        annotation symbolically.

        A type \xcd"T" is annotated by interface types
        \xcdmath"A$_1$", \dots,
        \xcdmath"A$_n$"
        using the syntax
        \xcdmath"@A$_1$ $\dots$ @A$_n$ T".

\section{Subtyping and type equivalence}\label{DepType:Equivalence}
\index{type equivalence}
\index{subtyping}

Intuitively, type \xcdmath"T$_1$" is a subtype of type \xcdmath"T$_2$", 
written \xcdmath"T$_1$ <: T$_2$", 
if
every instance of \xcdmath"T$_1$" is also an instance of \xcdmath"T$_2$".  For
example, \xcd`Child` is a subtype of \xcd`Person` (assuming a suitably defined
class hierarchy): every child is a person.  Similarly, \xcd`Int{self != 0}`
is a subtype of \xcd`Int` -- every non-zero integer is an integer.  

This section formalizes the concept of subtyping. Subtyping of types depends
on a {\em type context}, \viz. a set of constraints on type parameters
and variables that occur in the type.
For example: 

%~~gen ^^^ Types400
% package Types.subtyping.cons;
% NOCOMPILE
%~~vis
\begin{xten}
class ConsTy[T,U] {
   def upcast(t:T){T <: U} :U = t;
}
\end{xten}
%~~siv
%
%~~neg
\noindent
Inside \xcd`upcast`, \xcd`T` is constrained to be a subtype of \xcd`U`, and so
\xcd`T <: U` is true, and \xcd`t` can be treated as a value of type \xcd`U`.  
Outside of \xcd`upcast`, there is no reason to expect any relationship between
them, and \xcd`T <: U` may be false.
However, subtyping of types that have no free variables does not depend
on the context.    \xcd`Int{self != 0} <: Int` is always
true.

\limitation{Subtyping of type variables does not work under all circumstances
in the X10 2.2 implementation.}


\begin{itemize}
\item {\bf Reflexivity:} Every type \xcd`T` is a subtype of itself: \xcd`T <: T`.

\item {\bf Transitivity:} If \xcd`T <: U` and \xcd`U <: V`, then \xcd`T <: V`. 

\iffalse
{\bf Class types:}  
Given the definition 
\xcd`class C[$\vec{X}$] extends D[$\vec{Y}$]{d} implements I1, ..., In {...}`
where {$\vec{X}$} is a vector of type variables, and 
{$\vec{Y$} a vector of types possibly involving variables from {$\vec{X}$}, 
and {$\vec{T$} an instantiation of {$\vec{X$} and {$\vec{U$} the corresponding
instantiation of {$\vec{Y$}, 
then 
\xcdmath"C[$\vec{T}$]`"is a subtype of \xcd`D[$\vec{U}$]{d}`, \xcd`I1`, ..., \xcd`In`. 

\item
{\bf Interface types:}  
Given the definition 
\xcdmath"interface I[$\vec{X}$] extends I1, ... In {...}`"
then \xcdmath"I` is a subtype of \xcd`"1`, ..., \xcd`In`.

\item 
{\bf Struct types:} 
Given the definition 
\xcdmath"struct S implements I1, ..., In {...}`"then \xcd`S` is a 
subtype of \xcd`I1`, ..., \xcd`In`. 
\fi

\item {\bf Direct Subclassing:} 
Let {$\vec{X}$} be a (possibly empty) vector of type variables, and
{$\vec{Y}$}, {$\vec{Y_i}$} be vectors of type terms over {$\vec{X}$}.
Let {$\vec{T}$} be an instantiation of {$\vec{X}$}, 
and {$\vec{U}$}, {$\vec{U_i}$} the corresponding instantiation of 
{$\vec{Y}$}, {$\vec{Y_i}$}.  Let \xcd`c` be a constraint, and \xcdmath"c$'$"
be the corresponding instantiation.
We elide properties, and interpret empty vectors as absence of the relevant
clauses. 
Suppose that \xcd`C` is declared by one of the
forms: 
\begin{enumerate}
\item \xcdmath"class C[$\vec{X}$]{c} extends D[$\vec{Y}$]{d}"\\
\xcdmath"implements I$_1[\vec{Y_1}]${i$_1$},...,I$_n[\vec{Y_n}]${i$_n$}{"
\item \xcdmath"interface C[$\vec{X}$]{c} extends I$_1[\vec{Y_1}]${i$_1$},...,I$_n[\vec{Y_n}]${i$_n$}{"
\item \xcdmath"struct C[$\vec{X}$]{c} implements I$_1[\vec{Y_1}]${i$_1$},...,I$_n[\vec{Y_n}]${i$_n$}{"
\end{enumerate}
Then: 
\begin{enumerate}
\item \xcdmath"C[$\vec{T}$] <: D[$\vec{U}$]{d}" for a class
\item \xcdmath"C[$\vec{T}$] <: I$_i$[$\vec{U_i}$]{i$_i$}" for all cases.
\item \xcdmath"C[$\vec{T}$] <: C[$\vec{T}$]{c$'$}" for all cases.
\end{enumerate}


\item
{\bf Function types:}
\begin{xtenmath}
(x$_1$: T$_1$, $\dots$, x$_n$: T$_n$){c} => T
\end{xtenmath}
is a  subtype of 
\begin{xtenmath}
(x$'_1$: T$'_1$, $\dots$, x$'_n$: T$'_n$){c$'$} => T$'$
\end{xtenmath}
if: 
\begin{enumerate}
\item Each \xcdmath"T$_i$ <: T$'_i$";
\item \xcdmath"c[x$'_1$, $\ldots$, x$'_n$ / x$_1$, $\ldots$, x$_n$]" entails \xcdmath"c$'$";
\item \xcdmath"T$'$ <: T";
\end{enumerate}

\item
{\bf Constrained types:}
\xcd`T{c}` is a subtype of \xcd`T{d}` if \xcd`c` entails \xcd`d`. 

\item {\bf Any:} 
Every type \xcd`T` is a subtype of \xcd`x10.lang.Any`.

\item 
{\bf Type Variables:}
Inside the scope of a constraint \xcd`c` which entails \xcd`A <: B`, we have
\xcd`A <: B`.  \eg, \xcd`upcast` above.


%%NO-VARIANCE%% \item 
%%NO-VARIANCE%% {\bf Covariant Generic Types:} 
%%NO-VARIANCE%% If \xcd`C` is a generic type whose {$i$}th type parameter is covariant, 
%%NO-VARIANCE%% and {\xcdmath"T$'_i$ <: T$_i$"}
%%NO-VARIANCE%% and  {\xcdmath"T$'_j$ == T$_j$"} for all {$j \ne i$}, 
%%NO-VARIANCE%% then {\xcdmath"C[T$'_1$, $\ldots$, T$'_n$] <: C[T$'_1$, $\ldots$, T$'_n$]"}.
%%NO-VARIANCE%% \Eg, \xcd`class C[T1, +T2, T3]` with {$i=2$}, and \xcd"U2 <: T2", then
%%NO-VARIANCE%% \xcd`C[T1,U2,T3] <: C[T1,T2,T3]`.
%%NO-VARIANCE%% 
%%NO-VARIANCE%% \item 
%%NO-VARIANCE%% {\bf Contravariant Generic Types:} 
%%NO-VARIANCE%% If \xcd`C` is a generic type whose {$i$}th type parameter is contravariant, 
%%NO-VARIANCE%% and \xcdmath"T$'_i$ <: T$_i$"
%%NO-VARIANCE%% and  \xcdmath"T$'_j$ == T$_j$" for all {$j \ne i$}, 
%%NO-VARIANCE%% then \xcdmath"C[T$'_1$, $\ldots$, T$'_n$] :> C[T$'_1$, $\ldots$, T$'_n$]".
%%NO-VARIANCE%% \Eg, \xcd`class C[T1, -T2, T3]` with {$i=2$}, and \xcdmath"U2 <: T2", then
%%NO-VARIANCE%% \xcd`C[T1,U2,T3] :> C[T1,T2,T3]`.
%%NO-VARIANCE%% 

\end{itemize}


Two types are {\em equivalent}, \xcd`T == U`, if \xcd`T <: U` and \xcd`U <: T`. 


\section{Common ancestors of types}
\label{LCA}

There are several situations where X10 must find a type \xcd`T` that describes
values of two or more different types.  This arises when X10 is trying to find
a good type for: 
\begin{itemize}
%~~exp~~`~~`~~test:Boolean ~~ ^^^ Types410
\item Conditional expressions, like \xcd`test ? 0 : "non-zero"` or even \\
%~~exp~~`~~`~~test:Boolean ~~ ^^^ Types420
      \xcd`test ? 0 : 1`;
%~~exp~~`~~`~~ ~~ ^^^ Types430
\item Array construction, like \xcd`[0, "non-zero"]` and 
%~~exp~~`~~`~~ ~~ ^^^ Types440
      \xcd`[0,1]`;
\item Functions with multiple returns, like
%~~gen ^^^ Types450
% package Types_odd_inferred_return_type;
% class Examplerator {
%~~vis
\begin{xten}
def f(a:Int) {
  if (a == 0) return 0;
  else return "non-zero";
}
\end{xten}
%~~siv
%}
%~~neg
\end{itemize}

In some cases, there is a unique best type describing the expression.  For
example, if \xcd`B` and \xcd`C` are direct subclasses of \xcd`A`, \xcd`pick`
will have return type \xcd`A`: 
%~~gen ^^^ Types_uniq
% package Types.For.Gripes.About.Pipes.Full.Of.Wipes;
%  class A {} class B extends A{} class C extends A{}
% class D {
%~~vis
\begin{xten}
static def pick(t:Boolean, b:B, c:C) = t ? b : c;  
\end{xten}
%~~siv
%}
%~~neg

However, in many common cases, there is no unique best type describing the
expression.  For example, consider the expression {$E$} 
\begin{xtenmath}
b ? 0 : 1   // Call this expression $E$
\end{xtenmath}
The
best type of \xcd`0` 
is \xcd`Int{self==0}`, and the best type of 1 is \xcd`Int{self==1}`.
Certainly {$E$} could be given the type \xcd`Int`, or even \xcd`Any`, and that
would describe all possible results.  However, we actually know more.
\xcd`Int{self != 2}` is a better description of the type of {$E$}---certainly
the result of {$E$} can never be \xcd`2`.   \xcd`Int{self != 2, self != 3}` is
an even better description; {$E$} can't be \xcd`3` either.  We can continue
this process forever, adding integers which {$E$} will definitely not return
and getting better and better approximations. (If the constraint
sublanguage had \xcd`||`, we could give it the type 
\xcd`Int{self == 0 || self == 1`, which would be nearly perfect.  But 
\xcd`||` makes typechecking far more expensive, so it is excluded.)
No X10 type is the best description of {$E$}; there is always a better one.

Similarly, consider two unrelated interfaces: 
%~~gen ^^^ Types460
% package Types.For.Gripes.About.Snipes;
%~~vis
\begin{xten}
interface I1 {}
interface I2 {}
class A implements I1, I2 {}
class B implements I1, I2 {}
class C {
  static def example(t:Boolean, a:A, b:B) = t ? a : b;
}
\end{xten}
%~~siv
%
%~~neg
\xcd`I1` and \xcd`I2` are both perfectly good descriptions of \xcd`t ? a : b`, 
but neither one is better than the other, and there is no single X10 type
which is better than both. (Some languages have {\em conjunctive
    types}, and could say that the return type of \xcd`example` was 
\xcd`I1 && I2`.  This, too, complicates typechecking.)


So, when confronted with expressions like this, X10 computes {\em some}
satisfactory type for the expression, but not necessarily the {\em best} type.  
X10 provides certain guarantees about the common type \xcd`V{v}` computed for 
\xcd`T{t}` and \xcd`U{u}`: 
\begin{itemize}
\item If \xcd`T{t} == U{u}`, then \xcd`V{v} == T{t} == U{u}`.  So, if X10's
      algorithm produces an utterly untenable type for \xcd`a ? b : c`, and
      you want the result to have type \xcd`T{t}`, you can 
      (in the worst case) rewrite it to 
\begin{xtenmath}
a ? b as T{t} : c as T{t}
\end{xtenmath}
\item If \xcd`T == U`, then \xcd`V == T == U`.  For example, 
      X10 will compute the type of \xcd`b ? 0 : 1` as 
      \xcd`Int{c}` for some constraint \xcd`c`---perhaps simply 
      picking \xcd`Int{true}`, \viz, \xcd`Int`. 
\item X10 preserves place information about \xcd`GlobalRef`s, because it is so important. If both
      \xcd`t` and \xcd`u` entail \xcd`self.home==p`, then  
      \xcd`v` will also entail \xcd`self.home==p`.  
\item X10 similarly preserves nullity information.  If \xcd`t` and \xcd`u`
      both entail \xcd`x == null` or \xcd`x != null` for some variable
      \xcd`x`, then \xcd`v` will also entail it as well.

\item The computed upper bound of function types with the {\em same} argument
      types is found by computing the upper bound of the result types.  
      If 
      \xcdmath"T = (T$_1$, $\ldots$, T$_n$) => T'"
      and 
      \xcdmath"U = (T$_1$, $\ldots$, T$_n$) => U'", 
      and \xcd`V'` is the computed upper bound of \xcd`T'` and \xcd`U'`, 
      then the computed upper bound of \xcd`T` and \xcd`U` is 
      \xcdmath"U = (T$_1$, $\ldots$, T$_n$) => V'".
      (But, if the argument types are different, the computed upper bound may
      be \xcd`Any`.)

\end{itemize}

%\subsection{Syntactic abbreviations}\label{DepType:SyntaxAbbrev}

\section{Fundamental types}

Certain types are used in fundamental ways by X10.  

\subsection{The interface {\tt Any}}

It is quite convenient to have a type which all values are instances of; that
is, a supertype of all types.\footnote{Java, for one, suffers a number of
  inconveniences because some built-in types like \xcd`int` and \xcd`char`
  aren't subtypes of anything else.}  X10's universal supertype is the
  interface \xcd`Any`. 

\begin{xten}
package x10.lang;
public interface Any {
  def toString():String;
  def typeName():String;
  def equals(Any):Boolean;
  def hashCode():Int;
}
\end{xten}

\xcd`Any` provides a handful of essential methods that make sense and are
useful for everything. \xcd`a.toString()` produces a
string representation of \xcd`a`, and \xcd`a.typeName()` the string
representation of its type; both are useful for debugging.  \xcd`a.equals(b)`
is the programmer-overridable equality test, and \xcd`a.hashCode()` an integer
useful for hashing.  


\subsection{The class {\tt Object}}
\label{Object}
\index{\Xcd{Object}}
\index{\Xcd{x10.lang.Object}}

The class \xcd"x10.lang.Object" is the supertype of all classes.
A variable of this type can hold a reference to any object.
\xcd`Object` implements \xcd`Any`.



\section{Type inference}
\label{TypeInference}
\index{type!inference}
\index{type inference}

\XtenCurrVer{} supports limited local type inference, permitting
certain variable types and return types to be elided.
It is a static error if an omitted type cannot be inferred or
uniquely determined. Type inference does not consider coercions.

\subsection{Variable declarations}

The type of a \xcd`val` variable declaration can be omitted if the
declaration has an initializer.  The inferred type of the
variable is the computed type of the initializer.
For example, 
%~~stmt~~`~~`~~ ~~ ^^^ Types470
\xcd`val seven = 7;`
is identical to 
\begin{xtenmath}
val seven: Int{self==7} = 7;
\end{xtenmath}
Note that type inference gives the most precise X10 type, which might be more
specific than the type that a programmer would write.



\limitation{At the moment,  \xcd`var` declarations may not have their types
elided in this way.  
}

\subsection{Return types}

The return type of a method can be omitted if the method has a body (\ie, is
not \xcd"abstract" or \xcd"native"). The inferred return type is the computed
type of the body.  In the following example, the return type inferred for
\xcd`isTriangle` is 
%~~type~~`~~`~~ ~~ ^^^ Types490
\xcd`Boolean{self==false}`
%~~gen ^^^ Types500
% package Types.Inferred.Return;
%~~vis
\begin{xten}
class Shape {
  def isTriangle() = false; 
}  
\end{xten}
%~~siv
%
%~~neg
Note that, as with other type inference, methods are given the most specific
type.  In many cases, this interferes with subtyping.  For example, if one
tried to write: 
\begin{xten}
class Triangle extends Shape {
  def isTriangle() = true;
}
\end{xten}
\noindent
the compiler would reject this program for attempting to override
\xcd`isTriangle()` by a method with the wrong type, \viz,
\xcd`Boolean{self==true}`.  In this case, supply the type that is actually
intended for \xcd`isTriangle`: 
\begin{xtenmath}def isTriangle() :Boolean =false;
\end{xtenmath}

The return type of a closure can be omitted.
The inferred return type is the computed type of the body.

The return type of a constructor can be omitted if the
constructor has a body.
The inferred return type is the enclosing class type with
properties bound to the arguments in the constructor's \xcd"property"
statement, if any, or to the unconstrained class type.
For example, the \xcd`Spot` class has two constructors, the first of which has
inferred return type \xcd`Spot{x==0}` and the second of which has 
inferred return type \xcd`Spot{x==xx}`. 
%~~gen ^^^ Types510
% package Types.Inferred.By.Phone;
%~~vis
\begin{xten}
class Spot(x:Int) {
  def this() {property(0);}
  def this(xx: Int) { property(xx); }
}
\end{xten}
%~~siv
%class Confirm{ 
% static val s0 : Spot{x==0} = new Spot();
% static val s1 : Spot{x==1} = new Spot(1);
%}
%~~neg


\index{void}

A method or closure that has expression-free \xcd`return` statements
(\xcd`return;` rather than \xcd`return e;`) is said to return \xcd`void`.
\xcd`void` is not a type; there are no \xcd`void` values, nor can \xcd`void`
be used as the argument of a generic type. However, \xcd`void` takes the
syntactic place of a type in a few contexts. A method returning \xcd`void` can be specified by
\xcd`def m():void`: 

%~~gen ^^^ Types520
% package Types.voidd;
% class voidddd {
% static 
%~~vis
\begin{xten}
val f : () => void = () => {return;};
\end{xten}
%~~siv
%}
%~~neg

By a convenient abuse of language, \xcd`void` is sometimes
lumped in with types; \eg, we may say ``return type of a method'' rather than
the formally correct but rather more awkward ``return type of a method, or
\xcd`void`''.   Despite this informal usage, \xcd`void` is not a type.  For
example, given 
%~~gen ^^^ Types530
% package Types.void_is_not_a_type;
% class EEEEVil {
%~~vis
\begin{xten}
  static def eval[T] (f:()=>T):T = f();
\end{xten}
%~~siv
% }
%~~neg
\noindent
The call \xcd`eval[void](f)` does {\em not} typecheck; \xcd`void` is not a
type and thus cannot be used as a type argument.  There is no way in X10 to
write a generic function which works with both functions which return a value
and functions which do not.  In most cases, functions which have no sensible
return value can be provided with a dummy return value.

\subsection{Inferring Type Arguments}
\label{TypeParamInfer}


A call to a polymorphic method %, closure, or constructor 
may omit the
explicit type arguments.  
X10 will compute a type from the types of the actual arguments. 

(Exception: it is an error if the method call provides no information about
a type parameter that must be inferred.  For example, given the method
definition \xcd`def m[T](){...}`, an invocation \xcd`m()` is considered a
static error.  The compiler has no idea what \xcd`T` the programmer intends.)



\begin{ex}Consider the following method, which chooses one of its arguments.  (A more
sophisticated one might sometimes choose the second argument, but that does
not matter for the sake of this example.)
\begin{xten}
static def choose[T](a: T, b: T): T = a; 
\end{xten}


The type argument \xcd`T` can always be supplied: 
\xcd`choose[Int](1, 2)` picks an integer, 
and \xcd`choose[Any](1, "yes")` picks a value that might be an integer or a
string.  
However, the type argument can be elided.  Suppose that \xcd`Sub <: Super`;
then the following compiles: 

%~~gen ^^^ Types540
% package Types.GenericInference;
% class Exampllll{ 
%~~vis
\begin{xten}
  static def choose[T](a: T, b: T): T = a; 
  static val j : Any = choose("string", 1);
  static val k : Super = choose(new Sub(), new Super());
\end{xten}
%~~siv
%}
% class Super {}
% class Sub extends Super {}
%~~neg
\end{ex}

The type parameter doesn't need to be the type of a variable. It can be found
inside of the type of a variable; X10 can extract it.

\begin{ex}
The \xcd`first` method below returns the first element of a one-dimensional
array.  The type parameter \xcd`T` represents the type of the array's
elements. There is no parameter of type \xcd`T`. There is one of type
\xcd`Array[T]{c}`; X10 strips off the constraint \xcd`{c}` and the
\xcd`Array[...]` type to get at the \xcd`T` inside.
%~~gen ^^^ Types3d5j
% package Types3d5j;
% class Example {
%~~vis
\begin{xten}
static def first[T](x:Array[T](1)) = x(0);
static def example() {
  val ss <: Array[String] = ["X10", "Scala", "Thorn"];
  val s1 = first(ss);
  assert s1.equals("X10");
}
\end{xten}
%~~siv
%}
% class Hook{ def run() {Example.example(); return true;}}
%~~neg

\end{ex}


\subsubsection{Sketch of X10 Type Inference for Method Calls}

When the X10 compiler sees a method call 
\begin{xtenmath}
a.m(b$_1$, $\ldots$,b$_n$)
\end{xtenmath}
and attempts to infer type parameters to
see if it could be a use of a
method 
\begin{xtenmath}
def m[X$_1$, $\ldots$, X$_t$](y$_1$: S$_1$, $\ldots$, y$_n$:S$_n$),
\end{xtenmath}
it reasons as follows. 



Suppose that \xcdmath"b$_i$" has type \xcdmath"T$_i$".  Then, X10 is seeking a
set of type {$B$} bindings 
\begin{xtenmath}
X$_j$ = U$_j$, 
\end{xtenmath}
for $1 \le j \le t$, 
such that 
\xcdmath"T$_i$ <: S$^*_i$" for {$1 \le i \le n$}, where \xcdmath"S$^*$" is
\xcd`S` with each type variable \xcdmath"X$_j$" replaced by the corresponding
\xcdmath"U$_j$".  If it can find such a {$B$}, it has a usable choice of type
arguments and can do the type inference.  If it cannot find {$B$}, then it
cannot do type inference.    (Note that X10's type inference algorithm is
incomplete -- there may {\em be} such a {$B$} that X10 cannot find.  If this
occurs in your program, you will have to write down the type arguments
explicitly.) 

Let $B_0$ be the set {$\{ T_i \subtype S_i | 1 \le i \le n\}$}.  Let
{$B_{n+1}$} be {$B_n$} with one element {$F \subtype G$} or 
{$F \typeeq G$} removed, and
{$C(F \subtype G)$} 
or {$C(F \typeeq G)$} (defined below) added.  Repeat this until 
{$B_n$} consists entirely of comparisons with type variables (\viz, 
\xcdmath"Y$_j$ == U", 
\xcdmath"Y$_j$ <: U", and
\xcdmath"Y$_j$ :> U"), 
or until some {$n$} exceeds a predefined compiler limit. 

The candidate inferred types may be read off of {$B_n$}.  The guessed binding
for \xcdmath"X$_j$" is: 
\begin{itemize}
\item If there is an equality \xcdmath"X$_j$==W" in {$B_n$}, then guess the
      binding \xcdmath"X$_j$=W".  Note that there may be several such
      equalities with different choices of \xcd`W`; pick any one.  If the
      chosen binding does not equal the others, the candidate binding will be
      rejected later. 
\item Otherwise, if there is one or more upper bounds 
\xcdmath"X$_j$ <: V$_k$" in {$B_n$}, guess the binding 
\xcdmath"X$_j$ = V$_+$", where 
\xcdmath"V$_+$" is the computed lower bound of all the \xcdmath"V$_k$"'s.
\item Otherwise, if there is one or more lower bounds 
\xcdmath"R$_k$ <: X$_j$", guess that
\xcdmath"X$_j$ = R$_+$", where 
\xcdmath"R$_+$" is the computed upper bound of all the \xcdmath"R$_k$"'s.
\end{itemize}
If this does not yield a binding for some variable \xcdmath"X$_j$", then type
inference fails.  Furthermore, if every variable \xcdmath"X$_j$" is given a
binding \xcdmath"U$_j$", but the 
bindings do not work --- 
that is, if 
\xcdmath"a.m[U$_1$, $\ldots$, U$_t$](b$_1$, $\ldots$,b$_n$)"
is not a call of 
the original method 
\xcdmath"def m[X$_1$, $\ldots$, X$_t$](y$_1$: S$_1$, $\ldots$, y$_n$:S$_n$)"
--- then type inference also fails.

\paragraph{Computation of the Replacement Elements}

Given a type relation
{$r$} of the form {$F \subtype G$}
or {$F \typeeq G$}, we compute the set {$C(r)$} of
replacement constraints.  There are a number of cases; we present only the
interesting ones. 

\begin{itemize}
\item If $F$ has the form \xcdmath"$F'${c}", then  
\xcdmath"$C(r)$" is defined to be
 \xcdmath"$F'$ == $G$" if $r$ is an equality, or 
 \xcdmath"$F'$ <: $G$" if {$r$} is a subtyping.
That is, we erase type constraints.  
Validity is not an issue at this point in the algorithm, as 
we check at the end that the result is valid.
Note that, if the equation had the form \xcdmath"Z{c} == A", it could be
solved by either \xcd`Z==A` or by \xcd`Z = A{c}`.  By dropping constraints in this
rule, we choose the former solution. 

\item Similarly, we drop constraints on {$G$} as well.

\item If {$F$} has the form \xcdmath"K[F$_1$, $\ldots$, F$_k$]"
and 
{$G$}
has the form \xcdmath"K[G$_1$, $\ldots$, G$_k$]", 
then {$C(r)$} has one type relation comparing each parameter of 
{$F$} with the corresponding one of {$G$}: 
\[C(r) = \{ F_l \typeeq G_l | 1 \le l \le k \} \]

For example, the constraint \xcdmath"List[X] == List[Y]" induces the
constraint \xcd`X==Y`.  
\xcd`List[X] <: List[Y]` also induces the same constraint.  The only way that
\xcd`List[X]` could be a subtype of \xcd`List[Y]` in X10 is if \xcd`X==Y`.
List of different types are incomparable.\footnote{The situation would be more
complex if X10 had covariant and contravariant types.}

\item Other cases are fairly routine.  \Eg, if {$F$} is a \xcd`type`-defined
      abbreviation, it is expanded.

\end{itemize}

\begin{ex}
Consider the program: 
%~~gen ^^^ Types1s4y
% package Types1s4y;
%~~vis
\begin{xten}
import x10.util.*;
class Cl[C1, C2, C3]{}
class Example {
  static def me[X1, X2](Cl[Int, X1, X2]) = 
     new Cl[X1, X2, Point]();
  static def example() {
    val a = new Cl[Int, Boolean, String]();
    val b : Cl[Boolean, String, Point] 
          = me[Boolean, String](a);
    val c : Cl[Boolean, String, Point] 
          = me(a);
  }
}
\end{xten}
%~~siv
%
%~~neg
The method call for \xcd`b` has explicit type parameters.  
The call for \xcd`c` infers the parameters.  The computation 
starts with one equation, saying that the type of the formal parameter of 
\xcd`me` has to be the same as the type of the actual parameter, \viz, the
type of \xcd`a`:
\begin{xtenmath}
Cl[Int, X1, X2] == Cl[Int, Boolean, String]
\end{xtenmath}
Note that both terms are \xcd`Cl` of three things. 
This is broken into three equations: 
\begin{xtenmath}
Int == Int
\end{xtenmath}
which is easy to satisfy,
\begin{xtenmath}
X1 == Boolean
\end{xtenmath}
which suggests a possible value for \xcd`X1`,  and 
\begin{xtenmath}
X2 == String
\end{xtenmath}
which suggests a value for \xcd`X2`.  
All of these equations are simple enough, so the algorithm terminates. 

Then, X10 confirms that the binding \xcd`X1==Boolean`, \xcd`X2==String`
atually generates a correct call, which it does.  
\end{ex}

\section{Type Dependencies}

Type definitions may not be circular, in the sense that no type may be its own
supertype, nor may it be a container for a supertype. This forbids interfaces
like \xcd`interface Loop extends Loop`, and indirect self-references such as
\xcd`interface A extends B.C` where \xcd`interface B extends A`.  
The formal definition of this is based on Java's.  

An {\em entity type} is a class, interface, or struct type.   

Entity type $E$ {\em directly depends on} entity type $F$ if $F$ is mentioned
in the \xcd`extends` or \xcd`implements` clause of $E$, either by itself or as
a qualifier within a super-entity-type name.  

\begin{ex}
In the following, \xcd`A` directly depends on \xcd`B`, \xcd`C`, \xcd`D`, 
\xcd`E`, and \xcd`F`.    It does not directly depend on \xcd`G`.
%~~gen ^^^ Types6a9m
% package Types6a9m;
% NOTEST
% class B{ static class C{}}
% class D{ static interface E{}}
% interface F[X]{}
% class G{}
%~~vis
\begin{xten}
class A extends B.C implements D.E, F[G] {}
\end{xten}
%~~siv
%
%~~neg

It is an ordinary programming idiom to use \xcd`A` as an argument to a generic
interface that \xcd`A` implements.  For example, \xcd`ComparableTo[T]`
describes things which can be compared to a value of type \xcd`T`. Saying that
\xcd`A` implements \xcd`ComparableTo[A]` means that one \xcd`A` can be
compared to another, which is reasonable and useful: 
%~~gen ^^^ Types2x6d
% package Types2x6d;
%~~vis
\begin{xten}
interface ComparableTo[T] {
  def eq(T):Boolean;
}
class A implements ComparableTo[A] {
  public def eq(other:A) = this.equals(other);
}
\end{xten}
%~~siv
%
%~~neg
\end{ex}

Entity type $E$ {\em depends on} entity type $F$ if
either $E$ directly depends on $F$, or $E$ directly depends on an entity type
that depends on $F$.   That is, the relation ``depends on'' is the transitive
closure of the relation ``directly depends on''.  

It is a static error if any entity type $E$ depends on itself.

\section{Limitations of Strict Typing}

X10's type checking provides substantial guarantees.  In most cases, a program
that passes the X10 type checker will not have any runtime type errors.
However, there are a modest number of compromises with practicality in the
type system: places where a program can pass the typechecker and still have a
type error.

\begin{enumerate}


\item The library type \xcd`IndexedMemoryChuck` provides a low-level interface
      to blocks of memory.  A few methods on that class are not type-safe. See
      the API if you must.

\item Custom serialization (\Sref{sect:ser+deser}) allows user code to
      construct new objects in ways that can subvert the type system.

\item Code written to use the underlying Java or C++ (\Sref{NativeCode}) can
      break X10's guarantees.

\end{enumerate}
	

\chapter{Variables}\label{XtenVariables}\index{variable}

%%OLDA variable is a storage location.  \Xten{} supports seven kinds of
%%OLDvariables: constant {\em class variables} (static variables), {\em
%%OLD  instance variables} (the instance fields of a class), {\em array
%%OLD  components}, {\em method parameters}, {\em constructor parameters},
%%OLD{\em exception-handler parameters} and {\em local variables}.

A {\em variable} is an X10 identifier associated with a value within some
context. Variable bindings have these essential properties:
\begin{itemize}
\item {\bf Type:} What sorts of values can be bound to the identifier;
\item {\bf Scope:} The region of code in which the identifier is associated
      with the entity;
\item {\bf Lifetime:} The interval of time in which the identifier is
      associated with the entity.
\item {\bf Visibility:} Which parts of the program can read or manipulate the
      value through the variable.
\end{itemize}



X10 has many varieties of variables, used for a number of purposes. 
\begin{itemize}
\item Class variables, also known as the static fields of a class, which hold
      their values for the lifetime of the class.  
\item Instance variables, which hold their values for the lifetime of an
      object;
\item Array elements, which are not individually named and hold their values
      for the lifetime of an array;
\item Formal parameters to methods, functions, and constructors, which hold
      their values for the duration of method (etc.) invocation;
\item Local variables, which hold their values for the duration of execution
      of a block.
\item Exception-handler parameters, which hold their values for the execution
      of the exception being handled. 
\end{itemize}
A few other kinds of things are called variables for historical reasons; \eg,
type parameters are often called type variables, despite not being variables
in this sense because they do not refer to X10 values.  Other named entities,
such as classes and methods, are not called variables.  However, all
name-to-whatever bindings enjoy similar concepts of scope and visibility.  

\begin{ex}
In the following example, 
\xcd`n` is an instance variable, and \xcd`nxt` is a
local variable defined within the method \xcd`bump`.\footnote{This code is
unnecessarily turgid for the sake of the example.  One would generally write
\xcd`public def bump() = ++n;`.   }
%~~gen ^^^ Vars10
% package Vars.For.Squares;
%~~vis
\begin{xten}
class Counter {
  private var n : Int = 0;
  public def bump() : Int {
    val nxt = n+1;
    n = nxt;
    return nxt;
    }
}
\end{xten}
%~~siv
% class Hook{ def run() { val c = new Counter(); val d = new Counter();
%   assert c.bump() == 1;  
%   assert c.bump() == 2;  
%   assert c.bump() == 3;  
%   assert c.bump() == 4;  
%   assert d.bump() == 1;  
%   assert c.bump() == 5;  
%   return true;
% } }
%~~neg
Both variables have type \xcd`Int` (or
perhaps something more specific).    The scope of \xcd`n` is the body of
\xcd`Counter`; the scope of \xcd`nxt` is the body of \xcd`bump`.  The
lifetime of \xcd`n` is the lifetime of the \xcd`Counter` object holding it;
the lifetime of \xcd`nxt` is the duration of the call to \xcd`bump`. Neither
variable can be seen from outside of its scope.
\end{ex}
\label{exploded-syntax}
\label{VariableDeclarations}
\index{variable declaration}


Variables whose value may not be changed after initialization are said to be
{\em immutable}, or {\em constants} (\Sref{FinalVariables}), or simply
\xcd`val` variables. Variables whose value may change are {\em mutable} or
simply \xcd`var` variables. \xcd`var` variables are declared by the \xcd`var`
keyword. \xcd`val` variables may be declared by the \xcd`val` keyword; when a
variable declaration does not include either \xcd`var` or \xcd`val`, it is
considered \xcd`val`. 

A variable---even a \xcd`val` -- can be declared in one statement, and then
initialized later on.  It must be initialized before it can be used
(\Sref{sect:DefiniteAssignment}).  


\begin{ex}
The following example illustrates many of the variations on variable
declaration: 
%~~gen ^^^ Vars20
%package Vars.For.Bears.In.Chairs;
%class VarExample{
%static def example() {
%~~vis
\begin{xten}
val a : Int = 0;               // Full 'val' syntax
b : Int = 0;                   // 'val' implied
val c = 0;                     // Type inferred
var d : Int = 0;               // Full 'var' syntax
var e : Int;                   // Not initialized
var f : Int{self != 100} = 0;  // Constrained type
val g : Int;                   // Init. deferred
if (a > b) g = 1; else g = 2;  // Init. done here.
\end{xten}
%~~siv
%}}
%~~neg
\end{ex}





\section{Immutable variables}
\label{FinalVariables}
\index{variable!immutable}
\index{immutable variable}
\index{variable!val}
\index{val}

%##(LocalVariableDeclarationStatement LocalVariableDeclaration
\begin{bbgrammar}
%(FROM #(prod:LocVarDeclnStmt)#)
     LocVarDeclnStmt \: LocVarDecln \xcd";" & (\ref{prod:LocVarDeclnStmt}) \\
%(FROM #(prod:LocVarDecln)#)
         LocVarDecln \: Mods\opt VarKeyword VariableDeclrs & (\ref{prod:LocVarDecln}) \\
                     \| Mods\opt VarDeclsWType \\
                     \| Mods\opt VarKeyword FormalDeclrs \\
\end{bbgrammar}
%##)

An immutable (\xcd`val`) variable can be given a value (by initialization or
assignment) at 
most once, and must be given a value before it is used.  Usually this is
achieved by declaring and initializing the variable in a single statement, 
such as \Xcd{val x = 3}, with syntax 
(\ref{prod:LocVarDecln}) using the {\it VariableDeclarators} or {\it
VarDelcsWType} alternatives.

\begin{ex}
After these declarations, \xcd`a` and \xcd`b` cannot be assigned to further,
or even redeclared:  
%~~gen ^^^ Vars30
% package Vars.In.Snares;
% class ABitTedious{
% def example() {
%~~vis
\begin{xten}
val a : Int = 10;
val b = (a+1)*(a-1);
// ERROR: a = 11;  // vals cannot be assigned to.
// ERROR: val a = 11; // no redeclaration.
\end{xten}
%~~siv
%}}
%~~neg

\end{ex}

In two special cases, the declaration and assignment are separate.  One 
case is how constructors give values to \xcd`val` fields of objects.  In this
case, production (\ref{prod:LocVarDecln}) is taken, with the {\it
FormalDeclarators} option, such as  \Xcd{var n:Int;}.  

\begin{ex} The
\xcd`Example` class has an immutable field \xcd`n`, which is given different
values depending on which constructor was called. \xcd`n` can't be given its
value by initialization when it is declared, since it is not knowable which
constructor is called at that point.  
%~~gen ^^^ Vars40
% package Vars.For.Cares;
%~~vis
\begin{xten}
class Example {
  val n : Int; // not initialized here
  def this() { n = 1; }
  def this(dummy:Boolean) { n = 2;}
}
\end{xten}
%~~siv
%
%~~neg
\end{ex}


The other case of separating declaration and assignment is in function
and method call, described in \Sref{sect:formal-parameters}.  The formal
parameters are bound to the corresponding actual 
parameters, but the binding does not happen until the function is called.  

\begin{ex}
In
the code below, \xcd`x` is initialized to \xcd`3` in the first call and
\xcd`4` in the second.
%~~gen ^^^ Vars50
%package Vars.For.Swears;
%class Examplement {
%static def whatever() {
%~~vis
\begin{xten}
val sq = (x:Int) => x*x;
x10.io.Console.OUT.println("3 squared = " + sq(3));
x10.io.Console.OUT.println("4 squared = " + sq(4));
\end{xten}
%~~siv
%}}
%~~neg
\end{ex}




%%IMMUTABLE%% An immutable variable satisfies two conditions: 
%%IMMUTABLE%% \begin{itemize}
%%IMMUTABLE%% \item it can be assigned to at most once, 
%%IMMUTABLE%% \item it must be assigned to before use. 
%%IMMUTABLE%% \end{itemize}
%%IMMUTABLE%% 
%%IMMUTABLE%% \Xten{} follows \java{} language rules in this respect \cite[\S
%%IMMUTABLE%% 4.5.4,8.3.1.2,16]{jls2}. Briefly, the compiler must undertake a
%%IMMUTABLE%% specific analysis to statically guarantee the two properties above.
%%IMMUTABLE%% 
%%IMMUTABLE%% Immutable local variables and fields are defined by the \xcd"val"
%%IMMUTABLE%% keyword.  Elements of value arrays are also immutable.
%%IMMUTABLE%% 
%%IMMUTABLE%% \oldtodo{Check if this analysis needs to be revisited.}

\section{Initial values of variables}
\label{NullaryConstructor}\index{nullary constructor}
\index{initial value}
\index{initialization}


Every assignment, binding, or initialization to a variable of type \xcd`T{c}`
must be an instance of type \xcd`T` satisfying the constraint \xcd`{c}`.
Variables must be given a value before they are used. This may be done by
initialization -- giving a variable a value as part
of its declaration. 

\begin{ex}
These variables are all initialized: 
%~~gen ^^^ Vars60
%package Vars.For.Bears;
%class VarsForBears{
%def check() {
%~~vis
\begin{xten}
  val immut : Int = 3;
  var mutab : Int = immut;
  val use = immut + mutab;
\end{xten}
%~~siv
%}}
%~~neg
\end{ex}

Or, a variable may be given a value by an assignment.  \xcd`var` variables may
be assigned to repeatedly.  \xcd`val` variables may only be assigned once; the
compiler will ensure that they are assigned before they are used.

\begin{ex}
The variables in the following example are given their initial values by
assignment.  Note that they could not be used before those assignments,
nor could \xcd`immu` be assigned repeatedly.
%~~gen ^^^ Vars70
%package Vars.For.Stars;
%abstract class VarsForStars{
% abstract def cointoss(): Boolean;
% abstract def println(Any):void;
%def check() {
%~~vis
\begin{xten}
  var muta : Int;
  // ERROR:  println(muta);
  muta = 4;
  val use2A = muta * 10;
  val immu : Int;
  // ERROR: println(immu);
  if (cointoss())   {immu = 1;}
  else              {immu = use2A;}
  val use2B = immu * 10;
  // ERROR: immu = 5;
\end{xten}
%~~siv
%}}
%~~neg
\end{ex}

Every class variable must be initialized before it is read, through
the execution of an explicit initializer. Every
instance variable must be initialized before it is read, through the
execution of an explicit or implicit initializer or a constructor.
Implicit initializers initialize \xcd`var`s to the default values of their
types (\Sref{DefaultValues}). Variables of types which do not have default
values are not implicitly initialized.



Each method and constructor parameter is initialized to the
corresponding argument value provided by the invoker of the method. An
exception-handling parameter is initialized to the object thrown by
the exception. A local variable must be explicitly given a value by
initialization or assignment, in a way that the compiler can verify
using the rules for definite assignment (\Sref{sect:DefiniteAssignment}).


\section{Destructuring syntax}
\index{variable declarator!destructuring}
\index{destructuring}
\Xten{} permits a \emph{destructuring} syntax for local variable
declarations with explicit initializers,  and for formal parameters, of type
\xcd`Point`, \Sref{point-syntax} and \xcd`Array`, \Sref{XtenArrays}.
A point is a sequence of zero or more \xcd`Int`-valued coordinates; an array
is an indexed collection of data. 
It is often useful to get at the coordinates or elements directly, in
variables.

%##(VariableDeclarator
\begin{bbgrammar}
%(FROM #(prod:VariableDeclr)#)
       VariableDeclr \: Id HasResultType\opt \xcd"=" VariableInitializer & (\ref{prod:VariableDeclr}) \\
                     \| \xcd"[" IdList \xcd"]" HasResultType\opt \xcd"=" VariableInitializer \\
                     \| Id \xcd"[" IdList \xcd"]" HasResultType\opt \xcd"=" VariableInitializer \\
\end{bbgrammar}
%##)

The syntax \xcdmath"val [a$_1$, $\ldots$, a$_n$] = e;", 
where \xcd`e` is a \xcd`Point`,
declares {$n$}
\xcd`Int` variables, bound to the precisely {$n$} components of the \xcd`Point` value of
\xcd`e`; it is an error if \xcd`e` is not a \xcd`Point` with precisely {$n$} components.
The syntax \xcdmath"val p[a$_1$, $\ldots$, a$_n$] = e;"  is similar, but also
declares the variable \xcd`p` to be of type \xcdmath"Point(n)".  


The syntax \xcdmath"val [a$_1$, $\ldots$, a$_n$] = e;", 
where \xcd`e` is an \xcd`Array[T]` for some type \xcd`T`,
declares {$n$}
variables of type \xcd`T`, bound to the precisely {$n$} components of the \xcd`Array[T]` value of
\xcd`e`; it is an error if \xcd`e` is not a \xcd`Array[T]` 
with \xcd`rank==1` and \xcdmath"size==$n$". 
The syntax \xcdmath"val p[a$_1$, $\ldots$, a$_n$] = e;"  is similar, but also
declares the variable \xcd`p` to be of type
\xcdmath"Array[T]{rank==1,size==n}".   


\begin{ex}
The following code makes an anonymous point with one coordinate \xcd`11`, and
binds \xcd`i` to \xcd`11`.  Then it makes a point with coordinates \xcd`22`
and \xcd`33`, binds \xcd`p` to that point, and \xcd`j` and \xcd`k` to \xcd`22`
and \xcd`33` respectively.
%~~gen ^^^ Vars80
% package Vars.For.Glares;
% class Example {
% static def example () {
%~~vis
\begin{xten}
val [i] : Point = Point.make(11);
assert i == 11;
val p[j,k] = Point.make(22,33);
assert j == 22 && k == 33;
val q[l,m] = [44,55] as Point; 
assert l == 44 && m == 55;
//ERROR: val [n] = p;
\end{xten}
%~~siv
%}}
% class Hook{ def run() {Example.example(); return true;}}
%~~neg

Destructuring is allowed wherever a \xcd`Point` or \xcd`Array[T]` variable is
declared, \eg, as the formal parameters of a method.
\begin{ex}
The methods below take a single argument each: a three-element point for
\xcd`example1`, a three-element array for \xcd`example2`.  The argument itself
is bound to \xcd`x` in both cases, and its elements are bound to \xcd`a`,
\xcd`b`, and \xcd`c`.  
%~~gen ^^^ Vars2e6j
% package Vars2e6j;
%class Example {
%~~vis
\begin{xten}
static def example1(x[a,b,c]:Point){}
static def example2(x[a,b,c]:Array[String]{rank==1,size==3}){}
\end{xten}
%~~siv
%}
%~~neg
\end{ex}

\end{ex}


\section{Formal parameters}
\label{sect:formal-parameters}
\index{formal parameter}
\index{parameter}


Formal parameters are the variables which hold values transmitted into a
method or function.  
They are always declared with a type.  (Type inference is not
available, because there is no single expression to deduce a type from.)
The variable name can be omitted if it is not to be used in the
scope of the declaration, as in the type of the method 
\xcd`static def main(Rail[String]):void` executed at the start of a program that
does not use its command-line arguments.

\xcd`var` and \xcd`val` behave just as they do for local
variables, \Sref{local-variables}.  In particular, the following \xcd`inc`
method is allowed, but, unlike some languages, does {\em not} increment its
actual parameter.  \xcd`inc(j)` creates a new local 
variable \xcd`i` for the method call, initializes \xcd`i` with the value of
\xcd`j`, increments \xcd`i`, and then returns.  \xcd`j` is never changed.
%~~gen ^^^ Vars100
% package Vars.For.Squares.Of.Mares;
% class Example {
%~~vis
\begin{xten}
static def inc(var i:Int) { i += 1; }
static def example() {
   var j : Int = 0;
   assert j == 0;
   inc(j);
   assert j == 0;
}
\end{xten}
%~~siv
%}
% class Hook{ def run() {Example.example(); return true;}}
%~~neg


\section{Local variables and Type Inference}
\label{local-variables}
\index{variable!local}
\index{local variable}
Local variables are declared in a limited scope, and, dynamically, keep their
values only for so long as the scope is being executed.  They may be \xcd`var`
or \xcd`val`.  
They may have 
initializer expressions: \xcd`var i:Int = 1;` introduces 
a variable \xcd`i` and initializes it to 1.
If the variable is immutable
(\xcd"val")
the type may be omitted and
inferred from the initializer type (\Sref{TypeInference}).

The variable declaration \xcd`val x:T=e;` confirms that \xcd`e`'s value is of
type \xcd`T`, and then introduces the variable \xcd`x` with type \xcd`T`.  For
example, consider a class \xcd`Tub` with a property \xcd`p`.
%~~gen ^^^ Vars_Tub
% package Vars.Local;
%~~vis
\begin{xten}
class Tub(p:Int){
  def this(pp:Int):Tub{self.p==pp} {property(pp);}
  def example() {
    val t : Tub = new Tub(3);
  }
}
\end{xten}
%~~siv
%
%~~neg
\noindent
produces a variable \xcd`t` of type \xcd`Tub`, even though the expression
\xcd`new Tub(3)` produces a value of type \xcd`Tub{self.p==3}` -- that is, a
\xcd`Tub`  whose \xcd`p` field is 3.  This can be inconvenient when the
constraint information is required.

\index{\Xcd{<:}}
\index{documentation type declaration}
Including type information in variable declarations is generally good
programming practice: it explains to both the compiler and human readers
something of the intent of the variable.  However, including types in 
\xcd`val t:T=e` can obliterate helpful information.  So, X10 allows a {\em
documentation type declaration}, written 
\begin{xtenmath}
val t <: T = e
\end{xtenmath}
This 
has the same effect as \xcd`val t = e`, giving \xcd`t` the full type inferred
from \xcd`e`; but it also confirms statically that that type is at least
\xcd`T`.  

\begin{ex}The following gives \xcd`t` the type \xcd`Tub{self.p==3}` as
desired.  However, a similar declaration with an inappropriate type will fail
to compile.
%~~gen ^^^ Vars_Var_Bounded
% package Vars.Local.not.the.express.plz;
% class Tub(p:Int){
%   def this(pp:Int):Tub{self.p==pp} {property(pp);}
%   def example() {
%     val t : Tub = new Tub(3);
%   }
% }
% class TubBounded{
% def example() {
%~~vis
\begin{xten}
   val t <: Tub = new Tub(3);
   // ERROR: val u <: Int = new Tub(3);
\end{xten}
%~~siv
%}}
%~~neg

\end{ex}



\section{Fields}
\index{field}
\index{object!field}
\index{struct!field}
\index{class!field}

%##(FieldDeclarators FieldDecln FieldDeclarator HasResultType  Mod
\begin{bbgrammar}
%(FROM #(prod:FieldDeclrs)#)
         FieldDeclrs \: FieldDeclr & (\ref{prod:FieldDeclrs}) \\
                     \| FieldDeclrs \xcd"," FieldDeclr \\
%(FROM #(prod:FieldDecln)#)
          FieldDecln \: Mods\opt VarKeyword FieldDeclrs \xcd";" & (\ref{prod:FieldDecln}) \\
                     \| Mods\opt FieldDeclrs \xcd";" \\
%(FROM #(prod:FieldDeclr)#)
          FieldDeclr \: Id HasResultType & (\ref{prod:FieldDeclr}) \\
                     \| Id HasResultType\opt \xcd"=" VariableInitializer \\
%(FROM #(prod:HasResultType)#)
       HasResultType \: ResultType & (\ref{prod:HasResultType}) \\
                     \| \xcd"<:" Type \\
%(FROM #(prod:Mod)#)
                 Mod \: \xcd"abstract" & (\ref{prod:Mod}) \\
                     \| Annotation \\
                     \| \xcd"atomic" \\
                     \| \xcd"final" \\
                     \| \xcd"native" \\
                     \| \xcd"private" \\
                     \| \xcd"protected" \\
                     \| \xcd"public" \\
                     \| \xcd"static" \\
                     \| \xcd"transient" \\
                     \| \xcd"clocked" \\
\end{bbgrammar}
%##)

Like most other kinds of variables in X10, 
the fields of an object can be either \xcd`val` or \xcd`var`. 
\xcd`val` fields can be \xcd`static` (\Sref{FieldDefinitions}).
Field declarations may have optional
initializer expressions, as for local variables, \Sref{local-variables}.
\xcd`var` fields without an initializer are initialized with the default value
of their type. \xcd`val` fields without an initializer must be initialized by
each constructor.


For \xcd`val` fields, as for \xcd`val` local variables, the type may be
omitted and inferred from the initializer type (\Sref{TypeInference}).
\xcd`var` fields, like \xcd`var` local variables, must be declared with a type.



%%GRAM%% \begin{grammar}
%%GRAM%% FieldDeclaration
%%GRAM%%         \: FieldModifier\star \xcd"var" FieldDeclaratorsWithType \\&& ( \xcd"," FieldDeclaratorsWithType )\star \\
%%GRAM%%         \| FieldModifier\star \xcd"val" FieldDeclarators \\&& ( \xcd"," FieldDeclarators )\star \\
%%GRAM%%         \| FieldModifier\star FieldDeclaratorsWithType \\&& ( \xcd"," FieldDeclaratorsWithType )\star \\
%%GRAM%% FieldDeclarators
%%GRAM%%         \: FieldDeclaratorsWithType \\
%%GRAM%%         \: FieldDeclaratorWithInit \\
%%GRAM%% FieldDeclaratorId
%%GRAM%%         \: Identifier  \\
%%GRAM%% FieldDeclaratorWithInit
%%GRAM%%         \: FieldDeclaratorId Init \\
%%GRAM%%         \| FieldDeclaratorId ResultType Init \\
%%GRAM%% FieldDeclaratorsWithType
%%GRAM%%         \: FieldDeclaratorId ( \xcd"," FieldDeclaratorId )\star ResultType \\
%%GRAM%% FieldModifier \: Annotation \\
%%GRAM%%                 \| \xcd"static" \\ \| \xcd`property` \\ \| \xcd`global` \\
%%GRAM%% \end{grammar}
%%GRAM%% 
%%GRAM%% 

%%ACC%%  \section{Accumulator Variables}
%%ACC%%  
%%ACC%%  Accumulator variables allow the accumulation of partial results to produce a
%%ACC%%  final result.  For example, an accumulator variable could compute a running
%%ACC%%  sum, product, maximum, or minimum of a collection of numbers.  In particular,
%%ACC%%  many concurrent activites can accumulate safely into the {\em same} local
%%ACC%%  variable, without need for \Xcd{atomic} blocks or other explicit coordination.  
%%ACC%%  
%%ACC%%  An accumulator variable is associated with a {\em reducer}, which explains how
%%ACC%%  new partial values are accumulated.
%%ACC%%  
%%ACC%%  \subsection{Reducers}
%%ACC%%  
%%ACC%%  A notion of accumulation has two aspects: 
%%ACC%%  \begin{enumerate}
%%ACC%%  \item A {\bf zero} value, which is the initial value of the accumulator,
%%ACC%%        before any partial results have been included.  When accumulating a sum,
%%ACC%%        the zero value is \Xcd{0}; when accumulating a product, it is \Xcd{1}.
%%ACC%%  \item A {\bf combining function}, explaining how to combine two partial
%%ACC%%        accumulations into a whole one.  When accumulating a sum, partial sums
%%ACC%%        should be added together; for a product, they should be multiplied.  
%%ACC%%  \end{enumerate}
%%ACC%%  
%%ACC%%  In X10, this is represented as a value of type
%%ACC%%  \Xcd{x10.lang.Reducer[T]}: 
%%ACC%%  %~acc~gen
%%ACC%%  %package Vars.Notx10lang.Reducerererer;
%%ACC%%  %~acc~vis
%%ACC%%  \begin{xten}
%%ACC%%  struct Reducer[T](zero:T, apply: (T,T)=>T){}
%%ACC%%  \end{xten}
%%ACC%%  %~acc~siv
%%ACC%%  %
%%ACC%%  %~acc~neg
%%ACC%%  \noindent 
%%ACC%%  If \Xcd{r:Reducer[T]}, then \Xcd{r.zero} is the zero element, and
%%ACC%%  \Xcd{r(a,b)} --- which can also be written \Xcd{r.apply(a,b)} --- is the
%%ACC%%  combination of \Xcd{a} and \Xcd{b}.
%%ACC%%  
%%ACC%%  For example, the reducers for adding and multiplying integers are: 
%%ACC%%  %~acc~gen
%%ACC%%  %package Vars.Notx10lang.Reducererererererer;
%%ACC%%  %struct Reducer[T](zero:T, apply: (T,T)=>T){}
%%ACC%%  %class Example{
%%ACC%%  %~acc~vis
%%ACC%%  \begin{xten}
%%ACC%%  val summer = Reducer[Int](0, Int.+);
%%ACC%%  val producter = Reducer[Int](1, Int.*);
%%ACC%%  \end{xten}
%%ACC%%  %~acc~siv
%%ACC%%  %}
%%ACC%%  %~acc~neg
%%ACC%%  
%%ACC%%  
%%ACC%%  Reduction is guaranteed to be deterministic if the reducer is {\em
%%ACC%%  Abelian},\footnote{This term is borrowed from abstract algebra, where such a
%%ACC%%  reducer, together with its type, forms an Abelian monoid.}
%%ACC%%  that is, 
%%ACC%%  \begin{enumerate}
%%ACC%%  \item \Xcd{r.apply} is pure; that is, has no side effects;
%%ACC%%  \item \Xcd{r.apply} is commutative; that is, \Xcd{r(a,b) == r(b,a)} for all
%%ACC%%        inputs \Xcd{a} and \Xcd{b};
%%ACC%%  \item \Xcd{r.apply} is associative; that is, 
%%ACC%%        \Xcd{r(a,r(b,c)) == r(r(a,b),c)} for all \Xcd{a}, \Xcd{b}, and \Xcd{c}.
%%ACC%%  \item \Xcd{r.zero} is the identity element for \Xcd{r.apply}; that is, 
%%ACC%%        \Xcd{r(a, r.zero) == a}
%%ACC%%        for all \Xcd{a}.
%%ACC%%  \end{enumerate}
%%ACC%%  
%%ACC%%  
%%ACC%%  
%%ACC%%  
%%ACC%%  \Xcd{summer} and \Xcd{producter} satisfy all these conditions, and give
%%ACC%%  determinate reductions. The compiler does not require or check these, though.
%%ACC%%  
%%ACC%%  
%%ACC%%  \subsection{Accumulators}
%%ACC%%  
%%ACC%%  If \Xcd{r} is a  value of type \Xcd{Reducer[T]}, then an accumulator of type
%%ACC%%  \Xcd{T} using \Xcd{r} is declared as:
%%ACC%%  %~accTODO~gen
%%ACC%%  % package Vars.Accumulators.Basic.Little.Idea;
%%ACC%%  % class C[T]{
%%ACC%%  % static def example (r:Reducer[T]) {
%%ACC%%  %~accTODO~vis
%%ACC%%  \begin{xten}
%%ACC%%  acc(r) x : T;
%%ACC%%  acc(r) y; 
%%ACC%%  \end{xten}
%%ACC%%  %~accTODO~siv
%%ACC%%  %
%%ACC%%  %~accTODO~neg
%%ACC%%  The type declaration \Xcd{T} is optional; if specified, it must be the same
%%ACC%%  type that the reducer \Xcd{r} uses.
%%ACC%%  
%%ACC%%  \subsection{Sequential Use of Accumulators}
%%ACC%%  
%%ACC%%  The sequential use of accumulator variables is straightforward, and could be
%%ACC%%  done as easily without accumulators.  (The power of accumulators is in their
%%ACC%%  concurrent use, \Sref{ConcurrentUseOfAccumulators}.)
%%ACC%%  
%%ACC%%  A variable declared as \Xcd{acc(r) x:T;} is initialized to \Xcd{r.zero}.  
%%ACC%%  
%%ACC%%  Assignment of values of \Xcd{acc} variables has nonstandard semantics.
%%ACC%%  \Xcd{x = v;} causes the value \Xcd{r(v,x)} to be stored in \Xcd{x} --- in
%%ACC%%  particular, {\em not} the value of \Xcd{v}.
%%ACC%%  
%%ACC%%  Reading a value from an accumulator retrieves the current accumulation.
%%ACC%%  
%%ACC%%  For example, the sum and product of a list \Xcd{L} of integers can be computed
%%ACC%%  by: 
%%ACC%%  %~accTODO~gen
%%ACC%%  %package Vars.Accumulators.Are.For.Bisimulators;
%%ACC%%  % import java.util.*;
%%ACC%%  % class Example{
%%ACC%%  % static def example(L: List[Int]) {
%%ACC%%  %~accTODO~vis
%%ACC%%  \begin{xten}
%%ACC%%  val summer = Reducer[Int](0, Int.+);
%%ACC%%  val producter = Reducer[Int](1, Int.*);
%%ACC%%  acc(summer) sum;
%%ACC%%  acc(producter) prod;
%%ACC%%  for (x in L) {
%%ACC%%    sum = x;
%%ACC%%    prod = x;
%%ACC%%  }
%%ACC%%  x10.io.Console.OUT.println("Sum = " + sum + "; Product = " + prod);
%%ACC%%  \end{xten}
%%ACC%%  %~accTODO~siv
%%ACC%%  %
%%ACC%%  %~accTODO~neg
%%ACC%%  
%%ACC%%  
%%ACC%%  
%%ACC%%  \subsection{Concurrent Use of Accumulators}
%%ACC%%  \label{ConcurrentUseOfAccumulators}
%%ACC%%  \index{accumulator!and activities}
%%ACC%%  
%%ACC%%  Accumulator variables are restricted and synchronized in ways that make them
%%ACC%%  ideally suited for concurrent accumulation of data.   The {\em governing
%%ACC%%  activity} of an accumulator is the activity in which the \Xcd{acc} variable is
%%ACC%%  declared.  
%%ACC%%  
%%ACC%%  \begin{enumerate}
%%ACC%%  \item The governing activity can read the accumulator at any point that it has
%%ACC%%        no running sub-activities.  
%%ACC%%  \item Any activity that has lexical access to the accumulator can write to it.  
%%ACC%%        All
%%ACC%%        writes are performed atomically, without need for \Xcd{atomic} or other
%%ACC%%        concurrency control.
%%ACC%%  \end{enumerate}
%%ACC%%  
%%ACC%%  If the reducer is Abelian, this guarantees that \Xcd{acc} variables cannot
%%ACC%%  cause race conditions; the result of such a computation is determinate,
%%ACC%%  independent of the scheduling of activities. Read-read conflicts are
%%ACC%%  impossible, as only a single activity, the governing activity, can read the
%%ACC%%  \Xcd{acc} variable. Read-write conflicts are impossible, as reads are only
%%ACC%%  allowed at points where the only activity which can refer to the \Xcd{acc}
%%ACC%%  variable is the governing activity. Two activities may try to write the
%%ACC%%  \Xcd{acc} variable at the same time. The writes are performed atomically, so
%%ACC%%  they behave as if they happened in some (arbitrary) order---and, because the
%%ACC%%  reducer is Abelian, the order of writes doesn't matter.
%%ACC%%  
%%ACC%%  If the reducer is not Abelian---\eg, it is accumulating a string result by
%%ACC%%  concatenating a lot of partial strings together---the result is indeterminate.
%%ACC%%  However, because the accumulator operations are atomic, it will be the result
%%ACC%%  of {\em some} combination of the individual elements by the reduction
%%ACC%%  operation, \eg, the concatenation of the partial strings in {\em some} order.  
%%ACC%%  
%%ACC%%  
%%ACC%%  
%%ACC%%  For example, the following code computes triangle numbers {$\sum_{i=1}^{n}i$}
%%ACC%%  concurrently.\footnote{This program is highly inefficient. Even ignoring the
%%ACC%%    constant-time formula {$\sum_{i=1}^{n}i = \frac{n(n+1)}{2}$}, this program
%%ACC%%    incurs the cost of starting {$n$} activities and coordinating {$n$} accesses
%%ACC%%    to the accumulator. Accumulator variables are of most value in multi-place,
%%ACC%%    multi-core computations.}
%%ACC%%  
%%ACC%%  
%%ACC%%  %~accTODO~gen
%%ACC%%  %package Vars.Accumulator.Concurrency.Example;
%%ACC%%  %class Example{
%%ACC%%  %
%%ACC%%  %~accTODO~vis
%%ACC%%  \begin{xten}
%%ACC%%  def triangle(n:Int) {
%%ACC%%    val summer = Reducer[Int](0, Int.+);
%%ACC%%    acc(summer) sum; 
%%ACC%%    finish {
%%ACC%%      for([i] in 1..n) async {
%%ACC%%        sum = i;  // (A)
%%ACC%%      }
%%ACC%%      // (C)
%%ACC%%    }
%%ACC%%    return sum; // (B)
%%ACC%%  }
%%ACC%%  \end{xten}
%%ACC%%  %~accTODO~siv
%%ACC%%  %}
%%ACC%%  %~accTODO~neg
%%ACC%%  
%%ACC%%  The governing activity of the \Xcd{acc} variable \Xcd{sum} is the activity
%%ACC%%  including the body of \Xcd{triangle}.  It starts up \Xcd{n} sub-activities,
%%ACC%%  each of which adds one value to \Xcd{sum} at point \Xcd{(A)}.  Note that these
%%ACC%%  activities cannot {\em read} the value of \Xcd{sum}---only the governing
%%ACC%%  activity can do that---but they can update it.  
%%ACC%%  
%%ACC%%  At point \Xcd{(B)}, \Xcd{triangle} returns the value in \Xcd{sum}. It is
%%ACC%%  clear, from the \Xcd{finish} statement, that all sub-activities started by the
%%ACC%%  governing process have finished at this point. X10 forbids reading of
%%ACC%%  \Xcd{sum}, even by the governing process, at point \Xcd{(C)}, since
%%ACC%%  sub-activities writing into it could still be active when the governing
%%ACC%%  activity reaches this point.  The \Xcd{return sum;} statement could not be
%%ACC%%  moved to \Xcd{(C)}, which is good, because the program would be wrong if it
%%ACC%%  were there.
%%ACC%%  
%%ACC%%  
%%ACC%%  
%%ACC%%  
%%ACC%%  \subsubsection{Accumulators and Places}
%%ACC%%  \index{accumulator!and places} Activity variables can be read and written from
%%ACC%%  any place, without need for \Xcd{GlobalRef}s. We may spread the previous
%%ACC%%  computation out among all the available processors by simply sticking in an
%%ACC%%  \Xcd{at(...)} statement at point \Xcd{(D)}, without need for other
%%ACC%%  restructuring of the program.
%%ACC%%  
%%ACC%%  %~accTODO~gen
%%ACC%%  %package Vars.Accumulator.Concurrency.Example.Multiplacey;
%%ACC%%  %class Example{
%%ACC%%  %~accTODO~vis
%%ACC%%  \begin{xten}
%%ACC%%  def triangle(n:Int) {
%%ACC%%    val summer = Reducer[Int](0, Int.+);
%%ACC%%    acc(summer) sum; 
%%ACC%%    finish {
%%ACC%%      for([i] in 1..n) async 
%%ACC%%        at(Places.place(i % Places.MAX_PLACES) { //(D)
%%ACC%%          sum = i;  // (A)
%%ACC%%      }
%%ACC%%    }
%%ACC%%    return sum; // (B)
%%ACC%%  }
%%ACC%%  \end{xten}
%%ACC%%  %~accTODO~siv
%%ACC%%  %}
%%ACC%%  %~accTODO~neg
%%ACC%%  
%%ACC%%  \subsubsection{Accumulator Parameters}
%%ACC%%  \index{accumulator variables!as parameters}
%%ACC%%  \index{parameters!accumulator}
%%ACC%%  
%%ACC%%  Accumulators can be passed to methods and closures, by giving the keyword 
%%ACC%%  \Xcd{acc} instead of \Xcd{var} or \Xcd{val}.  Reducers are not specified; each
%%ACC%%  accumulator comes with its own reducer.  However, the type \Xcd{T} of the
%%ACC%%  accumulator {\em is} required.
%%ACC%%  
%%ACC%%  For example, the following method takes a list of numbers, and accumulates
%%ACC%%  those that are divisible by 2 in \Xcd{evens}, and those that are divisible by
%%ACC%%  3 in \Xcd{triples}: 
%%ACC%%  %~accTODO~gen
%%ACC%%  %package Vars.accumulators.parameters.oscillators.convulsitors.proximators;
%%ACC%%  %import x10.util.*;
%%ACC%%  %class Whatever {
%%ACC%%  %~accTODO~vis
%%ACC%%  \begin{xten}
%%ACC%%  static def split23(L:List[Int], acc evens:Int, acc triples:Int) {
%%ACC%%    for(n in L) {
%%ACC%%       if (n % 2 == 0) evens = n;
%%ACC%%       if (n % 3 == 0) triples = n;
%%ACC%%    }
%%ACC%%  }
%%ACC%%  static val summer = Reducer[Int](0, Int.+);
%%ACC%%  static val producter = Reducer[Int](1, Int.*);
%%ACC%%  static def sumEvenPlusProdTriple(L:List[Int]) {
%%ACC%%    acc(summer) sumEven;
%%ACC%%    acc(producter) prodTriple;
%%ACC%%    split23(L, sumEven, prodTriple);
%%ACC%%    return sumEven + prodTriple;
%%ACC%%  }
%%ACC%%  \end{xten}
%%ACC%%  %~accTODO~siv
%%ACC%%  %}
%%ACC%%  %~accTODO~neg
%%ACC%%  
%%ACC%%  \subsection{Indexed Accumulators}
%%ACC%%  \index{accumulator!indexed}
%%ACC%%  \index{accumulator!array}
%%ACC%%  
%%ACC%%  
%%ACC%%  \noo{Define this!}
%%ACC%%  
%%ACC%%  %~accTODO~gen
%%ACC%%  % package Vars.Indexed.Accumulators;
%%ACC%%  %~accTODO~vis
%%ACC%%  \begin{xten}
%%ACC%%  class BoolAccum implements SelfAccumulator[Boolean, Int] {
%%ACC%%    var sumTrue = 0, sumFalse = 0;
%%ACC%%    def update(k:Boolean, v:Int) { 
%%ACC%%       if (k) sumTrue += k; else sumFalse += k;
%%ACC%%    }
%%ACC%%    def update(ks:Array[Boolean]{rail}, vs:Array[Int]{ks.size == vs.size}) {
%%ACC%%       for([i] in ks.region) update(ks(i), vs(i));  }
%%ACC%%    
%%ACC%%  }
%%ACC%%  \end{xten}
%%ACC%%  %~accTODO~siv
%%ACC%%  %
%%ACC%%  %~accTODO~neg

\chapter{Names and packages}
\label{packages} \index{names}\index{packages}\index{public}\index{protected}\index{private}

\Xten{} supports mechanisms for names and packages in the style of Java
\cite[\S 6,\S 7]{jls2}, including \xcd"public", \xcd"protected", \xcd"private"
and package-specific access control.

\section{Packages}

A package is a named collection of top-level type declarations, \viz, class,
interface, and struct declarations. Package names are sequences of
identifiers, like \xcd`x10.lang` and \xcd`com.ibm.museum`. The multiple names
are simply a convenience. Packages \xcd`a`, \xcd`a.b`, and \xcd`a.c` have only
a very tenuous relationship, despite the similarity of their names.

Packages and protection modifiers determine which top-level names can be used
where. Only the \xcd`public` members of package \xcd`pack.age` can be accessed
outside of \xcd`pack.age` itself.  
%~~gen~~Stimulus
%
%~~vis
\begin{xten}
package pack.age;
class Deal {
  public def make() {}
}
public class Stimulus {
  private def taxCut() = true;
  protected def benefits() = true;
  public def jobCreation() = true;
  /*package*/ def jumpstart() = true;
}
\end{xten}
%~~siv
%
%~~neg

The class \xcd`Stimulus` can be referred to from anywhere outside of
\xcd`pack.age` by its full name of \xcd`pack.age.Stimulus`, or can be imported
and referred to simply as \xcd`Stimulus`.  The public \xcd`jobCreation()`
method of a \xcd`Stimulus` can be referred to from anywhere as well; the other
methods have smaller visibility.  The non-\xcd`public` class \xcd`Deal` cannot
be used from outside of \xcd`pack.age`.  



\subsection{Name Collisions}

It is a static error for a package to have two members, or apparent members,
with the same name.  For example, package \xcd`pack.age` cannot define two
classes both named \xcd`Crash`, nor a class and an interface with that name.

Furthermore, \xcd`pack.age` cannot define a member \xcd`Crash` if there is
another package named \xcd`pack.age.Crash`, nor vice-versa. (This prohibition
is the only actual relationship between the two packages.)  This prevents the
ambiguity of whether \xcd`pack.age.Crash` refers to the class or the package.  
Note that the naming convention that package names are lower-case and package
members are capitalized prevents such collisions.


\section{\xcd`import` Declarations}

Any public member of a package can be referred to from anywhere through a
fully-qualified name: \xcd`pack.age.Stimulus`.    

Often, this is too awkward.  X10 has two ways to allow code outside of a class
to refer to the class by its short name (\xcd`Stimulus`): single-type imports
and on-demand imports.   

Imports of either kind appear at the start of the file, immediately after the
\xcd`package` directive if there is one; their scope is the whole file.

\subsection{Single-Type Import}

The declaration \xcd`import ` {\em TypeName} \xcd`;` imports a single type
into the current namespace.  The type it imports must be a fully-qualified
name of an extant type, and it must either be in the same package (in which
case the \xcd`import` is redundant) or be declared \xcd`public`.  

Furthermore, when importing \xcd`pack.age.T`, there must not be another type
named \xcd`T` at that point: neither a  \xcd`T` declared in \xcd`pack.age`,
nor a \xcd`inst.ant.T` imported from some other package.

\subsection{Automatic Import}

The automatic import \xcd`import pack.age.*;`, loosely, imports all the public
members of \xcd`pack.age`.  In fact, it does so somewhat carefully, avoiding
certain errors that could occur if it were done naively.  Types defined in the
current package, and those imported by single-type imports, shadow those
imported by automatic imports.  

\subsection{Implicit Imports}

The packages \xcd`x10.lang` and \xcd`x10.array` are imported in all files
without need for further specification.

%%BARD-HERE



\section{Conventions on Type Names}

\begin{grammar}
TypeName   \: Identifier \\
        \| TypeName \xcd"." Identifier \\
        \| PackageName \xcd"." Identifier \\
PackageName   \: Identifier \\
        \| PackageName \xcd"." Identifier \\
\end{grammar}


While not enforced by the compiler, classes and interfaces
in the \Xten{} library follow the following naming conventions.
Names of types---including classes,
type parameters, and types specified by type definitions---are in
CamelCase and begin with an uppercase letter.  (Type variables are often
single capital letters, such as \xcd`T`.)
For backward
compatibility with languages such as C and \java{}, type
definitions are provided to allow primitive types
such as \xcd"int" and \xcd"boolean" to be written in lowercase.
Names of methods, fields, value properties, and packages are in camelCase and
begin with a lowercase letter. 
Names of \xcd"static val" fields are in all uppercase with words
separated by `\xcd"_"''s.



\chapter{Interfaces}
\label{XtenInterfaces}\index{interface}

An interface specifies signatures for zero or more public methods, property
methods,
\xcd`static val`s, 
classes, structs, interfaces, types
and an invariant. 

The following puny example illustrates all these features: 
% TODO Well, it would if there weren't a compiler bug in the way.
%~~gen ^^^Interfaces_static_val
% package Interfaces_static_val;
% 
%~~vis
\begin{xten}
interface Pushable{prio() != 0} {
  def push(): void;
  static val MAX_PRIO = 100;
  abstract class Pushedness{}
  struct Pushy{}
  interface Pushing{}
  static type Shove = Int;
  property text():String;
  property prio():Int;
}
class MessageButton(text:String)
  implements Pushable{self.prio()==Pushable.MAX_PRIO} {
  public def push() { 
    x10.io.Console.OUT.println(text + " pushed");
  }
  public property text() = text;
  public property prio() = Pushable.MAX_PRIO;
}
\end{xten}
%~~siv
%
%~~neg
\noindent
\xcd`Pushable` defines two property methods, one normal method, and a static
value.  It also 
establishes an invariant, that \xcd`prio() != 0`. 
\xcd`MessageButton` implements a constrained version of \xcd`Pushable`,
\viz\ one with maximum priority.  It
defines the \xcd`push()` method given in the interface, as a \xcd`public`
method---interface methods are implicitly \xcd`public`.

\limitation{X10 may not always detect that type invariants of interfaces are
satisfied, even when they obviously are.}
%% TODO - is this a JIRA?  

A container---a class or struct---can {\em implement} an interface,
typically by having all the methods and property methods that the interface
requires, and by providing a suitable \xcd`implements` clause in its definition.

A variable may be declared to be of interface type.  Such a variable has all
the property and normal methods declared (directly or indirectly) by the
interface; 
nothing else is statically available.  Values of any concrete type which
implement the interface may be stored in the variable.  

\begin{ex}
The following code puts two quite different objects into the variable
\xcd`star`, both of which satisfy the interface \xcd`Star`.
%~~gen ^^^ Interfaces6l3f
% package Interfaces6l3f;
%~~vis
\begin{xten}
interface Star { def rise():void; }
class AlphaCentauri implements Star {
   public def rise() {}
}
class ElvisPresley implements Star {
   public def rise() {}
}
class Example {
   static def example() {
      var star : Star;
      star = new AlphaCentauri();
      star.rise();
      star = new ElvisPresley();
      star.rise();
   }
}
\end{xten}
%~~siv
%
%~~neg
\end{ex}
An interface may extend several interfaces, giving
X10 a large fraction of the power of multiple inheritance at a tiny fraction
of the cost.

\begin{ex}
%~~gen ^^^ Interfaces6g4u
% package Interfaces6g4u;
%~~vis
\begin{xten}
interface Star{}
interface Dog{}
class Sirius implements Dog, Star{}
class Lassie implements Dog, Star{}
\end{xten}
%~~siv
%
%~~neg
\end{ex}


\section{Interface Syntax}

\label{DepType:Interface}

%##(NormalInterfaceDecl TypeParamsI TypeParamI Guard ExtendsInterfaces InterfaceBody InterfaceMemberDecl
\begin{bbgrammar}
%(FROM #(prod:NormalInterfaceDecl)#)
 NormalInterfaceDecl \: Mods\opt \xcd"interface" Id TypeParamsI\opt Properties\opt Guard\opt ExtendsInterfaces\opt InterfaceBody & (\ref{prod:NormalInterfaceDecl}) \\
%(FROM #(prod:TypeParamsI)#)
         TypeParamsI \: \xcd"[" TypeParamIList \xcd"]" & (\ref{prod:TypeParamsI}) \\
%(FROM #(prod:TypeParamI)#)
          TypeParamI \: Id & (\ref{prod:TypeParamI}) \\
                     \| \xcd"+" Id \\
                     \| \xcd"-" Id \\
%(FROM #(prod:Guard)#)
               Guard \: DepParams & (\ref{prod:Guard}) \\
%(FROM #(prod:ExtendsInterfaces)#)
   ExtendsInterfaces \: \xcd"extends" Type & (\ref{prod:ExtendsInterfaces}) \\
                     \| ExtendsInterfaces \xcd"," Type \\
%(FROM #(prod:InterfaceBody)#)
       InterfaceBody \: \xcd"{" InterfaceMemberDecls\opt \xcd"}" & (\ref{prod:InterfaceBody}) \\
%(FROM #(prod:InterfaceMemberDecl)#)
 InterfaceMemberDecl \: MethodDecl & (\ref{prod:InterfaceMemberDecl}) \\
                     \| PropertyMethodDecl \\
                     \| FieldDecl \\
                     \| ClassDecl \\
                     \| InterfaceDecl \\
                     \| TypeDefDecl \\
                     \| \xcd";" \\
\end{bbgrammar}
%##)


\noindent
The invariant associated with an interface is the conjunction of the
invariants associated with its superinterfaces and the invariant
defined at the interface. 



A class \xcd"C"  implements an interface \xcd"I" if \xcd`I`, or a subtype of \xcd`I`, appears in the \xcd`implements` list
of \xcd`C`.  
In this case,
 \xcd`C` implicitly gets all the methods and property methods of \xcd`I`,
      as \xcd`abstract` \xcd`public` methods.  If \xcd`C` does not declare
      them explicitly, then they are \xcd`abstract`, and \xcd`C` must be
      \xcd`abstract` as well.   If \xcd`C` does declare them all, \xcd`C` may
      be concrete.



If \xcd`C` implements \xcd`I`, then the class invariant
(\Sref{DepType:ClassGuardDef}) for \xcd`C`,   $\mathit{inv}($\xcd"C"$)$, implies
the class invariant for \xcd`I`, $\mathit{inv}($\xcd"I"$)$.  That is, if the
interface \xcd`I` specifies some requirement, then every class \xcd`C` that
implements it satisfies that requirement.

\section{Access to Members}

All interface members are \xcd`public`, whether or not they are declared
public.  There is little purpose to non-public methods of an interface; they
would specify that implementing classes and structs have methods that cannot
be seen.

\section{Property Methods}

An interface may declare \xcd`property` methods.  All non-\xcd`abstract`
containers implementing such an interface must provide all the \xcd`property`
methods specified.  

\section{Field Definitions}
\index{interface!field definition in}

An interface may declare a \xcd`val` field, with a value.  This field is implicitly
\xcd`public static val`.  In particular, it is {\em not} an instance field. 
%~~gen ^^^ Interfaces10
% package Interface.Field;
%~~vis
\begin{xten}
interface KnowsPi {
  PI = 3.14159265358;
}
\end{xten}
%~~siv
%
%~~neg

Classes and structs implementing such an interface get the interface's fields as
\xcd`public static` fields.  Unlike  methods, there is no need
for the implementing class to declare them. 
%~~gen ^^^ Interfaces20
% package Interface.Field.Two;
% interface KnowsPi {PI = 3.14159265358;}
%~~vis
\begin{xten}
class Circle implements KnowsPi {
  static def area(r:Double) = PI * r * r;
}
class UsesPi {
  def circumf(r:Double) = 2 * r * KnowsPi.PI;
}
\end{xten}
%~~siv
%
%~~neg

\subsection{Fine Points of Fields}

If two parent interfaces give different static fields of the same name, 
those fields must be referred to by qualified names.
%~~gen ^^^ Interface_field_name_collision
% 
%~~vis
\begin{xten}
interface E1 {static val a = 1;}
interface E2 {static val a = 2;}
interface E3 extends E1, E2{}
class Example implements E3 {
  def example() = E1.a + E2.a;
}
\end{xten}
%~~siv
%
%~~neg

If the {\em same} field \xcd`a` is inherited through many paths, there is no need to
disambiguate it:
%~~gen ^^^ Interfaces_multi
% package Interfaces.Mult.Inher.Field;
%~~vis
\begin{xten}
interface I1 { static val a = 1;} 
interface I2 extends I1 {}
interface I3 extends I1 {}
interface I4 extends I2,I3 {}
class Example implements I4 {
  def example() = a;
}
\end{xten}
%~~siv
%
%~~neg

The initializer of a field in an interface may be any expression.  It is
evaluated under the same rules as a \xcd`static` field of a class. 

\begin{ex}
In this example, a class \xcd`TheOne` is defined,
with an inner interface \xcd`WelshOrFrench`, whose field \xcd`UN` (named in
either Welsh or French) has value 1.  Note that \xcd`WelshOrFrench` does not
define any methods, so it can be trivially added to the \xcd`implements`
clause of any class, as it is for \xcd`Onesome`. 
This allows the body of \xcd`Onesome` to use \xcd`UN` through an unqualified
name, as is done in \xcd`example()`.

%~~gen ^^^ Interfaces3l4a
% package Interfaces3l4a;
%~~vis
\begin{xten}
class TheOne {
  static val ONE = 1;
  interface WelshOrFrench {
    val UN = 1;
  }
  static class Onesome implements WelshOrFrench {
    static def example() {
      assert UN == ONE;
    }
  }
}
\end{xten}
%~~siv
% class Hook{ def run() {TheOne.Onesome.example(); return true;}}
%~~neg
\end{ex}

\section{Generic Interfaces}

Interfaces, like classes and structs, can have type parameters.  
The discussion of generics in \Sref{TypeParameters} applies to interfaces,
without modification.

\begin{ex}
%~~gen ^^^ Interfaces7n1z
% package Interfaces7n1z;
%~~vis
\begin{xten}
interface ListOfFuns[T,U] extends x10.util.List[(T)=>U] {}
\end{xten}
%~~siv
%
%~~neg

\end{ex}

\section{Interface Inheritance}

The {\em direct superinterfaces} of a non-generic interface \xcd`I` are the interfaces
(if any) mentioned in the \xcd`extends` clause of \xcd`I`'s definition.
If \xcd`I`  is generic, the direct superinterfaces are of an instantiation of
\xcd`I` are the corresponding instantiations of those interfaces.
A {\em superinterface} of \xcd`I` is either \xcd`I` itself, or a direct
superinterface of a superinterface of \xcd`I`, and similarly for generic
interfaces.    

\xcd`I` inherits the members of all of its superinterfaces. Any class or
struct that has \xcd`I` in its \xcd`implements` clause also implements all of
\xcd`I`'s superinterfaces. 






\section{Members of an Interface}

The members of an interface \xcd`I` are the union of the following sets: 
\begin{enumerate}
\item All of the members appearing in \xcd`I`'s declaration;
\item All the members of its direct super-interfaces, except those which are
      hidden (\Sref{sect:Hiding}) by \xcd`I`
\item The members of \xcd`Any`.
\end{enumerate}

Overriding for instances is defined as for classes, \Sref{MethodOverload}



\chapter{Classes}
\label{XtenClasses}\index{class}
\label{ReferenceClasses}





\section{Principles of X10 Objects}\label{XtenObjects}\index{object}
\index{class}

\subsection{Basic Design}

Objects are instances of classes: the most common and most powerful sort of
value in X10.  The other kinds of values, structs and functions, are more
specialized, better in some circumstances but not in all.

Classes are structured in a forest of single-inheritance code
hierarchies. Like C++, but unlike Java, there is no single root
class (\Xcd{java.lang.Object}) that all classes inherit from.  Classes
may have any or all of these features: 
\begin{itemize}
\item Implementing any number of interfaces;
\item Static and instance \xcd`val` fields; 
\item Instance \xcd`var` fields; 
\item Static and instance methods;
\item Constructors;
\item Properties;
\item Static and instance nested containers.
\item Static type definitions
\end{itemize}


\Xten{} objects (unlike Java objects) do not have locks associated with them.
Programmers may use atomic blocks (\Sref{AtomicBlocks}) for mutual
exclusion and clocks (\Sref{XtenClocks}) for sequencing multiple parallel
operations.

An object exists in a single location: the place that it was created.  One
place cannot use or even directly refer to an object in a different place.   A
special type, \Xcd{GlobalRef[T]}, allows explicit cross-place references. 

The basic operations on objects are:
\begin{itemize}

\item Construction (\Sref{ObjectInitialization}).  Objects are created, 
      their \xcd`var` and \xcd`val` fields initialized, and other invariants
      established.

\item Field access (\Sref{FieldAccess}). 
The static, instance, and property fields of an object can be retrieved; \xcd`var` fields
can be set.  

\item Method invocation (\Sref{MethodInvocation}).  
Static, instance, and property methods of an object can be invoked.

\item Casting (\Sref{ClassCast}) and instance testing with \xcd`instanceof`
(\Sref{instanceOf}) Objects can be cast or type-tested.  

\item The equality operators \xcd"==" and \xcd"!=".  
Objects can be compared for equality with the \Xcd{==} operation.  This checks
object {\em identity}: two objects are \Xcd{==} iff they are the same object.

\end{itemize}

  

\subsection{Class Declaration Syntax}
\label{sect:ClassDeclSyntax}

The {\em class declaration} has a list of type parameters, a list of
properties, a constraint (the {\em class invariant}), zero or one superclass,
zero or more interfaces that it implements, and a class body containing the
the definition of fields, properties, methods, and member types. Each such
declaration introduces a class type (\Sref{ReferenceTypes}).

%##(ClassDecln TypeParamsI TypeParamIList Properties PropertyList Property Guard Super Interfaces InterfaceTypeList ClassBody ClassMemberDeclns ClassMemberDecln
\begin{bbgrammar}
%(FROM #(prod:ClassDecln)#)
          ClassDecln \: Mods\opt \xcd"class" Id TypeParamsI\opt Properties\opt Guard\opt Super\opt Interfaces\opt ClassBody & (\ref{prod:ClassDecln}) \\
%(FROM #(prod:TypeParamsI)#)
         TypeParamsI \: \xcd"[" TypeParamIList \xcd"]" & (\ref{prod:TypeParamsI}) \\
%(FROM #(prod:TypeParamIList)#)
      TypeParamIList \: TypeParam & (\ref{prod:TypeParamIList}) \\
                     \| TypeParamIList \xcd"," TypeParam \\
                     \| TypeParamIList \xcd"," \\
%(FROM #(prod:Properties)#)
          Properties \: \xcd"(" PropList \xcd")" & (\ref{prod:Properties}) \\
%(FROM #(prod:PropList)#)
            PropList \: Prop & (\ref{prod:PropList}) \\
                     \| PropList \xcd"," Prop \\
%(FROM #(prod:Prop)#)
                Prop \: Annotations\opt Id ResultType & (\ref{prod:Prop}) \\
%(FROM #(prod:Guard)#)
               Guard \: DepParams & (\ref{prod:Guard}) \\
%(FROM #(prod:Super)#)
               Super \: \xcd"extends" ClassType & (\ref{prod:Super}) \\
%(FROM #(prod:Interfaces)#)
          Interfaces \: \xcd"implements" InterfaceTypeList & (\ref{prod:Interfaces}) \\
%(FROM #(prod:InterfaceTypeList)#)
   InterfaceTypeList \: Type & (\ref{prod:InterfaceTypeList}) \\
                     \| InterfaceTypeList \xcd"," Type \\
%(FROM #(prod:ClassBody)#)
           ClassBody \: \xcd"{" ClassMemberDeclns\opt \xcd"}" & (\ref{prod:ClassBody}) \\
%(FROM #(prod:ClassMemberDeclns)#)
   ClassMemberDeclns \: ClassMemberDecln & (\ref{prod:ClassMemberDeclns}) \\
                     \| ClassMemberDeclns ClassMemberDecln \\
%(FROM #(prod:ClassMemberDecln)#)
    ClassMemberDecln \: InterfaceMemberDecln & (\ref{prod:ClassMemberDecln}) \\
                     \| CtorDecln \\
\end{bbgrammar}
%##)




\section{Fields}
\label{FieldDefinitions}
\index{object!field}
\index{field}

Objects may have {\em instance fields}, or simply {\em fields} (called
``instance variables'' in C++ and Smalltalk, and ``slots'' in CLOS): places to
store data that is pertinent to the object.  Fields, like variables, may be
mutable (\xcd`var`) or immutable (\xcd`val`).  

A class may have {\em static fields}, which store data pertinent to the
entire class of objects.  See \Sref{StaticInitialization} for more
information. 
Because of its emphasis on safe concurrency, \Xten{} requires static
fields to be immutable (\xcd`val`). 

No two fields of the same class may have the same name.  A field may have the
same name as a method, although for fields of functional type there is a
subtlety (\Sref{sect:disambiguations}).  

\subsection{Field Initialization}
\index{field!initialization}
\index{initialization!of field}

Fields may be given values via {\em field initialization expressions}:
\xcd`val f1 = E;` and \xcd`var f2 : Int = F;`. Other fields of \xcd`this` may
be referenced, but only those that {\em precede} the field being initialized.


\begin{ex}The following is correct, but would not be if the fields were
reversed:

%~~gen ^^^ Classes10
%package Classes_field_init_expr_a;
%~~vis
\begin{xten}
class Fld{
  val a = 1;
  val b = 2+a;
}
\end{xten}
%~~siv
% class Hook{ def run() {
%   val f = new Fld();
%   assert f.a == 1 && f.b == 3;
%   return true;}}
%~~neg
\end{ex}

\subsection{Field hiding}
\label{sect:FieldHiding}
\index{field!hiding}


A subclass that defines a field \xcd"f" hides any field \xcd"f"
declared in a superclass, regardless of their types.  The
superclass field \xcd"f" may be accessed within the body of
the subclass via the reference \xcd"super.f".

With inner classes, it is occasionally necessary to 
write \xcd`Cls.super.f` to get at a hidden field \xcd`f` of an outer class
\xcd`Cls`. 

\begin{ex}
The \xcd`f` field in \xcd`Sub` hides the \xcd`f` field in \xcd`Super`
The \xcd`superf` method provides access to the \xcd`f` field in \xcd`Super`.
%~~gen ^^^ Classes20
% package classes.fields.primus;
%~~vis
\begin{xten}
class Super{ 
  public val f = 1; 
}
class Sub extends Super {
  val f = true;
  def superf() : Int = super.f; // 1
}
\end{xten}
%~~siv
% class Hook { def run() { 
%   val sub = new Sub();
%   assert sub.f == true;
%   assert sub.superf() == 1;
%   return true;} }
%~~neg
\end{ex}

\begin{ex}
Hidden fields of outer classes can be accessed by suitable forms: 
%~~gen ^^^ Classes30
% package classes.fields.secundus; 
% // NOTEST
%~~vis
\begin{xten}
class A {
   val f = 3;
}
class B extends A {
   val f = 4;
   class C extends B {
      // C is both a subclass and inner class of B
      val f = 5;
       def example() {
         assert f         == 5 : "field of C";
         assert super.f   == 4 : "field of superclass";
         assert B.this.f  == 4 : "field of outer instance";
         assert B.super.f == 3 : "super.f of outer instance";
       }
    }
}
\end{xten}
%~~siv
% class Hook { def run() { ((new B()).new C()).example(); return true; } }
%~~neg
\end{ex}

\subsection{Field qualifiers}
\label{FieldQualifier}
\index{qualifier!field}
\index{field!qualifier}

The behavior of a field may be changed by a field qualifier, such as
\xcd`static` or \xcd`transient`.  


\subsubsection{\Xcd{static} qualifier}
\index{field!static}

A \xcd`val` field may be declared to be {\em static}, as described in
\Sref{FieldDefinitions}. 

\subsubsection{\Xcd{transient} Qualifier}
\label{TransientFields}
\index{transient}
\index{field!transient}

A field may be declared to be {\em transient}.  Transient fields are excluded
from the deep copying that happens when information is sent from place to
place in an \Xcd{at} statement.    The value of a transient field of a copied
object is the default value of its type, regardless of the value of the field
in the original.  If the type of a field has no
default value, it cannot be marked \Xcd{transient}.

%%AT-COPY%% %~~gen ^^^ Classes40
%%AT-COPY%% % package Classes.Transient.Example;
%%AT-COPY%% % KNOWNFAIL-at
%%AT-COPY%% %~~vis
%%AT-COPY%% \begin{xten}
%%AT-COPY%% class Trans { 
%%AT-COPY%%    val copied = "copied";
%%AT-COPY%%    transient var transy : String = "a very long string";
%%AT-COPY%%    def example() {
%%AT-COPY%%       at (here; this) { // causes copying of 'this'
%%AT-COPY%%          assert(this.copied.equals("copied"));
%%AT-COPY%%          assert(this.transy == null);
%%AT-COPY%%       }
%%AT-COPY%%    }
%%AT-COPY%% }
%%AT-COPY%% \end{xten}
%%AT-COPY%% %~~siv
%%AT-COPY%% % class Hook{ def run() {(new Example()).example(); return true;}}
%%AT-COPY%% %~~neg
%%AT-COPY%% 

%~~gen ^^^ Classes40
% package Classes.Transient.Example;
% 
%~~vis
\begin{xten}
class Trans { 
   val copied = "copied";
   transient var transy : String = "a very long string";
   def example() {
      at (here) { // causes copying of 'this'
         assert(this.copied.equals("copied"));
         assert(this.transy == null);
      }
   }
}
\end{xten}
%~~siv
% class Hook{ def run() {(new Trans()).example(); return true;}}
%~~neg


\section{Properties}
\label{PropertiesInClasses}
\index{property}

The properties of an object (or struct) are a restricted form of public
\xcd`val` fields.\footnote{In many cases, a 
\xcd`val` field can be upgraded to a \xcd`property`, which 
entails no compile-time or runtime cost.  Some cannot be, \eg, in cases where
cyclic structures of \xcd`val` fields are required.} 
For example,  every array has a \xcd`rank` telling
how many subscripts it takes.  User-defined classes can have whatever
properties are desired. 

Properties differ from public \xcd`val` fields in a few ways: 
\begin{enumerate}
\item Property references are allowed on \xcd`self` in constraints:
      \xcd`self.prop`.  Field references are not.
\item Properties are in scope for all instance initialization expressions.
      \xcd`val` fields are not.
\item The graph of values reachable from a given object by following only
      property links is acyclic.  Conversely, it is possible (and routine) for
      two objects to point to each other with \xcd`val` fields.
\item Properties are declared in the class header; \xcd`val` fields are
      defined in the class body.
\item Properties are set in constructors by a \xcd`property` statement.
      \xcd`val` fields are set by assignment.
\end{enumerate}



Properties are defined in parentheses, after the name of the class.  They are
given values by the \xcd`property` command in constructors.

\begin{ex}
\xcd`Proper` has a single property, \xcd`t`.  \xcd`new Proper(4)` creates a
\xcd`Proper` object with \xcd`t==4`. 
%~~gen ^^^ Classes50
% package Classes.Toss.Freedom.Disk2;
%~~vis
\begin{xten}
class Proper(t:Int) {
  def this(t:Int) {property(t);}
}
\end{xten}
%~~siv
% class Hook{ def run() {
%   val p = new Proper(4);
%   return p.t == 4;
% } } 
%~~neg

\end{ex}


It is a static error for a class
defining a property \xcd"x: T" to have a subclass class that defines
a property or a field with the name \xcd"x".


A property \xcd`x:T` induces a field with the same name and type, 
as if defined with: 
%~~gen ^^^ Classes60
% package Classes.For.Masses.Of.NevermindTheRest;
% class Exampll[T] {
%~~vis
\begin{xten}
public val x : T;
\end{xten} 
%~~siv
% def this(y:T) { x=y; }
% }
%~~neg

\index{property!initialization}
Properties are initialized in a constructor by the invocation of a special \Xcd{property}
statement. The requirement to use the \xcd`property` statement means that all properties
must be given values at the same time: a container either has its properties
or it does not.
\begin{xten}
property(e1,..., en);
\end{xten}
The number and types of arguments to the \Xcd{property} statement must match
the number and types of the properties in the class declaration, in order.  
Every constructor of a class with properties must invoke \xcd`property(...)`
precisely once; it is a static error if X10 cannot prove that this holds.



By construction, the graph whose nodes are values and whose edges are
properties is acyclic.  \Eg, there cannot be values \xcd`a` and \xcd`b` with
properties \xcd`c` and \xcd`d` such that \xcd`a.c == b` and \xcd`b.d == a`.

\begin{ex}
%~~gen ^^^ Classes7h2f
% package Classes7h2f;
%~~vis
\begin{xten}
class Proper(a:Int, b:String) {
  def this(a:Int, b:String) {
      property(a, b);
  }
  def this(z:Int) {
      val theA = z+5;
      val theB = "X"+z;
      property(theA, theB);
  }
  static def example() {
      val p = new Proper(1, "one");
      assert p.a == 1 && p.b.equals("one");
      val q = new Proper(10);
      assert q.a == 15 && q.b.equals("X10");
  }
}
\end{xten}
%~~siv
% class Hook{ def run() {Proper.example(); return true;}}
%~~neg
\end{ex}

\subsection{Properties and Field Initialization}

Fields with explicit initializers are evaluated immediately after the
\xcd`property` command, and all properties are in scope when initializers are
evaluated.  

\begin{ex}
Class \xcd`Init` initializes the field \xcd`a` to be an array of \xcd`n`
elements, where \xcd`n` is a property.    
When \xcd`new Init(4)` is executed, the constructor first sets \xcd`n` to
\xcd`4` via the \xcd`property` statement, and then initializes \xcd`a` to a
4-element array.

However, \xcd`Outit` uses a field rather than a property for \xcd`n`.  
If the \xcd`ERROR` line were present, it would not compile.  \xcd`n` has not
been definitely assigned (\Sref{sect:DefiniteAssignment}) at this point, and
\xcd`n` has not been given its value, so \xcd`a` cannot be computed.  
(If one insisted that \xcd`n` be a property, \xcd`a` would have to be
initialized in the constructor, rather than by an initialization expression.)
%~~gen ^^^ Classes9c9r
% package Classes9c9r;
%~~vis
\begin{xten}
class Init(n:Int) {
  val a = new Rail[String](n, "");
  def this(n:Int) { property(n); }
}
class Outit {
  val n : Int;
  //ERROR: val a = new Rail[String](n, "");
  def this(m:Int) { this.n = m; }
}
\end{xten}
%~~siv
%
%~~neg


\end{ex}

\subsection{Properties and Fields}

A container with a property named \xcd`p`, or a nullary property method named
\xcd`p()`, cannot have a field named \xcd`p` --- either defined in that
container, or inherited from a superclass.

\subsection{Acyclicity of Properties}
\index{properties!acyclic}

X10 has certain restrictions that, ultimately, require that properties are
simpler than their containers.  For example, \xcd`class A(a:A){}` is not
allowed.  
Formally, this requirement is that there is  a total order $\preceq$ 
on all classes and
structs such that, if $A$ extends $B$, then $A \prec B$, and
if $A$ has a property of type $B$, then $A \prec B$, where $A \prec B$ means
$A \preceq B$ and $A \ne B$.   
For example, the preceding class \xcd`A` is ruled out because we would need
\xcd`A`$\prec$\xcd`A`, which violates the definition of $\prec$.
The programmer need not (and cannot) specify
$\preceq$, and rarely need worry about its existence.  

Similarly, 
the type of a property may not simply be a type parameter.  
For example, \xcd`class A[X](x:X){}` is illegal.





\section{Methods}
\label{sect:Methods}
\index{method}
\index{signature}
\index{method!signature}
\index{method!instance}
\index{method!static}

As is common in object-oriented languages, objects can have {\em methods}, of
two sorts.  {\em Static methods} are functions, conceptually associated with a
class and defined in its namespace.  {\em Instance methods} are parameterized
code bodies associated with an instance of the class, which execute with
convenient access to that instance's fields. 

Each method has a {\em signature}, telling what arguments it accepts, what
type it returns, and what precondition it requires. Method definitions may be
overridden by subclasses; the overriding definition may have a declared return
type that is a subtype of the return type of the definition being overridden.
Multiple methods with the same name but different signatures may be provided
\index{overloading}
\index{polymorphism}
on a class (called ``overloading'' or ``ad hoc polymorphism''). Methods may be
declared \Xcd{public}, \Xcd{private}, \Xcd{protected}, or given default package-level access
rights.

%##(MethMods MethodDeclaration TypeParams Formals FormalList HasResultType MethodBody BinOpDecln PrefixOpDecln ApplyOpDecln ConversionOpDecln
\begin{bbgrammar}
%(FROM #(prod:MethMods)#)
            MethMods \: Mods\opt & (\ref{prod:MethMods}) \\
                     \| MethMods \xcd"property"  \\
                     \| MethMods Mod \\
%(FROM #(prod:MethodDecln)#)
         MethodDecln \: MethMods \xcd"def" Id TypeParams\opt Formals Guard\opt Throws\opt HasResultType\opt MethodBody & (\ref{prod:MethodDecln}) \\
                     \| BinOpDecln \\
                     \| PrefixOpDecln \\
                     \| ApplyOpDecln \\
                     \| SetOpDecln \\
                     \| ConversionOpDecln \\
%(FROM #(prod:TypeParams)#)
          TypeParams \: \xcd"[" TypeParamList \xcd"]" & (\ref{prod:TypeParams}) \\
%(FROM #(prod:Formals)#)
             Formals \: \xcd"(" FormalList\opt \xcd")" & (\ref{prod:Formals}) \\
%(FROM #(prod:FormalList)#)
          FormalList \: Formal & (\ref{prod:FormalList}) \\
                     \| FormalList \xcd"," Formal \\
%(FROM #(prod:Throws)#)
             Throws \: \xcd"throws" ThrowList & (\ref{prod:Throws}) \\
%(FROM #(prod:ThrowsList)#)
          ThrowsList \: Type & (\ref{prod:ThrowsList}) \\
                     \| ThrowsList \xcd"," Type \\
%(FROM #(prod:HasResultType)#)
       HasResultType \: ResultType & (\ref{prod:HasResultType}) \\
                     \| \xcd"<:" Type \\
%(FROM #(prod:MethodBody)#)
          MethodBody \: \xcd"=" LastExp \xcd";" & (\ref{prod:MethodBody}) \\
                     \| \xcd"=" Annotations\opt \xcd"{" BlockStmts\opt LastExp \xcd"}" \\
                     \| \xcd"=" Annotations\opt Block \\
                     \| Annotations\opt Block \\
                     \| \xcd";" \\
%(FROM #(prod:BinOpDecln)#)
          BinOpDecln \: MethMods \xcd"operator" TypeParams\opt \xcd"(" Formal  \xcd")" BinOp \xcd"(" Formal  \xcd")" Guard\opt HasResultType\opt MethodBody & (\ref{prod:BinOpDecln}) \\
                     \| MethMods \xcd"operator" TypeParams\opt \xcd"this" BinOp \xcd"(" Formal  \xcd")" Guard\opt HasResultType\opt MethodBody \\
                     \| MethMods \xcd"operator" TypeParams\opt \xcd"(" Formal  \xcd")" BinOp \xcd"this" Guard\opt HasResultType\opt MethodBody \\
%(FROM #(prod:PrefixOpDecln)#)
       PrefixOpDecln \: MethMods \xcd"operator" TypeParams\opt PrefixOp \xcd"(" Formal  \xcd")" Guard\opt HasResultType\opt MethodBody & (\ref{prod:PrefixOpDecln}) \\
                     \| MethMods \xcd"operator" TypeParams\opt PrefixOp \xcd"this" Guard\opt HasResultType\opt MethodBody \\
%(FROM #(prod:ApplyOpDecln)#)
        ApplyOpDecln \: MethMods \xcd"operator" \xcd"this" TypeParams\opt Formals Guard\opt HasResultType\opt MethodBody & (\ref{prod:ApplyOpDecln}) \\
%(FROM #(prod:ConversionOpDecln)#)
   ConversionOpDecln \: ExplConvOpDecln & (\ref{prod:ConversionOpDecln}) \\
                     \| ImplConvOpDecln \\
\end{bbgrammar}
%##)


\index{parameter!var}
\index{parameter!val}
A formal parameter may have a \xcd"val" or \xcd"var"
% , or \Xcd{ref}
modifier; \xcd`val` is the default.
The body of the method is executed in an environment in which 
each formal parameter corresponds to a local variable (\xcd`var` iff the
formal parameter is \xcd`var`)
and is initialized with the value of the actual parameter.

\subsection{Forms of Method Definition}

There are several syntactic forms for definining methods.   The forms that
include a block, such as \xcd`def m(){S}`, allow an arbitrary block.  These
forms can define a \xcd`void` method, which does not return a value. 

The
forms that include an expression, such as \xcd`def m()=E`, require a
syntactically and semantically valid expression.   These forms cannot define a
\xcd`void` method, because expressions cannot be \xcd`void`.  

There are no other semantic differences between the two forms. 

\subsection{Method Return Types}

A method with an explicit return type returns that type.
A method without an
explicit return type is given a return type by type inference.
A {\em call} to a method has type given by substituting information about the
actual \xcd`val` parameters for the formals.

\begin{ex}

In the example below, \xcd`met1` has an explicit return type \xcd`Ret{n==a}`.
\xcd`met2` does not, so its return type is computed, also to be
\xcd`Ret{n==a}`, because that's what the implicitly-defined constructor 
returns.

\xcd`use3` requires that its argument have \xcd`n==3`.  
\xcd`example` shows that both \xcd`met1` and \xcd`met2` can be used to produce
such an object.  In both cases, the actual argument \xcd`3` is substituted for
the formal argument \xcd`a` in the return type expression for the method
\xcd`Ret{n==a}`, giving the type \xcd`Ret{n==3}` as required by \xcd`use3`.

%~~gen ^^^ Classes9q2w
% package Classes9q2w;
%~~vis
\begin{xten}
class Ret(n:Int) {
  static def met1(a:Int) : Ret{n==a} = new Ret(a);
  static def met2(a:Int)             = new Ret(a);
  static def use3(Ret{n==3}) {}
  static def example() {
     use3(met1(3));
     use3(met2(3));
  }  
}
\end{xten}
%~~siv
%
%~~neg


\end{ex}


\subsection{Throws Clause}
The \xcd`throws` clause indicates what checked exceptions may be
raised during the execution of the method and are not handled by
\xcd'catch' blocks within the method.  If a checked exception may
escape from the method, then it must be by a subtype of one of the
types listed in the \xcd`throws` clause of the method.   Checked
exceptions are defined to be any subclass of
\xcd{x10.lang.CheckedThrowable} that are not also subclasses of
either \xcd{x10.lang.Exception} or \xcd{x10.lang.Error}. 

If a method is implementing an interface or overriding a superclass
method the set of types represented by its \xcd'throws' clause must by
a (potentially improper) subset of the types of the \xcd'throws'
clause of the method it is overriding. 

\subsection{Final Methods}
\index{final}
\index{method!final}
An instance method may be given the \xcd`final` qualifier.  \xcd`final`
methods may not be overridden.

\subsection{Generic Instance Methods}
\index{method!generic instance}

\limitationx{}
In X10, an instance method may be generic: 
%~~gen ^^^ Classes1b7z
% package Classes1b7z;
% NOTEST
%~~vis
\begin{xten}
class Example {
  def example[T](t:T) = "I like " + t;
}
\end{xten}
%~~siv
%
%~~neg

However, the C++ back end does not currently support generic virtual instance
methods like \xcd`example`.  It does allow generic instance methods which are
\xcd`final` or \xcd`private`, and it does allow generic static methods.  


\subsection{Method Guards}
\label{MethodGuard}
\index{method!guard}
\index{guard!on method}

Often, a method will only make sense to invoke under certain
statically-determinable conditions.  These conditions may be expressed as a
guard on the method.

\begin{ex}
For example, \xcd`example(x)` is only
well-defined when \xcd`x != null`, as \xcd`null.toString()` throws a null
pointer exception, and returns nothing: 
%~~gen ^^^ Classes80
% package Classes.methodwithconstraintthingie;
% 
%~~vis
\begin{xten}
class Example {
   var f : String = "";
   def setF(x:Any){x != null} : void = {
      this.f = x.toString();
   }
}
\end{xten}
%~~siv
%
%~~neg
\noindent
(We could have used a constrained type \xcd`Any{self!=null}` for \xcd`x`
instead; in
most cases it is a matter of personal preference or convenience of expression
which one to use.) 
\end{ex}


The requirement of having a method guard 
is that callers must demonstrate to
the X10
compiler that the guard is satisfied.  
With the \xcd`STATIC_CHECKS` compiler option in force (\Sref{sect:Callstyle}), this is
checked at compile time. 
As usual with static constraint
checking, there is no runtime cost.  Indeed, this code can be more efficient
than usual, as it is statically provable that \xcd`x != null`.

When \xcd`STATIC_CHECKS` is not in force, dynamic checks are generated as
needed; method guards are checked at runtime. This is potentially more
expensive, but may be more convenient. 

\begin{ex}
The following code fragment contains a line which will not compile 
with \xcd`STATIC_CHECKS` on (assuming the guarded \xcd`example` method above).  (X10's type system does not attempt to propagate 
information from \xcd`if`s.)  It will compile with \xcd`STATIC_CHECKS` off,
but it may insert an extra \xcd`null`-test for \xcd`x`.  
%~~gen ^^^ Classes90
% package Classes.methodguardnadacastthingie;
%//OPTIONS: -STATIC_CHECKS
% class Example {var f : String = ""; def example(x:Any){x != null} = {this.f = x.toString();}}
% class Eyample {
%~~vis
\begin{xten}
  def exam(e:Example, x:Any) {
    if (x != null) 
       e.example(x as Any{x != null});
       // If STATIC_CHECKS is in force: 
       // ERROR: if (x != null) e.example(x); 
  }
\end{xten}
%~~siv
%}
%~~neg
\end{ex}


The guard \xcd`{c}` 
in a guarded method 
\xcd`def m(){c} = E;`
specifies a constraint \xcd"c" on the
properties of the class \xcd"C" on which the method is being defined. The
method, in effect, only exists  for those instances of \xcd"C" which satisfy
\xcd"c".  It is 
illegal for code to invoke the method on objects whose static type is
not a subtype of \xcd"C{c}".

Specifically: 
    the compiler checks that every method invocation
    \xcdmath"o.m(e$_1$, $\dots$, e$_n$)"
    is type correct. Each argument
    \xcdmath"e$_i$" must have a
    static type \xcdmath"S$_i$" that is a subtype of the declared type
    \xcdmath"T$_i$" for the $i$th
    argument of the method, and the conjunction of the constraints on the
    static types 
    of the arguments must entail the guard in the parameter list
    of the method.

    The compiler checks that in every method invocation
    \xcdmath"o.m(e$_1$, $\dots$, e$_n$)"
    the static type of \xcd"o", \xcd"S", is a subtype of \xcd"C{c}", where the method
    is defined in class \xcd"C" and the guard for \xcd"m" is equivalent to
    \xcd"c".

    Finally, if the declared return type of the method is
    \xcd"D{d}", the
    return type computed for the call is
    \xcdmath"D{a: S; x$_1$: S$_1$; $\dots$; x$_n$: S$_n$; d[a/this]}",
    where \xcd"a" is a new
    variable that does not occur in
    \xcdmath"d, S, S$_1$, $\dots$, S$_n$", and
    \xcdmath"x$_1$, $\dots$, x$_n$" are the formal
    parameters of the method.


\limitation{
Using a reference to an outer class, \xcd`Outer.this`, in a constraint, is not supported.
}


\subsection{Property methods}
\index{method!property}
\index{property method}

%##(PropertyMethodDeclaration
\begin{bbgrammar}
%(FROM #(prod:PropMethodDecln)#)
     PropMethodDecln \: MethMods Id TypeParams\opt Formals Guard\opt Throws\opt HasResultType\opt MethodBody & (\ref{prod:PropMethodDecln}) \\
                     \| MethMods Id Guard\opt HasResultType\opt MethodBody \\
\end{bbgrammar}
%##)

Property methods are methods that can be evaluated in constraints, as
properties can.   They provide a means of abstraction over properties; \eg,
interfaces can specify property methods that implementing containers must
provide, but, just as they cannot specify ordinary fields, they cannot specify
property fields.   Property methods are very limited in computing power: they
must obey the same restrictions as constraint expressions.  In particular,
they cannot have side effects, or even much code in their bodies.


\begin{ex}
The \xcd`eq()` method below tells if the \xcd`x` and \xcd`y`
properties are equal; the \xcd`is(z)` method tells if they are both equal to
\xcd`z`.  
The \xcd`eq` and \xcd`is` property methods are used in types in the
\xcd`example` method.
%~~gen ^^^ Classes100
%package Classes.PropertyMethods;
%~~vis
\begin{xten}
class Example(x:Int, y:Int) {
   def this(x:Int, y:Int) { property(x,y); }
   property eq() = (x==y);
   property is(z:Int) = x==z && y==z;
   def example( a : Example{eq()}, b : Example{is(3)} ) {}
}
\end{xten}
%~~siv
%
%~~neg
\end{ex}

A property method declared in a class must have
a body and must not be \xcd"void".  The body of the method must
consist of only a single \xcd"return" statement with an expression,  or a single
expression.  It is a static error if the expression cannot be
represented in the constraint system.   Property methods may be \xcd`abstract`
in \xcd`abstract` classes, and may be specified in interfaces, but are
implicitly \xcd`final` in 
non-\xcd`abstract` classes. 

The expression may contain invocations of other property methods.  The
compiler ensures that there are no circularities in property methods, so
property method evaluations always terminate.

Property methods in classes are implicitly \xcd"final"; they cannot be
overridden.  It is a static error if a superclass has a property method with a
given signature, and a subclass has a method or property method with the same
signature.   It is a static error if a superclass has a property with some
name \xcd`p`, and a subclass has a nullary method of any kind (instance,
static, or property) also named \xcd`p`. 



A nullary property method definition may omit 
the \xcd"def" keyword.  That is, the following are equivalent:

%~~gen ^^^ Classes110
% package classes.waifsome1;
% class Waif(rect:Boolean, onePlace:Place, zeroBased:Boolean) {
%~~vis
\begin{xten}
property def rail(): Boolean = 
   rect && onePlace == here && zeroBased;
\end{xten}
%~~siv
%}
%~~neg
and
%~~gen ^^^ Classes120
% package classes.waifsome2;
% class Waif(rect:Boolean, onePlace:Place, zeroBased:Boolean) {
%~~vis
\begin{xten}
property rail(): Boolean = 
   rect && onePlace == here && zeroBased;
\end{xten}
%~~siv
%}
%~~neg

Similarly, nullary property methods can be inspected in constraints without
\xcd`()`. If \xcd`ob`'s type has a property \xcd`p`, then \xcd`ob.p` is that
property. Otherwise, if it has a nullary property method \xcd`p()`, \xcd`ob.p`
is equivalent to \xcd`ob.p()`. As a consequence, if the type provides both a
property \xcd`p` and a nullary method \xcd`p()`, then the property can be
accessed as \xcd`ob.p` and the method as \xcd`ob.p()`.\footnote{This only
applies to nullary property methods, not nullary instance methods.  Nullary
property methods perform limited computations, have no side effects, and
always return the same value, since
they have to be expressed in the constraint sublanguage.  In this sense, a
nullary property method does not behave hugely different from a property.
Indeed, a compilation scheme which cached the value of the property method
would all but erase the distinction.  Other methods may
have more behavior, \eg, side effects, so we keep the \xcd`()` to make it
clear that a method call is potentially large.
}

%~~longexp~~`~~` ^^^ Classes130
% package classes.not.weasels;
% class Waif(rect:Boolean, onePlace:Place, zeroBased:Boolean) {
%   def this(rect:Boolean, onePlace:Place, zeroBased:Boolean) 
%          :Waif{self.rect==rect, self.onePlace==onePlace, self.zeroBased==zeroBased}
%          = {property(rect, onePlace, zeroBased);}
%   property rail(): Boolean = rect && onePlace == here && zeroBased;
%   static def zoink() {
%      val w : Waif{
%~~vis
\xcd`w.rail`, with either definition above, 
% }= new Waif(true, here, true);
% }}
%~~pxegnol
is equivalent to 
%~~longexp~~`~~` ^^^ Classes140
% package classes.not.ferrets;
% class Waif(rect:Boolean, onePlace:Place, zeroBased:Boolean) {
%   def this(rect:Boolean, onePlace:Place, zeroBased:Boolean) 
%          :Waif{self.rect==rect, self.onePlace==onePlace, self.zeroBased==zeroBased}
%          = {property(rect, onePlace, zeroBased);}
%   property rail(): Boolean = rect && onePlace == here && zeroBased;
%   static def zoink() {
%      val w : Waif{
%~~vis
\xcd`w.rail()`
% }= new Waif(true, here, true);
% }}
%~~pxegnol


\subsubsection{Limitation of Property Methods}

\limitationx{} 
Currently, X10 forbids the use of property methods which have all the
following features: 
\begin{itemize}
\item they are abstract, and
\item they have one or more arguments, and
\item they appear as subterms in constraints.
\end{itemize}
Any two of these features may be combined, but the three together may not be. 

\begin{ex} 
The constraint in \xcd`example1` is concrete, not abstract.  The constraint in
\xcd`example2` is nullary, and has no arguments.  The constraint in
\xcd`example3` appears at the top level, rather than as a subterm ({\em cf.}
the equality expressions \xcd`A==B` in the other examples).    However,
\xcd`example4` combines all three features, and is not allowed.
%~~gen ^^^ Classes7a5j
% package Classes7a5j;
% // If example4() compiles, then the limitation in Classes7a5j's section is
% // gone, so delete the whole subsection from the spec.
%~~vis
\begin{xten}
class Cls {
  property concrete(a:Int) = 7;
}
interface Inf {
  property nullary(): Int;
  property topLevel(z:Int):Boolean;
  property allThree(z:Int):Int;
}
class Example{
  def example1(Cls{self.concrete(3)==7}) = 1;
  def example2(Inf{self.nullary()==7})   = 2;
  def example3(Inf{self.topLevel(3)})    = 3;
  //ERROR: def example4(Inf{self.allThree(3)==7}) = "fails";
}
\end{xten}
%~~siv
%
%~~neg
\end{ex}


\subsection{Method overloading, overriding, hiding, shadowing and obscuring}
\label{MethodOverload}
\index{method!overloading}



The definitions of method overloading, overriding, hiding, shadowing and
obscuring in \Xten{} are familiar from languages such as Java, modulo the
following considerations motivated by type parameters and dependent types.



Two or more methods of a class or interface may have the same
name if they have a different number of type parameters, or
they have formal parameters of different constraint-erased types (in some instantiation of the
generic parameters). 



\begin{ex}
The following overloading of \xcd`m` is unproblematic.
%~~gen ^^^ Classes150
% package Classes.Mful;
%~~vis
\begin{xten}
class Mful{
   def m() = 1;
   def m[T]() = 2;
   def m(x:Int) = 3;
   def m[T](x:Int) = 4;
}
\end{xten}
%~~siv
%
%~~neg
\end{ex}


A class definition may include methods which are ambiguous in {\em some}
generic instantiation. (It is a compile-time error if the methods are
ambiguous in {\em every} generic instantiation, but excluding class
definitions which are are ambiguous in {\em some} instantiation would exclude
useful cases.)  It is a compile-time error to {\em use} an ambiguous method
call. 

\begin{ex}
The following class definition is acceptable.  However, the marked method
calls are ambiguous, and hence not acceptable.
%~~gen ^^^ Classes4d5e
% package Classes4d5e;
%~~vis
\begin{xten}
class Two[T,U]{
  def m(x:T)=1;
  def m(x:Int)=2;
  def m[X](x:X)=3;
  def m(x:U)=4;
  static def example() {
    val t12 = new Two[Int, Any]();
    // ERROR: t12.m(2);
    val t13  = new Two[String, Any]();
    t13.m("ferret");
    val t14 = new Two[Boolean,Boolean]();
    // ERROR: t14.m(true);
  }
}
\end{xten}
%~~siv
%~~neg
\noindent
The call \xcd`t12.m(2)` could refer to either the \xcd`1` or \xcd`2`
definition of \xcd`m`, so it is not allowed.   
The call \xcd`t14.m(true)` could refer to either the \xcd`1` or \xcd`4`
definition, so it, too, is not allowed.

The call \xcd`t13.m("ferret")` refers only to the \xcd`1` definition.  If
the \xcd`1` definition were absent, type argument inference would make it
refer to the \xcd`3` definition.  However, X10 will choose a fully-specified
call if there is one, before trying type inference, so this call unambiguously
refers to \xcd`1`.
\end{ex}


\XtenCurrVer{} does not permit overloading based on constraints. That is, the
following is {\em not} legal, although either method definition individually
is legal:
\begin{xten}
   def n(x:Int){x==1} = "one";
   def n(x:Int){x!=1} = "not";
\end{xten}




The definition of a method declaration \xcdmath"m$_1$" ``having the same signature
as'' a method declaration \xcdmath"m$_2$" involves identity of types. 



The {\em constraint erasure} of a type \xcdmath"T", 
\xcdmath"$ce$(T)",
is obtained by removing all the constraints outside of functions in \xcd`T`,
specificially: 
\begin{eqnarray}
ce({\tt T}) &=& {\tt T} \mbox{ if \xcd`T` is a container or interface}\\
ce({\tt T\{c\}}) &=& ce({\tt T})\\
ce({\tt T[S}_1{\tt,}\ldots{\tt,S}_n{\tt ]})
  &=&
ce({\tt T}){\tt [} ce({\tt S}_1){\tt,}\ldots{\tt,} ce({\tt S}_n){\tt ]}\\
ce({\tt (S}_1{\tt,}\ldots{\tt,S}_n{\tt ) => T})
  &=&
{\tt }{\tt (} ce({\tt S}_1){\tt,}\ldots{\tt,} ce({\tt S}_n){\tt ) => } 
ce({\tt T})
\end{eqnarray}



 Two methods are said to have {\em erasedly equivalent signatures} if (a) they have the
 same number of type parameters, 
(b) they have the same number of formal (value) parameters, and (c)
for each formal parameter the constraint erasure of its types are erasedly equivalent.
It is a 
compile-time error for there to be two methods with the same name and
erasedly equivalent signatures in a class (either defined in that class or in a
superclass), unless the signatures are identical (without erasures) and one of the methods is
defined in a superclass (in which case the superclass's method is overridden
by the subclass's, and the subclass's method's return type must be a subtype of
the superclass's method's return type).  

 



In addition, the guard of an overridden method
must entail
the guard of the overriding method.   This
ensures that any virtual call to the method
satisfies the guard of the callee.

\begin{ex}
In the following example, the call to \xcd`s.recip(3)` in \xcd`example()`
will invoke \xcd`Sub.recip(n)`.  The call is legitimate because
\xcd`Super.recip`'s guard, \xcd`n != 0`, is satisfied by \xcd`3`.  
The guard on \xcd`Sub.recip(n)` is simply
\xcd`true`, which is also satisfied.  However, if we had used the \xcd`ERROR`
line's definition, the guard on \xcd`Sub.recip(n)` would be \xcd`n != 0, n != 3`, which
is not satisfied by \xcd`3`, so -- despite the call statically type-checking
-- at runtime the call would violate its guard and (in this case) throw an exception.
%~~gen ^^^ Classes5l3r
% package Classes5l3r;
% // FOR-ERR-CASE-DELETE: def recip(m:Int){true} = 1.0/m;
%~~vis
\begin{xten}
class Super {
  def recip(n:Int){n != 0} = 1.0/n;
}
class Sub extends Super{
  //ERROR: def recip(n:Int){n != 0, n != 3} = 1.0/(n * (n-3));
  def recip(m:Int){true} = 1.0/m;
}
class Example{
  static def example() {
     val s : Super = new Sub();
     s.recip(3);
  }
}
\end{xten}
%~~siv
%
%~~neg

\end{ex}


  If a class \xcd"C" overrides a method of a class or interface
  \xcd"B", the guard of the method in \xcd"B" must entail
  the guard of the method in \xcd"C".


A class \xcd"C" inherits from its direct superclass and superinterfaces all
their methods visible according to the access modifiers
of the superclass/superinterfaces that are not hidden or overridden. A method \xcdmath"M$_1$" in a class
\xcd"C" overrides
a method \xcdmath"M$_2$" in a superclass \xcd"D" if
\xcdmath"M$_1$" and \xcdmath"M$_2$" have erasedly equivalent signatures.
Methods are overriden on a signature-by-signature basis.  It is a compile-time
error if an instance method overrides a static method.  (But is it permitted
for an instance {\em field} to hide a static {\em field}; that's hiding
(\Sref{sect:FieldHiding}), not 
overriding, and hence totally different.)

\section{Constructors}
\label{sect:constructors}
\index{constructor}

Instances of classes are created by the \xcd`new` expression: \\
%##(ClassInstCreationExp
\begin{bbgrammar}
%(FROM #(prod:ObCreationExp)#)
       ObCreationExp \: \xcd"new" TypeName TypeArgs\opt \xcd"(" ArgumentList\opt \xcd")" ClassBody\opt & (\ref{prod:ObCreationExp}) \\
                     \| Primary \xcd"." \xcd"new" Id TypeArgs\opt \xcd"(" ArgumentList\opt \xcd")" ClassBody\opt \\
                     \| FullyQualifiedName \xcd"." \xcd"new" Id TypeArgs\opt \xcd"(" ArgumentList\opt \xcd")" ClassBody\opt \\
\end{bbgrammar}
%##)

This constructs a new object, and calls some code, called a {\em constructor},
to initialize the newly-created object properly.

Constructors are defined like methods, except that they must be named \xcd`this`
and ordinary methods may not be.    The content of a constructor body has
certain capabilities (\eg, \xcd`val` fields of the object may be initialized)
and certain restrictions (\eg, most methods cannot be called); see
\Sref{ObjectInitialization} for the details.

\begin{ex}

The following class provides two constructors.  The unary constructor 
\xcd`def this(b : Int)` allows initialization of the \xcd`a` field to an 
arbitrary value.  The nullary constructor \xcd`def this()` gives it a default
value of 10.  The \xcd`example` method illustrates both of these calls.


%~~gen ^^^ ClassesCtor10
% package ClassesCtor10;
%~~vis
\begin{xten}
class C {
  public val a : Int;
  def this(b : Int) { a = b; } 
  def this()        { a = 10; }
  static def example() {
     val two = new C(2);
     assert two.a == 2;
     val ten = new C(); 
     assert ten.a == 10;
  }
}
\end{xten}
%~~siv
% class Hook{ def run() {C.example(); return true;}}
%~~neg
\end{ex}


\subsection{Automatic Generation of Constructors}
\index{constructor!generated}

Classes that have no constructors written in the class declaration are
automatically given a constructor which sets the class properties and does
nothing else. If this automatically-generated constructor is not valid (\eg,
if the class has \xcd`val` fields that need to be initialized in a
constructor), the class has no constructor, which is a static error.

\begin{ex}
The following class has no explicit constructor.
Its implicit constructor is 
\xcd`def this(x:Int){property(x);}`
This implicit constructor is valid, and so is the class. 
%~~gen ^^^ ClassesCtor20
% package ClassesCtor20;
%~~vis
\begin{xten}
class C(x:Int) {
  static def example() {
    val c : C = new C(4);
    assert c.x == 4;
  }
}
\end{xten}
%~~siv
% class Hook{ def run() {C.example(); return true;}}
%~~neg
\noindent 


The following class has the same default constructor.  However, that
constructor does not initialize \xcd`d`, and thus is invalid.  This 
class does not compile; it needs an explicit constructor.
%~~gen ^^^ ClassCtor30_MustFailCompile
% NOCOMPILE
%~~vis
\begin{xten}
// THIS CODE DOES NOT COMPILE
class Cfail(x:Int) {
  val d: Int;
  static def example() {
    val wrong = new Cfail(40);
  }
}
\end{xten}
%~~siv
%
%~~neg


\end{ex}

\subsection{Calling Other Constructors}
\label{sect:call-another-ctor}

The {\em first} statement of a constructor body may be a call of the form 
\xcd`this(a,b,c)` or \xcd`super(a,b,c)`.  The former will execute the body of
the matching constructor of the current class; the latter, of the superclass. 
This allows a measure of abstraction in constructor definitions; one may be
defined in terms of another.

\begin{ex}
The following class has two constructors.  \xcd`new Ctors(123)` constructs a
new \xcd`Ctors` object with parameter 123.  \xcd`new Ctors()` constructs one
whose parameter has a default value of 100: 
%~~gen ^^^ Classes5q6q
% package Classes5q6q;
%~~vis
\begin{xten}
class Ctors {
  public val a : Int;
  def this(a:Int) { this.a = a; }
  def this()      { this(100);  }
}
\end{xten}
%~~siv
%class Hook{ def run() {
% val x = new Ctors(10); assert x.a == 10;
% val y = new Ctors(); assert y.a == 100;
% return true;}}
%~~neg
\end{ex}

In the case of a class which implements \xcd`operator ()` 
--- or any other constructor and application with the same signature --- 
this can be ambiguous.  If \xcd`this()` appears as the first statement of a
constructor body, it could, in principle, mean either a constructor call or an
operator evaluation.   This ambiguity is resolved so that \xcd`this()` always
means the constructor invocation.  If, for some reason, it is necessary to
invoke an application operator as the first meaningful statement of a
constructor, write the target of the application as \xcd`(this)`, \eg,
\xcd`(this)(a,b);`. 

\subsection{Return Type of Constructor}

A constructor for class \xcd`C` may have a return type \xcd`C{c}`.  The return
type specifies a constraint on the kind of object returned.  It cannot change
its {\em class} --- a constructor for class \xcd`C` always returns an instance
of class \xcd`C`.  
If no explicit return type is specified, the constructor's return type is
inferred.

\begin{ex}
The constructor \xcd`(A)` below, having no explicit return type, 
has its return type inferred.  
\xcd`n` is set by the \xcd`property` statement to \xcd`1`, so the return type
is inferred as \xcd`Ret{self.n==1}.`
The constructor \xcd`(B)` has \xcd`Ret{n==self.n}` as an explicit return type.
The \xcd`example()` code shows both of these in action.

%~~gen ^^^ Classes1v9a
% package Classes1v9a;
%~~vis
\begin{xten}
class Ret(n:Int) {
   def this()    { property(1); }     // (A)
   def this(n:Int) : Ret{n==self.n} { // (B)
      property(n);
   }
   static def typeIs[T](x:T){}
   static def example() {
     typeIs[Ret{self.n==1}](new Ret());  // uses (A)
     typeIs[Ret{self.n==3}](new Ret(3)); // uses (B)
   }
}
\end{xten}
%~~siv
%
%~~neg


\end{ex}

\section{Static initialization}
\label{StaticInitialization}
\index{initialization!static} 
Static fields in \Xten{} are immutable and are guaranteed to be
initialized before they are accessed. Static fields are initialized on
a per-Place basis; thus an activity that reads a static field in two
different Places may read different values for the content of the
field in each Place.  Static fields are not eagerly initialized, thus
if a particular static field is not accessed in a given Place then the
initializer expression for that field may not be evaluated in that
Place.

When an activity running in a Place \Xcd{P} attempts to read a static
field \Xcd{F} that has not yet been initialized in \Xcd{P}, then the
activity will evaluate the initializer expression for \Xcd{F} and
store the resulting value in \Xcd{F}. It is guaranteed that at most
one activity in each Place will attempt to evaluate the initializer
expression for a given static field.  If a second activity attempts to
read \Xcd{F} while the first activity is still executing the
initializer expression the second activity will be suspended until the
first activity finishes evaluating the initializer and stores the
resulting value in \Xcd{F}.

The initializer expression may directly or indirectly read other
static fields in the program.  If there is a cycle in the field
initialization dependency graph for a set of static fields, then any
activities accessing those fields may deadlock, which in turn may
result in the program deadlocking.\footnote{The current \Xten{}
  runtime does not dynamically detect this situation. Future versions
  of \Xten may be able to detect this and convert such a deadlock into the
  throwing of an \Xcd{ExceptionInInitializer} exception.}.

If an exception is raised during the evaluation of a static field's
initializer expression, then the field is deemed uninitializable in
that Place and any subsequent attempt to access the static field's
value by another activity in the Place will also result in an
exception being raised.\footnote{The implementation will make a best
  effort attempt to present stack trace information about the original
  cause of the exception in all subsequent raised exceptions}.  Failure
to initialize a field in one Place does not impact the initialization
status of the field in other Places.

\subsection{Compatability with Prior Versions of \Xten{}}
Previous versions of \Xten{} eagerly initialized all static fields in
the program at Place 0 and serialized the resulting values to all
other Places before beginning execution of the user main function.  It
is possible to simulate these serialization semantics for specific
static fields under the lazy per-Place initialization semantics
by using the idiom below:

\begin{xten}
// Pre X10 2.3 code
// expr evaluated once in Place 0 and resulting value 
// serialized to all other places
public static val x:T = expr;

// X10 2.3 code when T haszero is false
private static val x_holder:Cell[T] = 
    (here == Place.FIRST_PLACE) ? new Cell[T](expr): null;
public static val x:T = at (Place.FIRST_PLACE) x_holder();

// simpler X10 2.3 code when T haszero is true
private static val x_holder:T = 
    (here == Place.FIRST_PLACE) ? expr : Zero.get[T]();
public static val x:T = at (Place.FIRST_PLACE) x_holder;

\end{xten}

A slightly more complex variant of the above idiom in which the
initializer expression for the public field conditionally does the \xcd{at}
only when not executed at \xcd{Place.FIRST_PLACE} can be used to
obtain exactly the same serialization behavior as the pre \Xten{} v2.3
semantics.  When necessary, eager initialization for specific static fields
can be simulated by reading the static fields in \xcd{main} before
executing the rest of the program.

\section{User-Defined Operators}
\label{sect:operators}
\index{operator}
\index{operator!user-defined}

%##(MethodDeclaration
\begin{bbgrammar}
%(FROM #(prod:MethodDecln)#)
         MethodDecln \: MethMods \xcd"def" Id TypeParams\opt Formals Guard\opt Throws\opt HasResultType\opt MethodBody & (\ref{prod:MethodDecln}) \\
                     \| BinOpDecln \\
                     \| PrefixOpDecln \\
                     \| ApplyOpDecln \\
                     \| SetOpDecln \\
                     \| ConversionOpDecln \\
\end{bbgrammar}
%##)


It is often convenient to have methods named by symbols rather than words.
For example, suppose that we wish to define a \xcd`Poly` class of
polynomials -- for the sake of illustration, single-variable polynomials with
\xcd`Int` coefficients.  It would be very nice to be able to manipulate these
polynomials by the usual operations: \xcd`+` to add, \xcd`*` to multiply,
\xcd`-` to subtract, and \xcd`p(x)` to compute the value of the polynomial at
argument \xcd`x`.  We would like to write code thus: 
%~~gen ^^^ Classes160
% package Classes.In.Poly101;
% // Integer-coefficient polynomials of one variable.
% class Poly {
%   public val coeff : Rail[Int];
%   public def this(coeff: Rail[Int]) { this.coeff = coeff;}
%   public def degree() = (coeff.size-1) as Int;
%   public def a(i:Int) = (i<0 || i>this.degree()) ? 0 : coeff(i);
%
%   public static operator (c : Int) as Poly = new Poly([c as Int]);
%
%   public operator this(x:Long) {
%     val d = this.degree();
%     var s : Long = this.a(d);
%     for( i in 1 .. this.degree() ) {
%        s = x * s + a(d-i);
%     }
%     return s;
%   }
%
%   public operator this + (p:Poly) =  new Poly(
%      new Rail[Int](
%         Math.max(this.coeff.size, p.coeff.size) as Int,
%         (i:Int) => this.a(i) + p.a(i)
%      ));
%   public operator this - (p:Poly) = this + (-1)*p;
%
%   public operator this * (p:Poly) = new Poly(
%      new Rail[Int](
%        this.degree() + p.degree() + 1,
%        (k:Int) => sumDeg(k, this, p)
%        )
%      );
%
%
%   public operator (n : Int) + this = (n as Poly) + this;
%   public operator this + (n : Int) = (n as Poly) + this;
%
%   public operator (n : Int) - this = (n as Poly) + (-1) * this;
%   public operator this - (n : Int) = ((-n) as Poly) + this;
%
%   public operator (n : Int) * this = new Poly(
%      new Rail[Int](
%        this.degree()+1,
%        (k:Int) => n * this.a(k)
%      ));
%   private static def sumDeg(k:Int, a:Poly, b:Poly) {
%      var s : Int = 0;
%      for( i in 0 .. k ) s += a.a(i) * b.a(k-i);
%        // x10.io.Console.OUT.println("sumdeg(" + k + "," + a + "," + b + ")=" + s);
%      return s;
%      };
%   public final def toString() = {
%      var allZeroSoFar : Boolean = true;
%      var s : String ="";
%      for( i in 0..this.degree() ) {
%        val ai = this.a(i);
%        if (ai == 0) continue;
%        if (allZeroSoFar) {
%           allZeroSoFar = false;
%           s = term(ai, i);
%        }
%        else
%           s +=
%              (ai > 0 ? " + " : " - ")
%             +term(ai, i);
%      }
%      if (allZeroSoFar) s = "0";
%      return s;
%   }
%   private final def term(ai: Int, n:Int) = {
%      val xpow = (n==0) ? "" : (n==1) ? "x" : "x^" + n ;
%      return (ai == 1) ? xpow : "" + Math.abs(ai) + xpow;
%   }
%
%   public static def Main(ss:Rail[String]):void {main(ss);};
%


%~~vis
\begin{xten}
  public static def main(Rail[String]):void {
     val X = new Poly([0,1]);
     val t <: Poly = 7 * X + 6 * X * X * X; 
     val u <: Poly = 3 + 5*X - 7*X*X;
     val v <: Poly = t * u - 1;
     for( i in -3 .. 3) {
       x10.io.Console.OUT.println(
         "" + i + "	X:" + X(i) + "	t:" + t(i) 
         + "	u:" + u(i) + "	v:" + v(i)
         );
     }
  }

\end{xten}
%~~siv
%}
%~~neg

Writing the same code with method calls, while possible, is far less elegant: 
%~~gen ^^^ Classes170

%package Classes.In.Remedial.Poly101;
% // Integer-coefficient polynomials of one variable.
% class UglyPoly {
%   public val coeff : Rail[Int];
%   public def this(coeff: Rail[Int]) { this.coeff = coeff;}
%   public def degree() = (coeff.size-1) as Int;
%   public  def  a(i:Int) = (i<0 || i>this.degree()) ? 0 : coeff(i);
%
%   public static operator (c : Int) as UglyPoly = new UglyPoly([c as Int]);
%
%   public def apply(x:Int) {
%     val d = this.degree();
%     var s : Int = this.a(d);
%     for( i in 1 .. this.degree() ) {
%        s = x * s + a(d-i);
%     }
%     return s;
%   }
%
%   public operator this + (p:UglyPoly) =  new UglyPoly(
%      new Rail[Int](
%         Math.max(this.coeff.size, p.coeff.size) as Int,
%         (i:Int) => this.a(i) + p.a(i)
%      ));
%   public operator this - (p:UglyPoly) = this + (-1)*p;
%
%   public operator this * (p:UglyPoly) = new UglyPoly(
%      new Rail[Int](
%        this.degree() + p.degree() + 1,
%        (k:Int) => sumDeg(k, this, p)
%        )
%      );
%
%
%   public operator (n : Int) + this = (n as UglyPoly) + this;
%   public operator this + (n : Int) = (n as UglyPoly) + this;
%
%   public operator (n : Int) - this = (n as UglyPoly) + (-1) * this;
%   public operator this - (n : Int) = ((-n) as UglyPoly) + this;
%
%   public operator (n : Int) * this = new UglyPoly(
%      new Rail[Int](
%        this.degree()+1,
%        (k:Int) => n * this.a(k)
%      ));
%   private static def sumDeg(k:Int, a:UglyPoly, b:UglyPoly) {
%      var s : Int = 0;
%      for( i in 0 .. k ) s += a.a(i) * b.a(k-i);
%        // x10.io.Console.OUT.println("sumdeg(" + k + "," + a + "," + b + ")=" + s);
%      return s;
%      };
%   public final def toString() = {
%      var allZeroSoFar : Boolean = true;
%      var s : String ="";
%      for( i in 0..this.degree() ) {
%        val ai = this.a(i);
%        if (ai == 0) continue;
%        if (allZeroSoFar) {
%           allZeroSoFar = false;
%           s = term(ai, i);
%        }
%        else
%           s +=
%              (ai > 0 ? " + " : " - ")
%             +term(ai, i);
%      }
%      if (allZeroSoFar) s = "0";
%      return s;
%   }
%   private final def term(ai: Int, n:Int) = {
%      val xpow = (n==0) ? "" : (n==1) ? "x" : "x^" + n ;
%      return (ai == 1) ? xpow : "" + Math.abs(ai) + xpow;
%   }
%
%   def mult(p:UglyPoly) : UglyPoly = this * p;
%   def mult(n:Int)      : UglyPoly = n * this;
%   def plus(p:UglyPoly) : UglyPoly = this + p;
%   def plus(n:Int)      : UglyPoly = n + this;
%   def minus(p:UglyPoly): UglyPoly = this - p;
%   def minus(n:Int)     : UglyPoly = this - n;
%   static def const(n:Int): UglyPoly = n as UglyPoly;
%
%~~vis
\begin{xten}
  public static def uglymain() {
     val X = new UglyPoly([0,1]);
     val t <: UglyPoly 
           = X.mult(7).plus(
               X.mult(X).mult(X).mult(6));  
     val u <: UglyPoly 
           = const(3).plus(
               X.mult(5)).minus(X.mult(X).mult(7));
     val v <: UglyPoly = t.mult(u).minus(1);
     for( i in -3 .. 3) {
       x10.io.Console.OUT.println(
         "" + i + "	X:" + X.apply(i) + "	t:" + t.apply(i) 
          + "	u:" + u.apply(i) + "	v:" + v.apply(i)
         );
     }
  }
\end{xten}
%~~siv
%}
%~~neg

The operator-using code can be written in X10, though a few variations are
necessary to handle such exotic cases as \xcd`1+X`.



Most X10 operators can be given definitions.\footnote{Indeed, even for the
standard types, these operators are defined in the library.  Not even as basic
an operation as integer addition is built into the language.  Conversely, if
you define a full-featured numeric type, it will have most of the privileges that
the standard ones enjoy.  The missing priveleges are (1) literals; (2) 
the \xcd`..` operator won't compute the \xcd`zeroBased` and \xcd`rail`
properties as it does for \xcd`Int` ranges; (3) \xcd`*` won't track ranks, as
it does for \xcd`Region`s; 
(4) \xcd`&&` and \xcd`||` won't short-circuit, as they do for \xcd`Boolean`s, 
(5) the built-in notion of equality \xcd`a==b` will only coincide with
the programmible notion \xcd`a.equals(b)`, as they do for most library types,
if coded that way; and (6) it is 
impossible to define an 
operation like \xcd`String.+` which converts both its left and right arguments
from any type.  For example, a \xcd`Polar` type might
have many representations for the origin, as radius 0 and any angle; these
will be \xcd`equals()`, but will not be \xcd`==`; whereas for the standard
\xcd`Complex` type, the two equalities are the same.}  (However, \xcd`&&` and
\xcd`||` 
are only short-circuiting for \xcd`Boolean` expressions; user-defined versions
of these operators have no special execution behavior.)

The user-definable operations are (in order of precedence): \\
\begin{tabular}{l}
implicit type coercions\\
postfix \xcd`()`\\
\xcd`as T`\\
these unary operators:  \xcd`- + ! ~ | & / ^ * %`\\
\xcd`..`\\
\xcd`*      /       %      **`\\
\xcd`+` \xcd`     -` \\
\xcd`<<     >>      >>>    ->     <-     >-      -<      !`\\
\xcd`>      ` \xcd`>=     ` \xcd`<     ` \xcd`<=     ` 
\xcd`~      !~`\\
\xcd`&` \\
\xcd`^` \\
\xcd`|` \\
\xcd`&&` \\
\xcd`||` \\
\end{tabular}

Several of these operators have no standard meaning on any library type, and
are included purely for programmer convenience.  


Many operators may be defined either in \xcd`static` or instance forms.  Those
defined in instance form are dynamically dispatched, just like an instance
method.  Those defined in static form are statically dispatched, just like a
static method.  Operators are scoped like methods; static operators are scoped
like static methods.

\begin{ex}
%~~gen ^^^ Classes6a1j
% package oifClasses6a1j;
% class Whatever {
% 
%~~vis
\begin{xten}
static class Trace(n:Int){
  public static operator !(f:Trace) 
      = new Trace(10 * f.n + 1);
  public operator -this = new Trace (10 * this.n + 2);
}
static class Brace extends Trace{
  def this(n:Int) { super(n); }
  public operator -this = new Brace (10 * this.n + 3);
  static def example() {
     val t = new Trace(1);
     assert (!t).n == 11;
     assert (-t).n == 12 && (-t instanceof Trace);
     val b = new Brace(1);
     assert (!b).n == 11;
     assert (-b).n == 13 && (-b instanceof Brace);
  }
}

\end{xten}
%~~siv
% // And checking the unambiguous syntax while I'm here...
% //static class Glook { def checky(t:Trace) { 
% //   Trace.operator !(t);
% //   t.operator -();
% //} }
% }
%~~neg
\end{ex}

%%OP%% Operators may be invoked by unambiguous syntax, loosely akin to a
%%OP%% fully-qualified name. For example, \xcd`!t` above may be invoked as
%%OP%% \xcd`Trace.operator !(t)`. This unambiguous syntax may be used even if there
%%OP%% are several \xcd`!` operators that could apply to \xcd`t`, rendering the
%%OP%% convenient short form \xcd`!t` unavailable in some context.




\subsection{Binary Operators}

Binary operators, illustrated by \xcd`+`, may be defined statically in a
container \xcd`A` as:
\begin{xten}
static operator (b:B) + (c:C) = ...;
\end{xten}
%%OP%% In this case it may be invoked as \xcd`A.operator +(b,c)`.
Or, it may be defined as  as an instance operator by one of the forms:
\begin{xten}
operator this + (b:B) = ...;
operator (b:B) + this = ...;
\end{xten}
%%OP%% and be invoked as 
%%OP%% \xcd`a.operator +(b)`
%%OP%% and as 
%%OP%% \xcd`a.operator ()+(b)` 
%%OP%% respectively.

\begin{ex}

Defining the sum \xcd`P+Q` of two polynomials looks much like a method
definition.  It uses the \xcd`operator` keyword instead of \xcd`def`, and
\xcd`this` appears in the definition in the place that a \xcd`Poly` would
appear in a use of the operator.  So, 
\xcd`operator this + (p:Poly)` explains how to add \xcd`this` to a
\xcd`Poly` value.
%~~gen ^^^ Classes180
% package Classes.In.Poly102;
%~~vis
\begin{xten}
class Poly {
  public val coeff : Rail[Int];
  public def this(coeff: Rail[Int]) { 
    this.coeff = coeff;}
  public def degree() = coeff.size-1 as Int;
  public def  a(i:Int) 
    = (i<0 || i>this.degree()) ? 0 : coeff(i);
  public operator this + (p:Poly) =  new Poly(
     new Rail[Int](
        Math.max(this.coeff.size, p.coeff.size) as Int,
        (i:Int) => this.a(i) + p.a(i)
     )); 
  // ... 
\end{xten}
%~~siv
%   public operator (n : Int) + this = new Poly([n as Int]) + this;
%   public operator this + (n : Int) = new Poly([n as Int]) + this;
% 
%   def makeSureItWorks() {
%      val x = new Poly([0,1]);
%      val p <: Poly = x+x+x;
%      val q <: Poly = 1+x;
%      val r <: Poly = x+1;
%   }
%     
% }
%~~neg


The sum of a polynomial and an integer, \xcd`P+3`, looks like
an overloaded method definition.  
%~~gen ^^^ Classes190
% package Classes.In.Poly103;
% class Poly {
%   public val coeff : Rail[Int];
%   public def this(coeff: Rail[Int]) { this.coeff = coeff;}
%   public def degree() = coeff.size-1 as Int;
%   public def  a(i:Int) = (i<0 || i>this.degree()) ? 0 : coeff(i);
% 
%   public operator this + (p:Poly) =  new Poly(
%      new Rail[Int](
%         Math.max(this.coeff.size, p.coeff.size) as Int,
%         (i:Int) => this.a(i) + p.a(i)
%      ));
%    public operator (n : Int) + this = new Poly([n as Int]) + this;
%~~vis
\begin{xten}
   public operator this + (n : Int) 
          = new Poly([n as Int]) + this;
\end{xten}
%~~siv
% 
%   def makeSureItWorks() {
%      val x = new Poly([0,1]);
%      val p <: Poly = x+x+x;
%      val q <: Poly = 1+x;
%      val r <: Poly = x+1;
%   }
%     
% }
%~~neg


However, we want to allow the sum of an integer and a polynomial as well:
\xcd`3+P`.  It would be quite inconvenient to have to define this as a method
on \xcd`Int`; changing \xcd`Int` is far outside of normal coding.  So, we
allow it as a method on \xcd`Poly` as well.


%~~gen ^^^ Classes200
% package Classes.In.Poly104o;
% class Poly {
%   public val coeff : Rail[Int];
%   public def this(coeff: Rail[Int]) { this.coeff = coeff;}
%   public def degree() = coeff.size-1 as Int;
%   public def  a(i:Int) = (i<0 || i>this.degree()) ? 0 : coeff(i);
% 
%   public operator this + (p:Poly) =  new Poly(
%      new Rail[Int](
%         Math.max(this.coeff.size, p.coeff.size) as Int,
%         (i:Int) => this.a(i) + p.a(i)
%      ));
%~~vis
\begin{xten}
   public operator (n : Int) + this 
          = new Poly([n as Int]) + this;
\end{xten}
%~~siv
% 
%   public operator this + (n : Int) = new Poly([n as Int]) + this;
%   def makeSureItWorks() {
%      val x = new Poly([0,1]);
%      val p <: Poly = x+x+x;
%      val q <: Poly = 1+x;
%      val r <: Poly = x+1;
%   }
%     
% }
%~~neg

Furthermore, it is sometimes convenient to express a binary operation as a
static method on a class. 
The definition for the sum of two
\xcd`Poly`s could have been written:
%~~gen ^^^ Classes210
% package Classes.In.Poly105;
% class Poly {
%   public val coeff : Rail[Int];
%   public def this(coeff: Rail[Int]) { this.coeff = coeff;}
%   public def degree() = coeff.size-1 as Int;
%   public def  a(i:Int) = (i<0 || i>this.degree()) ? 0 : coeff(i);
%~~vis
\begin{xten}
  public static operator (p:Poly) + (q:Poly) =  new Poly(
     new Rail[Int](
        Math.max(q.coeff.size, p.coeff.size) as Int,
        (i:Int) => q.a(i) + p.a(i)
     ));
\end{xten}
%~~siv
%
%   public operator (n : Int) + this = new Poly([n as Int]) + this;
%   public operator this + (n : Int) = new Poly([n as Int]) + this;
% 
%   def makeSureItWorks() {
%      val x = new Poly([0,1]);
%      val p <: Poly = x+x+x;
%      val q <: Poly = 1+x;
%      val r <: Poly = x+1;
%   }
%     
% }
%~~neg

\end{ex}

When X10 attempts to typecheck a binary operator expression like \xcd`P+Q`, it
first typechecks \xcd`P` and \xcd`Q`. Then, it looks for operator declarations
for \xcd`+` in the types of \xcd`P` and \xcd`Q`. If there are none, it is a
static error. If there is precisely one, that one will be used. If there are
several, X10 looks for a {\em best-matching} operation, \viz{} one which does
not require the operands to be converted to another type. For example,
\xcd`operator this + (n:Long)` and \xcd`operator this + (n:Int)` both apply to
\xcd`p+1`, because \xcd`1` can be converted from an \xcd`Int` to a \xcd`Long`.
However, the \xcd`Int` version will be chosen because it does not require a
conversion. If even the best-matching operation is not uniquely determined,
the compiler will report a static error.


\subsection{Unary Operators}

Unary operators,  illustrated by \xcd`!`, may be defined statically in
container 
\xcd`A` as 
\begin{xten}
static operator !(x:A) = ...;
\end{xten}
or as instance operators by: 
\begin{xten}
operator !this = ...;
\end{xten}

%%OP%% A statically-defined unary operator \xcd`!` may be invoked on \xcd`a:A` as 
%%OP%% \xcd`A.operator !(a)`.  An instance operator may be invoked as
%%OP%% \xcd`a.operator !()`.  

The rules for typechecking a unary operation are the same as for methods; the
complexities of binary operations are not needed.

\begin{ex}
The operator to negate a polynomial is: 

%~~gen ^^^ Classes220
% package Classes.In.Poly106;
% class Poly {
%   public val coeff : Rail[Int];
%   public def this(coeff: Rail[Int]) { this.coeff = coeff;}
%   public def degree() = coeff.size-1 as Int;
%   public def  a(i:Int) = (i<0 || i>this.degree()) ? 0 : coeff(i);
%~~vis
\begin{xten}
  public operator - this = new Poly(
    new Rail[Int](coeff.size as Int, (i:Int) => -coeff(i))
    );
\end{xten}
%~~siv
%   def makeSureItWorks() {
%      val x = new Poly([0,1]);
%      val p <: Poly = -x;
%   }
% }
%~~neg



\end{ex}


\subsection{Type Conversions}
\label{sect:type-conv}
\index{type conversion!user-defined}


Explicit type conversions, \xcd`e as A`, can be defined as operators on
class \xcd`A`, or on the container type of \xcd`e`.  These must be static
operators.  

To define an operator in \xcd`class A` (or \xcd`struct A)` converting values
of type \xcd`B` into type \xcd`A`, use the syntax: 
\begin{xten}
static operator (x:B) as ? {c} = ... 
\end{xten}
The \xcd`?` indicates the containing type \xcd`A`.  
The guard clause \xcd`{c}` may be omitted.



\begin{ex}
%~~gen ^^^ Classes230
% package Classes_explicit_type_conversions_a;
%~~vis
\begin{xten}
class Poly {
  public val coeff : Rail[Int];
  public def this(coeff: Rail[Int]) { this.coeff = coeff;}
  public static operator (a:Int) as ? = new Poly([a as Int]);
  public static def main(Rail[String]):void {
     val three : Poly = 3 as Poly;
  }
}
\end{xten}
%~~siv
%
%~~neg
\end{ex}
The \xcd`?` may be given a bound, such as \xcd`as ? <: Caster`, if desired.
  

There is little difference between an explicit conversion \xcd`e as T` and a
method call \xcd`e.asT()`.  The explicit conversion does say undeniably what
the result type will be.  However, as described in \Sref{sect:ambig-cast},
sometimes the built-in meaning of \xcd`as` as a cast overrides the
user-defined explicit conversion.  

Explicit casts are most suitable for cases
which resemble the use of explicit casts among the arithmetic types, where, 
for example, \xcd`1.0 as Int` is a way to turn a floating-point number into the
corresponding integer.  
While there is nothing in X10 which
requires it, \xcd`e as T` has the connotation that it gives a good
approximation of \xcd`e` in type \xcd`T`, just as \xcd`1` is a good
(indeed, perfect) approximation of \xcd`1.0` in type \xcd`Int`.  

\subsection{Implicit Type Coercions}
\label{sect:ImplicitCoercion}
\index{type conversion!implicit}

An implicit type conversion from \xcd`U`  to \xcd`T` may be specified in
container \xcd`T`.  
The syntax for it is: 
\begin{xten}
static operator (u:U) : T = e;
\end{xten}
%%OP%% which may be invoked by the unambiguous syntax 
%%OP%% \xcd`T.operator[T](u)` or \xcd`U.operator[T](u)`.
%%OP%% 



Implicit coercions are used automatically by the compiler on method calls 
(\Sref{sect:MethodResolution}) and assignments (\Sref{AssignmentStatement}).
Implicit coercions may be used when a value of type \xcd`T` appears in a
context expecting a value of type \xcd`U`.  If \xcd`T <: U`, no implicit
coercion is needed; \eg, a method \xcd`m` expecting an \xcd`Int` argument may 
be called as \xcd`m(3)`, with an argument of type \xcd`Int{self==3}`, which is
a subtype of \xcd`m`'s argument type \xcd`Int`. 
However, if it is not the case that \xcd`T <: U`, but there is an implicit
coercion from \xcd`T` to \xcd`U` defined in container \xcd`U`, then this
implicit coercion will be applied.

\begin{ex}
We can define an implicit coercion from \xcd`Int` to \xcd`Poly`,
and avoid having to define the sum of an integer and a polynomial
as many special cases.  In the following example, we only define \xcd`+` on
two polynomials.  The
calculation \xcd`1+x` coerces \xcd`1` to a polynomial and uses polynomial
addition to add it to \xcd`x`.

%~~gen ^^^ Classes240
% package Classes.And.Implicit.Coercions;
% class Poly {
%   public val coeff : Rail[Int];
%   public def this(coeff: Rail[Int]) { this.coeff = coeff;}
%   public def degree() = (coeff.size-1) as Int;
%   public def  a(i:Int) = (i<0 || i>this.degree()) ? 0 : coeff(i);
%   public final def toString() = {
%      var allZeroSoFar : Boolean = true;
%      var s : String ="";
%      for( i in 0..this.degree() ) {
%        val ai = this.a(i);
%        if (ai == 0) continue;
%        if (allZeroSoFar) {
%           allZeroSoFar = false;
%           s = term(ai, i);
%        }
%        else 
%           s += 
%              (ai > 0 ? " + " : " - ")
%             +term(ai, i);
%      }
%      if (allZeroSoFar) s = "0";
%      return s;
%   }
%   private final def term(ai: Int, n:Int) = {
%      val xpow = (n==0) ? "" : (n==1) ? "x" : "x^" + n ;
%      return (ai == 1) ? xpow : "" + Math.abs(ai) + xpow;
%   }

%~~vis
\begin{xten}
  public static operator (c : Int) : Poly 
     = new Poly([c as Int]);

  public static operator (p:Poly) + (q:Poly) = new Poly(
      new Rail[Int](
        Math.max(p.coeff.size, q.coeff.size) as Int,
        (i:Int) => p.a(i) + q.a(i)
     ));

  public static def main(Rail[String]):void {
     val x = new Poly([0,1]);
     x10.io.Console.OUT.println("1+x=" + (1+x));
  }
\end{xten}
%~~siv
%}
%~~neg
\end{ex}



\subsection{Assignment and Application Operators}
\index{assignment operator}
\index{application operator}
\index{()}
\index{()=}
\label{set-and-apply}
X10 allows types to implement the subscripting / function application
operator, and indexed assignment.  The \xcd`Array`-like classes take advantage
of both of these in \xcd`a(i) = a(i) + 1`.  

\xcd`a(b,c,d)`
is an operator call, to an operator defined with 
\xcd`public operator this(b:B, c:C, d:D)`.  It may be overloaded.
For
example, an ordered dictionary structure could allow subscripting by numbers
with \xcd`public operator this(i:Int)`, and by strings with 
\xcd`public operator this(s:String)`.  


\xcd`a(i,j)=b` is an \xcd`operator` as well, with zero or more indices
\xcd`i,j`.  It may also be overloaded. 

The update operations \xcd`a(i) += b` 
(for all binary operators in place of \xcd`+`)
are defined to be the same as the
corresponding \xcd`a(i) = a(i) + b`. This applies for all arities of
arguments, and all types, and all binary operations. Of course to use this,
the \xcd`+`, application and assignment \xcd`operator`s must be defined.


\begin{ex}

The \xcd`Oddvec` class of somewhat peculiar vectors illustrates this.

\xcd`a()` returns a string representation of the oddvec, which ordinarily
would 
0be done by \xcd`toString()` instead.  
\xcd`a(i)` sensibly picks out one of the three
coordinates of \xcd`a`.
\xcd`a()=b` sets all the coordinates of \xcd`a` to \xcd`b`.
\xcd`a(i)=b` assigns to one of the
coordinates.  \xcd`a(i,j)=b` assigns different values to \xcd`a(i)` and
\xcd`a(j)`.  

%~~gen ^^^ Classes250
% package Classes.Assignments1_oddvec;
%~~vis
\begin{xten}
class Oddvec {
  var v : Rail[Int] = new Rail[Int](3);
  public operator this () = 
      "(" + v(0) + "," + v(1) + "," + v(2) + ")";
  public operator this () = (newval: Int) { 
    for(p in v.range) v(p) = newval;
  }
  public operator this(i:Int) = v(i);
  public operator this(i:Int, j:Int) = [v(i),v(j)];
  public operator this(i:Int) = (newval:Int) 
      = {v(i) = newval;}
  public operator this(i:Int, j:Int) = (newval:Int) 
      = { v(i) = newval; v(j) = newval+1;} 
  public def example() {
    this(1) = 6;   assert this(1) == 6;
    this(1) += 7;  assert this(1) == 13;
  }
\end{xten}
%~~siv
% }
%  class Hook { def run() {
%     val a = new Oddvec();
%     assert a().equals("(0,0,0)");
%     a() = 1;
%     assert a().equals("(1,1,1)");
%     a(1) = 4;
%     assert a().equals("(1,4,1)");
%     a(0,2) = 5;
%     assert a().equals("(5,4,6)");
%     return true;
%   }
% }
%~~neg

\end{ex}

\section{Class Guards and Invariants}\label{DepType:ClassGuard}
\index{type invariants}
\index{class invariants}
\index{invariant!type}
\index{invariant!class}
\index{guard}


Classes (and structs and interfaces) may specify a {\em class guard}, a
constraint which must hold on all values of the class.    In the following
example, a \xcd`Line` is defined by two distinct \xcd`Pt`s\footnote{We use \xcd`Pt`
to avoid any possible confusion with the built-in class \xcd`Point`.}
%~~gen ^^^ Classes260
% package classes.guards.invariants.glurp;
%~~vis
\begin{xten}
class Pt(x:Int, y:Int){}
class Line(a:Pt, b:Pt){a != b} {}
\end{xten}
%~~siv
%
%~~neg

In most cases the class guard could be phrased as a type constraint on a property of
the class instead, if preferred.  Arguably, a symmetric constraint like two
points being different is better expressed as a class guard, rather than
asymmetrically as a constraint on one type: 
%~~gen ^^^ Classes270
% package classes.guards.invariants.glurp2;
% class Pt(x:Int, y:Int){}
%~~vis
\begin{xten}
class Line(a:Pt, b:Pt{a != b}) {}
\end{xten}
%~~siv
%
%~~neg



\label{DepType:TypeInvariant}
\index{class invariant}
\index{invariant!class}
\index{class!invariant}
\label{DepType:ClassGuardDef}



With every container  or interface \xcd"T" we associate a {\em type
invariant} $\mathit{inv}($\xcd"T"$)$, which describes the guarantees on the
properties of values of type \xcd`T`.  

Every value of \xcd`T` satisfies $\mathit{inv}($\xcd"T"$)$ at all times.  This
is somewhat stronger than the concept of type invariant in most languages
(which only requires that the invariant holds when no method calls are
active).  X10 invariants only concern properties, which are immutable; thus,
once established, they cannot be falsified.

The type
invariant associated with \xcd"x10.lang.Any"
is 
\xcd"true".

The type invariant associated with any interface or struct \xcd"I" that extends
interfaces \xcdmath"I$_1$, $\dots$, I$_k$" and defines properties
\xcdmath"x$_1$: P$_1$, $\dots$, x$_n$: P$_n$" and
specifies a guard \xcd"c" is given by:

\begin{xtenmath}
$\mathit{inv}$(I$_1$) && $\dots$ && $\mathit{inv}$(I$_k$) &&
self.x$_1$ instanceof P$_1$ &&  $\dots$ &&  self.x$_n$ instanceof P$_n$ 
&& c  
\end{xtenmath}

Similarly the type invariant associated with any class \xcd"C" that
implements interfaces \xcdmath"I$_1$, $\dots$, I$_k$",
extends class \xcd"D" and defines properties
\xcdmath"x$_1$: P$_1$, $\dots$, x$_n$: P$_n$" and
specifies a guard \xcd"c" is
given by the same thing with the invariant of the superclass \xcd`D` conjoined:
\begin{xtenmath}
$\mathit{inv}$(I$_1$) && $\dots$ && $\mathit{inv}$(I$_k$) 
&& self.x$_1$ instanceof P$_1$ &&  $\dots$ &&  self.x$_n$ instanceof P$_n$ 
&& c  
&& $\mathit{inv}$(D)
\end{xtenmath}


Note that the type invariant associated with a class entails the type
invariants of each interface that it implements (directly or indirectly), and
the type invariant of each ancestor class.
It is guaranteed that for any variable \xcd"v" of
type \xcd"T{c}" (where \xcd"T" is an interface name or a class name) the only
objects \xcd"o" that may be stored in \xcd"v" are such that \xcd"o" satisfies
$\mathit{inv}(\mbox{\xcd"T"}[\mbox{\xcd"o"}/\mbox{\xcd"this"}])
\wedge \mbox{\xcd"c"}[\mbox{\xcd"o"}/\mbox{\xcd"self"}]$.



\subsection{Invariants for {\tt implements} and {\tt extends} clauses}\label{DepType:Implements}
\label{DepType:Extends}
\index{type-checking!implements clause}
\index{type-checking!extends clause}
\index{implements}
\index{extends}
Consider a class definition
\begin{xtenmath}
$\mbox{\emph{ClassModifiers}}^{\mbox{?}}$
class C(x$_1$: P$_1$, $\dots$, x$_n$: P$_n$){c} extends D{d}
   implements I$_1${c$_1$}, $\dots$, I$_k${c$_k$}
$\mbox{\emph{ClassBody}}$
\end{xtenmath}

These two rules must be satisfied:


\begin{itemize}

\item 
The type invariant \xcdmath"$\mathit{inv}$(C)" of \xcd"C" must entail
\xcdmath"c$_i$[this/self]" for each $i$ in $\{1, \dots, k\}$


\item The return type \xcd"c" of each constructor in a class \xcd`C`
must entail the invariant \xcdmath"$\mathit{inv}$(C)".
\end{itemize}

\subsection{Timing of Invariant Checks}

\index{invariant!checked}

The invariants for a container are checked immediately after the
\xcd`property` statement in the container's constructor. 
This is the earliest that the invariant could possibly be checked. 
Recall that an invariant 
can mention the properties of the container (which are set, forever, at that
point in the code), but cannot mention the \xcd`val`
or \xcd`var` fields (which might not be set at that point), or \xcd`this`
(which might not have been fully initialized).  

If X10 can prove that the invariant always holds given the \xcd`property`
statement and other known information, it may omit the actual check.




\subsection{Invariants and constructor definitions}
\index{invariant!and constructor}
\index{constructor!and invariant}

A constructor for a class \xcd"C" is guaranteed to return an object of the
class on successful termination. This object must satisfy  \xcdmath"$\mathit{inv}$(C)", the
class invariant associated with \xcd"C" (\Sref{DepType:TypeInvariant}).
However,
often the objects returned by a constructor may satisfy {\em stronger}
properties than the class invariant. \Xten{}'s dependent type system
permits these extra properties to be asserted with the constructor in
the form of a constrained type (the ``return type'' of the constructor):

%##(CtorDeclaration
\begin{bbgrammar}
%(FROM #(prod:CtorDecln)#)
           CtorDecln \: Mods\opt \xcd"def" \xcd"this" TypeParams\opt Formals Guard\opt HasResultType\opt CtorBody & (\ref{prod:CtorDecln}) \\
\end{bbgrammar}
%##)

\label{ConstructorGuard}

The parameter list for the constructor
may specify a \emph{guard} that is to be satisfied by the parameters
to the list.

\begin{ex}
%%TODO--rewrite this
Here is another example, constructed as a simplified 
version of \Xcd{x10.array.Region}.  The \xcd`mockUnion` method 
has the type, though not the value, that a true \xcd`union` method would have.

%~~gen ^^^ Classes280
%package Classes.SimplifiedRegion;
%~~vis
\begin{xten}
class MyRegion(rank:Int) {
  static type MyRegion(n:Int)=MyRegion{rank==n};
  def this(r:Int):MyRegion(r) {
    property(r);
  }
  def this(diag:Rail[Int]):MyRegion(diag.size){ 
    property(diag.size);
  }
  def mockUnion(r:MyRegion(rank)):MyRegion(rank) = this;
  def example() {
    val R1 : MyRegion(3) = new MyRegion([4,4,4 as Int]); 
    val R2 : MyRegion(3) = new MyRegion([5,4,1]); 
    val R3 = R1.mockUnion(R2); // inferred type MyRegion(3)
  }
}
\end{xten}
%~~siv
%
%~~neg
The first constructor returns the empty region of rank \Xcd{r}.  The
second constructor takes a \Xcd{Array[Int](1)} of arbitrary length
\Xcd{n} and returns a \Xcd{MyRegion(n)} (intended to represent the set
of points in the rectangular parallelopiped between the origin and the
\Xcd{diag}.)

The code in \xcd`example` typechecks, and \xcd`R3`'s type is inferred as
\xcd`MyRegion(3)`.  


\end{ex}

   Let \xcd"C" be a class with properties
   \xcdmath"p$_1$: P$_1$, $\dots$, p$_n$: P$_n$", and invariant \xcd"c"
   extending the constrained type \xcd"D{d}" (where \xcd"D" is the name of a
   class).



   For every constructor in \xcd"C" the compiler checks that the call to
   super invokes a constructor for \xcd"D" whose return type is strong enough
   to entail \xcd"d". Specifically, if the call to super is of the form 
     \xcdmath"super(e$_1$, $\dots$, e$_k$)"
   and the static type of each expression \xcdmath"e$_i$" is
   \xcdmath"S$_i$", and the invocation
   is statically resolved to a constructor
\xcdmath"def this(x$_1$: T$_1$, $\dots$, x$_k$: T$_k$){c}: D{d$_1$}"
   then it must be the case that 
\begin{xtenmath}
x$_1$: S$_1$, $\dots$, x$_i$: S$_i$ entails x$_i$: T$_i$  (for $i \in \{1, \dots, k\}$)
x$_1$: S$_1$, $\dots$, x$_k$: S$_k$ entails c  
d$_1$[a/self], x$_1$: S$_1$, ..., x$_k$: S$_k$ entails d[a/self]      
\end{xtenmath}
\noindent where \xcd"a" is a constant that does not appear in 
\xcdmath"x$_1$: S$_1$ $\wedge$ ... $\wedge$ x$_k$: S$_k$".

   The compiler checks that every constructor for \xcd"C" ensures that
   the properties \xcdmath"p$_1$,..., p$_n$" are initialized with values which satisfy
   $\mathit{inv}($\xcd"T"$)$, and its own return type \xcd"c'" as follows.  In each constructor, the
   compiler checks that the static types \xcdmath"T$_i$" of the expressions \xcdmath"e$_i$"
   assigned to \xcdmath"p$_i$" are such that the following is
   true:
\begin{xtenmath}
p$_1$: T$_1$, $\dots$, p$_n$: T$_n$ entails $\mathit{inv}($T$)$ $\wedge$ c'     
\end{xtenmath}

(Note that for the assignment of \xcdmath"e$_i$" to \xcdmath"p$_i$"
to be type-correct it must be the
    case that \xcdmath"p$_i$: T$_i$ $\wedge$ p$_i$: P$_i$".) 



The compiler must check that every invocation \xcdmath"C(e$_1$, $\dots$, e$_n$)" to a
constructor is type correct: each argument \xcdmath"e$_i$" must have a static type
that is a subtype of the declared type \xcdmath"T$_i$" for the $i$th
argument of the
constructor, and the conjunction of static types of the argument must
entail the constraint in the parameter list of the constructor.

\section{Generic Classes}

Classes, like other units, can be generic.  They can be parameterized by
types.  The parameter types are used just like ordinary types inside the body
of the generic class -- with a few exceptions.  

\begin{ex}
A \xcd`Colorized[T]` holds a thing of type \xcd`T`, and a string which is intended to represent
its color.  Any type can be used for \xcd`T`; the \xcd`example` method shows
\xcd`Int` and \xcd`Boolean`.  The \xcd`thing()` method retrieves the thing;
note that its return type is the generic type variable \xcd`T`.  X10 is aware 
that \xcd`colInt.thing()` is an \xcd`Int` and \xcd`colTrue.thing()` is a
\xcd`Boolean`, and uses those typings in \xcd`example`. 
%~~gen ^^^ Classes6d9u
% package Classes6d9u;
%~~vis
\begin{xten}
class Colorized[T] {
  private var thing:T; 
  private var color:String; 
  def this(thing:T, color:String) {
     this.thing = thing;
     this.color = color;
  }
  public def thing():T = thing;
  public def color():String = color;  
  public static def example() {
    val colInt  : Colorized[Int] 
                = new Colorized[Int](3, "green");
    assert colInt.thing() == 3 
                && colInt.color().equals("green");
    val colTrue : Colorized[Boolean] 
                = new Colorized[Boolean](true, "blue");
    assert colTrue.thing() 
                && colTrue.color().equals("blue");
  }
}
\end{xten}
%~~siv
%class Hook{ def run() {Colorized.example(); return true;}}
%~~neg


\end{ex}



\subsection{Use of Generics}

An unconstrained type variable \Xcd{X} can be instantiated by any type. All
the operations of \Xcd{Any} are available on a 
variable of type \Xcd{X}. Additionally, variables of type
\Xcd{X} may be used with \Xcd{==, !=}, in \Xcd{instanceof}, and casts.  

If a type variable is constrained, the operations implied by its constraint
are available as well.

\begin{ex}
The interface \xcd`Named` describes entities which know their own name.  The
class \xcd`NameMap[T]` is a specialized map which stores and retrieves
\xcd`Named` entities by name.  The call \xcd`t.name()` in \xcd`put()` is only
valid because the constraint \xcd`{T <: Named}` implies that \xcd`T` is a
subtype of \xcd`Named`, and hence provides all the operations of \xcd`Named`. 
%~~gen ^^^ Types6e6x
% package Types6e6x;
% import x10.util.*;
%~~vis
\begin{xten}
interface Named { def name():String; }
class NameMap[T]{T <: Named} {
   val m = new HashMap[String, T]();
   def put(t:T) { m.put(t.name(), t); }
   def get(s:String):T = m.getOrThrow(s);
}
\end{xten}
%~~siv
%
%~~neg


\end{ex}





\section{Object Initialization}
\label{ObjectInitialization}
\index{initialization}
\index{constructor}
\index{object!constructor}
\index{struct!constructor}

% \noo{Confirm this chapter with the paper}

X10 does object initialization safely.  It avoids certain bad things which
trouble some other languages:
\begin{enumerate}
\item Use of a field before the field has been initialized.
\item A program reading two different values from a \xcd`val` field of a
      container. 
\item \Xcd{this} escaping from a constructor, which can cause problems as
      noted below. 

\end{enumerate}

It should be unsurprising that fields must not be used before they are
initialized. At best, it is uncertain what value will be in them, as in
\Xcd{x} below. Worse, the value might not even be an allowable value; \Xcd{y},
declared to be nonzero in the following example, might be zero before it is
initialized.
\begin{xten}
// Not correct X10
class ThisIsWrong {
  val x : Int;
  val y : Int{y != 0};
  def this() {
    x10.io.Console.OUT.println("x=" + x + "; y=" + y);
    x = 1; y = 2;
  }
}
\end{xten}

One particularly insidious way to read uninitialized fields is to allow
\Xcd{this} to escape from a constructor. For example, the constructor could
put \Xcd{this} into a data structure before initializing it, and another
activity could read it from the data structure and look at its fields:
\begin{xten}
class Wrong {
  val shouldBe8 : Int;
  static Cell[Wrong] wrongCell = new Cell[Wrong]();
  static def doItWrong() {
     finish {
       async { new Wrong(); } // (A)
       assert( wrongCell().shouldBe8 == 8); // (B)
     }
  }
  def this() {
     wrongCell.set(this); // (C) - ILLEGAL
     this.shouldBe8 = 8; // (D)
  }
}
\end{xten}
\noindent
In this example, the underconstructed \Xcd{Wrong} object is leaked into a
storage cell at line \Xcd{(C)}, and then initialized.  The \Xcd{doItWrong}
method constructs a new \Xcd{Wrong} object, and looks at the \Xcd{Wrong}
object in the storage cell to check on its \Xcd{shouldBe8} field.  One
possible order of events is the following:
\begin{enumerate}
\item \Xcd{doItWrong()} is called.
\item \Xcd{(A)} is started.  Space for a new \Xcd{Wrong} object is allocated.
      Its \Xcd{shouldBe8} field, not yet initialized, contains some garbage
      value.
\item \Xcd{(C)} is executed, as part of the process of constructing a new
      \Xcd{Wrong} object.  The new, uninitialized object is stored in
      \Xcd{wrongCell}.
\item Now, the initialization activity is paused, and execution of the main activity
      proceeds from \Xcd{(B)}.
\item The value in \Xcd{wrongCell} is retrieved, and is \Xcd{shouldBe8} field
      is read.  This field contains garbage, and the assertion fails.
\item Now let the initialization activity proceed with \Xcd{(D)},
      initializing \Xcd{shouldBe8} --- too late.
\end{enumerate}

The \xcd`at` statement (\Sref{AtStatement}) introduces the potential for
escape as well. The following class prints an uninitialized value: 
%~~gen ^^^ ThisEscapingViaAt_MustFailCompile
% package ObjInit_at;
% NOCOMPILE
%~~vis
\begin{xten}
// This code violates this chapter's constraints
// and thus will not compile in X10.
class Example {
  val a: Int;
  def this() { 
    at(here.next()) {
      // Recall that 'this' is a copy of 'this' outside 'at'.
      Console.OUT.println("this.a = " + this.a);
    }
    this.a = 1;
  }
}
\end{xten}
%~~siv
%
%~~neg


X10 must protect against such possibilities.  The rules explaining how
constructors can be written are somewhat intricate; they are designed to allow
as much programming as possible without leading to potential problems.
Ultimately, they simply are elaborations of the fundamental principles that
uninitialized fields must never be read, and \Xcd{this} must never be leaked.

%%RAW%% \subsection{Raw and Cooked Objects}
%%RAW%% \index{raw}
%%RAW%% \index{cooked}
%%RAW%% 
%%RAW%% An object is {\em raw} before its constructor ends, and {\em cooked} after its
%%RAW%% constructor ends. Note that, when an object is cooked, all its subobjects are
%%RAW%% cooked.  
%%RAW%% 



\subsection{Constructors and Non-Escaping Methods}
\index{non-escaping}
\label{sect:nonescaping}

In general, constructors must not be allowed to call methods with \Xcd{this} as
an argument or receiver. Such calls could leak references to \Xcd{this},
either directly from a call to \Xcd{cell.set(this)}, or indirectly because
\Xcd{toString} leaks \Xcd{this}, and the concatenation
\Xcd`"Escaper = "+this` calls \Xcd{toString}.\footnote{This is abominable behavior for
\Xcd{toString}, but it cannot be prevented -- save by a scheme such as we
present in this section.}
%~WRONG~gen
%package ObjectInit.CtorAndNonEscaping.One;
%~WRONG~vis
\begin{xten}
// This code violates this chapter's constraints
// and thus will not compile in X10.
class Escaper {
  static val Cell[Escaper] cell = new Cell[Escaper]();
  def toString() {
    cell.set(this);
    return "Evil!";
  }
  def this() {
    cell.set(this);
    x10.io.Console.OUT.println("Escaper = " + this);
  }
}
\end{xten}
%~WRONG~siv
%
%~WRONG~neg

However, it is convenient to be able to call methods from constructors; {\em
e.g.}, a class might have eleven constructors whose common behavior is best
described by three methods.
Under certain stringent conditions, it {\em is}
safe to call a method: the method called must not leak references to
\Xcd{this}, and must not read \Xcd{val}s or \Xcd{var}s which might not have
been assigned.

So, X10 performs a static dataflow analysis, sufficient to guarantee that
method calls in constructors are safe.  This analysis requires having access
to or guarantees about all the code that could possibly be called.  This can
be accomplished in two ways:
\begin{enumerate}
\item Ensuring that only code from the class itself can be called, by
      forbidding overriding of
      methods called from the constructor: they can be marked \Xcd{final} or
      \Xcd{private}, or the whole class can be \Xcd{final}.
\item Marking the methods called from the constructor by
      \xcd`@NonEscaping` or \xcd`@NoThisAccess`
\end{enumerate}

\subsubsection{Non-Escaping Methods}
\index{method!non-escaping}
\index{method!implicitly non-escaping}
\index{method!NonEscaping}
\index{implicitly non-escaping}
\index{non-escaping}
\index{non-escaping!implicitly}
\index{NonEscaping}


A method may be annotated with \xcd`@NonEscaping`.  This
imposes several restrictions on the method body, and on all methods overriding
it.  However, it is the only way that a method can be called from
constructors.  The
\Xcd{@NonEscaping} annotation makes explicit all the X10 compiler's needs for
constructor-safety.

A method can, however, be safe to call from constructors without being marked
\Xcd{@NonEscaping}. We call such methods {\em implicitly non-escaping}.
Implicitly non-escaping methods need to obey the same constraints on
\Xcd{this}, \Xcd{super}, and variable usage as \Xcd{@NonEscaping} methods. An
implicitly non-escaping method {\em could} be marked as
\xcd`@NonEscaping`; the compiler, in
effect, infers the annotation. In addition, all non-escaping methods
must be \Xcd{private} or \Xcd{final} or members of a \Xcd{final} class; this
corresponds to the hereditary nature of \Xcd{@NonEscaping} (by forbidding
inheritance of implicitly non-escaping methods).

We say that a method is {\em non-escaping} if it is either implicitly
non-escaping, or annotated \Xcd{@NonEscaping}.

The first requirement on non-escaping methods is that they do not allow
\Xcd{this} to escape. Inside of their bodies, \Xcd{this} and \Xcd{super} may
only be used for field access and assignment, and as the receiver of
non-escaping methods.


The following example uses the possible variations.  \Xcd{aplomb()} 
explicitly forbids reading any field but
\Xcd{a}. \Xcd{boric()} is called after \Xcd{a} and \Xcd{b} are set, but
\Xcd{c} is not.
The \xcd`@NonEscaping` annotation on \xcd`boric()` is optional, but the
compiler will print a warning if it is left out.
\Xcd{cajoled()} is only called after all fields are set, so it
can read anything; its annotation, too, is not required.   \Xcd{SeeAlso} is able to override \Xcd{aplomb()}, because
\Xcd{aplomb()} is \xcd`@NonEscaping`; it cannot override the final method
\Xcd{boric()} or the private one \Xcd{cajoled()}.  
%~~gen ^^^ ObjectInitialization10
%package ObjInit.C2;
%~~vis
\begin{xten}
import x10.compiler.*;

final class C2 {
  protected val a:Int; protected val b:Int; protected val c:Int;
  protected var x:Int; protected var y:Int; protected var z:Int;
  def this() {
    a = 1;
    this.aplomb();
    b = 2;
    this.boric();
    c = 3;
    this.cajoled();
  }
  @NonEscaping def aplomb() {
    x = a;
    // this.boric(); // not allowed; boric reads b.
    // z = b; // not allowed -- only 'a' can be read here
  }
  @NonEscaping final def boric() {
    y = b;
    this.aplomb(); // allowed; 
       // a is definitely set before boric is called
    // z = c; // not allowed; c is not definitely written
  }
  @NonEscaping private def cajoled() {
    z = c;
  }
}

\end{xten}
%~~siv
%
%~~neg

\subsubsection{NoThisAccess Methods}

A method may be annotated \xcd`@NoThisAccess`.  \xcd`@NoThisAccess` methods
may be called from constructors, and they may be overridden in subclasses.
However, they may not refer to \xcd`this` in any way -- in particular, they
cannot refer to fields of \xcd`this`, nor to \xcd`super`.

\begin{ex}

The class \xcd`IDed` has an \xcd`Float`-valued \xcd`id` field.  The method
\xcd`count()` is used to initialize the \xcd`id`.  For \xcd`IDed` objects,
the \xcd`id` is the count of \xcd`IDed`s created with the same parity of its
\xcd`kind`.   Note that \xcd`count()` does not refer to \xcd`this`, though
it does refer to a \xcd`static` field \xcd`counts`. 

The subclass \xcd`SubIDed` has \xcd`id`s that depend on \xcd`kind%3`
as well as the parity of \xcd`kind`.  It overrides the \xcd`count()`
method.  The body of \xcd`count()` still cannot refer to \xcd`this`.
Nor can it refer to \xcd`super` (which is \xcd`self` under another name).
This precludes the use of a \xcd`super` call.  This is why we have separated
the body of \xcd`count` out as the static method \xcd`kind2count` -- without
that, we would have had to duplicate its body, as we could not call 
\xcd`super.count(kind)` in a \xcd`NoThisAccess` method, as is shown by 
the \xcd`ERROR` line \xcd`(A)`. 

Note that \xcd`NoThisAccess` is in \xcd`x10.compiler` and must be imported,
and that the overriding method \xcd`SubIDed.count` must be declared
\xcd`@NoThisAccess` as well as the overridden method.
Line \xcd`(B)` is not allowed because \xcd`code` is a field of \xcd`this`, 
and field accesses are forbidden.   Line \xcd`(C)` references \xcd`this`
directly, which, of course, is forbidden by \xcd`@NoThisAccess`.  


%~~gen ^^^ ObjectInitialization7p2v
% package ObjectInitialization7p2v;
%~~vis
\begin{xten}
import x10.compiler.*;
class UseNoThisAccess {
  static class IDed {
    protected static val counts = [0 as Int,0];
    protected var code : Int;
    val id: Float;
    public def this(kind:Int) { 
      code = kind;
      this.id = this.count(kind); 
    }
    protected static def kind2count(kind:Int) = ++counts(kind % 2);
    @NoThisAccess def count(kind:Int) : Float = kind2count(kind);
  }
  static class SubIDed extends IDed {
    protected static val subcounts = [0 as Int, 0, 0];
    public static val all = new x10.util.ArrayList[SubIDed]();
    public def this(kind:Int) { 
       super(kind); 
    }
    @NoThisAccess
    def count(kind:Int) : Float {
       val subcount <: Int = ++subcounts(kind % 3);
       val supercount <: Float = kind2count(kind);
       //ERROR: val badSuperCount = super.count(kind); //(A)
       //ERROR: code = kind;                           //(B)
       //ERROR: all.add(this);                         //(C)
       return  supercount + 1.0f / subcount;
    }
  }
}
\end{xten}
%~~siv
%
%~~neg


\end{ex}

\subsection{Fine Structure of Constructors}
\label{SFineStructCtors}

The code of a constructor consists of four segments, three of them optional
and one of them implicit.
\begin{enumerate}
\item The first segment is an optional call to \Xcd{this(...)} or
      \Xcd{super(...)}.  If this is supplied, it must be the first statement
      of the constructor.  If it is not supplied, the compiler treats it as a
      nullary super-call \Xcd{super()};
\item If the class or struct has properties, there must be a single
      \Xcd{property(...)} command in the constructor, or a \xcd`this(...)`
      constructor call.  Every execution path
      through the constructor must go through this \Xcd{property(...)} command
      precisely once.   The second segment of the constructor is the code
      following the first segment, up to and including the \Xcd{property()}
      statement.

      If the class or struct has no properties, the \Xcd{property()} call must
      be omitted. If it is present, the second segment is defined as before.  If
      it is absent, the second segment is empty.
\item The third segment is automatically generated.  Fields with initializers
      are initialized immediately after the \Xcd{property} statement.
      In the following example, \Xcd{b} is initialized to \Xcd{y*9000} in
      segment three.  The initialization makes sense and does the right
      thing; \Xcd{b} will be \Xcd{y*9000} for every \Xcd{Overdone} object.
      (This would not be possible if field initializers were processed
      earlier, before properties were set.)
\item The fourth segment is the remainder of the constructor body.
\end{enumerate}

The segments in the following code are shown in the comments.
%~~gen ^^^ ObjectInitialization20
% package ObjectInitialization.ShowingSegments;
%~~vis
\begin{xten}
class Overlord(x:Int) {
  def this(x:Int) { property(x); }
}//Overlord
class Overdone(y:Int) extends Overlord  {
  val a : Int;
  val b =  y * 9000;
  def this(r:Int) {
    super(r);                      // (1)
    x10.io.Console.OUT.println(r); // (2)
    val rp1 = r+1;
    property(rp1);                 // (2)
    // field initializations here  // (3)
    a = r + 2 + b;                 // (4)
  }
  def this() {
    this(10);                      // (1), (2), (3)
    val x = a + b;                 // (4)
  }
}//Overdone
\end{xten}
%~~siv
%
%~~neg

The rules of what is allowed in the three segments are different, though
unsurprising.  For example, properties of the current class can only be read
in segment 3 or 4---naturally, because they are set at the end of segment 2.

\subsubsection{Initialization and Inner Classses}
\index{constructor!inner classes in}

Constructors of inner classes are tantamount to method calls on \Xcd{this}.
For example, the constructor for Inner {\bf is} acceptable.  It does not leak
\Xcd{this}.  It leaks \Xcd{Outer.this}, which is an utterly different object.
So, the call to \Xcd{this.new Inner()} in the \Xcd{Outer} constructor {\em
is} illegal.  It would leak \Xcd{this}.  There is no special rule in effect
preventing this; a constructor call of an inner class is no
different from a method as far as leaking is concerned.
%~~gen ^^^ ObjectInitialization30
% package ObjInit.InnerClass; 
% // NOTEST-packaging-issue
%~~vis
\begin{xten}
class Outer {
  static val leak : Cell[Outer] = new Cell[Outer](null);
  class Inner {
     def this() {Outer.leak.set(Outer.this);}
  }
  def /*Outer*/this() {
     //ERROR: val inner = this.new Inner();
  }
}
\end{xten}
%~~siv
%
%~~neg



\subsubsection{Initialization and Closures}
\index{constructor!closure in}

Closures in constructors may not refer to \xcd`this`.  They may not even refer
to fields of \xcd`this` that have been initialized.   For example, the
closure \xcd`bad1` is not allowed because it refers to \xcd`this`; \xcd`bad2`
is not allowed because it mentions \xcd`a` --- which is, of course, identical
to \xcd`this.a`. 

%%-deleted-%% valid if they were invoked (or inlined) at the
%%-deleted-%%place of creation. For example, \Xcd{closure} below is acceptable because it
%%-deleted-%%only refers to fields defined at the point it was written.  \Xcd{badClosure}
%%-deleted-%%would not be acceptable, because it refers to fields that were not defined at
%%-deleted-%%that point, although they were defined later.
%~~gen ^^^ ObjectInitialization40
% package ObjectInitialization.Closures; 
%~~vis
\begin{xten}
class C {
  val a:Int;
  def this() {
    this.a = 1;
    //ERROR: val bad1 = () => this; 
    //ERROR: val bad2 = () => a*10;
  }
}
\end{xten}
%~~siv
%
%~~neg


\subsection{Definite Initialization in Constructors}


An instance field \Xcd{var x:T}, when \Xcd{T} has a default value, need not be
explicitly initialized.  In this case, \Xcd{x} will be initialized to the
default value of type \Xcd{T}.  For example, a \Xcd{Score} object will have
its \Xcd{currently} field initialized to zero, below:
%~~gen ^^^ ObjectInitialization50
% package ObjectInit.DefaultInit;
%~~vis
\begin{xten}
class Score {
  public var currently : Int;
}
\end{xten}
%~~siv
%
%~~neg

All other sorts of instance fields do need to be initialized before they can
be used.  \Xcd{val} fields must be initialized in the constructor, even if
their type has a 
default value.  It would be silly to have a field \Xcd{val z : Int} that was
always given default value of \Xcd{0} and, since it is \Xcd{val}, can never be
changed.  \Xcd{var} fields whose type has no default value must be initialized
as well, such as \xcd`var y : Int{y != 0}`, since it cannot be assigned a
sensible initial value.

The fundamental principles are:
\begin{enumerate}
\item \Xcd{val} fields must be assigned precisely once in each constructor on every
possible execution path.
\item \Xcd{var} fields of defaultless type must be
assigned at least once on every possible execution path, but may be assigned
more than once.
\item No variable may be read before it is guaranteed to have been
assigned.
\item Initialization may be by field initialization expressions (\Xcd{val x :
      Int = y+z}), or by uninitialized fields \Xcd{val x : Int;} plus
an initializing assignment \Xcd{x = y+z}.  Recall that field initialization
expressions are performed after the \Xcd{property} statement, in segment 3 in
the terminology of \Sref{SFineStructCtors}.
\end{enumerate}



\subsection{Summary of Restrictions on Classes and Constructors}

The following table tells whether a given feature is (yes), is not (no) or is
with some conditions (note) allowed in a given context.   For example, a
property method is allowed with the type of another property, as long as it
only mentions the preceding properties. The first column of the table gives
examples, by line of the following code body.

\begin{tabular}{||l|l|c|c|c|c|c|c||}
\hline
~
  & {\bf Example}
  & {\bf Prop.}
  & {\bf {\tt \small self==this}(1)}
  & {\bf Prop.Meth.}
  & {\bf {\tt this}}
  & {\bf {fields}}
\\\hline
Type of property
  & (A)
  & %?properties
    yes (2)
  & no %? self==this
  & no %? property methods
  & no %? this may be used
  & no %? fields may be used
\\\hline
Class Invariant
  & (B)
  & yes %?properties
  & yes %? self==this
  & yes %? property methods
  & yes %? this may be used
  & no %? fields may be used
\\\hline
Supertype (3)
  & (C), (D)
  & yes%?properties
  & yes%? self==this
  & yes%? property methods
  & no%? this may be used
  & no%? fields may be used
\\\hline
Property Method Body
  & (E)
  & yes %?properties
  & yes %? self==this
  & yes %? property methods
  & yes %? this may be used
  & no %? fields may be used
\\\hline

Static field (4)
  & (F) (G)
  & no %?properties
  & no %? self==this
  & no %? property methods
  & no %? this may be used
  & no %? fields may be used
\\\hline

Instance field (5)
  & (H), (I)
  & yes %?properties
  & yes %? self==this
  & yes %? property methods
  & yes %? this may be used
  & yes %? fields may be used
\\\hline

Constructor arg. type
  & (J)
  & no %?properties
  & no %? self==this
  & no  %? property methods
  & no %? this may be used
  & no %? fields may be used
\\\hline

Constructor guard
  & (K)
  & no %?properties
  & no %? self==this
  & no %? property methods
  & no %? this may be used
  & no %? fields may be used
\\\hline

Constructor ret. type
  & (L)
  & yes %?properties
  & yes %? self==this
  & yes %? property methods
  & yes %? this may be used
  & yes %? fields may be used
\\\hline

Constructor segment 1
  & (M)
  & no%?properties
  & yes%? self==this
  & no%? property methods
  & no%? this may be used
  & no%? fields may be used
\\\hline


Constructor segment 2
  & (N)
  & no%?properties
  & yes%? self==this
  & no%? property methods
  & no%? this may be used
  & no%? fields may be used
\\\hline

Constructor segment 4
  & (O)
  & yes%?properties
  & yes%? self==this
  & yes%? property methods
  & yes%? this may be used
  & yes%? fields may be used
\\\hline

Methods
  & (P)
  & yes %?properties
  & yes %? self==this
  & yes %? property methods
  & yes %? this may be used
  & yes %? fields may be used
\\\hline



\iffalse
place
  & (pos)
  & %?properties
  & %? self==this
  & %? property methods
  & %? this may be used
  & %? fields may be used
\\\hline
\fi
\end{tabular}

Details:

\begin{itemize}
\item (1) {Top-level {\tt self} only.}
\item (2) {The type of the {$i^{th}$} property may only mention
                 properties {$1$} through {$i$}.}
\item (3) Super-interfaces follow the same rules as supertypes.
\item (4) The same rules apply to types and initializers.
\end{itemize}



The example indices refer to the following code:
%~~gen ^^^ ObjectInitialization60
% package ObjectInit.GrandExample;
% class Supertype[T]{}
% interface SuperInterface[T]{}
%~~vis
\begin{xten}
class Example (
   prop : Int,
   proq : Int{prop != proq},                    // (A)
   pror : Int
   )
   {prop != 0}                                  // (B)
   extends Supertype[Int{self != prop}]         // (C)
   implements SuperInterface[Int{self != prop}] // (D)
{
   property def propmeth() = (prop == pror);    // (E)
   static staticField
      : Cell[Int{self != 0}]                    // (F)
      = new Cell[Int{self != 0}](1);            // (G)
   var instanceField
      : Int {self != prop}                      // (H)
      = (prop + 1) as Int{self != prop};        // (I)
   def this(
      a : Int{a != 0},
      b : Int{b != a}                           // (J)
      )
      {a != b}                                  // (K)
      : Example{self.prop == a && self.proq==b} // (L)
   {
      super();                                  // (M)
      property(a,b,a);                          // (N)
      // fields initialized here
      instanceField = b as Int{self != prop};   // (O)
   }

   def someMethod() =
        prop + staticField() + instanceField;   // (P)
}
\end{xten}
%~~siv
%
%~~neg

\section{Method Resolution}
\index{method!resolution}
\index{method!which one will get called}
\label{sect:MethodResolution}

Method resolution is the problem of determining, statically, which method (or
constructor or operator)
should be invoked, when there are several choices that could be invoked.  For
example, the following class has two overloaded \xcd`zap` methods, one taking
an \Xcd{Any}, and the other a \Xcd{Resolve}.  Method resolution will figure
out that the call \Xcd{zap(1..4)} should call \xcd`zap(Any)`, and
\Xcd{zap(new Resolve())} should call \xcd`zap(Resolve)`.  

\begin{ex}
%~~gen ^^^ MethodResolution10
%package MethodResolution.yousayyouwantaresolution;
% // This depends on https://jira.codehaus.org/browse/XTENLANG-2696
%~~vis
\begin{xten}
class Res {
  public static interface Surface {}
  public static interface Deface {}

  public static class Ace implements Surface {
    public static operator (Boolean) : Ace = new Ace();
    public static operator (Place) : Ace = new Ace();
  }
  public static class Face implements Surface, Deface{}

  public static class A {}
  public static class B extends A {}
  public static class C extends B {}

  def m(x:A) = 0;
  def m(x:Int) = 1;
  def m(x:Boolean) = 2;
  def m(x:Surface) = 3;
  def m(x:Deface) = 4; 

  def example() {
     assert m(100) == 1 : "Int"; 
     assert m(new C()) == 0 : "C";
     // An Ace is a Surface, unambiguous best choice
     assert m(new Ace()) == 3 : "Ace";
     // ERROR: m(new Face());

     // The match must be exact.
     // ERROR: assert m(here) == 3 : "Place";

     // Boolean could be handled directly, or by 
     // implicit coercion Boolean -> Ace.
     // Direct matches always win.
     assert m(true) == 2 : "Boolean"; 
  }
\end{xten}
%~~siv
%  public static def main(argv:Rail[String]) {(new Res()).example(); Console.OUT.println("That's all!");}
% public def claim() { val ace : Ace = here; assert m(ace)==3; }
% }
% class Hook{ def run(){ (new Res()).example(); return true;} }
%~~neg

In the \xcd`"Int"` line, there is a very close match.  \xcd`100` is an
\xcd`Int`.  In fact, \xcd`100` is an \xcd`Int{self==100}`, so even in this
case the type of the actual parameter is not {\em precisely} equal to the type
of the method.

In the \xcd`"C"` line of the example, \xcd`new C()` is an instance of \xcd`C`,
which is a subtype of \xcd`A`, so the \xcd`A` method applies.  No other method
does, and so the \xcd`A` method will be invoked.

Similarly, in the \xcd`"Ace"` line, the \xcd`Ace` class implements
\xcd`Surface`, and so \xcd`new Ace()` matches the \xcd`Surface` method. 

However, a \xcd`Face` is both a \xcd`Surface` and a \xcd`Deface`, so there is
no unique best match for the invocation \xcd`m(new Face())`.  This invocation
would be forbidden, and a compile-time error issued.


The match must be exact.  There is an implicit coercion 
from \xcd`Place` to \xcd`Ace`, and \xcd`Ace` implements \xcd`Surface`, so the
code
\begin{xten}
val ace : Ace = here;
assert m(ace) == 3;
\end{xten}
works, by using the \xcd`Surface` form of \xcd`m`.  But doing it in one step
requires a deeper search than X10 performs\footnote{In general this search is
unbounded, so X10 can't perform it.}, and is not allowed.


For \xcd`m(true)`, both the \xcd`Boolean` and, with the implicit coercion,
\xcd`Ace` methods could apply.  Since the \xcd`Boolean` method applies
directly, and the \xcd`Ace` method requires an implicit coercion, this call
resolves to the \xcd`Boolean` method, without an error.

\end{ex}


The basic concept of method resolution is:
\begin{enumerate}
\item List all the methods that could possibly be used, inferring generic
      types but not performing implicit coercions.    
\item If one possible method is more specific than all the others, that one 
      is the desired method.
\item If there are two or more methods neither of which is more specific than
      the others, then the method invocation is ambiguous.  Method resolution
      fails and reports an error.
\item Otherwise, no possible methods were found without implicit coercions.
      Try the preceding steps again, but with coercions allowed: zero or one
      implicit coercion for each argument.  If a single
      most specific method is found with coercions, it is the desired method.
      If there are several, the invocation is ambiguous and erronious.
\item If no methods were found even with coercions, then the method invocation
      is undetermined.  Method resolution fails and reports an error.
\end{enumerate}

After method resolution is done, there is a validation phase that checks the
legality of the call, based on the \xcd`STATIC_CHECKS` compiler flag.  
With \xcd`STATIC_CHECKS`, the method's constraints must be satisfied; that is,
they must be entailed (\Sref{SemanticsOfConstraints}) by the information in
force at the point of the call.  With \xcd`DYNAMIC_CHECKS`, if the constraint
is not entailed at that point, a dynamic check is inserted to make sure that
it is true at runtime.

\noindent
In the presence of X10's highly-detailed type system, some subtleties arise. 
One point, at least, is {\em not} subtle. The same procedure is used, {\em
mutatis mutandis} for method, constructor, and operator resolution.  



\subsection{Space of Methods}

X10 allows some constructs, particularly \xcd`operator`s, to be defined in a
number of ways, and invoked in a number of ways. This section specifies which
forms of definition could correspond to a given definiendum.
%%OP%% , and (redundantly)
%%OP%% the syntax for invoking that definition unambiguously.  

Method invocations \xcd`a.m(b)`, where \xcd`a` is an expression, can be either
of the following forms.  There may be any number of arguments.
\begin{itemize}
\item An instance method on \xcd`a`, of the form \xcd`def m(B)`.
%%OP%% , so that the   invocation is \xcd`a.m(b)`;
\item A static method on \xcd`a`'s class, of the form \xcd`static def m(B)`.
%%OP%%       so that the invocation is \xcd`A.m(b)`.
\end{itemize}

The meaning of an invocation of the form \xcd`m(b)`, with any number of
arguments, depends slightly on its context.  Inside of a constraint, it might
mean \xcd`self.m(b)`.  Outside of a constraint, there is no \xcd`self`
defined, so it can't mean that.  The first of these that applies will be
chosen. 
\begin{enumerate}
\item Invoke a method on \xcd`this`, \viz{} \xcd`this.m(b)`.  Inside a
      constraint, it may also invoke a property method on \xcd`self`, \viz.
      \xcd`self.m(b)`.  It is an error if both \xcd`this.m(b)` and
      \xcd`self.m(b)` are possible.
\item Invoke a function named \xcd`m` in a local or field.
\item Construct a structure named \xcd`m`.
\end{enumerate}

Static method invocations, \xcd`A.m(b)`, where \xcd`A` is a container name,
can only be static.  There may be any number of arguments.
\begin{itemize}
\item A static method on \xcd`A`, of the form \xcd`static def m(B)`.
%%OP%%       the invocation is \xcd`A.m(b)`; 
\end{itemize}


Constructor invocations, \xcd`new A(b)`, must invoke constructors. There may
be any number of arguments. 
\begin{itemize}
\item A constructor on \xcd`A`, of the form \xcd`def this(B)`.
%%OP%% , so that the
%%OP%%       invocation is \xcd`new A(b)`.
\end{itemize}


A unary operator \xcdmath"$\star$ a" may be defined as: 
\begin{itemize}
\item An instance operator on \xcd`A`, defined as 
      \xcdmath"operator $\star$ this()".
%%OP%%       so that the invocation is 
%%OP%%       \xcdmath"a.operator $\star$()"; or
\item A static operator on \xcd`A`, defined as 
      \xcdmath"operator $\star$(a:A)".
%%OP%%       so that the invocation is 
%%OP%%       \xcdmath"A.operator $\star$(a)"
\end{itemize}

A binary operator \xcdmath"a $\star$ b" may be defined as: 
\begin{itemize}
\item An instance operator on \xcd`A`, defined as 
      \xcdmath"operator this $\star$(b:B)";
%%OP%%       so that the invocation is \xcdmath"a.operator $\star$(b)", 
or
\item A right-hand operator on \xcd`B`, defined as
      \xcdmath"operator (a:A) $\star$ this"; or
%%OP%%       so that the invocation is \xcdmath"b.operator ()$\star$(b)"

\item A static operator on \xcd`A`, defined as
      \xcdmath"operator (a:A) $\star$ (b:B)", 
%%OP%%       so that the invocation is \xcdmath"A.operator $\star$(a,b)"
; or
\item A static operator on \xcd`B`, if \xcd`A` and \xcd`B` are different
      classes, defined as
      \xcdmath"operator (a:A) $\star$ (b:B)"
%%OP%% , so that the invocation is 
%%OP%%       \xcdmath"B.operator $\star$(a,b)".
\end{itemize}
\noindent
If none of those resolve to a method, then either operand may be implicitly
coerced to the
other.  If one of the following two situations obtains, it will be done; if
both, the expression causes a static error.
\begin{itemize}
\item An implicit coercion from \xcd`A` to \xcd`B`, and 
      an operator \xcdmath"B $\star$ B" can be used, by 
      coercing \xcd`a` to be of type \xcd`B`, and then using \xcd`B`'s
      $\star$.  
\item An implicit coercion from \xcd`B` to \xcd`A`, and 
      an operator \xcdmath"A $\star$ A" can be used,
      coercing \xcd`b` to be of type \xcd`A`, and then using \xcd`A`'s
      $\star$.  
\end{itemize}

An application \xcd`a(b)`, for any number of arguments, can come from a number
of things. 
\begin{itemize}
\item an application operator on \xcd`a`, defined as \xcd`operator this(b:B)`;
%%OP%% , so that the 
%%OP%% invocation is \xcd`a.operator()(b)`
\item If \xcd`a` is an identifier, \xcd`a(b)` can also be a method invocation
      equivalent to \xcd`this.a(b)`, which  invokes \xcd`a` as
      either an instance or static method on \xcd`this`
\item If \xcd`a` is a qualified identifier, \xcd`a(b)` can also be an
      invocation of a struct constructor.
\end{itemize}


An indexed assignment, \xcd`a(b)=c`, for any number of \xcd`b`'s, can only
come from an indexed assignment definition: 
\begin{itemize}
\item \xcd`operator this(b:B)=(c:C) {...}`
%%OP%%       so that the invocation is \xcd`a.operator()=(b,c)`.
\end{itemize}

An implicit coercion, in 
which a value \xcd`a:A` is used in a context which requires a value of some
other non-subtype \xcd`B`, 
can only come from implicit coercion operation defined on
\xcd`B`: 
\begin{itemize}
\item an implicit coercion in \xcd`B`:
      \xcd`static operator (a:A):B`;
%%OP%%       so that the coercion is \xcd`B.operator[B](a)`;
\end{itemize}

An explicit conversion \xcd`a as B` can come from an explicit conversion
operator, or an implicit coercion operator.  X10 tries two things, in order,
only checking 2 if 1 fails: 
\begin{enumerate}
\item An \xcd`as` operator in \xcd`B`: 
      \xcdmath"static operator (a:A) as ?";
%%OP%%       so that the conversion is \xcd`B.operator as[B](a)`

\item or, failing that, an implicit coercion in \xcd`B`:
      \xcd`static operator (a:A):B`.
%%OP%% , so that the conversion is 
%%OP%%       \xcd`B.operator[B](a)`;

\end{enumerate}



\subsection{Possible Methods}

This section describes what it means for a method to be a {\em possible}
resolution of a method invocation.  



Generics introduce several subtleties, especially with the inference of
generic types. 
For the purposes of method resolution, all that matters about a method,
constructor, or operator \xcd`M` --- we use the word ``method'' to include all
three choices for this section --- is its signature, plus which method it is.
So, a typical \xcd`M` might look like 
\xcdmath"def m[G$_1$,$\ldots$, G$_g$](x$_1$:T$_1$,$\ldots$, x$_f$:T$_f$){c} =...".  The code body \xcd`...` is irrelevant for the purpose of whether a
given method call means \xcd`M` or not, so we ignore it for this section.

All that matters about a method definition, for the purposes of method
resolution, is: 
\begin{enumerate}
\item The method name \xcd`m`;
\item The generic type parameters of the method \xcd`m`,  \xcdmath"G$_1$,$\ldots$, G$_g$".  If there
      are no generic type parameters, {$g=0$}.  
\item The types \xcdmath"x$_1$:T$_1$,$\ldots$, x$_f$:T$_f$" of the formal parameters.  If
      there are no formal parameters, {$f=0$}. In the case of an instance
      method, the receiver will be the first formal parameter.\footnote{The
      variable names are relevant because one formal can be mentioned in a
      later type, or even a constraint: {\tt def f(a:Int, b:Point\{rank==a\})=...}.}
\item A {\em unique identifier} \xcd`id`, sufficient to tell the compiler
      which method body is intended.  A file name and position in that file
      would suffice.  The details of the identifier are not relevant.
\end{enumerate}

For the purposes of understanding method resolution, we assume that all the
actual parameters of an invocation are simply variables: \xcd`x1.meth(x2,x3)`.
This is done routinely by the compiler in any case; the code 
\xcd`tbl(i).meth(true, a+1)` would be treated roughly as 
\begin{xten}
val x1 = tbl(i);
val x2 = true;
val x3 = a+1;
x1.meth(x2,x3);
\end{xten}

All that matters about an invocation \xcd`I` is: 
\begin{enumerate}
\item The method name \xcdmath"m$'$";
\item The generic type parameters \xcdmath"G$'_1$,$\ldots$, G$'_g$".  If there
      are no generic type parameters, {$g=0$}.  
\item The names and types \xcdmath"x$_1$:T$'_1$,$\ldots$, x$_f$:T$'_f$" of the
      actual parameters.
      If
      there are no actual parameters, {$f=0$}. In the case of an instance
      method, the receiver is the first actual parameter.
\end{enumerate}

The signature of the method resolution procedure is: 
\xcd`resolve(invo : Invocation, context: Set[Method]) : MethodID`.  
Given a particular invocation and the set \xcd`context` of all methods
which could be called at that point of code, method resolution either returns
the unique identifier of the method that should be called, or (conceptually)
throws an exception if the call cannot be resolved.

The procedure for computing \xcd`resolve(invo, context)` is: 
\begin{enumerate}
\item Eliminate from \xcd`context` those methods which are not {\em
      acceptable}; \viz, those whose name, type parameters, and formal parameters
      do not suitably match \xcd`invo`.  In more detail:
      \begin{itemize}
      \item The method name \xcd`m` must simply equal the invocation name \xcdmath"m$'$";
      \item X10 infers type parameters, by an algorithm given in \Sref{TypeParamInfer}.
      \item The method's type parameters are bound to the invocation's for the
            remainder of the acceptability test.
      \item The actual parameter types must be subtypes of the formal
            parameter types, or be coercible to such subtypes.  Parameter $i$
            is a subtype if \xcdmath"T$'_i$ <: T$_i$".  It is implicitly
            coercible to a subtype if either it is a subtype, or if there is
            an implicit coercion operator 
            defined from \xcdmath"T$'_i$" to some type \xcd`U`, and 
            \xcdmath"U <: T$_i$". \index{method resolution!implicit coercions
            and} \index{implicit coercion}\index{coercion}.  If coercions are
            used to resolve the method, they will be called on the arguments
            before the method is invoked.
            
      \end{itemize}
\item Eliminate from \xcd`context` those methods which are not {\em
      available}; \viz, those which cannot be called due to visibility
      constraints, such as methods from other classes marked \xcd`private`.
      The remaining methods are both acceptable and available; they might be
      the one that is intended.
\item If the method invocation is a \xcd`super` invocation appearing in class
      \xcd`Cl`, methods of \xcd`Cl` and its subclasses are considered
      unavailable as well.
      
\item From the remaining methods, find the unique \xcd`ms` which is more specific than all the
      others, \viz, for which \xcd`specific(ms,mo) = true` for all other
      methods \xcd`mo`.
      The specificity test \xcd`specific` is given next.
      \begin{itemize}
      \item If there is a unique such \xcd`ms`, then
            \xcd`resolve(invo,context)` returns the \xcd`id` of \xcd`ms`.  
      \item If there is not a unique such \xcd`ms`, then \xcd`resolve` reports
            an error.
      \end{itemize}

\end{enumerate}

The subsidiary procedure \xcd`specific(m1, m2)` determines whether method
\xcd`m1` is equally or more specific than \xcd`m2`.  \xcd`specific` is not a
total order: is is possible for each one to be considered more specific than
the other, or either to be more specific.  \xcd`specific` is computed as: 
\begin{enumerate}
\item Construct an invocation \xcd`invo1` based on \xcd`m1`: 
      \begin{itemize}
      \item \xcd`invo1`'s method name is \xcd`m1`'s method name;
      \item \xcd`invo1`'s generic parameters are those of \xcd`m1`--- simply
            some type variables.
      \item \xcd`invo1`'s parameters are those of \xcd`m1`.
      \end{itemize}
\item If \xcd`m2` is acceptable for the invocation \xcd`invo1`,
      \xcd`specific(m1,m2)` returns true; 
\item Construct an invocation \xcd`invo2p`, which is \xcd`invo1` with the
      generic parameters erased.  Let \xcd`invo2` be \xcd`invo2p` with generic
      parameters as inferred by X10's type inference algorithm.  If type
      inference fails, \xcd`specific(m1,m2)` returns false.
\item If \xcd`m2` is acceptable for the invocation \xcd`invo2`,
      \xcd`specific(m1,m2)` returns true; 
\item Otherwise, \xcd`specific(m1,m2)` returns false.
\end{enumerate}

\subsection{Field Resolution}

An identifier \xcd`p` can refer to a number of things.  The rules are somewhat
different inside and outside of a constraint.

Outside of a constraint, the compiler chooses
the first one from the following list which applies: 
\begin{enumerate}
\item A local variable named \xcd`p`.
\item A field of \xcd`this`, \viz{} \xcd`this.p`.
\item A nullary property method, \xcd`this.p()`
\item A member type named \xcd`p`.
\item A package named \xcd`p`.
\end{enumerate}

Inside of a constraint, the rules are slightly different, because \xcd`self`
is available, and packages cannot be used per se.
\begin{enumerate}
\item A local variable named \xcd`p`.
\item A property of \xcd`this` or of \xcd`self`, \viz{} \xcd`this.p` or
      \xcd`self.p`.  If both are available, report an error.
\item A nullary property method, \xcd`this.p()`
\item A member type named \xcd`p`.
\end{enumerate}

\subsection{Other Disambiguations}
\label{sect:disambiguations}

It is possible to have a field of the same name as a method.
Indeed, it is a common pattern to have private field and a public
method of the same name to access it:
\begin{ex}
%~~gen ^^^ MethodResolution_disamb_a
%package MethodResolution_disamb_a;
%~~vis
\begin{xten}
class Xhaver {
  private var x: Int = 0;
  public def x() = x;
  public def bumpX() { x ++; }
}
\end{xten}
%~~siv
%
%~~neg
\end{ex}

\begin{ex}
However, this can lead to syntactic ambiguity in the case where the field
\Xcd{f} of object \xcd`a` is a
function, array, list, or the like, and where \xcd`a` has a method also named
\xcd`f`.  The term \Xcd{a.f(b)} could either mean ``call method \xcd`f` of \xcd`a` upon
\xcd`b`'', or ``apply the function \xcd`a.f` to argument \xcd`b`''.  

%~~gen  ^^^ MethodResolution_disamb_b
%package MethodResolution_disamb_b;
%NOCOMPILE
%~~vis
\begin{xten}
class Ambig {
  public val f : (Int)=>Int =  (x:Int) => x*x;
  public def f(y:int) = y+1;
  public def example() {
      val v = this.f(10);
      // is v 100, or 11?
  }
}
\end{xten}
%~~siv
%
%~~neg
\end{ex}

In the case where a syntactic form \xcdmath"E.m(F$_1$, $\ldots$, F$_n$)" could
be resolved as either a method call, or the application of a field \xcd`E.m`
to some arguments, it will be treated as a method call.  
The application of \xcd`E.m` to some arguments can be specified by adding
parentheses:  \xcdmath"(E.m)(F$_1$, $\ldots$, F$_n$)".

\begin{ex}

%~~gen ^^^ MethodResolution_disamb_c
%package MethodResolution_disamb_c;
%NOCOMPILE
%~~vis
\begin{xten}
class Disambig {
  public val f : (Int)=>Int =  (x:Int) => x*x;
  public def f(y:int) = y+1;
  public def example() {
      assert(  this.f(10)  == 11  );
      assert( (this.f)(10) == 100 );
  }
}
\end{xten}
%~~siv
%
%~~neg

\end{ex}

Similarly, it is possible to have a method with the same name as a struct, say
\xcd`ambig`, giving an ambiguity as to whether \xcd`ambig()` is a struct
constructor invocation or a method invocation.  This ambiguity is resolved by
treating it as a method invocation.  If the constructor invocation is desired,
it can be achieved by including the optional \xcd`new`.  That is, 
\xcd`new ambig()` is struct constructor invocation; \xcd`ambig()` is a 
method invocation.

\section{Static Nested Classes}
\label{StaticNestedClasses}
\index{class!static nested}
\index{class!nested}
\index{static nested class}

One class (or struct or interface) may be nested within another.  The simplest
way to do this is as a \xcd`static` nested class, written by putting one class
definition at top level inside another, with the inner one having a
\xcd`static` modifier.  
For most purposes, a static nested class behaves like a top-level class.
However, a static nested class has access to private static
fields and methods of its containing class.  

Nested interfaces and static structs are permitted as well.

%~~gen ^^^ InnerClasses10
% package Classes.StaticNested; 
% NOTEST
%~~vis
\begin{xten}
class Outer {
  private static val priv = 1;
  private static def special(n:Int) = n*n;
  public static class StaticNested {
     static def reveal(n:Int) = special(n) + priv;
  }
}
\end{xten}
%~~siv
%
%~~neg

\section{Inner Classes}
\label{InnerClasses}
\index{class!inner}
\index{inner class}


Non-static nested classes are called {\em inner classes}. An inner class
instance can be thought of as a very elaborate member of an object --- one
with a full class structure of its own.   The crucial characteristic of an
inner class instance is that it has an implicit reference to an instance of
its containing class.  

\begin{ex}
This feature is particularly useful when an instance of the inner class makes
no sense without reference to an instance of the outer, and is closely tied to
it.  For example, consider a range class, describing a span of integers {$m$}
to {$n$}, and an iterator over the range.  The iterator might as well have
access to the range object, and there is little point to discussing
iterators-over-ranges without discussing ranges as well.
In the following example, the inner class \xcd`RangeIter` iterates over the
enclosing \xcd`Range`.  

It has its own private cursor field \xcd`n`, telling
where it is in the iteration; different iterations over the same \xcd`Range`
can exist, and will each have their own cursor.
It is perhaps unwise to use the name \xcd`n` for a field of the inner class,
since it is also a field of the outer class, but it is legal.  (It can happen
by accident as well -- \eg, if a programmer were to add a field \xcd`n` to a
superclass of the  outer class, the inner class would still work.)
It does not even
interfere with the inner class's ability to refer to the outer class's \xcd`n`
field: the cursor initialization 
refers to the \xcd`Range`'s lower bound through a fully qualified name
\xcd`Range.this.n`.
The initialization of its \xcd`n` field refers to the outer class's \xcd`m` field, which is
not shadowed and can be referred to directly, as \xcd`m`.


%~~gen ^^^ InnerClasses20
% package Classes.InnerClasses_a; 
% NOTEST
%~~vis
\begin{xten}
class Range(m:Int, n:Int) implements Iterable[Int]{
  public def iterator ()  = new RangeIter();
  private class RangeIter implements Iterator[Int] {
     private var n : Int = m;
     public def hasNext() = n <= Range.this.n;
     public def next() = n++;
  }
  public static def main(argv:Rail[String]) {
    val r = new Range(3,5);
    for(i in r) Console.OUT.println("i=" + i);
  }
}
\end{xten}
%~~siv
%
%~~neg
\end{ex}

An inner class has full access to the members of its enclosing class, both
static and instance.  In particular, it can access \xcd`private` information,
just as methods of the enclosing class can.  

An inner class can have its own members.  
Inside instance methods of an inner class, \xcd`this` refers to the instance
of the {\em inner} class.  The instance of the outer class can be accessed as
{\em Outer}\xcd`.this` (where {\em Outer} is the name of the outer class).
If, for some dire reason, it is necessary to have an inner class within an inner
class, the innermost class can refer to the \xcd`this` of either outer class
by using its name.

An inner class can inherit from any class in scope,
with no special restrictions. \xcd`super` inside an inner class refers to the
inner class's superclass. If it is necessary to refer to the outer classes's
superclass, use a qualified name of the form {\em Outer}\xcd`.super`.

The members of inner classes must be instance members.  They cannot be static
members.  Classes, interfaces, static methods, static fields, and typedefs are
not allowed as members of inner classes. 
The same restriction applies to local classes (\Sref{sect:LocalClasses}).

\index{inner class!extending}
Consider
an inner class \xcd`IC1` of some outer class \xcd`OC1`, being extended by 
another class \xcd`IC2`. However, since an \xcd`IC1` only exists as a
dependent of an \xcd`OC1`, each \xcd`IC2` must be associated with an \xcd`OC1`
--- or a subtype thereof --- as well.   So, \xcd`IC2` must be an inner class
of either \xcd`OC1` or some subclass \xcd`OC2 <: OC1`.

\begin{ex}For example, one often extends an
inner class when one extends its outer class: 
%~~gen ^^^ InnerClasses30
% package Classes.Innerclasses.Are.For.Innermasses;
%~~vis
\begin{xten}
class OC1 {
   class IC1 {}
}
class OC2 extends OC1 {
   class IC2 extends IC1 {} 
}
\end{xten}
%~~siv
%
%~~neg
\end{ex}


The hiding of method names has one fine point.  If an inner class defines a
method named \xcd`doit`, then {\em all} methods named \xcd`doit` from the
outer class are hidden --- even if they have different argument types than the
one defined in the inner class.
They are still accessible via
\xcd`Outer.this.doit()`, but not simply via \xcd`doit()`.  The following code
is correct, but would not be correct if the ERROR line were uncommented.

%~~gen ^^^ InnerClasses40
% package Classes.Innerclasses.StupidOverloading; 
% NOTEST
%~~vis
\begin{xten}
class Outer {
  def doit() {}
  def doit(String) {}
  class Inner { 
     def doit(Boolean, Outer) {}
     def example() {
        doit(true, Outer.this);
        Outer.this.doit();
        //ERROR: doit("fails");
     }
  }
}
\end{xten}
%~~siv
%
%~~neg


\subsection{Constructors and Inner Classes}
\label{sect:InnerClassCtor}
\index{inner class!constructor}

If \xcd`IC` is an inner class of \xcd`OC`, then instance code in the body of
\xcd`OC` can create instances of \xcd`IC` simply by calling a constructor
\xcd`new IC(...)`: 
%~~gen ^^^ InnerClasses50
% package Classes.Innerclasses.Constructors.Easy;
%~~vis
\begin{xten}
class OC {
  class IC {}
  def method(){
    val ic = new IC();
  }
}
\end{xten}
%~~siv
%
%~~neg

Instances of \xcd`IC` can be constructed from elsewhere as well.  Since every
instance of \xcd`IC` is associated with an instance of \xcd`OC`, an \xcd`OC`
must be supplied to the \xcd`IC` constructor.  The syntax for doing so is: 
\xcd`oc.new IC()`.  For example: 
%~~gen ^^^ InnerClasses60
% package Classes.Inner_a; 
% NOTEST
% /*NONSTATIC*/
%~~vis
\begin{xten}
class OC {
  class IC {}
  static val oc1 = new OC();
  static val oc2 = new OC();
  static val ic1 = oc1.new IC();
  static val ic2 = oc2.new IC();
}
class Elsewhere{
  def method(oc : OC) {
    val ic = oc.new IC();
  }
}
\end{xten}
%~~siv
%
%~~neg


\section{Local Classes}
\label{sect:LocalClasses}

Classes can be defined and instantiated in the middle of methods and other
code blocks.
A local class in a static method is a static class; a local class in an
instance method is an inner class.
 Local classes are local to the block in which they are defined.
They have access to almost everything defined at that point in the method; the
one exception is that they cannot use \xcd`var` variables. Local classes
cannot be \xcd`public`, \xcd`protected`, or \xcd`private`, because they are
only visible from within the block of declaration. They cannot be
\xcd`static`.

\begin{ex}
The following example illustrates the use of a local class \xcd`Local`, 
defined inside the body of method \xcd`m()`. 
%~~gen ^^^ InnerClasses5p9v
% package InnerClasses5p9v;
% NOTEST
%~~vis
\begin{xten}
class Outer {
  val a = 1;
  def m() {
    val a = -2; 
    val b = 2;
    class Local {
      val a = 3;
      def m() = 100*Outer.this.a + 10*b + a; 
    }
    val l : Local = new Local();
    assert l.m() == 123;
  }//end of m()
}
\end{xten}
%~~siv
% class Hook{ def run() {
%   val o <: Outer = new Outer();
%   o.m();
%   return true;
% } }
%~~neg
Note that the middle \xcd`a`,
whose value is \xcd`-2`, is not accessible inside of \xcd`Local`; it is
shadowed by \xcd`Local`'s \xcd`a` field.  \xcd`Outer`'s \xcd`a` is also
shadowed, but the notation \xcd`Outer.this` gives a reference to the enclosing
\xcd`Outer` object.  There is no corresponding notation to access shadowed local
variables from the enclosing block; if you need to get them, rename the fields
of \xcd`Local`.    
\end{ex}


The members of inner classes must be instance members.  They cannot be static
members.  Classes, interfaces, static methods, static fields, and typedefs are
not allowed as members of local classes. 
The same restriction applies to inner classes (\Sref{InnerClasses}). 





\section{Anonymous Classes}
\index{class!anonymous}
\index{anonymous class}

It is possible to define a new local class and instantiate it as part of an
expression.  The new class can extend an existing class or interface.  Its body
can include all of the usual members of a local class. It can refer to any
identifiers available at that point in the expression --- except for \xcd`var`
variables.  An anonymous class in a static context is a static inner class.

Anonymous classes are useful when you want to package several pieces of
behavior together (a single piece of behavior can often be expressed as a
function, which is syntactically lighter-weight), or if you want to extend and
vary an extant class without going through the trouble of actually defining a
whole new class.

The syntax for an anonymous class is a constructor call followed immediately
by a braced class body: \xcd`new C(1){def foo()=2;}`.

\begin{ex}In the following minimalist example, the abstract class \xcd`Choice`
encapsulates a decision.   A \xcd`Choice` has a \xcd`yes()` and a \xcd`no()`
method.  The \xcd`choose(b)` method will invoke one of the two.  \xcd`Choice`s
also have names.

The \xcd`main()` method creates a specific \xcd`Choice`.  \xcd`c` is not a
immediate instance of \xcd`Choice` --- as an abstract class, \xcd`Choice` has
no immediate instances. \xcd`c` is an instance of an anonymous class which
inherits from \xcd`Choice`, but supplies \xcd`yes()` and \xcd`no()` methods.
These methods modify the contents of the \xcd`Cell[Int]` \xcd`n`.  (Note that,
as \xcd`n` is a local variable, it would take a few lines more coding to
extract \xcd`c`'s class, name it, and make it an inner class.)  The call to
\xcd`c.choose(true)`  will call \xcd`c.yes()`, incrementing \xcd`n()`, in a
rather roundabout manner.

%~~gen ^^^ InnerClasses70
% package ClassInnnerclassAnonclassOw; 
%~~vis
\begin{xten}
abstract class Choice(name: String) {
  def this(name:String) {property(name);}
  def choose(b:Boolean) { 
     if (b) this.yes(); else this.no(); }
  abstract def yes():void;
  abstract def no():void;
}

class Example {
  static def main(Rail[String]) {
    val n = new Cell[Int](0);
    val c = new Choice("Inc Or Dec") {
      def yes() { n() += 1; }
      def no()  { n() -= 1; }
      };
    c.choose(true);
    Console.OUT.println("n=" + n());
  }
}

\end{xten}
%~~siv
%
%~~neg
\end{ex}

Anonymous classes have many of the features of classes in general.  A few
features are unavailable because they don't make sense.

\begin{itemize}

\item Anonymous classes don't have constructors.  Since they don't have names,
      there's no way a constructor could get called in the ordinary way.
      Instead, the \xcd`new C(...)` expression must match a constructor of the
      parent class \xcd`C`, which will be called to initialize the
      newly-created object of the anonymous class.

\item The \xcd`public`,
      \xcd`private`, and \xcd`protected`  modifiers don't make sense for
      anonymous classes:  
      Anonymous classes, being anonymous,
      cannot be referenced at all, so references to them can't be public,
      private, or protected.

\item Anonymous classes cannot be \xcd`abstract`.  Since they only exist in
      combination with a constructor call, they must be constructable.  The
      parent class of the anonymous class may be abstract, or may be an
      interface; in this case, the anonymous class must provide all the
      methods that the parent demands.

\item Anonymous classes cannot have explicit \xcd`extends` or \xcd`implements`
      clauses; there's no place in the syntax for them. They have a single
      parent and that is that. 
\end{itemize}

\chapter{Structs}
\label{XtenStructs}
\label{StructClasses}
\label{Structs}
\index{struct}

X10 objects are a powerful general-purpose programming tool. However, the
power must be paid for in space and time. In space, a typical object
implementation requires some extra memory for run-time class information, as
well as a pointer for each reference to the object.  In time, a typical object
requires an extra indirection to read or write data, and some computation to
figure out which method body to call.  

For high-performance computing, this overhead may not be acceptable for all
objects. X10 provides structs, which are stripped-down objects. They are less
powerful than objects; in particular they lack inheritance and mutable fields.
Without inheritance, method calls do not need to do any lookup; they can be
implemented directly. Accordingly, structs can be implemented and used more
cheaply than objects, potentially avoiding the space and time overhead.
(Currently, the C++ back end avoids the overhead, but the Java back end
implements structs as Java objects and does not avoid it.)

Structs and classes are interoperable. Both can implement interfaces (in
particular, like all X10 values they implement \xcd`Any`), and subprocedures
whose arguments are defined by interfaces can take both structs and classes.
(Some caution is necessary here: referring to a struct through an interface
requires overhead similar to that required for an object.)

They are also interconvertable, within the constraints of structs. If you
start off defining a struct and decide you need a class instead, the code
change required is simply changing the keyword \xcd`struct` to \xcd`class`. If
you have a class that does not use inheritance or mutable fields, it can be
converted to a struct by changing its keyword. Client code using the struct
that was a class will need certain changes: the \xcd`new` keyword must be
added in constructor calls, and structs (unlike classes) do not have default values.



\section{Struct declaration}
\index{struct!declaration}
A struct declaration has the structure: 
\begin{xtenmath}
$\mbox{\emph{StructModifiers}}^{\mbox{?}}$
struct C[X$_1$, $\ldots$, X$_n$](p$_1$:T$_1$, $\ldots$, p$_n$:T$_n$){c} 
   implements I$_1$, $\ldots$, I$_k$ {
$\mbox{\emph{StructBody}}$
}
\end{xtenmath}

All fields of a struct must be \xcd`val`.

A struct \Xcd{S} cannot contain a field of type \Xcd{S}, or a field of struct
type \Xcd{T} which, recursively, contains a field of type \Xcd{S}.  This
restriction is necessary to permit \xcd`S` to be implemented as a contiguous
block of memory of size equal to the sum of the sizes of its fields.  


Values of a struct \Xcd{C} type can be created by invoking a constructor
defined in \Xcd{C}, but without prefixing it with \Xcd{new}: 
%~~gen
% package Structs.For.Muckts;
%~~vis
\begin{xten}
struct Polar(r:Double, theta:Double){
  def this(r:Double, theta:Double) {property(r,theta);}
  static val Origin = Polar(0,0);
  static val x0y1 = Polar(1, 3.14159/2);
}
\end{xten}
%~~siv
%
%~~neg

Structs support the same notions of generics, properties, and constrained
types that classes do.  For example, the \xcd`Pair` type below provides pairs
of values; the \xcd`diag()` method applies only when the two elements of the
pair are equal, and returns that common value: 
%~~gen
% package Structs.For.Muckts;
%~~vis
\begin{xten}
struct Pair[T,U](t:T, u:U) {
  def this(t:T, u:U) { property(t,u); }
  def diag(){T==U && t==u} = t;
}
\end{xten}
%~~siv
%
%~~neg


\section{Boxing of structs}
\index{auto-boxing!struct to interface}
\index{struct!auto-boxing}
\index{struct!casting to interface}
\label{auto-boxing} 
If a struct \Xcd{S} implements an interface \Xcd{I} (\eg, \Xcd{Any}),
a value \Xcd{v} of type \Xcd{S} can be assigned to a variable of type
\Xcd{I}. The implementation creates an object \Xcd{o} that is an
instance of an anonymous class implementing \Xcd{I} and containing
\Xcd{v}.  The result of invoking a method of \Xcd{I} on \Xcd{o} is the
same as invoking it on \Xcd{v}. This operation is termed {\em auto-boxing}.
It allows full interoperability of structs and objects---at the cost of losing
the extra efficiency of the structs when they are boxed.

In a generic class or struct obtained by instantiating a type parameter
\Xcd{T} with a struct \Xcd{S}, variables declared at type \Xcd{T} in the body
of the class are not boxed. They are implemented as if they were declared at
type \Xcd{S}.

\section{Optional Implementation of \Xcd{Any} methods}
\label{StructAnyMethods}
\index{Any!structs}

Two
structs are equal (\Xcd{==}) if and only if their corresponding fields
are equal (\Xcd{==}). 

All structs implement \Xcd{x10.lang.Any}. 
Structs are required to implement the following methods from \xcd`Any`.  
Programmers need not provide them; X10 will produce them automatically if 
the program does not include them. 
\begin{xten}
  public def equals(Any):Boolean;
  public def hashCode():Int;
  public def typeName():String;
  public def toString():String;  
\end{xten}


A programmer who provides an explicit implementation
of \Xcd{equals(Any)} for a struct \Xcd{S} should also consider
supplying a definition for \Xcd{equals(S):Boolean}. This will often
yield better performance since the cost of an upcast to \Xcd{Any} and
then a downcast to \Xcd{S} can be avoided.

\section{Primitive Types}
\index{types!primitive}
\index{primitive types}
\index{Int}
\index{UInt}
\index{Long}
\index{ULong}
\index{Char}
\index{Byte}
\index{UByte}
\index{Boolean}
\index{Short}
\index{UShort}
\index{Float}
\index{Double}

Certain types that might be built in to other languages are in fact
implemented as structs in package \xcd`x10.lang` in X10. Their methods and
operations are often provided with \xcd`@Native` (\Sref{NativeCode}) rather
than X10 code, however. These types are:
\begin{xten}
Boolean, Char, Byte, Short, Int, Long
Float, Double, UByte, UShort, UInt, ULong
\end{xten}
 
\section{Generic programming with structs}
\section{struct!generic}
\section{generics!struct}

An unconstrained type variable \Xcd{X} can be instantiated with \Xcd{Object} or
its subclasses or structs or functions.

Within a generic struct, all the operations of \Xcd{Any} are available
on a variable of type \Xcd{X}. Additionally, variables of type \Xcd{X} may
be used with \Xcd{==, !=}, in \Xcd{instanceof}, and casts.

\bard{The rest of this section is under discussion.  The example is wrong; it
ignores the fact that values can be functions.}
The programmer must be aware of the different interpretations of
equality for structs and classes and ensure that the code is correctly
written for both cases. If necessary the programmer can write code
that distinguishes between the two cases (a type parameter \Xcd{X} is
instantiated to a struct or not) as follows:

\begin{xten}
val x:X = ...;
if (x instanceof Object) { // x is a real object
   val x2 = x as Object; // this cast will always succeed.
   ...
} else { // x is a struct
   ...
}
\end{xten}
 
  
\section{Example structs}

\xcd`x10.lang.Complex` provides a detailed example of a practical struct,
suitable for use in a library.  For a shorter example, we define the
\xcd`Pair` struct---available in \xcd`x10.util.Pair`.  A \xcd`Pair` packages
two values of possibly unrelated type together in a single value, \eg, to
return two values from a function.

%~~gen
% package Structs.Pairs.Are.For.Squares;
%~~vis
\begin{xten}
struct Pair[T,U] {
    public val first:T;
    public val second:U;
    public def this(first:T, second:U):Pair[T,U] {
        this.first = first;
        this.second = second;
    }
    public def toString():String {
        return "(" + first + ", " + second + ")";
    }
}
\end{xten}
%~~siv
%
%~~neg

\section{Nested Structs}
\index{struct!static nested}
\index{static nested struct}

Static nested structs may be defined, essentially as static nested classes
except for making them structs
(\Sref{StaticNestedClasses}).  Inner structs may be defined, essentially as
inner classes except making them structs (\Sref{InnerClasses}).



\chapter{Functions}
\label{Functions}
\label{functions}
\index{functions}
\label{Closures}

\section{Overview}
Functions, the last of the three kinds of values in X10, encapsulate pieces of
code which can be applied to a vector of arguments to produce a value.
Functions, when applied, can do nearly anything that any other code could do:
fail to terminate, throw an exception, modify variables, spawn activities,
execute in several places, and so on. X10 functions are not mathematical
functions: the \xcd`f(1)` may return \xcd`true` on one call and \xcd`false` on
an immediately following call.

It is a limitation of \XtenCurrVer{} that functions do not support
type arguments. This limitation may be removed in future versions of
the language.

A \emph{function literal} \xcd"(x1:T1,..,xn:Tn){c}:T=>e" creates a function of
type\\ \xcd"(x1:T1,...,xn:Tn){c}=>T" (\Sref{FunctionType}).  For example, 
\xcd`(x:Int) => x*x` is a function literal describing the squaring function on
integers.   
\xcd`null` is also a function value.

Function application is written \xcd`f(a,b,c)`, following common mathematical
usage. 
\index{Exception!unchecked}
Function invocation may throw unchecked exceptions. 

The function body may be a block.  To compute integer squares by repeated
addition (inefficiently), one may write: 
%~~gen
% package Functions.Are.For.Spunctions;
% class Examplllll {
% static 
%~~vis
\begin{xten}
val sq: (Int) => Int 
      = (n:Int) => {
           var s : Int = 0;
           val abs_n = n < 0 ? -n : n;
           for ([i] in 1..abs_n) s += abs_n;
           s
        };
\end{xten}
%~~siv
%}
%~~neg




A function literal evaluates to a function entity {$\phi$}. When {$\phi$} is
applied to a suitable list of actual parameters \xcd`a1`-\xcd`an`, it
evaluates \xcd`e` with the formal parameters bound to the actual parameters.
So, the following are equivalent, where \xcd`e` is an expression involving
\xcd`x1` and \xcd`x2`\footnote{Strictly, there are a few other requirements;
  \eg, \xcd`result` must be a \xcd`var` of type \xcd`T` defined outside the
  outer block, the variables \xcd`a1` and \xcd`a2` had better not appear in
  \xcd`e`, and everything in sight had better typecheck properly.}

%~~gen
% package functions2.why.is.there.a.two;
% abstract class FunctionsTooManyFlippingFunctions[T, T1, T2]{
% abstract def arg1():T1;
% abstract def arg2():T2;
% def thing1(e:T) {var result:T;
%~~vis
\begin{xten}
{
  val f = (x1:T1,x2:T2){true}:T => e;
  val a1 : T1 = arg1();
  val a2 : T2 = arg2();
  result = f(a1,a2);
}
\end{xten}
%~~siv
%}}
%~~neg
and 
%~~gen
% package functions2.why.is.there.a.two.but.here.is.the.other.one;
% abstract class FunctionsTooManyFlippingFunctions[T, T1, T2]{
% abstract def arg1():T1;
% abstract def arg2():T2;
% def thing1(e:T) {var result:T;
%~~vis
\begin{xten}
{
  val a1 : T1 = arg1();
  val a2 : T2 = arg2();
  {
     val x1 : T1 = a1;
     val x2 : T2 = a2;
     result = e;
  }  
}
\end{xten}
%~~siv
%}}
%~~neg
\noindent
This doesn't quite work if the body is a statement rather than an expression.
A few language features are forbidden (\xcd`break` or \xcd`continue` of a loop
that surrounds the function literal) or mean something different (\xcd`return`
inside a function returns from the function). 





The \emph{method selector expression} \Xcd{e.m.(x1:T1,...,xn:Tn)} (\Sref{MethodSelectors})
permits the specification of the function underlying
the method \Xcd{m}, which takes arguments of type \Xcd{(x1:T1,..., xn:Tn)}.
Within this function, \Xcd{this} is bound to the result of evaluating \Xcd{e}.

Function types may be used in \Xcd{implements} clauses of class
definitions. Instances of such classes may be used as functions of the
given type.  Indeed, an object may behave like any (fixed) number of
functions, since the class it is an instance of may implement any
(fixed) number of function types.

%\section{Implementation Notes}
%\begin{itemize}
%
%\item Note that e.m.(T1,...,Tn) will evaluate e to create a
%  function. This function will be applied later to given
%  arguments. Thus this syntax can be used to evaluate the receiver of
%  a method call ahead of the actual invocation. The resulting function
%  can be used multiple times, of course.
%\end{itemize}


\section{Function Literals}
\index{literal!function}
\label{FunctionLiteral}

\Xten{} provides first-class, typed functions, including
\emph{closures}, \emph{operator functions}, and \emph{method
  selectors}.

\begin{grammar}
ClosureExpression \:
        \xcd"("
        Formals\opt
        \xcd")"
\\ &&
        Guard\opt
        ReturnType\opt
        Throws\opt
        Offers\opt
        \xcd"=>" ClosureBody \\
ClosureBody \:
        Expression \\
        \| \xcd"{" Statement\star \xcd"}" \\
        \| \xcd"{" Statement\star Expression \xcd"}" \\
\end{grammar}

Functions have zero or more formal parameters and an optional return type.
The body has the 
same syntax as a method body; it may be either an expression, a block
of statements, or a block terminated by an expression to return. In
particular, a value may be returned from the body of the function
using a return statement (\Sref{ReturnStatement}). 

The type of a
function is a function type (\Sref{FunctionType}).  In some cases the
return type \Xcd{T} is also optional and defaults to the type of the
body. If a formal \Xcd{xi} does not occur in any
\Xcd{Tj}, \Xcd{c}, \Xcd{T} or \Xcd{e}, the declaration \Xcd{xi:Ti} may
be replaced by just \Xcd{Ti}: \xcd`(Int)=>7` is the integer function returning
7 for all inputs.

\label{ClosureGuard}

As with methods, a function may declare a guard to
constrain the actual parameters with which it may be invoked.
The guard may refer to the type parameters, formal parameters,
and any \xcd`val`s in scope at the function expression.

The body of the function is evaluated when the function is
invoked by a call expression (\Sref{Call}), not at the function's
place in the program text.

As with methods, a function with return type \xcd"Void" cannot
have a terminating expression. 
If the return type is omitted, it is inferred, as described in
\Sref{TypeInference}.
It is a static error if the return type cannot be inferred.  \Eg,
\xcd`(Int)=>null` is not well-defined; X10 does not know which type of
\xcd`null` is intended.  
%~~exp~~`~~`~~ ~~
But \xcd`(Int):Rail[Double] => null` is legal.


\begin{example}
The following method takes a function parameter and uses it to
test each element of the list, returning the first matching
element.  It returns \xcd`otherwise` if no element matches.

%~~gen
% package functions2.oh.no;
% import x10.util.*;
% class Finder {
% static 
%~~vis
\begin{xten}

def find[T](f: (T) => Boolean, xs: List[T], absent:T): T = {
  for (x: T in xs)
    if (f(x)) return x;
  absent
  }
\end{xten}
%~~siv
% }
%~~neg

The method may be invoked thus:
%~~gen
% package functions2.oh.no.my.ears;
% import x10.util.*;
% class Finderator {
% static def find[T](f: (T) => Boolean, xs: x10.util.List[T], absent:T): T = {
%  for (x: T in xs)
%    if (f(x)) return x;
%  absent
%}
% static def checkery() {
%~~vis
\begin{xten}
xs: List[Int] = new ArrayList[Int]();
x: Int = find((x: Int) => x>0, xs, 0);
\end{xten}
%~~siv
%}}
%~~neg

\end{example}

As with a normal method, the function may have a \xcd"throws"
clause. It is a static error if the body of the function throws a
checked exception that is not declared in the function's \xcd"throws"
clause.


\subsection{Outer variable access}

In a function
\xcdmath"(x$_1$: T$_1$, $\dots$, x$_n$: T$_n$){c} => { s }"
the types \xcdmath"T$_i$", the guard \xcd"c" and the body \xcd"s"
may access many, though not all, sorts of variables from outer scopes.  
Specifically, they can access: 
\begin{itemize}
\item All fields of the enclosing object and class;
\item All type parameters;
\item All \xcd`val` variables;
\item \xcd`var` variables with the \xcd`shared` annotation. 
\end{itemize}


\limitation{\xcd`shared` is not currently supported.}

The function body may refer to instances of enclosing classes using
the syntax \xcd"C.this", where \xcd"C" is the name of the
enclosing class.  \xcd`this` refers to the instance of the immediately
enclosing class, as usual.

For example, the following is legal.  However, it would not be legal to add
\xcd`e` or \xcd`h` to the sum; they are non-\xcd`shared` \xcd`var`s from the
surrounding scope.

%%TODO -- this example uses 'shared', which is not currently available.
\begin{xten}
class Lambda {
   var a : Int = 0;
   val b = 0;
   def m(var c : Int, shared var d : Int,  val e : Int) {
      var f : Int = 0;
      shared var g : Int = 0;
      val h : Int = 0;
      val closure = (var i: Int, val j: Int) => {
    	  return a + b + d + g + i + j + this.a + Lambda.this.a;
      };
      return closure;
   }
}
\end{xten}


{\bf Rationale:} Non-\xcd`shared` \xcd`var`s like \xcd`e` and \xcd`h` are
excluded in X10, as in many other languages, for practical implementation
reasons. They are allocated on the stack, which is desirable for efficiency.
However, the closure may exist for long after the stack frame containing
\xcd`e` and \xcd`h` has been freed, so those storage locations are no longer
valid for those variables. \xcd`shared var`s are heap-allocated, which is less
efficient but allows them to exist after \xcd`m` returns. 


\xcd`shared` does not guarantee {\bf atomic} access to the shared variable. As
with any code that might mutate shared data concurrently, be sure to protect
references to mutable shared state with \xcd`atomic`. For example, the
following code returns a pair of closures which operate on the same shared
variable \xcd`a`, which are concurrency-safe---even if invoked many times
simultaneously. Without \xcd`atomic`, it would no longer be concurrency-safe.


%~fails~gen
% package Functions2.Are.All.Too.Much;
% class Fun2Frivols {
%~fails~vis
\begin{xten}
  def counters() {
      shared var a : Int = 0;
       return [
          () => {atomic a ++;},
          () => {atomic return a;}
          ];
   }
\end{xten}
%~fails~siv
%}
%
%~fails~neg


%SHARED% \begin{note}
%SHARED% The main activity may run in parallel with any
%SHARED% functions it creates. Hence even the read of an outer variable by the
%SHARED% body of a function may result in a race condition. Since functions are
%SHARED% first-class, the analysis of whether a function may execute in parallel
%SHARED% with the activity that created it may be difficult.
%SHARED% \end{note}

%% vj: This should be verified.
%\begin{note}
%The rule for accessing outer variables from function bodies
%should be the same as the rule for accessing outer variables from local
%or anonymous classes.
%\end{note}

\section{Method selectors}
\label{MethodSelectors}
\index{function!method selector}
\index{method!underlying function}

A method selector expression allows a method to be used as a
first-class function, without writing a function expression for it.
For example, consider a class \xcd`Span` defining ranges of integers.  

%~~gen
% package Functions2.Span;
%~~vis
\begin{xten}
class Span(low:Int, high:Int) {
  def this(low:Int, high:Int) {property(low,high);}
  def between(n:Int) = low <= n && n <= high;
  def example() {
    val digit = new Span(0,9);
    val isDigit : (Int) => Boolean = digit.between.(Int);
    if (isDigit(8)) Console.OUT.println("8 is!");
  }
}
\end{xten}
%~~siv
%
%~~neg
\noindent


In \xcd`example()`, 
%~~exp~~`~~`~~ digit:Span~~class Span(low:Int, high:Int) {def this(low:Int, high:Int) {property(low,high);} def between(n:Int) = low <= n && n <= high;}
\xcd`digit.between.(Int)` 
is a unary function testing whether its argument is between zero
and nine.  It could also be written 
%~~exp~~`~~`~~ digit:Span~~class Span(low:Int, high:Int) {def this(low:Int, high:Int) {property(low,high);} def between(n:Int) = low <= n && n <= high;}
\xcd`(n:Int) => digit.between(n)`.

This is formalized thus:

\begin{grammar}
MethodSelector \:
        Primary \xcd"."
        MethodName \xcd"."
                TypeParameters\opt \xcd"(" Formals\opt \xcd")" \\
      \|
        TypeName \xcd"."
        MethodName \xcd"."
                TypeParameters\opt \xcd"(" Formals\opt \xcd")" \\
\end{grammar}

The \emph{method selector expression} \Xcd{e.m.(T1,...,Tn)} is type
correct only if  the static type of \Xcd{e} is a
class or struct or interface \xcd`V` with a method
\Xcd{m(x1:T1,...xn:Tn)\{c\}:T} defined on it (for some
\Xcd{x1,...,xn,c,T)}. At runtime the evaluation of this expression
evaluates \Xcd{e} to a value \Xcd{v} and creates a function \Xcd{f}
which, when applied to an argument list \Xcd{(a1,...,an)} (of the right
type) yields the value obtained by evaluating \Xcd{v.m(a1,...,an)}.

Thus, the method selector

\begin{xtenmath}
e.m.[X$_1$, $\dots$, X$_m$](T$_1$, $\dots$, T$_n$)
\end{xtenmath}
\noindent behaves as if it were the function
\begin{xtenmath}
((v:V)=>
  [X$_1$, $\dots$, X$_m$](x$_1$: T$_1$, $\dots$, x$_n$: T$_n$){c} 
  => v.m[X$_1$, $\dots$, X$_m$](x$_1$, $\dots$, x$_n$))
(e)
\end{xtenmath}


\limitation{X10 functions, including method selectors, do not currently accept
generic arguments.}

Because of overloading, a method name is not sufficient to
uniquely identify a function for a given class (in Java-like languages).
One needs the argument type information as well.
The selector syntax (dot) is used to distinguish \xcd"e.m()" (a
method invocation on \xcd"e" of method named \xcd"m" with no arguments)
from \xcd"e.m.()"
(the function bound to the method). 

A static method provides a binding from a name to a function that is
independent of any instance of a class; rather it is associated with the
class itself. The static function selector
\xcdmath"T.m.(T$_1$, $\dots$, T$_n$)" denotes the
function bound to the static method named \xcd"m", with argument types
\xcdmath"(T$_1$, $\dots$, T$_n$)" for the type \xcd"T". The return type
of the function is specified by the declaration of \xcd"T.m".

There is no difference between using a function defined directly 
directly using the function syntax, or obtained via static or
instance function selectors.


\section{Operator functions}
\label{OperatorFunction}
\index{function!operator}
Every operator (e.g.,
\xcd"+",
\xcd"-",
\xcd"*",
\xcd"/",
\dots) has a family of functions, one for
each type on which the operator is defined. The function can be
selected using the ``\xcd`.`'' syntax:

\begin{grammar}
OperatorFunction
        \: TypeName \xcd"." Operator \xcd"(" Formals\opt \xcd")" \\
        \| TypeName \xcd"." Operator \\
\end{grammar}

If an operator has more than one arity (\eg, unary and binary
\xcd"-"), the unary version may be selected by giving the
formal parameter types.  The binary version is selected by
default, or the types may be specified for clarity.
For example, the following equivalences hold:

\begin{xtenmath}
String.+             $\equiv$ (x: String, y: String): String => x + y
Long.-               $\equiv$ (x: Long, y: Long): Long => x - y
Float.-(Float,Float) $\equiv$ (x: Float, y: Float): Float => x - y
Int.-(Int)           $\equiv$ (x: Int): Int => -x
Boolean.&            $\equiv$ (x: Boolean, y: Boolean): Boolean => x & y
Boolean.!            $\equiv$ (x: Boolean): Boolean => !x
Int.<(Int,Int)       $\equiv$ (x: Int, y: Int): Boolean => x < y
Dist.|(Place)        $\equiv$ (d: Dist, p: Place): Dist => d | p
\end{xtenmath}


%%TODO -- fix commented-out lines!

%~~gen
% package Functions2.For.The.Lose;
% class TypecheckThatSillyExample {
%   def checker() {
%    val l1 : (String, String) => String = String.+;
%    val r1 : (String, String) => String = (x: String, y: String): String => x + y;
%    val l2 : (Long,Long) => Long = Long.-;
%    val r2 : (Long,Long) => Long = (x: Long, y: Long): Long => x - y;
%//var v1 : (Float,Float) => Float = Float.-(Float,Float) ;
%var v2 : (Float,Float) => Float = (x: Float, y: Float): Float => x - y;
%//var v3 : (Int) => Int =  Int.-(Int)     ;      ;
%var v4  : (Int) => Int  =  (x: Int): Int => -x;
%var v5 : (Boolean,Boolean) => Boolean = Boolean.&            ;
%var v6 : (Boolean,Boolean) => Boolean =  (x: Boolean, y: Boolean): Boolean => x & y;
%//var v7 : (Boolean) => Boolean = Boolean.!            ;
%var v8 : (Boolean) => Boolean =  (x: Boolean): Boolean => !x;
%//var v9 : (Int,Int) => Boolean = Int.<(Int,Int)       ;
%var v10: (Int,Int) => Boolean =  (x: Int, y: Int): Boolean => x < y;
%//var v11 : (Dist,Place)=>Dist = Dist.|(Place)        ;
%var v12 : (Dist,Place)=>Dist=  (d: Dist, p: Place): Dist => d | p;
%}
% }
%~~vis
%~~siv
%
%~~neg

Unary and binary promotion (\Sref{XtenPromotions}) is not performed
when invoking these
operations; instead, the operands are coerced individually via implicit
coercions (\Sref{XtenConversions}), as appropriate.


%%WE-NEVER-GOT-TO-IT%%  \begin{planned}
%%WE-NEVER-GOT-TO-IT%%  
%%WE-NEVER-GOT-TO-IT%%  {\bf The following is not implemented in version 2.0.3:}
%%WE-NEVER-GOT-TO-IT%%  
%%WE-NEVER-GOT-TO-IT%%  Additionally, for every expression \xcd"e" of a type \xcd"T" at which a binary
%%WE-NEVER-GOT-TO-IT%%  operator \xcd"OP" is defined, the expression \xcd"e.OP" or
%%WE-NEVER-GOT-TO-IT%%  \xcd"e.OP(T)" represents the function
%%WE-NEVER-GOT-TO-IT%%  defined by:
%%WE-NEVER-GOT-TO-IT%%  
%%WE-NEVER-GOT-TO-IT%%  \begin{xten}
%%WE-NEVER-GOT-TO-IT%%  (x: T): T => { e OP x }
%%WE-NEVER-GOT-TO-IT%%  \end{xten}
%%WE-NEVER-GOT-TO-IT%%  
%%WE-NEVER-GOT-TO-IT%%  \begin{grammar}
%%WE-NEVER-GOT-TO-IT%%  Primary \: Expr \xcd"." Operator \xcd"(" Formals\opt \xcd")" \\
%%WE-NEVER-GOT-TO-IT%%          \| Expr \xcd"." Operator \\
%%WE-NEVER-GOT-TO-IT%%  \end{grammar}
%%WE-NEVER-GOT-TO-IT%%  
%%WE-NEVER-GOT-TO-IT%%  %% For every expression \xcd"e" of a type \xcd"T" at which a unary
%%WE-NEVER-GOT-TO-IT%%  %%operator \xcd"OP" is defined, the expression \xcd"e.OP()"
%%WE-NEVER-GOT-TO-IT%%  %% represents the function defined by:
%%WE-NEVER-GOT-TO-IT%%  
%%WE-NEVER-GOT-TO-IT%%  %% \begin{xten}
%%WE-NEVER-GOT-TO-IT%%  %% (): T => { OP e }
%%WE-NEVER-GOT-TO-IT%%  %% \end{xten}
%%WE-NEVER-GOT-TO-IT%%  
%%WE-NEVER-GOT-TO-IT%%  For example,
%%WE-NEVER-GOT-TO-IT%%  one may write an expression that adds one to each member of a
%%WE-NEVER-GOT-TO-IT%%  list \xcd"xs" by:
%%WE-NEVER-GOT-TO-IT%%  
%%WE-NEVER-GOT-TO-IT%%  %%TODO -- when this topic works, make the example wwork too.
%%WE-NEVER-GOT-TO-IT%%  %~x~gen
%%WE-NEVER-GOT-TO-IT%%  % package Functions2.Wants.A.Dinner.Reservation;
%%WE-NEVER-GOT-TO-IT%%  % import x10.util.*;
%%WE-NEVER-GOT-TO-IT%%  % class Reservation {
%%WE-NEVER-GOT-TO-IT%%  % def smerp() {
%%WE-NEVER-GOT-TO-IT%%  %   val xs = new ArrayList[Int]();
%%WE-NEVER-GOT-TO-IT%%  %~x~vis
%%WE-NEVER-GOT-TO-IT%%  \begin{xten}
%%WE-NEVER-GOT-TO-IT%%  xs.map(1.+);
%%WE-NEVER-GOT-TO-IT%%  \end{xten}
%%WE-NEVER-GOT-TO-IT%%  %~x~siv
%%WE-NEVER-GOT-TO-IT%%  % }
%%WE-NEVER-GOT-TO-IT%%  % }
%%WE-NEVER-GOT-TO-IT%%  %
%%WE-NEVER-GOT-TO-IT%%  %~x~neg
%%WE-NEVER-GOT-TO-IT%%  \end{planned}
%%WE-NEVER-GOT-TO-IT%%  
%%WE-NEVER-GOT-TO-IT%%  
\section{Functions as objects of type \Xcd{Any}}
\label{FunctionAnyMethods}

\label{FunctionEquality}
\index{function!equality} \index{equality!function} Two functions \Xcd{f} and
\Xcd{g} are equal, \xcd`f==g` if both are instances of classes and the same
object, or if both were obtained by the same evaluation of a function
literal.\footnote{A literal may occur in program text within a loop, and hence
  may be evaluated multiple times.} Further, it is guaranteed that if two
functions are equal then they refer to the same locations in the environment
and represent the same code, so their executions in an identical situation are
indistinguishable. (Specifically, if \xcd`f == g`, then \xcd`f(1)` can be
substituted for \xcd`g(1)` and the result will be identical. However, there is
no guarantee that \xcd`f(1)==g(1)` will evaluate to true. Indeed, there is no
guarantee that \xcd`f(1)==f(1)` will evaluate to true either, as \xcd`f` might
be a function which returns {$n$} on its {$n^{th}$} invocation. However,
\xcd`f(1)==f(1)` and \xcd`f(1)==g(1)` are interchangeable.)
\index{function!==}


Every function type implements all the methods of \Xcd{Any}.
\xcd`f.equals(g)` is equivalent to \xcd`f==g`.  \xcd`f.hashCode()`, 
\xcd`f.toString()`, and \xcd`f.typeName()` are implementation-dependent, but
respect \xcd`equals` and the basic contracts of \xcd`Any`. 

\index{function!equals}
\index{function!hashCode}
\index{function!toString}
\index{function!typeName}
\index{function!home}
\index{function!at(Place)}
\index{function!at(Object)}



\chapter{Expressions}\label{XtenExpressions}\index{expression}

\Xten{} has a rich expression language.
Evaluating an expression produces a value, or, in a few cases, no value. 
Expression evaluation may have side effects, such as change of the value of a 
\xcd`var` variable or a data structure, allocation of new values, or throwing
an exception. 



\section{Literals}
\index{literal}

Literals denote fixed values of built-in types. 
The syntax for literals is given in \Sref{Literals}. 

The type that \Xten{} gives a literal often includes its value. \Eg, \xcd`1`
is of type \xcd`Int{self==1}`, and \xcd`true` is of type
\xcd`Boolean{self==true}`.

\section{{\tt this}}
\index{this}
\index{\Xcd{this}}

\begin{bbgrammar}
%(FROM #(prod:Primary)#)
             Primary \: \xcd"this" (\ref{prod:Primary}) \\
                    \| \xcd"this" \\
                    \| ClassName \xcd"." \xcd"this" \\
\end{bbgrammar}


The expression \xcd"this" is a  local \xcd`val` containing a reference
to an instance of the lexically enclosing class.
It may be used only within the body of an instance method, a
constructor, or in the initializer of a instance field -- that is, the places
where there is an instance of the class under consideration.

Within an inner class, \xcd"this" may be qualified with the
name of a lexically enclosing class.  In this case, it
represents an instance of that enclosing class.  


\begin{ex}
\xcd`Outer` is a class containing \xcd`Inner`.  Each instance of
\xcd`Inner` has a reference \xcd`Outer.this` to the \xcd`Outer` involved in its
creation.  \xcd`Inner` has access to the fields of \xcd`Outer.this`, as seen
in the \xcd`outerThree` and \xcd`alwaysTrue` methods.  Note that \xcd`Inner`
has its own \xcd`three` field, which is different from and not even the same
type as \xcd`Outer.this.three`. 
%~~gen ^^^ Expressions10
% package exp.vexp.pexp.lexp.shexp; 
% NOTEST
%~~vis 
\begin{xten}
class Outer {
  val three = 3;
  class Inner {
     val three = "THREE";
     def outerThree() = Outer.this.three;
     def alwaysTrue() = outerThree() == 3;
  }
}
\end{xten}
%~~siv
%
%~~neg
\end{ex}

The type of a \xcd"this" expression is the
innermost enclosing class, or the qualifying class,
constrained by the class invariant and the
method guard, if any.

The \xcd"this" expression may also be used within constraints in
a class or interface header (the class invariant and
\xcd"extends" and \xcd"implements" clauses).  Here, the type of
\xcd"this" is restricted so that only properties declared in the
class header itself, and specifically not any members declared in the class
body or in supertypes, are accessible through \xcd"this".

\section{Local variables}

%##(Id
\begin{bbgrammar}
%(FROM #(prod:Id)#)
                  Id \: identifier & (\ref{prod:Id}) \\
\end{bbgrammar}
%##)

A local variable expression consists simply of the name of the local variable,
field of the current object, formal parameter in scope, etc. It evaluates to
the value of the local variable. 


\begin{ex}
\xcd`n` in the second line below is a local
variable expression.  The \xcd`n` in the first line is not; it is part of a
local variable declaration.
%~~gen  ^^^ Expressions20
% package exp.loc.al.varia.ble; 
% class Example {
% def example() { 
%~~vis
\begin{xten}
val n = 22;
val m = n + 56;
\end{xten}
%~~siv
%} }
%~~neg

\end{ex}

\section{Field access}
\label{FieldAccess}
\index{field!access to}

%##(FieldAccess
\begin{bbgrammar}
%(FROM #(prod:FieldAccess)#)
         FieldAccess \: Primary \xcd"." Id & (\ref{prod:FieldAccess}) \\
                    \| \xcd"super" \xcd"." Id \\
                    \| ClassName \xcd"." \xcd"super"  \xcd"." Id \\
                    \| Primary \xcd"." \xcd"class"  \\
                    \| \xcd"super" \xcd"." \xcd"class"  \\
                    \| ClassName \xcd"." \xcd"super"  \xcd"." \xcd"class"  \\
\end{bbgrammar}
%##)

A field of an object instance may be  accessed
with a field access expression.

The type of the access is the declared type of the field with the
actual target substituted for \xcd"this" in the type. 

\begin{ex}
The declaration of \xcd`b` below has a constraint involving \xcd`this`.  
The use of an instance of it, \xcd`f.b`, has the same constraint involving
\xcd`f` instead of \xcd`this`, as required.
%~~gen ^^^ Expressions5s7v
% package Expressions5s7v;
%~~vis
\begin{xten}
class Fielded {
  public val a : Int = 1;
  public val b : Int{this.a == b} = this.a;
  static def example() {
    val f : Fielded = new Fielded();
    val fb : Int{fb == f.a} = f.b;
  }
}
\end{xten}
%~~siv
%
%~~neg

\end{ex}
% If the actual
%target is not a final access path (\Sref{FinalAccessPath}),
%an anonymous path is substituted for \xcd"this".

The field accessed is selected from the fields and value properties
of the static type of the target and its superclasses.

If the field target is given by the keyword \xcd"super", the target's type is
the superclass of the enclosing class.  This form is used to access fields of
the parent class shadowed by same-named fields of the current class.

If the field target is \xcd`Cls.super`, then the target's type is \xcd`Cls`,
which must be an  enclosing class.  This (admittedly
obscure) form is used to access fields of an ancestor class which are shadowed
by same-named fields of some more recent ancestor.  

\begin{ex}
This illustrates all four cases of field access.
%~~gen ^^^ Expressions30
% package exp.re.ssio.ns.fiel.dacc.ess;
% NOTEST
%~~vis
\begin{xten}
class Uncle {
  public static val f = 1;
}
class Parent {
  public val f = 2;
}
class Ego extends Parent {
  public val f = 3;
  class Child extends Ego {
     public val f = 4;
     def classNameDotId() =  Uncle.f;     // 1
     def cnDotSuperDotId() = Ego.super.f; // 2
     def superDotId() =      super.f;     // 3
     def expDotId() =        this.f;      // 4
  }
}
\end{xten}
%~~siv
%
%~~neg
\end{ex}

If the field target is \xcd"null", a \xcd"NullPointerException"
is thrown.
If the field target is a class name, a static field is selected.
It is illegal to access  a field that is not visible from
the current context.
It is illegal to access a non-static field
through a static field access expression.  However, it is legal to access a
static field through a non-static reference.

\section{Function Literals}
Function literals are described in \Sref{Functions}.

\section{Calls}
\label{Call}
\label{MethodInvocation}
\label{MethodInvocationSubstitution}
\index{invocation}
\index{call}
\index{invocation!method}
\index{call!method}
\index{invocation!function}
\index{call!function}
\index{method!calling}
\index{method!invoking}


%##(MethodInvocation ArgumentList
\begin{bbgrammar}
%(FROM #(prod:MethodInvocation)#)
    MethodInvocation \: MethodPrimaryPrefix \xcd"(" ArgumentList\opt \xcd")" & (\ref{prod:MethodInvocation}) \\
                    \| MethodSuperPrefix \xcd"(" ArgumentList\opt \xcd")" \\
                    \| MethodClassNameSuperPrefix \xcd"(" ArgumentList\opt \xcd")" \\
                    \| MethodName TypeArguments\opt \xcd"(" ArgumentList\opt \xcd")" \\
                    \| Primary \xcd"." Id TypeArguments\opt \xcd"(" ArgumentList\opt \xcd")" \\
                    \| \xcd"super" \xcd"." Id TypeArguments\opt \xcd"(" ArgumentList\opt \xcd")" \\
                    \| ClassName \xcd"." \xcd"super"  \xcd"." Id TypeArguments\opt \xcd"(" ArgumentList\opt \xcd")" \\
                    \| Primary TypeArguments\opt \xcd"(" ArgumentList\opt \xcd")" \\
%(FROM #(prod:ArgumentList)#)
        ArgumentList \: Exp & (\ref{prod:ArgumentList}) \\
                    \| ArgumentList \xcd"," Exp \\
\end{bbgrammar}
%##)


A \grammarrule{MethodInvocation} may be to either a \xcd"static" method, an
instance method, or a closure.

The syntax for method invocations is ambiguous. \xcd`ob.m()` could either be
the invocation of a method named \xcd`m` on object \xcd`ob`, or the
application of a function held in a field \xcd`ob.m`. The target \xcd`ob` must
be type-checked to determine which of these it is.  It is a static error if
both cases are possible after type checking.

\begin{ex}
%~~gen ^^^ Expressions40
% package expres.sio.nsca.lls;
%~~vis
\begin{xten}
class Callsome {
  static val closure : () => Int = () => 1;
  static def method()            = 2;
  static def example() {
     assert Callsome.closure() == 1;
     assert Callsome.method()  == 2;
  } 
}
\end{xten}
%~~siv
% class Hook{ def run() { Callsome.example(); return true; } }
%~~neg
However, adding a static method [mis]named \xcd`closure` makes
\xcd`Callsome.closure()` 
ambiguous: it could be a call to the closure, or to the static method: 
%~~gen ^^^ Expressions50
% package expres.sio.nsca.lls.twoooo;
% class Callsome {static val closure = () => 1; static def method () = 2; static val methodEvaluated = Callsome.method();
%~~vis
\begin{xten}
  static def closure () = 3;
  // ERROR: static errory = Callsome.closure();
\end{xten}
%~~siv
% }
%~~neg
\end{ex}

The application form \xcd`e(f,g)`, when \xcd`e` evaluates to an object or
struct, invokes the application \xcd`operator`, 
defined in the form 
%~~gen ^^^ Expressions2x1f
% package Expressions2x1f;
% class Example[F,G] {
%~~vis
\begin{xten}
public operator this(f:F, g:G) = "value";
\end{xten}
%~~siv
%  }
%~~neg


Method selection rules are given in \Sref{sect:MethodResolution}.

It is a static error if a method's \grammarrule{Guard} is not statically
satisfied by the 
caller.  

\begin{ex}
In this example, a \xcd`DivideBy` object provides the service of dividing
numbers by \xcd`denom` --- so long as \xcd`denom` is not zero. 
In the \xcd`example` method, \xcd`this.div(100)`  is not allowed; there is no
guarantee that \xcd`denom != 0`.  Casting \xcd`this` to a type 
whose constraint implies \xcd`denom != 0` permits the method call.
%~~gen ^^^ Expressions60
%package Expressions.Calls.Guarded.By.Walls;
% KNOWNFAIL
%~~vis
\begin{xten}
class DivideBy(denom:Int) {
  def div(numer:Int){denom != 0} = numer / denom;
  def example() {
     val thisCast = (this as DivideBy{self.denom != 0});
     thisCast.div(100);
     //ERROR: this.div(100); 
  }
}
\end{xten}
%~~siv
% class Hook{ def run() { (new DivideBy(1)).example(); return true; } }
%~~neg
\end{ex}

\section{Assignment}\index{assignment}\label{AssignmentStatement}

%##(Assignment LeftHandSide AssignmentOperator
\begin{bbgrammar}
%(FROM #(prod:Assignment)#)
          Assignment \: LeftHandSide AssignmentOperator AssignmentExp & (\ref{prod:Assignment}) \\
                    \| ExpName  \xcd"(" ArgumentList\opt \xcd")" AssignmentOperator AssignmentExp \\
                    \| Primary  \xcd"(" ArgumentList\opt \xcd")" AssignmentOperator AssignmentExp \\
%(FROM #(prod:LeftHandSide)#)
        LeftHandSide \: ExpName & (\ref{prod:LeftHandSide}) \\
                    \| FieldAccess \\
%(FROM #(prod:AssignmentOperator)#)
  AssignmentOperator \: \xcd"=" & (\ref{prod:AssignmentOperator}) \\
                    \| \xcd"*=" \\
                    \| \xcd"/=" \\
                    \| \xcd"%=" \\
                    \| \xcd"+=" \\
                    \| \xcd"-=" \\
                    \| \xcd"<<=" \\
                    \| \xcd">>=" \\
                    \| \xcd">>>=" \\
                    \| \xcd"&=" \\
                    \| \xcd"^=" \\
                    \| \xcd"|=" \\
\end{bbgrammar}
%##)



The assignment expression \xcd"x = e" assigns a value given by
expression \xcd"e"
to a variable \xcd"x".  
Most often, \xcd`x` is mutable, a \xcd`var` variable.  The same syntax is
used for delayed initialization of a \xcd`val`, but \xcd`val`s can only be
initialized once.
%~~gen ^^^ Expressions70
% package express.ions.ass.ignment;
% class Example {
% static def exasmple() {
%~~vis
\begin{xten}
  var x : Int;
  val y : Int;
  x = 1;
  y = 2; // Correct; initializes y
  x = 3; 
  // ERROR: y = 4;
\end{xten}
%~~siv
% } } 
%~~neg


There are three syntactic forms of
assignment: 
\begin{enumerate}
\item \xcd`x = e;`, assigning to a local variable, formal parameter, field of
      \xcd`this`, etc. 
\item \xcd`x.f = e;`, assigning to a field of an object.
\item \xcdmath`a(i$_1$,$\ldots$,i$_n$) = v;`, where {$n \ge 0$}, assigning to
      an element of an array or some other such structure. This is an operator
      call (\Sref{sect:operators}).  For well-behaved classes it works like
      array assignment, mutatis mutandis, but there is no actual guarantee,
      and the compiler makes no assumptions about how this works for arbitrary \xcd`a`.
      Naturally, it is a static error if no suitable assignment operator
      for \xcd`a`.
\end{enumerate}

For a binary operator $\diamond$, the $\diamond$-assignment expression
\xcdmath"x $\diamond$= e" combines the current value of \xcd`x` with the value
of \xcd`e` by {$\diamond$}, and stores the result back into \xcd`x`.  
\xcd`i += 2`, for example, adds 2 to \xcd`i`. For variables and fields, 
\xcdmath"x $\diamond$= e" behaves just like \xcdmath"x = x $\diamond$ e". 

The subscripting forms of \xcdmath"a(i) $\diamond$= b" are slightly subtle.
Subexpressions of \xcd`a` and \xcd`i` are only evaluated once.  However,
\xcd`a(i)` and \xcd`a(i)=c` are each executed once---in particular, there is
one call to the application operator, and one to the assignment operator.
If subscripting is implemented strangely for
the class of \xcd`a`, the behavior is {\em not} necessarily updating a single
storage location. Specifically, \xcd`A()(I()) += B()` is tantamount to: 
%~~gen ^^^ Expressions80
% package expressions.stupid.addab;
% class Example {
% def example(A:()=>Rail[Int], I: () => Int, B: () => Int ) {
%~~vis
\begin{xten}
{
  val aa = A();  // Evaluate A() once
  val ii = I();  // Evaluate I() once
  val bb = B();  // Evaluate B() once
  val tmp = aa(ii) + bb; // read aa(ii)
  aa(ii) = tmp;  // write sum back to aa(ii)
}
\end{xten}
%~~siv
%}}
%~~neg

\limitation{+= does not currently meet this specification.}




\section{Increment and decrement}
\index{increment}
\index{decrement}
\index{\Xcd{++}}
\index{\Xcd{--}}


The operators \xcd"++" and \xcd"--" increment and decrement
a variable, respectively.  
\xcd`x++` and \xcd`++x` both increment \xcd`x`, just as the statement 
\xcd`x += 1` would, and similarly for \xcd`--`.  

The difference between the two is the return value.  
\xcd`++x` and \xcd`--x` return the {\em new} value of \xcd`x`, after
incrementing or decrementing.
\xcd`x++` and \xcd`x--` return the {\em old} value of \xcd`x`, before
incrementing or decrementing.


\limitation{This currently only works for numeric types.}

\section{Numeric Operations}
\label{XtenPromotions}
\index{promotion}
\index{numeric promotion}
\index{numeric operations}
\index{operation!numeric}

Numeric types (\xcd`Byte`, \xcd`Short`, \xcd`Int`, \xcd`Long`, \xcd`Float`,
\xcd`Double`, \xcd`Complex`, and unsigned variants of fixed-point types) are normal X10
structs, though most of their methods are implemented via native code. They
obey the same general rules as other X10 structs. For example, numeric
operations, coercions, and conversions are defined by \xcd`operator` definitions, the same way you could
for any struct.

Promoting a numeric value to a longer numeric type preserves the sign of the
value.  For example, \xcd`(255 as UByte) as UInt` is 255. 

Most of these operations can be defined on user-defined types as well.  While
it is good practice to keep such operations consistent with the numeric
operations whenever possible, the compiler neither enforces nor assumes any
particular semantics of user-defined operations. 

\subsection{Conversions and coercions}

Specifically, each numeric type can be converted or coerced into each other
numeric type, perhaps with loss of accuracy.
%~~gen ^^^ Expressions90
% package exp.ress.io.ns.numeric.conversions;
% class ExampleOfConversionAndStuff {
% def example() {
%~~vis
\begin{xten}
val n : Byte = 123 as Byte; // explicit 
val f : (Int)=>Boolean = (Int) => true; 
val ok = f(n); // implicit
\end{xten}
%~~siv
% } }
%~~neg



\subsection{Unary plus and unary minus}

The unary \xcd`+` operation on numbers is an identity function.
The unary \xcd`-` operation on numbers is a negation function.
On unsigned numbers, these are two's-complement.  For example, 
\xcd`-(0x0F as UByte)` is 
\xcd`(0xF1 as UByte)`.
\bard{UInts and such are closed under negation -- the negative of a UInt is
done binarily.  }



\section{Bitwise complement}

The unary \xcd"~" operator, only defined on integral types, complements each
bit in its operand.  

\section{Binary arithmetic operations} 

The binary arithmetic operators perform the familiar binary arithmetic
operations: \xcd`+` adds, \xcd`-` subtracts, \xcd`*` multiplies, 
\xcd`/` divides, and \xcd`%`
computes remainder.

On integers, the operands are coerced to the longer of their two types, and
then operated upon.  
Floating point operations are determined by the IEEE 754
standard. 
The integer \xcd"/" and \xcd"%" throw an exception 
if the right operand is zero.



\section{Binary shift operations}

The operands of the binary shift operations must be of integral type.
The type of the result is the type of the left operand.
The right operand, describing a number of bits, must be unsigned: 
%~~exp~~`~~`~~ x:Int ~~ ^^^Expressions1l4m
\xcd`x << 1U`.  


If the promoted type of the left operand is \xcd"Int",
the right operand is masked with \xcd"0x1f" using the bitwise
AND (\xcd"&") operator, giving a number at most the number of bits in an
\xcd`Int`. 
If the promoted type of the left operand is \xcd"Long",
the right operand is masked with \xcd"0x3f" using the bitwise
AND (\xcd"&") operator, giving a number at most the number of bits in a
\xcd`Long`. 

The \xcd"<<" operator left-shifts the left operand by the number of
bits given by the right operand.
The \xcd">>" operator right-shifts the left operand by the number of
bits given by the right operand.  The result is sign extended;
that is, if the right operand is $k$,
the most significant $k$ bits of the result are set to the most
significant bit of the operand.

The \xcd">>>" operator right-shifts the left operand by the number of
bits given by the right operand.  The result is not sign extended;
that is, if the right operand is $k$,
the most significant $k$ bits of the result are set to \xcd"0".
This operation is deprecated, and may be removed in a later version of the
language. 


\section{Binary bitwise operations}

The binary bitwise operations operate on integral types, which are promoted to
the longer of the two types.
The \xcd"&" operator  performs the bitwise AND of the promoted operands.
The \xcd"|" operator  performs the bitwise inclusive OR of the promoted operands.
The \xcd"^" operator  performs the bitwise exclusive OR of the promoted operands.

\section{String concatenation}
\index{string!concatenation}

The \xcd"+"  operator is used for string concatenation 
 as well as addition.
If either operand is of static type \xcd"x10.lang.String",
 the other operand is converted to a \xcd"String" , if needed,
  and  the two strings  are concatenated.
 String conversion of a non-\xcd"null" value is  performed by invoking the
 \xcd"toString()" method of the value.
  If the value is \xcd"null", the value is converted to 
  \xcd'"null"'.

The type of the result is \xcd"String".

 For example, 
%~~exp~~`~~`~~ ~~ ^^^ Expressions100
      \xcd`"one " + 2 + here` 
      evaluates to  \xcd`one 2(Place 0)`.  

\section{Logical negation}

The operand of the  unary \xcd"!" operator 
must be of type \xcd"x10.lang.Boolean".
The type of the result is \xcd"Boolean".
If the value of the operand is \xcd"true", the result is \xcd"false"; if
if the value of the operand  is \xcd"false", the result is \xcd"true".

\section{Boolean logical operations}

Operands of the binary boolean logical operators must be of type \xcd"Boolean".
The type of the result is \xcd"Boolean"

The \xcd"&" operator  evaluates to \xcd"true" if both of its
operands evaluate to \xcd"true"; otherwise, the operator
evaluates to \xcd"false".

The \xcd"|" operator  evaluates to \xcd"false" if both of its
operands evaluate to \xcd"false"; otherwise, the operator
evaluates to \xcd"true".

\section{Boolean conditional operations}

Operands of the binary boolean conditional operators must be of type
\xcd"Boolean". 
The type of the result is \xcd"Boolean"

The \xcd"&&" operator  evaluates to \xcd"true" if both of its
operands evaluate to \xcd"true"; otherwise, the operator
evaluates to \xcd"false".
Unlike the logical operator \xcd"&",
if the first operand is \xcd"false",
the second operand is not evaluated.

The \xcd"||" operator  evaluates to \xcd"false" if both of its
operands evaluate to \xcd"false"; otherwise, the operator
evaluates to \xcd"true".
Unlike the logical operator \xcd"||",
if the first operand is \xcd"true",
the second operand is not evaluated.

\section{Relational operations} 

The relational operations on numeric types compare numbers, producing
\xcd`Boolean` results.

The \xcd"<" operator evaluates to \xcd"true" if the left operand is
less than the right.
The \xcd"<=" operator evaluates to \xcd"true" if the left operand is
less than or equal to the right.
The \xcd">" operator evaluates to \xcd"true" if the left operand is
greater than the right.
The \xcd">=" operator evaluates to \xcd"true" if the left operand is
greater than or equal to the right.

Floating point comparison is determined by the IEEE 754
standard.  Thus,
if either operand is NaN, the result is \xcd"false".
Negative zero and positive zero are considered to be equal.
All finite values are less than positive infinity and greater
than negative infinity.



\section{Conditional expressions}
\index{\Xcd{? :}}
\index{conditional expression}
\index{expression!conditional}
\label{Conditional}

%##(ConditionalExp
\begin{bbgrammar}
%(FROM #(prod:ConditionalExp)#)
      ConditionalExp \: ConditionalOrExp & (\ref{prod:ConditionalExp}) \\
                    \| ClosureExp \\
                    \| AtExp \\
                    \| FinishExp \\
                    \| ConditionalOrExp \xcd"?" Exp \xcd":" ConditionalExp \\
\end{bbgrammar}
%##)

A conditional expression evaluates its first subexpression (the
condition); if \xcd"true"
the second subexpression (the consequent) is evaluated; otherwise,
the third subexpression (the alternative) is evaluated.

The type of the condition must be \xcd"Boolean".
The type of the conditional expression is some common 
ancestor (as constrained by \Sref{LCA}) of the types of the consequent and the
alternative. 

\begin{ex}
%~~exp~~`~~`~~a:Int,b:Int ~~ ^^^ Expressions110
\xcd`a == b ? 1 : 2`
evaluates to \xcd`1` if \xcd`a` and \xcd`b` are the same, and \xcd`2` if they
are different.   As the type of \xcd`1` is \xcd`Int{self==1}` and of \xcd`2`
is \xcd`Int{self==2}`, the type of the conditional expression has the form
\xcd`Int{c}`, where \xcd`self==1` and \xcd`self==2` both imply \xcd`c`.  For
example, it might be \xcd`Int{true}` -- or perhaps it might be 
\xcd`Int{self != 8}`. Note that this term has no most accurate type in the X10
type system.
\end{ex}

The subexpression not selected is not evaluated.

\begin{ex}
The following use of the conditional expression prevents division by zero.  If
\xcd`den==0`, the division is not performed at all.
%~~gen ^^^ Expressions4t3m
% package Expressions4t3m;
% class Hook {
% static def example(num:Int, den:Int ) =
%~~vis
\begin{xten}
(den == 0) ? 0 : num/den
\end{xten}
%~~siv
%; 
% def run() { 
%   return example(1,0) == 0 && example(6,3) == 2;
% } }
%~~neg

Similarly, the following code performs a method call if \xcd`op` is non-null,
and avoids the null pointer error if it is null.  Defensive coding like this
is quite common when working with possibly-null objects.
%~~gen ^^^ Expressions6o2b
% package Expressions6o2b;
% class Hook { 
% static def example(ob:Object) = 
%~~vis
\begin{xten}
(ob == null) ? null : ob.toString();
\end{xten}
%~~siv
%def run() {
%  return example(null) == null && example("yes").equals("yes"); 
% } } 
%~~neg



\end{ex}

\section{Stable equality}
\label{StableEquality}
\index{\Xcd{==}}
\index{equality}

\begin{bbgrammar}
 EqualityExp    \: RelationalExp & (\ref{prod:EqualityExp})\\
%<FROM #(prod:EqualityExp)#
    \| EqualityExp \xcd"==" RelationalExp\\
    \| EqualityExp \xcd"!=" RelationalExp\\
    \| Type  \xcd"==" Type \\
\end{bbgrammar}


The \xcd"==" and \xcd"!=" operators provide a fundamental, though
non-abstract, notion of equality.  \xcd`a==b` is true if the values of \xcd`a`
and \xcd`b` are extremely identical.

\begin{itemize}
\item If \xcd`a` and \xcd`b` are values of object type, then \xcd`a==b` holds
      if \xcd`a` and \xcd`b` are the same object.
\item If one operand is \xcd`null`, then \xcd`a==b` holds iff the other is
      also \xcd`null`.
\item If the operands both have struct type, then they must be structurally equal;
that is, they must be instances of the same struct
and all their fields or components must be \xcd"==". 
\item The definition of equality for function types is specified in
      \Sref{FunctionEquality}.
\item No implicit coercions are performed by \xcd`==`.  
\item It is a static error to have an expression \xcd`a == b` if the types of
      \xcd`a` and \xcd`b` are disjoint.  
\end{itemize}

\xcd`a != b`
is true iff \xcd`a==b` is false.

The predicates \xcd"==" and \xcd"!=" may not be overridden by the programmer.

\xcd`==` provides a {\em stable} notion of equality.  If two values are
\xcd`==` at any time, they remain \xcd`==` forevermore, regardless of what
happens to the mutable state of the program. 

\begin{ex}
Regardless of the values and types of \xcd`a` and \xcd`b`, 
or the behavior of \xcd`any_code_at_all` (which may, indeed, be
any code at all---not just a method call), the value of 
\xcd`a==b` does not change: 
%~~gen ^^^ Expressions1i5k
% package Expressions1i5k;
% class Example{ 
% def example( something: ()=>Int, something_else: ()=>Int,
%   any_code_at_all: () => Int) {
%~~vis
\begin{xten}
val a = something();
val b = something_else();
val eq1 = (a == b);
any_code_at_all();
val eq2 = (a == b);
assert eq1 == eq2;
\end{xten}
%~~siv
%} } 
%~~neg
\end{ex}

\subsection{No Implicit Coercions}
\label{sect:eqeq-no-coerce}

\xcd`==` is a primitive operation in X10 -- one of very few. Most operations,
like \xcd`+` and \xcd`<=`, are defined as \xcd`operator`s. \xcd`==` and
\xcd`!=` are not. As non-\xcd`operator`s, they need not and do not follow the
general method resolution procedure of \Sref{sect:MethodResolution}. In
particular, while \xcd`operator`s perform implicit conversions on their
arguments, \xcd`==` and \xcd`!=` do not.

The advantage of this restriction is that \xcd`==`'s behavior is as simple and
efficient as possible.  It never runs user-defined code, and the compiler can
analyze and understand it in detail -- and guarantee that it is efficient.

The disadvantage is that certain straightforward-looking idioms do not work.
One may not, for example, write
\begin{xten}
// NOT ALLOWED
for(var i : Long = 0; i != 100; i++) 
\end{xten}
A \xcd`Long` like \xcd`i` can never \xcd`==` an \xcd`Int` like \xcd`100`.

We can write \xcd`i = i + 1;`, adding an \xcd`Int` to \xcd`i`. This works 
because the expression uses \xcd`+`,  an ordinary \xcd`operator`.
There is an implicit coercion from \xcd`Int` to \xcd`Long`, so the
\xcd`1` can be converted to \xcd`1L`, which can be added to \xcd`i`.  

However, \xcd`==` does not permit implicit coercions, and so the \xcd`100`
stays an \xcd`Int`.  The loop must be written with a comparison of two
\xcd`Long`s: 
\begin{xten}
for(var i : Long = 0; i != 100L; i++) 
\end{xten}

Incidentally, it could also be written 
\begin{xten}
for(var i : Long = 0; i <= 100; i++) 
\end{xten}
The operation \xcd`<=` is a regular operator, and thus uses coercions in its
arguments, so \xcd`100` gets coerced to \xcd`100L`.  

\subsection{Non-Disjointness Requirement}

It is a static error to have an expression \xcd`a==b` where \xcd`a` and
\xcd`b` could not possibly be equal, based on their types.  This is a
practical codicil to \Sref{sect:eqeq-no-coerce}.  Consider the illegal code 
\begin{xten}
// NOT ALLOWED
for(var i : Long = 0; i != 100; i++) 
\end{xten}

\xcd`100` and \xcd`100L` are different values; they are not \xcd`==`. A
coercion could make them equal, but \xcd`==` does not allow coercions. So, if
\xcd`100 == 100L` were going to return anything, it would have to return
\xcd`false`. This would have the unfortunate effect of making the \xcd`for`
loop diverge.

Since this and related idioms are so common, and since so many programmers are
used to languages which are less precise about their numeric types, X10 avoids
the mistake by declaring it a static error in most cases.  Specifically,
\xcd`a==b` is not allowed if, by inspection of the types, \xcd`a` and \xcd`b`
could not possibly be equal.


\begin{itemize}

\item Numbers of different base types cannot be compared for equality.  
\xcd`100==100L` is a static error.  To compare numbers, explicitly cast them
%~~exp~~`~~`~~ ~~ ^^^Expressions2g6f
to the same type: \xcd`100 as Long == 100L`.

\item Indeed, structs of different types cannot be equal, and so they cannot be
compared for equality.  

\item For objects, the story is different. Unconstrained object types can
      always be compared for equality. Given objects \xcd`a:Person` and
      \xcd`b:Theory`, \xcd`a==b` could be true if \xcd`a==null` and
      \xcd`b==null`. 

\item Constrained object types may or may not be comparable.  For example,  
      if \xcd`Person` and \xcd`Theory` are both direct subclasses of
      \xcd`Object`, and \xcd`a:Person{self!=null}` and \xcd`b:Theory`, then
      \xcd`a==b` is not allowed, since the two could not possibly be equal.

\item Explicit casts erase type information.  If you wanted
      to have a comparison \xcd`a==b` for \xcd`a:Person{self!=null}` and
      \xcd`b:Theory`, you could write it as \xcd`a as Object == b as Object`.
      It would, of course, return \xcd`false`, but it would not be a compiler
      error.\footnote{Code generators often find this trick too be useful.}
      A struct and an object may both be cast to \xcd`Any` and compared for
      equality, though they, too, will always be different.

\end{itemize}





\section{Allocation}
\label{ClassCreation}
\index{new}
\index{allocation}
\index{class!instantation}
\index{class!construction}
\index{struct!instantation}
\index{struct!construction}
\index{instantation}

%##(ClassInstCreationExp
\begin{bbgrammar}
%(FROM #(prod:ClassInstCreationExp)#)
ClassInstCreationExp \: \xcd"new" TypeName TypeArguments\opt \xcd"(" ArgumentList\opt \xcd")" ClassBody\opt & (\ref{prod:ClassInstCreationExp}) \\
                    \| \xcd"new" TypeName \xcd"[" Type \xcd"]" \xcd"[" ArgumentList\opt \xcd"]" \\
                    \| Primary \xcd"." \xcd"new" Id TypeArguments\opt \xcd"(" ArgumentList\opt \xcd")" ClassBody\opt \\
                    \| AmbiguousName \xcd"." \xcd"new" Id TypeArguments\opt \xcd"(" ArgumentList\opt \xcd")" ClassBody\opt \\
\end{bbgrammar}
%##)

An allocation expression creates a new instance of a class and
invokes a constructor of the class.
The expression designates the class name and passes
type and value arguments to the constructor.

The allocation expression may have an optional class body.
In this case, an anonymous subclass of the given class is
allocated.   An anonymous class allocation may also specify a
single super-interface rather than a superclass; the superclass
of the anonymous class is \xcd"x10.lang.Object".

If the class is anonymous---that is, if a class body is
provided---then the constructor is selected from the superclass.
The constructor to invoke is selected using the same rules as
for method invocation (\Sref{MethodInvocation}).

The type of an allocation expression
is the return type of the constructor invoked, with appropriate
substitutions  of actual arguments for formal parameters, as
specified in \Sref{MethodInvocationSubstitution}.

It is illegal to allocate an instance of an \xcd"abstract" class.
The usual visibility rules apply to allocations: 
it is illegal to allocate an instance of a class or to invoke a
constructor that is not visible at
the allocation expression.

Note that instantiating a struct type can use function application syntax; 
\xcd`new` is optional.  As structs do not have subclassing, there is no need or
possibility of a {\em ClassBody}.


\section{Casts}\label{ClassCast}\index{cast}
\index{type conversion}

The cast operation may be used to cast an expression to a given type:

%##(CastExp
\begin{bbgrammar}
%(FROM #(prod:CastExp)#)
             CastExp \: Primary & (\ref{prod:CastExp}) \\
                    \| ExpName \\
                    \| CastExp \xcd"as" Type \\
\end{bbgrammar}
%##)

The result of this operation is a value of the given type if the cast
is permissible at run time, and either a compile-time error or a runtime
exception 
(\xcd`x10.lang.TypeCastException`) if it is not.  

When evaluating \xcd`E as T{c}`, first the value of \xcd`E` is converted to
type \xcd`T` (which may fail), and then the constraint \xcd`{c}` is checked. 



\begin{itemize}
\item If \xcd`T` is a primitive type, then \xcd`E`'s value is converted to type
      \xcd`T` according to the rules of
      \Sref{sec:effects-of-explicit-numeric-coercions}. 
      
\item If \xcd`T` is a class, then the first half of the cast succeeds if the
      run-time value of \xcd`E` is an instance of class \xcd`T`, or of a
      subclass. 

\item If \xcd`T` is an interface, then the first half of the cast succeeds if
      the run-time value of \xcd`E` is an instance of a class implementing
      \xcd`T`. 

\item If \xcd`T` is a struct type, then the first half of the cast succeeds if
      the run-time value of \xcd`E` is an instance of \xcd`T`.  

\item If \xcd`T` is a function type, then the first half of the cast succeeds
      if the run-time value of \xcd`X` is a function of that type, or a
      subtype of it.
\end{itemize}

If the first half of the cast succeeds, the second half -- the constraint
\xcd`{c}` -- must be checked.  In general this will be done at runtime, though
in special cases it can be checked at compile time.   For example, 
\xcd`n as Int{self != w}` succeeds if \xcd`n != w` --- even if \xcd`w` is a value
read from input, and thus not determined at compile time.

The compiler may forbid casts that it knows cannot possibly work. If there is
no way for the value of \xcd`E` to be of type \xcd`T{c}`, then 
\xcd`E as T{c}` can result in a static error, rather than a runtime error.  
For example, \xcd`1 as Int{self==2}` may fail to compile, because the compiler
knows that \xcd`1`, which has type \xcd`Int{self==1}`, cannot possibly be of
type \xcd`Int{self==2}`. 


%BB% \bard{This section need serious whomping.  The Java mention needs to go.  The
%BB% rules for coercions are given in \Sref{sec:effects-of-explicit-numeric-coercions}.
%BB% If the \xcd`Type` has a constraint, the constraint will be checked at runtime. 
%BB% We need to give examples. 
%BB% }
%BB% 
%BB% Type conversion is checked according to the
%BB% rules of the \java{} language (e.g., \cite[\S 5.5]{jls2}).
%BB% For constrained types, both the base
%BB% type and the constraint are checked.
%BB% If the
%BB% value cannot be cast to the appropriate type, a
%BB% \xcd"ClassCastException"
%BB% is thrown. 



% {\bf Conversions of numeric values}
% {\bf Can't do (a as T) if a can't be a T.}


%If the value cannot be cast to the
%appropriate place type a \xcd"BadPlaceException" is thrown. 

% Any attempt to cast an expression of a reference type to a value type
% (or vice versa) results in a compile-time error. Some casts---such as
% those that seek to cast a value of a subtype to a supertype---are
% known to succeed at compile-time. Such casts should not cause extra
% computational overhead at run time.

\section{\Xcd{instanceof}}
\label{instanceOf}
\index{\Xcd{instanceof}}
\index{instanceof}

\Xten{} permits types to be used in an in instanceof expression
to determine whether an object is an instance of the given type:

%##(RelationalExp
\begin{bbgrammar}
%(FROM #(prod:RelationalExp)#)
       RelationalExp \: RangeExp & (\ref{prod:RelationalExp}) \\
                    \| SubtypeConstraint \\
                    \| RelationalExp \xcd"<" RangeExp \\
                    \| RelationalExp \xcd">" RangeExp \\
                    \| RelationalExp \xcd"<=" RangeExp \\
                    \| RelationalExp \xcd">=" RangeExp \\
                    \| RelationalExp \xcd"instanceof" Type \\
                    \| RelationalExp \xcd"in" ShiftExp \\
\end{bbgrammar}
%##)

In the above expression, \grammarrule{Type} is any type. At run time, the
result of \xcd`e instanceof T`
is \xcd"true" if the
value of \xcd`e` is an instance of type \xcd`T`.
Otherwise the result is \xcd"false". This determination may involve checking
that the constraint, if any, associated with the type is true for the given
expression.

%~~exp~~`~~`~~x:Int~~ ^^^ Expressions120
For example, \xcd`3 instanceof Int{self==x}` is an overly-complicated way of
saying \xcd`3==x`.


However, it is a static error if \xcd`e` cannot possibly be an instance of
\xcd`C{c}`; the compiler will reject \xcd`1 instanceof Int{self == 2}` because
\xcd`1` can never satisfy \xcd`Int{self == 2}`. Similarly, \Xcd{1 instanceof
String} is a static error, rather than an expression always returning false. 

\limitationx
X10 does not currently handle \xcd`instanceof` of generics in the way you
%~NO~exp~~`~~`~~r:Array[Int](1) ~~
might expect.  For example, \xcd`r instanceof Array[Int{self != 0}]` does
not test that every element of \xcd`r` is non-zero; instead, the compiler
rejects it.


\section{Subtyping expressions}
\index{\Xcd{<:}}
\index{\Xcd{:>}}
\index{subtype!test}


%##(SubtypeConstraint
\begin{bbgrammar}
%(FROM #(prod:SubtypeConstraint)#)
   SubtypeConstraint \: Type  \xcd"<:" Type  & (\ref{prod:SubtypeConstraint}) \\
                    \| Type  \xcd":>" Type  \\
\end{bbgrammar}
%##)

The subtyping expression \xcdmath"T$_1$ <: T$_2$" evaluates to \xcd"true" if
\xcdmath"T$_1$" is a subtype of \xcdmath"T$_2$".

The expression \xcdmath"T$_1$ :> T$_2$" evaluates to \xcd"true" if
\xcdmath"T$_2$" is a subtype of \xcdmath"T$_1$".

The expression \xcdmath"T$_1$ == T$_2$"
evaluates to  \xcd"true" if 
\xcdmath"T$_1$" is a subtype of \xcdmath"T$_2$" and
if \xcdmath"T$_2$" is a subtype of \xcdmath"T$_1$".

\begin{ex}
Subtyping expressions are particularly useful in giving constraints on generic
types.  \xcd`x10.util.Ordered[T]` is an interface whose values can be compared
with values of type \xcd`T`. 
In particular, \xcd`T <: x10.util.Ordered[T]` is
true if values of type \xcd`T` can be compared to other values of type
\xcd`T`.  So, if we wish to define a generic class \xcd`OrderedList[T]`, of
lists whose elements are kept in the right order, we need the elements to be
ordered.  This is phrased as a constraint on \xcd`T`: 
%~~gen ^^^ Expressions130
% package expre.ssi.onsfgua.rde.dq.uantification;
%~~vis
\begin{xten}
class OrderedList[T]{T <: x10.util.Ordered[T]} {
  // ...
}
\end{xten}
%~~siv
%
%~~neg
\end{ex}


\section{Contains expressions}
\index{in}

\begin{bbgrammar}
       RelationalExp \:RelationalExp \xcd"in" ShiftExp & (\ref{prod:RelationalExp}) \\
\end{bbgrammar}

\xcd`in` is a binary operator, definable in \Sref{sect:operators}.  It is
conventionally used for checking containment.

\begin{ex}
The built-in type \xcd`Region` provides \xcd`in`, testing whether a
\xcd`Point` is in the region: 
%~~gen ^^^ Expressions6d2z
% package Expressions6d2z;
% class Hook { def run() {
%~~vis
\begin{xten}
assert 3 in 1..10;
assert !(10 in 1..3);
\end{xten}
%~~siv
% return true;
%}}
%~~neg

Other types can provide them as well:
%~~gen ^^^ Expressions3c4m
% package Expressions3c4m;
%~~vis
\begin{xten}
class Cont {
   operator this in (Int) = true;
   operator (String) in this = false;
   static operator (Cont) in (b:Boolean) = b;
   static def example() {
      val c:Cont = new Cont();
      assert c in 4 && !("odd" in c) && (c in true);
   }
}
\end{xten}
%~~siv
%class Hook{ def run() { Cont.example(); return true; } }
%~~neg


\end{ex}

\section{Array Constructors}
\label{sect:ArrayCtors}
\index{array!construction}
\index{array!literal}


\begin{bbgrammar}
             Primary \: 
                    \xcd"[" ArgumentList\opt \xcd"]" 
\end{bbgrammar}
%##)

X10 includes short syntactic forms for constructing one-dimensional arrays.
The shortest form is to enclose some expressions in brackets: 
%~~gen ^^^ Expressions140
% package Expressions.ArrayCtor.Primo;
% class Example {
% def example() {
%~~vis
\begin{xten}
val ints <: Array[Int](1) = [1,3,7,21];
\end{xten}
%~~siv
%}}
%~~neg

The expression \xcdmath"[e$_1$, $\ldots$, e$_n$]" produces an \Xcd{n}-element
\xcd`Array[T](1)`, where \xcd`T` is the computed common supertype (\Sref{LCA}) of the {\bf
base types} of the expressions  \xcdmath"e$_i$". 

\begin{ex}
The type of
\xcd`[0,1,2]` is \Xcd{Array[Int](1)}.    
More importantly, the type of 
\xcd`[0]` is also \xcd`Array[Int](1)`.  It is {\em not} 
\xcd`Array[Int{self==0}](1)`, even though all the elements are all 
of type \xcd`Int{self==0}`.  This is subtle but important. There are many
functions that take \xcd`Array[Int](1)`s, such as conversions to \xcd`Point`.
These functions do {\em not} take
\xcd`Array[Int{self==0}]`'s.

(Suppose that the function took \xcd`a:Array[Int](1)` and did 
the operation \xcd`a(i)=100`.   This operation is perfectly fine for
an \xcd`Array[Int](1)`, which is all the compiler knows about \xcd`a`.  
However, it is invalid for an \xcd`Array[Int{self==0}](1)`, because it assigns
a non-zero value to an element of the array, violating the type constraint
which says that all the elements are zero.  So, \xcd`Array[Int{self==0}](1)`
is not and must not be a subtype of \xcd`Array[Int](1)` --- the two types are simply unrelated.
%~~type~~`~~`~~ ~~ ^^^ Expressions150
Since there are far more uses for \xcd`Array[Int](1)` than
%~~type~~`~~`~~ ~~ ^^^ Expressions160
\xcd`Array[Int{self==0}](1)`, the compiler produces the former.)
\end{ex}



\begin{ex}
Occasionally one does actually need \xcd`Array[Int{self==0}](1)`, 
or, say, \xcd`Array[Eel{self != null}](1)`, an array of non-null \xcd`Eel`s.  
For these cases, cast one of the elements of the array to the desired type,
and the array constructor will do the right thing.  
%~~gen ^^^ Expressions170
%package Expressions.ArrayCtor.Details;
%class Eel{}
%class Example{
%def example(){
%~~vis
\begin{xten}
val zero <: Array[Int{self == 0}](1) 
          = [0];
val non1 <: Array[Int{self != 1}](1) 
          = [0 as Int{self != 1}];
val eels <: Array[Eel{self != null}](1) 
          = [ new Eel() as Eel{self != null}, new Eel(), new Eel()];
\end{xten}
%~~siv
%}}
%~~neg
\end{ex}


\section{Coercions and conversions}
\label{XtenConversions}
\label{User-definedCoercions}
\index{conversion}\index{coercion}
\index{type!conversion}\index{type!coercion}

\XtenCurrVer{} supports the following coercions and conversions.

\subsection{Coercions}

%##(CastExp
\begin{bbgrammar}
%(FROM #(prod:CastExp)#)
             CastExp \: CastExp \xcd"as" Type \\
\end{bbgrammar}
%##)


A {\em coercion} does not change object identity; a coerced object may
be explicitly coerced back to its original type through a cast. A {\em
  conversion} may change object identity if the type being converted
to is not the same as the type converted from. \Xten{} permits
user-defined conversions (\Sref{sec:user-defined-conversions}).

\paragraph{Subsumption coercion.}
A value of a subtype may be implicitly coerced to any supertype.  
\index{coercion!subsumption}

\begin{ex}
If \xcd`Child <: Person` and \xcd`val rhys:Child`, then \xcd`rhys` may be used
in any context that expects a \xcd`Person`.  For example, 
%~~gen ^^^ Expressions7f1h
% package Expressions7f1h;
% class Person{}
% class Child extends Person {}
%~~vis
\begin{xten}
class Example {
  def greet(Person) = "Hi!";
  def example(rhys: Child) {
     greet(rhys);
  }
}
\end{xten}
%~~siv
%
%~~neg

Similarly, \xcd`2` (whose innate type is \xcd`Int{self==2}`)
is usable in a context requiring a non-zero integer
(\xcd`Int{self != 0}`).  
\end{ex}

\paragraph{Explicit Coercion (Casting with \xcd"as")}

All classes and interfaces allow the use of the \xcd`as` operator for explicit
type coercion.  
Any class or
interface may be cast to any interface.  
Any interface may be cast to
any class.  Also, any interface can be cast to a struct that implements
(directly or indirectly) that interface.

\begin{ex}
In the following code, a \xcd`Person` is cast to \xcd`Childlike`.  There is
nothing in the class definition of \xcd`Person` that suggests that a
\xcd`Person` can be \xcd`Childlike`.  However, the \xcd`Person` in question,
\xcd`p`, is actually a \xcd`HappyChild` --- a subclass of \xcd`Person` --- and
is, in fact, \xcd`Childlike`.  

Similarly, the \xcd`Childlike` value \xcd`cl` is cast to \xcd`Happy`.  Though
these two interfaces are unrelated, the value of \xcd`cl` is, in fact,
\xcd`Happy`.  And the \xcd`Happy` value \xcd`hc` is cast to the class
\xcd`Child`, though there is no relationship between the two, but the actual
value is a \xcd`HappyChild`, and thus the cast is correct at runtime.

\xcd`Cyborg` is a struct rather than a class.  So, it cannot have substructs,
and all the interfaces of all \xcd`Cyborg`s are known: a \xcd`Cyborg` is
\xcd`Personable`, but not \xcd`Childlike` or \xcd`Happy`.  So, it is correct
and meaningful to cast \xcd`r` to \xcd`Personable`.  There is no way that a
cast to \xcd`Childlike` could succeed, so \xcd`r as Childlike` is a static error.

%~~gen ^^^ Expressions180
% package Types.Coercions;
%~~vis
\begin{xten}
interface Personable {}
class Person implements Personable {}
interface Childlike extends Personable {}
class Child extends Person implements Childlike {}
struct Cyborg implements Personable {}
interface Happy {}
class HappyChild extends Child implements Happy {}
class Example {
  static def example() {
    var p : Person = new HappyChild();
    val cl : Childlike = p as Childlike; // class -> interface
    val hc : Happy = cl as Happy; //        interface -> interface
    val ch : Child = hc as Child; //        interface -> class
    var r : Cyborg = Cyborg();
    val rl : Personable = r as Personable; 
    // ERROR: r as Childlike
  }
}
\end{xten}
%~~siv
% class Hook{ def run(){ Example.example(); return true; } }
%~~neg




\end{ex}


If the value coerced is not an instance of the target type,
and no coercion operators that can convert it to that type are defined, 
a \xcd"ClassCastException" is thrown.  Casting to a constrained
type may require a run-time check that the constraint is
satisfied.
\index{coercion!explicit}
\index{cast}
\index{\Xcd{as}}

\limitation{It is currently a static error, rather than the specified
\xcd`ClassCastException`, when the cast is statically determinable to be
impossible.}



\paragraph{Effects of explicit numeric coercion}
\label{sec:effects-of-explicit-numeric-coercions}

Coercing a number of one type to another type gives the best approximation of
the number in the result type, or a suitable disaster value if no
approximation is good enough.  

\begin{itemize}
\item Casting a number to a {\em wider} numeric type is safe and effective,
      and can be done by an implicit conversion as well as an explicit
%~~exp~~`~~`~~ ~~ ^^^ Expressions190
      coercion.  For example, \xcd`4 as Long` produces the \xcd`Long` value of
      4. 
\item Casting a floating-point value to an integer value truncates the digits
      after the decimal point, thereby rounding the number towards zero.  
%~~exp~~`~~`~~ ^^^ Expressions200
      \xcd`54.321 as Int` is \xcd`54`, and 
%~~exp~~`~~`~~ ~~ ^^^ Expressions210
      \xcd`-54.321 as Int` is \xcd`-54`.
      If the floating-point value is too large to represent as that kind of
      integer, the coercion returns the largest or smallest value of that type
      instead: \xcd`1e110 as Int` is 
      \xcd`Int.MAX_VALUE`, \viz{} \xcd`2147483647`. 

\item Casting a \xcd`Double` to a \xcd`Float` normally truncates binary digits: 
%~~exp~~`~~`~~ ~~ ^^^ Expressions220
      \xcd`0.12345678901234567890 as Float` is approximately \xcd`0.12345679f`.  This can
      turn a nonzero \xcd`Double` into \xcd`0.0f`, the zero of type
      \xcd`Float`: 
%~~exp~~`~~`~~ ~~ ^^^ Expressions230
      \xcd`1e-100 as Float` is \xcd`0.0f`.  Since 
      \xcd`Double`s can be as large as about \xcd`1.79E308` and \xcd`Float`s
      can only be as large as about \xcd`3.4E38f`, a large \xcd`Double` will
      be converted to the special \xcd`Float` value of \xcd`Infinity`: 
%~~exp~~`~~`~~ ~~ ^^^ Expressions240
      \xcd`1e100 as Float` is \xcd`Infinity`.
\item Integers are coerced to smaller integer types by truncating the
      high-order bits. If the value of the large integer fits into the smaller
      integer's range, this gives the same number in the smaller type: 
%~~exp~~`~~`~~ ~~ ^^^ Expressions250
      \xcd`12 as Byte` is the \xcd`Byte`-sized 12, 
%~~exp~~`~~`~~ ~~ ^^^ Expressions260
      \xcd`-12 as Byte` is -12. 
      However, if the larger integer {\em doesn't} fit in the smaller type,
%~~exp~~`~~`~~ ~~ ^^^ Expressions270
      the numeric value and even the sign can change: \xcd`254 as Byte` is
      the \xcd`Byte`sized \xcd`-2y`.  

\item Casting an unsigned integer type to a signed integer type of the same
      size (\eg, \xcd`UInt` to \xcd`Int`) preserves 2's-complement bit pattern
      (\eg,  
      \xcd`UInt.MAX_VALUE as Int == -1`.   Casting an unsigned integer type to
      a signed integer type of a different size is equivalent to first casting
      to an unsigned integer type of the target size, and then casting to a
      signed integer type.

\item Casting a signed integer type to an unsigned one is similar.  

\end{itemize}

\subsubsection{User-defined Coercions}
\index{coercion!user-defined}

Users may define coercions from arbitrary types into the container type
\xcd`B`, and coercions from \xcd`B` to arbitrary types, by providing
\xcd`static operator` definitions for the \xcd`as` operator in the definition of
\xcd`B`.  

\begin{ex}

%~~gen ^^^ Expressions2j7z
% package Expressions2j7z;
% KNOWNFAIL
%~~vis
\begin{xten}
class Bee {
  public static operator (x:Bee) as Int = 1;
  public static operator (x:Int) as Bee = new Bee();
  def example() {
    val b:Bee = 2 as Bee; 
    assert (b as Int) == 1;
  }
}
\end{xten}
%~~siv
%
%~~neg


\end{ex}



\subsection{Conversions}
\index{conversion}
\index{type!conversion}

\paragraph{Widening numeric conversion.}
\label{WideningConversions}
A numeric type may be implicitly converted to a wider numeric type. In
particular, an implicit conversion may be performed between a numeric
type and a type to its right, below:

\begin{xten}
Byte < Short < Int < Long < Float < Double
UByte < UShort < UInt < ULong
\end{xten}

Furthermore, an unsigned integer type may be implicitly coerced a signed type
large 
enough to hold any value of the type: \xcd`UByte` to \xcd`Short`, \xcd`UShort`
to \xcd`Int`, \xcd`UInt` to \xcd`Long`.  There are no implicit conversions
from signed to unsigned numbers, since they cannot treat negatives properly.

There are no implicit conversions in cases when overflow is possible.  For
example, there is no implicit conversion between \xcd`Int` and \xcd`UInt`.  If
it is necessary to convert between these types, use \xcd`n as Int` or 
\xcd`n as UInt`, generally with a test to ensure that the value will fit and
code to handle the case in which it does not.  


\index{conversion!widening}
\index{conversion!numeric}

\paragraph{String conversion.}
Any value that is an operand of the binary
\xcd"+" operator may
be converted to \xcd"String" if the other operand is a \xcd"String".
A conversion to \xcd"String" is performed by invoking the \xcd"toString()"
method.

\index{conversion!string}

\paragraph{User defined conversions.}\label{sec:user-defined-conversions}
\index{conversion!user-defined}

The user may define implicit conversion operators from type \Xcd{A} {\em to} a
container type \Xcd{B} by specifying an operator in \Xcd{B}'s definition of the form:

\begin{xten}
  public static operator (r: A): T = ... 
\end{xten}

The return type \Xcd{T} should be a subtype of \Xcd{B}. The return
type need not be specified explicitly; it will be computed in the
usual fashion if it is not. However, it is good practice for the
programmer to specify the return type for such operators explicitly.
The return type can be more specific than simply \xcd`B`, for cases when there
is more information available.


\begin{ex}
The code for \Xcd{x10.lang.Point} contains a conversion from 
one-dimensional \xcd`Array`s of integers to \xcd`Point`s of the same length: 
\begin{xten}
  public operator (r: Array[Int](1)): Point(r.length) = make(r);
\end{xten}
This conversion is used whenever an array of integers appears in a 
context that requires a \xcd`Point`, such as subscripting. Note 
that \xcd`a` requires a \xcd`Point` of rank 2 as a subscript, and that 
a two-element \xcd`Array` (like \xcd`[2,4]`) is converted to a 
\xcd`Point(2)`.
%~~gen ^^^ Expressions4f4y
% package Expressions4f4y;
% class Example { def example() {
%~~vis
\begin{xten}
val a = new Array[String]((2..3) * (4..5), "hi!");
a([2,4]) = "converted!";
\end{xten}
%~~siv
%} } 
%~~neg


\end{ex}

\section{Parenthesized Expressions}

If \xcd`E` is any expression, \xcd`(E)` is an expression which, when
evaluated, produces the same result as \xcd`E`.   

\begin{ex}
The main use of parentheses is to write complex expressions for which the 
standard precedence order of operations is not appropriate: \xcd`1+2*3` is 7,
but \xcd`(1+2)*3` is 9.  

Similarly, but perhaps less familiarly, 
parentheses can disambiguate other expressions.  In the following code, 
\xcd`funny.f` is a field-selection expression, and so \xcd`(funny.f)()` means
``select the \xcd`f` field from \xcd`funny`, and evaluate it''.  However, 
\xcd`funny.f()` means ``evaluate the \xcd`f` method on object \xcd`funny`.''  
%~~gen ^^^ Expressions3f6f
% package Expressions3f6f;
%~~vis
\begin{xten}
class Funny {
  def f () = 1;
  val f = () => 2;
  static def example() {
    val funny = new Funny();
    assert funny.f() == 1;
    assert (funny.f)() == 2;
  }
}
\end{xten}
%~~siv
% class Hook{ def run() { Funny.example(); return true; }}
%~~neg


\end{ex}

Note that this does {\em
not} mean that \xcd`E` and \xcd`(E)` are identical in all respects; for
example, if \xcd`i` is an \xcd`Int` variable, \xcd`i++` increments \xcd`i`,
but \xcd`(i)++` is not allowed.    \xcd`++` is an assignment; it operates on
variables, not merely values, and \xcd`(i)` is simply an expression whose {\em
value} is the same as that of \xcd`i`. 
	
\chapter{Statements}\label{XtenStatements}\index{statement}

This chapter describes the statements in the sequential core of
\Xten{}.  Statements involving concurrency and distribution
are described in \Sref{XtenActivities}.

\section{Empty statement}

The empty statement \xcd";" does nothing.  

\begin{ex}
Sometimes, the syntax of X10 requires a statement in some position, but you do
not actually want to do any computation there.   
The following code searches the array \xcd`a` for the value \xcd`v`, assumed
to appear somewhere in \xcd`a`, and returns the index at which it was found.  
There is no computation to do in the loop body, so we use an empty statement
there. 
%~~gen ^^^ Statements10
% package statements.emptystatement;
% class EmptyStatementExample {
%~~vis
\begin{xten}
static def search[T](a: Array[T](1), v: T):Int {
  var i : Int;
  for(i = a.region.min(0); a(i) != v; i++)
     ;
  return i;
}
\end{xten}
%~~siv
%}
%~~neg

\end{ex}

\section{Local variable declaration}
\label{sect:LocalVarDecl}
\index{variable!declaration}
\index{var}
\index{val}

%##(LocVarDecl LocVarDeclStmt VarDeclWType VarDeclsWType VariableDeclarators VariableInitializer FormalDeclarators
\begin{bbgrammar}
%(FROM #(prod:LocVarDecl)#)
          LocVarDecl \: Mods\opt VarKeyword VariableDeclarators & (\ref{prod:LocVarDecl}) \\
                     \| Mods\opt VarDeclsWType \\
                     \| Mods\opt VarKeyword FormalDeclarators \\
%(FROM #(prod:LocVarDeclStmt)#)
      LocVarDeclStmt \: LocVarDecl \xcd";" & (\ref{prod:LocVarDeclStmt}) \\
%(FROM #(prod:VarDeclWType)#)
        VarDeclWType \: Id HasResultType \xcd"=" VariableInitializer & (\ref{prod:VarDeclWType}) \\
                     \| \xcd"[" IdList \xcd"]" HasResultType \xcd"=" VariableInitializer \\
                     \| Id \xcd"[" IdList \xcd"]" HasResultType \xcd"=" VariableInitializer \\
%(FROM #(prod:VarDeclsWType)#)
       VarDeclsWType \: VarDeclWType & (\ref{prod:VarDeclsWType}) \\
                     \| VarDeclsWType \xcd"," VarDeclWType \\
%(FROM #(prod:VariableDeclarators)#)
 VariableDeclarators \: VariableDeclarator & (\ref{prod:VariableDeclarators}) \\
                     \| VariableDeclarators \xcd"," VariableDeclarator \\
%(FROM #(prod:VariableInitializer)#)
 VariableInitializer \: Exp & (\ref{prod:VariableInitializer}) \\
%(FROM #(prod:FormalDeclarators)#)
   FormalDeclarators \: FormalDeclarator & (\ref{prod:FormalDeclarators}) \\
                     \| FormalDeclarators \xcd"," FormalDeclarator \\
\end{bbgrammar}
%##)

Short-lived variables are introduced by local variables declarations, as
described in \Sref{VariableDeclarations}. Local variables may be declared only
within a block statement (\Sref{Blocks}). The scope of a local variable
declaration is the subsequent statements in the
block.   
%~~gen ^^^ Statements20
% package statements.should.have.locals;
% class LocalExample {
% def example(a:Int) {
%~~vis
\begin{xten}
  if (a > 1) {
     val b = a/2;
     var c : Int = 0;
     // b and c are defined here
  }
  // b and c are not defined here.
\end{xten}
%~~siv
%} }
%~~neg

Variables declared in such statements shadow variables of the same
name declared elsewhere.
A local variable of a given name, say \xcd`x`, cannot shadow another local
variable or parameter named \xcd`x` unless there is an intervening method,
constructor, initializer, or
closure declaration.
%%, or unless the inner \xcd`x` is declared inside an
%%\xcd`async` or \xcd`at` statement and the outer variable is declared outside
%% of that.   Strictly, \xcd`at` introduces a new scope that does not share the
%%variables of the external scope, so its variables do not actually shadow those
%%outside.  

\begin{ex}
The following code illustrates both legal and illegal uses of shadowing.
Note that a shadowed {\em field} name \xcd`x` can still be accessed 
as \xcd`this.x`. 

%%AT-COPY%% %~~gen ^^^ Statements4h6p
%%AT-COPY%% % package Statements4h6p;
%%AT-COPY%% %~~vis
%%AT-COPY%% \begin{xten}
%%AT-COPY%% class Shadow{
%%AT-COPY%%   var x : Int; 
%%AT-COPY%%   def this(x:Int) { 
%%AT-COPY%%      // Parameter can shadow field
%%AT-COPY%%      this.x = x; 
%%AT-COPY%%   }
%%AT-COPY%%   def example(y:Int) {
%%AT-COPY%%      val x = "shadows a field";
%%AT-COPY%%      // ERROR: val y = "shadows a param";
%%AT-COPY%%      val z = "local";
%%AT-COPY%%      for (a in [1,2,3]) {
%%AT-COPY%%         // ERROR: val x = "can't shadow local var";
%%AT-COPY%%      }
%%AT-COPY%%      async {
%%AT-COPY%%         val x = "can shadow through async";
%%AT-COPY%%      }        
%%AT-COPY%%      at (here;) {
%%AT-COPY%%         val x = "at gives a whole new namespace";
%%AT-COPY%%      }        
%%AT-COPY%%      val f = () => { 
%%AT-COPY%%        val x = "can shadow through closure";
%%AT-COPY%%        x
%%AT-COPY%%      };
%%AT-COPY%%   }
%%AT-COPY%% }
%%AT-COPY%% \end{xten}
%%AT-COPY%% %~~siv
%%AT-COPY%% %
%%AT-COPY%% %~~neg
%%AT-COPY%% 
%~~gen ^^^ Statements4h6p
% package Statements4h6p;
% // NOTEST-stupid-packaging-issue
%~~vis
\begin{xten}
class Shadow{
  var x : Int; 
  def this(x:Int) { 
     // Parameter can shadow field
     this.x = x; 
  }
  def example(y:Int) {
     val x = "shadows a field";
     // ERROR: val y = "shadows a param";
     val z = "local";
     for (a in [1,2,3]) {
        // ERROR: val x = "can't shadow local var";
     }
     async {
        // ERROR: val x = "can't shadow through async";
     }        
     val f = () => { 
       val x = "can shadow through closure";
       x
     };
     class Local {
        val f = at(here.next()){ val x = "can here"; x };
        def this() { val x = "can here, too"; }
     }
  }
}
\end{xten}
%~~siv
%
%~~neg



\end{ex}

\begin{ex}
Note that recursive definitions of local variables is not allowed.  There are
few useful recursive declarations of objects and structs; \xcd`x`, in the
following example, has no meaningful definition.  Recursive declarations of
local functions is forbidden, even though (like \xcd`f` below) there are
meaningful uses of it.  
\begin{xten}
val x : Int = x + 1; // ERROR: recursive local declaration
val f : (Int)=>Int 
      = (n:Int) => (n <= 2) ? 1 : f(n-1) + f(n-2);
      // ERROR: recursive local declaration
\end{xten}

\end{ex}



\section{Block statement}
\index{block}
\label{Blocks}

%##(Block BlockStatements BlockStatement
\begin{bbgrammar}
%(FROM #(prod:Block)#)
               Block \: \xcd"{" BlockStatements\opt \xcd"}" & (\ref{prod:Block}) \\
%(FROM #(prod:BlockStatements)#)
     BlockStatements \: BlockStatement & (\ref{prod:BlockStatements}) \\
                     \| BlockStatements BlockStatement \\
%(FROM #(prod:BlockStatement)#)
      BlockStatement \: LocVarDeclStmt & (\ref{prod:BlockStatement}) \\
                     \| ClassDecl \\
                     \| TypeDefDecl \\
                     \| Statement \\
\end{bbgrammar}
%##)


A block statement consists of a sequence of statements delimited by
``\xcd"{"'' and ``\xcd"}"''. When a block is evaluated, the statements inside
of it are evaluated in order.  Blocks are useful for putting several
statements in a place where X10 asks for a single one, such as the consequent
of an \xcd`if`, and for limiting the scope of local variables.
%~~gen ^^^ Statements30
% package statements.FOR.block.heads;
% class Example {
% def example(b:Boolean, S1:(Int)=>void, S2:(Int)=>void ) {
%~~vis
\begin{xten}
if (b) {
  // This is a block
  val v = 1;
  S1(v); 
  S2(v);
}
\end{xten}
%~~siv
%  } } 
%~~neg



\section{Expression statement}

Any expression may be used as a statement.

%##(ExpStatement StatementExp
\begin{bbgrammar}
%(FROM #(prod:ExpStatement)#)
        ExpStatement \: StatementExp \xcd";" & (\ref{prod:ExpStatement}) \\
%(FROM #(prod:StatementExp)#)
        StatementExp \: Assignment & (\ref{prod:StatementExp}) \\
                     \| PreIncrementExp \\
                     \| PreDecrementExp \\
                     \| PostIncrementExp \\
                     \| PostDecrementExp \\
                     \| MethodInvocation \\
                     \| ClassInstCreationExp \\
\end{bbgrammar}
%##)

The expression statement evaluates an expression. The value of the expression
is not used. Side effects of the expression occur, and may produce results
used by following statements. Indeed, statement expressions which terminate
without side effects cannot have any visible effect on the results of the
computation. 


\begin{ex}
%~~gen ^^^ Statements40
% package Sta.tem.ent.s.expressions;
% import x10.util.*;
%~~vis
\begin{xten}
class StmtEx {
  def this() { 
     x10.io.Console.OUT.println("New StmtEx made");  }
  static def call() { 
     x10.io.Console.OUT.println("call!");}
  def example() {
     var a : Int = 0;
     a = 1; // assignment
     new StmtEx(); // allocation
     call(); // call
  }
}
\end{xten}
%~~siv
%
%~~neg
\end{ex}



\section{Labeled statement}
\index{label}
\index{statement label}


\begin{bbgrammar}
    LabeledStatement \: Id \xcd":" Statement 
\end{bbgrammar}


Statements may be labeled. The label may be used to describe the target of a
\xcd`break` statement appearing within a substatement (which, when executed,
ends the labeled statement), or, in the case of a loop, a \xcd`continue` as
well (which, when executed, proceeds to the next iteration of the loop). The
scope of a label is the statement labeled.

\begin{ex}
The label on the outer \xcd`for` statement allows \xcd`continue` and
\xcd`break` statements to continue or break it.  Without the label,
\xcd`continue` or \xcd`break` would only continue or break the inner \xcd`for`
loop. 
%~~gen ^^^ Statements50
% package state.meant.labe.L;
% class Example {
% def example(a:(Int,Int) => Int, do_things_to:(Int)=>Int) {
%~~vis
\begin{xten}
lbl : for (i in 1..10) {
   for (j in i..10) {  
      if (a(i,j) == 0) break lbl;
      if (a(i,j) == 1) continue lbl;
      if (a(i,j) == a(j,i)) break lbl;
   }
}
\end{xten}
%~~siv
%} } 
%~~neg
\end{ex}

In particular, a block statement may be labeled: \xcd` L:{S}`.  This allows
the use of \xcd`break L` within \xcd`S` to leave \xcd`S`, which can, if
carefully used, avoid deeply-nested \xcd`if`s. 

\begin{ex}
%~~gen ^^^ Statements51
% package statements51;
% abstract class Example {
% abstract def phase1(String):void;
% abstract def phase2(String):void;
% abstract def phase3(String):void;
% abstract def suitable_for_phase_2(String):Boolean;
% abstract def suitable_for_phase_3(String):Boolean;
% def example(filename: String) {
% KNOWNFAIL-labelled-blocks
%~~vis
\begin{xten}
multiphase: {
  if (!exists(filename)) break multiphase;
  phase1(filename);
  if (!suitable_for_phase_2(filename)) break multiphase;
  phase2(filename);
  if (!suitable_for_phase_3(filename)) break multiphase;
  phase3(filename);
}
// Now the file has been phased as much as possible
\end{xten}
%~~siv
%}
%~~neg
\end{ex}

\limitation{Blocks cannot currently be labeled.}

\section{Break statement}
\index{break}

%##(BreakStatement
\begin{bbgrammar}
%(FROM #(prod:BreakStatement)#)
      BreakStatement \: \xcd"break" Id\opt \xcd";" & (\ref{prod:BreakStatement}) \\
\end{bbgrammar}
%##)


An unlabeled break statement exits the currently enclosing loop or switch
statement. A labeled break statement exits the enclosing 
statement with the given label.
It is illegal to break out of a statement not defined in the current
method, constructor, initializer, or closure.  
\xcd`break` is only allowed in sequential code.

\begin{ex}
The following code searches for an element of a two-dimensional
array and breaks out of the loop when it is found:
%~~gen ^^^ Statements60
% package statements.come.from.banks.and.cranks;
% class LabelledBreakeyBreakyHeart {
% def findy(a:Array[Array[Int](1)](1), v:Int): Boolean {
%~~vis
\begin{xten}
var found: Boolean = false;
outer: for (var i: Int = 0; i < a.size; i++)
    for (var j: Int = 0; j < a(i).size; j++)
        if (a(i)(j) == v) {
            found = true;
            break outer;
        }
\end{xten}
%~~siv
% return found;
%}}
%~~neg
\end{ex}

\section{Continue statement}
\index{continue}

%##(ContinueStatement
\begin{bbgrammar}
%(FROM #(prod:ContinueStatement)#)
   ContinueStatement \: \xcd"continue" Id\opt \xcd";" & (\ref{prod:ContinueStatement}) \\
\end{bbgrammar}
%##)
An unlabeled \xcd`continue` skips the rest of the current iteration of the
innermost enclosing loop, and proceeds on to the next.  A labeled
\xcd`continue` does the same to the enclosing loop with that label.
It is illegal to continue a loop not defined in the current
method, constructor, initializer, or closure.
\xcd`continue` is only allowed in sequential code.



\section{If statement}
\index{if}

%##(IfThenStatement IfThenElseStatement
\begin{bbgrammar}
%(FROM #(prod:IfThenStatement)#)
     IfThenStatement \: \xcd"if" \xcd"(" Exp \xcd")" Statement & (\ref{prod:IfThenStatement}) \\
%(FROM #(prod:IfThenElseStatement)#)
 IfThenElseStatement \: \xcd"if" \xcd"(" Exp \xcd")" Statement  \xcd"else" Statement  & (\ref{prod:IfThenElseStatement}) \\
\end{bbgrammar}
%##)

An if statement comes in two forms: with and without an else
clause.

The if-then statement evaluates a condition expression, which must be of type
\xcd`Boolean`. If the condition is \xcd`true`, it evaluates the then-clause.
If the condition is \xcd"false", the if-then statement completes normally.

The if-then-else statement evaluates a \xcd`Boolean` expression and 
evaluates the then-clause if the condition is
\xcd"true"; otherwise, the \xcd`else`-clause is evaluated.

As is traditional in languages derived from Algol, the if-statement is syntactically
ambiguous.  That is, 
\begin{xten}
if (B1) if (B2) S1 else S2
\end{xten}
could be intended to mean either 
\begin{xten}
if (B1) { if (B2) S1 else S2 }
\end{xten} 
or 
\begin{xten}
if (B1) {if (B2) S1} else S2
\end{xten}
X10, as is traditional, attaches an \xcd`else` clause to the most recent
\xcd`if` that doesn't have one.
This example is interpreted as 
\xcd`if (B1) { if (B2) S1 else S2 }`. 



\section{Switch statement}
\index{switch}

%##(SwitchStatement SwitchBlock SwitchBlockGroups SwitchBlockGroup SwitchLabels SwitchLabel
\begin{bbgrammar}
%(FROM #(prod:SwitchStatement)#)
     SwitchStatement \: \xcd"switch" \xcd"(" Exp \xcd")" SwitchBlock & (\ref{prod:SwitchStatement}) \\
%(FROM #(prod:SwitchBlock)#)
         SwitchBlock \: \xcd"{" SwitchBlockGroups\opt SwitchLabels\opt \xcd"}" & (\ref{prod:SwitchBlock}) \\
%(FROM #(prod:SwitchBlockGroups)#)
   SwitchBlockGroups \: SwitchBlockGroup & (\ref{prod:SwitchBlockGroups}) \\
                     \| SwitchBlockGroups SwitchBlockGroup \\
%(FROM #(prod:SwitchBlockGroup)#)
    SwitchBlockGroup \: SwitchLabels BlockStatements & (\ref{prod:SwitchBlockGroup}) \\
%(FROM #(prod:SwitchLabels)#)
        SwitchLabels \: SwitchLabel & (\ref{prod:SwitchLabels}) \\
                     \| SwitchLabels SwitchLabel \\
%(FROM #(prod:SwitchLabel)#)
         SwitchLabel \: \xcd"case" ConstantExp \xcd":" & (\ref{prod:SwitchLabel}) \\
                     \| \xcd"default" \xcd":" \\
\end{bbgrammar}
%##)

A switch statement evaluates an index expression and then branches to
a case whose value is equal to the value of the index expression.
If no such case exists, the switch branches to the 
\xcd"default" case, if any.

Statements in each case branch are evaluated in sequence.  At the
end of the branch, normal control-flow falls through to the next case, if
any.  To prevent fall-through, a case branch may be exited using
a \xcd"break" statement.

The index expression must be of type \xcd"Int".
Case labels must be of type \xcd"Int", \xcd`Byte`, or \xcd`Short`, 
and must be compile-time 
constants.  Case labels cannot be duplicated within the
\xcd"switch" statement.

\begin{ex}
In this \xcd`switch`, case \xcd`1` falls through to case \xcd`2`.  The
other cases are separated by \xcd`break`s.
%~~gen ^^^ Statements70
% package Statement.Case;
% class Example {
% def example(i : Int, println: (String)=>void) {
%~~vis
\begin{xten}
switch (i) {
  case 1: println("one, and ");
  case 2: println("two"); 
          break;
  case 3: println("three");
          break;
  default: println("Something else");
           break;
}
\end{xten}
%~~siv
% } } 
%~~neg
\end{ex}

\section{While statement}
\index{while}

%##(WhileStatement
\begin{bbgrammar}
%(FROM #(prod:WhileStatement)#)
      WhileStatement \: \xcd"while" \xcd"(" Exp \xcd")" Statement & (\ref{prod:WhileStatement}) \\
\end{bbgrammar}
%##)

A while statement evaluates a \xcd`Boolean`-valued condition and executes a
loop body if \xcd"true". If the loop body completes normally (either by
reaching the end or via a \xcd"continue" statement with the loop header as
target), the condition is reevaluated and the loop repeats if \xcd"true". If
the condition is \xcd"false", the loop exits.

\begin{ex}
A loop to execute the process in the Collatz conjecture (a.k.a. 3n+1 problem,
Ulam conjecture, Kakutani's problem, Thwaites conjecture, Hasse's algorithm,
and Syracuse problem) can be written as follows:
%~~gen ^^^ Statements80
% package Statements.AreFor.Flatements;
% class Example {
% def example(var n:Int) {
%~~vis
\begin{xten}
  while (n > 1) {
     n = (n % 2 == 1) ? 3*n+1 : n/2;
  }
\end{xten}
%~~siv
% } } 
%~~neg
\end{ex}
\section{Do--while statement}
\index{do}

%##(DoStatement
\begin{bbgrammar}
%(FROM #(prod:DoStatement)#)
         DoStatement \: \xcd"do" Statement \xcd"while" \xcd"(" Exp \xcd")" \xcd";" & (\ref{prod:DoStatement}) \\
\end{bbgrammar}
%##)


A \Xcd{do-while} statement executes the loop body, and then evaluates a
\xcd`Boolean`-valued condition expression. If \xcd"true", the loop repeats.
Otherwise, the loop exits.


\section{For statement}
\index{for}

%##(ForStatement BasicForStatement ForInit ForUpdate StatementExpList EnhancedForStatement
\begin{bbgrammar}
%(FROM #(prod:ForStatement)#)
        ForStatement \: BasicForStatement & (\ref{prod:ForStatement}) \\
                     \| EnhancedForStatement \\
%(FROM #(prod:BasicForStatement)#)
   BasicForStatement \: \xcd"for" \xcd"(" ForInit\opt \xcd";" Exp\opt \xcd";" ForUpdate\opt \xcd")" Statement & (\ref{prod:BasicForStatement}) \\
%(FROM #(prod:ForInit)#)
             ForInit \: StatementExpList & (\ref{prod:ForInit}) \\
                     \| LocVarDecl \\
%(FROM #(prod:ForUpdate)#)
           ForUpdate \: StatementExpList & (\ref{prod:ForUpdate}) \\
%(FROM #(prod:StatementExpList)#)
    StatementExpList \: StatementExp & (\ref{prod:StatementExpList}) \\
                     \| StatementExpList \xcd"," StatementExp \\
%(FROM #(prod:EnhancedForStatement)#)
EnhancedForStatement \: \xcd"for" \xcd"(" LoopIndex \xcd"in" Exp \xcd")" Statement & (\ref{prod:EnhancedForStatement}) \\
                     \| \xcd"for" \xcd"(" Exp \xcd")" Statement \\
\end{bbgrammar}
%##)

\xcd`for` statements provide bounded iteration, such as looping over a list.
It has two forms: a basic form allowing near-arbitrary iteration, {\em a la}
C, and an enhanced form designed to iterate over a collection.

A basic \xcd`for` statement provides for arbitrary iteration in a somewhat
more organized fashion than a \xcd`while`.  The loop 
\xcd`for(init; test; step)body` is
similar to: 
\begin{xten}
{
   init;
   while(test) {
      body;
      step;
   }
}
\end{xten}
\noindent
except that \xcd`continue` statements which continue the \xcd`for` loop will
perform the \xcd`step`, which, in the \xcd`while` loop, they will not do. 

\xcd`init` is performed before the loop, and is traditionally used to declare
and/or initialize the loop variables. It may be a single variable binding
statement, such as \xcd`var i:Int = 0` or \xcd`var i:Int=0, j:Int=100`. (Note
that a single variable binding statement may bind multiple variables.)
Variables introduced by \xcd`init` may appear anywhere in the \xcd`for`
statement, but not outside of it.  Or, it may be a sequence of expression
statements, such as \xcd`i=0, j=100`, operating on already-defined variables.
If omitted, \xcd`init` does nothing.

\xcd`test` is a Boolean-valued expression; an iteration of the loop will only
proceed if \xcd`test` is true at the beginning of the loop, after \xcd`init`
on the first iteration or after \xcd`step` on later ones. If omitted, \xcd`test`
defaults to \xcd`true`, giving a loop that will run until stopped by some
other means such as \xcd`break`, \xcd`return`, or \xcd`throw`.

\xcd`step` is performed after the loop body, between one iteration and the
next. It traditionally updates the loop variables from one iteration to the
next: \eg, \xcd`i++` and \xcd`i++,j--`.  If omitted, \xcd`step` does nothing.

\xcd`body` is a statement, often a code block, which is performed whenever
\xcd`test` is true.  If omitted, \xcd`body` does nothing.




\label{ForAllLoop}


An enhanced for statement is used to iterate over a collection, or other
structure designed to support iteration by implementing the interface
\xcd`Iterable[T]`.    The loop variable must be of type \xcd`T`, 
or destructurable from a value of type \xcd`T`
(\Sref{exploded-syntax}).  
Each iteration of the loop
binds the iteration variable to another element of the collection.
The loop \xcd`for(x in c)S` behaves like: 
%~~gen ^^^ Statements5e4u
% package Statements5e4u;
% class ForAll {
% def forall[T](c:Iterable[T], S: () => void) {
%~~vis
\begin{xten}
val iterator: Iterator[T] = c.iterator();
while (iterator.hasNext()) {
  val x : T = iterator.next();
  S();
}
\end{xten}
%~~siv
%} }
%~~neg

A number of library classes implement \xcd`Iterable`, and thus can be iterated
over.  For example, iterating over a \xcd`Region` iterates the \xcd`Point`s in
the region, and iterating over an \xcd`Array` iterates over the
\xcd`Point`s at which the  array is defined.

The type of the loop variable may be supplied as \xcd`x <: T`.  In this case
the iterable \xcd`c` must have type \xcd`Iterable[U}` for some \xcd`U <: T`,
and \xcd`x` will be given the type \xcd`U`.

\begin{ex}
This loop adds up the elements of a \xcd`List[Int]`.
Note that iterating over a list yields the elements of the list, as specified
in the \xcd`List` API. 
%~~gen ^^^ Statements3d9l
% package Statements3d9l;
% class Example {
%~~vis
\begin{xten}
static def sum(a:x10.util.List[Int]):Int {
  var s : Int = 0;
  for(x in a) s += x;
  return s;
}
\end{xten}
%~~siv
%}
%~~neg

The following code sums the elements of an integer array.  Note that the
\xcd`for` loop iterates over the indices of the array, not the elements, as
specified in the \xcd`Array` API.  
%~~gen ^^^ Statements2d4h
% package Statements2d4h;
% class Example { 
%~~vis
\begin{xten}
static def sum(a: Array[Int]): Int {
  var s : Int = 0;
  for(p in a) s += a(p);
  return s;
}
\end{xten}
%~~siv
%}
%~~neg

Iteration over an \xcd`IntRange` (\Sref{sect:intrange}) is quite common. This
allows looping while varying an integer index: 
%~~gen ^^^ Statements3o9s
% package Statements3o9s;
% class Example { static def example() {
%~~vis
\begin{xten}
var sum : Int = 0;
for(i in 1..10) sum += i;
assert sum == 55;
\end{xten}
%~~siv
%} } 
% class Hook { def run() { Example.example(); return true; } }
%~~neg


\end{ex}

Iteration variables have the \xcd`for` statement as scope.  They shadow other
variables of the same names.


\section{Return statement}
\label{ReturnStatement}
\index{return}

%##(ReturnStatement
\begin{bbgrammar}
%(FROM #(prod:ReturnStatement)#)
     ReturnStatement \: \xcd"return" Exp\opt \xcd";" & (\ref{prod:ReturnStatement}) \\
\end{bbgrammar}
%##)

Methods and closures may return values using a return statement.
If the method's return type is explicitely declared \xcd"void",
the method must return without a value; otherwise, it must return
a value of the appropriate type.

\begin{ex}
The following code illustrates returning values from a closure and a method.
The \xcd`return` inside of \xcd`closure` returns from \xcd`closure`, not from
\xcd`method`.  
%~~gen ^^^ Statements2j1d
% package Statements2j1d;
% class Example {
%~~vis
\begin{xten}
def method(x:Int) {
  val closure = (y:Int) => {return x+y;}; 
  val res = closure(0);
  assert res == x;
  return res == x;
}
\end{xten}
%~~siv
%}
%~~neg


\end{ex}


\section{Assert statement} 
\index{assert}

%##(AssertStatement
\begin{bbgrammar}
%(FROM #(prod:AssertStatement)#)
     AssertStatement \: \xcd"assert" Exp \xcd";" & (\ref{prod:AssertStatement}) \\
                     \| \xcd"assert" Exp  \xcd":" Exp  \xcd";" \\
\end{bbgrammar}
%##)

The statement \xcd`assert E` checks that the Boolean expression \xcd`E`
evaluates to true, and, if not, throws an \xcd`x10.lang.Error`  exception.  
The annotated assertion statement \xcd`assert E : F;` checks \xcd`E`, and, if
it is 
false, throws an \xcd`x10.lang.Error` exception with \xcd`F`'s value attached
to it. 

\begin{ex}
The following code compiles properly.  
%~~gen ^^^ Statements100
% package Statements.Assert;
% 
%~~vis
\begin{xten}
class Example {
  public static def main(argv:Array[String](1)) {
    val a = 1;
    assert a != 1 : "Changed my mind about a.";
  }
}
\end{xten}
%~~siv
%~~neg
\noindent
However, when run, it 
prints a stack trace starting with 
\begin{xten}
x10.lang.Error: Changed my mind about a.
\end{xten}
\end{ex}

\section{Exceptions in X10}
\index{exception}
\index{termination!abrupt}
\index{termination!normal}

X10 programs can throw {\em Exceptions} to indicate unusual or problematic
situations; this is {\em abrupt termination}.  Exceptions, as data values, are
objects which which inherit from 
\xcd`x10.lang.Throwable`.    Exceptions may be thrown intentionally with the
\xcd`throw` statement. Many primitives and library functions throw exceptions
if they encounter problems; \eg, dividing by zero throws an instance of
\xcd`x10.lang.ArithmeticException`. 

When an exception is thrown, statically and dynamically enclosing
\xcd`try`-\xcd`catch` blocks in the same activity can attempt to handle it.   If the throwing
statement in inside some \xcd`try` clause, and some matching \xcd`catch`
clause catches that type of exception, the corresponding \xcd`catch` body will
be executed, and the process of throwing is finished.  
If no statically-enclosing \xcd`try`-\xcd`catch` block can handle the
exception, the current method call returns (abnormally), throwing the same
exception from the point at which the method was called.  

This process continues until the exception is handled or there are no more
calling methods in the activity. In the latter case, the activity will
terminate abnormally, and the exception will propagate to the activity's root;
see \Sref{ExceptionModel} for details.

Unlike some statically-typed languages with exceptions, X10's exceptions are
all {\em unchecked}. Methods do not declare which exceptions they might throw;
any method can, potentially, throw any exception.


\section{Throw statement}
\index{throw}

%##(ThrowStatement
\begin{bbgrammar}
%(FROM #(prod:ThrowStatement)#)
      ThrowStatement \: \xcd"throw" Exp \xcd";" & (\ref{prod:ThrowStatement}) \\
\end{bbgrammar}
%##)

\index{Exception}
\xcd"throw E" throws an exception whose value is \xcd`E`, which must be an
instance of a subtype of \xcd`x10.lang.Throwable`. 

\begin{ex}
The following code checks if an index is in range and
throws an exception if not.

%~~gen ^^^ Statements110
% package Statements_index_check;
% class ThrowStatementExample {
% def thingie(i:Int, x:Array[Boolean](1))  {
%~~vis
\begin{xten}
if (i < 0 || i >= x.size)
    throw new MyIndexOutOfBoundsException();
\end{xten}
%~~siv
%} }
% class MyIndexOutOfBoundsException extends Exception {}
%~~neg
\end{ex}

\section{Try--catch statement}
\index{try}
\index{catch}
\index{finally}
\index{exception}

%##(TryStatement Catches CatchClause Finally
\begin{bbgrammar}
%(FROM #(prod:TryStatement)#)
        TryStatement \: \xcd"try" Block Catches & (\ref{prod:TryStatement}) \\
                     \| \xcd"try" Block Catches\opt Finally \\
%(FROM #(prod:Catches)#)
             Catches \: CatchClause & (\ref{prod:Catches}) \\
                     \| Catches CatchClause \\
%(FROM #(prod:CatchClause)#)
         CatchClause \: \xcd"catch" \xcd"(" Formal \xcd")" Block & (\ref{prod:CatchClause}) \\
%(FROM #(prod:Finally)#)
             Finally \: \xcd"finally" Block & (\ref{prod:Finally}) \\
\end{bbgrammar}
%##)

Exceptions are handled with a \xcd"try" statement.
A \xcd"try" statement consists of a \xcd"try" block, zero or more
\xcd"catch" blocks, and an optional \xcd"finally" block.

First, the \xcd"try" block is evaluated.  If the block throws an
exception, control transfers to the first matching \xcd"catch"
block, if any.  A \xcd"catch" matches if the value of the
exception thrown is a subclass of the \xcd"catch" block's formal
parameter type.

The \xcd"finally" block, if present, is evaluated on all normal
and exceptional control-flow paths from the \xcd"try" block.
If the \xcd"try" block completes normally
or via a \xcd"return", a \xcd"break", or a
\xcd"continue" statement, 
the \xcd"finally"
block is evaluated, and then control resumes at
the statement following the \xcd"try" statement, at the branch target, or at
the caller as appropriate.
If the \xcd"try" block completes
exceptionally, the \xcd"finally" block is evaluated after the
matching \xcd"catch" block, if any, and when and if the \xcd`finally` block
finishs normally, the
exception is rethrown.


The parameter of a \xcd`catch` block has the block as scope.  It shadows other
variables of the same name.

\begin{ex}
The \xcd`example()` method below executes without any assertion errors
%~~gen ^^^ Statements9x3m
% package Statements9x3m;
% 
%~~vis
\begin{xten}
class Example {
  class SeriousExn extends Throwable {}
  class SillyExn   extends Throwable {}
  var didFinally : Boolean = false;
  def example() {
    try {
       throw new SillyExn();
       assert false; // This cannot happen
    }
    catch(SillyExn)   {return true;}
    catch(SeriousExn) {return false;}
    finally {
       this.didFinally = true;
    }
    return false;
  }
  static def doExample() {
    val e = new Example();
    assert e.example();
    assert e.didFinally == true;
  }
}
\end{xten}
%~~siv
% 
% class Hook { def run() { Example.doExample(); return true; } }
%~~neg

\end{ex}

\limitation{Constraints on exception types in \xcd`catch` blocks are not
currently supported. 
}

\section{Assert}

The \xcd`assert` statement 
%~~stmt~~`~~`~~B:Boolean ~~
\xcd`assert B;` 
checks that the Boolean expression \xcd`B` evaluates to true.  If so,
computation proceeds.  If not, it throws \xcd`x10.lang.AssertionError`.

The extended form 
%~~stmt~~`~~`~~B:Boolean, A:Any ~~ 
\xcd`assert B:A;`
is similar, but provides more debugging information.  The value of the
expression \xcd`A` is available as part of the \xcd`AssertionError`, \eg, to
be printed on the console.

\begin{ex}
\xcd`assert` is useful for confirming properties that you believe to be true
and wish to rely on.  In particular, well-chosen \xcd`assert`s make a program
robust in the face of code changes and unexpected uses of methods.
For example, the following method compute percent differences, but asserts
that it is not dividing by zero.  If the mean is zero, it throws an exception,
including the values of the numbers as potentially useful debugging
information. 
%~~gen ^^^ StmtAssert10
%package StmtAssert10;
% class Example {
%~~vis
\begin{xten}
static def percentDiff(x:Double, y:Double) {
  val diff = x-y;
  val mean = (x+y)/2;
  assert mean != 0.0  : [x,y]; 
  return Math.abs(100 * (diff / mean));
}
\end{xten}
%~~siv
% }
%~~neg

\end{ex}


At times it may be considered important not to check \xcd`assert` statements;
\eg, if the test is expensive and the code is sufficiently well-tested.  The
\xcd`-noassert` command line option causes the compiler to ignore all
\xcd`assert` statements. 
	

\chapter{Places}
\label{XtenPlaces}
\index{place}

An \Xten{} place is a repository for data and activities, corresponding
loosely to a process or a processor. Places induce a concept of ``local''. The
activities running in a place may access data items located at that place with
the efficiency of on-chip access. Accesses to remote places may take orders of
magnitude longer. X10's system of places is designed to make this obvious.
Programmers are aware of the places of their data, and know when they are
incurring communication costs, but the actual operation to do so is easy. It's
not hard to use non-local data; it's simply hard to to do so accidentally.

The set of places available to a computation is determined at the time that
the program is started, and remains fixed through the run of the program. See
the {\tt README} documentation on how to set command line and configuration
options to set the number of places.

Places are first-class values in X10, as instances 
\xcd"x10.lang.Place".   \xcd`Place` provides a number of useful ways to
query places, such as \xcd`Place.places`, which is a  \xcd`Sequence[Place]` of 
the places
available to the current run of the program.

Objects and structs (with one exception) are created in a single place -- the
place that the constructor call was running in. They cannot change places.
They can be {\em copied} to other places, and the special library struct
\Xcd{GlobalRef} allows values at one place to point to values at another.  

\section{The Structure of Places}
\index{place!MAX\_PLACES}
\index{place!FIRST\_PLACES}
\index{MAX\_PLACES}
\index{FIRST\_PLACE}

%~~exp~~`~~`~~ ~~ ^^^ Places10
Places are numbered 0 through \xcd`Place.MAX_PLACES-1`; the number is stored
in the field 
\xcd`pl.id`.  The \xcd`Sequence[Place]` \xcd`Place.places()` contains the places of the
program, in numeric order. 
The program starts by executing a \xcd`main` method at
%~~exp~~`~~`~~ ~~ ^^^ Places20
\xcd`Place.FIRST_PLACE`, which is 
%~~exp~~`~~`~~ ~~ ^^^ Placesoik
\xcd`Place.places()(0)`; see
\Sref{initial-computation}. 

Operations on places include \xcd`pl.next()`, which gives the next entry
(looping around) in \xcd`Place.places` and its opposite \xcd`pl.prev()`. 
In multi-place executions, 
\xcd`here.next()` is a convenient way to express ``a place other than \xcd`here`''.
There are also tests, like  
%~~exp~~`~~`~~pl:Place ~~ ^^^ Placesoid
\xcd`pl.isCUDA()`, which test for particular kinds of processors.




\section{{\tt here}}\index{here}\label{Here}

The variable \xcd"here" is always bound to the place at which the current
computation is running, in the same way that \xcd`this` is always bound to the
instance of the current class (for non-static code), or \xcd`self` is bound to
the instance of the type currently being constrained.  
\xcd`here` may denote different places in the same method body or even the
same expression, due to
place-shifting operations.


This is not unusual for automatic variables:  \Xcd{self} denotes 
two different values (one \xcd`List`, one \xcd`Int`) 
when one describes a non-null list of non-zero numbers as
\xcd`List[Int{self!=0}]{self!=null}`. In the following 
code, \xcd`here` has one value at 
\xcd`h0`, and a different one at \xcd`h1` (unless there is only one place).
%~~gen ^^^ Placesoijo
% package places.are.For.Graces;
% class Example {
% def example() {
%~~vis
\begin{xten}
val h0 = here;
at (here.next()) {
  val h1 = here; 
  assert (h0 != h1);
}
\end{xten}
%~~siv
%} } 
% 
%~~neg
\noindent
(Similar examples show that \xcd`self` and \xcd`this` have the same behavior:
\xcd`self` can be shadowed by constrained types appearing inside of type
constraints, and \xcd`this` by inner classes.)



The following example looks through a list of references to \Xcd{Thing}s.  
It finds those references to things that are \Xcd{here}, and deals with them.  
%~~gen ^^^ Places70
%package Places.Are.For.Graces.2;
%import x10.util.*;
%abstract class Thing {}
%class DoMine {
%  static def dealWith(Thing) {}	
%~~vis
\begin{xten}
  public static def deal(things: List[GlobalRef[Thing]]) {
     for(gr in things) {
        if (gr.home == here) {
           val grHere = 
               gr as GlobalRef[Thing]{gr.home == here};
           val thing <: Thing = grHere();
           dealWith(thing);
        }
     }
  }
\end{xten}
%~~siv
%}
% 
%~~neg

\section{ {\tt at}: Place Changing}\label{AtStatement}
\index{at}
\index{place!changing}

An activity may change place synchronously using the \xcd"at" statement or
\xcd"at" expression. Like any distributed operation, it is 
potentially expensive, as it requires, at a minimum, two messages
and the copying of all data used in the operation, and must be used with care
-- but it provides the basis for multicore programming in X10.

%##(AtStatement AtExp
\begin{bbgrammar}
%(FROM #(prod:AtStmt)#)
              AtStmt \: \xcd"at" \xcd"(" Exp \xcd")" Stmt & (\ref{prod:AtStmt}) \\
%(FROM #(prod:AtExp)#)
               AtExp \: \xcd"at" \xcd"(" Exp \xcd")" ClosureBody & (\ref{prod:AtExp}) \\
\end{bbgrammar}
%##)

The {\it PlaceExp} must be an expression of type \xcd`Place` or some
subtype. For programming convenience, if {\it PlaceExp} is of type
\xcd`GlobalRef[T]` then the \xcd'home' property of \xcd'GlobalRef' is
used as the value of {\it PlaceExp}.

%%AT-COPY%% The \xcd`at`-statment \xcd`at(p;F)S` first evaluates \xcd`p` to a place, then
%%AT-COPY%% copies information to that place as determined by \xcd`F`, and then executes
%%AT-COPY%% \xcd`S` using the resulting copies.  The \xcd`at`-{\em expression}
%%AT-COPY%% \xcd`at(p;F)E` is similar, but it copies the result of the expression \xcd`E`
%%AT-COPY%% and returns the copy as its result.
%%AT-COPY%% 
%%AT-COPY%% The clause \xcd`F` in \xcd`at(p;F)S` is a list of zero or more {\em copy
%%AT-COPY%% specifiers}, explaining what values are to be copied to the place \xcd`p`, and
%%AT-COPY%% how they are to be referred to at \xcd`p`.  
%%AT-COPY%% 

%%AT-COPY%% \begin{ex}
%%AT-COPY%% The following example creates a rail \xcd`a` located \xcd`here`, and copies
%%AT-COPY%% it to another place, giving the copy the name \xcd`a2` there.  The copy is
%%AT-COPY%% modified and examined.  After the \xcd`at` finishes, the original is also
%%AT-COPY%% examined, and (since only the copy, not the original, was modified) is observed
%%AT-COPY%% to be unchanged. 
%%AT-COPY%% %~x~gen ^^^ Places6e1o
%%AT-COPY%% % package Places6e1o;
%%AT-COPY%% % KNOWNFAIL-at
%%AT-COPY%% % class Example { static def example() { 
%%AT-COPY%% %~x~vis
%%AT-COPY%% \begin{xten}
%%AT-COPY%% val a = [1,2,3];
%%AT-COPY%% at(here.next(); a2 = a) {
%%AT-COPY%%   a2(1) = 4;
%%AT-COPY%%   assert a2(0)==1 && a2(1)==4 && a2(2)==3; 
%%AT-COPY%%   // 'a' is not accessible here
%%AT-COPY%% }
%%AT-COPY%% assert  a(0)==1 && a(1)==2 && a(2)==3; 
%%AT-COPY%% \end{xten}
%%AT-COPY%% %~x~siv
%%AT-COPY%% %} } 
%%AT-COPY%% % class Hook { def run() { Example.example(); return true; }}
%%AT-COPY%% %~x~neg
%%AT-COPY%% \end{ex}
%%AT-COPY%% 

\begin{ex}
The following example creates a rail \xcd`a` located \xcd`here`, and copies
it to another place.  \xcd`a` in the second place (\xcd`here.next()`) refers
to the copy.  The copy is
modified and examined.  After the \xcd`at` finishes, the original is also
examined, and (since only the copy, not the original, was modified) is observed
to be unchanged. 
%~~gen ^^^ Places6e1o
% package Places6e1o;
% KNOWNFAIL-at
% class Example { static def example() { 
%~~vis
\begin{xten}
val a = [1,2,3];
at(here.next()) {
  a(1) = 4;
  assert a(0)==1 && a(1)==4 && a(2)==3; 
}
assert  a(0)==1 && a(1)==2 && a(2)==3; 
\end{xten}
%~~siv
%} } 
% class Hook { def run() { Example.example(); return true; }}
%~~neg
\end{ex}

%%AT-COPY%% \subsection{Copy Specifiers}
%%AT-COPY%% \label{sect:copy-spec}
%%AT-COPY%% \index{copy specifier}
%%AT-COPY%% \index{at!copy specifier}
%%AT-COPY%% 
%%AT-COPY%% A single copy specifier can be one of the following forms.   
%%AT-COPY%% Each copy specifier determines an {\em original-expression}, saying what value
%%AT-COPY%% will be copied, and a {\em target variable}, saying what it will be called.
%%AT-COPY%% 
%%AT-COPY%% \begin{itemize}
%%AT-COPY%% 
%%AT-COPY%% \item \xcd`val x = E`, and its usual variants \xcd`val x:T = E`, 
%%AT-COPY%%       \xcd`x : T = E`, and 
%%AT-COPY%%       \xcd`val x <: T = E`, evaluate the expression \xcd`E` at the initial
%%AT-COPY%%       place, copy it to \xcd`p`, and bind \xcd`x` to the copy, as normal for a
%%AT-COPY%%       local \xcd`val` binding.  If a type is supplied, it is checked
%%AT-COPY%%       statically in the usual way.  
%%AT-COPY%%       The original-expression is \xcd`E`, and the target variable is \xcd`x`.
%%AT-COPY%% 
%%AT-COPY%% \begin{ex}
%%AT-COPY%% The following code copies a variable \xcd`a` located \xcd`here` to a variable
%%AT-COPY%% \xcd`d` located \xcd`there`.  
%%AT-COPY%% Note that, while the copy \xcd`d` is available \xcd`there` inside of the \xcd`at`-block,
%%AT-COPY%% the original \xcd`a` is not.  (\xcd`a` could not be available in the block in
%%AT-COPY%% any case; it is not located \xcd`there`.)
%%AT-COPY%% %~~gen ^^^ Places9v2e1
%%AT-COPY%% % package Places9v2e1;
%%AT-COPY%% % KNOWNFAIL-at
%%AT-COPY%% % class Example{ 
%%AT-COPY%% % static def use(Any) = 1;
%%AT-COPY%% % static def example() { 
%%AT-COPY%% %  val there = here.next();
%%AT-COPY%% %~~vis
%%AT-COPY%% \begin{xten}
%%AT-COPY%% var a : Int = 1;
%%AT-COPY%% at(there; val d = a) {
%%AT-COPY%%    assert d == 1;
%%AT-COPY%%    // ERROR: assert a == 1;
%%AT-COPY%% }
%%AT-COPY%% \end{xten}
%%AT-COPY%% %~~siv
%%AT-COPY%% % } } 
%%AT-COPY%% % class Hook{ def run() {Example.example(); return true;}}
%%AT-COPY%% %~~neg
%%AT-COPY%% \end{ex}
%%AT-COPY%% 
%%AT-COPY%% \item \xcd`var x : T = E` evaluates \xcd`E` at the initial place, copies it to
%%AT-COPY%%       \xcd`p`, and binds \xcd`x` to a new \xcd`var` whose initial value is the
%%AT-COPY%%       copy, as normal for a local \xcd`var` binding.
%%AT-COPY%%       If a type is supplied, it is checked
%%AT-COPY%%       statically in the usual way.
%%AT-COPY%%       The original-expression is \xcd`E`, and the target variable is \xcd`x`.
%%AT-COPY%%       Note that, like a \xcd`var` parameter to a method, \xcd`x` is a local
%%AT-COPY%%       variable.  Changes to \xcd`x` will not change anything else. In
%%AT-COPY%%       particular, even if \xcd`x` has the same name as a \xcd`var` variable
%%AT-COPY%%       outside, the two \xcd`var`s are unconnected.  
%%AT-COPY%%       See \Sref{sect:athome} for the way to modify a variable from the
%%AT-COPY%%       surrounding scope.
%%AT-COPY%% 
%%AT-COPY%% \begin{ex}
%%AT-COPY%% The following code copies \xcd`a` to a \xcd`var` named \xcd`e`.  Changing
%%AT-COPY%% \xcd`e` does not change \xcd`a`; the two \xcd`var`s have no ongoing relationship.
%%AT-COPY%% %~~gen ^^^ Places9v2e2
%%AT-COPY%% % package Places9v2e2;
%%AT-COPY%% % KNOWNFAIL-at
%%AT-COPY%% % class Example{ 
%%AT-COPY%% % static def use(Any) = 1;
%%AT-COPY%% % static def example() { 
%%AT-COPY%% %  val there = here.next();
%%AT-COPY%% %~~vis
%%AT-COPY%% \begin{xten}
%%AT-COPY%% var a : Int = 1;
%%AT-COPY%% assert a == 1;
%%AT-COPY%% at(there; var e = a) { 
%%AT-COPY%%    assert e == 1;
%%AT-COPY%%    e += 1;
%%AT-COPY%%    assert e == 2;
%%AT-COPY%% }
%%AT-COPY%% assert a == 1; 
%%AT-COPY%% \end{xten}
%%AT-COPY%% %~~siv
%%AT-COPY%% % 
%%AT-COPY%% % }  } 
%%AT-COPY%% % class Hook{ def run() {Example.example(); return true;}}
%%AT-COPY%% %~~neg
%%AT-COPY%% \end{ex}
%%AT-COPY%% 
%%AT-COPY%% \item \xcd`x = E`, as a copy specifier, is equivalent to \xcd`val x = E`.
%%AT-COPY%%       Note that this abbreviated form is not available as a local variable
%%AT-COPY%%       definition, (because it is used as an assignment statement), but in a
%%AT-COPY%%       copy specifier there are no assignment statements and so the
%%AT-COPY%%       abbreviation is allowed.
%%AT-COPY%%       The original-expression is \xcd`E`, and the target variable is \xcd`x`.
%%AT-COPY%% 
%%AT-COPY%% \begin{ex}
%%AT-COPY%% The following code evaluates an expression \xcd`a+b(0)`.  The result of this
%%AT-COPY%% expression is stored \xcd`there`, in the \xcd`val` variable \xcd`f`, but is
%%AT-COPY%% not stored \xcd`here`. 
%%AT-COPY%% %~~gen ^^^ Places9v2e3
%%AT-COPY%% % package Places9v2e3;
%%AT-COPY%% % KNOWNFAIL-at
%%AT-COPY%% % class Example{ 
%%AT-COPY%% % static def use(Any) = 1;
%%AT-COPY%% % static def example() { 
%%AT-COPY%% %  val there = here.next();
%%AT-COPY%% %~~vis
%%AT-COPY%% \begin{xten}
%%AT-COPY%% var a : Int = 1;
%%AT-COPY%% var b : Rail[Int] = [2,3,4];
%%AT-COPY%% at(there; f = a + b(0)) {
%%AT-COPY%%    assert f == 3;
%%AT-COPY%% }
%%AT-COPY%% \end{xten}
%%AT-COPY%% %~~siv
%%AT-COPY%% % }  } 
%%AT-COPY%% % class Hook{ def run() {Example.example(); return true;}}
%%AT-COPY%% % 
%%AT-COPY%% %~~neg
%%AT-COPY%% 
%%AT-COPY%% 
%%AT-COPY%% \end{ex}
%%AT-COPY%% 
%%AT-COPY%% \item \xcd`x` alone, as a copy specifier, is equivalent to \xcd`val x = x`.
%%AT-COPY%%       It says that the variable \xcd`x` will be copied, and the copy will also
%%AT-COPY%%       be named \xcd`x`.  
%%AT-COPY%%       The original-expression is \xcd`x`, and the target variable is \xcd`x`.
%%AT-COPY%% 
%%AT-COPY%% \begin{ex}
%%AT-COPY%% The following code copies \xcd`b` to \xcd`there`.  The copy is also called
%%AT-COPY%% \xcd`b`.  The two \xcd`b`'s are not connected; \eg, changing one does not
%%AT-COPY%% change the other.
%%AT-COPY%% %~~gen ^^^ Places9v2e4
%%AT-COPY%% % package Places9v2e4;
%%AT-COPY%% % KNOWNFAIL-at
%%AT-COPY%% % class Example{ 
%%AT-COPY%% % static def use(Any) = 1;
%%AT-COPY%% % static def example() { 
%%AT-COPY%% %  val there = here.next();
%%AT-COPY%% %~~vis
%%AT-COPY%% \begin{xten}
%%AT-COPY%% var b : Rail[Int] = [2,3,4];
%%AT-COPY%% assert b(0) == 2;
%%AT-COPY%% at(there; b) {
%%AT-COPY%%   b(0) = 200;  // Modify copy of b.
%%AT-COPY%%   assert b(0) == 200;
%%AT-COPY%% }
%%AT-COPY%% assert b(0) == 2; 
%%AT-COPY%% \end{xten}
%%AT-COPY%% %~~siv
%%AT-COPY%% % 
%%AT-COPY%% % }  } 
%%AT-COPY%% % class Hook{ def run() {Example.example(); return true;}}
%%AT-COPY%% %~~neg
%%AT-COPY%% \end{ex}
%%AT-COPY%% 
%%AT-COPY%% \item A field assignment statements \xcdmath"a.fld = $E_2$", evaluates 
%%AT-COPY%%       \xcd`a` and $E_2$ on the sending side to values $v_1$ and {$v_2$}.  
%%AT-COPY%%       {$v_1$} must be an object with a mutable field \xcd`fld`.  {$v_1$} and
%%AT-COPY%%       {$v_2$} are sent to place \xcd`p`, and the field assignment is performed
%%AT-COPY%%       there.  The modified version of {$v_1$} is available as a \xcd`val`
%%AT-COPY%%       variable \xcd`a`.   The compiler may optimize this, \eg, by neglecting to
%%AT-COPY%%       deserialize \xcdmath"$v_1$.fld", and deserializing {$v_2$} directly into
%%AT-COPY%%       that field rather than into a separate buffer.
%%AT-COPY%% 
%%AT-COPY%% \begin{ex}
%%AT-COPY%% %~~gen ^^^ Places9v2e5
%%AT-COPY%% % package Places9v2e5;
%%AT-COPY%% % KNOWNFAIL
%%AT-COPY%% % class Example {
%%AT-COPY%% % static def use(Any) = 1;
%%AT-COPY%% % static def example() { 
%%AT-COPY%% %  val there = here.next();
%%AT-COPY%% %~~vis
%%AT-COPY%% \begin{xten}
%%AT-COPY%% class Example{ 
%%AT-COPY%%    var f : Int = 1;
%%AT-COPY%%    var g : Int = 2;
%%AT-COPY%%    static def example() { 
%%AT-COPY%%       val there = here.next();
%%AT-COPY%%       val e : Example = new Example();
%%AT-COPY%%       assert e.f == 1 && e.g == 2;
%%AT-COPY%%       at(there; e.f = 3) {
%%AT-COPY%%           assert e.f == 3; && e.g == 2;
%%AT-COPY%%       }
%%AT-COPY%%       assert e.f == 1 && e.g == 2;
%%AT-COPY%%    }
%%AT-COPY%% }
%%AT-COPY%% \end{xten}
%%AT-COPY%% %~~siv
%%AT-COPY%% % class Hook{ def run() {Example.example(); return true;}}
%%AT-COPY%% %~~neg
%%AT-COPY%% %
%%AT-COPY%% \end{ex}
%%AT-COPY%% 
%%AT-COPY%% \item A rail-element assignment 
%%AT-COPY%%       \xcdmath"a($E_1$, $\ldots$, $E_n$) = $E_+$".
%%AT-COPY%%       This copies and transmits \xcd`a` as normal for a rail.  In addition,
%%AT-COPY%%       and 
%%AT-COPY%%       much like a field assignment, it also evaluates all the expressions $E_i$
%%AT-COPY%%       at the sending side to values $v_i$, and transmits them.  \xcd`a`'s value must
%%AT-COPY%%       admit a suitably-typed $n$-ary subscripting operation.  That operation
%%AT-COPY%%       is applied after the values are deserialized at \xcd`p`.  The compiler
%%AT-COPY%%       may optimize this, \eg, by neglecting to deserialize one element of the
%%AT-COPY%%       rail $v_0$, and deserializing $v_+$ directly into that location.  
%%AT-COPY%% 
%%AT-COPY%% 
%%AT-COPY%% \begin{ex}
%%AT-COPY%% The following code sends a modified \xcd`b` to \xcd`there`, while (as always)
%%AT-COPY%% keeping an unmodified version \xcd`here`.   X10 may perform optimizations to
%%AT-COPY%% avoid transmitting the original value of \xcd`b(1)`, since it will be
%%AT-COPY%% overwritten immediately in any case.
%%AT-COPY%% %~~gen ^^^ Places9v2e6
%%AT-COPY%% % package Places9v2e6;
%%AT-COPY%% % KNOWNFAIL
%%AT-COPY%% % class Example{ 
%%AT-COPY%% % static def use(Any) = 1;
%%AT-COPY%% % static def example() { 
%%AT-COPY%% %  val there = here.next();
%%AT-COPY%% %~~vis
%%AT-COPY%% \begin{xten}
%%AT-COPY%% var b = [2,3,4];
%%AT-COPY%% assert b(0) == 2 && b(1) == 3;
%%AT-COPY%% at(there; b(1) = 300) {
%%AT-COPY%%   assert b(0) == 2 && b(1) == 300;
%%AT-COPY%% }
%%AT-COPY%% assert b(0) == 2 && b(1) == 3;
%%AT-COPY%% \end{xten}
%%AT-COPY%% %~~siv
%%AT-COPY%% % 
%%AT-COPY%% %~~neg
%%AT-COPY%% % }  }
%%AT-COPY%% % class Hook{ def run() {Example.example(); return true;}}
%%AT-COPY%% \end{ex}
%%AT-COPY%% 
%%AT-COPY%% \item \xcd`*` may appear as the last copy specifier in the list, indicating
%%AT-COPY%%       that all \xcd`val` variables from outside \xcd`S` which are used in
%%AT-COPY%%       \xcd`S` should be copied. Specifically, let 
%%AT-COPY%%       \xcdmath"x$_1, \ldots, $x$_n$" be all the \xcd`val` variables defined
%%AT-COPY%%       outside of \xcd`S` 
%%AT-COPY%%       mentioned in \xcd`S`. The \xcd`*` copy specifier is equivalent to 
%%AT-COPY%%       the list of variables 
%%AT-COPY%%       \xcdmath"x$_1, \ldots, $x$_n$".
%%AT-COPY%% 
%%AT-COPY%% \begin{ex}
%%AT-COPY%% %~~gen ^^^ Places9v2e7
%%AT-COPY%% % package Places9v2e7;
%%AT-COPY%% % KNOWNFAIL-at
%%AT-COPY%% % class Example{ 
%%AT-COPY%% % static def use(Any) = 1;
%%AT-COPY%% % static def example() { 
%%AT-COPY%% %  val there = here.next();
%%AT-COPY%% %~~vis
%%AT-COPY%% \begin{xten}
%%AT-COPY%% var a : Int = 1;
%%AT-COPY%% val b = [2,3,4];
%%AT-COPY%% at(there; *) {
%%AT-COPY%%   assert a + b(0) == b(1);
%%AT-COPY%% }
%%AT-COPY%% \end{xten}
%%AT-COPY%% %~~siv
%%AT-COPY%% % }  }
%%AT-COPY%% % class Hook{ def run() {Example.example(); return true;}}
%%AT-COPY%% %~~neg
%%AT-COPY%% 
%%AT-COPY%% \end{ex}
%%AT-COPY%% 
%%AT-COPY%% \end{itemize}
%%AT-COPY%% 
%%AT-COPY%% As an important special case, \xcd`at(p;)S` copies {\em nothing} to \xcd`S`.
%%AT-COPY%% This must not be confused with \xcd`at(p)S`, which copies {\em everything}.
%%AT-COPY%% 
%%AT-COPY%% 
%%AT-COPY%% 
%%AT-COPY%% Note that \xcd`at(p;x,*)use(x,y);` is equivalent to \xcd`at(p;*)use(x,y);`.
%%AT-COPY%% In both statements, the \xcd`*` indicates that all variables used in the body
%%AT-COPY%% are to be copied in.  The former makes clear that \xcd`x` is one of the things
%%AT-COPY%% being copied, but, from the \xcd`*`, there may be others. 
%%AT-COPY%% 
%%AT-COPY%% However, other copy specifiers may be used to compute
%%AT-COPY%% values in \xcd`S` which are not available (and thus need not be stored)
%%AT-COPY%% outside of it.  
%%AT-COPY%% 
%%AT-COPY%% \begin{ex}The following code may end up with a large object \xcd`c` in
%%AT-COPY%% memory at \xcd`p` but not at the initial place: 
%%AT-COPY%% %~~gen ^^^ Places3q9u
%%AT-COPY%% % package Places3q9u;
%%AT-COPY%% % KNOWNFAIL-at
%%AT-COPY%% % class Example { 
%%AT-COPY%% % def use(Example, Example, Example) = 1;
%%AT-COPY%% % def Elephant(Example) = 1;
%%AT-COPY%% % static def example(a: Example, b:Example, p:Place) { 
%%AT-COPY%% %~~vis
%%AT-COPY%% \begin{xten}
%%AT-COPY%% at(p; c = a.Elephant(b), *) {
%%AT-COPY%%   use(a,b,c);
%%AT-COPY%% }
%%AT-COPY%% \end{xten}
%%AT-COPY%% %~~siv
%%AT-COPY%% %} } 
%%AT-COPY%% %~~neg
%%AT-COPY%% \end{ex}
%%AT-COPY%% 
%%AT-COPY%% The blanket \xcd`at`-statement \xcd`at(p)S` copies everything.  It is an
%%AT-COPY%% abbreviation for \xcd`at(p;*)S`.  
%%AT-COPY%% When this manual refers to a generic \xcd`at`-statement as \xcd`at(p;F)S`, it
%%AT-COPY%% should be understood as including the blanket \xcd`at` statement \xcd`at(p)S`
%%AT-COPY%% with this interpretation.
%%AT-COPY%% 


\subsection{Copying Values}
%%AT-COPY%% An activity executing statement \xcd"at (q;F) S" at a place \xcd`p`
%%AT-COPY%% evaluates \xcd`q` at \xcd`p` and then moves to \xcd`q` to execute
%%AT-COPY%% \xcd`S`.  
%%AT-COPY%% The original-expressions of \xcd`F` are evaluated at \xcd`p`.
%%AT-COPY%% Their values are copied (\Sref{sect:at-init-val}) to \xcd`q`, and bound to 
%%AT-COPY%% names there, as specified by \xcd`F`.  
%%AT-COPY%% \xcd`S` is evaluated in an environment containing the target variables of
%%AT-COPY%% \xcd`F`, and \xcd`here` and {\em no} other variables.  (In particular, if this
%%AT-COPY%% statement appears in an instance method body and \xcd`this` is not copied,
%%AT-COPY%% \xcd`this` is not accessible.  This fact is important: it allows the
%%AT-COPY%% programmer to control when \xcd`this` is copied, which may be expensive for
%%AT-COPY%% large containers.)

An activity executing \xcd`at(q)S` at a place \xcd`p` evaluates \xcd`q` at
place \xcd`p`, which should be a \xcd`Place`.  It then moves to place \xcd`q`
to execute \xcd`S`.  The values variables that \xcd`S` refers to are copied
(\Sref{sect:at-init-val}) to \xcd`q`, and bound to the variables of the same
name.   If the \xcd`at` is inside of an instance method and \xcd`S` uses
\xcd`this`, \xcd`this` is copied as well.  Note that a field reference
\xcd`this.fld` or a method call \xcd`this.meth()` will cause \xcd`this` to be
copied --- as will their abbreviated forms \xcd`fld` and \xcd`meth()`, despite
the lack of a visible \xcd`this`.  


Note that the value obtained by evaluating \xcd`q`
is not necessarily distinct from \xcd`p` (\eg, \xcd`q` may be
\xcd`here`). 
This does not alter the behavior of \xcd`at`.  
%%AT-COPY%%  \xcd`at(here;F)S` will copy all the values specified by \xcd`F`, 
%%AT-COPY%% even though there is no actual change of place, and even though the original
%%AT-COPY%% values already exist there.
\xcd`at(here)S` will copy all the values mentioned in \xcd`S`, even though
there is no actual change of place, and even though the original values
already exist there. 

On normal termination of \xcd`S` control returns to \xcd`p` and
execution is continued with the statement following 
%%AT-COPY%% \xcd`at (q;F) S`. 
\xcd`at (q) S`. 
If
\xcd`S` terminates abruptly with exception \xcd`E`, \xcd`E` is
serialized into a buffer, the buffer is communicated to \xcd`p` where
it is deserialized into an exception \xcd`E1` and \xcd`at (p) S`
throws \xcd`E1`.

Since 
%%AT-COPY%% \xcd`at(p;F) S` 
\xcd`at(p) S` 
is a synchronous construct, usual control-flow
constructs such as \xcd`break`, \xcd`continue`, \xcd`return` and 
\xcd`throw` are permitted in \xcd`S`.  All concurrency related
constructs -- \xcd`async`, \xcd`finish`, \xcd`atomic`, \xcd`when` are
also permitted.

The \xcd`at`-expression 
%%AT-COPY%% \xcd`at(p;F)E` 
\xcd`at(p)E` 
is similar, except that, in the case of
normal termination of \xcd`E`, the value that \xcd`E` produces is serialized
into a buffer, transported to the starting place, and deserialized, and the
value of the \xcd`at`-expression is the result of deserialization.

\limitation{
X10 does not currently allow {\tt break}, {\tt continue}, or {\tt return}
to exit from an {\tt at}.
}



\subsection{How {\tt at} Copies Values}
\label{sect:at-init-val}

%%AT-COPY%% The values of the original-expressions  specified by \xcd`F` in 
%%AT-COPY%% \xcd`at (p;F)S` are copied to \xcd`p`, as follows.

The values mentioned in \xcd`S` are copied to place \xcd`p` by \xcd`at(p)S` as follows.

First, the original-expressions are evaluated to give a vector of X10 values.
Consider the graph of all values reachable from these values (except for 
\xcd`transient` fields 
(\Sref{sect:transient}, \xcd`GlobalRef`s (\Sref{GlobalRef}); also custom
serialization (\Sref{sect:ser+deser} may alter this behavior)). 

Second this graph is {\em
serialized} into a buffer and transmitted to place \xcd`q`.  Third,
the vector of X10 values is 
re-created at \xcd`q` 
by deserializing the buffer at
\xcd`q`. Fourth, \xcd`S` is executed at \xcd`q`, in an environment in
which each variable \xcd`v` declared in \xcd`F` 
refers to the corresponding deserialized value.  

Note that since values accessed across an \xcd`at` boundary are
copied, the programmer may wish to adopt the discipline that either
variables accessed across an \xcd`at` boundary  contain only structs 
or stateless objects, or the methods invoked on them do not access any
mutable state on the objects. Otherwise the programmer has to ensure
that side effects are made to the correct copy of the object. For this
the struct \xcd`x10.lang.GlobalRef[T]` is often useful.


\subsubsection{Serialization and deserialization.}
\label{sect:ser+deser}
\index{transient}
\index{field!transient}
The X10 runtime provides a default mechanism for
serializing/deserializing an object graph with a given set of roots.
This mechanism may be overridden by the programmer on a per class or
struct basis as described in the API documentation for
\xcd`x10.io.CustomSerialization`.  
The default mechanism performs a
deep copy of the object graph (that is, it copies the object or struct
and, recursively, the values contained in its fields), but does not
traverse or copy \xcd`transient` fields. \xcd`transient` fields are omitted from the
serialized data.   On deserialization, \xcd`transient` fields are initialized
with their default values (\Sref{DefaultValues}).    The types of
\xcd`transient` fields must therefore have default values.



A struct \xcd`s` of type \xcd`x10.lang.GlobalRef[T]` \ref{GlobalRef}
is serialized as a unique global reference to its contained object
\xcd`o` (of type \xcd`T`).  Please see the documentation
of \xcd`x10.lang.GlobalRef[T]` for more details.



\subsection{{\tt at} and Activities}
%%AT-COPY%% \xcd`at(p;F)S` 
\xcd`at(p)S` 
does {\em not} start a new activity.  It should be thought of as
transporting the current activity to \xcd`p`, running \xcd`S` there, and then
transporting it back.  \xcd`async` is the only construct in the
language that starts a new activity. In different contexts, each one
of the following makes sense:
%%AT-COPY%% (1)~\xcd`async at(p;F) S` 
(1)~\xcd`async at(p) S` 
(spawn an activity locally to execute \xcd`S` at
\xcd`p`; here \xcd`p` is evaluated by the spawned activity) , 
%%AT-COPY%% (2)~\xcd`at(p;F) async S` 
(2)~\xcd`at(p) async S` 
(evaluate \xcd`p` and then at \xcd`p` spawn an
activity to execute \xcd`S`), and,
%%AT-COPY%% (3)~\xcd`async at(p;F) async S`. 
(3)~\xcd`async at(p) async S`. 
%%AT-COPY%% In most cases, \xcd`at(p;F) async S` is preferred to
%%\xcd`async at(p;F)`, since In most cases, \xcd`at(p) async S` is
preferred to \xcd`async at(p) S`, since the former form enables a more
efficient runtime implementation.  In the first case, the expression
\xcd`p` is evaluated synchronously by the current activity and then a
single remote async is spawned.  In the second case, \xcd`p` is
semantically required to be evaluated asynchronously with the parent
async as it is contained in the body of an async.  Therefore, if the
compiler cannot prove that "async at (p)" can be safely rewritten into
"at (p) async", a first local async is spawned to evaluate \xcd`p`
then a remote async is spawned to evaluate \xcd`S`.

Since 
%%AT-COPY%% \Xcd{at(p;F) S} 
\Xcd{at(p) S} 
does not start a new activity, 
\xcd`S` may contain constructs which only make sense
within a single activity.  
For example, 
\begin{xten}
    for(x in globalRefsToThings) 
      if (at(x.home) x().isNice()) 
        return x();
\end{xten}
returns the first nice thing in a collection.   If we had used 
\xcd`async at(x.home)`, this would not be allowed; 
you can't \xcd`return` from an
\xcd`async`. 

\limitation{
X10 does not currently allow {\tt break}, {\tt continue}, or {\tt return}
to exit from an {\tt at}.
}



\subsection{Copying from {\tt at} }
\index{at!copying}

%%AT-COPY%% \xcd`at(p;F)S` copies data as specified by \xcd`F`, and sends it
\xcd`at(p)S` copies data required in \xcd`S`, and sends it
to place \xcd`p`, before executing \xcd`S` there. The only things that are not
copied are values only reachable through \xcd`GlobalRef`s and \xcd`transient`
fields, and data omitted by custom serialization.    
%%AT-COPY%% Several choices of copy specifier use the same identifier for the original
%%AT-COPY%% variable outside of 
%%AT-COPY%% \xcd`at(p)S` 
%%AT-COPY%% and its copy inside of \xcd`S`.  
%%AT-COPY%% 

\begin{ex}
%%AT-COPY%% 
%%AT-COPY%% %~~gen ^^^ Places_implicit_copy_from_at_example_1
%%AT-COPY%% % package Places.implicitcopyfromat;
%%AT-COPY%% % class Example {
%%AT-COPY%% % static def example() {
%%AT-COPY%% % 
%%AT-COPY%% %~~vis
%%AT-COPY%% \begin{xten}
%%AT-COPY%% val c = new Cell[Int](9); // (1)
%%AT-COPY%% at (here;c) {             // (2)
%%AT-COPY%%    assert(c() == 9);      // (3)
%%AT-COPY%%    c.set(8);              // (4)
%%AT-COPY%%    assert(c() == 8);      // (5)
%%AT-COPY%% }
%%AT-COPY%% assert(c() == 9);         // (6)
%%AT-COPY%% \end{xten}
%%AT-COPY%% %~~siv
%%AT-COPY%% %}}
%%AT-COPY%% % class Hook{ def run() { Example.example(); return true; } }
%%AT-COPY%% %~~neg
%%AT-COPY%% 

%~~gen ^^^ Places_implicit_copy_from_at_example_1
% package Places.implicitcopyfromat;
% class Example {
% static def example() {
% 
%~~vis
\begin{xten}
val c = new Cell[Int](9); // (1)
at (here) {               // (2) 
   assert(c() == 9);      // (3)
   c.set(8);              // (4)
   assert(c() == 8);      // (5)
}
assert(c() == 9);         // (6)
\end{xten}
%~~siv
%}}
% class Hook{ def run() { Example.example(); return true; } }
%~~neg


The \xcd`at` statement copies the \xcd`Cell` and its contents.  
After \xcd`(1)`, \xcd`c` is a \xcd`Cell` containing 9; call that cell {$c_1$}
At \xcd`(2)`, that cell is copied, resulting in another cell {$c_2$} whose
contents are also 9, as tested at \xcd`(3)`.
(Note that the copying behavior of \xcd`at` happens {\em even when the
destination place is the same as the starting place}--- even with
\xcd`at(here)`.)
At \xcd`(4)`, the contents of {$c_2$} are changed to 8, as confirmed at \xcd`(5)`; the contents of
{$c_1$} are of course untouched.    Finally, at \xcd`(c)`, outside the scope
of the \xcd`at` started at line \xcd`(2)`, \xcd`c` refers to its original
value {$c_1$} rather than the copy {$c_2$}.  
\end{ex}

The \xcd`at` statement induces a {\em deep copy}.  Not only does it copy the
values of variables, it copies values that they refer to through zero or more
levels of reference.  Structures are preserved as well: if two fields
\xcd`x.f` and \xcd`x.g` refer to the same object {$o_1$} in the original, then
\xcd`x.f` and \xcd`x.g` will both refer to the same object {$o_2$} in the
copy.  

\begin{ex}
In the following variation of the preceding example,
\xcd`a`'s original value {$a_1$} is a rail with two references to the same
\xcd`Cell[Int]` {$c_1$}.  The fact that {$a_1(0)$} and {$a_1(1)$} are both
identical to {$c_1$} is demonstrated in \xcd`(A)`-\xcd`(C)`, as {$a_1(0)$} is modified
and {$a_1(1)$} is observed to change.  In \xcd`(D)`-\xcd`(F)`, the copy
{$a_2$} is tested in the same way, showing that {$a_2(0)$} and {$a_2(1)$} both
refer to the same \xcd`Cell[Int]` {$c_2$}.  However, the test at \xcd`(G)`
shows that {$c_2$} is a different cell from {$c_1$}, because changes to
{$c_2$} did not propagate to {$c_1$}.  

%%AT-COPY%% %~~gen ^^^ PlacesAtCopy
%%AT-COPY%% %package Places.AtCopy2;
%%AT-COPY%% %class example {
%%AT-COPY%% %static def Example() {
%%AT-COPY%% %
%%AT-COPY%% %~~vis
%%AT-COPY%% \begin{xten}
%%AT-COPY%% val c = new Cell[Int](5);
%%AT-COPY%% val a : Rail[Cell[Int]] = [c,c as Cell[Int]];
%%AT-COPY%% assert(a(0)() == 5 && a(1)() == 5);     // (A)
%%AT-COPY%% c.set(6);                               // (B)
%%AT-COPY%% assert(a(0)() == 6 && a(1)() == 6);     // (C)
%%AT-COPY%% at(here;a) {
%%AT-COPY%%   assert(a(0)() == 6 && a(1)() == 6);   // (D)
%%AT-COPY%%   c.set(7);                             // (E)
%%AT-COPY%%   assert(a(0)() == 7 && a(1)() == 7);   // (F)
%%AT-COPY%% }
%%AT-COPY%% assert(a(0)() == 6 && a(1)() == 6);     // (G)
%%AT-COPY%% \end{xten}
%%AT-COPY%% %~~siv
%%AT-COPY%% %}}
%%AT-COPY%% %class Hook{ def run() { example.Example(); return true; } }
%%AT-COPY%% %~~neg

%~~gen ^^^ PlacesAtCopy
%package Places.AtCopy2;
%class example {
%static def Example() {
%
%~~vis
\begin{xten}
val c = new Cell[Int](5);
val a : Rail[Cell[Int]] = [c,c as Cell[Int]];
assert(a(0)() == 5 && a(1)() == 5);     // (A)
c.set(6);                               // (B)
assert(a(0)() == 6 && a(1)() == 6);     // (C)
at(here) {
  assert(a(0)() == 6 && a(1)() == 6);   // (D)
  c.set(7);                             // (E)
  assert(a(0)() == 7 && a(1)() == 7);   // (F)
}
assert(a(0)() == 6 && a(1)() == 6);     // (G)
\end{xten}
%~~siv
%}}
%class Hook{ def run() { example.Example(); return true; } }
%~~neg


\end{ex}

\subsection{Copying and Transient Fields}
\label{sect:transient}
\index{at!transient fields and}
\index{transient}
\index{field!transient}

Recall that fields of classes and structs marked \xcd`transient` are not copied by
\xcd`at`.  Instead, they are set to the default values for their types. Types
that do not have default values cannot be used in \xcd`transient` fields.

\begin{ex}
Every \xcd`Trans` object has an \xcd`a`-field equal
to 1.  However, despite the initializer on the \xcd`b` field, it is not the
case that every \xcd`Trans` has \xcd`b==2`.  Since \xcd`b` is \xcd`transient`,
when the \xcd`Trans` value \xcd`this` is copied at \xcd`at(here){...}` in
\xcd`example()`, its \xcd`b` field is not copied, and the default value for an
\xcd`Int`, 0, is used instead.  
Note that we could not make a transient field \xcd`c : Int{c != 0}`, since the
type has no default value, and copying would in fact set it to zero.

%%AT-COPY%% %~~gen ^^^ Places40
%%AT-COPY%% %package Places_transient_a;
%%AT-COPY%% % 
%%AT-COPY%% %~~vis
%%AT-COPY%% \begin{xten}
%%AT-COPY%% class Trans {
%%AT-COPY%%    val a : Int = 1;
%%AT-COPY%%    transient val b : Int = 2;
%%AT-COPY%%    //ERROR transient val c : Int{c != 0} = 3;
%%AT-COPY%%    def example() {
%%AT-COPY%%      assert(a == 1 && b == 2);
%%AT-COPY%%      at(here;a) {
%%AT-COPY%%         assert(a == 1 && b == 0);
%%AT-COPY%%      }
%%AT-COPY%%    }
%%AT-COPY%% }
%%AT-COPY%% \end{xten}
%%AT-COPY%% %~~siv
%%AT-COPY%% %class Hook{ def run() { (new Trans()).example(); return true; } }
%%AT-COPY%% %~~neg

%~~gen ^^^ Places40
%package Places_transient_a;
% 
%~~vis
\begin{xten}
class Trans {
   val a : Int = 1;
   transient val b : Int = 2;
   //ERROR: transient val c : Int{c != 0} = 3;
   def example() {
     assert(a == 1 && b == 2);
     at(here) {
        assert(a == 1 && b == 0);
     }
   }
}
\end{xten}
%~~siv
%class Hook{ def run() { (new Trans()).example(); return true; } }
%~~neg



\end{ex}

\subsection{Copying and GlobalRef}
\label{GlobalRef}
\index{at!GlobalRef}
\index{at!blocking copying}

%%The other barrier to the potentially copious copying behavior of \xcd`at`
%%is the \xcd`GlobalRef` struct.  
A \xcd`GlobalRef[T]` (say \xcd`g`) contains a reference to
a value \xcd`v` of type \xcd`T`, in a form which can be transmitted, and a \xcd`Place`
\xcd`g.home` indicating where the value lives. When a 
\xcd`GlobalRef` is serialized an opaque, globally unique handle to
\xcd`v` is created.  

\begin{ex}The following example does not copy the value \xcd`huge`.  However, \xcd`huge`
would have been copied if it had been put into a \xcd`Cell`, or simply used
directly. 

%%AT-COPY%% %~~gen ^^^ Places50
%%AT-COPY%% %package Places.copyingblockingwithglobref;
%%AT-COPY%% % class GR {
%%AT-COPY%% %  static def use(Any){}
%%AT-COPY%% %  static def example() {
%%AT-COPY%% % 
%%AT-COPY%% %~~vis
%%AT-COPY%% \begin{xten}
%%AT-COPY%% val huge = "A potentially big thing";
%%AT-COPY%% val href = GlobalRef(huge);
%%AT-COPY%% at (here;href) {
%%AT-COPY%%    use(href);
%%AT-COPY%%   }
%%AT-COPY%% }
%%AT-COPY%% \end{xten}
%%AT-COPY%% %~~siv
%%AT-COPY%% %}
%%AT-COPY%% % class Hook{ def run() { GR.example(); return true; } }
%%AT-COPY%% %~~neg

%~~gen ^^^ Places50
%package Places.copyingblockingwithglobref;
% class GR {
%  static def use(Any){}
%  static def example() {
% 
%~~vis
\begin{xten}
val huge = "A potentially big thing";
val href = GlobalRef(huge);
at (here) {
   use(href);
  }
}
\end{xten}
%~~siv
%}
% class Hook{ def run() { GR.example(); return true; } }
%~~neg


\end{ex}

Values protected in \xcd`GlobalRef`s can be retrieved by the application
%~~exp~~`~~`~~ g:GlobalRef[Any]{here == g.home}~~ ^^^Places4e7q
operation \xcd`g()`.  \xcd`g()` is guarded; it can 
only be called when \xcd`g.home == here`.  If you  want to do anything other
than pass a global reference around or compare two of them for equality, you
need to placeshift back to the home place of the reference, often with
\xcd`at(g.home)`.   

\begin{ex}The following program, for reasons best known to the programmer,
modifies the 
command-line argument array.

%%AT-COPY%% 
%%AT-COPY%% %~~gen ^^^ Places60
%%AT-COPY%% % package Places.Atsome.Globref2;
%%AT-COPY%% % class GR2 {
%%AT-COPY%% % 
%%AT-COPY%% %~~vis
%%AT-COPY%% \begin{xten}
%%AT-COPY%%   public static def main(argv:Rail[String]) {
%%AT-COPY%%     val argref = GlobalRef[Rail[String]](argv);
%%AT-COPY%%     at(here.next(); argref) 
%%AT-COPY%%         use(argref);
%%AT-COPY%%   }
%%AT-COPY%%   static def use(argref : GlobalRef[Rail[String]]) {
%%AT-COPY%%     at(argref.home; argref) {
%%AT-COPY%%       val argv = argref();
%%AT-COPY%%       argv(0) = "Hi!";
%%AT-COPY%%     }
%%AT-COPY%%   }
%%AT-COPY%% \end{xten}
%%AT-COPY%% %~~siv
%%AT-COPY%% %} 
%%AT-COPY%% % class Hook{ def run() { GR2.main(["what, me weasel?" as String]); return true; }}
%%AT-COPY%% %~~neg
%%AT-COPY%% 

%~~gen ^^^ Places60
% package Places.Atsome.Globref2;
% class GR2 {
% 
%~~vis
\begin{xten}
  public static def main(argv:Rail[String]) {
    val argref = GlobalRef[Rail[String]](argv);
    at(here.next()) 
        use(argref);
  }
  static def use(argref : GlobalRef[Rail[String]]) {
    at(argref) {
      val argv = argref();
      argv(0) = "Hi!";
    }
  }
\end{xten}
%~~siv
%} 
% class Hook{ def run() { GR2.main(["what, me weasel?" as String]); return true; }}
%~~neg

\end{ex}

There is an implicit coercion from \xcd`GlobalRef[T]` to \xcd`Place`, so
\xcd`at(argref)S` goes to \xcd`argref.home`.  


\subsection{Warnings about \xcd`at`}
There are two dangers involved with \xcd`at`: 
\begin{itemize}
\item Careless use of \xcd`at` can result in copying and transmission
of very large data structures.  
%%AT-COPY%% This is particularly an issue with the blanket
%%AT-COPY%% \xcd`at` statement, \xcd`at(p)S`, where everything used in \xcd`S` is copied.  
In particular, it is very easy to capture
\xcd`this` -- a field reference will do it -- and accidentally copy everything
that \xcd`this` refers to, which can be very large.  A disciplined use of copy
specifiers to make explicit just what gets copied can ameliorate this issue.

\item As seen in the examples above, a local variable reference
  \xcd`x` may refer to different objects in different nested \xcd`at`
  scopes. The programmer must either ensure that a variable accessed
  across an \xcd`at` boundary has no mutable state or be prepared to
  reason about which copy gets modified.   A disciplined use of copy specifiers to give
  different names to variables can ameliorate this concern.
\end{itemize}


%%AT-COPY%% \section{{\tt athome}: Returning Values from {\tt at}-Blocks}
%%AT-COPY%% \label{sect:athome}
%%AT-COPY%% \index{athome}
%%AT-COPY%% 
%%AT-COPY%% The 
%%AT-COPY%% \xcd`at(p;F)S` 
%%AT-COPY%% construct renders external variables unavailable within
%%AT-COPY%% \xcd`S`.  However, it is often useful to transmit values back from \xcd`S`,
%%AT-COPY%% and store them in external variables. 
%%AT-COPY%% 
%%AT-COPY%% The \xcd`athome(V;F)S` construct provides
%%AT-COPY%% this ability.  \xcd`V` is a list of variables, which must all be defined at
%%AT-COPY%% the same place.  \xcd`athome(V;F)S` goes to the place where the variables are
%%AT-COPY%% defined, copying \xcd`F` as for \xcd`at(p;F)S`, and executes \xcd`S` ---
%%AT-COPY%% allowing reading, assignment and initialization of the listed variables in
%%AT-COPY%% \xcd`V`. 
%%AT-COPY%% 
%%AT-COPY%% \xcd`V`, the list of variables, may include one or more variables.  It is a
%%AT-COPY%% static error if X10 cannot determine that all the variables in the list are
%%AT-COPY%% defined at the same place.
%%AT-COPY%% 
%%AT-COPY%% 
%%AT-COPY%% 
%%AT-COPY%% 
%%AT-COPY%% \begin{ex}
%%AT-COPY%% \xcd`athome` allows returning multiple pieces of information from an
%%AT-COPY%% \xcd`at`-statement.  In the following example, we return two data: 
%%AT-COPY%% one as a \xcd`val` named \xcd`square`, and the other as an addition in to a
%%AT-COPY%% partially-computed polynomial named \xcd`poly`.  
%%AT-COPY%% %~~gen ^^^ Places5f9g
%%AT-COPY%% % package Places5f9g;
%%AT-COPY%% % % KNOWNFAIL-at
%%AT-COPY%% % class Example { 
%%AT-COPY%% %~~vis
%%AT-COPY%% \begin{xten}
%%AT-COPY%% static def example(a: Int, mathProc: Place) { 
%%AT-COPY%%   val square : Int;
%%AT-COPY%%   var poly : Int = 1 + a; // will be 1+a+a*a
%%AT-COPY%%   at(mathProc; a) {
%%AT-COPY%%     val sq = a*a; 
%%AT-COPY%%     athome(square, poly; sq) {
%%AT-COPY%%        square = sq;  // initialization
%%AT-COPY%%        poly += sq;   // read and update
%%AT-COPY%%     }
%%AT-COPY%%   return [square, poly];
%%AT-COPY%%   }
%%AT-COPY%% \end{xten}
%%AT-COPY%% %~~siv
%%AT-COPY%% %}}
%%AT-COPY%% % class Hook { def run() { 
%%AT-COPY%% %   val e = example(2, here);
%%AT-COPY%% %   assert e(0) == 4 && e(1) == 7;
%%AT-COPY%% %   return true;
%%AT-COPY%% % }} 
%%AT-COPY%% %~~neg
%%AT-COPY%% \end{ex}
%%AT-COPY%% 
%%AT-COPY%% The abbreviated forms 
%%AT-COPY%% \xcd`athome (*) S` and 
%%AT-COPY%% \xcd`athome S` 
%%AT-COPY%% allow a block of assignments without specifying the variables being assigned
%%AT-COPY%% to, which is convenient for a small set of assignments. 
%%AT-COPY%% They 
%%AT-COPY%% are both equivalent to \xcd`athome(V;F)S`,
%%AT-COPY%% where: 
%%AT-COPY%% \begin{itemize}
%%AT-COPY%% \item \xcd`V` is the list of all variables appearing on the left-hand side of
%%AT-COPY%%       an assignment or update statement in \xcd`S`, excluding those which
%%AT-COPY%%       appear inside the body of an \xcd`at` or \xcd`athome` statement in \xcd`S`;
%%AT-COPY%% \item \xcd`F` is the same as for \xcd`at(p)S` (\Sref{sect:copy-spec})
%%AT-COPY%% \end{itemize}
%%AT-COPY%% 
%%AT-COPY%% 
%%AT-COPY%% \begin{ex}
%%AT-COPY%% 
%%AT-COPY%% Much as the blanket \xcd`at` construct \xcd`at(p)S` is convenient for
%%AT-COPY%% executing a small code body at another place, the blanket \xcd`athome`
%%AT-COPY%% construct \xcd`athome(*) S` 
%%AT-COPY%% (which may be written as simply \xcd`athome S`)
%%AT-COPY%% is convenient for returning a result or two.   The
%%AT-COPY%% preceding example could have been written using blanket statements.
%%AT-COPY%% 
%%AT-COPY%% %~~gen ^^^ Places5f9gblanket
%%AT-COPY%% % package Places5f9gblanket;
%%AT-COPY%% % class Example { 
%%AT-COPY%% % KNOWNFAIL-at
%%AT-COPY%% %~~vis
%%AT-COPY%% \begin{xten}
%%AT-COPY%% static def example(a: Int, mathProc: Place) { 
%%AT-COPY%%   val square : Int;
%%AT-COPY%%   var poly : Int = 1 + a; // will be 1+a+a*a
%%AT-COPY%%   at(mathProc) {
%%AT-COPY%%     val sq = a*a; 
%%AT-COPY%%     athome {
%%AT-COPY%%        square = sq;  // initialization
%%AT-COPY%%        poly += sq;   // read and update
%%AT-COPY%%     }
%%AT-COPY%%   return [square, poly];
%%AT-COPY%%   }
%%AT-COPY%% \end{xten}
%%AT-COPY%% %~~siv
%%AT-COPY%% %}}
%%AT-COPY%% % class Hook { def run() { 
%%AT-COPY%% %   val e = example(2, here);
%%AT-COPY%% %   assert e(0) == 4 && e(1) == 7;
%%AT-COPY%% %   return true;
%%AT-COPY%% % }} 
%%AT-COPY%% %~~neg
%%AT-COPY%% \end{ex}
%%AT-COPY%% 
%%AT-COPY%% {\bf Design:} It is not fundamentally essential to distinguish \xcd`at` from
%%AT-COPY%% \xcd`athome`.  \xcd`at(p;F)S` could allow writing to variables whose homes are
%%AT-COPY%% known at compile-time to be equal to \xcd`p`.  Indeed, in earlier versions of
%%AT-COPY%% X10, it did so.    This required an idiom in which programmers had to manage
%%AT-COPY%% the home locations of variables directly, and keep track of which home
%%AT-COPY%% location corresponded to which variable.  The \xcd`athome` construct makes
%%AT-COPY%% this idiom more convenient. 
	
\chapter{Activities}\label{XtenActivities}

An {\em activity} is a statement being executed, independently, with its own
local variables; it may be thought of as a very light-weight thread. An
\Xten{} computation may have many concurrent {\em activities} executing at any
give time.  All X10 code runs as part of an activity; when an X10 program is
started, the \xcd`main` method is invoked in an activity, called the {\em root
activity}.\index{root
activity}


Activities coordinate their execution by various control and data structures.
For example, `
%~~stmt~~`~~`~~x:Int, var y:Int ~~
\xcd`when(x==0);` blocks the current activity until some other activity
sets \xcd`x` to zero.  However, activities determine the places at which they
may be blocked and resumed, by \xcd`when` and similar constructs.  There are
no means by which one activity can arbitrarily interrupt, block, or resume
another, no method  \xcd`activity.interrupt()`.

An activity may be {\em running}, {\em blocked} on some condition or {\em
terminated}. If terminated, it is terminated in the same way that its
statement is: in particular, if the statement terminates abruptly, the
activity terminates abruptly for the same reason.
(\Sref{ExceptionModel}).

Activities can be long-running entities with a good deal of local state.  In
particular they can involve recursive method calls (and therefore have runtime
stacks).  However, activities can also be short-running light-weight entities,
\eg, it is reasonable to have an activity that simply increments a variable.

An activity may asynchronously and in parallel launch activities at
other places.  Every activity save the initial \xcd`main` activity is spawned
by another.  Thus, at any instant, the activities in a program form a tree.

X10 uses this tree in crucial ways.  
First is the distinction 
between {\em local} termination and {\em global}
termination of a statement. The execution of a statement by an
activity is said to terminate locally when the activity has finished
all its computation. (For instance the
creation of an asynchronous activity terminates locally when the
activity has been created.)  It is said to terminate globally when it
has terminated locally and all activities that it may have spawned at
any place have, recursively, terminated globally.
For example, consider: 
%~~gen
% package Activites.Are.For.Whacktivities;
% class Example {
% def example( s1:() => Void, s2 : () => Void ) {
%~~vis
\begin{xten}
async {s1();}
async {s2();}
\end{xten}
%~~siv
% } } 
%~~neg
The primary activity spawns two child activities and then terminates locally,
very quickly.  The child activities may take arbitrary amounts of time to
terminate (and may spawn grandchildren).  When \xcd`s1()`, \xcd`s2()`, and
all their descendants terminate locally, then the primary activity terminates
globally. 

The program as a whole terminates when the root activity terminates globally.
In particular, X10 does not permit the creation of 
daemon threads---threads that outlive the lifetime of the root
activity.  We say that an \Xten{} computation is {\em rooted}
(\Sref{initial-computation}).

\futureext{ We may permit the initial activity to be a daemon activity
to permit reactive computations, such as webservers, that may not
terminate.}

\section{The \Xten{} rooted exception model}
\label{ExceptionModel}
\index{Exception!model}

The rooted nature of \Xten{} computations permits the definition of a
{\em rooted exception model.} In multi-threaded programming languages
there is a natural parent-child relationship between a thread and a
thread that it spawns. Typically the parent thread continues execution
in parallel with the child thread. Therefore the parent thread cannot
serve to catch any exceptions thrown by the child thread. 

The presence of a root activity and the concept of global termination permits
\Xten{} to adopt a more powerful exception model. In any state of the
computation, say that an activity $A$ is {\em a root of} an activity $B$ if
$A$ is an ancestor of $B$ and $A$ is blocked at a statement (such as the
\xcd"finish" statement \Sref{finish}) awaiting the termination of $B$ (and
possibly other activities). For every \Xten{} computation, the \emph{root-of}
relation is guaranteed to be a tree. The root of the tree is the root activity
of the entire computation. If $A$ is the nearest root of $B$, the path from
$A$ to $B$ is called the {\em activation path} for the activity.\footnote{Note
  that depending on the state of the computation the activation path may
  traverse activities that are running, blocked or terminated.}

We may now state the exception model for \Xten.  An uncaught exception
propagates up the activation path to its nearest root activity, where
it may be handled locally or propagated up the \emph{root-of} tree when
the activity terminates (based on the semantics of the statement being
executed by the activity).\footnote{In \XtenCurrVer{} the \xcd"finish"
statement is the only statement that marks its activity as a root
activity. Future versions of the language may introduce more such
statements.}  In Java, exceptions may be overlooked because there is no good
place to put a \xcd`try`-\xcd`catch` block; this is never the case in X10.

\section{\xcd`at`: Place changing}\label{AtStatement}

An activity may change place using the \xcd"at" statement or
\xcd"at" expression:

\begin{grammar}
Statement \: AtStatement \\
AtStatement \: \xcd"at" PlaceExpressionSingleList Statement \\
Expression \: AtExpression \\
AtExpression \: \xcd"at" PlaceExpressionSingleList ClosureBody 
\end{grammar}

The statement \xcd"at (p) S" executes the statement \xcd"S"
synchronously at a place described by \xcd"p".
The expression \xcd"at (p) E" executes the statement \xcd"E"
synchronously at place \xcd"p", returning the result to the
originating place.  




\xcd`p` may be an expression of type \xcd`Place`, in which case its value is
used as the place to execute the body: 
%~~gen
% package Activities.At.A.Standstill;
% class Example {
% def example(ob: Object, S: ()=>Void) {
%~~vis
\begin{xten}
   at (here.next()) S();
\end{xten}
%~~siv
% } } 
%~~neg
\noindent



\xcd`at(p)S` does {\em not} start a new activity.  It should be thought of as
transporting the current activity to \xcd`p`, running \xcd`S` there, and then
transporting it back.    If you want to start a new activity, use \xcd`async`;
if you want to start a new activity at \xcd`p`, use 
\xcd`at(p) async S`.  

As a consequence of this, \xcd`S` may contain constructs which only make sense
within a single activity.  
For example, 
\begin{xten}
    for(x in globalRefsToThings) 
    if (at(x.home) x().nice()) 
        return x();
\end{xten}
returns the first nice thing in a collection.   If we had used 
\xcd`async at(x.home)`, this would not be allowed; 
you can't \xcd`return` from an
\xcd`async`. 



\section{\xcd`async`: Spawning an activity}\label{AsynchronousActivity}\label{AsyncActivity}

Asynchronous activities serve as a single abstraction for supporting a
wide range of concurrency constructs such as message passing, threads,
DMA, streaming, data prefetching. (In general, asynchronous operations
are better suited for supporting scalability than synchronous
operations.)

An activity is created by executing the \xcd`async` statement: 

\begin{grammar}
Statement \: AsyncStatement \\
AsyncStatement \: \xcd"async"  Statement \\
PlaceExpressionSingleList \: \xcd"(" PlaceExpression \xcd")" \\
PlaceExpression \: Expression 
\end{grammar} 


The basic form of \xcd`async` is \xcd`async S`, which starts a new activity
located \xcd`here` executing \xcd`S`.   


\bard{The followingin para is under investigation:}
In many cases the compiler may infer the unique place at which the
statement is to be executed by an analysis of the types of the
variables occurring in the statement. (The place must be such that the
statement can be executed safely, without generating a
\xcd"BadPlaceException".) In such cases the programmer may omit the
place designator; the compiler will throw an error if it cannot
determine the unique designated place.\footnote{\XtenCurrVer{} does
not specify a particular algorithm; this will be fixed in future
versions.}

An activity $A$ executes the statement \xcd"async (P) S" by launching
a new activity $B$ at place \xcd`P` (or \xcd`P.home` if \xcd`P` is of an
object type), to execute \xcd`S`. The statement terminates locally as soon as $B$ is
launched.  The activation path for $B$ is that of $A$ augmented by the
information that {$A$} is the parent of {$B$}. 
$B$
terminates normally when $S$ terminates normally.  It terminates
abruptly if $S$ throws an uncaught exception. The exception is
propagated to $A$ if $A$ is a root activity (see \Sref{finish}),
otherwise it is propagated through $A$ to $A$'s root activity. Note that while
{$A$} is running, exceptions thrown by activities it has already
spawned may propagate through it up to its root activity, without {$A$} noticing.

Multiple activities launched by a single activity at another place are not
ordered in any way. They are added to the set of activities at the target
place and will be executed based on the local scheduler's decisions.
If some particular sequencing of events is needed, \xcd`when`, \xcd`atomic`,
\xcd`finish`, clocks, and other X10 constructs can be used.
\Xten{} implementations are not required to have fair schedulers,
though every implementation should make a best faith effort to ensure
that every activity eventually gets a chance to make forward progress.

\begin{staticrule*}
The statement in the body of an \xcd"async" is subject to the
restriction that it must be acceptable as the body of a \xcd"void"
method for an anonymous inner class declared at that point in the code,
which throws no checked exceptions. As such, it may reference
variables in lexically enclosing scopes (including \xcd"clock"
variables, \Sref{XtenClocks}) provided that such variables are
(implicitly or explicitly) \xcd"val".
\end{staticrule*}

\section{Finish}\index{finish}\label{finish}
The statement \xcd"finish S" converts global termination to local
termination and introduces a root activity.   It executes \xcd`S`, and then
waits for all activities spawned by \xcd`S`, directly or indirectly, to
finish. It also collects exceptions thrown by those activities.

\begin{grammar}
Statement \: FinishStatement \\
FinishStatement \: \xcd"finish" Statement 
\end{grammar}

An activity $A$ executes \xcd"finish S" by executing \xcd"S".  The
execution of \xcd"S" may spawn other asynchronous activities (here or
at other places).  Uncaught exceptions thrown or propagated by any
activity spawned by \xcd"S" are accumulated at \xcd"finish S".
\xcd"finish S" terminates locally when all activities spawned by
\xcd"S" terminate globally (either abruptly or normally). If \xcd"S"
terminates normally, then \xcd"finish S" terminates normally and $A$
continues execution with the next statement after \xcd"finish S".  If
\xcd"S" terminates abruptly, then \xcd"finish S" terminates abruptly
and throws a single exception, \Xcd{x10.lang.MultipleExceptions}
formed from the collection of exceptions accumulated at \xcd"finish S".

Thus a \xcd"finish S" statement serves as a collection point for
uncaught exceptions generated during the execution of \xcd"S".

Note that repeatedly \xcd"finish"ing a statement has little effect after
the first \xcd"finish": \xcd"finish finish S" is indistinguishable
from \xcd"finish S" if \xcd`S` throws no exceptions.  (If \xcd`S` throws
exceptions, \xcd`finish S` wraps them in one layer of 
\xcd`MultipleExceptions` and \xcd`finish finish S` in two layers.)

%%OLIVIER-DENIES%% \paragraph{Interaction with clocks.}\label{sec:finish:clock-rule}
%%OLIVIER-DENIES%% 
%%OLIVIER-DENIES%% \xcd"finish S" interacts with clocks (\Sref{XtenClocks}). 
%%OLIVIER-DENIES%% While executing \xcd"S", an activity must not spawn any \xcd"clocked"
%%OLIVIER-DENIES%% asyncs. (Asyncs spawned during the execution of \xcd"S" may spawn
%%OLIVIER-DENIES%% clocked asyncs.) A
%%OLIVIER-DENIES%% \xcd"ClockUseException"\index{clock!ClockUseException} is thrown
%%OLIVIER-DENIES%% if (and when) this condition is violated.
%%OLIVIER-DENIES%% 
%%OLIVIER-DENIES%% This is necessary to prevent deadlocks.  In the following invalid code 
%%OLIVIER-DENIES%% %~s~gen
%%OLIVIER-DENIES%% % package Activities.Finish.Hates.Clocks;
%%OLIVIER-DENIES%% % class Example{
%%OLIVIER-DENIES%% % def example() {
%%OLIVIER-DENIES%% %~s~vis
%%OLIVIER-DENIES%% \begin{xten}
%%OLIVIER-DENIES%% val c:Clock = Clock.make();
%%OLIVIER-DENIES%% async clocked(c) {                // (A) 
%%OLIVIER-DENIES%%       finish async clocked(c) {   // (B) INVALID
%%OLIVIER-DENIES%%             next;                 // (Bnext)
%%OLIVIER-DENIES%%       }
%%OLIVIER-DENIES%%       next;                       // (Anext)
%%OLIVIER-DENIES%% }
%%OLIVIER-DENIES%% \end{xten}
%%OLIVIER-DENIES%% %~s~siv
%%OLIVIER-DENIES%% % } } 
%%OLIVIER-DENIES%% %~s~neg
%%OLIVIER-DENIES%% \xcd`(A)`, first of all, waits for the \xcd`finish` containing \xcd`(B)` to
%%OLIVIER-DENIES%% finish.  
%%OLIVIER-DENIES%% \xcd`(B)` will execute its \xcd`next` at \xcd`(Bnext)`, and then wait for all
%%OLIVIER-DENIES%% other activities registered on \xcd`c` to execute their \xcd`next`s.
%%OLIVIER-DENIES%% However, \xcd`(A)` is registered on \xcd`c`.  So, \xcd`(B)` cannot finish
%%OLIVIER-DENIES%% until \xcd`(A)` has proceeded to \xcd`(Anext)`, and \xcd`(A)` cannot proceed
%%OLIVIER-DENIES%% until \xcd`(B)` finishes. Thus, this causes deadlock.
%%OLIVIER-DENIES%% 
%%OLIVIER-DENIES%% 
%%OLIVIER-DENIES%% 
%%OLIVIER-DENIES%% In \XtenCurrVer{} this condition is checked dynamically; future
%%OLIVIER-DENIES%% versions of the language will introduce type qualifiers which permit
%%OLIVIER-DENIES%% this condition to be checked statically.
%%OLIVIER-DENIES%% 
%%OLIVIER-DENIES%% \futureext{
%%OLIVIER-DENIES%% The semantics of \xcd"finish S" is conjunctive; it terminates when all
%%OLIVIER-DENIES%% the activities created during the execution of \xcd"S" (recursively)
%%OLIVIER-DENIES%% terminate. In many situations (e.g., nondeterministic search) it is
%%OLIVIER-DENIES%% natural to require a statement to terminate when any {\em one} of the
%%OLIVIER-DENIES%% activities it has spawned succeeds. The other activities may then be
%%OLIVIER-DENIES%% safely aborted. Future versions of the language may introduce a
%%OLIVIER-DENIES%% \xcd"finishone S" construct to support such speculative or nondeterministic
%%OLIVIER-DENIES%% computation.
%%OLIVIER-DENIES%% }
%%OLIVIER-DENIES%% 



\section{Initial activity}\label{initial-computation}\index{initial activity}

An \Xten{} computation is initiated from the command line on the
presentation of a classname \xcd"C". The class must have a
\xcd"public static def main(a: Rail[String]):Void" method, otherwise an
exception is thrown
and the computation terminates.  The single statement
\begin{xten}
finish async (Place.FIRST_PLACE) {
  C.main(s);
}
\end{xten} 
\noindent is executed where \xcd"s" is an Rail of strings created
from the command line arguments. This single activity is the root activity
for the entire computation. (See \Sref{XtenPlaces} for a discussion of
places.)

%% Say something about configuration information? 




\section{Ateach statements}\index{\Xcd{ateach}}\label{ateach-section}

\begin{grammar}
Statement \: AtEachStatement \\
AtEachStatement \:
      \xcd"ateach" \xcd"(" Formal \xcd"in" Expression \xcd")"
         Statement \\
AtEachStatement \:
      \xcd"ateach" \xcd"(" Expression \xcd")"
         Statement 
\end{grammar}

The \xcd"ateach" statement \xcd`ateach (p in D) S`
spawns an activity \xcd`S` at each place \xcd`p` of a distribution \xcd`D`. 
In \xcd`ateach(p in D) S`, 
\xcd`D` must be either of type \xcd"Dist" 
(see \Sref{XtenDistributions})
or of type
\xcd`DistArray[T]` (see \Sref{XtenArrays}), 
and \xcd`p` will be of type \xcd"Point" (see \Sref{point-syntax}).

\xcd`ateach(p in D)S` is equivalent to 
\xcd`for(p in D) at(D(p)) async S`.  That is, the elements of \xcd`D` are all
points \xcd`p`.  \xcd`D(p)` is a \xcd`Place`.  \xcd`ateach(p in D)S` executes
the body \xcd`S` at the place \xcd`D(p)` (and may use the point \xcd`p`
there). 


However, the compiler may implement it more efficiently to avoid extraneous
communications.  In particular, it is recommended that \xcd`ateach(p in D)S`
be implemented as the following code, which coordinates with each place of
\xcd`D` just once, rather than once per element of \xcd`D` at that place: 

%~~gen
% package Activities.Activities.Activities;
% class EquivCode {
% static def S(pt:Point) {}
% static def example(D:Dist) {
%~~vis
\begin{xten}
for (p in D.places()) async at (p) {
    for (pt in D|here) async {
        S(pt);
    }
}
\end{xten}
%~~siv
%}} 
%~~neg

If \xcd`e` is an \xcd`DistArray[T]`, then \xcd`ateach (p in e)S` is identical to
\xcd`ateach(p in e.dist)S`; the iteration is over the array's underlying
distribution.   
The code below is a common and generally efficient way to work with the
elements of a distributed array:
%~~gen
%package Activities.For.Fnu.And.Pforit;
%class Example[T]{
%  def dealWith(T):Void = {}
% def idiom(A:DistArray[T]){
%~~vis
\begin{xten}
ateach(p in A) 
  dealWith(A(p));
\end{xten}
%~~siv
%}}
%~~neg








\section{At expressions}

\begin{grammar}
Expression \: \xcd"at" \xcd"(" Expression \xcd")" Expression
\end{grammar}

An \Xcd{at} expression evaluates an expression synchronously at the
given place and returns its value.  For instance a copy of the
value pointed to by a \Xcd{GlobalRef} may be obtained using
the \Xcd{fetch} method:
%~~gen
% package Activities.AtExpressions.Fetching;
% class Example[T] {
%~~vis
\begin{xten}
  def fetch(g:GlobalRef[T]):T = at (g) g();
\end{xten}
%~~siv
% } 
%~~neg

The expression evaluation may spawn asynchronous activities. The \Xcd{at}
expression will return without waiting for those activities to terminate. That
is, \Xcd{at} does not have built-in \Xcd{finish} semantics.

\section{Atomic blocks}\label{AtomicBlocks}\index{atomic blocks}
Languages such as \java{} use low-level synchronization locks to allow
multiple interacting threads to coordinate the mutation of shared
data. \Xten{} eschews locks in favor of a very simple high-level
construct, the {\em atomic block}.

A programmer may use atomic blocks to guarantee that invariants of
shared data-structures are maintained even as they are being accessed
simultaneously by multiple activities running in the same place.  

For example, consider a class \xcd`Redund[T]`, which encapsulates a list
\xcd`list` and, (redundantly) keeps the size of the list in a second field
\xcd`size`.  Then \xcd`r:Redund[T]` has the invariant 
\xcd`r.list.size() == r.size`, which must be true at any point that there are
no method calls on \xcd`r` active.

If the \xcd`add` method on \xcd`Redund` (which adds an element to the list) 
were defined as: 
%~~gen
% package Activities.Atomic.Redund.One;
% import x10.util.*;
% class Redund[T] {
%   val list = new ArrayList[T]();
%   var size : Int = 0;
%~~vis
\begin{xten}
def add(x:T) { // Incorrect
  this.list.add(x);
  this.size = this.size + 1;
}
\end{xten}
%~~siv
%}
%~~neg
Then two activities simultaneously adding elements to the same \xcd`r` could break the
invariant.  Suppose that \xcd`r` starts out empty.  Let the first activity
perform the \xcd`list.add`, and compute \xcd`this.size+1`, which is 1, but not store it
back into \xcd`this.size` yet.  
(At this point, \xcd`r.list.size()==1` and \xcd`r.size==0`; the invariant
expression is false, but, as the first call to \xcd`r.add()` is active, the
invariant does not need to be true -- it only needs to be true when the
call finishes.)
Now, let the second activity do its call to
\xcd`add` to completion, which finishes with \xcd`r.size==1`.  
(As before, the invariant expression is false, but a call to \xcd`r.add()` is
still active, so the invariant need not be true.)
Finally, let
the first activity finish, which assigns the \xcd`1` computed before back into
\xcd`this.size`.  At the end, there are two elements in \xcd`r.list`, but
\xcd`r.size==1`. Since there are no calls to \xcd`r.add()` active, the
invariant must be true, but it is not.

In this case, the invariant can be maintained by making the increment atomic.
Doing so forbids that sequence of events; the \xcd`atomic` block cannot be
stopped partway.  
%~~gen
% package Activities.Atomic.Redund.Two;
% import x10.util.*;
% class Redund[T] {
%   val list = new ArrayList[T]();
%   var size : Int = 0;
%~~vis
\begin{xten}
def add(x:T) { 
  this.list.add(x);
  atomic { this.size = this.size + 1; }
}
\end{xten}
%~~siv
%}
%~~neg



\subsection{Unconditional atomic blocks}
The simplest form of an atomic block is the {\em unconditional
atomic block}:

\begin{grammar}
Statement \: AtomicStatement \\
AtomicStatement \: \xcd"atomic"  Statement \\
MethodModifier \: \xcd"atomic" \\
\end{grammar}

For the sake of efficient implementation \XtenCurrVer{} requires
that the atomic block be {\em analyzable}, that is, the set of
locations that are read and written by the \grammarrule{BlockStatement} are
bounded and determined statically.\footnote{A static bound is a constant
that depends only on the program text, and is independent 
of any runtime parameters. }
The exact algorithm to be used by
the compiler to perform this analysis will be specified in future
versions of the language.
\tbd{}

Such a statement is executed by an activity as if in a single step
during which all other concurrent activities in the same place are
blocked. If execution of the statement may throw an exception, it is
the programmer's responsibility to wrap the atomic block within a
\xcd"try"/\xcd"finally" clause and include undo code in the finally
clause. Thus the \xcd"atomic" statement only guarantees atomicity on
successful execution, not on a faulty execution.


We allow methods of an object to be annotated with \xcd"atomic". Such
a method is taken to stand for a method whose body is wrapped within an
\xcd"atomic" statement.

Atomic blocks are closely related to non-blocking synchronization
constructs \cite{herlihy91waitfree}, and can be used to implement 
non-blocking concurrent algorithms.

\begin{staticrule*}
In \xcd"atomic S", \xcd"S" may include calls to \xcd`safe` methods, and use of
sequential control structures.

It may {\em not} include an \xcd"async" activity (such as creation
of a \Xcd{future}).

It may {\em not} include any statement that may potentially block at
runtime (\eg, \xcd"when", \xcd"force" operations, \xcd"next"
operations on clocks, \xcd"finish"). 

It may {\em not} include any \xcd`at` expressions or
statements. (Hence all locations accessed in the atomic block must
belong to the current place.)
\index{locality condition}\label{LocalityCondition} 

\end{staticrule*}

The compiler checks for this condition by checking whether the statement
could be the body of a \xcd"void" method annotated with \xcd"safe" at
that point in the code (\Sref{SafeAnnotation}).

\paragraph{Consequences.}
Note an important property of an (unconditional) atomic block:

\begin{eqnarray}
 \mbox{\xcd"atomic \{s1; atomic s2\}"} &=& \mbox{\xcd"atomic \{s1; s2\}"}
\end{eqnarray}

Atomic blocks do not introduce deadlocks.    They may exhibit all the bad
behavior of sequential programs, including throwing exceptions and running
forever, but they are guaranteed not to deadlock.


\subsubsection{Example}

The following class method implements a (generic) compare and swap (CAS) operation:


%~~gen
% package Activities.And.Protein;
% class CASSizer{
%~~vis
\begin{xten}
var target:Object = null;
public atomic def CAS(old1: Object, new1: Object): Boolean {
   if (target.equals(old1)) {
     target = new1;
     return true;
   }
   return false;
}
\end{xten}
%~~siv
%}
%~~neg

\subsection{Conditional atomic blocks}

Conditional atomic blocks allow the activity to wait for some condition to be
satisfied before executing an atomic block. For example, consider a
\xcd`Redund` class holding a list \xcd`r.list` and, redundantly, its length
\xcd`r.size`.  A \xcd`pop` operation will delay until the \xcd`Redund` is
nonempty, and then remove an element and update the length.  
%~~gen
% package Activities.Condato.Example.Not.A.Tree;
% import x10.util.*;
% class Redund[T] {
% val list = new ArrayList[T]();
% var size : Int = 0;
%~~vis
\begin{xten}
def pop():T {
  var ret : T;
  when(size>0) {
    ret = list.removeAt(0);
    size --;
    }
  return ret;
}
\end{xten}
%~~siv
% }
%~~neg


The execution of the test is atomic with the execution of the block.  This is
important; it means that no other activity can sneak in and make the condition
be false before the block is executed.  In this example, two \xcd`pop`s
executing on a list with one element would work properly. Without the
conditional atomic block -- even doing the decrement atomically -- one call to
\xcd`pop` could pass the \xcd`size>0` guard; then the other call could run to
completion (removing the only element of the list); then, when the first call
proceeds, its \xcd`removeAt` will fail.  

Note that \xcd`if` would not work here.  
\xcd`if(size>0) atomic{size--; return list.removeAt(0);}` allows another
activity to act between the test and the atomic block.  
And 
\xcd`atomic{ if(size>0) {size--; ret = list.removeAt(0);}}` 
does not wait for \xcd`size>0` to become true.


Conditional atomic blocks are of the form:

\begin{grammar}
Statement \:  WhenStatement \\
WhenStatement \:  \xcd"when" \xcd"(" Expression \xcd")" Statement \\
            \| WhenStatement \xcd"or" \xcd"(" Expression \xcd")" Statement 
\end{grammar}

In such a statement the one or more expressions are called {\em
guards} and must be \xcd"Boolean" expressions. The statements are the
corresponding {\em guarded statements}.  

An activity executing such a statement suspends until such time as any
one of the guards is true in the current state. In that state, the
statement corresponding to the first guard that is true is executed.
The checking of the guards and the execution of the corresponding
guarded statement is done atomically. 

\Xten{} does not guarantee that a conditional atomic block
will execute if its condition holds only intermittently. For, based on
the vagaries of the scheduler, the precise instant at which a
condition holds may be missed. Therefore the programmer is advised to
ensure that conditions being tested by conditional atomic blocks are
eventually stable, \ie, they will continue to hold until the block
executes (the action in the body of the block may cause the condition
to not hold any more).

%%Fourth, \Xten{} does not guarantees only {\em weak fairness} when executing
%%conditional atomic blocks. Let $c$ be the guard of some conditional
%%atomic block $A$. $A$ is required to make forward progress only if
%%$c$ is {\em eventually stable}. That is, any execution $s_1, s_2,
%%\ldots$ of the program is considered illegal only if there is a $j$
%%such that $c$ holds in all states $s_k$ for $k > j$ and in which $A$
%%does not execute. Specifically, if the system executes in such a way
%%that $c$ holds only intermmitently (that is, for some state in which
%%$c$ holds there is always a later state in which $c$ does not hold),
%%$A$ is not required to be executed (though it may be executed).

\begin{rationale}
The guarantee provided by \xcd"wait"/\xcd"notify" in \java{} is no
stronger. Indeed conditional atomic blocks may be thought of as a
replacement for \java's wait/notify functionality.
\end{rationale} 


The statement \xcd"when (true) S" is
behaviorally identical to \xcd"atomic S": it never suspends.

\begin{staticrule*}
For the sake of efficient implementation certain restrictions are
placed on the guards and statements in a conditional atomic
block. 
\end{staticrule*}

Guards are statically required not to have side-effects, not to spawn
asynchronous activities (as for the \xcd`sequential` qualifier on methods) and
to have a statically determinable upper bound on their execution (as for the
\xcd`nonblocking` qualifier on methods).

The body of a \xcd"when" statement must satisfy the conditions
for the body of an \xcd"atomic" block.

Note that this implies that guarded statements are required to be {\em
flat}, that is, they may not contain conditional atomic blocks. (The
implementation of nested conditional atomic blocks may require
sophisticated operational techniques such as rollbacks.)


\begin{example}
The following class shows how to implement a bounded buffer of size
$1$ in \Xten{} for repeated communication between a sender and a
receiver.  The call \xcd`buf.send(ob)` waits until the buffer has space, and
then puts \xcd`ob` into it.  Dually, \xcd`buf.receive()` waits until the
buffer has something in it, and then returns that thing.


%~~gen
% package Activities;
%~~vis
\begin{xten}
class OneBuffer[T] {
  var datum: T;
  def this(t:T) { this.datum = t; this.filled = true; }
  var filled: Boolean;
  public def send(v: T) {
    when (!filled) {
      this.datum = v;
      this.filled = true;
    }
  }
  public def receive(): T {
    when (filled) {
      v: T = datum;
      filled = false;
      return v;
    }
  }
}
\end{xten}
%~~siv
%
%~~neg
\end{example}

	
\chapter{Clocks}\label{XtenClocks}\index{clocks}

Many concurrent algorithms proceed in phases: in phase {$k$}, several
activities work independently, but synchronize together before proceeding on
to phase {$k+1$}. X10 supports this communication structure (and many
variations on it) with a generalization of barriers 
called {\em clocks}. Clocks are designed so that programs which follow a
simple syntactic discipline will not have either deadlocks or race conditions.


The following minimalist example of clocked code has two worker activities A
and B, and three phases. In the first phase, each worker activity says its
name followed by 1; in the second phase, by a 2, and in the third, by a 3.  
So, if \xcd`say` prints its argument, 
\xcd`A-1 B-1 A-2 B-2 B-3 A-3`
would be a legitimate run of the program, but
\xcd`A-1 A-2 B-1 B-2 A-3 B-3`
(with \xcd`A-2` before \xcd`B-1`) would not.

The program creates a clock \xcd`cl` to manage the phases.  Each participating
activity does
the work of its first phase, and then executes \xcd`next;` to signal that it
is finished with that work. \xcd`next;` is blocking, and causes the participant to
wait until all participant have finished with the phase -- as measured by the
clock \xcd`cl` to which they are both registered.  
Then they do the second phase, and another \xcd`next;` to make sure that
neither proceeds to the third phase until both are ready.  This example uses
\xcd`finish` to wait for both particiants to finish.  The parent thread is also
registered on the clock just as the particiants are, and executes \xcd`next;next;`
to run through the phases.


%%TODO -- put the 'atomic' back in when that's legal.

%~~gen
%package Clocks.For.Spock;
%class ClockEx {
%  static def say(s:String) = 
% { /*atomic{x10.io.Console.OUT.println(s);}*/ }
%  public static def main(argv:Rail[String]) {
%~~vis
\begin{xten}
    finish async{
      val cl = Clock.make();
      async clocked(cl) {// Activity A
        say("A-1");
        next;
        say("A-2");
        next;
        say("A-3"); 
      }// Activity A

      async clocked(cl) {// Activity B
        say("B-1");
        next;
        say("B-2");
        next;
        say("B-3"); 
      }// Activity B
    }
\end{xten}
%~~siv
%  }
% }
%~~neg

This chapter describes the syntax and semantics of clocks and
statements in the language that have parameters of type \xcd"Clock". 

The key invariants associated with clocks are as follows.  At any
stage of the computation, a clock has zero or more {\em registered}
activities. An activity may perform operations only on those clocks it
is registered with (these clocks constitute its {\em clock set}). 
An attempt by an activity to operate on a clock it is not registered with
will cause a 
\xcd"ClockUseException"\index{clock!ClockUseException}. 
to be thrown.  
An activity is registered with zero or more clocks when it is created.
During its lifetime the only additional clocks it is registered with
are exactly those that it creates. In particular it is not possible
for an activity to register itself with a clock it discovers by
reading a data structure.

The primary operations that an activity \xcd`a` may perform on a clock \xcd`c`
that it is registered upon are: 
\begin{itemize}
\item It may spawn and simultaneously  {\em register} a new activity on
      \xcd`c`, with the statement       \xcd`async clocked(c){S}`.
\item It may {\em unregister} itself from \xcd`c`, with \xcd`c.drop()`.  After
      doing so, it can no longer use most primary operations on \xcd`c`.
\item It may {\em resume} the clock, with \xcd`c.resume()`, indicating that it
      has finished with the current phase associated with \xcd`c` and is ready
      to move on to the next one.
\item It may {\em wait} on the clock, with \xcd`c.next()`.  This first does
      \xcd`c.resume()`, and then blocks the current activity until the start
      of the next phase, \viz, until all other activities registered on that
      clock have called \xcd`c.resume()`.
\item It may {\em block} on all the clocks it is registered with
      simultaneously, by the command \xcd`next;`.  This, in effect, calls
      \xcd`c.next()` simultaneously 
      on all clocks \xcd`c` that the current activity is registered with.
\item Other miscellaneous operations are available as well; see the
      \xcd`Clock` API.
\end{itemize}

%%CLOCK%% An activity may perform the following operations on a clock \xcd"c".
%%CLOCK%% It may {\em unregister} with \xcd"c" by executing \xcd"c.drop();".
%%CLOCK%% After this, it may perform no further actions on \xcd"c"
%%CLOCK%% for its lifetime. It may {\em check} to see if it is unregistered on a
%%CLOCK%% clock. It may {\em register} a newly forked activity with \xcd"c".
%%CLOCK%% %% It may {\em post} a statement \xcd"S" for completion in the current phase
%%CLOCK%% %% of \xcd"c" by executing the statement \xcd"now(c) S". 
%%CLOCK%% Once registered and "active" (see below), it may also perform the following operations.
%%CLOCK%% It may {\em resume} the clock by executing \xcd"c.resume();". This
%%CLOCK%% indicates to \xcd"c" that it has finished posting all statements it
%%CLOCK%% wishes to perform in the current phase. Finally, it may {\em block}
%%CLOCK%% (by executing \xcd"next;") on all the clocks that it is registered
%%CLOCK%% with. (This operation implicitly \xcd"resume"'s all clocks for the
%%CLOCK%% activity.) It will resume from this statement only when all these
%%CLOCK%% clocks are ready to advance to the next phase.

%%CLOCK%% A clock becomes ready to advance to the next phase when every activity
%%CLOCK%% registered with the clock has executed at least one \xcd"resume"
%%CLOCK%% operation on that clock and all statements posted for completion in
%%CLOCK%% the current phase have been completed.

%%OLIVIER-DENIES%% Though clocks introduce a blocking statement (\xcd"next") an important
%%OLIVIER-DENIES%% property of \Xten{} is that clocks -- when used with the \xcd`next;` {\em
%%OLIVIER-DENIES%%   statement} only, without the \xcd`c.next()` method call -- cannot introduce
%%OLIVIER-DENIES%% deadlocks. That is, the system cannot reach a quiescent state (in which no
%%OLIVIER-DENIES%% activity is progressing) from which it is unable to progress. For, before
%%OLIVIER-DENIES%% blocking each activity resumes all clocks it is registered with. Thus if a
%%OLIVIER-DENIES%% configuration were to be stuck (that is, no activity can progress) all clocks
%%OLIVIER-DENIES%% will have been resumed. But this implies that all activities blocked on
%%OLIVIER-DENIES%% \xcd"next" may continue and the configuration is not stuck. The only other
%%OLIVIER-DENIES%% possibility is that an activity may be stuck on \xcd"finish". But the
%%OLIVIER-DENIES%% interaction rule between \xcd"finish" and clocks
%%OLIVIER-DENIES%% (\Sref{sec:finish:clock-rule}) guarantees that this cannot cause a cycle in
%%OLIVIER-DENIES%% the wait-for graph. A more rigorous proof may be found in \cite{X10-concur05}.

\section{Clock operations}\label{sec:clock}
There are two language constructs for working with clocks. 
\xcd`async clocked(cl) S` starts a new activity registered on one or more
clocks.  \xcd`next;` blocks the current activity until all the activities
sharing clocks with it are ready to proceed to the next clock phase. 
Clocks are objects, and have a number of useful methods on them as well.

\subsection{Creating new clocks}\index{clock!creation}\label{sec:clock:create}

Clocks are created using a factory method on \xcd"x10.lang.Clock":


%~~gen
% package Clocks.For.Spocks;
%class Clockuser {
% def example() {
%~~vis
\begin{xten}
val c: Clock = Clock.make();
\end{xten}
%~~siv
%}}
%~~neg

%%CLOCKVAR%% \eat{All clocked variables are implicitly \xcd`val`. The initializer for a
%%CLOCKVAR%% local variable declaration of type \xcd"Clock" must be a new clock
%%CLOCKVAR%% expression. Thus \Xten{} does not permit aliasing of clocks.
%%CLOCKVAR%% Clocks are created in the place global heap and hence outlive the
%%CLOCKVAR%% lifetime of the creating activity.  Clocks are structs, hence may be freely
%%CLOCKVAR%% copied from place to 
%%CLOCKVAR%% place. (Clock instances typically contain references to mutable state
%%CLOCKVAR%% that maintains the current state of the clock.)
%%CLOCKVAR%% }

The current activity is automatically registered with the newly
created clock.  It may deregister using the \xcd"drop" method on
clocks (see the documentation of \xcd"x10.lang.Clock"). All activities
are automatically deregistered from all clocks they are registered
with on termination (normal or abrupt).

\subsection{Registering new activities on clocks}
\index{clock!clocked statements}\label{sec:clock:register}

The statement 

%~~gen
%package Clocks.For.Jocks;
%class Qlocked{
%static def S():void{}
%static def flock() { 
% val c1 = Clock.make(), c2 = Clock.make(), c3 = Clock.make();
%~~vis
\begin{xten}
  async clocked (c1, c2, c3) S
\end{xten}
%~~siv
%();
%}}
%~~neg
starts a new activity, initially registered with
clocks \xcd`c1`, \xcd`c2`, and \xcd`c3`, and  running \xcd`S`. The activity running this code must
be registered on those clocks. 
Violations of these conditions are punished by the throwing of a
\xcd"ClockUseException"\index{clock!ClockUseException}. 

% An activity may transmit only those clocks that are registered with and
% has not quiesced on (\Sref{resume}). 
% A \xcd"ClockUseException"\index{clock!ClockUseException} is
%thrown if (and when) this condition is violated.

If an activity {$a$} that has executed \xcd`c.resume()` then starts a
new activity {$b$} also registered on \xcd`c` (\eg, via \Xcd{async
clocked(c) S}), the new activity {$b$} starts out having also resumed
\xcd`c`, as if it too had executed \xcd`c.resume()`.  
%~~gen
% package Clocks.For.Jocks.In.Clicky.Smocks;
%class Example{
%static def S():void{}
%static def a_phase_two():void{}
%static def b_phase_two():void{}
%static def example() {
%~~vis
\begin{xten}
//a
val c = Clock.make();
c.resume();
async clocked(c) {
  // b
  c.next();
  b_phase_two();
}
c.next();
a_phase_two();
\end{xten}
%~~siv
%} }
%~~neg
In the proper execution, {$a$} and {$b$} both perform
\xcd`c.next()` and then their phase-2 actions.  
However, if {$b$} were not
initially in the resume state for \xcd`c`, there would be a race condition;
{$b$} could perform \xcd`c.next()` and proceed to \xcd`b_phase_two`
before {$a$} performed \xcd`c.next()`.


An activity may check that it is registered on a clock \xcd"c" by
%~~exp~~`~~`~~c:Clock ~~
the predicate \xcd`c.registered()`


\begin{note}
\Xten{} does not contain a ``register'' operation that would allow an activity
to discover a clock in a datastructure and register itself on it. Therefore,
while a clock \xcd`c` may be stored in a data structure by one activity
\xcd`a` and read from it by another activity \xcd`b`, \xcd`b` cannot do much
with \xcd`c` unless it is already registered with it.  In particular, it
cannot register itself on \xcd`c`, and, lacking that registration, cannot
register a sub-activity on it with \xcd`async clocked(c) S`.
\end{note}


\subsection{Resuming clocks}\index{clock!resume}\label{resume}\label{sec:clock:resume}
\Xten{} permits {\em split phase} clocks. An activity may wish
to indicate that it has completed whatever work it wishes to perform
in the current phase of a  clock \xcd"c" it is registered with, without
suspending altogether. It may do so  by executing 
%~~exp~~`~~`~~c:Clock ~~
\xcd`c.resume()`.



An activity may invoke \xcd`resume()` only on a clock it is registered with,
and has not yet dropped (\Sref{sec:clock:drop}). A
\xcd"ClockUseException"\index{clock!ClockUseException} is thrown if this
condition is violated. Nothing happens if the activity has already invoked a
\xcd"resume" on this clock in the current phase.
%%OLIVIER-DENIES%%  Otherwise, \xcd`c.resume()`
%%OLIVIER-DENIES%% indicates that the activity will not transmit \xcd"c" to an 
%%OLIVIER-DENIES%% \xcd"async" (through a \xcd"clocked" clause), 
%%OLIVIER-DENIES%% until it terminates, drops \xcd"c" or executes a \xcd"next".

%%OLIVIER-DENIES%% \bard{The following is under investigation}
%%OLIVIER-DENIES%% \begin{staticrule*}
%%OLIVIER-DENIES%% It is a static error if any activity has a potentially
%%OLIVIER-DENIES%% live execution path from a \xcd"resume" statement on a clock \xcd"c"
%%OLIVIER-DENIES%% to a
%%OLIVIER-DENIES%% %\xcd"now" or
%%OLIVIER-DENIES%% async spawn statement (which registers the new
%%OLIVIER-DENIES%% activity on \xcd"c") unless the path goes through a \xcd"next"
%%OLIVIER-DENIES%% statement. (A \xcd"c.drop()" following a \xcd"c.resume()" is legal,
%%OLIVIER-DENIES%% as is \xcd"c.resume()" following a \xcd"c.resume()".)
%%OLIVIER-DENIES%% \end{staticrule*}

\subsection{Advancing clocks}\index{clock!next}\label{sec:clock:next}
An activity may execute the statement
\begin{xten}
next;
\end{xten}

\noindent 
Execution of this statement blocks until all the clocks that the
activity is registered with (if any) have advanced. (The activity
implicitly issues a \xcd"resume" on all clocks it is registered
with before suspending.)

\xcd`next;` may be thought of as calling \xcd`c.next()` in parallel for all
clocks that the current activity is registered with.  (The parallelism is
conceptually important: if activities {$a$} and {$b$} are both
registered on clocks \xcd`c` and \xcd`d`, and {$a$} executes
\xcd`c.wait(); d.wait()` while {$b$} executes \xcd`d.wait(); c.wait()`,
then the two will deadlock.  However, if the two clocks are waited on in
parallel, as \xcd`next;` does, {$a$} and {$b$} will not deadlock.)

Equivalently, \xcd`next;` sequentially calls \xcd`c.resume()` for each
registered clock \xcd`c`, in arbitrary order, and then \xcd`c.wait()` for each
clock, again in arbitrary order.  


%%OLIVIER-DENIES%% An \Xten{} computation is said to be {\em quiescent} on a clock
%%OLIVIER-DENIES%% \xcd"c" if each activity registered with \xcd"c" has resumed \xcd"c".
%%OLIVIER-DENIES%% Note that once a computation is quiescent on \xcd"c", it will remain
%%OLIVIER-DENIES%% quiescent on \xcd"c" forever (unless the system takes some action),
%%OLIVIER-DENIES%% since no other activity can become registered with \xcd"c".  That is,
%%OLIVIER-DENIES%% quiescence on a clock is a {\em stable property}.

%%OLIVIER-DENIES%% Once the implementation has detected quiescence on \xcd"c", the system
%%OLIVIER-DENIES%% marks all activities registered with \xcd"c" as being able to progress
%%OLIVIER-DENIES%% on \xcd"c". 
%%OLIVIER-DENIES%% 
An activity blocked on \xcd"next" resumes execution once
it is marked for progress by all the clocks it is registered with.

\subsection{Dropping clocks}\index{clock!drop}\label{sec:clock:drop}
%~~exp~~`~~`~~ c:Clock~~
An activity may drop a clock by executing \xcd`c.drop()`.



\noindent{} The activity is no longer considered registered with this
clock.  A \xcd"ClockUseException" is thrown if the activity has
already dropped \xcd"c".

\section{Deadlock Freedom}

In general, programs using clocks can deadlock, just as programs using loops
can fail to terminate.  However, programs written with a particular syntactic
discipline {\em are} guaranteed to be deadlock-free, just as programs which
use only bounded loops are guaranteed to terminate.  The syntactic discipline
is: 
\begin{itemize}
\item The \xcd`next()` {\bf method} may not be called on any clock. (The
      \xcd`next;` statement is allowed.)
\item Inside of \xcd`finish{S}`, all clocked \xcd`async`s must be in the scope
      an unclocked \xcd`async`.  
\end{itemize}


The second clause prevents the following deadlock.  
%~~gen
% package Clocks.Finish.Hates.Clocks;
% class Example{
% def example() {
%~~vis
\begin{xten}
val c:Clock = Clock.make();
async clocked(c) {                // (A) 
      finish async clocked(c) {   // (B) Violates clause 2
            next;                 // (Bnext)
      }
      next;                       // (Anext)
}
\end{xten}
%~~siv
% } } 
%~~neg
\xcd`(A)`, first of all, waits for the \xcd`finish` containing \xcd`(B)` to
finish.  
\xcd`(B)` will execute its \xcd`next` at \xcd`(Bnext)`, and then wait for all
other activities registered on \xcd`c` to execute their \xcd`next`s.
However, \xcd`(A)` is registered on \xcd`c`.  So, \xcd`(B)` cannot finish
until \xcd`(A)` has proceeded to \xcd`(Anext)`, and \xcd`(A)` cannot proceed
until \xcd`(B)` finishes. Thus, this causes deadlock.


\section{Program equivalences}
From the discussion above it should be clear that the following
equivalences hold:

\begin{eqnarray}
 \mbox{\xcd"c.resume(); next;"}       &=& \mbox{\xcd"next;"}\\
 \mbox{\xcd"c.resume(); d.resume();"} &=& \mbox{\xcd"d.resume(); c.resume();"}\\
 \mbox{\xcd"c.resume(); c.resume();"} &=& \mbox{\xcd"c.resume();"}
\end{eqnarray}

Note that \xcd"next; next;" is not the same as \xcd"next;". The
first will wait for clocks to advance twice, and the second
once.  

%\notinfouro{\subsection{Implementation Notes}
Clocks may be implemented efficiently with message passing, e.g.{} by
using short-circuit ideas in \cite{SaraswatPODC88}.  Recall that every
activity is spawned with references to a fixed number of clocks. Each
reference should be thought of as a global pointer to a location in
some place representing the clock. (We shall discuss a further
optimization below.) Each clock keeps two counters: the total number
of outstanding references to the clock, and the number of activities
that are currently suspended on the clock.

When an activity $A$ spawns another activity $B$ that will reference a
clock $c$ referenced by $A$, $A$ {\em splits} the reference by sending
a message to the clock. Whenever an activity drops a reference to a
clock, or suspends on it, it sends a message to the clock. 

The optimization is that the clock can be represented in a distributed
fashion. Each place keeps a local counter for each clock that is
referenced by an activity in that place. The global location for the
clock simply keeps track of the places that have references and that
are quiescent. This can reduce the inter-place message traffic
significantly.
}
%\notinfouro{\section{Clocked types}\index{types!clocked}

We allow types to specify clocks, via a {\cf clocked(c)} modifier,
e.g.{}

\begin{x10}
  clocked(c) int r;
\end{x10}

This declaration asserts that {\cf r} is accessible
(readable/writable) only by those statements that are clocked on {\cf
c}. Thus propagation of the clock provides some control over the
``visibility'' of {\cf r}.

The declaration 

\begin{x10}
  clocked(c) final int l = r;
\end{x10}

\noindent asserts additionally that in each clock instant {\cf l} is final, 
i.e.{} the value of {\cf l} may be reset at the beginning of each phase
of {\tt c} but stays constant during the phase.

This statement terminates when the computation of {\tt r} has
terminated and the assignment has been performed.

\todo{Generalize the syntax so that aggregate variables can be clocked with an aggregate clock of the same shape.}

\subsection{Clocked assignment}\index{assignment!clocked}
We expect that most arrays containing application data will be
declared to be {\cf clocked final}. We support this very powerful type
declaration by providing a new statement:
{\footnotesize
\begin{verbatim}
  next(c) l = r; 
\end{verbatim}}


\noindent 
for a variable $l$ declared to be clocked on $c$. The statement
assigns $r$ to the {\em next} value of $l$. There may be multiple such
assignments before the clock advances. The last such assignment
specifies the value of the variable that will be visible after the
clock has advanced.  If the variable is {\cf clocked final} it is
guaranteed that {\em all} readers of the variable throughout this
phase will see the value $r$.

The expression {\tt r} is implicitly treated as {\tt now(c) r}. That
is, the clock {\tt c} will not advance until the computation of {\tt r} has
terminated.

}
%\notinfouro{\section{Examples}
\todo{Bring in other examples from Concur paper.}
Consider the core of the ASCI Benchmark Sweep3D program for computing
solutions to mass transport problems.

In a nutshell the core computation is a triply nested sequential loop
in which the value of a variable in the current iteration is dependent
on the values of neighboring variables in a past iteration. Such a
problem can be parallelized through pipelining. One visualizes a
diagonal wavefront sweeping through the array. An MPI version of the
program may be described as follows. There is a two dimensional grid
of processors which performs the following computation
repeatedly. Each processor synchronously receives a value from the
processor to its west, then to its north, then computes some function
of these values and computes a new value to be sent to the processor
to its east and then to its south.  Ignoring the behavior of the
boundary processors for the moment such a computation may be described
by the following \Xten{} program:

\begin{x10}
region R = [1..n0,1..m0];
clock[R] W,N;
clock(W) final double [cyclic(R)] A; 
for (int t : 1..TMax) \{
  ateach( i,j:A) 
    clock (W[i-1,j],N[i,j-1],W[i,j],N[i,j]) \{
      double west = now (W[i-1,j]) future\{A[i-1,j]\}; 
      W[i-1,j].continue();           
      double north = now (N[i,j-1]) future\{A[i,j-1]\}; 
      N[i,j-1].continue();
      next(W[i,j]) A[i,j] = compute(west, north);
      next W[i-1,j],N[i,j-1],W[i,j],N[i,j];
  \}
\}
\end{x10}
}

\section{Clocked Finish}
\index{finish!clocked}
\index{async!clocked}
\index{clocked!finish}
\index{clocked!async}
\label{ClockedFinish}

In the most common case of a single clock coordinating a few behaviors, X10
allows coding with an implicit clock.  \xcd`finish` and \xcd`async` statements
may be qualified with \xcd`clocked`.  

A \xcd`clocked finish` introduces a new clock.  It executes its body in the
usual way that a \xcd`finish` does--- except that, when its body completes,
the activity executing the \xcd`clocked finish` drops the clock, while it
waits for asynchronous spawned \xcd`async`s to terminate.  

A \xcd`clocked async` registers its async with the implicit clock of
the surrounding \xcd`clocked finish`.   

Both the \xcd`clocked finish` and \xcd`clocked async` may use the \xcd`next`
statement to advance implicit clock.  Since the implicit clock is not
available in a variable, it cannot be manipulated directly. (If you want to
manipulate the clock directly, use an explicit clock.)

The following code starts two activities, each of which perform their first
phase, wait for the other to finish phase 1, and then perform their second
phase.  
%~~gen
%package Clocks.ClockedFinish;
%class Example{
%static def phase(String, Int) {}
%def example() {
%~~vis
\begin{xten}
clocked finish {
  clocked async {
     phase("A", 1);
     next;
     phase("A", 2);
  }
  clocked async {
     phase("B", 1);
     next;
     phase("B", 2);
  }
}
\end{xten}
%~~siv
%}}
%~~neg


\index{finish!nested clocked}
\index{clocked finish!nested}

Clocked finishes may be nested.  The inner \xcd`clocked finish` operates in a
single phase of the outer one.  
	
\chapter{Local and Distributed Arrays}\label{XtenArrays}\index{array}

\Xcd{Array}s provide indexed access to data at a single \Xcd{Place}, {\em via}
\Xcd{Point}s---indices of any dimensionality. \Xcd{DistArray}s is similar, but
spreads the data across multiple \xcd`Place`s, {\em via} \Xcd{Dist}s.  
We refer to arrays either sort as ``general arrays''.  


This chapter provides an overview of the \Xcd{x10.array} classes \Xcd{Array}
and \Xcd{DistArray}, and their supporting classes \Xcd{Point}, \Xcd{Region}
and \Xcd{Dist}.  


\section{Points}\label{point-syntax}
\index{point}
\index{point!syntax}


General arrays are indexed by \xcd`Point`s, which are $n$-dimensional tuples of
integers.  The \xcd`rank`
property of a point gives its dimensionality.  Points can be constructed from
integers or \xcd`Array[Int](1)`s by
the \xcd`Point.make` factory methods:
%~~gen
% package Arrays.Points.Example1;
% class Example1 {
% def example1() {
%~~vis
\begin{xten}
val origin_1 : Point{rank==1} = Point.make(0);
val origin_2 : Point{rank==2} = Point.make(0,0);
val origin_5 : Point{rank==5} = Point.make([0,0,0,0,0]);
\end{xten}
%~~siv
% } } 
%~~neg

%~~type~~`~~`~~ ~~
There is an implicit conversion from \xcd`Array[Int](1)` to 
%~~type~~`~~`~~ ~~
\xcd`Point`, giving
a convenient syntax for constructing points: 

%~~gen
% package Arrays.Points.Example2;
% class Example{
% def example() {
%~~vis
\begin{xten}
val p : Point = [1,2,3];
val q : Point{rank==5} = [1,2,3,4,5];
val r : Point(3) = [11,22,33];
\end{xten}
%~~siv
% } } 
%~~neg

The coordinates of a point are available by subscripting; \xcd`p(i)` is the
\xcd`i`th coordinate of the point \xcd`p`. 
\xcdmath`Point($n$)` is a \Xcd{type}-defined shorthand  for 
\xcdmath`Point{rank==$n$}`.


\section{Regions}\label{XtenRegions}\index{region}
\index{region!syntax}

A region is a set of points of the same rank.  {}\Xten{}
provides a built-in class, \xcd`x10.array.Region`, to allow the
creation of new regions and to perform operations on regions. 
Each region \xcd`R` has a property \xcd`R.rank`, giving the dimensionality of
all the points in it.

%~~gen
% package Arrays.Some.Examples.Fidget.Fidget;
% class Example {
% static def example() {
%~~vis
\begin{xten}
val MAX_HEIGHT=20;
val Null = Region.makeUnit();  // Empty 0-dimensional region
val R1 = 1..100; // 1-dim region with extent 1..100
val R2 = (1..100) as Region(1); // same as R1
val R3 = (0..99) * (-1..MAX_HEIGHT);
val R4 = Region.makeUpperTriangular(10);
val R5 = R4 && R3; // intersection of two regions
\end{xten}
%~~siv
% } } 
%~~neg

The expression \xcdmath`m..n`, for integer expressions \Xcd{m} and \Xcd{n},
evaluates to the rectangular, rank-1 region consisting of the points
$\{$\xcdmath`[m]`, \dots, \xcdmath`[n]`$\}$. If \xcdmath`m` is greater than
\xcdmath`n`, the region \Xcd{m..n} is empty.

%%MAYBE%% A region may be constructed by converting from a rail of
%%MAYBE%% regions (\eg, \xcd`R4` above).
%%MAYBE%% The region constructed from a rail of regions represents the Cartesian product
%%MAYBE%% of the arguments. \Eg, \Xcd{R8} is a region of {$100 \times 16 \times 78$}
%%MAYBE%% points, in
%%MAYBE%% %~s~gen
%%MAYBE%% %package Arrays.Region.RailOfRegions;
%%MAYBE%% %class Example{
%%MAYBE%% %def example(){
%%MAYBE%% %~s~vis
%%MAYBE%% \begin{xten}
%%MAYBE%%   val R8 = [1..100, 3..18, 1..78] as Region(3);
%%MAYBE%% \end{xten}
%%MAYBE%% %~s~siv
%%MAYBE%% %}}
%%MAYBE%% %~s~neg


\index{region!upperTriangular}
\index{region!lowerTriangular}\index{region!banded}

Various built-in regions are provided through  factory
methods on \xcd`Region`.  
\begin{itemize}
\item \Xcd{Region.makeEmpty(n)} returns an empty region of rank \Xcd{n}.
\item \Xcd{Region.makeFull(n)} returns the region containing all points of
      rank \Xcd{n}.  
\item \Xcd{Region.makeUnit()} returns the region of rank 0 containing the
      unique point of rank 0.  It is useful as the identity for Cartesian
      product of regions.
\item \Xcd{Region.makeHalfspace(normal:Point, k:Int)} returns the unbounded
      half-space of rank \Xcd{normal.rank}, consisting of all points \Xcd{p}
      satisfying \xcdmath`p$\cdot$normal $\le$ k`.
\item \Xcd{Region.makeRectangular(min, max)}, where \Xcd{min} and \Xcd{max}
      are \Xcd{Int} rails or valrails of length \Xcd{n}, returns a
      \Xcd{Region(n)} equal to: 
      \xcdmath`[min(0) .. max(0), $\ldots$, min(n-1)..max(n-1)]`.
\item \Xcd{Region.make(regions)} constructs the Cartesian product of the
      \Xcd{Region(1)}s in \Xcd{regions}.
\item \Xcd{Region.makeBanded(size, upper, lower)} constructs the
      banded \Xcd{Region(2)} of size \Xcd{size}, with \Xcd{upper} bands above
      and \Xcd{lower} bands below the diagonal.
\item \Xcd{Region.makeBanded(size)} constructs the banded \Xcd{Region(2)} with
      just the main diagonal.
\item \xcd`Region.makeUpperTriangular(N)` returns a region corresponding
to the non-zero indices in an upper-triangular \xcd`N x N` matrix.
\item \xcd`Region.makeLowerTriangular(N)` returns a region corresponding
to the non-zero indices in a lower-triangular \xcd`N x N` matrix.
\item 
  If \xcd`R` is a region, and \xcd`p` a Point of the same rank, then 
%~~exp~~`~~`~~R:Region, p:Point(R.rank) ~~
  \xcd`R+p` is \xcd`R` translated forwards by 
  \xcd`p` -- the region whose
%~~exp~~`~~`~~r:Point, p:Point(r.rank) ~~
  points are \xcd`r+p` 
  for each \xcd`r` in \xcd`R`.
\item 
  If \xcd`R` is a region, and \xcd`p` a Point of the same rank, then 
%~~exp~~`~~`~~R:Region, p:Point(R.rank) ~~
  \xcd`R-p` is \xcd`R` translated backwards by 
  \xcd`p` -- the region whose
%~~exp~~`~~`~~r:Point, p:Point(r.rank) ~~
  points are \xcd`r-p` 
  for each \xcd`r` in \xcd`R`.

\end{itemize}

All the points in a region are ordered canonically by the
lexicographic total order. Thus the points of the region \xcd`(1..2)*(1..2)`
are ordered as 
\begin{xten}
(1,1), (1,2), (2,1), (2,2)
\end{xten}
Sequential iteration statements such as \xcd`for` (\Sref{ForAllLoop})
iterate over the points in a region in the canonical order.

A region is said to be {\em rectangular}\index{region!convex} if it is of
the form \xcdmath`(T$_1$ * $\cdots$ * T$_k$)` for some set of intervals
\xcdmath`T$_i = $ l$_i$ .. h$_i$ `. Such a
region satisfies the property that if two points $p_1$ and $p_3$ are
in the region, then so is every point $p_2$ between them (that is, it is {\em convex}). 
(Banded and triangular regions are not rectangular.)
The operation
%~~exp~~`~~`~~R:Region ~~
\xcd`R.boundingBox()` gives the smallest rectangular region containing
\xcd`R`.

\subsection{Operations on regions}
\index{region!operations}

Let \xcd`R` be a region. A {\em sub-region} is a subset of \Xcd{R}.
\index{region!sub-region}

Let \xcdmath`R1` and \xcdmath`R2` be two regions whose types establish that
they are of the same rank. Let \xcdmath`S` be another region; its rank is
irrelevant. 

\xcdmath`R1 && R2` is the intersection of \xcdmath`R1` and
\xcdmath`R2`, \viz, the region containing all points which are in both
\Xcd{R1} and \Xcd{R2}.  \index{region!intersection}
%~~exp~~`~~`~~ ~~
For example, \xcd`1..10 && 2..20` is \Xcd{2..10}.


%%NO-DIFF%% \xcdmath`R1 - R2` is the set difference of \xcdmath`R1` and
%%NO-DIFF%% \xcdmath`R2`; \viz, the points in \xcdmath`R1` which are not in
%%NO-DIFF%% \xcdmath`R2`.\index{region!set difference}
%%NO-DIFF%% For example, 
%%NO-DIFF%% ~~exp~~`~~`~~ ~~
%%NO-DIFF%% \xcd`(1..10) - (1..3)` 
%%NO-DIFF%% is 
%%NO-DIFF%% \Xcd{4..10}.

\xcdmath`R1 * S` is the Cartesian product of \xcdmath`R1` and
\xcdmath`S`,  formed by pairing each point in \xcdmath`R1` with every  point in \xcdmath`S`.
\index{region!product}
%~~exp~~`~~`~~ ~~
Thus, \xcd`(1..2)*(3..4)*(5..6)`
is the region of rank \Xcd{3} containing the eight points with coordinates
\xcd`[1,3,5]`, \xcd`[1,3,6]`, \xcd`[1,4,5]`, \xcd`[1,4,6]`,
\xcd`[2,3,5]`, \xcd`[2,3,6]`, \xcd`[2,4,5]`, \xcd`[2,4,6]`.


For a region \xcdmath`R` and point \xcdmath`p` of the same rank,
%~~exp~~`~~`~~R:Region, p:Point{p.rank==R.rank} ~~
\xcd`R+p` 
and
%~~exp~~`~~`~~R:Region, p:Point{p.rank==R.rank} ~~
\xcd`R-p` 
represent the translation of the region
forward 
and backward 
by \xcdmath`p`. That is, \Xcd{R+p} is the set of points
\Xcd{p+q} for all \Xcd{q} in \Xcd{R}, and \Xcd{R-p} is the set of \Xcd{q-p}.

More \Xcd{Region} methods are described in the API documentation.

\section{Arrays}
\index{array}

Arrays are organized data, arranged so that it can be accessed by subscript.
An \xcd`Array[T]` \Xcd{A} has a \Xcd{Region} \Xcd{A.region}, telling which
\Xcd{Point}s are in \Xcd{A}.  For each point \Xcd{p} in \Xcd{A.region},
\Xcd{A(p)} is the datum of type \Xcd{T} associated with \Xcd{p}.  X10
implementations should 
attempt to store \xcd`Array`s efficiently, and to make array element accesses
quick---\eg, avoiding constructing \Xcd{Point}s when unnecessary.

This generalizes the concepts of arrays appearing in many other programming
languages.  A \Xcd{Point} may have any number of coordinates, so an
\xcd`Array` can have, in effect, any number of integer subscripts.  

Indeed, it is possible to write code that works on \Xcd{Array}s regardless 
of dimension.  For example, to add one \Xcd{Array[Int]} \Xcd{src} into another
\Xcd{dest}, 
%~~gen
%package Arrays.Arrays.Arrays.Example;
% class Example{
%~~vis
\begin{xten}
static def addInto(src: Array[Int], dest:Array[Int])
  {src.region == dest.region}
  = {
    for (p in src.region) 
       dest(p) += src(p);
  }
\end{xten}
%~~siv
%}
%~~neg
\noindent
Since \Xcd{p} is a \Xcd{Point}, it can hold as many coordinates as are
necessary for the arrays \Xcd{src} and \Xcd{dest}.

The basic operation on arrays is subscripting: if \Xcd{A} is an \Xcd{Array[T]}
and \Xcd{p} a point with the same rank as \xcd`A.region`, then
%~~exp~~`~~`~~A:Array[Int], p:Point{self.rank == A.region.rank} ~~
\xcd`A(p)`
is the value of type \Xcd{T} associated with point \Xcd{p}.

Array elements can be changed by assignment. If \Xcd{t:T}, 
%~~gen
%package Arrays.Arrays.Subscripting.Is.From.Mars;
%class Example{
%def example[T](A:Array[T], p: Point{rank == A.region.rank}, t:T){
%~~vis
\begin{xten}
A(p) = t;
\end{xten}
%~~siv
%} } 
%~~neg
modifies the value associated with \Xcd{p} to be \Xcd{t}, and leaves all other
values in \Xcd{A} unchanged.

An \Xcd{Array[T]} \Xcd{A} has: 
\begin{itemize}
%~~exp~~`~~`~~A:Array[Int] ~~
\item \xcd`A.region`: the \Xcd{Region} upon which \Xcd{A} is defined.
%~~exp~~`~~`~~A:Array[Int] ~~
\item \xcd`A.size`: the number of elements in \Xcd{A}.
%~~exp~~`~~`~~A:Array[Int] ~~
\item \xcd`A.rank`, the rank of the points usable to subscript \Xcd{A}.
      Identical to \Xcd{A.region.rank}.
\end{itemize}

\subsection{Array Constructors}
\index{array!constructor}

To construct an array whose elements all have the same value \Xcd{init}, call
\Xcd{new Array[T](R, init)}. 
For example, an array of a thousand \xcd`"oh!"`s can be made by:
%~~exp~~`~~`~~ ~~
\xcd`new Array[String](1..1000, "oh!")`.


To construct and initialize an array, call the two-argument constructor. 
\Xcd{new Array[T](R, f)} constructs an array of elements of type \Xcd{T} on
region \Xcd{R}, with \Xcd{A(p)} initialized to \Xcd{f(p)} for each point
\Xcd{p} in \Xcd{R}.  \Xcd{f} must be a function taking a point of rank
\Xcd{R.rank} to a value of type \Xcd{T}.  \Eg, to construct an array of a
hundred zero values, call
%~~exp~~`~~`~~ ~~
\xcd`new Array[Int](1..100, (Point(1))=>0)`. 
To construct a multiplication table, call
%~~exp~~`~~`~~ ~~
\xcd`new Array[Int]((0..9)*(0..9), (p:Point(2)) => p(0)*p(1))`.

Other constructors are available; see the API documentation and
\Sref{sect:ArrayCtors}. 

\subsection{Array Operations}
\index{array!operations on}

The basic operation on \Xcd{Array}s is subscripting.  If \Xcd{A:Array[T]} and 
\xcd`p:Point{rank == A.rank}`, then \Xcd{a(p)} is the value of type \Xcd{T}
appearing at position \Xcd{p} in \Xcd{A}.    The syntax is identical to
function application, and, indeed, arrays may be used as functions.
\Xcd{A(p)} may be assigned to, as well, by the usual assignment syntax
%~~exp~~`~~`~~A:Array[Int], p:Point{rank == A.rank}, t:Int ~~
\xcd`A(p)=t`.
(This uses the application and setting syntactic sugar, as given in \Sref{set-and-apply}.)

Sometimes it is more convenient to subscript by integers.  Arrays of rank 1-4
can, in fact, be accessed by integers: 
%~~gen
%package Arrays.Arrays.wombatsfromlemuria;
%class Example{
%def example(){
%~~vis
\begin{xten}
val A1 = new Array[Int](1..10, 0);
A1(4) = A1(4) + 1;
val A4 = new Array[Int]((1..2)*(1..3)*(1..4)*(1..5), 0);
A4(2,3,4,5) = A4(1,1,1,1)+1;
\end{xten}
%~~siv
%}}
%~~neg



Iteration over an \Xcd{Array} is defined, and produces the \Xcd{Point}s of the
array's region.  If you want to use the values in the array, you have to
subscript it.  For example, you could double every element of an
\Xcd{Array[Int]} by: 
%~~gen
%package Arrays.Arrays.mostly_dire_dreams_tonight;
%class Example{
%def example(A:Array[Int]) {
%~~vis
\begin{xten}
for (p in A) A(p) = 2*A(p);
\end{xten}
%~~siv
%}}
%~~neg



\section{Distributions}\label{XtenDistributions}
\index{distribution}

Distributed arrays are spread across multiple \xcd`Place`s.  
A {\em distribution}, a mapping from a region to a set of places, 
describes where each element of a distributed array is kept.
Distributions are embodied by the class \Xcd{x10.array.Dist}.
This class is \xcd`final` in
{}\XtenCurrVer; future versions of the language may permit
user-definable distributions. 
The {\em rank} of a distribution is the rank of the underlying region, and
thus the rank of every point that the distribution applies to.



%~~gen
%package Arrays.Dists.Examples.Examples.EXAMPLESDAMMIT;
% class Example{
% def example() {
%~~vis
\begin{xten}
val R  <: Region = 1..100;
val D1 <: Dist = Dist.makeBlock(R);
val D2 <: Dist = R -> here;
\end{xten}
%~~siv
% } } 
%~~neg

Let \xcd`D` be a distribution. 
%~~exp~~`~~`~~D:Dist ~~
\xcd`D.region` 
denotes the underlying
region. 
Given a point \xcd`p`, the expression
%~~exp~~`~~`~~ D:Dist, p:Point{p.rank == D.rank}~~
\xcd`D(p)` represents the application of \xcd`D` to \xcd`p`, that is,
the place that \xcd`p` is mapped to by \xcd`D`. The evaluation of the
expression \xcd`D(p)` throws an \xcd`ArrayIndexOutofBoundsException`
if \xcd`p` does not lie in the underlying region.
%%NO-R2D2-CONV%% 
%%NO-R2D2-CONV%% When operated on as a distribution, a region \xcd`R` implicitly
%%NO-R2D2-CONV%% behaves as the distribution mapping each item in \xcd`R` to \xcd`here`
%%NO-R2D2-CONV%% (\ie, \xcd`R->here`, see below). Conversely, when used in a context
%%NO-R2D2-CONV%% expecting a region, a distribution \xcd`D` should be thought of as
%%NO-R2D2-CONV%% standing for \xcd`D.region`.


\subsection{Operations returning distributions}
\index{distribution!operations}

Let \xcd`R` be a region, \xcd`Q` a Sequence of places \{\xcd`p1`, \dots,
\xcd`pk`\} (enumerated in canonical order), and \xcd`P` a place.

\paragraph{Unique distribution} \index{distribution!unique}
%~~exp~~`~~`~~Q:Sequence[Place] ~~
The distribution \xcd`Dist.makeUnique(Q)` is the unique distribution from the
region \xcd`1..k` to \xcd`Q` mapping each point \xcd`i` to \xcd`pi`.

\paragraph{Constant distributions.} \index{distribution!constant}
%~~exp~~`~~`~~R:Region, P:Place ~~
The distribution \xcd`R->P` maps every point in region \xcd`R` to place \xcd`P`, as does
%~~exp~~`~~`~~R:Region, P:Place ~~
\xcd`Dist.makeConstant(R,P)`. 

\paragraph{Block distributions.}\index{distribution!block}
%~~exp~~`~~`~~R:Region ~~
The distribution \xcd`Dist.makeBlock(R)` distributes the elements of \xcd`R`,
in order, over all the places available to the program. 
Let $p$ equal \xcd`|R| div N` and $q$ equal \xcd`|R| mod N`,
where \xcd`N` is the size of \xcd`Q`, and 
\xcd`|R|` is the size of \xcd`R`.  The first $q$ places get
successive blocks of size $(p+1)$ and the remaining places get blocks of
size $p$.

There are other \xcd`Dist.makeBlock` methods capable of controlling the
distribution and the set of places used; see the API documentation.


%%NO-CYCLIC-DIST%%  \paragraph{Cyclic distributions.} \index{distribution!cyclic}
%%NO-CYCLIC-DIST%%  The distribution \xcd`Dist.makeCyclic(R, Q)` distributes the points in \xcd`R`
%%NO-CYCLIC-DIST%%  cyclically across places in \xcd`Q` in order.
%%NO-CYCLIC-DIST%%  
%%NO-CYCLIC-DIST%%  The distribution \xcd`Dist.makeCyclic(R)` is the same distribution as
%%NO-CYCLIC-DIST%%  \xcd`Dist.makeCyclic(R, Place.places)`. 
%%NO-CYCLIC-DIST%%  
%%NO-CYCLIC-DIST%%  Thus the distribution \xcd`Dist.makeCyclic(Place.MAX_PLACES)` provides a 1--1
%%NO-CYCLIC-DIST%%  mapping from the region \xcd`Place.MAX_PLACES` to the set of all
%%NO-CYCLIC-DIST%%  places and is the same as the distribution \xcd`Dist.makeCyclic(Place.places)`.
%%NO-CYCLIC-DIST%%  
%%NO-CYCLIC-DIST%%  \paragraph{Block cyclic distributions.}\index{distribution!block cyclic}
%%NO-CYCLIC-DIST%%  The distribution \xcd`Dist.makeBlockCyclic(R, N, Q)` distributes the elements
%%NO-CYCLIC-DIST%%  of \xcd`R` cyclically over the set of places \xcd`Q` in blocks of size
%%NO-CYCLIC-DIST%%  \xcd`N`.
%%NO-ARB-DIST%%  
%%NO-ARB-DIST%%  \paragraph{Arbitrary distributions.} \index{distribution!arbitrary}
%%NO-ARB-DIST%%  The distribution \xcd`Dist.makeArbitrary(R,Q)` arbitrarily allocates points in {\cf
%%NO-ARB-DIST%%  R} to \xcd`Q`. As above, \xcd`Dist.makeArbitrary(R)` is the same distribution as
%%NO-ARB-DIST%%  \xcd`Dist.makeArbitrary(R, Place.places)`.
%%NO-ARB-DIST%%  
%%NO-ARB-DIST%%  \oldtodo{Determine which other built-in distributions to provide.}
%%NO-ARB-DIST%%  
\paragraph{Domain Restriction.} \index{distribution!restriction!region}

If \xcd`D` is a distribution and \xcd`R` is a sub-region of {\cf
%~~exp~~`~~`~~D:Dist,R :Region{R.rank==D.rank} ~~
D.region}, then \xcd`D | R` represents the restriction of \xcd`D` to
\xcd`R`---that is, the distribution that takes each point \xcd`p` in \xcd`R`
to 
%~~exp~~`~~`~~D:Dist, p:Point{p.rank==D.rank} ~~
\xcd`D(p)`, 
but doesn't apply to any points but those in \xcd`R`.

\paragraph{Range Restriction.}\index{distribution!restriction!range}

If \xcd`D` is a distribution and \xcd`P` a place expression, the term
%~~exp~~`~~`~~ D:Dist, P:Place~~
\xcd`D | P` 
denotes the sub-distribution of \xcd`D` defined over all the
points in the region of \xcd`D` mapped to \xcd`P`.

Note that \xcd`D | here` does not necessarily contain adjacent points
in \xcd`D.region`. For instance, if \xcd`D` is a cyclic distribution,
\xcd`D | here` will typically contain points that differ by the number of
places. 
An implementation may find a
way to still represent them in contiguous memory, \eg, using a
complex arithmetic function to map from the region index to an index
into the array.

%%NO-USER-DIST%%  \subsection{User-defined distributions}\index{distribution!user-defined}
%%NO-USER-DIST%%  
%%NO-USER-DIST%%  Future versions of \Xten{} may provide user-defined distributions, in
%%NO-USER-DIST%%  a way that supports static reasoning.

%%NO-flinking-operations-on-DIST%%  \subsection{Operations on distributions}
%%NO-flinking-operations-on-DIST%%  
%%NO-flinking-operations-on-DIST%%  A {\em sub-distribution}\index{sub-distribution} of \xcd`D` is
%%NO-flinking-operations-on-DIST%%  any distribution \xcd`E` defined on some subset of the region of
%%NO-flinking-operations-on-DIST%%  \xcd`D`, which agrees with \xcd`D` on all points in its region.
%%NO-flinking-operations-on-DIST%%  We also say that \xcd`D` is a {\em super-distribution} of
%%NO-flinking-operations-on-DIST%%  \xcd`E`. A distribution \xcdmath`D1` {\em is larger than}
%%NO-flinking-operations-on-DIST%%  \xcdmath`D2` if \xcdmath`D1` is a super-distribution of
%%NO-flinking-operations-on-DIST%%  \xcdmath`D2`.
%%NO-flinking-operations-on-DIST%%  
%%NO-flinking-operations-on-DIST%%  Let \xcdmath`D1` and \xcdmath`D2` be two distributions with the same rank.  
%%NO-flinking-operations-on-DIST%%  

%%NO-&&-DIST%%  \paragraph{Intersection of distributions.}\index{distribution!intersection}
%%NO-&&-DIST%%  ~~exp~~`~~`~~D1:Dist, D2:Dist{D1.rank==D2.rank} ~~
%%NO-&&-DIST%%  \xcdmath`D1 && D2`, the intersection 
%%NO-&&-DIST%%  of \xcdmath`D1`
%%NO-&&-DIST%%  and \xcdmath`D2`, is the largest common sub-distribution of
%%NO-&&-DIST%%  \xcdmath`D1` and \xcdmath`D2`.

%%NO-overlay-DIST%%  \paragraph{Asymmetric union of distributions.}\index{distribution!union!asymmetric}
%%NO-overlay-DIST%%  ~~exp~~`~~`~~D1:Dist, D2:Dist{D1.rank==D2.rank} ~~
%%NO-overlay-DIST%%  \xcdmath`D1.overlay(D2)`, the asymmetric union of
%%NO-overlay-DIST%%  \xcdmath`D1` and \xcdmath`D2`, is the distribution whose
%%NO-overlay-DIST%%  region is the union of the regions of \xcdmath`D1` and
%%NO-overlay-DIST%%  \xcdmath`D2`, and whose value at each point \xcd`p` in its
%%NO-overlay-DIST%%  region is \xcdmath`D2(p)` if \xcdmath`p` lies in
%%NO-overlay-DIST%%  \xcdmath`D2.region` otherwise it is \xcdmath`D1(p)`.
%%NO-overlay-DIST%%  (\xcdmath`D1` provides the defaults.)

%%NO-flinking-operations-on-DIST%%  \paragraph{Disjoint union of distributions.}\index{distribution!union!disjoint}
%%NO-flinking-operations-on-DIST%%  ~~exp~~`~~`~~D1:Dist, D2:Dist{D1.rank==D2.rank} ~~
%%NO-flinking-operations-on-DIST%%  \xcdmath`D1 || D2`, the disjoint union of
%%NO-flinking-operations-on-DIST%%  \xcdmath`D1`
%%NO-flinking-operations-on-DIST%%  and \xcdmath`D2`, is defined only if the regions of
%%NO-flinking-operations-on-DIST%%  \xcdmath`D1` and \xcdmath`D2` are disjoint. Its value is
%%NO-flinking-operations-on-DIST%%  \xcdmath`D1.overlay(D2)` (or equivalently
%%NO-flinking-operations-on-DIST%%  \xcdmath`D2.overlay(D1)`.  (It is the least
%%NO-flinking-operations-on-DIST%%  super-distribution of \xcdmath`D1` and \xcdmath`D2`.)
%%NO-flinking-operations-on-DIST%%  
%%NO-flinking-operations-on-DIST%%  \paragraph{Difference of distributions.}\index{distribution!difference}
%%NO-flinking-operations-on-DIST%%  \xcdmath`D1 - D2` is the largest sub-distribution of
%%NO-flinking-operations-on-DIST%%  \xcdmath`D1` whose region is disjoint from that of
%%NO-flinking-operations-on-DIST%%  \xcdmath`D2`.
%%NO-flinking-operations-on-DIST%%  
%%NO-flinking-operations-on-DIST%%  
%%What-Is-This-Example%% \subsection{Example}
%%What-Is-This-Example%% \begin{xten}
%%What-Is-This-Example%% def dotProduct(a: Array[T](D), b: Array[T](D)): Array[Double](D) =
%%What-Is-This-Example%%   (new Array[T]([1:D.places],
%%What-Is-This-Example%%       (Point) => (new Array[T](D | here,
%%What-Is-This-Example%%                     (i): Point) => a(i)*b(i)).sum())).sum();
%%What-Is-This-Example%% \end{xten}
%%What-Is-This-Example%% 
%%What-Is-This-Example%% This code returns the inner product of two \xcd`T` vectors defined
%%What-Is-This-Example%% over the same (otherwise unknown) distribution. The result is the sum
%%What-Is-This-Example%% reduction of an array of \xcd`T` with one element at each place in the
%%What-Is-This-Example%% range of \xcd`D`. The value of this array at each point is the sum
%%What-Is-This-Example%% reduction of the array formed by multiplying the corresponding
%%What-Is-This-Example%% elements of \xcd`a` and \xcd`b` in the local sub-array at the current
%%What-Is-This-Example%% place.
%%What-Is-This-Example%% 

\section{Distributed Arrays}
\index{array!distributed}
\index{distributed array}
\index{\Xcd{DistArray}}
\index{DistArray}

Distributed arrays, instances of \xcd`DistArray[T]`, are very much like
\xcd`Array`s, except that they distribute information among multiple
\xcd`Place`s according to a \xcd`Dist` value passed in as a constructor
argument.  For example, the following code creates a distributed array holding
a thousand cells, each initialized to 0.0, distributed via a block
distribution over all places.
%~~gen
% package Arrays.Distarrays.basic.example;
% class Example {
% def example() {
%~~vis
\begin{xten}
val R <: Region = 1..1000;
val D <: Dist = Dist.makeBlock(R);
val da <: DistArray[Float] = DistArray.make[Float](D, (Point(1))=>0.0f);
\end{xten}
%~~siv
%}}
%~~neg




\section{Distributed Array Construction}\label{ArrayInitializer}
\index{distributed array!creation}
\index{\Xcd{DistArray}!creation}
\index{DistArray!creation}

\xcd`DistArray`s are instantiated by invoking one of the \xcd`make` factory
methods of the \xcd`DistArray` class.
A \xcd`DistArray` creation 
must take either an \xcd`Int` as an argument or a \xcd`Dist`. In the first
case,  a distributed array is created over the distribution \xcd`[0:N-1]->here`;
in the second over the given distribution. 

A distributed array creation operation may also specify an initializer
function.
The function is applied in parallel
at all points in the domain of the distribution. The
construction operation terminates locally only when the \xcd`DistArray` has been
fully created and initialized (at all places in the range of the
distribution).

For instance:
%~~gen
% package Arrays.DistArray.Construction.FeralWolf;
% class Example {
% def example() {
%~~vis
\begin{xten}
val data : DistArray[Int]
    = DistArray.make[Int](1..1000->here, ([i]:Point(1)) => i);
val blocked = Dist.makeBlock((1..1000)*(1..1000));
val data2 : DistArray[Int]
    = DistArray.make[Int](blocked, ([i,j]:Point(2)) => i*j);
\end{xten}
%~~siv
% }  }
%~~neg


{}\noindent 
The first declaration stores in \xcd`data` a reference to a mutable
distributed array with \xcd`1000` elements each of which is located in the
same place as the array. The element at \Xcd{[i]} is initialized to its index
\xcd`i`. 

The second declaration stores in \xcd`data2` a reference to a mutable
two-dimensional distributed array, whose coordinates both range from 1 to
1000, distributed in blocks over all \xcd`Place`s, 
initialized with \xcd`i*j`
at point \xcd`[i,j]`.

%%WHY-THIS-EXAMPLE%% In the following 
%%WHY-THIS-EXAMPLE%% %~x~gen
%%WHY-THIS-EXAMPLE%% % package Arrays.DistArrays.FlistArrays.GlistArrays;
%%WHY-THIS-EXAMPLE%% %~x~vis
%%WHY-THIS-EXAMPLE%% \begin{xten}
%%WHY-THIS-EXAMPLE%% val D1:Dist(1) = Dist.makeBlock(1..100);
%%WHY-THIS-EXAMPLE%% val D2:Dist(2) = Dist.makeBlock((1..100)*(-1..1));
%%WHY-THIS-EXAMPLE%% val ints : Array[Int]
%%WHY-THIS-EXAMPLE%%     = Array.make[Int](1000, ((i):Point) => i*i);
%%WHY-THIS-EXAMPLE%% val floats1 : Array[Float]
%%WHY-THIS-EXAMPLE%%     = Array.make[Float](D1, ((i):Point) => i*i as Float);
%%WHY-THIS-EXAMPLE%% val floats2 : Array[Float]
%%WHY-THIS-EXAMPLE%%    = Array.make[Float](D2, ((i,j):Point) => i+j as Float);;
%%WHY-THIS-EXAMPLE%% \end{xten}
%%WHY-THIS-EXAMPLE%% %~x~siv
%%WHY-THIS-EXAMPLE%% %
%%WHY-THIS-EXAMPLE%% %~x~neg


\section{Operations on Arrays and Distributed Arrays}

Arrays and distributed arrays share many operations.
In the following, let \xcd`a` be an array with base type T, and \xcd`da` be an
array with distribution \xcd`D` and base type \xcd`T`.




\subsection{Element operations}\index{array!access}
The value of \xcd`a` at a point \xcd`p` in its region of definition is
%~~exp~~`~~`~~a:Array[Int](3), p:Point(3) ~~
obtained by using the indexing operation \xcd`a(p)`. 
The value of \xcd`da` at \xcd`p` is similarly
%~~exp~~`~~`~~da:DistArray[Int](3), p:Point(3) ~~
\xcd`da(p)`
This operation
may be used on the left hand side of an assignment operation to update
the value: 
%~~stmt~~`~~`~~a:Array[Int](3), p:Point(3), t:Int ~~
\xcd`a(p)=t;`
and 
%~~stmt~~`~~`~~da:DistArray[Int](3), p:Point(3), t:Int ~~
\xcd`da(p)=t;`
The operator assignments, \xcd`a(i) += e` and so on,  are also
available. 

It is a runtime error to use either \xcd`da(p)` or \xcd`da(p)=v` at a place
other than \xcd`da.dist(p)`, \viz{} at the place that the element exists. 

%%HUH%%  For distributed array variables, the right-hand-side of an assignment must
%%HUH%%  have the same distribution \xcd`D` as an array being assigned. This
%%HUH%%  assignment involves
%%HUH%%  control communication between the sites hosting \xcd`D`. Each
%%HUH%%  site performs the assignment(s) of array components locally. The
%%HUH%%  assignment terminates when assignment has terminated at all
%%HUH%%  sites hosting \xcd`D`.

\subsection{Constant promotion}\label{ConstantArray}
\index{array!constant promotion}

For a region \xcd`R` and a value \xcd`v` of type \xcd`T`, the expression 
%~~genexp~~`~~`~~T~~R:Region, v:T ~~
\xcd`new Array[T](R, v)` 
produces an array on region \xcd`R` initialized with value \xcd`v`
Similarly, 
for a distribution \xcd`D` and a value \xcd`v` of
type \xcd`T` the expression 
%~~genexp~~`~~`~~T ~~D:Dist, v:T ~~
\xcd`DistArray.make[T](D, (Point(D.rank))=>v)`
constructs a distributed array with
distribution \xcd`D` and base type \xcd`T` initialized with \xcd`v`
at every point.

Note that \xcd`Array`s are constructed by constructor calls, but
\xcd`DistArrays` are constructed by calls to the factory methods
\xcd`DistArray.make`. This is because \xcd`Array`s are fairly simple objects,
but \xcd`DistArray`s may be implemented by different classes for different
distributions. The use of the factory method gives the library writer the
freedom to select appropriate implementations.


\subsection{Restriction of an array}\index{array!restriction}

Let \xcd`R` be a sub-region of \xcd`da.region`. Then 
%~~exp~~`~~`~~da:DistArray[Int](3), p:Point(3), R: Region(da.rank) ~~
\xcd`da | R`
represents the sub-\xcd`DistArray` of \xcd`da` on the region \xcd`R`.
That is, \xcd`da | R` has the same values as \xcd`da` when subscripted by a
%~~exp~~`~~`~~R:Region, da: DistArray[Int]{da.region.rank == R.rank} ~~
point in region \xcd`R && da.region`, and is undefined elsewhere.
`
Recall that a rich set of operators are available on distributions
(\Sref{XtenDistributions}) to obtain sub-distributions
(e.g. restricting to a sub-region, to a specific place etc).

%%GONE-AWAY%%  \subsection{Assembling an array}
%%GONE-AWAY%%  Let \xcd`da1,da2` be distributed arrays of the same base type \xcd`T` defined over
%%GONE-AWAY%%  distributions \xcd`D1` and \xcd`D2` respectively. 
%%GONE-AWAY%%  \paragraph{Assembling arrays over disjoint regions}\index{array!union!disjoint}
%%GONE-AWAY%%  
%%GONE-AWAY%%  
%%GONE-AWAY%%  If \xcd`D1` and \xcd`D2` are disjoint then the expression 
%%GONE-AWAY%%  %~x~genexp~~`~~`~~T ~~ da1: Array[T], da2: Array[T](da1.rank)~~
%%GONE-AWAY%%  \xcd`da1 || da2` denotes the unique array of base type \xcd`T` defined over the
%%GONE-AWAY%%  distribution \xcd`D1 || D2` such that its value at point \xcd`p` is
%%GONE-AWAY%%  \xcd`a1(p)` if \xcd`p` lies in \xcd`D1` and \xcd`a2(p)`
%%GONE-AWAY%%  otherwise. This array is a reference (value) array if \xcd`a1` is.
%%GONE-AWAY%%  
%%GONE-AWAY%%  \paragraph{Overlaying an array on another}\index{array!union!asymmetric}
%%GONE-AWAY%%  The expression
%%GONE-AWAY%%  \xcd`a1.overlay(a2)` (read: the array \xcd`a1` {\em overlaid with} \xcd`a2`)
%%GONE-AWAY%%  represents an array whose underlying region is the union of that of
%%GONE-AWAY%%  \xcd`a1` and \xcd`a2` and whose distribution maps each point \xcd`p`
%%GONE-AWAY%%  in this region to \xcd`D2(p)` if that is defined and to \xcd`D1(p)`
%%GONE-AWAY%%  otherwise. The value \xcd`a1.overlay(a2)(p)` is \xcd`a2(p)` if it is defined and \xcd`a1(p)` otherwise.
%%GONE-AWAY%%  
%%GONE-AWAY%%  This array is a reference (value) array if \xcd`a1` is.
%%GONE-AWAY%%  
%%GONE-AWAY%%  The expression \xcd`a1.update(a2)` updates the array \xcd`a1` in place
%%GONE-AWAY%%  with the result of \xcd`a1.overlay(a2)`.
%%GONE-AWAY%%  
%%GONE-AWAY%%  \oldtodo{Define Flooding of arrays}
%%GONE-AWAY%%  
%%GONE-AWAY%%  \oldtodo{Wrapping an array}
%%GONE-AWAY%%  
%%GONE-AWAY%%  \oldtodo{Extending an array in a given direction.}
%%GONE-AWAY%%  
\subsection{Operations on Whole Arrays}

\paragraph{Pointwise operations}\label{ArrayPointwise}\index{array!pointwise operations}
The unary \xcd`map` operation applies a function to each element of
a distributed or non-distributed array, returning a new distributed array with
the same distribution, or a non-distributed array with the same region.
For example, the following produces an array of cubes: 
%~~gen
%package Arrays.arrays.ginungagap.bakery.treats;
%class Example{
%def example() {
%~~vis
\begin{xten}
val A = new Array[Int](1..10, (p:Point(1))=>p(0) );
// A = 1,2,3,4,5,6,7,8,9,10
val cube = (i:Int) => i*i*i;
val B = A.map(cube);
// B = 1,8,27,64,216,343,512,729,1000
\end{xten}
%~~siv
%} } 
%~~neg

A variant operation lets you specify the array \Xcd{B} into which the result
will be stored.   
%~~gen
%package Arrays.arrays.ginungagap.bakery.treats.doomed;
%class Example{
%def example() {
%~~vis
\begin{xten}
val A = new Array[Int](1..10, (p:Point(1))=>p(0) );
// A = 1,2,3,4,5,6,7,8,9,10
val cube = (i:Int) => i*i*i;
val B = new Array[Int](A.region); // B = 0,0,0,0,0,0,0,0,0,0
A.map(B, cube);
// B = 1,8,27,64,216,343,512,729,1000
\end{xten}
%~~siv
%} } 
%~~neg
\noindent
This is convenient if you have an already-allocated array lying around unused.
In particular, it can be used if you don't need \Xcd{A} afterwards and want to
reuse its space:
%~~gen
%package Arrays.arrays.ginungagap.bakery.treats.doomed.spackled;
%class Example{
%def example() {
%~~vis
\begin{xten}
val A = new Array[Int](1..10, (p:Point(1))=>p(0) );
// A = 1,2,3,4,5,6,7,8,9,10
val cube = (i:Int) => i*i*i;
A.map(A, cube);
// A = 1,8,27,64,216,343,512,729,1000
\end{xten}
%~~siv
%} } 
%~~neg


The binary \xcd`map` operation takes a binary function and
another
array over the same region or distributed array over the same  distribution,
and applies the function 
pointwise to corresponding elements of the two arrays, returning
a new array or distributed array of the same shape.
The following code adds two distributed arrays: 
%~~gen
% package Arrays.Pointwise.Pointless.Map2;
% class Example{
%~~vis
\begin{xten}
static def add(da:DistArray[Int], db: DistArray[Int]{da.dist==db.dist})
    = da.map(db, Int.+);
\end{xten}
%~~siv
%}
%~~neg



\paragraph{Reductions}\label{ArrayReductions}\index{array!reductions}

Let \xcd`f` be a function of type \xcd`(T,T)=>T`.  Let
\xcd`a` be an array over base type \xcd`T`.
Let \xcd`unit` be a value of type \xcd`T`.
Then the
%~~genexp~~`~~`~~ T ~~ f:(T,T)=>T, a : Array[T], unit:T ~~
operation \xcd`a.reduce(f, unit)` returns a value of type \xcd`T` obtained
by combining all the elements of \xcd`a` by use of  \xcd`f` in some unspecified order
(perhaps in parallel).   
The following code gives one method which 
meets the definition of \Xcd{reduce},
having a running total \Xcd{r}, and accumulating each value \xcd`a(p)` into it
using \Xcd{f} in turn.  (This code is simply given as an example; \Xcd{Array}
has this operation defined already.)
%~~gen
%package Arrays.Reductions.And.Eliminations;
% class Example {
%~~vis
\begin{xten}
def oneWayToReduce[T](a:Array[T], f:(T,T)=>T, unit:T):T {
  var r : T = unit;
  for(p in a.region) r = f(r, a(p));
  return r;
}
\end{xten}
%~~siv
%}
%~~neg


For example,  the following sums an array of integers.  \Xcd{f} is addition,
and \Xcd{unit} is zero.  
%~~gen
% package Arrays.Reductions.And.Emulsions;
% class Example {
% def example() {
%~~vis
\begin{xten}
val a = [1,2,3,4];
val sum = a.reduce(Int.+, 0); 
assert(sum == 10); // 10 == 1+2+3+4
\end{xten}
%~~siv
%}}
%~~neg

Other orders of evaluation, degrees of parallelism, and applications of
\Xcd{f(x,unit)} and \xcd`f(unit,x)`are also correct.
In order to guarantee that the result is precisely
determined, the  function \xcd`f` should be associative and
commutative, and the value \xcd`unit` should satisfy
\xcd`f(unit,x)` \xcd`==` \xcd`x` \xcd`==` \xcd`f(x,unit)`
for all \Xcd{x:T}.  




\xcd`DistArray`s have the same operation.
This operation involves communication between the places over which
the \xcd`DistArray` is distributed. The \Xten{} implementation guarantees that
only one value of type \xcd`T` is communicated from a place as part of
this reduction process.

\paragraph{Scans}\label{ArrayScans}\index{array!scans}


Let \xcd`f:(T,T)=>T`, \xcd`unit:T`, and \xcd`a` be an \xcd`Array[T]` or
\xcd`DistArray[T]`.  Then \xcd`a.scan(f,unit)` is the array or distributed
array of type \xcd`T` whose {$i$}th element in canonical order is the
reduction by \xcd`f` with unit \xcd`unit` of the first {$i$} elements of
\xcd`a`. 


This operation involves communication between the places over which the
distributed array is distributed. The \Xten{} implementation will endeavour to
minimize the communication between places to implement this operation.

Other operations on arrays, distributed arrays, and the related classes may be
found in the \xcd`x10.array` package.
	
\chapter{Annotations}\label{XtenAnnotations}\index{annotations}


\Xten{} provides an 
an annotation system  system for to allow the
compiler to be extended with new static analyses and new
transformations.

Annotations are constraint-free interface types that decorate the abstract syntax tree
of an \Xten{} program.  The \Xten{} type-checker ensures that an annotation
is a legal interface type.
In \Xten{}, interfaces may declare
both methods and properties.  Therefore, like any interface type, an
annotation may instantiate
one or more of its interface's properties.
%%PLUGINNERY%%  Unlike with Java
%%PLUGINNERY%%  annotations,
%%PLUGINNERY%%  property initializers need not be
%%PLUGINNERY%%  compile-time constants;
%%PLUGINNERY%%  however, a given compiler plugin
%%PLUGINNERY%%  may do additional checks to constrain the allowable
%%PLUGINNERY%%  initializer expressions.
%%PLUGINNERY%%  The \Xten{} type-checker does not check that
%%PLUGINNERY%%  all properties of an annotation are initialized,
%%PLUGINNERY%%  although this could be enforced by
%%PLUGINNERY%%  a compiler plugin.

\section{Annotation syntax}

The annotation syntax consists of an ``\texttt{@}'' followed by an interface type.

%##(Annotations Annotation
\begin{bbgrammar}
%(FROM #(prod:Annotations)#)
         Annotations \: Annotation & (\ref{prod:Annotations}) \\
                     \| Annotations Annotation \\
%(FROM #(prod:Annotation)#)
          Annotation \: \xcd"@" NamedTypeNoConstraints & (\ref{prod:Annotation}) \\
\end{bbgrammar}
%##)

Annotations can be applied to most syntactic constructs in the language
including class declarations, constructors, methods, field declarations,
local variable declarations and formal parameters, statements,
expressions, and types.
Multiple occurrences of the same annotation (i.e., multiple
annotations with the same interface type) on the same entity are permitted.

%%OBSOLETE%% \begin{grammar}
%%OBSOLETE%% ClassModifier \: Annotation \\
%%OBSOLETE%% InterfaceModifier \: Annotation \\
%%OBSOLETE%% FieldModifier \: Annotation \\
%%OBSOLETE%% MethodModifier \: Annotation \\
%%OBSOLETE%% VariableModifier \: Annotation \\
%%OBSOLETE%% ConstructorModifier \: Annotation \\
%%OBSOLETE%% AbstractMethodModifier \: Annotation \\
%%OBSOLETE%% ConstantModifier \: Annotation \\
%%OBSOLETE%% Type \: AnnotatedType \\
%%OBSOLETE%% AnnotatedType \: Annotation\plus Type \\
%%OBSOLETE%% Statement \: AnnotatedStatement \\
%%OBSOLETE%% AnnotatedStatement \: Annotation\plus Statement \\
%%OBSOLETE%% Expression \: AnnotatedExpression \\
%%OBSOLETE%% AnnotatedExpression \: Annotation\plus Expression \\
%%OBSOLETE%% \end{grammar}

\noindent
Recall that interface types may have dependent parameters.

\noindent
The following examples illustrate the syntax:

\begin{itemize}
\item Declaration annotations:
\begin{xtennoindent}
  // class annotation
  @Value
  class Cons { ... }

  // method annotation
  @PreCondition(0 <= i && i < this.size)
  public def get(i: Int): Object { ... }

  // constructor annotation
  @Where(x != null)
  def this(x: T) { ... }

  // constructor return type annotation
  def this(x: T): C@Initialized { ... }

  // variable annotation
  @Unique x: A;
\end{xtennoindent}
\item Type annotations:
\begin{xtennoindent}
  List@Nonempty

  Int@Range(1,4)

  Array[Array[Double]]@Size(n * n)
\end{xtennoindent}
\item Expression annotations:
\begin{xtennoindent}
  m()  @RemoteCall
\end{xtennoindent}
\item Statement annotations:
\begin{xtennoindent}
  @Atomic { ... }

  @MinIterations(0)
  @MaxIterations(n)
  for (var i: Int = 0; i < n; i++) { ... }

  // An annotated empty statement ;
  @Assert(x < y);
\end{xtennoindent}
\end{itemize}

\section{Annotation declarations}

Annotations are declared as interfaces.  They must be
subtypes of the interface \texttt{x10.lang.annotation.Annotation}.
Annotations on particular static entities must extend the corresponding
\xcd`Annotation` subclasses, as follows: 
\begin{itemize}
\item Expressions---\xcd`ExpressionAnnotation`
\item Statements---\xcd`StatementAnnotation`
\item Classes---\xcd`ClassAnnotation`
\item Fields---\xcd`FieldAnnotation`
\item Methods---\xcd`MethodAnnotation`
\item Imports---\xcd`ImportAnnotation`
\item Packages---\xcd`PackageAnnotation`
\end{itemize}

%% vj Thu Sep 19 21:39:41 EDT 2013
% updated for v2.4 -- no change.
\chapter{Interoperability with Other Languages}
\label{NativeCode}
\index{native code}
\label{Interoperability}
\index{interoperability}

The ability to interoperate with other programming languages is an
essential feature of the \Xten{} implementation.  Cross-language
interoperability enables both the incremental adoption of \Xten{} in
existing applications and the usage of existing libraries and
frameworks by newly developed \Xten{} programs. 

There are two primary interoperability scenarios that are supported by
\XtenCurrVer{}: inline substitution of fragments of foreign code for
\Xten program fragments (expressions, statements) and external linkage
to foreign code.

\section{Embedded Native Code Fragments}

The
\xcd`@Native(lang,code) Construct` annotation from \xcd`x10.compiler.Native` in
\Xten{} can be used to tell the \Xten{} compiler to substitute \xcd`code` for
whatever it would have generated when compiling \xcd`Construct`
with the \xcd`lang` back end.

The compiler cannot analyze native code the same way it analyzes \Xten{} code.  In
particular, \xcd`@Native` fields and methods must be explicitly typed; the
compiler will not infer types.


\subsection{Native {\tt static} Methods}

\xcd`static` methods can be given native implementations.  Note that these
implementations are syntactically {\em expressions}, not statements, in C++ or
\Java{}.   Also, it is possible (and common) to provide native implementations
into both \Java{} and C++ for the same method.
%~~gen ^^^ extern10
% package Extern.or_current_turn;
%~~vis
\begin{xten}
import x10.compiler.Native;
class Son {
  @Native("c++", "printf(\"Hi!\")")
  @Native("java", "System.out.println(\"Hi!\")")
  static def printNatively():void = {};
}
\end{xten}
%~~siv
%
%~~neg

If only some back-end languages are given, the \Xten{} code will be used for the
remaining back ends: 
%~~gen ^^^ extern20
% package Extern.or.burn;
%~~vis
\begin{xten}
import x10.compiler.Native;
class Land {
  @Native("c++", "printf(\"Hi from C++!\")")
  static def example():void = {
    x10.io.Console.OUT.println("Hi from X10!");
  };
}
\end{xten}
%~~siv
%
%~~neg

The \xcd`native` modifier on methods indicates that the method must not have
an \Xten{} code body, and \xcd`@Native` implementations must be given for all back
ends:
%~~gen ^^^ extern30
% package Extern.or_maybe_getting_back_together;
%~~vis
\begin{xten}
import x10.compiler.Native;
class Plants {
  @Native("c++", "printf(\"Hi!\")")
  @Native("java", "System.out.println(\"Hi!\")")
  static native def printNatively():void;
}
\end{xten}
%~~siv
%
%~~neg


Values may be returned from external code to \Xten{}.  Scalar types in \Java{} and
C++ correspond directly to the analogous types in \Xten{}.  
%~~gen ^^^ extern40
% package Extern.hardy;
%~~vis
\begin{xten}
import x10.compiler.Native;
class Return {
  @Native("c++", "1")
  @Native("java", "1")
  static native def one():Int;
}
\end{xten}
%~~siv
%
%~~neg
Types are {\em not} inferred for methods marked as \xcd`@Native`.

Parameters may be passed to external code.  \xcd`(#1)`  is the first parameter,
\xcd`(#2)` the second, and so forth.  (\xcd`(#0)` is the name of the enclosing
class, or the \xcd`this` variable.)  Be aware that this is macro substitution
rather than normal parameter 
passing; \eg, if the first actual parameter is \xcd`i++`, and \xcd`(#1)`
appears twice in the external code, \xcd`i` will be incremented twice.
For example, a (ridiculous) way to print the sum of two numbers is: 
%~~gen ^^^ extern50
% package Extern.or_turnabout_is_fair_play;
%~~vis
\begin{xten}
import x10.compiler.Native;
class Species {
  @Native("c++","printf(\"Sum=%d\", ((#1)+(#2)) )")
  @Native("java","System.out.println(\"\" + ((#1)+(#2)))")
  static native def printNatively(x:Int, y:Int):void;
}
\end{xten}
%~~siv
%
%~~neg


Static variables in the class are available in the external code.  For \Java{},
the static variables are used with their \Xten{} names.  For C++, the names
must be mangled, by use of the \xcd`FMGL` macro.  

%~~gen ^^^ extern60
%package Extern.or.Die;
%~~vis
\begin{xten}
import x10.compiler.Native;
class Ability {
  static val A : Int = 1n;
  @Native("java", "A+2")
  @Native("c++", "Ability::FMGL(A)+2")
  static native def fromStatic():Int;
}
\end{xten}
%~~siv
%
%~~neg




\subsection{Native Blocks}

Any block may be annotated with \xcd`@Native(lang,stmt)`, indicating that, in
the given back end, it should be implemented as \xcd`stmt`. All 
variables  from the surrounding context are available inside \xcd`stmt`. For
example, the method call \xcd`born.example(10n)`, if compiled to \Java{}, changes
the field \xcd`y` of a \xcd`Born` object to 10. If compiled to C++ (for which
there is no \xcd`@Native`), it sets it to 3. 
%~~gen ^^^ extern70
%package Extern.me.plz; 
%~~vis
\begin{xten}
import x10.compiler.Native;
class Born {
  var y : Int = 1n; 
  public def example(x:Int):Int{
    @Native("java", "y=x;") 
    {y = 3n;}
    return y;
  }
}
\end{xten}
%~~siv
%
%~~neg

Note that the code being replaced is a statement -- the block \xcd`{y = 3n;}`
in this case -- so the replacement should also be a statement. 


Other \Xten{} constructs may or may not be available in \Java{} and/or C++ code.  For
example, type variables do not correspond exactly to type variables in either
language, and may not be available there.  The exact compilation scheme is
{\em not} fully specified.  You may inspect the generated \Java{} or C++ code and
see how to do specific things, but there is no guarantee that fancy external
coding will continue to work in later versions of \Xten{}.



The full facilities of C++ or \Java{} are available in native code blocks.
However, there is no guarantee that advanced features behave sensibly. You
must follow the exact conventions that the code generator does, or you will
get unpredictable results.  Furthermore, the code generator's conventions may
change without notice or documentation from version to version.  In most cases
the  code should either be a very simple expression, or a method
or function call to external code.


\section{Interoperability with External Java Code}

With Managed \Xten{}, we can seamlessly call existing \Java{} code from \Xten{},
and call \Xten{} code from \Java{}.  We call this the 
\emph{Java interoperability}~\cite{TakeuchiX1013} feature.

By combining \Java{} interoperability with \Xten{}'s distributed
execution features, we can even make existing \Java{} applications, which
are originally designed to run on a single \Java{} VM, scale-out with
minor modifications.

\subsection{How Java program is seen in X10}

Managed \Xten{} does not pre-process the existing \Java{} code to make it
accessible from \Xten{}.  \Xten{} programs compiled with Managed \Xten{} will call
existing \Java{} code as is.

\paragraph{Types}

In \Xten{}, both at compile time and run time, there is no way to
distinguish \Java{} types from \Xten{} types.  \Java{} types can be referred to
with regular \xcd{import} statement, or their qualified names.  The
package \xcd{java.lang} is not auto-imported into \Xten.  In Managed
x10, the resolver is enhanced to resolve types with \Xten{} source files
in the source path first, then resolve them with \Java{} class files in
the class path. Note that the resolver does not resolve types with
\Java{} source files, therefore \Java{} source files must be compiled in
advance.  To refer to \Java{} types listed in
Tables~\ref{tab:specialtypes}, and \ref{tab:otherspecialtypes}, which
include all \Java{} primitive types, use the corresponding \Xten{} type
(e.g. use \xcd{x10.lang.Int} (or in short, \xcd{Int}) instead of
\xcd{int}).

\paragraph{Fields}

Fields of \Java{} types are seen as fields of \Xten{} types.

Managed \Xten{} does not change the static initialization semantics of
\Java{} types, which is per-class, at load time, and per-place (\Java{} VM),
therefore, it is subtly different than the per-field lazy
initialization semantics of \Xten{} static fields.

\paragraph{Methods}

Methods of \Java{} types are seen as methods of \Xten{} types.

\paragraph{Generic types}

Generic \Java{} types are seen as their raw types 
(\S 4.8 in~\cite{java-lang-spec2005}).  Raw type is a mechanism to handle generic
\Java{} types as non-generic types, where the type parameters are assumed
as \verb|java.lang.Object| or their upperbound if they have it.  \Java{}
introduced generics and raw type at the same time to facilitate
generic \Java{} code interfacing with non-generic legacy \Java{} code.
Managed \Xten{} uses this mechanism for a slightly different purpose.
\Java{} erases type parameters at compile time, whereas \Xten{} preserves
their values at run time.  To manifest this semantic gap in generics,
Managed \Xten{} represents \Java{} generic types as raw types and eliminates
type parameters at source code level.  For more detailed discussions,
please refer to~\cite{TakeuchiX1011,TakeuchiX1012}.

\paragraph{Arrays}

\Xten{} rail and array types are generic types whose representation is different
from \Java{} array types.

Managed \Xten{} provides a special \Xten{} type
\xcd{x10.interop.Java.array[T]} which represents \Java{} array type
\xcd{T[]}.  Note that for \Xten{} types in Table~\ref{tab:specialtypes},
this type means the \Java{} array type of their primary type.  For
example, \xcd{array[Int]} and \xcd{array[String]} mean
\xcd{int[]} and \xcd{java.lang.String[]}, respectively.  Managed \Xten{}
also provides conversion methods between \Xten{} \xcd`Rail`s and \Java{}
arrays (\xcd{Java.convert[T](a:Rail[T]):array[T]} and
\xcd{Java.convert[T](a:array[T]):Rail[T]}),
and creation methods for \Java{} arrays 
(\xcd{Java.newArray[T](d0:Int):array[T]}
etc.).

\paragraph{Exceptions}

The \XtenCurrVer{} exception hierarchy has been designed so that there is a
natural correspondence with the \Java{} exception hierarchy. As shown in
Table~\ref{tab:otherspecialtypes}, many commonly used \Java{}
exception types are directly mapped to \Xten{} exception types. 
Exception types that are thus aliased may be caught (and thrown) using
either their \Java{} or \Xten types.  In \Xten code, it is stylistically
preferable to use the \Xten type to refer to the exception as shown in 
Figure~\ref{fig:javaexceptions}.

%----------------
\begin{figure}
\begin{xten}
import x10.interop.Java;
public class XClass {   
  public static def main(args:Rail[String]):void {
    try {
      val a = Java.newArray[Int](2n);
      a(0n) = 0n;
      a(1n) = 1n;
      a(2n) = 2n;
    } catch (e:x10.lang.ArrayIndexOutOfBoundsException) {
      Console.OUT.println(e);
    }
  }
}
\end{xten}
\begin{verbatim}
> x10c -d bin src/XClass.x10
> x10 -cp bin XClass
x10.lang.ArrayIndexOutOfBoundsException: Array index out of range: 2
\end{verbatim}
\caption{Java exceptions in X10}
\label{fig:javaexceptions}
\end{figure}
%----------------

\paragraph{Compiling and executing X10 programs}

We can compile and run \Xten{} programs that call existing \Java{} code with
the same \verb|x10c| and \verb|x10| command by specifying the location
of \Java{} class files or jar files that your \Xten{} programs refer to, with
\verb|-classpath| (or in short, \verb|-cp|) option.

\subsection{How X10 program is translated to Java}

Managed \Xten{} translates \Xten{} programs to \Java{} class files. 

\Xten{} does not provide a \Java{} reflection-like mechanism to resolve \Xten{}
types, methods, and fields with their names at runtime, nor a code
generation tool, such as \verb|javah|, to generate stub code to access
them from other languages.  \Java{} programmers, therefore, need to
access \Xten{} types, methods, and fields in the generated \Java{} code
directly as they access \Java{} types, methods, and fields.  To make it
possible, \Java{} programmers need to understand how \Xten{} programs are
translated to \Java{}.

Some aspects of the \Xten{} to \Java{} translation scheme may change in
future version of \Xten{}; therefore in this document only a stable
subset of translation scheme will be explained.  Although it is a
subset, it has been extensively used by \Xten{} core team and proved to be
useful to develop \Java{} Hadoop interop layer for a Main-memory Map
Reduce (M3R) engine~\cite{Shinnar12M3R} in \Xten{}.

In the following discussions, we deliberately ignore generic \Xten{}
types because the translation of generics is an area of active
development and will undergo some changes in future versions of \Xten{}.
For those who are interested in the implementation of generics
in Managed \Xten{}, please consult~\cite{TakeuchiX1012}.  We also don't
cover function types, function values, and all non-static methods.
Although slightly outdated, another paper~\cite{TakeuchiX1011}, which
describes translation scheme in \Xten{} 2.1.2, is still useful to
understand the overview of \Java{} code generation in Managed \Xten{}.


\paragraph{Types}

\Xten{} classes and structs are translated to \Java{} classes with the same
names.  \Xten{} interfaces are translated to \Java{} interfaces with the same
names.

Table~\ref{tab:specialtypes} shows the list of special types that are
mapped to \Java{} primitives.  Primitives are their primary
representations that are useful for good performance.  Wrapper classes
are used when the reference types are needed.  Each wrapper class has
two static methods \verb|$box()| and \verb|$unbox()| to convert its
value from primary representation to wrapper class, and vice versa,
and \Java{} backend inserts their calls as needed.  An you notice, every
unsigned type uses the same \Java{} primitive as its corresponding signed
type for its representation.

Table~\ref{tab:otherspecialtypes} shows a non-exhaustive list of
another kind of special types that are mapped (not translated) to \Java{}
types.  As you notice, since an interface \verb|Any| is mapped to a
class |java.lang.Object| and \verb|Object| is hidden from the
language, there is no direct way to create an instance of
\verb|Object|. As a workaround, \verb|Java.newObject()| is provided.

As you also notice, \verb|x10.lang.Comparable[T]| is mapped to \verb|java.lang.Comparable|.
This is needed to map \verb|x10.lang.String|, which implements \verb|x10.lang.Compatable[String]|, to \verb|java.lang.String| for performance, but as a trade off, this mapping results in the lost of runtime type information for \verb|Comparable[T]| in Managed \Xten{}.
The runtime of Managed \Xten{} has built-in knowledge for \verb|String|, but for other \Java{} classes that implement \verb|java.lang.Comparable|, \verb|instanceof Comparable[Int]| etc. may return incorrect results.
In principle, it is impossible to map \Xten{} generic type to the existing \Java{} generic type without losing runtime type information.

%----------------
\begin{table}
%\scriptsize
\small
\centering
\mbox{
\begin{tabular}{|lr|lr|l|}												   \hline
\multicolumn{2}{|c|}{\textbf{X10}}	& \multicolumn{2}{|c|}{\textbf{Java (primary)}}	& \textbf{Java (wrapper class)}	\\ \hline
															   \hline
{\tt x10.lang.Byte}	& {\tt 1y}	& {\tt byte}		& {\tt (byte)1}		& {\tt x10.core.Byte}		\\ \hline
{\tt x10.lang.UByte}	& {\tt 1uy}	& {\tt byte}		& {\tt (byte)1}		& {\tt x10.core.UByte}		\\ \hline
{\tt x10.lang.Short}	& {\tt 1s}	& {\tt short}		& {\tt (short)1}	& {\tt x10.core.Short}		\\ \hline
{\tt x10.lang.UShort}	& {\tt 1us}	& {\tt short}		& {\tt (short)1}	& {\tt x10.core.UShort} 	\\ \hline
{\tt x10.lang.Int}	& {\tt 1n}	& {\tt int}		& {\tt 1}		& {\tt x10.core.Int}		\\ \hline
{\tt x10.lang.UInt}	& {\tt 1un}	& {\tt int}		& {\tt 1}		& {\tt x10.core.UInt}		\\ \hline
{\tt x10.lang.Long}	& {\tt 1}	& {\tt long}		& {\tt 1l}		& {\tt x10.core.Long}	 	\\ \hline
{\tt x10.lang.ULong}	& {\tt 1u}	& {\tt long}		& {\tt 1l}		& {\tt x10.core.ULong}	 	\\ \hline
{\tt x10.lang.Float}	& {\tt 1.0f}	& {\tt float}		& {\tt 1.0f}		& {\tt x10.core.Float}	 	\\ \hline
{\tt x10.lang.Double}	& {\tt 1.0}	& {\tt double}		& {\tt 1.0}		& {\tt x10.core.Double} 	\\ \hline
{\tt x10.lang.Char}	& {\tt 'c'}	& {\tt char}		& {\tt 'c'}		& {\tt x10.core.Char}		\\ \hline
{\tt x10.lang.Boolean}	& {\tt true}	& {\tt boolean}		& {\tt true}		& {\tt x10.core.Boolean}	\\ \hline
%{\tt x10.lang.String} 	& {\tt "abc"}	& {\tt java.lang.String}& {\tt "abc"}		& {\tt x10.core.String}		\\ \hline
\end{tabular}
}
\caption{X10 types that are mapped to Java primitives}
\label{tab:specialtypes}
\end{table}
%----------------


%----------------
\begin{table}
%\scriptsize
\small
\centering
\mbox{
\begin{tabular}{|l|l|}										   \hline
\multicolumn{1}{|c|}{\textbf{X10}}		& \multicolumn{1}{|c|}{\textbf{Java}}		\\ \hline
												   \hline
{\tt x10.lang.Any} 				& {\tt java.lang.Object}			\\ \hline
{\tt x10.lang.Comparable[T]} 			& {\tt java.lang.Comparable}			\\ \hline
{\tt x10.lang.String}		 		& {\tt java.lang.String}			\\ \hline
{\tt x10.lang.CheckedThrowable}		 	& {\tt java.lang.Throwable}			\\ \hline
{\tt x10.lang.CheckedException}		 	& {\tt java.lang.Exception}			\\ \hline
{\tt x10.lang.Exception} 			& {\tt java.lang.RuntimeException}		\\ \hline
{\tt x10.lang.ArithmeticException} 		& {\tt java.lang.ArithmeticException}		\\ \hline
{\tt x10.lang.ClassCastException} 		& {\tt java.lang.ClassCastException}		\\ \hline
{\tt x10.lang.IllegalArgumentException} 	& {\tt java.lang.IllegalArgumentException}	\\ \hline
{\tt x10.util.NoSuchElementException}	 	& {\tt java.util.NoSuchElementException}	\\ \hline
{\tt x10.lang.NullPointerException} 		& {\tt java.lang.NullPointerException}		\\ \hline
{\tt x10.lang.NumberFormatException} 		& {\tt java.lang.NumberFormatException}		\\ \hline
{\tt x10.lang.UnsupportedOperationException} 	& {\tt java.lang.UnsupportedOperationException}	\\ \hline
{\tt x10.lang.IndexOutOfBoundsException} 	& {\tt java.lang.IndexOutOfBoundsException}	\\ \hline
{\tt x10.lang.ArrayIndexOutOfBoundsException} 	& {\tt java.lang.ArrayIndexOutOfBoundsException}\\ \hline
{\tt x10.lang.StringIndexOutOfBoundsException} 	& {\tt java.lang.StringIndexOutOfBoundsException}\\ \hline
{\tt x10.lang.Error} 				& {\tt java.lang.Error}				\\ \hline
{\tt x10.lang.AssertionError} 			& {\tt java.lang.AssertionError}		\\ \hline
{\tt x10.lang.OutOfMemoryError} 		& {\tt java.lang.OutOfMemoryError}		\\ \hline
{\tt x10.lang.StackOverflowError} 		& {\tt java.lang.StackOverflowError}		\\ \hline
{\tt void} 					& {\tt void}					\\ \hline
\end{tabular}
}
\caption{X10 types that are mapped (not translated) to Java types}
\label{tab:otherspecialtypes}
\end{table}
%----------------


\paragraph{Fields}

As shown in Figure~\ref{fig:fields}, instance fields of \Xten{} classes and structs are translated to the instance fields of the same names of the generated \Java{} classes.
Static fields of \Xten{} classes and structs are translated to the static methods of the generated \Java{} classes, whose name has \verb|get$| prefix.
Static fields of \Xten{} interfaces are translated to the static methods of the special nested class named \verb|$Shadow| of the generated \Java{} interfaces.

%----------------
\begin{figure}
\begin{xten}
class C {
  static val a:Int = ...;
  var b:Int;
}
interface I {
  val x:Int = ...;
}
\end{xten}
\begin{xten}
class C {
  static int get$a() { return ...; }
  int b;
}
interface I {
  abstract static class $Shadow {
    static int get$x() { return ...; }
  }
}
\end{xten}
\caption{X10 fields in Java}
\label{fig:fields}
\end{figure}
%----------------


\paragraph{Methods}

As shown in Figure~\ref{fig:methods}, methods of \Xten{} classes or structs are translated to the methods of the same names of the generated \Java{} classes.
Methods of \Xten{} interfaces are translated to the methods of the same names of the generated \Java{} interfaces.

Every method whose return type has two representations, such as the types in Table~\ref{tab:specialtypes}, will have \verb|$O| suffix with its name.
For example, \verb|def f():Int| in \Xten{} will be compiled as \verb|int f$O()| in \Java{}.
For those who are interested in the reason, please look at our paper~\cite{TakeuchiX1012}.

%----------------
\begin{figure}
\begin{xten}
interface I {
  def f():Int;
  def g():Any;
}
class C implements I {
  static def s():Int = 0n;
  static def t():Any = null;
  public def f():Int = 1n;
  public def g():Any = null;
}
\end{xten}
\begin{xten}
interface I {
  int f$O();
  java.lang.Object g();
}
class C implements I {
  static int s$O() { return 0; }
  static java.lang.Object t() { return null; }
  public int f$O() { return 1; }
  public java.lang.Object g() { return null; }
}
\end{xten}
\caption{X10 methods in Java}
\label{fig:methods}
\end{figure}
%----------------


\paragraph{Compiling Java programs}

To compile \Java{} program that calls \Xten{} code, you should use
\verb|x10cj| command instead of javac (or whatever your \Java{}
compiler). It invokes the post \Java{}-compiler of \verb|x10c| with the
appropriate options. You need to specify the location of \Xten{}-generated
class files or jar files that your \Java{} program refers to.

\verb|x10cj -cp MyX10Lib.jar MyJavaProg.java|


\paragraph{Executing Java programs}

Before executing any \Xten{}-generated \Java{} code, the runtime of Managed
\Xten{} needs to be set up properly at each place.  To set up the runtime,
a special launcher named \verb|runjava| is used to run \Java{} programs.
All \Java{} programs that call \Xten{} code should be launched with it, and
no other mechanisms, including direct execution with \verb|java| command, are
supported.

\begin{verbatim}
Usage: runjava <Java-main-class> [arg0 arg1 ...]
\end{verbatim}


\section{Interoperability with External C and C++ Code}

C and C++ code can be linked to \Xten{} code, either by writing auxiliary C++ files and
adding them with suitable annotations, or by linking libraries.

\subsection{Auxiliary C++ Files}

Auxiliary C++ code can be written in \xcd`.h` and \xcd`.cc` files, which
should be put in the same directory as the the \Xten{} file using them.
Connecting with the library uses the \xcd`@NativeCPPInclude(dot_h_file_name)`
annotation to include the header file, and the 
\xcd`@NativeCPPCompilationUnit(dot_cc_file_name)` annotation to include the
C++ code proper.  For example: 

{\bf MyCppCode.h}: 
\begin{xten}
void foo();
\end{xten}


{\bf MyCppCode.cc}:
\begin{xten}
#include <cstdlib>
#include <cstdio>
void foo() {
    printf("Hello World!\n");
}
\end{xten}

{\bf Test.x10}:
\begin{xten}
import x10.compiler.Native;
import x10.compiler.NativeCPPInclude;
import x10.compiler.NativeCPPCompilationUnit;

@NativeCPPInclude("MyCPPCode.h")
@NativeCPPCompilationUnit("MyCPPCode.cc")
public class Test {
    public static def main (args:Rail[String]) {
        { @Native("c++","foo();") {} }
    }
}
\end{xten}

\subsection{C++ System Libraries}

If we want to additionally link to more libraries in \xcd`/usr/lib` for
example, it is necessary to adjust the post-compilation directly.  The
post-compilation is the compilation of the C++ which the \Xten{}-to-C++ compiler
\xcd`x10c++` produces.  

The primary mechanism used for this is the \xcd`-cxx-prearg` and
\xcd`-cxx-postarg` command line arguments to
\xcd`x10c++`. The values of any \xcd`-cxx-prearg` commands are placed
in the post compiler command before the list of .cc files to compile.
The values of any \xcd`-cxx-postarg` commands are placed in the post
compiler command after the list of .cc files to compile. Typically
pre-args are arguments intended for the C++ compiler itself, while
post-args are arguments intended for the linker. 

The following example shows how to compile \xcd`blas` into the
executable via these commands. The command must be issued on one line.

\begin{xten}
x10c++ Test.x10 -cxx-prearg -I/usr/local/blas 
  -cxx-postarg -L/usr/local/blas -cxx-postarg -lblas'
\end{xten}


\chapter{Definite Assignment}
\label{sect:DefiniteAssignment}
\index{definite assignment}
\index{assignment!definite}
\index{definitely assigned}
\index{definitely not assigned}

X10 requires, reasonably enough, that every variable be set before it is read.
Sometimes this is easy, as when a variable is declared and assigned together: 
%~~gen ^^^ DefiniteAssignment4x1u
% package DefiniteAssignment4x1u;
% class Example {
% def example() {
%~~vis
\begin{xten}
  var x : Int = 0;
  assert x == 0;
\end{xten}
%~~siv
%}}
%~~neg
However, it is convenient to allow programs to make decisions before
initializing variables.
%~~gen ^^^ DefiniteAssignment4u7z
% package DefiniteAssignment4u7z;
% class Example {
%~~vis
\begin{xten}
def example(a:Int, b:Int) {
  val max:Int;
  //ERROR: assert max==max; // can't read 'max'
  if (a > b) max = a;
  else max = b;
  assert max >= a && max >= b;
}
\end{xten}
%~~siv
%}
%~~neg
This is particularly useful for \xcd`val` variables.  \xcd`var`s could be
initialized to a default value and then reassigned with the right value.
\xcd`val`s must be initialized once and cannot be changed, so they must be
initialized with the correct value. 

However, one must be careful -- and the X10 compiler enforces this care.
Without the \xcd`else` clause, the preceding code might not give \xcd`max` a
value by the \xcd`assert`.  

This leads to the concept of {\em definite assignment} \cite{Javasomething}.
A variable is definitely assigned at a point in code if, no matter how that
point in code is reached, the variable has been assigned to.  In X10,
variables must be definitely assigned before they can be read.


As X10 requires that \xcd`val` variables {\em not} be initialized
twice,  we need the dual concept as well.  A variable is {\em definitely
unassigned} at a point in code if it cannot have been assigned there.  For
example, immediately after \xcd`val x:Int`, \xcd`x` is definitely unassigned. 

Finally, we need the concept of {\em singly} and {\em multiply assigned}.
A variable is singly assigned in a block if it is assigned precisely
once; it is multiply assigned if it could possibly be assigned more than once.  
\xcd`var`s can  multiply assigned as desired. \xcd`val`s must be singly
assigned.  For example, the code \xcd`x = 1; x = 2;` is perfectly fine if
\xcd`x` is a \xcd`var`, but incorrect (even in a constructor) if \xcd`x` is a
\xcd`val`.  

At some points in code, a variable might be neither definitely assigned nor
definitely unassigned.    Such states are not always useful.  
%~~gen ^^^ DefiniteAssignment4f5z
% package DefiniteAssignment4f5z;
% class Example {
% 
%~~vis
\begin{xten}
def example(flag : Boolean) {
  var x : Int;
  if (flag) x = 1;
  // x is neither def. assigned nor unassigned.
  x = 2; 
  // x is def. assigned.
\end{xten}
%~~siv
% } } 
%~~neg
This shows that we cannot simply define ``definitely unassigned'' as ``not
definitely assigned''.   If \xcd`x` had been a \xcd`val` rather than a
\xcd`var`, the previous example would not be allowed.    

Unfortunately, a completely accurate definition of ``definitely assigned''
or ``definitely unassigned'' is undecidable -- impossible for the compiler to
determine.  So, X10 takes a {\em conservative approximation} of these
concepts.  If X10's definition says that \xcd`x` is definitely assigned (or
definitely unassigned), then it will be assigned (or not assigned) in every
execution of the program.  

However, there are programs which X10's algorithm says are incorrect, but
which actually would behave properly if they were executed.   In the following
example, \xcd`flag` is either \xcd`true` or \xcd`false`, and in either case
\xcd`x` will be initialized.  However, X10's analysis does not understand this
--- thought it {\em would} understand if the example were coded with an
\xcd`if-else` rather than a pair of \xcd`if`s.  So, after the two \xcd`if`
statements, \xcd`x` is not definitely assigned, and thus the \xcd`assert`
statement, which reads it, is forbidden.  
%~~gen ^^^ DefiniteAssignment3x6i
% package DefiniteAssignment3x6i;
% class Example{ 
%~~vis
\begin{xten}
def example(flag:Boolean) {
  var x : Int;
  if (flag) x = 1;
  if (!flag) x = 2;
  // ERROR: assert x < 3;
}
\end{xten}
%~~siv
%}
%~~neg

\section{Asynchronous Definite Assignment}


Local variables (but not fields) allow {\em asynchronous assignment}. A local
variable can be assigned in an \xcd`async` statement, and, when the
\xcd`async` is \xcd`finish`ed, the variable is definitely assigned.  

\begin{ex}
%~~gen ^^^ DefiniteAssignment4a6n
% package DefiniteAssignment4a6n;
% class Example {
% def example() {
%~~vis
\begin{xten}
val a : Int;
finish {
  async {
    a = 1;
  } 
  // a is not definitely assigned here
}
// a is definitely assigned after 'finish'
assert a==1; 
\end{xten}
%~~siv
%} } 
%~~neg
\end{ex}

This concept supports a core X10 programming idiom.  A \xcd`val` variable may
be initialized asynchronously, thereby providing a means for returning a value
from an \xcd`async` to be used after the enclosing \xcd`finish`.  

\section{Characteristics of Definite Assignment}

The properties ``definitely assigned'', ``singly assigned'', and
``definitely unassigned'' are computed by a conservative approximation of
X10's evaluation rules.

The precise details are up to the implementation. 
Many basic cases must be handled accurately; \eg, \xcd`x=1;` definitely and
singly assigns \xcd`x`.  

However, in more complicated cases, a conforming X10 may mark as invalid 
some code which, when executed, would actually be correct.  
For example, the following
program fragment will always result in \xcd`x` being definitely and singly
assigned:  
\begin{xten}
val x : Int;
var b : Boolean = mysterious();
if (b) {
   x = cryptic();
}
if (!b) { 
   x = unknown();
}
\end{xten}
However, most conservative approximations of program execution won't mark
\xcd`x` as properly initialized. For this to be correct, precisely one of the
two assignments to \xcd`x` must be executed. If \xcd`b` were true initially,
it would still be true after the call to \xcd`cryptic()` --- since methods
cannot modify their caller's local variables -- and so the first but not the
second assignment would happen. If \xcd`b` were false initially, it would
still be false when \xcd`!b` is tested, and so the second but not the first
assignment would happen.  Either way, \xcd`x` is definitely and singly assigned.

However, for a slightly different program, this analysis would be wrong. \Eg,
if  \xcd`b` were a field of \xcd`this` rather than a local variable,
\xcd`cryptic()` could change \xcd`b`; if \xcd`b` were true initially, both
assignments might happen, which is incorrect for a \xcd`val`.  

This sort of reasoning is beyond  most conservative approximation algorithms.
(Indeed, many do not bother checking that \xcd`!b` late in the program is the
opposite of \xcd`b` earlier.)
Algorithms that pay attention to such details and subtleties tend to be
fairly expensive, which would lead to very slow compilation for X10 -- for the
sake of obscure cases.

X10's analysis provides at least the following guarantees. We describe them in
terms of a statement \xcd`S` performing some collection of possible numbers of
assignments to variables --- on a scale of ``0'', ``1'', and ``many''. For
example, \xcd`if(b) x=1; else {x=1;x=2;y=2;}` might assign to \xcd`x` one or
many times, and might assign to \xcd`y` zero or one time. Hence, after it,
\xcd`x` is definitely assigned and may be multiply assigned, and \xcd`y` is
neither definitely assigned nor definitely unassigned.  

These descriptions are combined in natural ways.  For example, if \xcd`R` says
that \xcd`x` will be assigned 0 or 1 times, and \xcd`S` says it will be
assigned precisely once, then \xcd`R;S` will assign it one or many times.  If
only one or \xcd`R` or \xcd`S` will occur, as from \xcd`if(b)R; else S;`, 
then \xcd`x` may be assigned 0 or 1 times. 

This information is sufficient for the tests X10 makes.  If \xcd`x` can is
assigned one or many times in \xcd`S`, it is definitely assigned.  It is an
error if 
\xcd`x` is ever read at a point where it have been assigned zero times.  It is
an error if a \xcd`val` may be assigned many times.


We do not guarantee that any particular X10 compiler uses this algorithm;
indeed, as of the time of writing, the X10 compiler uses a somewhat more
precise one.  However, any conformant X10 compiler must provide results which
are at least as accurate as this analysis.




\subsubsection{Assignment: {\tt x = e}}   

\xcd`x = e` assigns to \xcd`x`, in addition to whatever assignments
\xcd`e` makes.   For example, if \xcd`this.setX(y)` sets a field \xcd`x` to
\xcd`y` and returns \xcd`y`, then \xcd`x = this.setX(y)` definitely and
multiply assigns \xcd`x`.  

\subsubsection{{\tt async} and {\tt finish}}

By itself, \xcd`async S` provides few guarantees.  After \xcd`async{x=1;}`
finishes, we know that there is a separate activity which will, when the
scheduler lets it, set \xcd`x` to \xcd`1`.  We do not know that anything has
happened yet.

However, if there is a \xcd`finish` around the \xcd`async`, the situation is
clearer.  After \xcd`finish{ async{ x=1; }}`, \xcd`x` has definitely been
assigned.  

In general, if an \xcd`async S` appears in the body of a \xcd`finish` in a way
that guarantees that it will be executed, then, after the \xcd`finish`, the
assignments made by \xcd`S` will have occurred.  For example, if \xcd`S`
definitely assigns to \xcd`x`, and the body of the \xcd`finish` guarantees
that \xcd`async S` will be executed, then \xcd`finish{...async S...}`
definitely assigns \xcd`x`.



\subsubsection{{\tt if} and {\tt switch}}

When \xcd`if(E) S else T` finishes, it will have performed the assignments of
\xcd`E`, together with those of either \xcd`S` or \xcd`T` but not both.  For
example, \xcd`if(b) x=1; else x=2;` definitely assigns \xcd`x`,
but \xcd`if(b) x=1;` does not.

{\tt switch} is more complex, but follows the same principles as \xcd`if`.
For example, \xcd`switch(E){case 1: A; break; case 2: B; default: C;}`  
performs the assignments of \xcd`E`, and those of precisely one of \xcd`A`, or
\xcd`B;C`, or \xcd`C`.  Note that case \xcd`2` falls through to the default
case, so it performs the same assignments as \xcd`B;C`.

\subsubsection{Sequencing}

When \xcd`R;S` finishes, it will have performed the assignments of \xcd`R` and
those of \xcd`S`.  For example, \xcd`x=1;y=2;` definitely assigns \xcd`x` and
\xcd`y`, and \xcd`x=1;x=2;` multiply assigns \xcd`x`. 


\subsubsection{Loops}

\xcd`while(E)S` performs the assignments of \xcd`E` one or more times, and
those of \xcd`S` zero or more times.  For example, if \xcd`while(b()){x=1;}`
might assign to \xcd`x` zero, one, or many times.  
\xcd`do S while(E)` performs the assignments of \xcd`E` one or more times, and
those of \xcd`S` one or more times. 

\xcd`for(A;B;C)D` performs the assignments of \xcd`A` once, those of \xcd`B`
one or more times, and those of \xcd`C` and \xcd`D` one or more times.
\xcd`for(x in E)S` performs the assignments of \xcd`E` once and those of
\xcd`S` zero or more times.  

Loops are of very little value for providing definite assignments, since X10
does not in general know how many times they will be executed. 

\xcd`continue` and \xcd`break` inside of a loop are hard to describe in simple
terms.  They may be conservatively assumed to cause the loop give no
information about the variables assigned inside of it.
For example, the analysis may conservatively conclude that 
\xcd`do{ x = 1; if (true) break; } while(true)` may 
assign to \xcd`x` zero, one, or many times, overlooking the more precise fact
that it is assigned once.  




\subsubsection{Method Calls}

A method call \xcd`E.m(A,B)` performs the assignments of \xcd`E`, \xcd`A`, and
\xcd`B` once each, and also those of \xcd`m`.  This implies that X10 must be
aware of the possible assignments performed by each method.


If X10 has complete information about \xcd`m` (as when \xcd`m` is a
\xcd`private` or \xcd`final` method), this is straightforward.  When such
information is fundamentally impossible to acquire, as when \xcd`m` is a
non-final method invocation, X10 has no choice but to assume that \xcd`m`
might do anything that a method can do.    (For this reason, the only methods
that can be called from within a constructor on a raw --
incompletely-constructed -- object) are the \xcd`private` and
\xcd`final` ones.)  
\begin{itemize}
\item \xcd`m` cannot assign to local fields of the caller; methods have no
      such power.
\item \xcd`m` can assign to \xcd`var` fields of \xcd`this` freely, unless this
      is prohibited by an annotation; see \Sref{somewhere}
\item \xcd`m` cannot initialize \xcd`val` fields of \xcd`this`.  (With one
      exception; see \Sref{sect:call-another-constructor}.) 
\end{itemize}

Recall that every container must be fully initialized (``cooked'') upon exit
from its constructor.  
X10 places certain restrictions on which methods can be called from a
constructor; see \Sref{sect:nonescaping}.  One of these restrictions is that
methods called before object initialization is complete must be \xcd`final` or
\xcd`private` --- and hence, available for static analysis.  So, when checking
field initialization, X10 will ensure: 
\begin{enumerate}
\item Each \xcd`val` field is initialized before it is read.   
      A method that does not read a \xcd`val` field \xcd`f` {\em may} be
      called before \xcd`f` is initialized; a method that reads \xcd`f` must
      not be called until \xcd`f` is initialized.        
      For example, 
      a constructor may have the form:
%~~gen ^^^ DefiniteAssignment4x6k
% package DefiniteAssignment4x6k;
%~~vis
\begin{xten}
class C {
  val f : Int;
  val g : String;
  def this() {
     f = fless();
     g = useF();
  }
  private def fless() = "f not used here".length();
  private def useF() = "f=" + this.f;
}
\end{xten}
%~~siv
%
%~~neg

\item \xcd`var` fields require a deeper analysis.  Consider a \xcd`var`
      field \xcd`var x:T`  without initializer.  If \xcd`T` has a default
      value, \xcd`x` may be read inside of a constructor before it is
      otherwise written, and it will 
      have its default value.   

      If \xcd`T` has no default value, an analysis
      like that used for \xcd`val`s must be performed to determine that
      \xcd`x` is initialized before it is used.  The situation is 
      more complex than for \xcd`val`s, however, because a method can assign to
      \xcd`x` as well read from it.  The X10 compiler computes a conservative
      approximation of which methods
      read and write which \xcd`var` fields. (Doing this carefully 
      requires finding a solution of a set of equations over sets of
      variables, with each callable method having equations describing what it
      reads and writes.)    

\end{enumerate}


\subsubsection{{\tt at} and \xcd`athome`}

%%AT-COPY%% \xcd`at(E)S` performs the assignments of \xcd`E`. Within \xcd`S`, only those
%%AT-COPY%% assignments to variables \xcd`x` from the surrounding environment which take
%%AT-COPY%% place within a suitable \xcd`athome(x)R` are counted. 
%%AT-COPY%% 
%%AT-COPY%% \begin{ex}
%%AT-COPY%% In the following code, the outer variable named \xcd`a` is definitely assigned
%%AT-COPY%% once, by the assignment \xcd`a = 3;`.  The inner variable (also named \xcd`a`)
%%AT-COPY%% is definitely multiply assigned 
%%AT-COPY%% by the two assignments \xcd`a = 1;` and \xcd`a = 2;` 
%%AT-COPY%% between the \xcd`at` and the \xcd`athome`.  
%%AT-COPY%% 
%%AT-COPY%% %~~gen ^^^ DefiniteAssignment3n5q
%%AT-COPY%% % package DefiniteAssignment3n5q;
%%AT-COPY%% % KNOWNFAIL-at
%%AT-COPY%% % class DefAss { def defass() { 
%%AT-COPY%% %~~vis
%%AT-COPY%% \begin{xten}
%%AT-COPY%% var a : Int;
%%AT-COPY%% at(here.next(); var a : Int = a) {
%%AT-COPY%%   a = 1;
%%AT-COPY%%   a = 2; 
%%AT-COPY%%   athome(a) a = 3;
%%AT-COPY%% }
%%AT-COPY%% \end{xten}
%%AT-COPY%% %~~siv
%%AT-COPY%% % } } 
%%AT-COPY%% %~~neg
%%AT-COPY%% 
%%AT-COPY%% 
%%AT-COPY%% \end{ex}
%%AT-COPY%% 

\xcd`at(p)S` cannot perform any assignments.

\subsubsection{{\tt atomic}}

\xcd`atomic S` performs the assignments of \xcd`S`, 
and \xcd`when(E)S` performs those of \xcd`E` and \xcd`S`.  

\subsubsection{{\tt try}}

\xcd`try S catch(x:T1) E1 ... catch(x:Tn) En  finally F` 
performs some or all of the assignments of \xcd`S`, plus all the assignments
of zero or one of the \xcd`E`'s, plus those of \xcd`F`.  
For example,
\begin{xten}
try {
  x = boomy();
  x = 0;
}
catch(e:Boom) { y = 1; }
finally { z = 1; }
\end{xten}
\noindent 
assigns \xcd`x` zero, one, or many times\footnote{A more precise
analysis could discover that \xcd`x` cannot be initialized only once.}, 
assigns \xcd`y` zero or one time, and assigns \xcd`z` exactly once.

\subsubsection{Expression Statements}

Expression statements \xcd`E;`, and other statements that execute an
expression and do something innocuous with it (local variable declaration and
\xcd`assert`) have the same effects as \xcd`E`. 

\subsubsection{{\tt return}, {\tt throw}}

Statements that do not finish normally, such as \xcd`return` and \xcd`throw`,
don't initialize anything (though the computation of the return or thrown
value may).    They also terminate a line of computation.  For example, 
\xcd`if(b) {x=1; return;}  x=2;` definitely and singly assigns \xcd`x`.  

\chapter{Grammar}


\begin{bbgrammar}

 MethodInvocation  \refstepcounter{equation}\label{prod:MethodInvocation}  \: MethodPrimaryPrefix \xcd"(" ArgumentList\opt \xcd")" & (\arabic{equation})\\
    \| MethodSuperPrefix \xcd"(" ArgumentList\opt \xcd")"\\
    \| MethodClassNameSuperPrefix \xcd"(" ArgumentList\opt \xcd")"\\
 Mod  \refstepcounter{equation}\label{prod:Mod}  \: \xcd"abstract" & (\arabic{equation})\\
    \| Annotation\\
    \| atomic\\
    \| \xcd"final"\\
    \| \xcd"native"\\
    \| \xcd"private"\\
    \| \xcd"protected"\\
    \| \xcd"public"\\
    \| \xcd"static"\\
    \| \xcd"transient"\\
    \| clocked\\
 TypeDefDecl  \refstepcounter{equation}\label{prod:TypeDefDecl}  \: Mods\opt type Id TypeParams\opt FormalParams\opt WhereClause\opt \xcd"=" Type \xcd";" & (\arabic{equation})\\
 Properties  \refstepcounter{equation}\label{prod:Properties}  \: \xcd"(" PropertyList \xcd")" & (\arabic{equation})\\
\end{bbgrammar}

\begin{bbgrammar}

 PropertyList  \refstepcounter{equation}\label{prod:PropertyList}  \: Property & (\arabic{equation})\\
    \| PropertyList \xcd"," Property\\
 Property  \refstepcounter{equation}\label{prod:Property}  \: Annotations\opt Id ResultType & (\arabic{equation})\\
 MethodDecl  \refstepcounter{equation}\label{prod:MethodDecl}  \: MethodMods\opt \xcd"def" Id TypeParams\opt FormalParams WhereClause\opt HasResultType\opt Offers\opt MethodBody & (\arabic{equation})\\
    \| MethodMods\opt \xcd"operator" TypeParams\opt \xcd"(" FormalParam  \xcd")" BinOp \xcd"(" FormalParam  \xcd")" WhereClause\opt HasResultType\opt Offers\opt MethodBody\\
    \| MethodMods\opt \xcd"operator" TypeParams\opt PrefixOp \xcd"(" FormalParam  \xcd")" WhereClause\opt HasResultType\opt Offers\opt MethodBody\\
    \| MethodMods\opt \xcd"operator" TypeParams\opt \xcd"this" BinOp \xcd"(" FormalParam  \xcd")" WhereClause\opt HasResultType\opt Offers\opt MethodBody\\
    \| MethodMods\opt \xcd"operator" TypeParams\opt \xcd"(" FormalParam  \xcd")" BinOp \xcd"this" WhereClause\opt HasResultType\opt Offers\opt MethodBody\\
    \| MethodMods\opt \xcd"operator" TypeParams\opt PrefixOp \xcd"this" WhereClause\opt HasResultType\opt Offers\opt MethodBody\\
    \| MethodMods\opt \xcd"operator" \xcd"this" TypeParams\opt FormalParams WhereClause\opt HasResultType\opt Offers\opt MethodBody\\
    \| MethodMods\opt \xcd"operator" \xcd"this" TypeParams\opt FormalParams \xcd"=" \xcd"(" FormalParam  \xcd")" WhereClause\opt HasResultType\opt Offers\opt MethodBody\\
    \| MethodMods\opt \xcd"operator" TypeParams\opt \xcd"(" FormalParam  \xcd")" \xcd"as" Type WhereClause\opt Offers\opt MethodBody\\
    \| MethodMods\opt \xcd"operator" TypeParams\opt \xcd"(" FormalParam  \xcd")" \xcd"as" \xcd"?" WhereClause\opt HasResultType\opt Offers\opt MethodBody\\
    \| MethodMods\opt \xcd"operator" TypeParams\opt \xcd"(" FormalParam  \xcd")" WhereClause\opt HasResultType\opt Offers\opt MethodBody\\
 PropertyMethodDecl  \refstepcounter{equation}\label{prod:PropertyMethodDecl}  \: MethodMods\opt Id TypeParams\opt FormalParams WhereClause\opt HasResultType\opt MethodBody & (\arabic{equation})\\
    \| MethodMods\opt Id WhereClause\opt HasResultType\opt MethodBody\\
 ExplicitCtorInvocation  \refstepcounter{equation}\label{prod:ExplicitCtorInvocation}  \: \xcd"this" TypeArguments\opt \xcd"(" ArgumentList\opt \xcd")" \xcd";" & (\arabic{equation})\\
    \| \xcd"super" TypeArguments\opt \xcd"(" ArgumentList\opt \xcd")" \xcd";"\\
    \| Primary \xcd"." \xcd"this" TypeArguments\opt \xcd"(" ArgumentList\opt \xcd")" \xcd";"\\
    \| Primary \xcd"." \xcd"super" TypeArguments\opt \xcd"(" ArgumentList\opt \xcd")" \xcd";"\\
 NormalInterfaceDecl  \refstepcounter{equation}\label{prod:NormalInterfaceDecl}  \: Mods\opt \xcd"interface" Id TypeParamsWithVariance\opt Properties\opt WhereClause\opt ExtendsInterfaces\opt InterfaceBody & (\arabic{equation})\\
 ClassInstCreationExp  \refstepcounter{equation}\label{prod:ClassInstCreationExp}  \: \xcd"new" TypeName TypeArguments\opt \xcd"(" ArgumentList\opt \xcd")" ClassBody\opt & (\arabic{equation})\\
    \| \xcd"new" TypeName \xcd"[" Type \xcd"]" \xcd"[" ArgumentList\opt \xcd"]"\\
    \| Primary \xcd"." \xcd"new" Id TypeArguments\opt \xcd"(" ArgumentList\opt \xcd")" ClassBody\opt\\
    \| AmbiguousName \xcd"." \xcd"new" Id TypeArguments\opt \xcd"(" ArgumentList\opt \xcd")" ClassBody\opt\\
 AssignPropertyCall  \refstepcounter{equation}\label{prod:AssignPropertyCall}  \: \xcd"property" TypeArguments\opt \xcd"(" ArgumentList\opt \xcd")" \xcd";" & (\arabic{equation})\\
 Type  \refstepcounter{equation}\label{prod:Type}  \: FunctionType & (\arabic{equation})\\
    \| ConstrainedType\\
 FunctionType  \refstepcounter{equation}\label{prod:FunctionType}  \: TypeParams\opt \xcd"(" FormalParamList\opt \xcd")" WhereClause\opt Offers\opt \xcd"=>" Type & (\arabic{equation})\\
 ClassType  \refstepcounter{equation}\label{prod:ClassType}  \: NamedType & (\arabic{equation})\\
 AnnotatedType  \refstepcounter{equation}\label{prod:AnnotatedType}  \: Type Annotations & (\arabic{equation})\\
 ConstrainedType  \refstepcounter{equation}\label{prod:ConstrainedType}  \: NamedType & (\arabic{equation})\\
    \| AnnotatedType\\
    \| \xcd"(" Type \xcd")"\\
 PlaceType  \refstepcounter{equation}\label{prod:PlaceType}  \: PlaceExp & (\arabic{equation})\\
 SimpleNamedType  \refstepcounter{equation}\label{prod:SimpleNamedType}  \: TypeName & (\arabic{equation})\\
    \| Primary \xcd"." Id\\
    \| DepNamedType \xcd"." Id\\
 DepNamedType  \refstepcounter{equation}\label{prod:DepNamedType}  \: SimpleNamedType DepParams & (\arabic{equation})\\
    \| SimpleNamedType Arguments\\
    \| SimpleNamedType Arguments DepParams\\
\end{bbgrammar}

\begin{bbgrammar}

    \| SimpleNamedType TypeArguments\\
    \| SimpleNamedType TypeArguments DepParams\\
    \| SimpleNamedType TypeArguments Arguments\\
    \| SimpleNamedType TypeArguments Arguments DepParams\\
 NamedType  \refstepcounter{equation}\label{prod:NamedType}  \: SimpleNamedType & (\arabic{equation})\\
    \| DepNamedType\\
 DepParams  \refstepcounter{equation}\label{prod:DepParams}  \: \xcd"{" ExistentialList\opt Conjunction\opt \xcd"}" & (\arabic{equation})\\
 TypeParamsWithVariance  \refstepcounter{equation}\label{prod:TypeParamsWithVariance}  \: \xcd"[" TypeParamWithVarianceList \xcd"]" & (\arabic{equation})\\
 TypeParams  \refstepcounter{equation}\label{prod:TypeParams}  \: \xcd"[" TypeParamList \xcd"]" & (\arabic{equation})\\
 FormalParams  \refstepcounter{equation}\label{prod:FormalParams}  \: \xcd"(" FormalParamList\opt \xcd")" & (\arabic{equation})\\
 Conjunction  \refstepcounter{equation}\label{prod:Conjunction}  \: Exp & (\arabic{equation})\\
    \| Conjunction \xcd"," Exp\\
 SubtypeConstraint  \refstepcounter{equation}\label{prod:SubtypeConstraint}  \: Type  \xcd"<:" Type  & (\arabic{equation})\\
    \| Type  \xcd":>" Type \\
 WhereClause  \refstepcounter{equation}\label{prod:WhereClause}  \: DepParams & (\arabic{equation})\\
 ExistentialList  \refstepcounter{equation}\label{prod:ExistentialList}  \: FormalParam & (\arabic{equation})\\
    \| ExistentialList \xcd";" FormalParam\\
 ClassDecl  \refstepcounter{equation}\label{prod:ClassDecl}  \: StructDecl & (\arabic{equation})\\
    \| NormalClassDecl\\
 NormalClassDecl  \refstepcounter{equation}\label{prod:NormalClassDecl}  \: Mods\opt \xcd"class" Id TypeParamsWithVariance\opt Properties\opt WhereClause\opt Super\opt Interfaces\opt ClassBody & (\arabic{equation})\\
 StructDecl  \refstepcounter{equation}\label{prod:StructDecl}  \: Mods\opt \xcd"struct" Id TypeParamsWithVariance\opt Properties\opt WhereClause\opt Interfaces\opt ClassBody & (\arabic{equation})\\
 CtorDecl  \refstepcounter{equation}\label{prod:CtorDecl}  \: Mods\opt \xcd"def" \xcd"this" TypeParams\opt FormalParams WhereClause\opt HasResultType\opt Offers\opt CtorBody & (\arabic{equation})\\
 Super  \refstepcounter{equation}\label{prod:Super}  \: \xcd"extends" ClassType & (\arabic{equation})\\
 FieldKeyword  \refstepcounter{equation}\label{prod:FieldKeyword}  \: val & (\arabic{equation})\\
    \| \xcd"var"\\
 VarKeyword  \refstepcounter{equation}\label{prod:VarKeyword}  \: val & (\arabic{equation})\\
    \| \xcd"var"\\
 FieldDecl  \refstepcounter{equation}\label{prod:FieldDecl}  \: Mods\opt FieldKeyword FieldDeclarators \xcd";" & (\arabic{equation})\\
    \| Mods\opt FieldDeclarators \xcd";"\\
 Statement  \refstepcounter{equation}\label{prod:Statement}  \: AnnotationStatement & (\arabic{equation})\\
    \| ExpStatement\\
 AnnotationStatement  \refstepcounter{equation}\label{prod:AnnotationStatement}  \: Annotations\opt NonExpStatement & (\arabic{equation})\\
 NonExpStatement  \refstepcounter{equation}\label{prod:NonExpStatement}  \: Block & (\arabic{equation})\\
    \| EmptyStatement\\
    \| AssertStatement\\
    \| SwitchStatement\\
    \| DoStatement\\
    \| BreakStatement\\
    \| ContinueStatement\\
    \| ReturnStatement\\
    \| ThrowStatement\\
\end{bbgrammar}

\begin{bbgrammar}

    \| TryStatement\\
    \| LabeledStatement\\
    \| IfThenStatement\\
    \| IfThenElseStatement\\
    \| WhileStatement\\
    \| ForStatement\\
    \| AsyncStatement\\
    \| AtStatement\\
    \| AtomicStatement\\
    \| WhenStatement\\
    \| AtEachStatement\\
    \| FinishStatement\\
    \| NextStatement\\
    \| ResumeStatement\\
    \| AssignPropertyCall\\
    \| OfferStatement\\
 OfferStatement  \refstepcounter{equation}\label{prod:OfferStatement}  \: offer Exp \xcd";" & (\arabic{equation})\\
 IfThenStatement  \refstepcounter{equation}\label{prod:IfThenStatement}  \: \xcd"if" \xcd"(" Exp \xcd")" Statement & (\arabic{equation})\\
 IfThenElseStatement  \refstepcounter{equation}\label{prod:IfThenElseStatement}  \: \xcd"if" \xcd"(" Exp \xcd")" Statement  \xcd"else" Statement  & (\arabic{equation})\\
 EmptyStatement  \refstepcounter{equation}\label{prod:EmptyStatement}  \: \xcd";" & (\arabic{equation})\\
 LabeledStatement  \refstepcounter{equation}\label{prod:LabeledStatement}  \: Id \xcd":" LoopStatement & (\arabic{equation})\\
 LoopStatement  \refstepcounter{equation}\label{prod:LoopStatement}  \: ForStatement & (\arabic{equation})\\
    \| WhileStatement\\
    \| DoStatement\\
    \| AtEachStatement\\
 ExpStatement  \refstepcounter{equation}\label{prod:ExpStatement}  \: StatementExp \xcd";" & (\arabic{equation})\\
 StatementExp  \refstepcounter{equation}\label{prod:StatementExp}  \: Assignment & (\arabic{equation})\\
    \| PreIncrementExp\\
    \| PreDecrementExp\\
    \| PostIncrementExp\\
    \| PostDecrementExp\\
    \| MethodInvocation\\
    \| ClassInstCreationExp\\
 AssertStatement  \refstepcounter{equation}\label{prod:AssertStatement}  \: \xcd"assert" Exp \xcd";" & (\arabic{equation})\\
    \| \xcd"assert" Exp  \xcd":" Exp  \xcd";"\\
 SwitchStatement  \refstepcounter{equation}\label{prod:SwitchStatement}  \: \xcd"switch" \xcd"(" Exp \xcd")" SwitchBlock & (\arabic{equation})\\
 SwitchBlock  \refstepcounter{equation}\label{prod:SwitchBlock}  \: \xcd"{" SwitchBlockStatementGroups\opt SwitchLabels\opt \xcd"}" & (\arabic{equation})\\
 SwitchBlockStatementGroups  \refstepcounter{equation}\label{prod:SwitchBlockStatementGroups}  \: SwitchBlockStatementGroup & (\arabic{equation})\\
    \| SwitchBlockStatementGroups SwitchBlockStatementGroup\\
 SwitchBlockStatementGroup  \refstepcounter{equation}\label{prod:SwitchBlockStatementGroup}  \: SwitchLabels BlockStatements & (\arabic{equation})\\
 SwitchLabels  \refstepcounter{equation}\label{prod:SwitchLabels}  \: SwitchLabel & (\arabic{equation})\\
\end{bbgrammar}

\begin{bbgrammar}

    \| SwitchLabels SwitchLabel\\
 SwitchLabel  \refstepcounter{equation}\label{prod:SwitchLabel}  \: \xcd"case" ConstantExp \xcd":" & (\arabic{equation})\\
    \| \xcd"default" \xcd":"\\
 WhileStatement  \refstepcounter{equation}\label{prod:WhileStatement}  \: \xcd"while" \xcd"(" Exp \xcd")" Statement & (\arabic{equation})\\
 DoStatement  \refstepcounter{equation}\label{prod:DoStatement}  \: \xcd"do" Statement \xcd"while" \xcd"(" Exp \xcd")" \xcd";" & (\arabic{equation})\\
 ForStatement  \refstepcounter{equation}\label{prod:ForStatement}  \: BasicForStatement & (\arabic{equation})\\
    \| EnhancedForStatement\\
 BasicForStatement  \refstepcounter{equation}\label{prod:BasicForStatement}  \: \xcd"for" \xcd"(" ForInit\opt \xcd";" Exp\opt \xcd";" ForUpdate\opt \xcd")" Statement & (\arabic{equation})\\
 ForInit  \refstepcounter{equation}\label{prod:ForInit}  \: StatementExpList & (\arabic{equation})\\
    \| LocalVariableDecl\\
 ForUpdate  \refstepcounter{equation}\label{prod:ForUpdate}  \: StatementExpList & (\arabic{equation})\\
 StatementExpList  \refstepcounter{equation}\label{prod:StatementExpList}  \: StatementExp & (\arabic{equation})\\
    \| StatementExpList \xcd"," StatementExp\\
 BreakStatement  \refstepcounter{equation}\label{prod:BreakStatement}  \: \xcd"break" Id\opt \xcd";" & (\arabic{equation})\\
 ContinueStatement  \refstepcounter{equation}\label{prod:ContinueStatement}  \: \xcd"continue" Id\opt \xcd";" & (\arabic{equation})\\
 ReturnStatement  \refstepcounter{equation}\label{prod:ReturnStatement}  \: \xcd"return" Exp\opt \xcd";" & (\arabic{equation})\\
 ThrowStatement  \refstepcounter{equation}\label{prod:ThrowStatement}  \: \xcd"throw" Exp \xcd";" & (\arabic{equation})\\
 TryStatement  \refstepcounter{equation}\label{prod:TryStatement}  \: \xcd"try" Block Catches & (\arabic{equation})\\
    \| \xcd"try" Block Catches\opt Finally\\
 Catches  \refstepcounter{equation}\label{prod:Catches}  \: CatchClause & (\arabic{equation})\\
    \| Catches CatchClause\\
 CatchClause  \refstepcounter{equation}\label{prod:CatchClause}  \: \xcd"catch" \xcd"(" FormalParam \xcd")" Block & (\arabic{equation})\\
 Finally  \refstepcounter{equation}\label{prod:Finally}  \: \xcd"finally" Block & (\arabic{equation})\\
 ClockedClause  \refstepcounter{equation}\label{prod:ClockedClause}  \: clocked \xcd"(" ClockList \xcd")" & (\arabic{equation})\\
 AsyncStatement  \refstepcounter{equation}\label{prod:AsyncStatement}  \: \xcd"async" ClockedClause\opt Statement & (\arabic{equation})\\
    \| clocked \xcd"async" Statement\\
 AtStatement  \refstepcounter{equation}\label{prod:AtStatement}  \: at PlaceExpSingleList Statement & (\arabic{equation})\\
 AtomicStatement  \refstepcounter{equation}\label{prod:AtomicStatement}  \: atomic Statement & (\arabic{equation})\\
 WhenStatement  \refstepcounter{equation}\label{prod:WhenStatement}  \: \xcd"when" \xcd"(" Exp \xcd")" Statement & (\arabic{equation})\\
 AtEachStatement  \refstepcounter{equation}\label{prod:AtEachStatement}  \: \xcd"ateach" \xcd"(" LoopIndex \xcd"in" Exp \xcd")" ClockedClause\opt Statement & (\arabic{equation})\\
    \| \xcd"ateach" \xcd"(" Exp \xcd")" Statement\\
 EnhancedForStatement  \refstepcounter{equation}\label{prod:EnhancedForStatement}  \: \xcd"for" \xcd"(" LoopIndex \xcd"in" Exp \xcd")" Statement & (\arabic{equation})\\
    \| \xcd"for" \xcd"(" Exp \xcd")" Statement\\
 FinishStatement  \refstepcounter{equation}\label{prod:FinishStatement}  \: \xcd"finish" Statement & (\arabic{equation})\\
    \| clocked \xcd"finish" Statement\\
 PlaceExpSingleList  \refstepcounter{equation}\label{prod:PlaceExpSingleList}  \: \xcd"(" PlaceExp \xcd")" & (\arabic{equation})\\
 PlaceExp  \refstepcounter{equation}\label{prod:PlaceExp}  \: Exp & (\arabic{equation})\\
 NextStatement  \refstepcounter{equation}\label{prod:NextStatement}  \: next \xcd";" & (\arabic{equation})\\
 ResumeStatement  \refstepcounter{equation}\label{prod:ResumeStatement}  \: resume \xcd";" & (\arabic{equation})\\
 ClockList  \refstepcounter{equation}\label{prod:ClockList}  \: Clock & (\arabic{equation})\\
    \| ClockList \xcd"," Clock\\
\end{bbgrammar}

\begin{bbgrammar}

 Clock  \refstepcounter{equation}\label{prod:Clock}  \: Exp & (\arabic{equation})\\
 CastExp  \refstepcounter{equation}\label{prod:CastExp}  \: Primary & (\arabic{equation})\\
    \| ExpName\\
    \| CastExp \xcd"as" Type\\
 TypeParamWithVarianceList  \refstepcounter{equation}\label{prod:TypeParamWithVarianceList}  \: TypeParamWithVariance & (\arabic{equation})\\
    \| TypeParamWithVarianceList \xcd"," TypeParamWithVariance\\
 TypeParamList  \refstepcounter{equation}\label{prod:TypeParamList}  \: TypeParam & (\arabic{equation})\\
    \| TypeParamList \xcd"," TypeParam\\
 TypeParamWithVariance  \refstepcounter{equation}\label{prod:TypeParamWithVariance}  \: Id & (\arabic{equation})\\
    \| \xcd"+" Id\\
    \| \xcd"-" Id\\
 TypeParam  \refstepcounter{equation}\label{prod:TypeParam}  \: Id & (\arabic{equation})\\
 AssignmentExp  \refstepcounter{equation}\label{prod:AssignmentExp}  \: Exp  \xcd"->" Exp  & (\arabic{equation})\\
 ClosureExp  \refstepcounter{equation}\label{prod:ClosureExp}  \: FormalParams WhereClause\opt HasResultType\opt Offers\opt \xcd"=>" ClosureBody & (\arabic{equation})\\
 LastExp  \refstepcounter{equation}\label{prod:LastExp}  \: Exp & (\arabic{equation})\\
 ClosureBody  \refstepcounter{equation}\label{prod:ClosureBody}  \: ConditionalExp & (\arabic{equation})\\
    \| Annotations\opt \xcd"{" BlockStatements\opt LastExp \xcd"}"\\
    \| Annotations\opt Block\\
 AtExp  \refstepcounter{equation}\label{prod:AtExp}  \: at PlaceExpSingleList ClosureBody & (\arabic{equation})\\
 FinishExp  \refstepcounter{equation}\label{prod:FinishExp}  \: \xcd"finish" \xcd"(" Exp \xcd")" Block & (\arabic{equation})\\
 identifier  \refstepcounter{equation}\label{prod:identifier}  \: \xcd"IDENTIFIER"  & (\arabic{equation})\\
 TypeName  \refstepcounter{equation}\label{prod:TypeName}  \: Id & (\arabic{equation})\\
    \| TypeName \xcd"." Id\\
 ClassName  \refstepcounter{equation}\label{prod:ClassName}  \: TypeName & (\arabic{equation})\\
 TypeArguments  \refstepcounter{equation}\label{prod:TypeArguments}  \: \xcd"[" TypeArgumentList \xcd"]" & (\arabic{equation})\\
 TypeArgumentList  \refstepcounter{equation}\label{prod:TypeArgumentList}  \: Type & (\arabic{equation})\\
    \| TypeArgumentList \xcd"," Type\\
 PackageName  \refstepcounter{equation}\label{prod:PackageName}  \: Id & (\arabic{equation})\\
    \| PackageName \xcd"." Id\\
 ExpName  \refstepcounter{equation}\label{prod:ExpName}  \: Id & (\arabic{equation})\\
    \| AmbiguousName \xcd"." Id\\
 MethodName  \refstepcounter{equation}\label{prod:MethodName}  \: Id & (\arabic{equation})\\
    \| AmbiguousName \xcd"." Id\\
 PackageOrTypeName  \refstepcounter{equation}\label{prod:PackageOrTypeName}  \: Id & (\arabic{equation})\\
    \| PackageOrTypeName \xcd"." Id\\
 AmbiguousName  \refstepcounter{equation}\label{prod:AmbiguousName}  \: Id & (\arabic{equation})\\
    \| AmbiguousName \xcd"." Id\\
 CompilationUnit  \refstepcounter{equation}\label{prod:CompilationUnit}  \: PackageDecl\opt TypeDecls\opt & (\arabic{equation})\\
    \| PackageDecl\opt ImportDecls TypeDecls\opt\\
    \| ImportDecls PackageDecl  ImportDecls\opt  TypeDecls\opt\\
    \| PackageDecl ImportDecls PackageDecl  ImportDecls\opt  TypeDecls\opt\\
\end{bbgrammar}

\begin{bbgrammar}

 ImportDecls  \refstepcounter{equation}\label{prod:ImportDecls}  \: ImportDecl & (\arabic{equation})\\
    \| ImportDecls ImportDecl\\
 TypeDecls  \refstepcounter{equation}\label{prod:TypeDecls}  \: TypeDecl & (\arabic{equation})\\
    \| TypeDecls TypeDecl\\
 PackageDecl  \refstepcounter{equation}\label{prod:PackageDecl}  \: Annotations\opt \xcd"package" PackageName \xcd";" & (\arabic{equation})\\
 ImportDecl  \refstepcounter{equation}\label{prod:ImportDecl}  \: SingleTypeImportDecl & (\arabic{equation})\\
    \| TypeImportOnDemandDecl\\
 SingleTypeImportDecl  \refstepcounter{equation}\label{prod:SingleTypeImportDecl}  \: \xcd"import" TypeName \xcd";" & (\arabic{equation})\\
 TypeImportOnDemandDecl  \refstepcounter{equation}\label{prod:TypeImportOnDemandDecl}  \: \xcd"import" PackageOrTypeName \xcd"." \xcd"*" \xcd";" & (\arabic{equation})\\
 TypeDecl  \refstepcounter{equation}\label{prod:TypeDecl}  \: ClassDecl & (\arabic{equation})\\
    \| InterfaceDecl\\
    \| TypeDefDecl\\
    \| \xcd";"\\
 Interfaces  \refstepcounter{equation}\label{prod:Interfaces}  \: \xcd"implements" InterfaceTypeList & (\arabic{equation})\\
 InterfaceTypeList  \refstepcounter{equation}\label{prod:InterfaceTypeList}  \: Type & (\arabic{equation})\\
    \| InterfaceTypeList \xcd"," Type\\
 ClassBody  \refstepcounter{equation}\label{prod:ClassBody}  \: \xcd"{" ClassBodyDecls\opt \xcd"}" & (\arabic{equation})\\
 ClassBodyDecls  \refstepcounter{equation}\label{prod:ClassBodyDecls}  \: ClassBodyDecl & (\arabic{equation})\\
    \| ClassBodyDecls ClassBodyDecl\\
 ClassBodyDecl  \refstepcounter{equation}\label{prod:ClassBodyDecl}  \: ClassMemberDecl & (\arabic{equation})\\
    \| CtorDecl\\
 ClassMemberDecl  \refstepcounter{equation}\label{prod:ClassMemberDecl}  \: FieldDecl & (\arabic{equation})\\
    \| MethodDecl\\
    \| PropertyMethodDecl\\
    \| TypeDefDecl\\
    \| ClassDecl\\
    \| InterfaceDecl\\
    \| \xcd";"\\
 FormalDeclarators  \refstepcounter{equation}\label{prod:FormalDeclarators}  \: FormalDeclarator & (\arabic{equation})\\
    \| FormalDeclarators \xcd"," FormalDeclarator\\
 FieldDeclarators  \refstepcounter{equation}\label{prod:FieldDeclarators}  \: FieldDeclarator & (\arabic{equation})\\
    \| FieldDeclarators \xcd"," FieldDeclarator\\
 VariableDeclaratorsWithType  \refstepcounter{equation}\label{prod:VariableDeclaratorsWithType}  \: VariableDeclaratorWithType & (\arabic{equation})\\
    \| VariableDeclaratorsWithType \xcd"," VariableDeclaratorWithType\\
 VariableDeclarators  \refstepcounter{equation}\label{prod:VariableDeclarators}  \: VariableDeclarator & (\arabic{equation})\\
    \| VariableDeclarators \xcd"," VariableDeclarator\\
 VariableInitializer  \refstepcounter{equation}\label{prod:VariableInitializer}  \: Exp & (\arabic{equation})\\
 ResultType  \refstepcounter{equation}\label{prod:ResultType}  \: \xcd":" Type & (\arabic{equation})\\
 HasResultType  \refstepcounter{equation}\label{prod:HasResultType}  \: \xcd":" Type & (\arabic{equation})\\
    \| \xcd"<:" Type\\
 FormalParamList  \refstepcounter{equation}\label{prod:FormalParamList}  \: FormalParam & (\arabic{equation})\\
\end{bbgrammar}

\begin{bbgrammar}

    \| FormalParamList \xcd"," FormalParam\\
 LoopIndexDeclarator  \refstepcounter{equation}\label{prod:LoopIndexDeclarator}  \: Id HasResultType\opt & (\arabic{equation})\\
    \| \xcd"[" IdList \xcd"]" HasResultType\opt\\
    \| Id \xcd"[" IdList \xcd"]" HasResultType\opt\\
 LoopIndex  \refstepcounter{equation}\label{prod:LoopIndex}  \: Mods\opt LoopIndexDeclarator & (\arabic{equation})\\
    \| Mods\opt VarKeyword LoopIndexDeclarator\\
 FormalParam  \refstepcounter{equation}\label{prod:FormalParam}  \: Mods\opt FormalDeclarator & (\arabic{equation})\\
    \| Mods\opt VarKeyword FormalDeclarator\\
    \| Type\\
 Offers  \refstepcounter{equation}\label{prod:Offers}  \: \xcd"offers" Type & (\arabic{equation})\\
 ExceptionTypeList  \refstepcounter{equation}\label{prod:ExceptionTypeList}  \: ExceptionType & (\arabic{equation})\\
    \| ExceptionTypeList \xcd"," ExceptionType\\
 ExceptionType  \refstepcounter{equation}\label{prod:ExceptionType}  \: ClassType & (\arabic{equation})\\
 MethodBody  \refstepcounter{equation}\label{prod:MethodBody}  \: \xcd"=" LastExp \xcd";" & (\arabic{equation})\\
    \| \xcd"=" Annotations\opt \xcd"{" BlockStatements\opt LastExp \xcd"}"\\
    \| \xcd"=" Annotations\opt Block\\
    \| Annotations\opt Block\\
    \| \xcd";"\\
 CtorBody  \refstepcounter{equation}\label{prod:CtorBody}  \: \xcd"=" CtorBlock & (\arabic{equation})\\
    \| CtorBlock\\
    \| \xcd"=" ExplicitCtorInvocation\\
    \| \xcd"=" AssignPropertyCall\\
    \| \xcd";"\\
 CtorBlock  \refstepcounter{equation}\label{prod:CtorBlock}  \: \xcd"{" ExplicitCtorInvocation\opt BlockStatements\opt \xcd"}" & (\arabic{equation})\\
 Arguments  \refstepcounter{equation}\label{prod:Arguments}  \: \xcd"(" ArgumentList\opt \xcd")" & (\arabic{equation})\\
 InterfaceDecl  \refstepcounter{equation}\label{prod:InterfaceDecl}  \: NormalInterfaceDecl & (\arabic{equation})\\
 ExtendsInterfaces  \refstepcounter{equation}\label{prod:ExtendsInterfaces}  \: \xcd"extends" Type & (\arabic{equation})\\
    \| ExtendsInterfaces \xcd"," Type\\
 InterfaceBody  \refstepcounter{equation}\label{prod:InterfaceBody}  \: \xcd"{" InterfaceMemberDecls\opt \xcd"}" & (\arabic{equation})\\
 InterfaceMemberDecls  \refstepcounter{equation}\label{prod:InterfaceMemberDecls}  \: InterfaceMemberDecl & (\arabic{equation})\\
    \| InterfaceMemberDecls InterfaceMemberDecl\\
 InterfaceMemberDecl  \refstepcounter{equation}\label{prod:InterfaceMemberDecl}  \: MethodDecl & (\arabic{equation})\\
    \| PropertyMethodDecl\\
    \| FieldDecl\\
    \| ClassDecl\\
    \| InterfaceDecl\\
    \| TypeDefDecl\\
    \| \xcd";"\\
 Annotations  \refstepcounter{equation}\label{prod:Annotations}  \: Annotation & (\arabic{equation})\\
    \| Annotations Annotation\\
 Annotation  \refstepcounter{equation}\label{prod:Annotation}  \: \xcd"@" NamedType & (\arabic{equation})\\
\end{bbgrammar}

\begin{bbgrammar}

 Id  \refstepcounter{equation}\label{prod:Id}  \: identifier & (\arabic{equation})\\
 Block  \refstepcounter{equation}\label{prod:Block}  \: \xcd"{" BlockStatements\opt \xcd"}" & (\arabic{equation})\\
 BlockStatements  \refstepcounter{equation}\label{prod:BlockStatements}  \: BlockStatement & (\arabic{equation})\\
    \| BlockStatements BlockStatement\\
 BlockStatement  \refstepcounter{equation}\label{prod:BlockStatement}  \: LocalVariableDeclStatement & (\arabic{equation})\\
    \| ClassDecl\\
    \| TypeDefDecl\\
    \| Statement\\
 IdList  \refstepcounter{equation}\label{prod:IdList}  \: Id & (\arabic{equation})\\
    \| IdList \xcd"," Id\\
 FormalDeclarator  \refstepcounter{equation}\label{prod:FormalDeclarator}  \: Id ResultType & (\arabic{equation})\\
    \| \xcd"[" IdList \xcd"]" ResultType\\
    \| Id \xcd"[" IdList \xcd"]" ResultType\\
 FieldDeclarator  \refstepcounter{equation}\label{prod:FieldDeclarator}  \: Id HasResultType & (\arabic{equation})\\
    \| Id HasResultType\opt \xcd"=" VariableInitializer\\
 VariableDeclarator  \refstepcounter{equation}\label{prod:VariableDeclarator}  \: Id HasResultType\opt \xcd"=" VariableInitializer & (\arabic{equation})\\
    \| \xcd"[" IdList \xcd"]" HasResultType\opt \xcd"=" VariableInitializer\\
    \| Id \xcd"[" IdList \xcd"]" HasResultType\opt \xcd"=" VariableInitializer\\
 VariableDeclaratorWithType  \refstepcounter{equation}\label{prod:VariableDeclaratorWithType}  \: Id HasResultType \xcd"=" VariableInitializer & (\arabic{equation})\\
    \| \xcd"[" IdList \xcd"]" HasResultType \xcd"=" VariableInitializer\\
    \| Id \xcd"[" IdList \xcd"]" HasResultType \xcd"=" VariableInitializer\\
 LocalVariableDeclStatement  \refstepcounter{equation}\label{prod:LocalVariableDeclStatement}  \: LocalVariableDecl \xcd";" & (\arabic{equation})\\
 LocalVariableDecl  \refstepcounter{equation}\label{prod:LocalVariableDecl}  \: Mods\opt VarKeyword VariableDeclarators & (\arabic{equation})\\
    \| Mods\opt VariableDeclaratorsWithType\\
    \| Mods\opt VarKeyword FormalDeclarators\\
 Primary  \refstepcounter{equation}\label{prod:Primary}  \: here & (\arabic{equation})\\
    \| \xcd"[" ArgumentList\opt \xcd"]"\\
    \| Literal\\
    \| \xcd"self"\\
    \| \xcd"this"\\
    \| ClassName \xcd"." \xcd"this"\\
    \| \xcd"(" Exp \xcd")"\\
    \| ClassInstCreationExp\\
    \| FieldAccess\\
    \| MethodInvocation\\
    \| MethodSelection\\
    \| OperatorFunction\\
 OperatorFunction  \refstepcounter{equation}\label{prod:OperatorFunction}  \: TypeName \xcd"." \xcd"+" & (\arabic{equation})\\
    \| TypeName \xcd"." \xcd"-"\\
    \| TypeName \xcd"." \xcd"*"\\
    \| TypeName \xcd"." \xcd"/"\\
\end{bbgrammar}

\begin{bbgrammar}

    \| TypeName \xcd"." \xcd"%"\\
    \| TypeName \xcd"." \xcd"&"\\
    \| TypeName \xcd"." \xcd"|"\\
    \| TypeName \xcd"." \xcd"^"\\
    \| TypeName \xcd"." \xcd"<<"\\
    \| TypeName \xcd"." \xcd">>"\\
    \| TypeName \xcd"." \xcd">>>"\\
    \| TypeName \xcd"." \xcd"<"\\
    \| TypeName \xcd"." \xcd"<="\\
    \| TypeName \xcd"." \xcd">="\\
    \| TypeName \xcd"." \xcd">"\\
    \| TypeName \xcd"." \xcd"=="\\
    \| TypeName \xcd"." \xcd"!="\\
 Literal  \refstepcounter{equation}\label{prod:Literal}  \: \xcd"IntegerLiteral"  & (\arabic{equation})\\
    \| \xcd"LongLiteral" \\
    \| \xcd"UnsignedIntegerLiteral" \\
    \| \xcd"UnsignedLongLiteral" \\
    \| \xcd"FloatingPointLiteral" \\
    \| \xcd"DoubleLiteral" \\
    \| BooleanLiteral\\
    \| \xcd"CharacterLiteral" \\
    \| \xcd"StringLiteral" \\
    \| \xcd"null"\\
 BooleanLiteral  \refstepcounter{equation}\label{prod:BooleanLiteral}  \: \xcd"true"  & (\arabic{equation})\\
    \| \xcd"false" \\
 ArgumentList  \refstepcounter{equation}\label{prod:ArgumentList}  \: Exp & (\arabic{equation})\\
    \| ArgumentList \xcd"," Exp\\
 FieldAccess  \refstepcounter{equation}\label{prod:FieldAccess}  \: Primary \xcd"." Id & (\arabic{equation})\\
    \| \xcd"super" \xcd"." Id\\
    \| ClassName \xcd"." \xcd"super"  \xcd"." Id\\
    \| Primary \xcd"." \xcd"class" \\
    \| \xcd"super" \xcd"." \xcd"class" \\
    \| ClassName \xcd"." \xcd"super"  \xcd"." \xcd"class" \\
 MethodInvocation  \refstepcounter{equation}\label{prod:MethodInvocation}  \: MethodName TypeArguments\opt \xcd"(" ArgumentList\opt \xcd")" & (\arabic{equation})\\
    \| Primary \xcd"." Id TypeArguments\opt \xcd"(" ArgumentList\opt \xcd")"\\
    \| \xcd"super" \xcd"." Id TypeArguments\opt \xcd"(" ArgumentList\opt \xcd")"\\
    \| ClassName \xcd"." \xcd"super"  \xcd"." Id TypeArguments\opt \xcd"(" ArgumentList\opt \xcd")"\\
    \| Primary TypeArguments\opt \xcd"(" ArgumentList\opt \xcd")"\\
 MethodSelection  \refstepcounter{equation}\label{prod:MethodSelection}  \: MethodName \xcd"." \xcd"(" FormalParamList\opt \xcd")" & (\arabic{equation})\\
    \| Primary \xcd"." Id \xcd"." \xcd"(" FormalParamList\opt \xcd")"\\
    \| \xcd"super" \xcd"." Id \xcd"." \xcd"(" FormalParamList\opt \xcd")"\\
\end{bbgrammar}

\begin{bbgrammar}

    \| ClassName \xcd"." \xcd"super"  \xcd"." Id \xcd"." \xcd"(" FormalParamList\opt \xcd")"\\
 PostfixExp  \refstepcounter{equation}\label{prod:PostfixExp}  \: CastExp & (\arabic{equation})\\
    \| PostIncrementExp\\
    \| PostDecrementExp\\
 PostIncrementExp  \refstepcounter{equation}\label{prod:PostIncrementExp}  \: PostfixExp \xcd"++" & (\arabic{equation})\\
 PostDecrementExp  \refstepcounter{equation}\label{prod:PostDecrementExp}  \: PostfixExp \xcd"--" & (\arabic{equation})\\
 UnannotatedUnaryExp  \refstepcounter{equation}\label{prod:UnannotatedUnaryExp}  \: PreIncrementExp & (\arabic{equation})\\
    \| PreDecrementExp\\
    \| \xcd"+" UnaryExpNotPlusMinus\\
    \| \xcd"-" UnaryExpNotPlusMinus\\
    \| UnaryExpNotPlusMinus\\
 UnaryExp  \refstepcounter{equation}\label{prod:UnaryExp}  \: UnannotatedUnaryExp & (\arabic{equation})\\
    \| Annotations UnannotatedUnaryExp\\
 PreIncrementExp  \refstepcounter{equation}\label{prod:PreIncrementExp}  \: \xcd"++" UnaryExpNotPlusMinus & (\arabic{equation})\\
 PreDecrementExp  \refstepcounter{equation}\label{prod:PreDecrementExp}  \: \xcd"--" UnaryExpNotPlusMinus & (\arabic{equation})\\
 UnaryExpNotPlusMinus  \refstepcounter{equation}\label{prod:UnaryExpNotPlusMinus}  \: PostfixExp & (\arabic{equation})\\
    \| \xcd"~" UnaryExp\\
    \| \xcd"!" UnaryExp\\
 MultiplicativeExp  \refstepcounter{equation}\label{prod:MultiplicativeExp}  \: UnaryExp & (\arabic{equation})\\
    \| MultiplicativeExp \xcd"*" UnaryExp\\
    \| MultiplicativeExp \xcd"/" UnaryExp\\
    \| MultiplicativeExp \xcd"%" UnaryExp\\
 AdditiveExp  \refstepcounter{equation}\label{prod:AdditiveExp}  \: MultiplicativeExp & (\arabic{equation})\\
    \| AdditiveExp \xcd"+" MultiplicativeExp\\
    \| AdditiveExp \xcd"-" MultiplicativeExp\\
 ShiftExp  \refstepcounter{equation}\label{prod:ShiftExp}  \: AdditiveExp & (\arabic{equation})\\
    \| ShiftExp \xcd"<<" AdditiveExp\\
    \| ShiftExp \xcd">>" AdditiveExp\\
    \| ShiftExp \xcd">>>" AdditiveExp\\
 RangeExp  \refstepcounter{equation}\label{prod:RangeExp}  \: ShiftExp & (\arabic{equation})\\
    \| ShiftExp  \xcd".." ShiftExp \\
 RelationalExp  \refstepcounter{equation}\label{prod:RelationalExp}  \: RangeExp & (\arabic{equation})\\
    \| SubtypeConstraint\\
    \| RelationalExp \xcd"<" RangeExp\\
    \| RelationalExp \xcd">" RangeExp\\
    \| RelationalExp \xcd"<=" RangeExp\\
    \| RelationalExp \xcd">=" RangeExp\\
    \| RelationalExp \xcd"instanceof" Type\\
    \| RelationalExp \xcd"in" ShiftExp\\
 EqualityExp  \refstepcounter{equation}\label{prod:EqualityExp}  \: RelationalExp & (\arabic{equation})\\
    \| EqualityExp \xcd"==" RelationalExp\\
\end{bbgrammar}

\begin{bbgrammar}

    \| EqualityExp \xcd"!=" RelationalExp\\
    \| Type  \xcd"==" Type \\
 AndExp  \refstepcounter{equation}\label{prod:AndExp}  \: EqualityExp & (\arabic{equation})\\
    \| AndExp \xcd"&" EqualityExp\\
 ExclusiveOrExp  \refstepcounter{equation}\label{prod:ExclusiveOrExp}  \: AndExp & (\arabic{equation})\\
    \| ExclusiveOrExp \xcd"^" AndExp\\
 InclusiveOrExp  \refstepcounter{equation}\label{prod:InclusiveOrExp}  \: ExclusiveOrExp & (\arabic{equation})\\
    \| InclusiveOrExp \xcd"|" ExclusiveOrExp\\
 ConditionalAndExp  \refstepcounter{equation}\label{prod:ConditionalAndExp}  \: InclusiveOrExp & (\arabic{equation})\\
    \| ConditionalAndExp \xcd"&&" InclusiveOrExp\\
 ConditionalOrExp  \refstepcounter{equation}\label{prod:ConditionalOrExp}  \: ConditionalAndExp & (\arabic{equation})\\
    \| ConditionalOrExp \xcd"||" ConditionalAndExp\\
 ConditionalExp  \refstepcounter{equation}\label{prod:ConditionalExp}  \: ConditionalOrExp & (\arabic{equation})\\
    \| ClosureExp\\
    \| AtExp\\
    \| FinishExp\\
    \| ConditionalOrExp \xcd"?" Exp \xcd":" ConditionalExp\\
 AssignmentExp  \refstepcounter{equation}\label{prod:AssignmentExp}  \: Assignment & (\arabic{equation})\\
    \| ConditionalExp\\
 Assignment  \refstepcounter{equation}\label{prod:Assignment}  \: LeftHandSide AssignmentOperator AssignmentExp & (\arabic{equation})\\
    \| ExpName  \xcd"(" ArgumentList\opt \xcd")" AssignmentOperator AssignmentExp\\
    \| Primary  \xcd"(" ArgumentList\opt \xcd")" AssignmentOperator AssignmentExp\\
 LeftHandSide  \refstepcounter{equation}\label{prod:LeftHandSide}  \: ExpName & (\arabic{equation})\\
    \| FieldAccess\\
 AssignmentOperator  \refstepcounter{equation}\label{prod:AssignmentOperator}  \: \xcd"=" & (\arabic{equation})\\
    \| \xcd"*="\\
    \| \xcd"/="\\
    \| \xcd"%="\\
    \| \xcd"+="\\
    \| \xcd"-="\\
    \| \xcd"<<="\\
    \| \xcd">>="\\
    \| \xcd">>>="\\
    \| \xcd"&="\\
    \| \xcd"^="\\
    \| \xcd"|="\\
 Exp  \refstepcounter{equation}\label{prod:Exp}  \: AssignmentExp & (\arabic{equation})\\
 ConstantExp  \refstepcounter{equation}\label{prod:ConstantExp}  \: Exp & (\arabic{equation})\\
 PrefixOp  \refstepcounter{equation}\label{prod:PrefixOp}  \: \xcd"+" & (\arabic{equation})\\
    \| \xcd"-"\\
    \| \xcd"!"\\
\end{bbgrammar}

\begin{bbgrammar}

    \| \xcd"~"\\
 BinOp  \refstepcounter{equation}\label{prod:BinOp}  \: \xcd"+" & (\arabic{equation})\\
    \| \xcd"-"\\
    \| \xcd"*"\\
    \| \xcd"/"\\
    \| \xcd"%"\\
    \| \xcd"&"\\
    \| \xcd"|"\\
    \| \xcd"^"\\
    \| \xcd"&&"\\
    \| \xcd"||"\\
    \| \xcd"<<"\\
    \| \xcd">>"\\
    \| \xcd">>>"\\
    \| \xcd">="\\
    \| \xcd"<="\\
    \| \xcd">"\\
    \| \xcd"<"\\
    \| \xcd"=="\\
    \| \xcd"!="\\
\end{bbgrammar}




\clearpage
\addcontentsline{toc}{chapter}{References}
\renewcommand{\bibname}{References}
\bibliographystyle{plain}
\bibliography{master}

%\documentclass[10pt,twoside,twocolumn,notitlepage]{report}
%\documentclass[12pt,twoside,notitlepage]{report}
\documentclass[10pt,twoside,notitlepage]{report}
\usepackage{tex/x10}
\usepackage{tex/tenv}
\def\Hat{{\tt \char`\^}}
\usepackage{url}
\usepackage{times}
\usepackage{tex/txtt}
\usepackage{ifpdf}
\usepackage{tocloft}
\usepackage{tex/bcprules}
\usepackage{xspace}
\usepackage{makeidx}

\newif\ifdraft
%\drafttrue
\draftfalse

\pagestyle{headings}
\showboxdepth=0
\makeindex

\usepackage{tex/commands}

\usepackage[
pdfauthor={Vijay Saraswat, Bard Bloom, Igor Peshansky, Olivier Tardieu, and David Grove},
pdftitle={X10 Language Specification},
pdfcreator={pdftex},
pdfkeywords={X10},
linkcolor=blue,
citecolor=blue,
urlcolor=blue
]{hyperref}

\ifpdf
          \pdfinfo {
              /Author   (Vijay Saraswat, Bard Bloom, Igor Peshansky, Olivier Tardieu, and David Grove)
              /Title    (X10 Language Specification)
              /Keywords (X10)
              /Subject  ()
              /Creator  (TeX)
              /Producer (PDFLaTeX)
          }
\fi

\def\headertitle{The \XtenCurrVer{} Report}
\def\integerversion{2.4}

% Sizes and dimensions

%\topmargin -.375in       %    Nominal distance from top of page to top of
                         %    box containing running head.
%\headsep 15pt            %    Space between running head and text.

%\textheight 9.0in        % Height of text (including footnotes and figures, 
                         % excluding running head and foot).

%\textwidth 5.5in         % Width of text line.
\columnsep 15pt          % Space between columns 
\columnseprule 0pt       % Width of rule between columns.

\parskip 5pt plus 2pt minus 2pt % Extra vertical space between paragraphs.
\parindent 0pt                  % Width of paragraph indentation.
%\topsep 0pt plus 2pt            % Extra vertical space, in addition to 
                                % \parskip, added above and below list and
                                % paragraphing environments.


\newif\iftwocolumn

\makeatletter
\twocolumnfalse
\if@twocolumn
\twocolumntrue
\fi
\makeatother

\iftwocolumn

\oddsidemargin  0in    % Left margin on odd-numbered pages.
\evensidemargin 0in    % Left margin on even-numbered pages.

\else

\oddsidemargin  .5in    % Left margin on odd-numbered pages.
\evensidemargin .5in    % Left margin on even-numbered pages.

\fi


\newtenv{example}{Example}[section]
\newtenv{planned}{Planned}[section]

\begin{document}

% \section{Work In Progress}
% \begin{itemize}
%     \item Rewrite first chapter
%     \item Describe library classes, including such fundamentals as Any and String
%     \item Examples for covariant/contravariant generics are wrong -- use Nate's examples
%     \item Describe local classes.
%     \item Reduce the use of \xcd`self` in constraints.
%     \item Copy sections of grammar to relevant sections of text.
%     \item Do something about 4.12.3
% \end{itemize}
% 
% {\bf Feedback:} 
% To help us the most, we would appreciate comments in one of these formats: 
% \begin{itemize}
% \item An annotated copy of the PDF document, if it's convenient.  Acrobat
%       Writer can produce helpful highlighting and sticky notes.  If you don't
%       use Acrobat Writer, don't fuss.
% \item Text comments.  Since the document is still being edited, page numbers
%       are going to be useless as pointers to the text.  If possible, we'd like
%       pointers to sections by number and title: {\em In 12.1, ``Empty
%       Statement'', please discuss side effects and performance implications
%       for this construct''}  If it's a long section, giving us a couple words
%       we can grep for would help too.
% \end{itemize}
% 
% Thank you very much!




% \parindent 0pt %!! 15pt                    % Width of paragraph indentation.

%\hfil {\bf 7 Feb 2005}
%\hfil \today{}

% First page

\thispagestyle{empty}

% \todo{"another" report?}

\title{ \Xten Language Specification \\
\large Version \integerversion}
%\ifdraft
\author{Vijay Saraswat, Bard Bloom, Igor Peshansky, Olivier Tardieu, and David Grove\\
\\
Please send comments to 
\texttt{vsaraswa@us.ibm.com}}
%\else
%\author{
%Vijay Saraswat \\
%Please send comments to \\
%\texttt{vsaraswa@us.ibm.com}}
%\fi
%\date{\today}

\maketitle

\newcommand\authorsc[1]{#1}
%\newcommand\authorsc[1]{\textsc{#1}}

This report provides a description of the programming
language \Xten. \Xten{} is a class-based object-oriented
programming language designed for high-performance, high-productivity
computing on high-end computers supporting $\approx 10^5$ hardware threads
and $\approx 10^{15}$ operations per second. 

\Xten{} is based on state-of-the-art object-oriented programming
languages and deviates from them only as necessary to support its
design goals. The language is intended to have a simple and clear
semantics and be readily accessible to mainstream OO programmers. It
is intended to support a wide variety of concurrent programming
idioms.
%, incuding data parallelism, task parallelism, pipelining.
%producer/consumer and divide and conquer.

%We expect to revise this document in the light of experience gained in implementing
%and using this language.

The \Xten{} design team consists of
\authorsc{David Cunningham},
\authorsc{David Grove},
\authorsc{Ben Herta},
\authorsc{Vijay Saraswat},
\authorsc{Avraham Shinnar},
\authorsc{Mikio Takeuchi},
\authorsc{Olivier Tardieu}.

Past members include
\authorsc{Shivali Agarwal}, 
\authorsc{Bowen Alpern}, 
\authorsc{David Bacon}, 
\authorsc{Raj Barik}, 
\authorsc{Ganesh Bikshandi}, 
\authorsc{Bob Blainey}, 
\authorsc{Bard Bloom}, 
\authorsc{Philippe Charles}, 
\authorsc{Perry Cheng}, 
\authorsc{Christopher Donawa}, 
\authorsc{Julian Dolby}, 
\authorsc{Kemal Ebcio\u{g}lu},
\authorsc{Stephen Fink},
\authorsc{Robert Fuhrer},
\authorsc{Patrick Gallop}, 
\authorsc{Christian Grothoff}, 
\authorsc{Hiroshi Horii}, 
\authorsc{Kiyokuni Kawachiya}, 
\authorsc{Allan Kielstra}, 
\authorsc{Sreedhar Kodali}, 
\authorsc{Sriram Krishnamoorthy}, 
\authorsc{Yan Li}, 
\authorsc{Bruce Lucas},
\authorsc{Yuki Makino}, 
\authorsc{Nathaniel Nystrom},
\authorsc{Igor Peshansky}, 
\authorsc{Vivek Sarkar},
\authorsc{Armando Solar-Lezama},  
\authorsc{S. Alexander Spoon}, 
\authorsc{Toshio Suganuma}, 
\authorsc{Sayantan Sur}, 
\authorsc{Toyotaro Suzumura}, 
\authorsc{Christoph von Praun},
\authorsc{Leena Unnikrishnan},
\authorsc{Pradeep Varma}, 
\authorsc{Krishna Nandivada Venkata},
\authorsc{Jan Vitek}, 
\authorsc{Hai Chuan Wang}, 
\authorsc{Tong Wen}, 
\authorsc{Salikh Zakirov}, and
\authorsc{Yoav Zibin}.


For extended discussions and support we would like to thank: 
Gheorghe Almasi,
Robert Blackmore,
Rob O'Callahan, 
Calin Cascaval, 
Norman Cohen, 
Elmootaz Elnozahy, 
John Field,
Kevin Gildea,
Chulho Kim,
Orren Krieger, 
Doug Lea, 
John McCalpin, 
Paul McKenney, 
Josh Milthorpe,
Andrew Myers,
Filip Pizlo, 
Ram Rajamony,
R.~K. Shyamasundar, 
V.~T. Rajan, 
Frank Tip,
Mandana Vaziri,
and
Hanhong Xue.


We thank Jonathan Rhees and William Clinger with help in obtaining the
\LaTeX{} style file and macros used in producing the Scheme report,
on which this document is based. We acknowledge the influence of
the $\mbox{\Java}^{\mbox{\textsc{tm}}}$ Language
Specification \cite{jls2}, the Scala language specification
\cite{scala-spec}, and ZPL \cite{zpl}.
%document, as evidenced by the numerous citations in the text.

This document specifies the language corresponding to Version
\integerversion{} of the implementation. The redesign and reimplementation of arrays and rails was  done by Dave Grove and Olivier Tardieu.
 Version 1.7 of the report was co-authored by Nathaniel Nystrom. The design of structs in \Xten{} was led by Olivier Tardieu and Nathaniel Nystrom.

Earlier implementations benefited from significant contributions by
Raj Barik, 
Philippe Charles, 
David Cunningham,
Christopher Donawa, 
Robert Fuhrer,
Christian Grothoff,
Nathaniel Nystrom,  
Igor Peshansky,  
Vijay Saraswat,
Vivek Sarkar, 
Olivier Tardieu,  
Pradeep Varma, 
Krishna Nandivada Venkata, and
Christoph von Praun.
Tong Wen has written many application programs
in \Xten{}. Guojing Cong has helped in the
development of many applications.
The implementation of generics in \Xten{} was influenced by the
implementation of PolyJ~\cite{polyj} by Andrew Myers and Michael Clarkson.
 

\clearpage

{\parskip 0pt
\addtolength{\cftsecnumwidth}{0.5em}
\addtolength{\cftsubsecnumwidth}{0.5em}
%\addtolength{\cftsecindent}{0.5em}
\addtolength{\cftsubsecindent}{0.5em}
\tableofcontents
}


\chapter{Introduction}

\subsection*{Background}



The era of the mighty single-processor computer is over. Now, when more
computing power is needed, one does not buy a faster uniprocessor---one buys
another processor just like those one already has, or another hundred, or
another million, and connects them with a high-speed communication network.
Or, perhaps, one rents them instead, with a cloud computer. This gives one
whatever quantity of computer cycles that one can desire and afford.

Then, one has the problem of how to use those computer cycles effectively.
Programming a multiprocessor is far more agonizing than programming a
uniprocessor.   One can use models of computation which give somewhat of the
illusion of programming a uniprocessor.  Unfortunately, the models which give
the closest imitations of uniprocessing are very expensive to implement,
either increasing the monetary cost of the computer tremendously, or slowing
it down dreadfully. 

One response to this problem has been to move to a {\em fragmented memory
  model}. Multiple processors are programmed largely as if they were
uniprocessors, but are made to interact via a relatively language-neutral
message-passing format such as MPI \cite{mpi}. This model has enjoyed some
success: several high-performance applications have been written in this
style. Unfortunately, this model leads to a {\em loss of programmer
  productivity}: the message-passing format is integrated into the host
language by means of an application-programming interface (API), the
programmer must explicitly represent and manage the interaction between
multiple processes and choreograph their data exchange; large data-structures
(such as distributed arrays, graphs, hash-tables) that are conceptually
unitary must be thought of as fragmented across different nodes; all
processors must generally execute the same code (in an SPMD fashion) etc.

One response to this problem has been the advent of the {\em
partitioned global address space} (PGAS) model underlying languages
such as UPC, Titanium and Co-Array Fortran \cite{pgas,titanium}. These
languages permit the programmer to think of a single computation
running across multiple processors, sharing a common address
space. All data resides at some processor, which is said to have {\em
affinity} to the data.  Each processor may operate directly on the
data it contains but must use some indirect mechanism to access or
update data at other processors. Some kind of global {\em barriers}
are used to ensure that processors remain roughly in lock-step.

\Xten{} is a modern object-oriented programming language
in the PGAS family. The fundamental goal of \Xten{} is to enable
scalable, high-performance, high-productivity transformational
programming for high-end computers---for traditional numerical
computation workloads (such as weather simulation, molecular dynamics,
particle transport problems etc) as well as commercial server
workloads.

\Xten{} is based on state-of-the-art object-oriented
programming ideas primarily to take advantage of their proven
flexibility and ease-of-use for a wide spectrum of programming
problems. \Xten{} takes advantage of several years of research (e.g.,
in the context of the Java Grande forum,
\cite{moreira00java,kava}) on how to adapt such languages to the context of
high-performance numerical computing. Thus \Xten{} provides support
for user-defined {\em struct types} (such as \xcd"Int", \xcd"Float",
\xcd"Complex" etc), supports a very
flexible form of multi-dimensional arrays (based on ideas in ZPL
\cite{zpl}) and supports IEEE-standard floating point arithmetic.
Some capabilities for supporting operator overloading are also provided.

{}\Xten{} introduces a flexible treatment of concurrency, distribution
and locality, within an integrated type system. \Xten{} extends the
PGAS model with {\em asynchrony} (yielding the {\em APGAS} programming
model). {}\Xten{} introduces {\em places} as an abstraction for a
computational context with a locally synchronous view of shared
memory. An \Xten{} computation runs over a large collection of places.
Each place hosts some data and runs one or more {\em
activities}. Activities are extremely lightweight threads of
execution. An activity may synchronously (and {\em atomically}) use
one or more memory locations in the place in which it resides,
leveraging current symmetric multiprocessor (SMP) technology.  
An activity may shift to another place to execute a statement block.
\Xten{} provides weaker ordering guarantees for
inter-place data access, enabling applications to scale.  
Multiple memory locations in multiple places cannot be
accessed atomically.  {\em
Immutable} data needs no consistency management and may be freely
copied by the implementation between places.  One or more {\em clocks}
may be used to order activities running in multiple
places.  \xcd`DistArray`s, distributed arrays,  may be distributed across
multiple 
places and  support parallel collective operations. A novel
exception flow model ensures that exceptions thrown by asynchronous
activities can be caught at a suitable parent activity.  The type
system tracks which memory accesses are local. The programmer may
introduce place casts which verify the access is local at run time.
Linking with native code is supported.

\chapter{Overview of \Xten}

\Xten{} is a statically typed object-oriented language, extending a sequential
core language with \emph{places}, \emph{activities}, \emph{clocks},
(distributed, multi-dimensional) \emph{arrays} and \emph{struct} types. All
these changes are motivated by the desire to use the new language for
high-end, high-performance, high-productivity computing.

\section{Object-oriented features}

The sequential core of \Xten{} is a {\em container-based} object-oriented language
similar to \java{} and C++, and more recent languages such as Scala.  
Programmers write \Xten{} code by defining containers for data and behavior
called 
\emph{classes}
(\Sref{XtenClasses}) and
\emph{structs}
(\Sref{XtenStructs}), 
often abstracted as 
\emph{interfaces}
(\Sref{XtenInterfaces}).
X10 provides inheritance and subtyping in fairly traditional ways. 

\begin{ex}

\xcd`Normed` describes entities with a \xcd`norm()` method. \xcd`Normed` is
intended to be used for entities with a position in some coordinate system,
and \xcd`norm()` gives the distance between the entity and the origin. A
\xcd`Slider` is an object which can be moved around on a line; a
\xcd`PlanePoint` is a fixed position in a plane. Both \xcd`Slider`s and
\xcd`PlanePoint`s have a sensible \xcd`norm()` method, and implement
\xcd`Normed`.

%~~gen ^^^ Overview10
% package Overview;
%~~vis
\begin{xten}
interface Normed {
  def norm():Double;
}
class Slider implements Normed {
  var x : Double = 0;
  public def norm() = Math.abs(x);
  public def move(dx:Double) { x += dx; }
}
struct PlanePoint implements Normed {
  val x : Double, y:Double;
  public def this(x:Double, y:Double) {
    this.x = x; this.y = y;
  }
  public def norm() = Math.sqrt(x*x+y*y);
}
\end{xten}
%~~siv
%
%~~neg
\end{ex}

\paragraph{Interfaces}

An \Xten{} interface specifies a collection of abstract methods; \xcd`Normed`
specifies just \xcd`norm()`. Classes and
structs can be specified to {\em implement} interfaces, as \xcd`Slider` and
\xcd`PlanePoint` implement \xcd`Normed`, and, when they do so, must provide
all the methods that the interface demands.

Interfaces are
purely abstract. Every value of type \xcd`Normed` must be an instance of some
class like \xcd`Slider` or some struct like \xcd`PlanePoint` which implements
\xcd`Normed`; no value can be \xcd`Normed` and nothing else. 


\paragraph{Classes and Structs}



There are two kinds of containers: \emph{classes}
(\Sref{ReferenceClasses}) and \emph{structs} (\Sref{Structs}). Containers hold
data in {\em fields}, and give concrete implementations of 
methods, as \xcd`Slider` and \xcd`PlainPoint` above.

Classes are organized in a single-inheritance tree: a class may have only a
single parent class, though it may implement many interfaces and have many
subclasses. Classes may have mutable fields, as \xcd`Slider` does.

In contrast, structs are headerless values, lacking the internal organs
which give objects their intricate behavior.  This makes them less powerful
than objects (\eg, structs cannot inherit methods, though objects can), but also
cheaper (\eg, they can be inlined, and they require less space than objects).  
Structs are immutable, though their fields may be immutably set to objects
which are themselves mutable.  They behave like objects in all ways consistent
with these limitations; \eg, while they cannot {\em inherit} methods, they can
have them -- as \xcd`PlanePoint` does.

\Xten{} has no primitive classes per se. However, the standard library
\xcd"x10.lang" supplies structs and objects \xcd"Boolean", \xcd"Byte",
\xcd"Short", \xcd"Char", \xcd"Int", \xcd"Long", \xcd"Float", \xcd"Double",
\xcd"Complex" and \xcd"String". The user may defined additional arithmetic
structs using the facilities of the language.



\paragraph{Functions.}

X10 provides functions (\Sref{Closures}) to allow code to be used
as values.  Functions are first-class data: they can be stored in lists,
passed between activities, and so on.  \xcd`square`, below, is a function
which squares an \xcd`Int`.  \xcd`of4` takes an \xcd`Int`-to-\xcd`Int`
function and applies it to the number \xcd`4`.  So, \xcd`fourSquared` computes
\xcd`of4(square)`, which is \xcd`square(4)`, which is 16, in a fairly
complicated way.
%~~gen ^^^ Overview20
% package Overview.of.Functions.one;
% class Whatever{
% def chkplz() {
%~~vis
\begin{xten}
  val square = (i:Int) => i*i;
  val of4 = (f: (Int)=>Int) => f(4);
  val fourSquared = of4(square);
\end{xten}
%~~siv
%}}
%~~neg



Functions are used extensively in X10
programs.  For example, a common way to construct and initialize an \xcd`Array[Int](1)` --
that is, a fixed-length one-dimensional array of numbers, like an \xcd`int[]` in Java -- is to
pass two arguments to a factory method: the first argument being the length of
the array, and the second being a function which computes the initial value of
the \xcd`i`{$^{th}$} element.  The following code constructs a 1-dimensional
array 
initialized to the squares of 0,1,...,9: \xcd`r(0) == 0`, \xcd`r(5)==25`, etc. 
%~~gen ^^^ Overview30
% package Overview.of.Functions.two;
% class Whatevermore {
%  def plzchk(){
%    val square = (i:Int) => i*i;
%~~vis
\begin{xten}
  val r : Array[Int](1) = new Array[Int](10, square);
\end{xten}
%~~siv
%}}
%~~neg








\paragraph{Constrained Types}

X10 containers may declare {\em properties}, which are fields bound immutably
at the creation of the container.  The static analysis system understands
properties, and can work with them logically.   


For example, an implementation of matrices \xcd`Mat` might have the numbers of
rows and columns as properties.  A little bit of care in definitions allows
the definition of a \xcd`+` operation that works on matrices of the same
shape, and \xcd`*` that works on matrices with appropriately matching shapes.
%~~gen ^^^ Overview40
%package Overview.Mat2;
%~~vis
\begin{xten}
abstract class Mat(rows:Int, cols:Int) {
 static type Mat(r:Int, c:Int) = Mat{rows==r&&cols==c};
 abstract operator this + (y:Mat(this.rows,this.cols))
                 :Mat(this.rows, this.cols);
 abstract operator this * (y:Mat) {this.cols == y.rows} 
                 :Mat(this.rows, y.cols);
\end{xten}
%~~siv
%  static def makeMat(r:Int,c:Int) : Mat(r,c) = null;
%  static def example(a:Int, b:Int, c:Int) {
%    val axb1 : Mat(a,b) = makeMat(a,b);
%    val axb2 : Mat(a,b) = makeMat(a,b);
%    val bxc  : Mat(b,c) = makeMat(b,c);
%    val axc  : Mat(a,c) = (axb1 +axb2) * bxc;
%  }
%}
%~~neg



The following code typechecks (assuming that \xcd`makeMat(m,n)` is a function
which creates an \xcdmath"m$\times$n" matrix).
However, an attempt to compute \xcd`axb1 + bxc` or
\xcd`bxc * axb1` would result in a compile-time type error:
%~~gen ^^^ Overview50
%package Overview.Mat1;
% // OPTIONS: -STATIC_CALLS 
%abstract class Mat(rows:Int, cols:Int) {
%  static type Mat(r:Int, c:Int) = Mat{rows==r&&cols==c};
%  public def this(r:Int, c:Int) : Mat(r,c) = {property(r,c);}
%  static def makeMat(r:Int,c:Int) : Mat(r,c) = null;
%  abstract  operator this + (y:Mat(this.rows,this.cols)):Mat(this.rows, this.cols);
%  abstract  operator this * (y:Mat) {this.cols == y.rows} : Mat(this.rows, y.cols);
%~~vis
\begin{xten}
  static def example(a:Int, b:Int, c:Int) {
    val axb1 : Mat(a,b) = makeMat(a,b);
    val axb2 : Mat(a,b) = makeMat(a,b);
    val bxc  : Mat(b,c) = makeMat(b,c);
    val axc  : Mat(a,c) = (axb1 +axb2) * bxc;
    //ERROR: val wrong1 = axb1 + bxc;
    //ERROR: val wrong2 = bxc * axb1;
  }

\end{xten}
%~~siv
%}
%~~neg

The ``little bit of care'' shows off many of the features of constrained
types.    
The \xcd`(rows:Int, cols:Int)` in the class definition declares two
properties, \xcd`rows` and \xcd`cols`.\footnote{The class is officially declared
abstract to allow for multiple implementations, like sparse and band matrices,
but in fact is abstract to avoid having to write the actual definitions of
\xcd`+` and \xcd`*`.}  

A constrained type looks like \xcd`Mat{rows==r && cols==c}`: a type
name, followed by a Boolean expression in braces.  
The \xcd`type` declaration on the second line makes
\xcd`Mat(r,c)` be a synonym for \xcd`Mat{rows==r && cols==c}`,
allowing for compact types in many places.

Functions can return constrained types.  
The \xcd`makeMat(r,c)` method returns a \xcd`Mat(r,c)` -- a matrix whose shape
is given by the arguments to the method.    In
particular, constructors can have constrained return types to provide specific
information about the constructed values.

The arguments of methods can have type constraints as well.  The 
\xcd`operator this +` line lets \xcd`A+B` add two matrices.  The type of the
second argument \xcd`y` is constrained to have the same number of rows and
columns as the first argument \xcd`this`. Attempts to add mismatched matrices
will be flagged as type errors at compilation.

At times it is more convenient to put the constraint on the method as a whole,
as seen in the \xcd`operator this *` line. Unlike for \xcd`+`, there is no
need to constrain both dimensions; we simply need to check that the columns of
the left factor match the rows of the right. This constraint is written in
\xcd`{...}` after the argument list.  The shape of the result is computed from
the shapes of the arguments.

And that is all that is necessary for a user-defined class of matrices to have
shape-checking for matrix addition and multiplication.  The \xcd`example`
method compiles under those definitions.








\paragraph{Generic types}

Containers may have type parameters, permitting the definition of
{\em generic types}.  Type parameters may be instantiated by any X10 type.  It
is thus possible to make a list of integers \xcd`List[Int]`, a list of
non-zero integers \xcd`List[Int{self != 0}]`, or a list of people
\xcd`List[Person]`.  In the definition of \xcd`List`, \xcd`T` is a type
parameter; it can be instantiated with any type.
%~~gen ^^^ Overview60
%~~vis
\begin{xten}
class List[T] {
    var head: T;
    var tail: List[T];
    def this(h: T, t: List[T]) { head = h; tail = t; }
    def add(x: T) {
        if (this.tail == null)
            this.tail = new List[T](x, null);
        else
            this.tail.add(x);
    }
}
\end{xten}
%~~siv
%~~neg
The constructor (\xcd"def this") initializes the fields of the new object.
The \xcd"add" method appends an element to the list.
\xcd"List" is a generic type.  When  instances of \xcd"List" are
allocated, the type \param{} \xcd"T" must be bound to a concrete
type.  \xcd"List[Int]" is the type of lists of element type
\xcd"Int", \xcd"List[List[String]]" is the type of lists whose elements are
themselves lists of string, and so on.

%%BARD-HERE

\section{The sequential core of X10}

The sequential aspects of X10 are mostly familiar from C and its progeny.
\Xten{} enjoys the familiar control flow constructs: \xcd"if" statements,
\xcd"while" loops, \xcd"for" loops, \xcd"switch" statements, \xcd`throw` to
raise exceptions and \xcd`try...catch` to handle them, and so on.

X10 has both implicit coercions and explicit conversions, and both can be
defined on user-defined types.  Explicit conversions are written with the
\xcd`as` operation: \xcd`n as Int`.  The types can be constrained: 
%~~exp~~`~~`~~n:Int~~ ^^^ Overview70
\xcd`n as Int{self != 0}` converts \xcd`n` to a non-zero integer, and throws a
runtime exception if its value as an integer is zero.

\section{Places and activities}

The full power of X10 starts to emerge with concurrency.
An \Xten{} program is intended to run on a wide range of computers,
from uniprocessors to large clusters of parallel processors supporting
millions of concurrent operations. To support this scale, \Xten{}
introduces the central concept of \emph{place} (\Sref{XtenPlaces}).
A place can be thought of as a virtual shared-memory multi-processor:
a computational unit with a finite (though perhaps changing) number of
hardware threads and a bounded amount of shared memory, uniformly
accessible by all threads.



An \Xten{} computation acts on \emph{values}(\Sref{XtenObjects}) through
the execution of lightweight threads called
\emph{activities}(\Sref{XtenActivities}). 
An {\em object}
 has a small, statically fixed set of fields, each of
which has a distinct name. A scalar object is located at a single place and
stays at that place throughout its lifetime. An \emph{aggregate} object has
many fields (the number may be known only when the object is created),
uniformly accessed through an index (\eg, an integer) and may be distributed
across many places. The distribution of an aggregate object remains unchanged
throughout the computation, thought different aggregates may be distributed
differently. Objects are garbage-collected when no longer useable; there are
no operations in the language to allow a programmer to explicitly release
memory.

{}\Xten{} has a \emph{unified} or \emph{global address space}. This means that
an activity can reference objects at other places. However, an activity may
synchronously access data items only in the current place, the place in which
it is running. It may atomically update one or more data items, but only in
the current place.   If it becomes necessary to read or modify an object at
some other place \xcd`q`, the {\em place-shifting} operation \xcd`at(q;F)` can
be used, to move part of the activity to \xcd`q`.  \xcd`F` is a specification
of what information will be sent to \xcd`q` for use by that part of the
computation. 
It is easy to compute
across multiple places, but the expensive operations (\eg, those which require
communication) are readily visible in the code. 

\paragraph{Atomic blocks.}

X10 has a control construct \xcd"atomic S" where \xcd"S" is a statement with
certain restrictions. \xcd`S` will be executed atomically, without
interruption by other activities. This is a common primitive used in
concurrent algorithms, though rarely provided in this degree of generality by
concurrent programming languages.

More powerfully -- and more expensively -- X10 allows conditional atomic
blocks, \xcd`when(B)S`, which are executed atomically at some point when
\xcd`B` is true.  Conditional atomic blocks are one of the strongest
primitives used in concurrent algorithms, and one of the least-often
available. 

\paragraph{Asynchronous activities.}

An asynchronous activity is created by a statement \xcd"async S", which starts
up a new activity running \xcd`S`.  It does not wait for the new activity to
finish; there is a separate statement (\xcd`finish`) to do that.




\section{Clocks}
The MPI style of coordinating the activity of multiple processes with
a single barrier is not suitable for the dynamic network of heterogeneous
activities in an \Xten{} computation.  
X10 allows multiple barriers in a form that supports determinate,
deadlock-free parallel computation, via the \xcd`Clock` type.

A single \xcd`Clock` represents a computation that occurs in phases.
At any given time, an activity is {\em registered} with zero or more clocks.
The X10 statement \xcd`next` tells all of an activity's registered clocks that
the activity has finished the current phase, and causes it to wait for the
next phase.  Other operations allow waiting on a single clock, starting
new clocks or new activities registered on an extant clock, and so on. 

%%INTRO-CLOCK%  Activities may use clocks to repeatedly detect quiescence of arbitrary
%%INTRO-CLOCK%  programmer-specified, data-dependent set of activities. Each activity
%%INTRO-CLOCK%  is spawned with a known set of clocks and may dynamically create new
%%INTRO-CLOCK%  clocks. At any given time an activity is \emph{registered} with zero or
%%INTRO-CLOCK%  more clocks. It may register newly created activities with a clock,
%%INTRO-CLOCK%  un-register itself with a clock, suspend on a clock or require that a
%%INTRO-CLOCK%  statement (possibly involving execution of new async activities) be
%%INTRO-CLOCK%  executed to completion before the clock can advance.  At any given
%%INTRO-CLOCK%  step of the execution a clock is in a given phase. It advances to the
%%INTRO-CLOCK%  next phase only when all its registered activities have \emph{quiesced}
%%INTRO-CLOCK%  (by executing a \xcd"next" operation on the clock).
%%INTRO-CLOCK%  When a clock advances, all its activities may now resume execution.
%%INTRO-CLOCK%  

Clocks act as {barriers} for a dynamically varying collection of activities.
They generalize the barriers found in MPI style program in that an activity
may use multiple clocks simultaneously. Yet programs using clocks properly are
guaranteed not to suffer from deadlock.

%%HERE

\section{Arrays, regions and distributions}

X10 provides \xcd`DistArray`s, {\em distributed arrays}, which spread data
across many places. An underlying \xcd`Dist` object provides the {\em
distribution}, telling which elements of the \xcd`DistArray` go in which
place. \xcd`Dist` uses subsidiary \xcd`Region` objects to abstract over the
shape and even the dimensionality of arrays.
Specialized X10 control statements such as \xcd`ateach` provide efficient
parallel iteration over distributed arrays.


\section{Annotations}

\Xten{} supports annotations on classes and interfaces, methods
and constructors,
variables, types, expressions and statements.
These annotations may be processed by compiler plugins.

\section{Translating MPI programs to \Xten{}}

While \Xten{} permits considerably greater flexibility in writing
distributed programs and data structures than MPI, it is instructive
to examine how to translate MPI programs to \Xten.

Each separate MPI process can be translated into an \Xten{}
place. Async activities may be used to read and write variables
located at different processes. A single clock may be used for barrier
synchronization between multiple MPI processes. \Xten{} collective
operations may be used to implement MPI collective operations.
\Xten{} is more general than MPI in (a)~not requiring synchronization
between two processes in order to enable one to read and write the
other's values, (b)~permitting the use of high-level atomic blocks
within a process to obtain mutual exclusion between multiple
activities running in the same node (c)~permitting the use of multiple
clocks to combine the expression of different physics (e.g.,
computations modeling blood coagulation together with computations
involving the flow of blood), (d)~not requiring an SPMD style of
computation.


%\note{Relaxed exception model}
\section{Summary and future work}
\subsection{Design for scalability}
\Xten{} is designed for scalability, by encouraging working with local data,
and limiting the ability of events at one place to delay those at another. For
example, an activity may atomically access only multiple locations in the
current place. Unconditional atomic blocks are dynamically guaranteed to be
non-blocking, and may be implemented using non-blocking techniques that avoid
mutual exclusion bottlenecks. 
%TODO: yoav says: ``no idea what [the following] means''
Data-flow synchronization permits point-to-point
coordination between reader/writer activities, obviating the need for
barrier-based or lock-based synchronization in many cases.

\subsection{Design for productivity}
\Xten{} is designed for productivity.

\paragraph{Safety and correctness.}

\bard{Confirm some of these claims}

Programs written in \Xten{} are guaranteed to be statically
\emph{type safe}, \emph{memory safe} and \emph{pointer safe}. 

Static type safety guarantees that every location contains only values whose
dynamic type agrees with the location's static type. The compiler allows a
choice of how to handle method calls. In strict mode, method calls are
statically checked to be permitted by the static types of operands. In lax
mode, dynamic checks are inserted when calls may or may not be correct,
providing weaker static correctness guarantees but more programming
convenience. 

Memory safety guarantees that an object may only access memory within its
representation, and other objects it has a reference to. \Xten{} does not
permit 
pointer arithmetic, and bound-checks array accesses dynamically if necessary.
\Xten{} uses garbage collection to collect objects no longer referenced by any
activity. \Xten{} guarantees that no object can retain a reference to an
object whose memory has been reclaimed. Further, \Xten{} guarantees that every
location is initialized at run time before it is read, and every value read
from a word of memory has previously been written into that word.

%XXX
%Pointer safety guarantees that a null pointer exception cannot be
%thrown by an operation on a value of a non-nullable type.

Because places are reflected in the type system, static type safety
also implies \emph{place safety}. All operations that need to be performed
locally are, in fact, performed locally.  All data which is declared to be
stored locally are, in fact, stored locally.

\Xten{} programs that use only clocks and unconditional atomic
blocks are guaranteed not to deadlock. Unconditional atomic blocks
are non-blocking, hence cannot introduce deadlocks.
Many concurrent programs can be shown to be determinate (hence
race-free) statically.

\paragraph{Integration.}
A key issue for any new programming language is how well it can be
integrated with existing (external) languages, system environments,
libraries and tools.

%TODO: Yoav asks ``you mean interop''?
We believe that \Xten{}, like \java{}, will be able to support a large
number of libraries and tools. An area where we expect future versions
of \Xten{} to improve on \java{} like languages is \emph{native
integration} (\Sref{NativeCode}). Specifically, \Xten{} will permit
permit multi-dimensional local arrays to be operated on natively by
native code.

\subsection{Conclusion}
{}\Xten{} is considerably higher-level than thread-based languages in
that it supports dynamically spawning lightweight activities, the
use of atomic operations for mutual exclusion, and the use of clocks
for repeated quiescence detection.

Yet it is much more concrete than languages like HPF in that it forces
the programmer to explicitly deal with distribution of data
objects. In this the language reflects the designers' belief that
issues of locality and distribution cannot be hidden from the
programmer of high-performance code in high-end computing.  A
performance model that distinguishes between computation and
communication must be made explicit and transparent.\footnote{In this
\Xten{} is similar to more modern languages such as ZPL \cite{zpl}.}
At the same time we believe that the place-based type system and
support for generic programming will allow the \Xten{} programmer to
be highly productive; many of the tedious details of
distribution-specific code can be handled in a generic fashion.

\chapter{Lexical structure}


Lexically a program consists of a stream of white space, comments,
identifiers, keywords, literals, separators and operators, all of them
composed of ASCII characters. 

\paragraph{Whitespace}
\index{white space}
% Whitespace \index{whitespace} follows \java{} rules \cite[Chapter 3.6]{jls2}.
ASCII space, horizontal tab (HT), form feed (FF) and line
terminators constitute white space.

\paragraph{Comments}
\index{comment}
% Comments \index{comments} follows \java{} rules
% \cite[Chapter 3.7]{jls2}. 
All text included within the ASCII characters ``\xcd"/*"'' and
``\xcd"*/"'' is
considered a comment and ignored; nested comments are not
allowed.  All text from the ASCII characters
``\xcd"//"'' to the end of line is considered a comment and is ignored.

\paragraph{Identifiers}
\index{identifier}
\index{variable name}

Identifiers consist of a single letter followed by zero or more
letters or digits.
The letters are the ASCII characters \xcd`a` through \xcd`z`, \xcd`A` through
\xcd`Z`, and \xcd`_`.
Digits are defined as the ASCII characters \xcd"0" through \xcd"9". Case is
significant; \xcd`a` and \xcd`A` are distinct identifiers, \xcd`as` is a
keyword, but \xcd`As` and \xcd`AS` are identifiers.

\paragraph{Keywords}
\index{keywords}
\Xten{} reserves the following keywords:
\begin{xten}
abstract       false          offers         transient      
as             final          operator       true           
assert         finally        package        try            
async          finish         private        var            
ateach         for            property       when           
break          goto           protected      while          
case           if             public         at             
catch          implements     return         atomic         
class          import         self           await          
continue       in             static         clocked        
def            instanceof     struct         here           
default        interface      super          next           
do             native         switch         offer          
else           new            this           resume         
extends        null           throw          type           
\end{xten}
Note that the primitive types are not considered keywords.

\paragraph{Literals}\label{Literals}\index{literal}

Briefly, \XtenCurrVer{} uses fairly standard syntax for its literals:
integers, unsigned integers, floating point numbers, booleans, 
characters, strings, and \xcd"null".  The most exotic points are (1) unsigned
numbers are marked by a \xcd`u` and cannot have a sign; (2) \xcd`true` and
\xcd`false` are the literals for the booleans; and (3) floating point numbers
are \xcd`Double` unless marked with an \xcd`f` for \xcd`Float`. 

Less briefly, we use the following abbreviations: 
\begin{displaymath}
\begin{array}{rcll}
d &=& \mbox{one or more decimal digits}\\
d_8 &=& \mbox{one or more octal digits}\\
d_{16} &=& \mbox{one or more hexadecimal digits, using \xcd`a`-\xcd`f`
for 10-15}\\
i &=& d 
        \mathbin{|} {\tt 0} d_8 
        \mathbin{|} {\tt 0x} d_{16}
        \mathbin{|} {\tt 0X} d_{16}
\\
s &=& \mbox{optional \xcd`+` or \xcd`-`}\\
b &=& d 
          \mathbin{|} d {\tt .}
          \mathbin{|} d {\tt .} d
          \mathbin{|}  {\tt .} d \\
x &=& ({\tt e } \mathbin{|} {\tt E})
         s
         d \\
f &=& b x
\end{array}
\end{displaymath}

\begin{itemize}

\item \xcd`true` and \xcd`false` are the \xcd`Boolean` literals. \index{Boolean!literal}\index{literal!Boolean}

\item \xcd`null` is a literal for the null value.  It has type
      \xcd`Any{self==null}`. \index{null} \index{object!literal}

\item \index{Int!literal}\index{literal!integer}
\xcd`Int` literals have the form {$si$}; \eg, \xcd`123`,
      \xcd`-321` are decimal \xcd`Int`s, \xcd`0123` and \xcd`-0321` are octal
      \xcd`Int`s, and \xcd`0x123`, \xcd`-0X321`,  \xcd`0xBED`, and \xcd`0XEBEC` are
      hexadecimal \xcd`Int`s.  

\item \xcd`Long` literals have the form {$si{\tt l}$} or
      {$si{\tt L}$}. \Eg, \xcd`1234567890L`  and \xcd`0xBAGEL` are \xcd`Long` literals. 

\item \xcd`UInt` literals have the form {$i{\tt u}$} or {$i {\tt U}$}.
      \Eg, \xcd`123u`, \xcd`0123u`, and \xcd`0xBEAU` are \xcd`UInt` literals.

\item \xcd`ULong` literals have the form {$i {\tt ul}$} or {$i {\tt
      lu}$}, or capital versions of those.  For example, 
      \xcd`123ul`, \xcd`0124567012ul`,  \xcd`0xFLU`, \xcd`OXba1eful`, and \xcd`0xDecafC0ffeefUL` are \xcd`ULong`
      literals. 

\item \xcd`Short` literals have the form {$si{\tt s}$} or
      {$si{\tt S}$}. \Eg,  414S, \xcd`OxACES` and \xcd`7001s` are short
      literals. 

\item \xcd`UShort` literals  form {$i {\tt us}$} or {$i {\tt
      su}$}, or capital versions of those.  For example, \xcd`609US`, 
      \xcd`107us`, and \xcd`OxBeaus` are unsigned short literals.

\item \xcd`Byte` literals have the form  {$si{\tt y}$} or
      {$si{\tt Y}$}.  (The letter \xcd`B` cannot be used for bytes, as it is
      a hexadecimal digit.)  \xcd`50Y` and \xcd`OxBABY` are byte literals.

\item \xcd`UByte` literals have the form {$i {\tt uy}$} or {$i {\tt yu}$}, or
      capitalized versions of those.  For example, \xcd`9uy` and \xcd`OxBUY`
      are \xcd`UByte` literals.
      


\item \xcd`Float` literals have the form {$s f {\tt f}$} or  {$s
\index{float!literal}
\index{literal!float}
      f {\tt F}$}.  Note that the floating-point marker letter \xcd`f` is
      required: unmarked floating-point-looking literals are \xcd`Double`. 
      \Eg, \xcd`1f`, \xcd`6.023E+32f`, \xcd`6.626068E-34F` are \xcd`Float`
      literals. 

\item \xcd`Double` literals have the form {$s f$}\footnote{Except that
\index{double!literal}
\index{literal!double}
      literals like \xcd`1` 
      which match both {$i$} and {$f$} are counted as
      integers, not \xcd`Double`; \xcd`Double`s require a decimal
      point, an exponent, or the \xcd`d` marker.
      }, {$s f {\tt
      D}$}, and {$s f {\tt d}$}.  
      \Eg, \xcd`0.0`, \xcd`0e100`, \xcd`1.3D`,  \xcd`229792458d`, and \xcd`314159265e-8`
      are \xcd`Double` literals.

\item 
\index{char!literal}
\index{literal!char}
\xcd`Char` literals have one of the following forms: 
      \begin{itemize}
      \item \xcd`'`{\it c}\xcd`'` where {\em c} is any printing ASCII
            character other than 
            \xcd`\` or \xcd`'`, representing the character {\em c} itself; 
            \eg, \xcd`'!'`;
      \item \xcd`'\b'`, representing backspace;
      \item \xcd`'\t'`, representing tab;
      \item \xcd`'\n'`, representing newline;
      \item \xcd`'\f'`, representing form feed;
      \item \xcd`'\r'`, representing return;
      \item \xcd`'\''`, representing single-quote;
      \item \xcd`'\"'`, representing double-quote;
      \item \xcd`'\\'`, representing backslash;
      \item \xcd`'\`{\em dd}\xcd`'`, where {\em dd} is one or more octal
            digits, representing the one-byte character numbered {\em dd}; it
            is an error if {\em dd}{$>0377$}.      
      \end{itemize}

\item
\index{string!literal} 
\index{literal!string}
\xcd`String` literals consist of a double-quote \xcd`"`, followed by
      zero or more of the contents of a \xcd`Char` literal, followed by
      another double quote.  \Eg, \xcd`"hi!"`, \xcd`""`.


\end{itemize}



\paragraph{Separators}
\Xten{} has the following separators and delimiters:
\begin{xten}
( )  { }  [ ]  ;  ,  .
\end{xten}

\paragraph{Operators}
\index{operator}
\Xten{} has the following operator,  type constructor, and miscellaneous symbols.  (\xcd`?` and
\xcd`:` comprise a single ternary operator, but are written separately.)
\begin{xten}
==  !=  <   >   <=  >=
&&  ||  &   |   ^
<<  >>  >>>
+   -   *   /   %
++  --  !   ~
&=  |=  ^=
<<= >>= >>>=
+=  -=  *=  /=  %=
=   ?   :  =>  ->
<:  :>  @   ..
\end{xten}





\chapter{Types}
\label{XtenTypes}\index{types}

{}\Xten{} is a {\em strongly typed} object-oriented language: every
variable and expression has a type that is known at compile-time.
Types limit the values that variables can hold.

{}\Xten{} supports three kinds of runtime entities, {\em objects},
{\em structs}, and {\em functions}. Objects are instances of {\em
  classes} (\Sref{ReferenceClasses}). They may contain zero or
more mutable fields, and a reference to the list of methods defined on them.

An object is represented by some (contiguous) memory chunk on the
heap. Entities (such as variables and fields) contain a {\em
  reference} to this chunk. That is, objects are represented through
an extra level of indirection.  A consequence of this flexibility is
that an entity containing a reference to an object \xcd{o} needs only
one word of memory to represent that reference, regardess of the
number of fields in \xcd{o}. An assignment to this entity simply
overwrites the reference with another reference (thus taking constant
time). Another consequence is that every class type contains the value
\Xcd{null} corresponding to the invalid reference. \Xcd{null} is often
useful as a default value. Further, two objects may be compared for
identity (\Xcd{==}) in constant time by simply comparing references to
the memory used to represent the objects. The default hash code for an
object is based on the value of this reference. A downside of this
flexibility is that the operations of accessing a field and invoking a
method are more expensive than simply reading a register and
invoking a static function.


Structs are instances of {\em struct types} (\Sref{StructClasses}).  A
struct is represented without the extra level of indirection, with a
memory chunk of size $N$ words precisely big enough to store the value
of every field of the struct (modulo alignment), plus whatever padding is needed. Thus structs cannot
be shared. Entities (such as variables and fields) refering to the
struct must allocate $N$ words to directly contain the chunk.  An
assignment to this entity must copy the $N$ words representing the
right hand side into the left hand side. Since there are no references
to structs, \Xcd{null} is not a legal value for a struct
type. Comparison for identity (\Xcd{==}) involves examining $N$
words. Additionally, structs do not have any mutable fields, hence
they can be freely copied. The payoff for these restrictions lies in
that fields can be stored in registers or local variables, and 
and method invocation is implemented by invoking a static function.

Functions, called closures, lambda-expressions, and blocks in other languages, are
instances of {\em function types} (\Sref{Functions}). 
A function has zero or more {\em formal parameters} (or {\em arguments}) and a
{\em body}, which is 
an expression that can reference the formal parameters and also other
variables in the surrounding block. For instance, \xcd`(x:Int)=>x*y`
is a unary integer function which multiplies its argument by the
variable \xcd`y` from the surrounding block.  Functions may be freely
copied from place to place and may be repeatedly applied. 

These runtime entities are classified by {\em types}. Types are used in
variable declarations, coercions and  explicit conversions, object creation,
array creation, static state and method accessors, and
\xcd"instanceof" and \xcd`as` expressions.


The basic relationship between values and types is the {\em is an
element of} relation.  We also often say ``$e$ has type $T$'' to
mean ``$e$ is an element of type $T$''.  For example, \xcd`1` has type
\xcd`Int` (the type of all integers representible in 32 bits). It also
has type \xcd`Any` (since all entitites have type \xcd`Any`), type
\xcd`Int{self != 0}` (the type of nonzero integers), type
\xcd`Int{self == 1}` (the type of integers which are equal to \xcd`1`, which
contains only one element), and many others. 

The basic relationship between types is {\em subtyping}: \xcd`T <: U`
holds if every instance of \xcd`T` is also an instance of \xcd`U`. Two
important kinds of subtyping are {\em subclassing} and {\em
  strengthening}. Subclassing is a familiar notion from
object-oriented programming. Here we use it to refer to the
relationship between a class and another class it extends, and the
relationship between a class and another interface it implements. For
instance, in a class hierarchy with classes \xcd`Animal` and \xcd`Cat`
such that \xcd`Cat` extends \xcd`Mammal` and \xcd`Mammal` extends
\xcd`Animal`, every instance of \xcd`Cat` is by definition an instance
of \xcd`Animal` (and \xcd`Mammal`). We say that \xcd`Cat` is a
subclass of \xcd`Animal`, or \xcd`Cat <: Animal` by subclassing. If
\xcd`Animal` implements \xcd`Thing`, then \xcd`Cat` also implements
\xcd`Thing`, and we say \xcd`Cat <: Thing` by subclassing.
Strengthening is an equally familiar notion from logic.  The instances
of \xcd`Int{self == 1}` are all elements of \xcd`Int{self != 0}` as well,
because \xcd`self == 1` logically implies \xcd`self != 0`; so 
\xcd`Int{self  == 1} <: Int{self !=0}` by strengthening.  X10 uses both notions
of subtyping.  See \Sref{DepType:Equivalence} for the full definition
of subtyping in X10.

\subsection{Type System}
\index{type system}
The types in X10 are as follows.  

These are the {\em elementary} types. Other
syntactic forms for types exist, but they are simply abbreviations for types
in the following system.  For example, \xcd`Array[Int](1)` is the type of
one-dimensional integer-valued arrays; it is an abbreviation for
\xcd`Array[Int]{rank==1}`.\\

% remove \refstepcounter{equation}
% snag the argument of \label{X}
% change the (\arabic{equation}) into (\ref{X})

%##(Type FunctionType ConstrainedType
\begin{bbgrammar}
%(FROM #(prod:Type)#)
                Type \: FunctionType & (\ref{prod:Type}) \\
                     \| ConstrainedType \\
                     \| VoidType \\
%(FROM #(prod:FunctionType)#)
        FunctionType \: TypeParams\opt \xcd"(" FormalList\opt \xcd")" Guard\opt Offers\opt \xcd"=>" Type & (\ref{prod:FunctionType}) \\
%(FROM #(prod:ConstrainedType)#)
     ConstrainedType \: NamedType & (\ref{prod:ConstrainedType}) \\
                     \| AnnotatedType \\
                     \| \xcd"(" Type \xcd")" \\
\end{bbgrammar}
%##)


Types may be given by name. 
For example, 
%~~type~~`~~`~~ ~~ ^^^ Types10
\xcd`Int`
is the type of 32-bit integers.
Given a class declaration 
%~~gen ^^^ Types20
%package Types.Core.TypeName; 
%~~vis
\begin{xten}
class Triple { /* ... */ }
\end{xten}
%~~siv
%
%~~neg
the identifier \xcd`Triple` may be used as a type.

The type {\em TypeName \xcd`[` Types{$^?$} \xcd`]`} is an instance of
a {\em generic} (or {\em parameterized}) type. 
 For example,
\xcd`Array[Int]` is the type of arrays of integers. 
\xcd`HashMap[String,Int]` is the type of hash maps from strings to
integers.

The type {\em Type \xcd`{` Constraint \xcd`}`} refers to a constrained type.
{\em Constraint} is a Boolean expression -- written in a {\em very} limited
subset of X10 -- describing the acceptable values of the constrained type.
%~~stmt~~`~~`~~ ~~ ^^^ Types30
For example, \xcd`var n : Int{self != 0};` guarantees that \xcd`n` is always a
non-zero integer. 
%~~stmt~~`~~`~~ ~~class Triple{} ^^^ Types40
Similarly, \xcd`var x : Triple{x != null};` defines a \xcd`Triple`-valued
variable \xcd`x` whose value is never null.

The qualified type {\em Type \xcd`.` Type} refers to an instance of a {\em
nested} type; that is, a class or struct defined inside of another class or
struct, and holding an implicit reference to the outer.  For example, given
the type declaration 
%~~gen ^^^ Types50
% package Types.Core.Hardcore.Qualified;
%~~vis
\begin{xten}
class Outer {
  class Inner { /* ... */ }
}
\end{xten}
%~~siv
%
%~~neg
then 
%~~exp~~`~~`~~ ~~ NOTEST class Outer {class Inner { /* ... */ }} ^^^ Types60
\xcd`(new Outer()).new Inner()` creates a value of type 
%~~type~~`~~`~~ ~~class Outer {class Inner { /* ... */ }} ^^^ Types70
\xcd`Outer.Inner`.

Type variables, {\em TypeVar}, refer to types that are parameters.  For
example, the following class defines a cell in a linked list.  
%~~gen ^^^ Types80
% package Types.Core.Bore.Lore;
%~~vis
\begin{xten}
class LinkedList[X] {
  val head : X;
  val tail : LinkedList[X];
  def this(head:X, tail:LinkedList[X]) {
     this.head = head; this.tail = tail;
  }
}
\end{xten}
%~~siv
%
%~~neg
It doesn't
matter what type the cell's element is, but it has to have {\em some} type.
\xcd`LinkedList[Int]` is a linked list of integers.
\xcd`LinkedList[LinkedList[Byte]]` is a list of lists of bytes.
Note that \xcd`LinkedList` is {\em not} a usable type -- it is missing a type parameter.



The function type 
{\em \xcd`(` Formals{$^?$} \xcd`) =>`  Type} 
refers to functions taking the
listed formal parameters and returning a result of {\em Type}.  In
\XtenCurrVer, function types may not be generic.
The closely-related void function type 
{\em \xcd`(` Formals{$^?$} \xcd`) =>`  \xcd`void`}  takes the listed
parameters and returns no value.
For example, 
\begin{xtenmath}
\xcd`(x:Int) => Int{self != x}` 
\end{xtenmath}
is the type of integer-valued functions which have no fixed points -- that is,
for which the output is an integer different from the input.
An example of such a function is \xcd`(x:Int) => x+1`.
For fundamental reasons, X10 --- or any other computer program --- cannot
tell in general whether a function has any fixed points or not.  So, X10
programs using such types must prove to X10 that they are correct. Often this
will involve a run-time check, expressed as a cast, such as: 
%~~gen ^^^ Types3x7m
% package Types3x7m;
% class Example {
%~~vis
\begin{xten}
  val plus1 : (x:Int) => Int{self != x} = 
     (x:Int) => (x+1) as Int{self != x}; 
\end{xten}
%~~siv
%}
%~~neg


The names of the formal parameters are bound in the type, and may be changed
consistently in the usual way without modifying the type.  
For example, \\
\xcd`(a:Int, b:Int{self!=a})=>Int{self!=a, self!=b}` 
and \\
\xcd`(c:Int, d:Int{self!=c})=>Int{self!=c, self!=d}` \\
are equivalent types.  



\section{Classes, Structs,  and interfaces}
\label{ReferenceTypes}

\subsection{Class types}

\index{type!class}
\index{class}
\index{class declaration}
\index{declaration!class declaration}
\index{declaration!reference class declaration}

A {\em class declaration} (\Sref{XtenClasses}) declares a {\em class type},
giving its name, behavior, data, and relationships to other classes and
interfaces. 

\begin{ex}
The \xcd`Position` class below could describe the position of a slider
control: 
%~~gen ^^^ Types100
% package Types.By.Cripes.Classes;
%~~vis
\begin{xten}
class Position {
  private var x : Int = 0;
  public def move(dx:Int) { x += dx; }
  public def pos() : Int = x;
}
\end{xten}
%~~siv
%
%~~neg
\end{ex}
Class instances, also called objects, are created by constructor calls, 
such as \xcd`new Position()`
Class
instances have fields and methods, type members, and value properties bound at
construction time. In addition, classes have static members: static \xcd`val` fields,
methods, type definitions, and member classes and member interfaces.

Classes may be {\em generic}, \ie, defined with one or more type
parameters (\Sref{TypeParameters}).  

%~~gen ^^^ Types110
%~~vis
\begin{xten}
class Cell[T] {
  var contents : T;
  public def this(t:T) { contents = t;  }
  public def putIn(t:T) { contents = t; }
  public def get() = contents;
  }
\end{xten}
%~~siv
%~~neg


%TODO: Yoav: ``This reasoning is no longer true in the new object model''
%% Why not?
\Xten{} does not permit mutable static state. A fundamental principle of the
X10 model of computation is that all mutable state be local to some place
(\Sref{XtenPlaces}), and, as static variables are
globally available, they
cannot be mutable. When mutable global state is necessary, programmers should
use singleton classes, putting the state in an object and using place-shifting
commands (\Sref{AtStatement}) and atomicity (\Sref{AtomicBlocks}) as necessary
to mutate it safely.

\index{\Xcd{Object}}
\index{\Xcd{x10.lang.Object}}

Classes are structured in a single-inheritance hierarchy. All classes extend
the class \xcd"x10.lang.Object", directly or indirectly. Each class other than
\xcd`Object` extends a single parent class.  \xcd`Object` provides no behaviors
of its own, beyond those required by \xcd`Any`.

\index{class!reference class}
\index{reference class type}
\index{\Xcd{Object}}
\index{\Xcd{x10.lang.Object}}


\index{null}


The null value, represented by the literal
\xcd"null", is a value of every class type \xcd`C`. The type whose values are
all instances of \xcd`C` except 
\xcd`null` can be defined as \xcd`C{self != null}`.

\subsection{Struct Types}

A {\em struct declaration} \Sref{XtenStructs} introduces a {\em struct type}
containing all instances of the struct.  The \xcd`Coords` struct below gives
an immutable position in 3-space: 
%~~gen ^^^ Types120
% package Types.Structs.Coords;
%~~vis
\begin{xten}
struct Position {
  public val x:Double, y:Double, z:Double; 
  def this(x:Double, y:Double, z:Double) {
     this.x = x; this.y = y; this.z = z;
  }
}
\end{xten}
%~~siv
%
%~~neg

Structs have many capabilities of classes: they can have methods, implement
interfaces, and be generic. However, they have certain restrictions; for
example, they cannot contain mutable (\xcd`var`) fields, or inherit from
superclasses. There is no \xcd`null` value for structs. Due to these
restrictions, structs may be implemented more efficiently than objects.


\subsection{Interface types}
\label{InterfaceTypes}

\index{type!interface}
\index{interface}
\index{interface declaration}
\index{declaration!interface declaration}

An {\em interface declaration} (\Sref{XtenInterfaces}) defines an {\em
interface type}, specifying a set of methods 
%type members, 
and properties which must be provided by any class declared to implement the
interface. 


Interfaces can also have static members: static fields, type
definitions, and member classes, structs and interfaces.  However,
interfaces cannot specify that implementing classes must provide
static members or constructors.

\begin{ex}
In the following interface, \xcd`PI` is a static field, 
\xcd`Vec` a static type definition, 
\xcd`Pair` a static member class.
It can't insist that implementations provide a static method 
like \xcd`meth`, or a nullary constructor.
%~~gen ^^^ Types2y3i
% NOTEST
% package Types2y3i;
%~~vis
\begin{xten}
interface Stat {
  static val PI = 3.14159; 
  static type R = Double;
  static class Pair(x:R, y:R) {}
  // ERROR: static def meth():Int;
  // ERROR: static def this();
}
class Example {
  static def example() {
     val p : Stat.Pair = new Stat.Pair(Stat.PI, Stat.PI);
  }
}
\end{xten}
%~~siv
%
%~~neg

\end{ex}

An interface may extend multiple interfaces.  
%~~gen ^^^ Types130
%package Types.For.Snipes.Interfaces;
%~~vis
\begin{xten}
interface Named {
  def name():String;
}
interface Mobile {
  def move(howFar:Int):void;
}
interface Person extends Named, Mobile {}
interface NamedPoint extends Named, Mobile {} 
\end{xten}
%~~siv
%
%~~neg


Classes and structs may be declared to implement multiple interfaces. Semantically, the
interface type is the set of all objects that are instances of classes
or structs that
implement the interface. A class or struct implements an interface if it is declared to
and if it concretely or abstractly implements all the methods and properties
defined in the interface. For example, \xcd`Kim` implements
\xcd`Person`, and hence \xcd`Named` and \xcd`Mobile`. It would be a static
error if \xcd`Kim` had no \xcd`name` method, unless \xcd`Kim` were also
declared \xcd`abstract`.

%~~gen ^^^ Types140
%interface Named {
%   def name():String;
% }
% interface Mobile {
%   def move(howFar:Int):void;
% }
% interface Person extends Named, Mobile {}
% interface NamedPoint extends Named, Mobile{} 
%~~vis
\begin{xten}
class Kim implements Person {
   var pos : Int = 0;
   public def name() = "Kim (" + pos + ")";
   public def move(dPos:Int) { pos += dPos; }
}
\end{xten}
%~~siv
%
%~~neg


\subsection{Properties}
\index{property}
\label{properties}

Classes, interfaces, and structs may have {\em properties}, specified in
parentheses after the type name. Properties are much like public \xcd`val`
instance fields. They have certain restrictions on their use, however, which
allows the compiler to understand them much better than other public \xcd`val`
fields. In particular, they can be used in types.  \Eg, the number of elements
in an array is a property of the array, and an X10 program can specify that
two arrays have the same number of elements.

\begin{ex}
The
following code declares a class named \xcd"Coords" with properties
\xcd"x" and \xcd"y" and a \xcd"move" method. The properties are bound
using the \xcd"property" statement in the constructor.

%~~gen ^^^ Types150
%package not.x10.lang;
%~~vis
\begin{xten}
class Coords(x: Int, y: Int) { 
  def this(x: Int, y: Int) :
    Coords{self.x==x, self.y==y} = { 
    property(x, y); 
  } 

  def move(dx: Int, dy: Int) = new Coords(x+dx, y+dy); 
}
\end{xten}
%~~siv
%~~neg
\end{ex}
Properties, unlike other public \xcd`val` fields, can be used  
at compile time in {constraints}. This allows us
to specify subtypes based on properties, by appending a boolean expression to
the type. For example, the type \xcd"Coords{x==0}" is the set of all points
whose \xcd"x" property is \xcd"0".  Details of this substantial topic are
found in \Sref{ConstrainedTypes}.



\section{Type Parameters and Generic Types}
\label{TypeParameters}

\index{type!parameter}
\index{method!parametrized}
\index{constructor!parametrized}
\index{closure!parametrized}
\label{Generics}
\index{type!generic}

A class, interface, method, or type definition  may have type
parameters.  Type parameters can be used as types, and will be bound to types
on instantiation.  
For example, a generic stack class may be defined as 
\xcd`Stack[T]{...}`.  Stacks can hold values of any type; \eg, 
%~~type~~`~~`~~ ~~class Stack[T]{} ^^^ Types160
\xcd`Stack[Int]` is a stack of integers, and 
%~~type~~`~~`~~ ~~class Stack[T]{} ^^^ Types170
\xcd`Stack[Point {self!=null}]` is a stack of non-null \xcd`Point`s.
Generics {\em must} be instantiated when they are used: \xcd`Stack`, by
itself, is not a valid type.
Type parameters may be constrained by a guard on the declaration
(\Sref{TypeDefGuard},
\Sref{MethodGuard},\Sref{ClosureGuard}).

\index{type!concrete}
\index{concrete type}
A {\em generic class} (or struct, interface, or type definition) 
is a class (resp. struct, interface, or type definition) 
declared with $k \geq 1$ type parameters. 
A generic class (or struct, interface, or type definition) 
can be used to form a type by supplying $k$ types as type arguments within
\xcd`[` \ldots \xcd`]`.
%%When instantiated,
%%with concrete (\viz, non-generic) types for its parameters, 
%%a generic type becomes a concrete type and can be
%%used like any other type. 
For example,
\xcd`Stack` is a generic class, 
%~~type~~`~~`~~ ~~class Stack[T]{} ^^^ Types180
\xcd`Stack[Int]` is a type, and can be used as one: 
%~~stmt~~`~~`~~ ~~class Stack[T]{} ^^^ Types190
\xcd`var stack : Stack[Int];`

\begin{ex}A \xcd`Cell[T]` is a generic object, capable of holding a value of type
\xcd`T`.  For example, a \xcd`Cell[Int]` can hold an \xcd`Int`, and a
\xcd`Cell[Cell[Int{self!=0}]]` can hold a \xcd`Cell` which in turn can
only hold non-zero numbers. 
%% vj: Dont know what this saying: bound immutably... but mutable?
%% \xcd`Cell`s are actually useful in situations
%%where values must be bound immutably for one reason, but need to be mutable.
%~~gen ^^^ Types200
% package ch4;
%~~vis
\begin{xten}
class Cell[T] {
    var x: T;
    def this(x: T) { this.x = x; }
    def get(): T = x;
    def set(x: T) = { this.x = x; }
}
\end{xten}
%~~siv
%~~neg


\xcd"Cell[Int]" is the type of \xcd`Int`-holding cells.  
The \xcd"get" method on a \xcd`Cell[Int]` returns an \xcd"Int"; the
\xcd"set" method takes an \xcd"Int" as argument.  Note that
\xcd"Cell" alone is not a legal type because the parameter is
not bound.
\end{ex}

A class (whether generic or not) may have generic methods.

\begin{ex}
\xcd`NonGeneric` has a generic method 
\xcd`first[T](x:List[T])`. An invocation of such a method may supply
the type parameters explicitly (\eg, \xcd`first[Int](z)`).
 In certain cases (\eg, \xcd`first(z)`)
type parameters may
be omitted and are inferred by the compiler (\Sref{TypeInference}).

%~~gen ^^^ Types210
% package Types.For.Cripes.Sake.Generic.Methods;
% import x10.util.*;
%~~vis
\begin{xten}
class NonGeneric {
  static def first[T](x:List[T]):T = x(0);
  def m(z:List[Int]) {
    val f = first[Int](z);
    val g = first(z);
    return f == g;
  }
}
\end{xten}
%~~siv
%
%~~neg


\end{ex}


\limitation{ \XtenCurrVer{}'s C++ back end requires generic methods to be
static or final; the Java back end can accomodate generic instance methods as well. }

Unlike other kinds of variables, type parameters may {\em not} be shadowed.  
If name \xcd`X` is in scope as a type, \xcd`X` may not be rebound as a type
variable.  

\begin{ex}
Neither \xcd`class B` nor \xcd`class C[B]` are allowed in the
following code, because they both shadow the type variable \xcd`B`.
%~~gen ^^^ TypesNoShadow
% package TypesNoShadow;
% KNOWNFAIL-https://jira.codehaus.org/browse/XTENLANG-2621
%~~vis
\begin{xten}
class A[B] {
  //ERROR: class B{} 
  //ERROR: class C[B]{} 
}
\end{xten}
%~~siv
%
%~~neg
\end{ex}

\subsection{Use of Generics}

An unconstrained type variable \Xcd{X} can be instantiated any type. Within a
generic struct or class, all the operations of \Xcd{Any} are available on a
variable of type unconstrained \Xcd{X}. Additionally, variables of type
\Xcd{X} may be used with \Xcd{==, !=}, in \Xcd{instanceof}, and casts.  

If a type variable is constrained, the operations implied by its constraint
are available as well.

\begin{ex}
The interface \xcd`Named` describes entities which know their own name.  The
class \xcd`NameMap[T]` is a specialized map which stores and retrieves
\xcd`Named` entities by name.  The call \xcd`t.name()` in \xcd`put()` is only
valid because the constraint \xcd`{T <: Named}` implies that \xcd`T` is a
subtype of \xcd`Named`, and hence provides all the operations of \xcd`Named`. 
%~~gen ^^^ Types6e6x
% package Types6e6x;
% import x10.util.*;
%~~vis
\begin{xten}
interface Named { def name():String; }
class NameMap[T]{T <: Named} {
   val m = new HashMap[String, T]();
   def put(t:T) { m.put(t.name(), t); }
   def get(s:String):T = m.getOrThrow(s);
}
\end{xten}
%~~siv
%
%~~neg


\end{ex}


%%NO-VARIANCE%% \subsection{Variance of Type Parameters}
%%NO-VARIANCE%% \index{covariant}
%%NO-VARIANCE%% \index{contravariant}
%%NO-VARIANCE%% \index{invariant}
%%NO-VARIANCE%% \index{type parameter!covariant}
%%NO-VARIANCE%% \index{type parameter!contravariant}
%%NO-VARIANCE%% \index{type parameter!invariant}
%%NO-VARIANCE%% 
%%NO-VARIANCE%% % Uncomment this when the language implementation properly supports variance.
%%NO-VARIANCE%% %\input{Variance}
%%NO-VARIANCE%% 
%%NO-VARIANCE%% Class, struct and interface definitions are permitted to specify a {\em
%%NO-VARIANCE%%   variance} 
%%NO-VARIANCE%% for each type parameter. 
%%NO-VARIANCE%% There are three variance specifications: 
%%NO-VARIANCE%% \xcd`+` indicates {\em co-variance},  \xcd`-` indicates {\em
%%NO-VARIANCE%%   contravariance} and the absence of  \xcd`+` and 
%%NO-VARIANCE%%  \xcd`-` indicates {\em invariance}. For a class (or struct or
%%NO-VARIANCE%%  interface) \xcd`S` specifying that a particular parameter position
%%NO-VARIANCE%%  (say, \xcd`i`) is covariant means that 
%%NO-VARIANCE%% if \xcd`Si <: Ti` then
%%NO-VARIANCE%% \xcdmath"S[S1,$\ldots$,Sn] <: S[S1,$\ldots$, Si-1,Ti,Si+1,$\ldots$ Sn]".
%%NO-VARIANCE%% Similarly, saying that position \xcd`i` is is contravariant means
%%NO-VARIANCE%% that 
%%NO-VARIANCE%% if \xcd`Si <: Ti` then
%%NO-VARIANCE%% \xcdmath"S[S1,$\ldots$, Si-1,Ti,Si+1,$\ldots$ Sn] <: S[S1,$\ldots$,Sn]". If the
%%NO-VARIANCE%% position is invariant, then no such relationship is asserted between
%%NO-VARIANCE%% \xcd`Si <: Ti` 
%%NO-VARIANCE%% and
%%NO-VARIANCE%% \xcdmath"S[S1,$\ldots$, Si-1,Ti,Si+1,$\ldots$ Sn]". The compiler must perform
%%NO-VARIANCE%% several checks on the body of the class (or struct or interface) to
%%NO-VARIANCE%% establish that type parameters with a variance are used in a manner
%%NO-VARIANCE%% that is consistent with their semantics.
%%NO-VARIANCE%% 
%%NO-VARIANCE%% \limitation{} The implementation of variance specifications  suffers from
%%NO-VARIANCE%% various limitations in \XtenCurrVer. Users are strongly encouraged not
%%NO-VARIANCE%% to use variance. (Some classes, structs, and interfaces in the standard
%%NO-VARIANCE%% libraries use variance specifications in a careful way that
%%NO-VARIANCE%% circumvents these limitations.)
%%NO-VARIANCE%% 

\section{Type definitions}
\label{TypeDefs}

\index{type!definitions}
\index{declaration!type}
A type definition can be thought of as a type-valued function,
mapping type parameters and value parameters to a concrete type.

%##(TypeDefDecl TypeParams Formals Guard
\begin{bbgrammar}
%(FROM #(prod:TypeDefDecl)#)
         TypeDefDecl \: Mods\opt \xcd"type" Id TypeParams\opt Guard\opt \xcd"=" Type \xcd";" & (\ref{prod:TypeDefDecl}) \\
                     \| Mods\opt \xcd"type" Id TypeParams\opt \xcd"(" FormalList \xcd")" Guard\opt \xcd"=" Type \xcd";" \\
%(FROM #(prod:TypeParams)#)
          TypeParams \: \xcd"[" TypeParamList \xcd"]" & (\ref{prod:TypeParams}) \\
%(FROM #(prod:Formals)#)
             Formals \: \xcd"(" FormalList\opt \xcd")" & (\ref{prod:Formals}) \\
%(FROM #(prod:Guard)#)
               Guard \: DepParams & (\ref{prod:Guard}) \\
\end{bbgrammar}
%##)

\noindent 
During type-checking the compiler replaces the use of such a defined
type with its body, substituting the actual type and value parameters
in the call for the formals. This replacement is performed recursively
until the type no longer contains a defined type or a predetermined
compiler limit is reached (in which case the compiler declares an
error). Thus, recursive type definitions are not permitted.

Thus type definitions are considered applicative and not generative --
they do not define new types, only aliases for existing types.

\label{TypeDefGuard}
Type definitions may have guards: an invocation of a type definition
is illegal unless the guard is satisified when formal types and values
are replaced by the actual parameters.

Type definitions may be overloaded: two type definitions with
the same name are permitted provided that they have a different number
of type parameters or different number or type of value parameters.  The rules
for type definition resolution are identical to those for method resolution.

However, \xcd`T()` is not allowed. If there is an argument list, it must be
nonempty.  This avoids a possible confusion between 
\xcd`type T = ...` and \xcd`type T() = ...`.  

A type definition for a type \xcd`T` must appear: 
\begin{itemize}
\item As a top-level definition in a file named \xcd`T.x10`; or
\item As a static member in a container definition; or
\item In a block statment.
\end{itemize}


\paragraph{Use of type definitions in constructor invocations}
If a type definition has no type parameters and no value
parameters and is an alias for a class type, a \xcd"new"
expression may be used to create an instance of the class using
the type definition's name.
Given the following type definition:
%TODO: Yoav says ``I just opened a jira on it: [1918].  I don't think you
% should be able to have {c} on the typedef A if you want to use it in a 'new'
% expression. If we do allow it, then we should allow: new
% Array[Int]{rank==1}(0..2) and new Array[Int](1)(0..2).
\begin{xtenmath}
type A = C[T$_1$, $\dots$, T$_k$]{c};
\end{xtenmath}
where 
\xcdmath"C[T$_1$, $\dots$, T$_k$]" is a
class type, a constructor of \xcdmath"C" may be invoked with
\xcdmath"new A(e$_1$, $\dots$, e$_n$)", if the
invocation
\xcdmath"new C[T$_1$, $\dots$, T$_k$](e$_1$, $\dots$, e$_n$)" is
legal and if the constructor return type is a subtype of
\xcd"A".

\paragraph{Automatically imported type definitions}
\index{import,type definitions}
\label{X10LangUnderscore}

The collection of type definitions in
\xcdmath"x10.lang._" is automatically imported in every compilation unit.


\subsection{Motivation and use}
The primary purpose of type definitions is to provide a succinct,
meaningful name for complex types
and combinations of types. 
With value arguments, type arguments, and constraints, the syntax for \Xten{}
types can often be verbose. 
For example, a non-null list of non-null strings is \\
%~~type~~`~~`~~ ~~import x10.util.*; ^^^ Types220
\xcd`List[String{self!=null}]{self!=null}`.

We could name that type: 
%~~gen ^^^ Types230
% package TypeDefs.glip.first;
% import x10.util.*;
% class LnSn {
% 
%~~vis
\begin{xten}
static type LnSn = List[String{self!=null}]{self!=null};
\end{xten}
%~~siv
%}
%~~neg
Or, we could abstract it somewhat, defining a type constructor
\xcd`Nonnull[T]` for the type of \xcd`T`'s which are not null:
%~~gen ^^^ Types240
% package TypeDefs.glip.second;
% import x10.util.*;
% 
%~~vis
\begin{xten}
class Example {
  static type Nonnull[T]{T <: Object}  = T{self!=null};
  var example : Nonnull[Example] = new Example();
}
\end{xten}
%~~siv
%
%~~neg

Type definitions can also refer to values, in particular, inside 
constraints.  The type of \xcd`n`-element \xcd`Array[Int](1)`s  is 
%~~type~~`~~`~~n:Int ~~ ^^^ Types250
\xcd`Array[Int]{self.rank==1 && self.size == n}`
but it is often convenient to give a shorter name: 
%~~gen ^^^ Types260
% package TypeDefs.glip.third;
% class Xmpl {
% def example() {
%~~vis
\begin{xten}
type Vec(n:Int) = Array[Int]{self.rank==1, self.size == n}; 
var example : Vec(78); 
\end{xten}
%~~siv
%}}
%~~neg

%
The following examples are legal type definitions, given \xcd`import x10.util.*`:
%~~gen ^^^ Types270
% package Types.TypeDef.Examples;
% import x10.util.*;
%~~vis
\begin{xten}
class TypeExamples {
  static type StringSet = Set[String];
  static type MapToList[K,V] = Map[K,List[V]];
  static type Int(x: Int) = Int{self==x};
  static type Dist(r: Int) = Dist{self.rank==r};
  static type Dist(r: Region) = Dist{self.region==r};
  static type Redund(n:Int, r:Region){r.rank==n} 
      = Dist{rank==n && region==r};
}
\end{xten}
%~~siv
% 
%~~neg

The following code illustrates that type definitions are applicative rather
than generative.  \xcd`B` and \xcd`C` are both aliases for \xcd`String`,
rather than new types, and so are interchangeable with each other and with
\xcd`String`. Similarly, \xcd`A` and \xcd`Int` are equivalent.
%~~gen ^^^ Types280
% package Types.TypeDef.Example.NonGenerative;
% import x10.util.*;
% class TypeDefNonGenerative {
%~~vis
\begin{xten}
def someTypeDefs () {
  type A = Int;
  type B = String;
  type C = String;
  a: A = 3;
  b: B = new C("Hi");
  c: C = b + ", Mom!";
  }
\end{xten}
%~~siv
% }
%~~neg
% An instance of a defined type with no type parameters and no
% value parameters may 


%%MEMBERSHIP%% All type definitions are members of their enclosing package or
%%MEMBERSHIP%% class.  A compilation unit may have one or more type definitions
%%MEMBERSHIP%% or class or interface declarations with the same name, as long
%%MEMBERSHIP%% as the definitions have distinct parameters according to the
%%MEMBERSHIP%% method overloading rules (\Sref{MethodOverload}).


\section{Constrained types}
\label{ConstrainedTypes}
\label{DepType:DepType}
\label{DepTypes}

\index{dependent type}
\index{type!dependent}
\index{constrained type}
\index{generic type}
\index{type!constrained}
\index{type!generic}


Basic types, like \xcd`Int` and \xcd`List[String]`, provide useful
descriptions of data.  

However, one frequently wants to say more.  One might want to know
that a \xcd`String` variable is not \xcd`null`, or that a matrix is
square, or that one matrix has the same number of columns as another
has rows (so they can be multiplied).  In the multicore setting, one
might wish to know that two values are located at the same processor,
or that one is located at the same place as the current computation.

In most languages, there is simply no way to say these things
statically.  Programmers must made do with comments, \xcd`assert`
statements, and dynamic tests.  X10 programs can do better, with {\em
  constraints} on types, and guards on class, method and type
definitions,

A constraint is a boolean expression \xcd`e` attached to a basic type \xcd`T`,
written \xcd`T{e}`.  (Only a limited selection of boolean expressions is
available.)  The values of type \xcd`T{e}` are the values of \xcd`T` for which
\xcd`e` is true.  

\index{self}When constraining a value of type \xcd`T`, \xcd`self` refers to the object of
type \xcd`T` which is being constrained.  For example, \xcd`Int{self == 4}` is
the type of \xcd`Int`s which are equal to 4 -- the best possible description
of \xcd`4`, and a very difficult type to express without using \xcd`self`.  

\begin{ex}

\begin{itemize}
%~~type~~`~~`~~ ~~ ^^^ Types290
\item \xcd`String{self != null}` is the type of non-null strings.  \xcd`self`
      is a special variable available only in constraints; it refers to the
      datum being constrained, and its type is the type to which the
      constraint is attached.
\item Suppose that \xcd`Matrix` is a matrix class with  properties \xcd`rows`
      and \xcd`cols`.  
%~~type~~`~~`~~ ~~class Matrix(rows:Int,cols:Int){} ^^^ Types300
      \xcd`Matrix{self.rows == self.cols}` is the type of square matrices.
\item One way to say that \xcd`a` has the same number of columns that \xcd`b`
      has rows (so that \xcd`a*b` is a valid matrix product), one could say: 
%~~gen ^^^ Types310
% package Types.cripes.whered.you.get.those.gripes;
% class Matrix(rows:Int, cols:Int){
% public static def someMatrix(): Matrix = null;
% public static def example(){
%~~vis
\begin{xten}
  val a : Matrix = someMatrix() ;
  var b : Matrix{b.rows == a.cols} ;
\end{xten}
%~~siv
%}}
%~~neg
\end{itemize}
\end{ex}



\xcd"T{e}" is a {\em dependent type}, that is, a type dependent on values. The
type \xcd"T" is called the {\em base type} and \xcd"e" is called the {\em
  constraint}. If the constraint is omitted, it is \xcd`true`---that is, the
  base type is unconstrained.

Constraints may refer to immutable values in the local environment: 
%~~gen ^^^ Types320
% class ConstraintsMayReferToValues {
% def thoseValues() {
%~~vis
\begin{xten}
     val n = 1;
     var p : Point{rank == n};
\end{xten}
%~~siv
%}}
%~~neg
In a variable declaration, the variable itself is in scope in its
type. For example, \xcd`val nz: Int{nz != 0} = 1;` declares a
non-zero variable \xcd`nz`.
\bard{This will need to be explained further once the language issues are
sorted out.}

%%TYPES-CONSTR-EXP%% We permit variable declarations \xcd"v: T" where \xcd"T" is obtained
%%TYPES-CONSTR-EXP%% from a dependent type \xcd"C{c}" by replacing one or more occurrences
%%TYPES-CONSTR-EXP%% of \xcd"self" in \xcd"c" by \xcd"v". (If such a declaration \xcd"v: T"
%%TYPES-CONSTR-EXP%% is type-correct, it must be the case that the variable \xcd"v" is not
%%TYPES-CONSTR-EXP%% visible at the type \xcd"T". Hence we can always recover the
%%TYPES-CONSTR-EXP%% underlying dependent type \xcd"C{c}" by replacing all occurrences of \xcd"v"
%%TYPES-CONSTR-EXP%% in the constraint of \xcd"T" by \xcd"self".)
%%TYPES-CONSTR-EXP%% 
%%TYPES-CONSTR-EXP%% For instance, \xcd"v: Int{v == 0}" is shorthand for \xcd"v: Int{self == 0}".
%%TYPES-CONSTR-EXP%% 
%%TYPES-CONSTR-EXP%% 
%%TYPES-CONSTR-EXP%% A variable occurring in the constraint \xcd"c" of a dependent type, other than
%%TYPES-CONSTR-EXP%% \xcd"self" or a property of \xcd"self", is said to be a {\em
%%TYPES-CONSTR-EXP%% parameter} of \xcd"c".\label{DepType:Parameter} \index{parameter}

A constrained type may be constrained further: the type \xcd`S{c}{d}`
is the same as the type \xcd`S{c,d}`.  Multiple constraints are equivalent to
conjoined constraints: \xcd`S{c,d}` in turn is the same as \xcd`S{c && d}`.

\subsection{Syntax of constraints}
\index{constraint!permitted}
\index{constraint!syntax}
\label{PermittedConstraints}
\index{constraint}
\index{expression!allowed in constraint}
\index{expression!constraint}

Only a few kinds of expressions can appear in constraints.  For fundamental
reasons of mathematical logic, the more kinds of expressions that can appear
in constraints, the harder it is to compute the essential properties of
constrained type -- in particular, the harder it is to compute 
\xcd`A{c} <: B{d}` or even \xcd`E : T{c}`.  It doesn't take much to make this
basic fact undecidable. 
In order to make sure that it stays decidable, X10 places stringent restrictions on
constraints.  

Only the following forms of expression are allowed in constraints.  

{\bf Value expressions in constraints} may be: 
\begin{enumerate}
\item Literal constants, like \xcd`3` and \xcd`true`;
% \item Expressions computable at compile time, like \Xcd{3*4+5};
\item Accessible, immutable (\xcd`val`) variables and parameters;
% \item Accessible and immutable fields of objects;
% \item Properties of the type being constrained;
\item \xcd`this`, if the constraint is in a place where \xcd`this` is defined;
\item \xcd`here`, if the constraint is in a place where \xcd`here` is defined;
\item \xcd`self`;
\item A field selection expression \xcd`t.f`, where \xcd`t` is a value
      expression allowed in constraints, and \xcd`f` is a field of \xcd`t`'s
      type.    
 \item Invocations of property methods,  \xcd`p(a,b,...,c)` or
      \xcd`a.p(b,c,...d)`, where the receiver and arguments must be
       value expressions acceptable in constraints, as long as the expansion
       (\viz, the expression obtained by taking the body of the definition of
       \xcd`p`, and replacing the formal parameters by the actual parameters)
       of the invocation is allowed as a value expression in constraints.  
\end{enumerate}
For an expression \xcd`self.p` to be legal in a constraint, 
\xcd`p` must be 
a property. However terms \xcd`t.f` may be
used in constraints (where \xcd`t` is a term other than \xcd`self` and
\xcd`f` is an immutable field.)

{\bf Constraints}  may be any of
the following, where 
all value expressions are of the forms which may appear in constraints: 
\begin{enumerate}
\item Equalities \xcd`e == f`;
\item Inequalities of the form \xcd`e != f`;\footnote{Currently inequalities
      of the form \xcd`e < f` are not supported.}
\item Conjunctions of Boolean expressions that may appear in constraints (but
      only in top-level constraints, not in Boolean expressions in constraints);
\item Subtyping and supertyping expressions: \xcd`T <: U` and \xcd`T :> U`; 
\item Type equalities and inequalities: \xcd`T == U` and \xcd`T != U`; 
\item Invocations of a property method, \xcd`p(a,b,...,c)` or
      \xcd`a.p(b,c,...d)`, where the receiver and arguments must be value
      expressions acceptable in constraints, as long as the expansion of the
      invocation is allowed as a constraint.
\item Testing a type for a default: \Xcd{T haszero}.
\end{enumerate}

All variables appearing in a constraint expression must be visible wherever
that expression can used.  \Eg, properties and public fields of an object are
always permitted, but private fields of an object can only constrain private
members.  (Consider a class \xcd`PriVio` with a private field \xcd`p` and a
public method \xcd`m(x: Int{self != p})`, and a call \xcd`ob.m(10)` made
outside of the class. Since \xcd`p` is only visible inside the class, there is
no way to tell if \xcd`10` is of type \xcd`Int{self != p}` at the call site.)

{\bf Note:} Constraints may not contain casts.   In particular, comparisons of
values of incompatible types are not allowed.  If \xcd`i:Int`, then \xcd`i==0`
is allowed as a constraint, but \xcd`i==0L` is an error, and 
\xcd`i as Long==0L` is outside of the constraint language.


\subsubsection{Semantics of constraints}
\index{constraint!semantics}
\label{SemanticsOfConstraints}
An assignment of values to variables is said to be a {\em solution} for a
constraint \xcd`c` if under this assignment \xcd`c` evaluates to
\xcd`true`. For instance, the assignment that maps 
the variables \xcd`a` and \xcd`b` to a value \xcd`t` is a solution for
the constraint \xcd`a==b`. An assignment that maps \xcd`a` to 
\xcd`s` and \xcd`b` to a distinct value \xcd`t` is a solution for 
\xcd`a != b`. 

An instance \xcd"o" of \xcd"C" is said to be of type \xcd"C{c}" (or {\em
belong to} \xcd"C{c}") if the constraint \xcd"c" evaluates to \xcd"true" in
the current lexical environment augmented with the binding \xcd"self"
$\mapsto$ \xcd"o".

A constraint \xcd`c` is said to {\em entail} a
constraint \xcd`d` if every solution for \xcd`c` is also a solution
for \xcd`d`. For instance the constraint
\xcd`x==y && y==z && z !=a` entails \xcd`x != a`.

The constraint solver considers the assignment \xcd`a` to \xcd`null`
to satisfy any constraint of the form \xcd`a.f==t`. 
Thus, the declaration 
\xcd`var x:Tree{self.r==p}=null` does not
produce an error, since \xcd`self==null` satisfies the constraint
\xcd`self.r==p`.  
(This only applies in constraints, not in expression evaluation.  
\xcd`if(a.f==t)...` will throw an exception if \xcd`a==null`.)
If \xcd`null` is not an acceptable value for some class \xcd`Tree`, 
add \xcd`self!=null` as a constraint: 
\xcd`Tree{self!=null}` is the class of non-\xcd`null` \xcd`Tree`s.

To ensure that type-checking is decidable, we require that property graphs be
acyclic.  The property graph, at an instant in an X10 execution, is the graph
whose nodes are all objects in existence at that instance, with an edge from
{$x$} to {$y$} if {$x$} is an object with a property whose value is {$y$}. 
The rules for constructors guarantee this.

Constraints participate in the subtyping relationship in a natural way:
\xcdmath"S[S1,$\ldots$, Sm]{c}" 
is a subtype of 
\xcdmath"T[T1,$\ldots$, Tn]{d}" 
if \xcdmath"S[S1,$\ldots$,Sm]" is a subtype of \xcdmath"T[T1,$\ldots$,Tn]" and
\xcd"c" entails \xcd"d".

For examples of constraints and entailment, see (\Sref{ConstraintExamples})
%%TYPES-CONSTR-EXP%% 
%%TYPES-CONSTR-EXP%% \begin{grammar}
%%TYPES-CONSTR-EXP%% Constraint \: ValueArguments     Guard\opt \\
%%TYPES-CONSTR-EXP%%            \| ValueArguments\opt Guard     \\
%%TYPES-CONSTR-EXP%%            \\
%%TYPES-CONSTR-EXP%% ValueArguments   \:  \xcd"(" ArgumentList\opt \xcd")" \\
%%TYPES-CONSTR-EXP%% ArgumentList     \:  Expression ( \xcd"," Expression )\star \\
%%TYPES-CONSTR-EXP%% Guard            \: \xcd"{" DepExpression \xcd"}" \\
%%TYPES-CONSTR-EXP%% DepExpression    \: ( Formal \xcd";" )\star ArgumentList \\
%%TYPES-CONSTR-EXP%% \end{grammar}
%%TYPES-CONSTR-EXP%% 
%%TYPES-CONSTR-EXP%% In \XtenCurrVer{} value constraints may be equalities (\xcd"=="),
%%TYPES-CONSTR-EXP%% disequalities (\xcd"!=") and conjunctions thereof.  The terms over
%%TYPES-CONSTR-EXP%% which these constraints are specified include literals and
%%TYPES-CONSTR-EXP%% (accessible, immutable) variables and fields, property methods, and the special
%%TYPES-CONSTR-EXP%% constants {\tt here}, {\tt self}, and {\tt this}. Additionally, place
%%TYPES-CONSTR-EXP%% types are permitted (\Sref{PlaceTypes}).
%%TYPES-CONSTR-EXP%% 
%%TYPES-CONSTR-EXP%% \index{self}

%%TYPES-CONSTR-EXP%% Type constraints may be subtyping and supertyping (\xcd"<:" and
%%TYPES-CONSTR-EXP%% \xcd":>") expressions over types.

\subsection{Constraint solver: incompleteness and approximation}
\index{constraint solver!incompleteness}
\index{constraint!entailment}
\index{constraint!subtyping}



The constraint solver is sound in that if it claims that \xcd`c` entails \xcd`d`
then in fact it is the case that every value that satisfies \xcd`c`
satisfies \xcd`d`. 

\limitationx{}
X10's constraint solver is incomplete. There are situations
in which \xcd`c` entails \xcd`d` but the solver cannot establish it. For
instance it cannot establish that \xcd`a != b && a != c && b != c`
entails \xcd`false` if \xcd`a`, \xcd`b`, and \xcd`c` are of type
\xcd`Boolean`.


Certain other constraint entailments are prohibitively expensive to calculate.  The issues
concern constraints that connect different levels of recursively-defined
types, such as the following.  
%~~gen ^^^ Types330
% package Types.Entailment.EntailFail;
%~~vis
\begin{xten}
class Listlike(x:Int) {
  val kid : Listlike{self.x == this.x};
  def this(x:Int, kid:Listlike) { 
     property(x); 
     this.kid = kid as Listlike{self.x == this.x};}
}
\end{xten}
%~~siv
%
%~~neg
There is nothing wrong with \xcd`Listlike` itself, or with most uses of it;
however, a sufficiently complicated use of it could, in principle, cause X10's
typechecker to fail. 
It is hard to give a plausible example of when X10's algorithm fails, as we
have not yet observed such a failure in practice for a correct program.  

The entailment algorithm of X10 imposes a certain limit on the number of
times such types will be unwound.   If this limit is exceeded, the compiler
will print a warning, and type-checking will fail in a situation where it is
semantically allowed.  In this case, insert a dynamic cast at the point where
type-checking failed.  

\limitation{ Support for comparisons of generic type variables is
  limited. This will be fixed in future releases.}
% //, and existential quantification over typed variables.

%%TYPES-CONSTR-EXP%% \emph{
%%TYPES-CONSTR-EXP%% Subsequent implementations are intended to support boolean algebra,
%%TYPES-CONSTR-EXP%% arithmetic, relational algebra, etc., to permit types over regions and
%%TYPES-CONSTR-EXP%% distributions. We envision this as a major step towards removing most,
%%TYPES-CONSTR-EXP%% if not all, dynamic array bounds and place checks from \Xten{}.
%%TYPES-CONSTR-EXP%% }




%%PLACE%%\subsection{Place constraints}
%%PLACE%%\label{PlaceTypes}
%%PLACE%%\label{PlaceType}
%%PLACE%%\index{place types}
%%PLACE%%\label{DepType:PlaceType}\index{placetype}
%%PLACE%%
%%PLACE%%An \Xten{} computation spans multiple places (\Sref{XtenPlaces}). Much data
%%PLACE%%can only be accessed from the proper place, and often it is preferable to
%%PLACE%%determine this statically. So, X10 has special syntax for working with places.
%%PLACE%%\xcd`T!` is a value of type \xcd`T` located at the right place for the current
%%PLACE%%computation, and \xcd`T!p` is one located at place \xcd`p`.
%%PLACE%%
%%PLACE%%\begin{grammar}
%%PLACE%%PlaceConstraint     \: \xcd"!" Place\opt \\
%%PLACE%%Place              \:   Expression \\
%%PLACE%%\end{grammar}
%%PLACE%%
%%PLACE%%More specifically, All \Xten{} classes extend the class \xcd"x10.lang.Object",
%%PLACE%%which defines a property \xcd"home" of type \xcd"Place".  \xcd`T!p`, when
%%PLACE%%\xcd`T` is a class, is \xcd`T{self.home==p}`.  If \xcd`p` is omitted, it
%%PLACE%%defaults to \xcd`here`.   \xcd`T!` is far and away the most common usage of
%%PLACE%%\xcd`!`. 
%%PLACE%%
%%PLACE%%Structs don't have \xcd`home`; they are available everywhere.  For structs, 
%%PLACE%%\xcd`T!` and \xcd`T!p` are synonyms for \xcd`T`. Since \xcd`T` is available
%%PLACE%%everywhere, it is available \xcd`here` and at \xcd`p`. 
%%PLACE%%
%%PLACE%%\xcd`!` may be combined with other constraints.  \xcd`T{c}!` is the type of
%%PLACE%%values of \xcd`T!` which satisfy \xcd`c`; it is \xcd`T{c && self.home==here}`
%%PLACE%%for an object type and \xcd`T{c}` for a struct type.  
%%PLACE%%\xcd`T{c}!p` is the type of
%%PLACE%%values of \xcd`T!p` which satisfy \xcd`c`; it is \xcd`T{c && self.home==p}`
%%PLACE%%for an object type and \xcd`T{c}` for a struct type.  
%%PLACE%%
%%PLACE%%
%%PLACE%%
%%PLACE%%% The place specifier \xcd"any" specifies that the object can be
%%PLACE%%% located anywhere.  Thus, the location is unconstrained; that is,
%%PLACE%%% \xcd"C{c}!any" is equivalent to \xcd"C{c}".
%%PLACE%%
%%PLACE%%% XXX ARRAY
%%PLACE%%%The place specifier \xcd"current" on an array base type
%%PLACE%%%specifies that an object with that type at point \xcd"p"
%%PLACE%%%in the array 
%%PLACE%%%is located at \xcd"dist(p)".  The \xcd"current" specifier can be
%%PLACE%%%used only with array types.
%%PLACE%%
%%PLACE%%

\subsection{Limitation: Runtime Constraint Erasure}
\index{cast!to generic type}

The X10 runtime does not maintain a representation of constraints.
In many cases, it does not need to.
If X10 has an object \xcd`x` of some type \xcd`T` around, it can check at
runtime whether or not \xcd`x` satisfies some constraint \xcd`c`, and hence
tell if \xcd`x` is a member of \xcd`T{c}`. 

\begin{ex}
Although there is no runtime representation of the constrained type 
\xcd`Int{self==1}`, X10 can generate a (correct) test for membership in it,
anyhow: 
%~~gen ^^^ Types3u5w
% package Types3u5w;
% class Example {
%~~vis
\begin{xten}
static def example(n:Int) {
  val b = (n instanceof Int{self == 1});
  assert b == (n == 1); 
}
\end{xten}
%~~siv
% }
% class Hook{ def run() {Example.example(0); Example.example(1);
% Example.example(2); return true; } }
%~~neg
\end{ex}

However, in cases where there is no object of type \xcd`T` around, there's
nothing that can be checked. For example, X10 cannot tell -- and in fact no
computer program can tell --  whether an
instance of a function type 
\begin{xtenmath}
(Int)=>Int
\end{xtenmath}
(unary functions returning
integers) is actually an instance of a more specific type
\begin{xtenmath}
(Int)=>Int{self!=0}
\end{xtenmath}
(unary functions returning non-zero integers).

In other cases, there might or might not be an object of type \xcd`T`, and X10
cannot tell until runtime.  Consider an array \xcd`a:Array[T]`.  If \xcd`a` is
nonempty, there is an instance of \xcd`T` at hand, and testing it for
constraints would be possible though potentially quite expensive. 
But \xcd`a` might be an
empty array, and testing its element type would be impossible. 

Rather than pay the runtime costs for keeping and manipulating constraints
(which can be considerable), X10 omits them.
However, this renders certain type checks uncertain: X10 needs some
information at runtime, but does not have it.  
Specifically, all casts to instances of generic types are forbidden.  

\begin{ex}
The following code  needs to be, and is, statically
rejected.  It constructs an array \xcd`a` of \xcd`Int{self==3}`'s -- integers
which 
are statically known to be 3. 
The only number that can be stored into \xcd`a` is \xcd`3`.  
Then (in the line that is rejected) it attempts to trick the compiler into
thinking that it is an array of \xcd`Int`, without restriction on the
elements, giving it the name \xcd`b` at that type.  
The cast \xcd`aa as Array[Int]` is a cast to an instance of a generic type,
and hence is forbidden. 

But, if that cast were allowed to work, it could store \xcd`1` into the array
under the alias 
\xcd`b`, thereby violating 
the invariant that all the elements of the array are 3.  
This could lead to program failures, as illustrated by the failing assertion.  
\begin{xten}
  val a = new Array[Int{self==3}](0..10, 3);
  // a(0) = 1; would be illegal
  a(0) = 3; // LEGAL
  val aa = a as Any;
  /* THE FOLLOWING IS A STATIC ERROR:
  val b = aa as Array[Int];
  b(0) = 1;
  val x : Int{self==3} = a(0);
  assert x == 3 : "This would fail at runtime.";
  */
\end{xten}
\end{ex}



\subsection{Example of Constraints}
\label{ConstraintExamples}

Example of entailment and subtyping involving constraints.
\begin{itemize}
\item \xcd`Int{self == 3} <: Int{self != 14}`.  The only value of
      \xcd`Int{self ==3}` is $3$.  All integers but $14$ are members of
      \xcd`Int{self != 14}`, and in particular $3$ is.  
\item Suppose we have classes \xcd`Child <: Person`, and \xcd`Person` has a
      \xcd`ssn:Long` property.  If \xcd`rhys : Child{ssn == 123456789}`, then
      \xcd`rhys` is also a \xcd`Person`.  
      \xcd`rhys`'s \xcd`ssn` field is the same, \xcd`123456789`, whether 
      \xcd`rhys` is regarded as a \xcd`Child` or a \xcd`Person`.  
      Thus, 
      \xcd`rhys : Person{ssn==123456789}` as well.  
      So, 
\begin{xtenmath}
Child{ssn == 123456789} <: Person{ssn == 123456789}.
\end{xtenmath}
\item Furthermore, since \xcd`123456789 != 555555555`, 
      is is clear that 
      \xcd`rhys : Person{ssn != 555555555}`.  
      So, 
\begin{xtenmath}
Child{ssn == 123456789} <: Person{ssn != 555555555}.  
\end{xtenmath}
\item \xcd`T{e} <: T` for any type \xcd`T`.  That is, if you have a value
      \xcd`v` of some base type \xcd`T` which satisfied \xcd`e`, then \xcd`v`
      is of that base type \xcd`T` (with the constraint ignored).
\item If \xcd`A <: B`, then \xcd`A{c} <: B{c}` for every constraint \xcd`{c}`
      for which \xcd`A{c}` and \xcd`B{c}` are defined.  That is, if every
      \xcd`A` is also a \xcd`B`, and \xcd`a : A{c}`, then 
      \xcd`a` is an \xcd`A` and \xcd`c` is true of it. So \xcd`a` is also a
      \xcd`B` (and \xcd`c` is still true of 
      it), so \xcd`a : B{c}`.  
\end{itemize}

Constraints can be used to express simple relationships between objects,
enforcing some class invariants statically.  For example, in geometry, a line
is determined by two {\em distinct} points; a \xcd`Line` struct can specify the
distinctness in a type constraint:\footnote{We call them
\xcd`Position` to avoid confusion with the built-in class \xcd`Point`. 
Also, \xcd`Position` is a struct rather than a class so that the non-equality
test \xcd`start != end` compares the coordinates.  If \xcd`Position` were a
class, \xcd`start != end` would check for different \xcd`Position` objects,
which might have the same coordinates.
}


%~~gen ^^^ Types340
% package triangleExample.partOne;
%~~vis
\begin{xten}
struct Position(x: Int, y: Int) {}
struct Line(start: Position, end: Position){start != end}
  {}
\end{xten}

%~~siv
%~~neg

Extending this concept, a \xcd`Triangle` can be defined as a figure with three
line segments which match up end-to-end.  Note that the degenerate case in
which two or three of the triangle's vertices coincide is excluded by the
constraint on \xcd`Line`.  However, not all degenerate cases can be excluded
by the type system; in particular, it is impossible to check that the three
vertices are not collinear. 

%~~gen ^^^ Types350
%package triangleExample.partTwo;
% struct Position(x: Int, y: Int) {
%    def this(x:Int,y:Int){property(x,y);}
%    }
% class Line(start: Position, 
%            end: Position{self != start}) {}
% 
%~~vis
\begin{xten}
struct Triangle 
 (a: Line, 
  b: Line{a.end == b.start}, 
  c: Line{b.end == c.start && c.end == a.start})  
 {}
\end{xten}
%~~siv
%
%~~neg

The \xcd`Triangle` class automatically gets a ternary constructor which takes
suitably constrained \xcd`a`, \xcd`b`, and \xcd`c` and produces a new
triangle. 

\section{Default Values}
\index{default value}
\index{type!default value}
\label{DefaultValues}

Some types have default values, and some do not. Default values are used in
situations where variables can legitimately be used without having been
initialized; types without default values cannot be used in such situations.
For example, a field of an object \xcd`var x:T` can be left uninitialized if
\xcd`T` has a default value; it cannot be if \xcd`T` does not. Similarly, a
transient (\Sref{TransientFields}) field \xcd`transient val x:T` is only
allowed if \xcd`T` has a default value.

Default values, or lack of them, is defined thus:
\begin{itemize}
\item The fundamental numeric types (\xcd`Int`, \xcd`UInt`,
      \xcd`Long`, \xcd`ULong`, 
%%limitation%%       \xcd`Short`, \xcd`UShort`, \xcd`Byte`,
%%limitation%%       \xcd`UByte`, 
      \xcd`Float`, \xcd`Double`) all have default value 0.
\item \xcd`Boolean` has default value \xcd`false`.
\item \xcd`Char` has default value \xcd`'\0'`.
\item Struct types other than those listed above have no default value.
\item A function type has a default value of \xcd`null`.
\item A class type has a default value of \xcd`null`.
\item The constrained type \xcd`T{c}` has the same default value as \xcd`T` if
      that default value satisfies \xcd`c`.  If the default value of \xcd`T`
      doesn't satisfy \xcd`c`, then \xcd`T{c}` has no default value.
\end{itemize}

\begin{ex}
\xcd`var x: Int{x != 4}` has default value 0, which is allowed
because \xcd`0 != 4` satisfies the constraint on \xcd`x`. 
\xcd`var y : Int{y==4}` has no default value, because \xcd`0` does not satisfy \xcd`y==4`.
The fact that \xcd`Int{y==4}` has precisely one value, \viz{} 4, doesn't
matter; the only candidate for its default value, as for any subtype of
\xcd`Int`, is 0. \xcd`y` must be initialized before it is used.
\end{ex}

The predicate \xcd`T haszero` tells if the type \xcd`T` has a default value.
\xcd`haszero` may be used in constraints. 

\begin{ex}
The following code defines a sort of cell holding a single value of type
\xcd`T`. The cell is initially empty -- that is, has \xcd`T`'s zero value --
but may be filled later. 
%~~gen ^^^ TypesHaszero10
% package TypesHaszero10;
%~~vis
\begin{xten}
class Cell0[T]{T haszero} {
  public var contents : T;
  public def put(t:T) { contents = t; }
}
\end{xten}
%~~siv
%
%~~neg
\end{ex}

The built-in type \xcd`Zero` has the method \xcd`get[T]()` which
returns the default value of type \xcd`T`.  

\begin{ex}
As a variation on a theme of \xcd`Cell0`, we define a class \xcd`Cell1[T]` which can be initialized with a value of an arbitrary
type
\xcd`T`, or, if \xcd`T` has a default value, can be created with the default
value.  Note that \xcd`T haszero` is a constraint on one of
the constructors, not the whole type:  
%~~gen ^^^ TypesHaszero20
% package TypesHaszero20;
%~~vis
\begin{xten}
class Cell1[T] {
  public var contents: T;
  def this(t:T) { contents = t; }
  def this(){T haszero} { contents = Zero.get[T](); }
  public def put(t:T) {contents = t;}
}
\end{xten}
%~~siv
%
%~~neg

\end{ex}

\section{Function types}
\label{FunctionTypes}
\label{FunctionType}
\index{function!types}
\index{type!function}

%##(FunctionType
\begin{bbgrammar}
%(FROM #(prod:FunctionType)#)
        FunctionType \: TypeParams\opt \xcd"(" FormalList\opt \xcd")" Guard\opt Offers\opt \xcd"=>" Type & (\ref{prod:FunctionType}) \\
\end{bbgrammar}
%##)


For every sequence of types \xcd"T1,..., Tn,T", and \xcd"n" distinct variables
\xcd"x1,...,xn" and constraint \xcd"c", the expression
\xcd"(x1:T1,...,xn:Tn){c}=>T" is a \emph{function type}. It stands for
 the set of all functions \xcd"f" which can be applied to a
 list of values \xcd"(v1,...,vn)" provided that the constraint
 \xcd"c[v1,...,vn,p/x1,...,xn]" is true, and which returns a value of
 type \xcd"T[v1,...vn/x1,...,xn]". When \xcd"c" is true, the clause \xcd"{c}" can be
 omitted. When \xcd"x1,...,xn" do not occur in \xcd"c" or \xcd"T", they can be
 omitted. Thus the type \xcd"(T1,...,Tn)=>T" is actually shorthand for
 \xcd"(x1:T1,...,xn:Tn){true}=>T", for some variables \xcd"x1,...,xn".

\limitationx{}
Constraints on closures are not supported.  They parse, but are not checked.

X10 functions, like mathematical functions, take some arguments and produce a
result.  X10 functions, like other X10 code, can change mutable state and
throw exceptions.  Closures (\Sref{Closures})  are of function type -- and so
are arrays.


\begin{ex}Typical functions are the reciprocal function: 
%~~gen ^^^ Types360
% package Types.Functions;
% class RecipEx {
% static 
%~~vis
\begin{xten}
val recip = (x : Double) => 1/x;
\end{xten}
%~~siv
%}
%~~neg
and a function which increments  element \xcd`i` of an array \xcd`r`, or throws an exception
if there is no such element, where, for the sake of example, we constrain the
type of \xcd`i` to avoid one of the many integers which are not possible subscripts:  
%~~gen ^^^ Types_constraint_b
% package Types_constraint_b;
% NOTEST
% /*NONSTATIC*/class IncrElEx {
% static def example()  {
%~~vis
\begin{xten}
val inc = (r:Array[Int](1), i: Int{i != r.size}) => {
  if (i < 0 || i >= r.size) throw new DoomExn();
  r(i)++;
};
\end{xten}
%~~siv
%}}
%class DoomExn extends Exception{}
%~~neg
\end{ex}

In general, a function type needs to list the types 
\xcdmath"T$_i$"
of all the formal parameters,
and their distinct names \xcdmath"x$_i$" in case other types refer to them; a
constraint 
\xcd"c" on the
function as a whole; a return type \xcd"T".

\begin{xtenmath}
(x$_1$: T$_1$, $\dots$, x$_n$: T$_n$){c} => T
\end{xtenmath}


The names \xcdmath"x$_i$" of the formal parameters are not relevant.  Types
which differ only in the names of formals (following the usual rules for
renaming of variables, as in {$\alpha$}-renaming in the {$\lambda$} calculus
\bard{cite something}) are considered equal.  \Eg, the two function types
%~~type~~`~~`~~ ~~ ^^^ Types370
\xcd`(a:Int, b:Array[String](1){b.size==a}) => Boolean`
and \\
%~~type~~`~~`~~ ~~ ^^^ Types380
\xcd`(b:Int, a:Array[String](1){a.size==b}) => Boolean`
are equivalent.

\limitation{
Function types differing only in the names of bound variables may wind up being
considered different in X10 v2.2, especially if the variables appear in
constraints.  
}

The formal parameter names are in scope from the point of definition to the
end of the function type---they may be used in the types of other formal parameters
and in the return type. 
Value parameters names may be
omitted if they are not used; the type of the reciprocal function can be
written as
%~~type~~`~~`~~ ~~ ^^^ Types390
\xcd`(Double)=>Double`. 

A function type is covariant in its result type and contravariant in
each of its argument types. That is, let 
\xcd"S1,...,Sn,S,T1,...Tn,T" be any
types satisfying \xcd"Si <: Ti" and \xcd"S <: T". Then
\xcd"(x1:T1,...,xn:Tn){c}=>S" is a subtype of
\xcd"(x1:S1,...,xn:Sn){c}=>T".

A class or struct definition may use a function type 
\begin{xtenmath}
F = (x1:T1,...,xn:Tn){c}=>T
\end{xtenmath}
in its 
implements clause; 
this is equivalent to implementing an interface requiring the single operator
\begin{xtenmath}
public operator this(x1:T1,...,xn:Tn){c}:T
\end{xtenmath}
Similarly, an interface
definition may specify a function type \xcd"F" in its \xcd"extends" clause.
Values of a class or struct implementing \xcd`F` 
can be used as functions of type \xcd`F` in all ways.  
In particular, applying one to suitable arguments calls the \xcd`apply`
method. 

\limitationx{} A class or struct may not implement two different
instantiations of a generic interface. In particular, a class or
struct can implement only one function type.


A function type \xcd"F" is not a class type in that it does not extend any
type or implement any interfaces, or support equality tests. 
\xcd`F` may be implemented, but not extended, by a class or function type. 
Nor is it a struct type, for it has no predefined notion of equality.


\section{Annotated types}
\label{AnnotatedTypes}

\index{type!annotated}
\index{annotations!type annotations}

        Any \Xten{} type may be annotated with zero or more
        user-defined \emph{type annotations}
        (\Sref{XtenAnnotations}).  

        Annotations are defined as (constrained) interface types and are
        processed by compiler plugins, which may interpret the
        annotation symbolically.

        A type \xcd"T" is annotated by interface types
        \xcdmath"A$_1$", \dots,
        \xcdmath"A$_n$"
        using the syntax
        \xcdmath"@A$_1$ $\dots$ @A$_n$ T".

\section{Subtyping and type equivalence}\label{DepType:Equivalence}
\index{type equivalence}
\index{subtyping}

Intuitively, type \xcdmath"T$_1$" is a subtype of type \xcdmath"T$_2$", 
written \xcdmath"T$_1$ <: T$_2$", 
if
every instance of \xcdmath"T$_1$" is also an instance of \xcdmath"T$_2$".  For
example, \xcd`Child` is a subtype of \xcd`Person` (assuming a suitably defined
class hierarchy): every child is a person.  Similarly, \xcd`Int{self != 0}`
is a subtype of \xcd`Int` -- every non-zero integer is an integer.  

This section formalizes the concept of subtyping. Subtyping of types depends
on a {\em type context}, \viz. a set of constraints on type parameters
and variables that occur in the type.
For example: 

%~~gen ^^^ Types400
% package Types.subtyping.cons;
% NOCOMPILE
%~~vis
\begin{xten}
class ConsTy[T,U] {
   def upcast(t:T){T <: U} :U = t;
}
\end{xten}
%~~siv
%
%~~neg
\noindent
Inside \xcd`upcast`, \xcd`T` is constrained to be a subtype of \xcd`U`, and so
\xcd`T <: U` is true, and \xcd`t` can be treated as a value of type \xcd`U`.  
Outside of \xcd`upcast`, there is no reason to expect any relationship between
them, and \xcd`T <: U` may be false.
However, subtyping of types that have no free variables does not depend
on the context.    \xcd`Int{self != 0} <: Int` is always
true.

\limitation{Subtyping of type variables does not work under all circumstances
in the X10 2.2 implementation.}


\begin{itemize}
\item {\bf Reflexivity:} Every type \xcd`T` is a subtype of itself: \xcd`T <: T`.

\item {\bf Transitivity:} If \xcd`T <: U` and \xcd`U <: V`, then \xcd`T <: V`. 

\iffalse
{\bf Class types:}  
Given the definition 
\xcd`class C[$\vec{X}$] extends D[$\vec{Y}$]{d} implements I1, ..., In {...}`
where {$\vec{X}$} is a vector of type variables, and 
{$\vec{Y$} a vector of types possibly involving variables from {$\vec{X}$}, 
and {$\vec{T$} an instantiation of {$\vec{X$} and {$\vec{U$} the corresponding
instantiation of {$\vec{Y$}, 
then 
\xcdmath"C[$\vec{T}$]`"is a subtype of \xcd`D[$\vec{U}$]{d}`, \xcd`I1`, ..., \xcd`In`. 

\item
{\bf Interface types:}  
Given the definition 
\xcdmath"interface I[$\vec{X}$] extends I1, ... In {...}`"
then \xcdmath"I` is a subtype of \xcd`"1`, ..., \xcd`In`.

\item 
{\bf Struct types:} 
Given the definition 
\xcdmath"struct S implements I1, ..., In {...}`"then \xcd`S` is a 
subtype of \xcd`I1`, ..., \xcd`In`. 
\fi

\item {\bf Direct Subclassing:} 
Let {$\vec{X}$} be a (possibly empty) vector of type variables, and
{$\vec{Y}$}, {$\vec{Y_i}$} be vectors of type terms over {$\vec{X}$}.
Let {$\vec{T}$} be an instantiation of {$\vec{X}$}, 
and {$\vec{U}$}, {$\vec{U_i}$} the corresponding instantiation of 
{$\vec{Y}$}, {$\vec{Y_i}$}.  Let \xcd`c` be a constraint, and \xcdmath"c$'$"
be the corresponding instantiation.
We elide properties, and interpret empty vectors as absence of the relevant
clauses. 
Suppose that \xcd`C` is declared by one of the
forms: 
\begin{enumerate}
\item \xcdmath"class C[$\vec{X}$]{c} extends D[$\vec{Y}$]{d}"\\
\xcdmath"implements I$_1[\vec{Y_1}]${i$_1$},...,I$_n[\vec{Y_n}]${i$_n$}{"
\item \xcdmath"interface C[$\vec{X}$]{c} extends I$_1[\vec{Y_1}]${i$_1$},...,I$_n[\vec{Y_n}]${i$_n$}{"
\item \xcdmath"struct C[$\vec{X}$]{c} implements I$_1[\vec{Y_1}]${i$_1$},...,I$_n[\vec{Y_n}]${i$_n$}{"
\end{enumerate}
Then: 
\begin{enumerate}
\item \xcdmath"C[$\vec{T}$] <: D[$\vec{U}$]{d}" for a class
\item \xcdmath"C[$\vec{T}$] <: I$_i$[$\vec{U_i}$]{i$_i$}" for all cases.
\item \xcdmath"C[$\vec{T}$] <: C[$\vec{T}$]{c$'$}" for all cases.
\end{enumerate}


\item
{\bf Function types:}
\begin{xtenmath}
(x$_1$: T$_1$, $\dots$, x$_n$: T$_n$){c} => T
\end{xtenmath}
is a  subtype of 
\begin{xtenmath}
(x$'_1$: T$'_1$, $\dots$, x$'_n$: T$'_n$){c$'$} => T$'$
\end{xtenmath}
if: 
\begin{enumerate}
\item Each \xcdmath"T$_i$ <: T$'_i$";
\item \xcdmath"c[x$'_1$, $\ldots$, x$'_n$ / x$_1$, $\ldots$, x$_n$]" entails \xcdmath"c$'$";
\item \xcdmath"T$'$ <: T";
\end{enumerate}

\item
{\bf Constrained types:}
\xcd`T{c}` is a subtype of \xcd`T{d}` if \xcd`c` entails \xcd`d`. 

\item {\bf Any:} 
Every type \xcd`T` is a subtype of \xcd`x10.lang.Any`.

\item 
{\bf Type Variables:}
Inside the scope of a constraint \xcd`c` which entails \xcd`A <: B`, we have
\xcd`A <: B`.  \eg, \xcd`upcast` above.


%%NO-VARIANCE%% \item 
%%NO-VARIANCE%% {\bf Covariant Generic Types:} 
%%NO-VARIANCE%% If \xcd`C` is a generic type whose {$i$}th type parameter is covariant, 
%%NO-VARIANCE%% and {\xcdmath"T$'_i$ <: T$_i$"}
%%NO-VARIANCE%% and  {\xcdmath"T$'_j$ == T$_j$"} for all {$j \ne i$}, 
%%NO-VARIANCE%% then {\xcdmath"C[T$'_1$, $\ldots$, T$'_n$] <: C[T$'_1$, $\ldots$, T$'_n$]"}.
%%NO-VARIANCE%% \Eg, \xcd`class C[T1, +T2, T3]` with {$i=2$}, and \xcd"U2 <: T2", then
%%NO-VARIANCE%% \xcd`C[T1,U2,T3] <: C[T1,T2,T3]`.
%%NO-VARIANCE%% 
%%NO-VARIANCE%% \item 
%%NO-VARIANCE%% {\bf Contravariant Generic Types:} 
%%NO-VARIANCE%% If \xcd`C` is a generic type whose {$i$}th type parameter is contravariant, 
%%NO-VARIANCE%% and \xcdmath"T$'_i$ <: T$_i$"
%%NO-VARIANCE%% and  \xcdmath"T$'_j$ == T$_j$" for all {$j \ne i$}, 
%%NO-VARIANCE%% then \xcdmath"C[T$'_1$, $\ldots$, T$'_n$] :> C[T$'_1$, $\ldots$, T$'_n$]".
%%NO-VARIANCE%% \Eg, \xcd`class C[T1, -T2, T3]` with {$i=2$}, and \xcdmath"U2 <: T2", then
%%NO-VARIANCE%% \xcd`C[T1,U2,T3] :> C[T1,T2,T3]`.
%%NO-VARIANCE%% 

\end{itemize}


Two types are {\em equivalent}, \xcd`T == U`, if \xcd`T <: U` and \xcd`U <: T`. 


\section{Common ancestors of types}
\label{LCA}

There are several situations where X10 must find a type \xcd`T` that describes
values of two or more different types.  This arises when X10 is trying to find
a good type for: 
\begin{itemize}
%~~exp~~`~~`~~test:Boolean ~~ ^^^ Types410
\item Conditional expressions, like \xcd`test ? 0 : "non-zero"` or even \\
%~~exp~~`~~`~~test:Boolean ~~ ^^^ Types420
      \xcd`test ? 0 : 1`;
%~~exp~~`~~`~~ ~~ ^^^ Types430
\item Array construction, like \xcd`[0, "non-zero"]` and 
%~~exp~~`~~`~~ ~~ ^^^ Types440
      \xcd`[0,1]`;
\item Functions with multiple returns, like
%~~gen ^^^ Types450
% package Types_odd_inferred_return_type;
% class Examplerator {
%~~vis
\begin{xten}
def f(a:Int) {
  if (a == 0) return 0;
  else return "non-zero";
}
\end{xten}
%~~siv
%}
%~~neg
\end{itemize}

In some cases, there is a unique best type describing the expression.  For
example, if \xcd`B` and \xcd`C` are direct subclasses of \xcd`A`, \xcd`pick`
will have return type \xcd`A`: 
%~~gen ^^^ Types_uniq
% package Types.For.Gripes.About.Pipes.Full.Of.Wipes;
%  class A {} class B extends A{} class C extends A{}
% class D {
%~~vis
\begin{xten}
static def pick(t:Boolean, b:B, c:C) = t ? b : c;  
\end{xten}
%~~siv
%}
%~~neg

However, in many common cases, there is no unique best type describing the
expression.  For example, consider the expression {$E$} 
\begin{xtenmath}
b ? 0 : 1   // Call this expression $E$
\end{xtenmath}
The
best type of \xcd`0` 
is \xcd`Int{self==0}`, and the best type of 1 is \xcd`Int{self==1}`.
Certainly {$E$} could be given the type \xcd`Int`, or even \xcd`Any`, and that
would describe all possible results.  However, we actually know more.
\xcd`Int{self != 2}` is a better description of the type of {$E$}---certainly
the result of {$E$} can never be \xcd`2`.   \xcd`Int{self != 2, self != 3}` is
an even better description; {$E$} can't be \xcd`3` either.  We can continue
this process forever, adding integers which {$E$} will definitely not return
and getting better and better approximations. (If the constraint
sublanguage had \xcd`||`, we could give it the type 
\xcd`Int{self == 0 || self == 1`, which would be nearly perfect.  But 
\xcd`||` makes typechecking far more expensive, so it is excluded.)
No X10 type is the best description of {$E$}; there is always a better one.

Similarly, consider two unrelated interfaces: 
%~~gen ^^^ Types460
% package Types.For.Gripes.About.Snipes;
%~~vis
\begin{xten}
interface I1 {}
interface I2 {}
class A implements I1, I2 {}
class B implements I1, I2 {}
class C {
  static def example(t:Boolean, a:A, b:B) = t ? a : b;
}
\end{xten}
%~~siv
%
%~~neg
\xcd`I1` and \xcd`I2` are both perfectly good descriptions of \xcd`t ? a : b`, 
but neither one is better than the other, and there is no single X10 type
which is better than both. (Some languages have {\em conjunctive
    types}, and could say that the return type of \xcd`example` was 
\xcd`I1 && I2`.  This, too, complicates typechecking.)


So, when confronted with expressions like this, X10 computes {\em some}
satisfactory type for the expression, but not necessarily the {\em best} type.  
X10 provides certain guarantees about the common type \xcd`V{v}` computed for 
\xcd`T{t}` and \xcd`U{u}`: 
\begin{itemize}
\item If \xcd`T{t} == U{u}`, then \xcd`V{v} == T{t} == U{u}`.  So, if X10's
      algorithm produces an utterly untenable type for \xcd`a ? b : c`, and
      you want the result to have type \xcd`T{t}`, you can 
      (in the worst case) rewrite it to 
\begin{xtenmath}
a ? b as T{t} : c as T{t}
\end{xtenmath}
\item If \xcd`T == U`, then \xcd`V == T == U`.  For example, 
      X10 will compute the type of \xcd`b ? 0 : 1` as 
      \xcd`Int{c}` for some constraint \xcd`c`---perhaps simply 
      picking \xcd`Int{true}`, \viz, \xcd`Int`. 
\item X10 preserves place information about \xcd`GlobalRef`s, because it is so important. If both
      \xcd`t` and \xcd`u` entail \xcd`self.home==p`, then  
      \xcd`v` will also entail \xcd`self.home==p`.  
\item X10 similarly preserves nullity information.  If \xcd`t` and \xcd`u`
      both entail \xcd`x == null` or \xcd`x != null` for some variable
      \xcd`x`, then \xcd`v` will also entail it as well.

\item The computed upper bound of function types with the {\em same} argument
      types is found by computing the upper bound of the result types.  
      If 
      \xcdmath"T = (T$_1$, $\ldots$, T$_n$) => T'"
      and 
      \xcdmath"U = (T$_1$, $\ldots$, T$_n$) => U'", 
      and \xcd`V'` is the computed upper bound of \xcd`T'` and \xcd`U'`, 
      then the computed upper bound of \xcd`T` and \xcd`U` is 
      \xcdmath"U = (T$_1$, $\ldots$, T$_n$) => V'".
      (But, if the argument types are different, the computed upper bound may
      be \xcd`Any`.)

\end{itemize}

%\subsection{Syntactic abbreviations}\label{DepType:SyntaxAbbrev}

\section{Fundamental types}

Certain types are used in fundamental ways by X10.  

\subsection{The interface {\tt Any}}

It is quite convenient to have a type which all values are instances of; that
is, a supertype of all types.\footnote{Java, for one, suffers a number of
  inconveniences because some built-in types like \xcd`int` and \xcd`char`
  aren't subtypes of anything else.}  X10's universal supertype is the
  interface \xcd`Any`. 

\begin{xten}
package x10.lang;
public interface Any {
  def toString():String;
  def typeName():String;
  def equals(Any):Boolean;
  def hashCode():Int;
}
\end{xten}

\xcd`Any` provides a handful of essential methods that make sense and are
useful for everything. \xcd`a.toString()` produces a
string representation of \xcd`a`, and \xcd`a.typeName()` the string
representation of its type; both are useful for debugging.  \xcd`a.equals(b)`
is the programmer-overridable equality test, and \xcd`a.hashCode()` an integer
useful for hashing.  


\subsection{The class {\tt Object}}
\label{Object}
\index{\Xcd{Object}}
\index{\Xcd{x10.lang.Object}}

The class \xcd"x10.lang.Object" is the supertype of all classes.
A variable of this type can hold a reference to any object.
\xcd`Object` implements \xcd`Any`.



\section{Type inference}
\label{TypeInference}
\index{type!inference}
\index{type inference}

\XtenCurrVer{} supports limited local type inference, permitting
certain variable types and return types to be elided.
It is a static error if an omitted type cannot be inferred or
uniquely determined. Type inference does not consider coercions.

\subsection{Variable declarations}

The type of a \xcd`val` variable declaration can be omitted if the
declaration has an initializer.  The inferred type of the
variable is the computed type of the initializer.
For example, 
%~~stmt~~`~~`~~ ~~ ^^^ Types470
\xcd`val seven = 7;`
is identical to 
\begin{xtenmath}
val seven: Int{self==7} = 7;
\end{xtenmath}
Note that type inference gives the most precise X10 type, which might be more
specific than the type that a programmer would write.



\limitation{At the moment,  \xcd`var` declarations may not have their types
elided in this way.  
}

\subsection{Return types}

The return type of a method can be omitted if the method has a body (\ie, is
not \xcd"abstract" or \xcd"native"). The inferred return type is the computed
type of the body.  In the following example, the return type inferred for
\xcd`isTriangle` is 
%~~type~~`~~`~~ ~~ ^^^ Types490
\xcd`Boolean{self==false}`
%~~gen ^^^ Types500
% package Types.Inferred.Return;
%~~vis
\begin{xten}
class Shape {
  def isTriangle() = false; 
}  
\end{xten}
%~~siv
%
%~~neg
Note that, as with other type inference, methods are given the most specific
type.  In many cases, this interferes with subtyping.  For example, if one
tried to write: 
\begin{xten}
class Triangle extends Shape {
  def isTriangle() = true;
}
\end{xten}
\noindent
the compiler would reject this program for attempting to override
\xcd`isTriangle()` by a method with the wrong type, \viz,
\xcd`Boolean{self==true}`.  In this case, supply the type that is actually
intended for \xcd`isTriangle`: 
\begin{xtenmath}def isTriangle() :Boolean =false;
\end{xtenmath}

The return type of a closure can be omitted.
The inferred return type is the computed type of the body.

The return type of a constructor can be omitted if the
constructor has a body.
The inferred return type is the enclosing class type with
properties bound to the arguments in the constructor's \xcd"property"
statement, if any, or to the unconstrained class type.
For example, the \xcd`Spot` class has two constructors, the first of which has
inferred return type \xcd`Spot{x==0}` and the second of which has 
inferred return type \xcd`Spot{x==xx}`. 
%~~gen ^^^ Types510
% package Types.Inferred.By.Phone;
%~~vis
\begin{xten}
class Spot(x:Int) {
  def this() {property(0);}
  def this(xx: Int) { property(xx); }
}
\end{xten}
%~~siv
%class Confirm{ 
% static val s0 : Spot{x==0} = new Spot();
% static val s1 : Spot{x==1} = new Spot(1);
%}
%~~neg


\index{void}

A method or closure that has expression-free \xcd`return` statements
(\xcd`return;` rather than \xcd`return e;`) is said to return \xcd`void`.
\xcd`void` is not a type; there are no \xcd`void` values, nor can \xcd`void`
be used as the argument of a generic type. However, \xcd`void` takes the
syntactic place of a type in a few contexts. A method returning \xcd`void` can be specified by
\xcd`def m():void`: 

%~~gen ^^^ Types520
% package Types.voidd;
% class voidddd {
% static 
%~~vis
\begin{xten}
val f : () => void = () => {return;};
\end{xten}
%~~siv
%}
%~~neg

By a convenient abuse of language, \xcd`void` is sometimes
lumped in with types; \eg, we may say ``return type of a method'' rather than
the formally correct but rather more awkward ``return type of a method, or
\xcd`void`''.   Despite this informal usage, \xcd`void` is not a type.  For
example, given 
%~~gen ^^^ Types530
% package Types.void_is_not_a_type;
% class EEEEVil {
%~~vis
\begin{xten}
  static def eval[T] (f:()=>T):T = f();
\end{xten}
%~~siv
% }
%~~neg
\noindent
The call \xcd`eval[void](f)` does {\em not} typecheck; \xcd`void` is not a
type and thus cannot be used as a type argument.  There is no way in X10 to
write a generic function which works with both functions which return a value
and functions which do not.  In most cases, functions which have no sensible
return value can be provided with a dummy return value.

\subsection{Inferring Type Arguments}
\label{TypeParamInfer}


A call to a polymorphic method %, closure, or constructor 
may omit the
explicit type arguments.  
X10 will compute a type from the types of the actual arguments. 

(Exception: it is an error if the method call provides no information about
a type parameter that must be inferred.  For example, given the method
definition \xcd`def m[T](){...}`, an invocation \xcd`m()` is considered a
static error.  The compiler has no idea what \xcd`T` the programmer intends.)



\begin{ex}Consider the following method, which chooses one of its arguments.  (A more
sophisticated one might sometimes choose the second argument, but that does
not matter for the sake of this example.)
\begin{xten}
static def choose[T](a: T, b: T): T = a; 
\end{xten}


The type argument \xcd`T` can always be supplied: 
\xcd`choose[Int](1, 2)` picks an integer, 
and \xcd`choose[Any](1, "yes")` picks a value that might be an integer or a
string.  
However, the type argument can be elided.  Suppose that \xcd`Sub <: Super`;
then the following compiles: 

%~~gen ^^^ Types540
% package Types.GenericInference;
% class Exampllll{ 
%~~vis
\begin{xten}
  static def choose[T](a: T, b: T): T = a; 
  static val j : Any = choose("string", 1);
  static val k : Super = choose(new Sub(), new Super());
\end{xten}
%~~siv
%}
% class Super {}
% class Sub extends Super {}
%~~neg
\end{ex}

The type parameter doesn't need to be the type of a variable. It can be found
inside of the type of a variable; X10 can extract it.

\begin{ex}
The \xcd`first` method below returns the first element of a one-dimensional
array.  The type parameter \xcd`T` represents the type of the array's
elements. There is no parameter of type \xcd`T`. There is one of type
\xcd`Array[T]{c}`; X10 strips off the constraint \xcd`{c}` and the
\xcd`Array[...]` type to get at the \xcd`T` inside.
%~~gen ^^^ Types3d5j
% package Types3d5j;
% class Example {
%~~vis
\begin{xten}
static def first[T](x:Array[T](1)) = x(0);
static def example() {
  val ss <: Array[String] = ["X10", "Scala", "Thorn"];
  val s1 = first(ss);
  assert s1.equals("X10");
}
\end{xten}
%~~siv
%}
% class Hook{ def run() {Example.example(); return true;}}
%~~neg

\end{ex}


\subsubsection{Sketch of X10 Type Inference for Method Calls}

When the X10 compiler sees a method call 
\begin{xtenmath}
a.m(b$_1$, $\ldots$,b$_n$)
\end{xtenmath}
and attempts to infer type parameters to
see if it could be a use of a
method 
\begin{xtenmath}
def m[X$_1$, $\ldots$, X$_t$](y$_1$: S$_1$, $\ldots$, y$_n$:S$_n$),
\end{xtenmath}
it reasons as follows. 



Suppose that \xcdmath"b$_i$" has type \xcdmath"T$_i$".  Then, X10 is seeking a
set of type {$B$} bindings 
\begin{xtenmath}
X$_j$ = U$_j$, 
\end{xtenmath}
for $1 \le j \le t$, 
such that 
\xcdmath"T$_i$ <: S$^*_i$" for {$1 \le i \le n$}, where \xcdmath"S$^*$" is
\xcd`S` with each type variable \xcdmath"X$_j$" replaced by the corresponding
\xcdmath"U$_j$".  If it can find such a {$B$}, it has a usable choice of type
arguments and can do the type inference.  If it cannot find {$B$}, then it
cannot do type inference.    (Note that X10's type inference algorithm is
incomplete -- there may {\em be} such a {$B$} that X10 cannot find.  If this
occurs in your program, you will have to write down the type arguments
explicitly.) 

Let $B_0$ be the set {$\{ T_i \subtype S_i | 1 \le i \le n\}$}.  Let
{$B_{n+1}$} be {$B_n$} with one element {$F \subtype G$} or 
{$F \typeeq G$} removed, and
{$C(F \subtype G)$} 
or {$C(F \typeeq G)$} (defined below) added.  Repeat this until 
{$B_n$} consists entirely of comparisons with type variables (\viz, 
\xcdmath"Y$_j$ == U", 
\xcdmath"Y$_j$ <: U", and
\xcdmath"Y$_j$ :> U"), 
or until some {$n$} exceeds a predefined compiler limit. 

The candidate inferred types may be read off of {$B_n$}.  The guessed binding
for \xcdmath"X$_j$" is: 
\begin{itemize}
\item If there is an equality \xcdmath"X$_j$==W" in {$B_n$}, then guess the
      binding \xcdmath"X$_j$=W".  Note that there may be several such
      equalities with different choices of \xcd`W`; pick any one.  If the
      chosen binding does not equal the others, the candidate binding will be
      rejected later. 
\item Otherwise, if there is one or more upper bounds 
\xcdmath"X$_j$ <: V$_k$" in {$B_n$}, guess the binding 
\xcdmath"X$_j$ = V$_+$", where 
\xcdmath"V$_+$" is the computed lower bound of all the \xcdmath"V$_k$"'s.
\item Otherwise, if there is one or more lower bounds 
\xcdmath"R$_k$ <: X$_j$", guess that
\xcdmath"X$_j$ = R$_+$", where 
\xcdmath"R$_+$" is the computed upper bound of all the \xcdmath"R$_k$"'s.
\end{itemize}
If this does not yield a binding for some variable \xcdmath"X$_j$", then type
inference fails.  Furthermore, if every variable \xcdmath"X$_j$" is given a
binding \xcdmath"U$_j$", but the 
bindings do not work --- 
that is, if 
\xcdmath"a.m[U$_1$, $\ldots$, U$_t$](b$_1$, $\ldots$,b$_n$)"
is not a call of 
the original method 
\xcdmath"def m[X$_1$, $\ldots$, X$_t$](y$_1$: S$_1$, $\ldots$, y$_n$:S$_n$)"
--- then type inference also fails.

\paragraph{Computation of the Replacement Elements}

Given a type relation
{$r$} of the form {$F \subtype G$}
or {$F \typeeq G$}, we compute the set {$C(r)$} of
replacement constraints.  There are a number of cases; we present only the
interesting ones. 

\begin{itemize}
\item If $F$ has the form \xcdmath"$F'${c}", then  
\xcdmath"$C(r)$" is defined to be
 \xcdmath"$F'$ == $G$" if $r$ is an equality, or 
 \xcdmath"$F'$ <: $G$" if {$r$} is a subtyping.
That is, we erase type constraints.  
Validity is not an issue at this point in the algorithm, as 
we check at the end that the result is valid.
Note that, if the equation had the form \xcdmath"Z{c} == A", it could be
solved by either \xcd`Z==A` or by \xcd`Z = A{c}`.  By dropping constraints in this
rule, we choose the former solution. 

\item Similarly, we drop constraints on {$G$} as well.

\item If {$F$} has the form \xcdmath"K[F$_1$, $\ldots$, F$_k$]"
and 
{$G$}
has the form \xcdmath"K[G$_1$, $\ldots$, G$_k$]", 
then {$C(r)$} has one type relation comparing each parameter of 
{$F$} with the corresponding one of {$G$}: 
\[C(r) = \{ F_l \typeeq G_l | 1 \le l \le k \} \]

For example, the constraint \xcdmath"List[X] == List[Y]" induces the
constraint \xcd`X==Y`.  
\xcd`List[X] <: List[Y]` also induces the same constraint.  The only way that
\xcd`List[X]` could be a subtype of \xcd`List[Y]` in X10 is if \xcd`X==Y`.
List of different types are incomparable.\footnote{The situation would be more
complex if X10 had covariant and contravariant types.}

\item Other cases are fairly routine.  \Eg, if {$F$} is a \xcd`type`-defined
      abbreviation, it is expanded.

\end{itemize}

\begin{ex}
Consider the program: 
%~~gen ^^^ Types1s4y
% package Types1s4y;
%~~vis
\begin{xten}
import x10.util.*;
class Cl[C1, C2, C3]{}
class Example {
  static def me[X1, X2](Cl[Int, X1, X2]) = 
     new Cl[X1, X2, Point]();
  static def example() {
    val a = new Cl[Int, Boolean, String]();
    val b : Cl[Boolean, String, Point] 
          = me[Boolean, String](a);
    val c : Cl[Boolean, String, Point] 
          = me(a);
  }
}
\end{xten}
%~~siv
%
%~~neg
The method call for \xcd`b` has explicit type parameters.  
The call for \xcd`c` infers the parameters.  The computation 
starts with one equation, saying that the type of the formal parameter of 
\xcd`me` has to be the same as the type of the actual parameter, \viz, the
type of \xcd`a`:
\begin{xtenmath}
Cl[Int, X1, X2] == Cl[Int, Boolean, String]
\end{xtenmath}
Note that both terms are \xcd`Cl` of three things. 
This is broken into three equations: 
\begin{xtenmath}
Int == Int
\end{xtenmath}
which is easy to satisfy,
\begin{xtenmath}
X1 == Boolean
\end{xtenmath}
which suggests a possible value for \xcd`X1`,  and 
\begin{xtenmath}
X2 == String
\end{xtenmath}
which suggests a value for \xcd`X2`.  
All of these equations are simple enough, so the algorithm terminates. 

Then, X10 confirms that the binding \xcd`X1==Boolean`, \xcd`X2==String`
atually generates a correct call, which it does.  
\end{ex}

\section{Type Dependencies}

Type definitions may not be circular, in the sense that no type may be its own
supertype, nor may it be a container for a supertype. This forbids interfaces
like \xcd`interface Loop extends Loop`, and indirect self-references such as
\xcd`interface A extends B.C` where \xcd`interface B extends A`.  
The formal definition of this is based on Java's.  

An {\em entity type} is a class, interface, or struct type.   

Entity type $E$ {\em directly depends on} entity type $F$ if $F$ is mentioned
in the \xcd`extends` or \xcd`implements` clause of $E$, either by itself or as
a qualifier within a super-entity-type name.  

\begin{ex}
In the following, \xcd`A` directly depends on \xcd`B`, \xcd`C`, \xcd`D`, 
\xcd`E`, and \xcd`F`.    It does not directly depend on \xcd`G`.
%~~gen ^^^ Types6a9m
% package Types6a9m;
% NOTEST
% class B{ static class C{}}
% class D{ static interface E{}}
% interface F[X]{}
% class G{}
%~~vis
\begin{xten}
class A extends B.C implements D.E, F[G] {}
\end{xten}
%~~siv
%
%~~neg

It is an ordinary programming idiom to use \xcd`A` as an argument to a generic
interface that \xcd`A` implements.  For example, \xcd`ComparableTo[T]`
describes things which can be compared to a value of type \xcd`T`. Saying that
\xcd`A` implements \xcd`ComparableTo[A]` means that one \xcd`A` can be
compared to another, which is reasonable and useful: 
%~~gen ^^^ Types2x6d
% package Types2x6d;
%~~vis
\begin{xten}
interface ComparableTo[T] {
  def eq(T):Boolean;
}
class A implements ComparableTo[A] {
  public def eq(other:A) = this.equals(other);
}
\end{xten}
%~~siv
%
%~~neg
\end{ex}

Entity type $E$ {\em depends on} entity type $F$ if
either $E$ directly depends on $F$, or $E$ directly depends on an entity type
that depends on $F$.   That is, the relation ``depends on'' is the transitive
closure of the relation ``directly depends on''.  

It is a static error if any entity type $E$ depends on itself.

\section{Limitations of Strict Typing}

X10's type checking provides substantial guarantees.  In most cases, a program
that passes the X10 type checker will not have any runtime type errors.
However, there are a modest number of compromises with practicality in the
type system: places where a program can pass the typechecker and still have a
type error.

\begin{enumerate}


\item The library type \xcd`IndexedMemoryChuck` provides a low-level interface
      to blocks of memory.  A few methods on that class are not type-safe. See
      the API if you must.

\item Custom serialization (\Sref{sect:ser+deser}) allows user code to
      construct new objects in ways that can subvert the type system.

\item Code written to use the underlying Java or C++ (\Sref{NativeCode}) can
      break X10's guarantees.

\end{enumerate}
	

\chapter{Variables}\label{XtenVariables}\index{variable}

%%OLDA variable is a storage location.  \Xten{} supports seven kinds of
%%OLDvariables: constant {\em class variables} (static variables), {\em
%%OLD  instance variables} (the instance fields of a class), {\em array
%%OLD  components}, {\em method parameters}, {\em constructor parameters},
%%OLD{\em exception-handler parameters} and {\em local variables}.

A {\em variable} is an X10 identifier associated with a value within some
context. Variable bindings have these essential properties:
\begin{itemize}
\item {\bf Type:} What sorts of values can be bound to the identifier;
\item {\bf Scope:} The region of code in which the identifier is associated
      with the entity;
\item {\bf Lifetime:} The interval of time in which the identifier is
      associated with the entity.
\item {\bf Visibility:} Which parts of the program can read or manipulate the
      value through the variable.
\end{itemize}



X10 has many varieties of variables, used for a number of purposes. 
\begin{itemize}
\item Class variables, also known as the static fields of a class, which hold
      their values for the lifetime of the class.  
\item Instance variables, which hold their values for the lifetime of an
      object;
\item Array elements, which are not individually named and hold their values
      for the lifetime of an array;
\item Formal parameters to methods, functions, and constructors, which hold
      their values for the duration of method (etc.) invocation;
\item Local variables, which hold their values for the duration of execution
      of a block.
\item Exception-handler parameters, which hold their values for the execution
      of the exception being handled. 
\end{itemize}
A few other kinds of things are called variables for historical reasons; \eg,
type parameters are often called type variables, despite not being variables
in this sense because they do not refer to X10 values.  Other named entities,
such as classes and methods, are not called variables.  However, all
name-to-whatever bindings enjoy similar concepts of scope and visibility.  

\begin{ex}
In the following example, 
\xcd`n` is an instance variable, and \xcd`nxt` is a
local variable defined within the method \xcd`bump`.\footnote{This code is
unnecessarily turgid for the sake of the example.  One would generally write
\xcd`public def bump() = ++n;`.   }
%~~gen ^^^ Vars10
% package Vars.For.Squares;
%~~vis
\begin{xten}
class Counter {
  private var n : Int = 0;
  public def bump() : Int {
    val nxt = n+1;
    n = nxt;
    return nxt;
    }
}
\end{xten}
%~~siv
% class Hook{ def run() { val c = new Counter(); val d = new Counter();
%   assert c.bump() == 1;  
%   assert c.bump() == 2;  
%   assert c.bump() == 3;  
%   assert c.bump() == 4;  
%   assert d.bump() == 1;  
%   assert c.bump() == 5;  
%   return true;
% } }
%~~neg
Both variables have type \xcd`Int` (or
perhaps something more specific).    The scope of \xcd`n` is the body of
\xcd`Counter`; the scope of \xcd`nxt` is the body of \xcd`bump`.  The
lifetime of \xcd`n` is the lifetime of the \xcd`Counter` object holding it;
the lifetime of \xcd`nxt` is the duration of the call to \xcd`bump`. Neither
variable can be seen from outside of its scope.
\end{ex}
\label{exploded-syntax}
\label{VariableDeclarations}
\index{variable declaration}


Variables whose value may not be changed after initialization are said to be
{\em immutable}, or {\em constants} (\Sref{FinalVariables}), or simply
\xcd`val` variables. Variables whose value may change are {\em mutable} or
simply \xcd`var` variables. \xcd`var` variables are declared by the \xcd`var`
keyword. \xcd`val` variables may be declared by the \xcd`val` keyword; when a
variable declaration does not include either \xcd`var` or \xcd`val`, it is
considered \xcd`val`. 

A variable---even a \xcd`val` -- can be declared in one statement, and then
initialized later on.  It must be initialized before it can be used
(\Sref{sect:DefiniteAssignment}).  


\begin{ex}
The following example illustrates many of the variations on variable
declaration: 
%~~gen ^^^ Vars20
%package Vars.For.Bears.In.Chairs;
%class VarExample{
%static def example() {
%~~vis
\begin{xten}
val a : Int = 0;               // Full 'val' syntax
b : Int = 0;                   // 'val' implied
val c = 0;                     // Type inferred
var d : Int = 0;               // Full 'var' syntax
var e : Int;                   // Not initialized
var f : Int{self != 100} = 0;  // Constrained type
val g : Int;                   // Init. deferred
if (a > b) g = 1; else g = 2;  // Init. done here.
\end{xten}
%~~siv
%}}
%~~neg
\end{ex}





\section{Immutable variables}
\label{FinalVariables}
\index{variable!immutable}
\index{immutable variable}
\index{variable!val}
\index{val}

%##(LocalVariableDeclarationStatement LocalVariableDeclaration
\begin{bbgrammar}
%(FROM #(prod:LocVarDeclnStmt)#)
     LocVarDeclnStmt \: LocVarDecln \xcd";" & (\ref{prod:LocVarDeclnStmt}) \\
%(FROM #(prod:LocVarDecln)#)
         LocVarDecln \: Mods\opt VarKeyword VariableDeclrs & (\ref{prod:LocVarDecln}) \\
                     \| Mods\opt VarDeclsWType \\
                     \| Mods\opt VarKeyword FormalDeclrs \\
\end{bbgrammar}
%##)

An immutable (\xcd`val`) variable can be given a value (by initialization or
assignment) at 
most once, and must be given a value before it is used.  Usually this is
achieved by declaring and initializing the variable in a single statement, 
such as \Xcd{val x = 3}, with syntax 
(\ref{prod:LocVarDecln}) using the {\it VariableDeclarators} or {\it
VarDelcsWType} alternatives.

\begin{ex}
After these declarations, \xcd`a` and \xcd`b` cannot be assigned to further,
or even redeclared:  
%~~gen ^^^ Vars30
% package Vars.In.Snares;
% class ABitTedious{
% def example() {
%~~vis
\begin{xten}
val a : Int = 10;
val b = (a+1)*(a-1);
// ERROR: a = 11;  // vals cannot be assigned to.
// ERROR: val a = 11; // no redeclaration.
\end{xten}
%~~siv
%}}
%~~neg

\end{ex}

In two special cases, the declaration and assignment are separate.  One 
case is how constructors give values to \xcd`val` fields of objects.  In this
case, production (\ref{prod:LocVarDecln}) is taken, with the {\it
FormalDeclarators} option, such as  \Xcd{var n:Int;}.  

\begin{ex} The
\xcd`Example` class has an immutable field \xcd`n`, which is given different
values depending on which constructor was called. \xcd`n` can't be given its
value by initialization when it is declared, since it is not knowable which
constructor is called at that point.  
%~~gen ^^^ Vars40
% package Vars.For.Cares;
%~~vis
\begin{xten}
class Example {
  val n : Int; // not initialized here
  def this() { n = 1; }
  def this(dummy:Boolean) { n = 2;}
}
\end{xten}
%~~siv
%
%~~neg
\end{ex}


The other case of separating declaration and assignment is in function
and method call, described in \Sref{sect:formal-parameters}.  The formal
parameters are bound to the corresponding actual 
parameters, but the binding does not happen until the function is called.  

\begin{ex}
In
the code below, \xcd`x` is initialized to \xcd`3` in the first call and
\xcd`4` in the second.
%~~gen ^^^ Vars50
%package Vars.For.Swears;
%class Examplement {
%static def whatever() {
%~~vis
\begin{xten}
val sq = (x:Int) => x*x;
x10.io.Console.OUT.println("3 squared = " + sq(3));
x10.io.Console.OUT.println("4 squared = " + sq(4));
\end{xten}
%~~siv
%}}
%~~neg
\end{ex}




%%IMMUTABLE%% An immutable variable satisfies two conditions: 
%%IMMUTABLE%% \begin{itemize}
%%IMMUTABLE%% \item it can be assigned to at most once, 
%%IMMUTABLE%% \item it must be assigned to before use. 
%%IMMUTABLE%% \end{itemize}
%%IMMUTABLE%% 
%%IMMUTABLE%% \Xten{} follows \java{} language rules in this respect \cite[\S
%%IMMUTABLE%% 4.5.4,8.3.1.2,16]{jls2}. Briefly, the compiler must undertake a
%%IMMUTABLE%% specific analysis to statically guarantee the two properties above.
%%IMMUTABLE%% 
%%IMMUTABLE%% Immutable local variables and fields are defined by the \xcd"val"
%%IMMUTABLE%% keyword.  Elements of value arrays are also immutable.
%%IMMUTABLE%% 
%%IMMUTABLE%% \oldtodo{Check if this analysis needs to be revisited.}

\section{Initial values of variables}
\label{NullaryConstructor}\index{nullary constructor}
\index{initial value}
\index{initialization}


Every assignment, binding, or initialization to a variable of type \xcd`T{c}`
must be an instance of type \xcd`T` satisfying the constraint \xcd`{c}`.
Variables must be given a value before they are used. This may be done by
initialization -- giving a variable a value as part
of its declaration. 

\begin{ex}
These variables are all initialized: 
%~~gen ^^^ Vars60
%package Vars.For.Bears;
%class VarsForBears{
%def check() {
%~~vis
\begin{xten}
  val immut : Int = 3;
  var mutab : Int = immut;
  val use = immut + mutab;
\end{xten}
%~~siv
%}}
%~~neg
\end{ex}

Or, a variable may be given a value by an assignment.  \xcd`var` variables may
be assigned to repeatedly.  \xcd`val` variables may only be assigned once; the
compiler will ensure that they are assigned before they are used.

\begin{ex}
The variables in the following example are given their initial values by
assignment.  Note that they could not be used before those assignments,
nor could \xcd`immu` be assigned repeatedly.
%~~gen ^^^ Vars70
%package Vars.For.Stars;
%abstract class VarsForStars{
% abstract def cointoss(): Boolean;
% abstract def println(Any):void;
%def check() {
%~~vis
\begin{xten}
  var muta : Int;
  // ERROR:  println(muta);
  muta = 4;
  val use2A = muta * 10;
  val immu : Int;
  // ERROR: println(immu);
  if (cointoss())   {immu = 1;}
  else              {immu = use2A;}
  val use2B = immu * 10;
  // ERROR: immu = 5;
\end{xten}
%~~siv
%}}
%~~neg
\end{ex}

Every class variable must be initialized before it is read, through
the execution of an explicit initializer. Every
instance variable must be initialized before it is read, through the
execution of an explicit or implicit initializer or a constructor.
Implicit initializers initialize \xcd`var`s to the default values of their
types (\Sref{DefaultValues}). Variables of types which do not have default
values are not implicitly initialized.



Each method and constructor parameter is initialized to the
corresponding argument value provided by the invoker of the method. An
exception-handling parameter is initialized to the object thrown by
the exception. A local variable must be explicitly given a value by
initialization or assignment, in a way that the compiler can verify
using the rules for definite assignment (\Sref{sect:DefiniteAssignment}).


\section{Destructuring syntax}
\index{variable declarator!destructuring}
\index{destructuring}
\Xten{} permits a \emph{destructuring} syntax for local variable
declarations with explicit initializers,  and for formal parameters, of type
\xcd`Point`, \Sref{point-syntax} and \xcd`Array`, \Sref{XtenArrays}.
A point is a sequence of zero or more \xcd`Int`-valued coordinates; an array
is an indexed collection of data. 
It is often useful to get at the coordinates or elements directly, in
variables.

%##(VariableDeclarator
\begin{bbgrammar}
%(FROM #(prod:VariableDeclr)#)
       VariableDeclr \: Id HasResultType\opt \xcd"=" VariableInitializer & (\ref{prod:VariableDeclr}) \\
                     \| \xcd"[" IdList \xcd"]" HasResultType\opt \xcd"=" VariableInitializer \\
                     \| Id \xcd"[" IdList \xcd"]" HasResultType\opt \xcd"=" VariableInitializer \\
\end{bbgrammar}
%##)

The syntax \xcdmath"val [a$_1$, $\ldots$, a$_n$] = e;", 
where \xcd`e` is a \xcd`Point`,
declares {$n$}
\xcd`Int` variables, bound to the precisely {$n$} components of the \xcd`Point` value of
\xcd`e`; it is an error if \xcd`e` is not a \xcd`Point` with precisely {$n$} components.
The syntax \xcdmath"val p[a$_1$, $\ldots$, a$_n$] = e;"  is similar, but also
declares the variable \xcd`p` to be of type \xcdmath"Point(n)".  


The syntax \xcdmath"val [a$_1$, $\ldots$, a$_n$] = e;", 
where \xcd`e` is an \xcd`Array[T]` for some type \xcd`T`,
declares {$n$}
variables of type \xcd`T`, bound to the precisely {$n$} components of the \xcd`Array[T]` value of
\xcd`e`; it is an error if \xcd`e` is not a \xcd`Array[T]` 
with \xcd`rank==1` and \xcdmath"size==$n$". 
The syntax \xcdmath"val p[a$_1$, $\ldots$, a$_n$] = e;"  is similar, but also
declares the variable \xcd`p` to be of type
\xcdmath"Array[T]{rank==1,size==n}".   


\begin{ex}
The following code makes an anonymous point with one coordinate \xcd`11`, and
binds \xcd`i` to \xcd`11`.  Then it makes a point with coordinates \xcd`22`
and \xcd`33`, binds \xcd`p` to that point, and \xcd`j` and \xcd`k` to \xcd`22`
and \xcd`33` respectively.
%~~gen ^^^ Vars80
% package Vars.For.Glares;
% class Example {
% static def example () {
%~~vis
\begin{xten}
val [i] : Point = Point.make(11);
assert i == 11;
val p[j,k] = Point.make(22,33);
assert j == 22 && k == 33;
val q[l,m] = [44,55] as Point; 
assert l == 44 && m == 55;
//ERROR: val [n] = p;
\end{xten}
%~~siv
%}}
% class Hook{ def run() {Example.example(); return true;}}
%~~neg

Destructuring is allowed wherever a \xcd`Point` or \xcd`Array[T]` variable is
declared, \eg, as the formal parameters of a method.
\begin{ex}
The methods below take a single argument each: a three-element point for
\xcd`example1`, a three-element array for \xcd`example2`.  The argument itself
is bound to \xcd`x` in both cases, and its elements are bound to \xcd`a`,
\xcd`b`, and \xcd`c`.  
%~~gen ^^^ Vars2e6j
% package Vars2e6j;
%class Example {
%~~vis
\begin{xten}
static def example1(x[a,b,c]:Point){}
static def example2(x[a,b,c]:Array[String]{rank==1,size==3}){}
\end{xten}
%~~siv
%}
%~~neg
\end{ex}

\end{ex}


\section{Formal parameters}
\label{sect:formal-parameters}
\index{formal parameter}
\index{parameter}


Formal parameters are the variables which hold values transmitted into a
method or function.  
They are always declared with a type.  (Type inference is not
available, because there is no single expression to deduce a type from.)
The variable name can be omitted if it is not to be used in the
scope of the declaration, as in the type of the method 
\xcd`static def main(Rail[String]):void` executed at the start of a program that
does not use its command-line arguments.

\xcd`var` and \xcd`val` behave just as they do for local
variables, \Sref{local-variables}.  In particular, the following \xcd`inc`
method is allowed, but, unlike some languages, does {\em not} increment its
actual parameter.  \xcd`inc(j)` creates a new local 
variable \xcd`i` for the method call, initializes \xcd`i` with the value of
\xcd`j`, increments \xcd`i`, and then returns.  \xcd`j` is never changed.
%~~gen ^^^ Vars100
% package Vars.For.Squares.Of.Mares;
% class Example {
%~~vis
\begin{xten}
static def inc(var i:Int) { i += 1; }
static def example() {
   var j : Int = 0;
   assert j == 0;
   inc(j);
   assert j == 0;
}
\end{xten}
%~~siv
%}
% class Hook{ def run() {Example.example(); return true;}}
%~~neg


\section{Local variables and Type Inference}
\label{local-variables}
\index{variable!local}
\index{local variable}
Local variables are declared in a limited scope, and, dynamically, keep their
values only for so long as the scope is being executed.  They may be \xcd`var`
or \xcd`val`.  
They may have 
initializer expressions: \xcd`var i:Int = 1;` introduces 
a variable \xcd`i` and initializes it to 1.
If the variable is immutable
(\xcd"val")
the type may be omitted and
inferred from the initializer type (\Sref{TypeInference}).

The variable declaration \xcd`val x:T=e;` confirms that \xcd`e`'s value is of
type \xcd`T`, and then introduces the variable \xcd`x` with type \xcd`T`.  For
example, consider a class \xcd`Tub` with a property \xcd`p`.
%~~gen ^^^ Vars_Tub
% package Vars.Local;
%~~vis
\begin{xten}
class Tub(p:Int){
  def this(pp:Int):Tub{self.p==pp} {property(pp);}
  def example() {
    val t : Tub = new Tub(3);
  }
}
\end{xten}
%~~siv
%
%~~neg
\noindent
produces a variable \xcd`t` of type \xcd`Tub`, even though the expression
\xcd`new Tub(3)` produces a value of type \xcd`Tub{self.p==3}` -- that is, a
\xcd`Tub`  whose \xcd`p` field is 3.  This can be inconvenient when the
constraint information is required.

\index{\Xcd{<:}}
\index{documentation type declaration}
Including type information in variable declarations is generally good
programming practice: it explains to both the compiler and human readers
something of the intent of the variable.  However, including types in 
\xcd`val t:T=e` can obliterate helpful information.  So, X10 allows a {\em
documentation type declaration}, written 
\begin{xtenmath}
val t <: T = e
\end{xtenmath}
This 
has the same effect as \xcd`val t = e`, giving \xcd`t` the full type inferred
from \xcd`e`; but it also confirms statically that that type is at least
\xcd`T`.  

\begin{ex}The following gives \xcd`t` the type \xcd`Tub{self.p==3}` as
desired.  However, a similar declaration with an inappropriate type will fail
to compile.
%~~gen ^^^ Vars_Var_Bounded
% package Vars.Local.not.the.express.plz;
% class Tub(p:Int){
%   def this(pp:Int):Tub{self.p==pp} {property(pp);}
%   def example() {
%     val t : Tub = new Tub(3);
%   }
% }
% class TubBounded{
% def example() {
%~~vis
\begin{xten}
   val t <: Tub = new Tub(3);
   // ERROR: val u <: Int = new Tub(3);
\end{xten}
%~~siv
%}}
%~~neg

\end{ex}



\section{Fields}
\index{field}
\index{object!field}
\index{struct!field}
\index{class!field}

%##(FieldDeclarators FieldDecln FieldDeclarator HasResultType  Mod
\begin{bbgrammar}
%(FROM #(prod:FieldDeclrs)#)
         FieldDeclrs \: FieldDeclr & (\ref{prod:FieldDeclrs}) \\
                     \| FieldDeclrs \xcd"," FieldDeclr \\
%(FROM #(prod:FieldDecln)#)
          FieldDecln \: Mods\opt VarKeyword FieldDeclrs \xcd";" & (\ref{prod:FieldDecln}) \\
                     \| Mods\opt FieldDeclrs \xcd";" \\
%(FROM #(prod:FieldDeclr)#)
          FieldDeclr \: Id HasResultType & (\ref{prod:FieldDeclr}) \\
                     \| Id HasResultType\opt \xcd"=" VariableInitializer \\
%(FROM #(prod:HasResultType)#)
       HasResultType \: ResultType & (\ref{prod:HasResultType}) \\
                     \| \xcd"<:" Type \\
%(FROM #(prod:Mod)#)
                 Mod \: \xcd"abstract" & (\ref{prod:Mod}) \\
                     \| Annotation \\
                     \| \xcd"atomic" \\
                     \| \xcd"final" \\
                     \| \xcd"native" \\
                     \| \xcd"private" \\
                     \| \xcd"protected" \\
                     \| \xcd"public" \\
                     \| \xcd"static" \\
                     \| \xcd"transient" \\
                     \| \xcd"clocked" \\
\end{bbgrammar}
%##)

Like most other kinds of variables in X10, 
the fields of an object can be either \xcd`val` or \xcd`var`. 
\xcd`val` fields can be \xcd`static` (\Sref{FieldDefinitions}).
Field declarations may have optional
initializer expressions, as for local variables, \Sref{local-variables}.
\xcd`var` fields without an initializer are initialized with the default value
of their type. \xcd`val` fields without an initializer must be initialized by
each constructor.


For \xcd`val` fields, as for \xcd`val` local variables, the type may be
omitted and inferred from the initializer type (\Sref{TypeInference}).
\xcd`var` fields, like \xcd`var` local variables, must be declared with a type.



%%GRAM%% \begin{grammar}
%%GRAM%% FieldDeclaration
%%GRAM%%         \: FieldModifier\star \xcd"var" FieldDeclaratorsWithType \\&& ( \xcd"," FieldDeclaratorsWithType )\star \\
%%GRAM%%         \| FieldModifier\star \xcd"val" FieldDeclarators \\&& ( \xcd"," FieldDeclarators )\star \\
%%GRAM%%         \| FieldModifier\star FieldDeclaratorsWithType \\&& ( \xcd"," FieldDeclaratorsWithType )\star \\
%%GRAM%% FieldDeclarators
%%GRAM%%         \: FieldDeclaratorsWithType \\
%%GRAM%%         \: FieldDeclaratorWithInit \\
%%GRAM%% FieldDeclaratorId
%%GRAM%%         \: Identifier  \\
%%GRAM%% FieldDeclaratorWithInit
%%GRAM%%         \: FieldDeclaratorId Init \\
%%GRAM%%         \| FieldDeclaratorId ResultType Init \\
%%GRAM%% FieldDeclaratorsWithType
%%GRAM%%         \: FieldDeclaratorId ( \xcd"," FieldDeclaratorId )\star ResultType \\
%%GRAM%% FieldModifier \: Annotation \\
%%GRAM%%                 \| \xcd"static" \\ \| \xcd`property` \\ \| \xcd`global` \\
%%GRAM%% \end{grammar}
%%GRAM%% 
%%GRAM%% 

%%ACC%%  \section{Accumulator Variables}
%%ACC%%  
%%ACC%%  Accumulator variables allow the accumulation of partial results to produce a
%%ACC%%  final result.  For example, an accumulator variable could compute a running
%%ACC%%  sum, product, maximum, or minimum of a collection of numbers.  In particular,
%%ACC%%  many concurrent activites can accumulate safely into the {\em same} local
%%ACC%%  variable, without need for \Xcd{atomic} blocks or other explicit coordination.  
%%ACC%%  
%%ACC%%  An accumulator variable is associated with a {\em reducer}, which explains how
%%ACC%%  new partial values are accumulated.
%%ACC%%  
%%ACC%%  \subsection{Reducers}
%%ACC%%  
%%ACC%%  A notion of accumulation has two aspects: 
%%ACC%%  \begin{enumerate}
%%ACC%%  \item A {\bf zero} value, which is the initial value of the accumulator,
%%ACC%%        before any partial results have been included.  When accumulating a sum,
%%ACC%%        the zero value is \Xcd{0}; when accumulating a product, it is \Xcd{1}.
%%ACC%%  \item A {\bf combining function}, explaining how to combine two partial
%%ACC%%        accumulations into a whole one.  When accumulating a sum, partial sums
%%ACC%%        should be added together; for a product, they should be multiplied.  
%%ACC%%  \end{enumerate}
%%ACC%%  
%%ACC%%  In X10, this is represented as a value of type
%%ACC%%  \Xcd{x10.lang.Reducer[T]}: 
%%ACC%%  %~acc~gen
%%ACC%%  %package Vars.Notx10lang.Reducerererer;
%%ACC%%  %~acc~vis
%%ACC%%  \begin{xten}
%%ACC%%  struct Reducer[T](zero:T, apply: (T,T)=>T){}
%%ACC%%  \end{xten}
%%ACC%%  %~acc~siv
%%ACC%%  %
%%ACC%%  %~acc~neg
%%ACC%%  \noindent 
%%ACC%%  If \Xcd{r:Reducer[T]}, then \Xcd{r.zero} is the zero element, and
%%ACC%%  \Xcd{r(a,b)} --- which can also be written \Xcd{r.apply(a,b)} --- is the
%%ACC%%  combination of \Xcd{a} and \Xcd{b}.
%%ACC%%  
%%ACC%%  For example, the reducers for adding and multiplying integers are: 
%%ACC%%  %~acc~gen
%%ACC%%  %package Vars.Notx10lang.Reducererererererer;
%%ACC%%  %struct Reducer[T](zero:T, apply: (T,T)=>T){}
%%ACC%%  %class Example{
%%ACC%%  %~acc~vis
%%ACC%%  \begin{xten}
%%ACC%%  val summer = Reducer[Int](0, Int.+);
%%ACC%%  val producter = Reducer[Int](1, Int.*);
%%ACC%%  \end{xten}
%%ACC%%  %~acc~siv
%%ACC%%  %}
%%ACC%%  %~acc~neg
%%ACC%%  
%%ACC%%  
%%ACC%%  Reduction is guaranteed to be deterministic if the reducer is {\em
%%ACC%%  Abelian},\footnote{This term is borrowed from abstract algebra, where such a
%%ACC%%  reducer, together with its type, forms an Abelian monoid.}
%%ACC%%  that is, 
%%ACC%%  \begin{enumerate}
%%ACC%%  \item \Xcd{r.apply} is pure; that is, has no side effects;
%%ACC%%  \item \Xcd{r.apply} is commutative; that is, \Xcd{r(a,b) == r(b,a)} for all
%%ACC%%        inputs \Xcd{a} and \Xcd{b};
%%ACC%%  \item \Xcd{r.apply} is associative; that is, 
%%ACC%%        \Xcd{r(a,r(b,c)) == r(r(a,b),c)} for all \Xcd{a}, \Xcd{b}, and \Xcd{c}.
%%ACC%%  \item \Xcd{r.zero} is the identity element for \Xcd{r.apply}; that is, 
%%ACC%%        \Xcd{r(a, r.zero) == a}
%%ACC%%        for all \Xcd{a}.
%%ACC%%  \end{enumerate}
%%ACC%%  
%%ACC%%  
%%ACC%%  
%%ACC%%  
%%ACC%%  \Xcd{summer} and \Xcd{producter} satisfy all these conditions, and give
%%ACC%%  determinate reductions. The compiler does not require or check these, though.
%%ACC%%  
%%ACC%%  
%%ACC%%  \subsection{Accumulators}
%%ACC%%  
%%ACC%%  If \Xcd{r} is a  value of type \Xcd{Reducer[T]}, then an accumulator of type
%%ACC%%  \Xcd{T} using \Xcd{r} is declared as:
%%ACC%%  %~accTODO~gen
%%ACC%%  % package Vars.Accumulators.Basic.Little.Idea;
%%ACC%%  % class C[T]{
%%ACC%%  % static def example (r:Reducer[T]) {
%%ACC%%  %~accTODO~vis
%%ACC%%  \begin{xten}
%%ACC%%  acc(r) x : T;
%%ACC%%  acc(r) y; 
%%ACC%%  \end{xten}
%%ACC%%  %~accTODO~siv
%%ACC%%  %
%%ACC%%  %~accTODO~neg
%%ACC%%  The type declaration \Xcd{T} is optional; if specified, it must be the same
%%ACC%%  type that the reducer \Xcd{r} uses.
%%ACC%%  
%%ACC%%  \subsection{Sequential Use of Accumulators}
%%ACC%%  
%%ACC%%  The sequential use of accumulator variables is straightforward, and could be
%%ACC%%  done as easily without accumulators.  (The power of accumulators is in their
%%ACC%%  concurrent use, \Sref{ConcurrentUseOfAccumulators}.)
%%ACC%%  
%%ACC%%  A variable declared as \Xcd{acc(r) x:T;} is initialized to \Xcd{r.zero}.  
%%ACC%%  
%%ACC%%  Assignment of values of \Xcd{acc} variables has nonstandard semantics.
%%ACC%%  \Xcd{x = v;} causes the value \Xcd{r(v,x)} to be stored in \Xcd{x} --- in
%%ACC%%  particular, {\em not} the value of \Xcd{v}.
%%ACC%%  
%%ACC%%  Reading a value from an accumulator retrieves the current accumulation.
%%ACC%%  
%%ACC%%  For example, the sum and product of a list \Xcd{L} of integers can be computed
%%ACC%%  by: 
%%ACC%%  %~accTODO~gen
%%ACC%%  %package Vars.Accumulators.Are.For.Bisimulators;
%%ACC%%  % import java.util.*;
%%ACC%%  % class Example{
%%ACC%%  % static def example(L: List[Int]) {
%%ACC%%  %~accTODO~vis
%%ACC%%  \begin{xten}
%%ACC%%  val summer = Reducer[Int](0, Int.+);
%%ACC%%  val producter = Reducer[Int](1, Int.*);
%%ACC%%  acc(summer) sum;
%%ACC%%  acc(producter) prod;
%%ACC%%  for (x in L) {
%%ACC%%    sum = x;
%%ACC%%    prod = x;
%%ACC%%  }
%%ACC%%  x10.io.Console.OUT.println("Sum = " + sum + "; Product = " + prod);
%%ACC%%  \end{xten}
%%ACC%%  %~accTODO~siv
%%ACC%%  %
%%ACC%%  %~accTODO~neg
%%ACC%%  
%%ACC%%  
%%ACC%%  
%%ACC%%  \subsection{Concurrent Use of Accumulators}
%%ACC%%  \label{ConcurrentUseOfAccumulators}
%%ACC%%  \index{accumulator!and activities}
%%ACC%%  
%%ACC%%  Accumulator variables are restricted and synchronized in ways that make them
%%ACC%%  ideally suited for concurrent accumulation of data.   The {\em governing
%%ACC%%  activity} of an accumulator is the activity in which the \Xcd{acc} variable is
%%ACC%%  declared.  
%%ACC%%  
%%ACC%%  \begin{enumerate}
%%ACC%%  \item The governing activity can read the accumulator at any point that it has
%%ACC%%        no running sub-activities.  
%%ACC%%  \item Any activity that has lexical access to the accumulator can write to it.  
%%ACC%%        All
%%ACC%%        writes are performed atomically, without need for \Xcd{atomic} or other
%%ACC%%        concurrency control.
%%ACC%%  \end{enumerate}
%%ACC%%  
%%ACC%%  If the reducer is Abelian, this guarantees that \Xcd{acc} variables cannot
%%ACC%%  cause race conditions; the result of such a computation is determinate,
%%ACC%%  independent of the scheduling of activities. Read-read conflicts are
%%ACC%%  impossible, as only a single activity, the governing activity, can read the
%%ACC%%  \Xcd{acc} variable. Read-write conflicts are impossible, as reads are only
%%ACC%%  allowed at points where the only activity which can refer to the \Xcd{acc}
%%ACC%%  variable is the governing activity. Two activities may try to write the
%%ACC%%  \Xcd{acc} variable at the same time. The writes are performed atomically, so
%%ACC%%  they behave as if they happened in some (arbitrary) order---and, because the
%%ACC%%  reducer is Abelian, the order of writes doesn't matter.
%%ACC%%  
%%ACC%%  If the reducer is not Abelian---\eg, it is accumulating a string result by
%%ACC%%  concatenating a lot of partial strings together---the result is indeterminate.
%%ACC%%  However, because the accumulator operations are atomic, it will be the result
%%ACC%%  of {\em some} combination of the individual elements by the reduction
%%ACC%%  operation, \eg, the concatenation of the partial strings in {\em some} order.  
%%ACC%%  
%%ACC%%  
%%ACC%%  
%%ACC%%  For example, the following code computes triangle numbers {$\sum_{i=1}^{n}i$}
%%ACC%%  concurrently.\footnote{This program is highly inefficient. Even ignoring the
%%ACC%%    constant-time formula {$\sum_{i=1}^{n}i = \frac{n(n+1)}{2}$}, this program
%%ACC%%    incurs the cost of starting {$n$} activities and coordinating {$n$} accesses
%%ACC%%    to the accumulator. Accumulator variables are of most value in multi-place,
%%ACC%%    multi-core computations.}
%%ACC%%  
%%ACC%%  
%%ACC%%  %~accTODO~gen
%%ACC%%  %package Vars.Accumulator.Concurrency.Example;
%%ACC%%  %class Example{
%%ACC%%  %
%%ACC%%  %~accTODO~vis
%%ACC%%  \begin{xten}
%%ACC%%  def triangle(n:Int) {
%%ACC%%    val summer = Reducer[Int](0, Int.+);
%%ACC%%    acc(summer) sum; 
%%ACC%%    finish {
%%ACC%%      for([i] in 1..n) async {
%%ACC%%        sum = i;  // (A)
%%ACC%%      }
%%ACC%%      // (C)
%%ACC%%    }
%%ACC%%    return sum; // (B)
%%ACC%%  }
%%ACC%%  \end{xten}
%%ACC%%  %~accTODO~siv
%%ACC%%  %}
%%ACC%%  %~accTODO~neg
%%ACC%%  
%%ACC%%  The governing activity of the \Xcd{acc} variable \Xcd{sum} is the activity
%%ACC%%  including the body of \Xcd{triangle}.  It starts up \Xcd{n} sub-activities,
%%ACC%%  each of which adds one value to \Xcd{sum} at point \Xcd{(A)}.  Note that these
%%ACC%%  activities cannot {\em read} the value of \Xcd{sum}---only the governing
%%ACC%%  activity can do that---but they can update it.  
%%ACC%%  
%%ACC%%  At point \Xcd{(B)}, \Xcd{triangle} returns the value in \Xcd{sum}. It is
%%ACC%%  clear, from the \Xcd{finish} statement, that all sub-activities started by the
%%ACC%%  governing process have finished at this point. X10 forbids reading of
%%ACC%%  \Xcd{sum}, even by the governing process, at point \Xcd{(C)}, since
%%ACC%%  sub-activities writing into it could still be active when the governing
%%ACC%%  activity reaches this point.  The \Xcd{return sum;} statement could not be
%%ACC%%  moved to \Xcd{(C)}, which is good, because the program would be wrong if it
%%ACC%%  were there.
%%ACC%%  
%%ACC%%  
%%ACC%%  
%%ACC%%  
%%ACC%%  \subsubsection{Accumulators and Places}
%%ACC%%  \index{accumulator!and places} Activity variables can be read and written from
%%ACC%%  any place, without need for \Xcd{GlobalRef}s. We may spread the previous
%%ACC%%  computation out among all the available processors by simply sticking in an
%%ACC%%  \Xcd{at(...)} statement at point \Xcd{(D)}, without need for other
%%ACC%%  restructuring of the program.
%%ACC%%  
%%ACC%%  %~accTODO~gen
%%ACC%%  %package Vars.Accumulator.Concurrency.Example.Multiplacey;
%%ACC%%  %class Example{
%%ACC%%  %~accTODO~vis
%%ACC%%  \begin{xten}
%%ACC%%  def triangle(n:Int) {
%%ACC%%    val summer = Reducer[Int](0, Int.+);
%%ACC%%    acc(summer) sum; 
%%ACC%%    finish {
%%ACC%%      for([i] in 1..n) async 
%%ACC%%        at(Places.place(i % Places.MAX_PLACES) { //(D)
%%ACC%%          sum = i;  // (A)
%%ACC%%      }
%%ACC%%    }
%%ACC%%    return sum; // (B)
%%ACC%%  }
%%ACC%%  \end{xten}
%%ACC%%  %~accTODO~siv
%%ACC%%  %}
%%ACC%%  %~accTODO~neg
%%ACC%%  
%%ACC%%  \subsubsection{Accumulator Parameters}
%%ACC%%  \index{accumulator variables!as parameters}
%%ACC%%  \index{parameters!accumulator}
%%ACC%%  
%%ACC%%  Accumulators can be passed to methods and closures, by giving the keyword 
%%ACC%%  \Xcd{acc} instead of \Xcd{var} or \Xcd{val}.  Reducers are not specified; each
%%ACC%%  accumulator comes with its own reducer.  However, the type \Xcd{T} of the
%%ACC%%  accumulator {\em is} required.
%%ACC%%  
%%ACC%%  For example, the following method takes a list of numbers, and accumulates
%%ACC%%  those that are divisible by 2 in \Xcd{evens}, and those that are divisible by
%%ACC%%  3 in \Xcd{triples}: 
%%ACC%%  %~accTODO~gen
%%ACC%%  %package Vars.accumulators.parameters.oscillators.convulsitors.proximators;
%%ACC%%  %import x10.util.*;
%%ACC%%  %class Whatever {
%%ACC%%  %~accTODO~vis
%%ACC%%  \begin{xten}
%%ACC%%  static def split23(L:List[Int], acc evens:Int, acc triples:Int) {
%%ACC%%    for(n in L) {
%%ACC%%       if (n % 2 == 0) evens = n;
%%ACC%%       if (n % 3 == 0) triples = n;
%%ACC%%    }
%%ACC%%  }
%%ACC%%  static val summer = Reducer[Int](0, Int.+);
%%ACC%%  static val producter = Reducer[Int](1, Int.*);
%%ACC%%  static def sumEvenPlusProdTriple(L:List[Int]) {
%%ACC%%    acc(summer) sumEven;
%%ACC%%    acc(producter) prodTriple;
%%ACC%%    split23(L, sumEven, prodTriple);
%%ACC%%    return sumEven + prodTriple;
%%ACC%%  }
%%ACC%%  \end{xten}
%%ACC%%  %~accTODO~siv
%%ACC%%  %}
%%ACC%%  %~accTODO~neg
%%ACC%%  
%%ACC%%  \subsection{Indexed Accumulators}
%%ACC%%  \index{accumulator!indexed}
%%ACC%%  \index{accumulator!array}
%%ACC%%  
%%ACC%%  
%%ACC%%  \noo{Define this!}
%%ACC%%  
%%ACC%%  %~accTODO~gen
%%ACC%%  % package Vars.Indexed.Accumulators;
%%ACC%%  %~accTODO~vis
%%ACC%%  \begin{xten}
%%ACC%%  class BoolAccum implements SelfAccumulator[Boolean, Int] {
%%ACC%%    var sumTrue = 0, sumFalse = 0;
%%ACC%%    def update(k:Boolean, v:Int) { 
%%ACC%%       if (k) sumTrue += k; else sumFalse += k;
%%ACC%%    }
%%ACC%%    def update(ks:Array[Boolean]{rail}, vs:Array[Int]{ks.size == vs.size}) {
%%ACC%%       for([i] in ks.region) update(ks(i), vs(i));  }
%%ACC%%    
%%ACC%%  }
%%ACC%%  \end{xten}
%%ACC%%  %~accTODO~siv
%%ACC%%  %
%%ACC%%  %~accTODO~neg

\chapter{Names and packages}
\label{packages} \index{names}\index{packages}\index{public}\index{protected}\index{private}

\Xten{} supports mechanisms for names and packages in the style of Java
\cite[\S 6,\S 7]{jls2}, including \xcd"public", \xcd"protected", \xcd"private"
and package-specific access control.

\section{Packages}

A package is a named collection of top-level type declarations, \viz, class,
interface, and struct declarations. Package names are sequences of
identifiers, like \xcd`x10.lang` and \xcd`com.ibm.museum`. The multiple names
are simply a convenience. Packages \xcd`a`, \xcd`a.b`, and \xcd`a.c` have only
a very tenuous relationship, despite the similarity of their names.

Packages and protection modifiers determine which top-level names can be used
where. Only the \xcd`public` members of package \xcd`pack.age` can be accessed
outside of \xcd`pack.age` itself.  
%~~gen~~Stimulus
%
%~~vis
\begin{xten}
package pack.age;
class Deal {
  public def make() {}
}
public class Stimulus {
  private def taxCut() = true;
  protected def benefits() = true;
  public def jobCreation() = true;
  /*package*/ def jumpstart() = true;
}
\end{xten}
%~~siv
%
%~~neg

The class \xcd`Stimulus` can be referred to from anywhere outside of
\xcd`pack.age` by its full name of \xcd`pack.age.Stimulus`, or can be imported
and referred to simply as \xcd`Stimulus`.  The public \xcd`jobCreation()`
method of a \xcd`Stimulus` can be referred to from anywhere as well; the other
methods have smaller visibility.  The non-\xcd`public` class \xcd`Deal` cannot
be used from outside of \xcd`pack.age`.  



\subsection{Name Collisions}

It is a static error for a package to have two members, or apparent members,
with the same name.  For example, package \xcd`pack.age` cannot define two
classes both named \xcd`Crash`, nor a class and an interface with that name.

Furthermore, \xcd`pack.age` cannot define a member \xcd`Crash` if there is
another package named \xcd`pack.age.Crash`, nor vice-versa. (This prohibition
is the only actual relationship between the two packages.)  This prevents the
ambiguity of whether \xcd`pack.age.Crash` refers to the class or the package.  
Note that the naming convention that package names are lower-case and package
members are capitalized prevents such collisions.


\section{\xcd`import` Declarations}

Any public member of a package can be referred to from anywhere through a
fully-qualified name: \xcd`pack.age.Stimulus`.    

Often, this is too awkward.  X10 has two ways to allow code outside of a class
to refer to the class by its short name (\xcd`Stimulus`): single-type imports
and on-demand imports.   

Imports of either kind appear at the start of the file, immediately after the
\xcd`package` directive if there is one; their scope is the whole file.

\subsection{Single-Type Import}

The declaration \xcd`import ` {\em TypeName} \xcd`;` imports a single type
into the current namespace.  The type it imports must be a fully-qualified
name of an extant type, and it must either be in the same package (in which
case the \xcd`import` is redundant) or be declared \xcd`public`.  

Furthermore, when importing \xcd`pack.age.T`, there must not be another type
named \xcd`T` at that point: neither a  \xcd`T` declared in \xcd`pack.age`,
nor a \xcd`inst.ant.T` imported from some other package.

\subsection{Automatic Import}

The automatic import \xcd`import pack.age.*;`, loosely, imports all the public
members of \xcd`pack.age`.  In fact, it does so somewhat carefully, avoiding
certain errors that could occur if it were done naively.  Types defined in the
current package, and those imported by single-type imports, shadow those
imported by automatic imports.  

\subsection{Implicit Imports}

The packages \xcd`x10.lang` and \xcd`x10.array` are imported in all files
without need for further specification.

%%BARD-HERE



\section{Conventions on Type Names}

\begin{grammar}
TypeName   \: Identifier \\
        \| TypeName \xcd"." Identifier \\
        \| PackageName \xcd"." Identifier \\
PackageName   \: Identifier \\
        \| PackageName \xcd"." Identifier \\
\end{grammar}


While not enforced by the compiler, classes and interfaces
in the \Xten{} library follow the following naming conventions.
Names of types---including classes,
type parameters, and types specified by type definitions---are in
CamelCase and begin with an uppercase letter.  (Type variables are often
single capital letters, such as \xcd`T`.)
For backward
compatibility with languages such as C and \java{}, type
definitions are provided to allow primitive types
such as \xcd"int" and \xcd"boolean" to be written in lowercase.
Names of methods, fields, value properties, and packages are in camelCase and
begin with a lowercase letter. 
Names of \xcd"static val" fields are in all uppercase with words
separated by `\xcd"_"''s.



\chapter{Interfaces}
\label{XtenInterfaces}\index{interface}

An interface specifies signatures for zero or more public methods, property
methods,
\xcd`static val`s, 
classes, structs, interfaces, types
and an invariant. 

The following puny example illustrates all these features: 
% TODO Well, it would if there weren't a compiler bug in the way.
%~~gen ^^^Interfaces_static_val
% package Interfaces_static_val;
% 
%~~vis
\begin{xten}
interface Pushable{prio() != 0} {
  def push(): void;
  static val MAX_PRIO = 100;
  abstract class Pushedness{}
  struct Pushy{}
  interface Pushing{}
  static type Shove = Int;
  property text():String;
  property prio():Int;
}
class MessageButton(text:String)
  implements Pushable{self.prio()==Pushable.MAX_PRIO} {
  public def push() { 
    x10.io.Console.OUT.println(text + " pushed");
  }
  public property text() = text;
  public property prio() = Pushable.MAX_PRIO;
}
\end{xten}
%~~siv
%
%~~neg
\noindent
\xcd`Pushable` defines two property methods, one normal method, and a static
value.  It also 
establishes an invariant, that \xcd`prio() != 0`. 
\xcd`MessageButton` implements a constrained version of \xcd`Pushable`,
\viz\ one with maximum priority.  It
defines the \xcd`push()` method given in the interface, as a \xcd`public`
method---interface methods are implicitly \xcd`public`.

\limitation{X10 may not always detect that type invariants of interfaces are
satisfied, even when they obviously are.}
%% TODO - is this a JIRA?  

A container---a class or struct---can {\em implement} an interface,
typically by having all the methods and property methods that the interface
requires, and by providing a suitable \xcd`implements` clause in its definition.

A variable may be declared to be of interface type.  Such a variable has all
the property and normal methods declared (directly or indirectly) by the
interface; 
nothing else is statically available.  Values of any concrete type which
implement the interface may be stored in the variable.  

\begin{ex}
The following code puts two quite different objects into the variable
\xcd`star`, both of which satisfy the interface \xcd`Star`.
%~~gen ^^^ Interfaces6l3f
% package Interfaces6l3f;
%~~vis
\begin{xten}
interface Star { def rise():void; }
class AlphaCentauri implements Star {
   public def rise() {}
}
class ElvisPresley implements Star {
   public def rise() {}
}
class Example {
   static def example() {
      var star : Star;
      star = new AlphaCentauri();
      star.rise();
      star = new ElvisPresley();
      star.rise();
   }
}
\end{xten}
%~~siv
%
%~~neg
\end{ex}
An interface may extend several interfaces, giving
X10 a large fraction of the power of multiple inheritance at a tiny fraction
of the cost.

\begin{ex}
%~~gen ^^^ Interfaces6g4u
% package Interfaces6g4u;
%~~vis
\begin{xten}
interface Star{}
interface Dog{}
class Sirius implements Dog, Star{}
class Lassie implements Dog, Star{}
\end{xten}
%~~siv
%
%~~neg
\end{ex}


\section{Interface Syntax}

\label{DepType:Interface}

%##(NormalInterfaceDecl TypeParamsI TypeParamI Guard ExtendsInterfaces InterfaceBody InterfaceMemberDecl
\begin{bbgrammar}
%(FROM #(prod:NormalInterfaceDecl)#)
 NormalInterfaceDecl \: Mods\opt \xcd"interface" Id TypeParamsI\opt Properties\opt Guard\opt ExtendsInterfaces\opt InterfaceBody & (\ref{prod:NormalInterfaceDecl}) \\
%(FROM #(prod:TypeParamsI)#)
         TypeParamsI \: \xcd"[" TypeParamIList \xcd"]" & (\ref{prod:TypeParamsI}) \\
%(FROM #(prod:TypeParamI)#)
          TypeParamI \: Id & (\ref{prod:TypeParamI}) \\
                     \| \xcd"+" Id \\
                     \| \xcd"-" Id \\
%(FROM #(prod:Guard)#)
               Guard \: DepParams & (\ref{prod:Guard}) \\
%(FROM #(prod:ExtendsInterfaces)#)
   ExtendsInterfaces \: \xcd"extends" Type & (\ref{prod:ExtendsInterfaces}) \\
                     \| ExtendsInterfaces \xcd"," Type \\
%(FROM #(prod:InterfaceBody)#)
       InterfaceBody \: \xcd"{" InterfaceMemberDecls\opt \xcd"}" & (\ref{prod:InterfaceBody}) \\
%(FROM #(prod:InterfaceMemberDecl)#)
 InterfaceMemberDecl \: MethodDecl & (\ref{prod:InterfaceMemberDecl}) \\
                     \| PropertyMethodDecl \\
                     \| FieldDecl \\
                     \| ClassDecl \\
                     \| InterfaceDecl \\
                     \| TypeDefDecl \\
                     \| \xcd";" \\
\end{bbgrammar}
%##)


\noindent
The invariant associated with an interface is the conjunction of the
invariants associated with its superinterfaces and the invariant
defined at the interface. 



A class \xcd"C"  implements an interface \xcd"I" if \xcd`I`, or a subtype of \xcd`I`, appears in the \xcd`implements` list
of \xcd`C`.  
In this case,
 \xcd`C` implicitly gets all the methods and property methods of \xcd`I`,
      as \xcd`abstract` \xcd`public` methods.  If \xcd`C` does not declare
      them explicitly, then they are \xcd`abstract`, and \xcd`C` must be
      \xcd`abstract` as well.   If \xcd`C` does declare them all, \xcd`C` may
      be concrete.



If \xcd`C` implements \xcd`I`, then the class invariant
(\Sref{DepType:ClassGuardDef}) for \xcd`C`,   $\mathit{inv}($\xcd"C"$)$, implies
the class invariant for \xcd`I`, $\mathit{inv}($\xcd"I"$)$.  That is, if the
interface \xcd`I` specifies some requirement, then every class \xcd`C` that
implements it satisfies that requirement.

\section{Access to Members}

All interface members are \xcd`public`, whether or not they are declared
public.  There is little purpose to non-public methods of an interface; they
would specify that implementing classes and structs have methods that cannot
be seen.

\section{Property Methods}

An interface may declare \xcd`property` methods.  All non-\xcd`abstract`
containers implementing such an interface must provide all the \xcd`property`
methods specified.  

\section{Field Definitions}
\index{interface!field definition in}

An interface may declare a \xcd`val` field, with a value.  This field is implicitly
\xcd`public static val`.  In particular, it is {\em not} an instance field. 
%~~gen ^^^ Interfaces10
% package Interface.Field;
%~~vis
\begin{xten}
interface KnowsPi {
  PI = 3.14159265358;
}
\end{xten}
%~~siv
%
%~~neg

Classes and structs implementing such an interface get the interface's fields as
\xcd`public static` fields.  Unlike  methods, there is no need
for the implementing class to declare them. 
%~~gen ^^^ Interfaces20
% package Interface.Field.Two;
% interface KnowsPi {PI = 3.14159265358;}
%~~vis
\begin{xten}
class Circle implements KnowsPi {
  static def area(r:Double) = PI * r * r;
}
class UsesPi {
  def circumf(r:Double) = 2 * r * KnowsPi.PI;
}
\end{xten}
%~~siv
%
%~~neg

\subsection{Fine Points of Fields}

If two parent interfaces give different static fields of the same name, 
those fields must be referred to by qualified names.
%~~gen ^^^ Interface_field_name_collision
% 
%~~vis
\begin{xten}
interface E1 {static val a = 1;}
interface E2 {static val a = 2;}
interface E3 extends E1, E2{}
class Example implements E3 {
  def example() = E1.a + E2.a;
}
\end{xten}
%~~siv
%
%~~neg

If the {\em same} field \xcd`a` is inherited through many paths, there is no need to
disambiguate it:
%~~gen ^^^ Interfaces_multi
% package Interfaces.Mult.Inher.Field;
%~~vis
\begin{xten}
interface I1 { static val a = 1;} 
interface I2 extends I1 {}
interface I3 extends I1 {}
interface I4 extends I2,I3 {}
class Example implements I4 {
  def example() = a;
}
\end{xten}
%~~siv
%
%~~neg

The initializer of a field in an interface may be any expression.  It is
evaluated under the same rules as a \xcd`static` field of a class. 

\begin{ex}
In this example, a class \xcd`TheOne` is defined,
with an inner interface \xcd`WelshOrFrench`, whose field \xcd`UN` (named in
either Welsh or French) has value 1.  Note that \xcd`WelshOrFrench` does not
define any methods, so it can be trivially added to the \xcd`implements`
clause of any class, as it is for \xcd`Onesome`. 
This allows the body of \xcd`Onesome` to use \xcd`UN` through an unqualified
name, as is done in \xcd`example()`.

%~~gen ^^^ Interfaces3l4a
% package Interfaces3l4a;
%~~vis
\begin{xten}
class TheOne {
  static val ONE = 1;
  interface WelshOrFrench {
    val UN = 1;
  }
  static class Onesome implements WelshOrFrench {
    static def example() {
      assert UN == ONE;
    }
  }
}
\end{xten}
%~~siv
% class Hook{ def run() {TheOne.Onesome.example(); return true;}}
%~~neg
\end{ex}

\section{Generic Interfaces}

Interfaces, like classes and structs, can have type parameters.  
The discussion of generics in \Sref{TypeParameters} applies to interfaces,
without modification.

\begin{ex}
%~~gen ^^^ Interfaces7n1z
% package Interfaces7n1z;
%~~vis
\begin{xten}
interface ListOfFuns[T,U] extends x10.util.List[(T)=>U] {}
\end{xten}
%~~siv
%
%~~neg

\end{ex}

\section{Interface Inheritance}

The {\em direct superinterfaces} of a non-generic interface \xcd`I` are the interfaces
(if any) mentioned in the \xcd`extends` clause of \xcd`I`'s definition.
If \xcd`I`  is generic, the direct superinterfaces are of an instantiation of
\xcd`I` are the corresponding instantiations of those interfaces.
A {\em superinterface} of \xcd`I` is either \xcd`I` itself, or a direct
superinterface of a superinterface of \xcd`I`, and similarly for generic
interfaces.    

\xcd`I` inherits the members of all of its superinterfaces. Any class or
struct that has \xcd`I` in its \xcd`implements` clause also implements all of
\xcd`I`'s superinterfaces. 






\section{Members of an Interface}

The members of an interface \xcd`I` are the union of the following sets: 
\begin{enumerate}
\item All of the members appearing in \xcd`I`'s declaration;
\item All the members of its direct super-interfaces, except those which are
      hidden (\Sref{sect:Hiding}) by \xcd`I`
\item The members of \xcd`Any`.
\end{enumerate}

Overriding for instances is defined as for classes, \Sref{MethodOverload}



\chapter{Classes}
\label{XtenClasses}\index{class}
\label{ReferenceClasses}





\section{Principles of X10 Objects}\label{XtenObjects}\index{object}
\index{class}

\subsection{Basic Design}

Objects are instances of classes: the most common and most powerful sort of
value in X10.  The other kinds of values, structs and functions, are more
specialized, better in some circumstances but not in all.

Classes are structured in a forest of single-inheritance code
hierarchies. Like C++, but unlike Java, there is no single root
class (\Xcd{java.lang.Object}) that all classes inherit from.  Classes
may have any or all of these features: 
\begin{itemize}
\item Implementing any number of interfaces;
\item Static and instance \xcd`val` fields; 
\item Instance \xcd`var` fields; 
\item Static and instance methods;
\item Constructors;
\item Properties;
\item Static and instance nested containers.
\item Static type definitions
\end{itemize}


\Xten{} objects (unlike Java objects) do not have locks associated with them.
Programmers may use atomic blocks (\Sref{AtomicBlocks}) for mutual
exclusion and clocks (\Sref{XtenClocks}) for sequencing multiple parallel
operations.

An object exists in a single location: the place that it was created.  One
place cannot use or even directly refer to an object in a different place.   A
special type, \Xcd{GlobalRef[T]}, allows explicit cross-place references. 

The basic operations on objects are:
\begin{itemize}

\item Construction (\Sref{ObjectInitialization}).  Objects are created, 
      their \xcd`var` and \xcd`val` fields initialized, and other invariants
      established.

\item Field access (\Sref{FieldAccess}). 
The static, instance, and property fields of an object can be retrieved; \xcd`var` fields
can be set.  

\item Method invocation (\Sref{MethodInvocation}).  
Static, instance, and property methods of an object can be invoked.

\item Casting (\Sref{ClassCast}) and instance testing with \xcd`instanceof`
(\Sref{instanceOf}) Objects can be cast or type-tested.  

\item The equality operators \xcd"==" and \xcd"!=".  
Objects can be compared for equality with the \Xcd{==} operation.  This checks
object {\em identity}: two objects are \Xcd{==} iff they are the same object.

\end{itemize}

  

\subsection{Class Declaration Syntax}
\label{sect:ClassDeclSyntax}

The {\em class declaration} has a list of type parameters, a list of
properties, a constraint (the {\em class invariant}), zero or one superclass,
zero or more interfaces that it implements, and a class body containing the
the definition of fields, properties, methods, and member types. Each such
declaration introduces a class type (\Sref{ReferenceTypes}).

%##(ClassDecln TypeParamsI TypeParamIList Properties PropertyList Property Guard Super Interfaces InterfaceTypeList ClassBody ClassMemberDeclns ClassMemberDecln
\begin{bbgrammar}
%(FROM #(prod:ClassDecln)#)
          ClassDecln \: Mods\opt \xcd"class" Id TypeParamsI\opt Properties\opt Guard\opt Super\opt Interfaces\opt ClassBody & (\ref{prod:ClassDecln}) \\
%(FROM #(prod:TypeParamsI)#)
         TypeParamsI \: \xcd"[" TypeParamIList \xcd"]" & (\ref{prod:TypeParamsI}) \\
%(FROM #(prod:TypeParamIList)#)
      TypeParamIList \: TypeParam & (\ref{prod:TypeParamIList}) \\
                     \| TypeParamIList \xcd"," TypeParam \\
                     \| TypeParamIList \xcd"," \\
%(FROM #(prod:Properties)#)
          Properties \: \xcd"(" PropList \xcd")" & (\ref{prod:Properties}) \\
%(FROM #(prod:PropList)#)
            PropList \: Prop & (\ref{prod:PropList}) \\
                     \| PropList \xcd"," Prop \\
%(FROM #(prod:Prop)#)
                Prop \: Annotations\opt Id ResultType & (\ref{prod:Prop}) \\
%(FROM #(prod:Guard)#)
               Guard \: DepParams & (\ref{prod:Guard}) \\
%(FROM #(prod:Super)#)
               Super \: \xcd"extends" ClassType & (\ref{prod:Super}) \\
%(FROM #(prod:Interfaces)#)
          Interfaces \: \xcd"implements" InterfaceTypeList & (\ref{prod:Interfaces}) \\
%(FROM #(prod:InterfaceTypeList)#)
   InterfaceTypeList \: Type & (\ref{prod:InterfaceTypeList}) \\
                     \| InterfaceTypeList \xcd"," Type \\
%(FROM #(prod:ClassBody)#)
           ClassBody \: \xcd"{" ClassMemberDeclns\opt \xcd"}" & (\ref{prod:ClassBody}) \\
%(FROM #(prod:ClassMemberDeclns)#)
   ClassMemberDeclns \: ClassMemberDecln & (\ref{prod:ClassMemberDeclns}) \\
                     \| ClassMemberDeclns ClassMemberDecln \\
%(FROM #(prod:ClassMemberDecln)#)
    ClassMemberDecln \: InterfaceMemberDecln & (\ref{prod:ClassMemberDecln}) \\
                     \| CtorDecln \\
\end{bbgrammar}
%##)




\section{Fields}
\label{FieldDefinitions}
\index{object!field}
\index{field}

Objects may have {\em instance fields}, or simply {\em fields} (called
``instance variables'' in C++ and Smalltalk, and ``slots'' in CLOS): places to
store data that is pertinent to the object.  Fields, like variables, may be
mutable (\xcd`var`) or immutable (\xcd`val`).  

A class may have {\em static fields}, which store data pertinent to the
entire class of objects.  See \Sref{StaticInitialization} for more
information. 
Because of its emphasis on safe concurrency, \Xten{} requires static
fields to be immutable (\xcd`val`). 

No two fields of the same class may have the same name.  A field may have the
same name as a method, although for fields of functional type there is a
subtlety (\Sref{sect:disambiguations}).  

\subsection{Field Initialization}
\index{field!initialization}
\index{initialization!of field}

Fields may be given values via {\em field initialization expressions}:
\xcd`val f1 = E;` and \xcd`var f2 : Int = F;`. Other fields of \xcd`this` may
be referenced, but only those that {\em precede} the field being initialized.


\begin{ex}The following is correct, but would not be if the fields were
reversed:

%~~gen ^^^ Classes10
%package Classes_field_init_expr_a;
%~~vis
\begin{xten}
class Fld{
  val a = 1;
  val b = 2+a;
}
\end{xten}
%~~siv
% class Hook{ def run() {
%   val f = new Fld();
%   assert f.a == 1 && f.b == 3;
%   return true;}}
%~~neg
\end{ex}

\subsection{Field hiding}
\label{sect:FieldHiding}
\index{field!hiding}


A subclass that defines a field \xcd"f" hides any field \xcd"f"
declared in a superclass, regardless of their types.  The
superclass field \xcd"f" may be accessed within the body of
the subclass via the reference \xcd"super.f".

With inner classes, it is occasionally necessary to 
write \xcd`Cls.super.f` to get at a hidden field \xcd`f` of an outer class
\xcd`Cls`. 

\begin{ex}
The \xcd`f` field in \xcd`Sub` hides the \xcd`f` field in \xcd`Super`
The \xcd`superf` method provides access to the \xcd`f` field in \xcd`Super`.
%~~gen ^^^ Classes20
% package classes.fields.primus;
%~~vis
\begin{xten}
class Super{ 
  public val f = 1; 
}
class Sub extends Super {
  val f = true;
  def superf() : Int = super.f; // 1
}
\end{xten}
%~~siv
% class Hook { def run() { 
%   val sub = new Sub();
%   assert sub.f == true;
%   assert sub.superf() == 1;
%   return true;} }
%~~neg
\end{ex}

\begin{ex}
Hidden fields of outer classes can be accessed by suitable forms: 
%~~gen ^^^ Classes30
% package classes.fields.secundus; 
% // NOTEST
%~~vis
\begin{xten}
class A {
   val f = 3;
}
class B extends A {
   val f = 4;
   class C extends B {
      // C is both a subclass and inner class of B
      val f = 5;
       def example() {
         assert f         == 5 : "field of C";
         assert super.f   == 4 : "field of superclass";
         assert B.this.f  == 4 : "field of outer instance";
         assert B.super.f == 3 : "super.f of outer instance";
       }
    }
}
\end{xten}
%~~siv
% class Hook { def run() { ((new B()).new C()).example(); return true; } }
%~~neg
\end{ex}

\subsection{Field qualifiers}
\label{FieldQualifier}
\index{qualifier!field}
\index{field!qualifier}

The behavior of a field may be changed by a field qualifier, such as
\xcd`static` or \xcd`transient`.  


\subsubsection{\Xcd{static} qualifier}
\index{field!static}

A \xcd`val` field may be declared to be {\em static}, as described in
\Sref{FieldDefinitions}. 

\subsubsection{\Xcd{transient} Qualifier}
\label{TransientFields}
\index{transient}
\index{field!transient}

A field may be declared to be {\em transient}.  Transient fields are excluded
from the deep copying that happens when information is sent from place to
place in an \Xcd{at} statement.    The value of a transient field of a copied
object is the default value of its type, regardless of the value of the field
in the original.  If the type of a field has no
default value, it cannot be marked \Xcd{transient}.

%%AT-COPY%% %~~gen ^^^ Classes40
%%AT-COPY%% % package Classes.Transient.Example;
%%AT-COPY%% % KNOWNFAIL-at
%%AT-COPY%% %~~vis
%%AT-COPY%% \begin{xten}
%%AT-COPY%% class Trans { 
%%AT-COPY%%    val copied = "copied";
%%AT-COPY%%    transient var transy : String = "a very long string";
%%AT-COPY%%    def example() {
%%AT-COPY%%       at (here; this) { // causes copying of 'this'
%%AT-COPY%%          assert(this.copied.equals("copied"));
%%AT-COPY%%          assert(this.transy == null);
%%AT-COPY%%       }
%%AT-COPY%%    }
%%AT-COPY%% }
%%AT-COPY%% \end{xten}
%%AT-COPY%% %~~siv
%%AT-COPY%% % class Hook{ def run() {(new Example()).example(); return true;}}
%%AT-COPY%% %~~neg
%%AT-COPY%% 

%~~gen ^^^ Classes40
% package Classes.Transient.Example;
% 
%~~vis
\begin{xten}
class Trans { 
   val copied = "copied";
   transient var transy : String = "a very long string";
   def example() {
      at (here) { // causes copying of 'this'
         assert(this.copied.equals("copied"));
         assert(this.transy == null);
      }
   }
}
\end{xten}
%~~siv
% class Hook{ def run() {(new Trans()).example(); return true;}}
%~~neg


\section{Properties}
\label{PropertiesInClasses}
\index{property}

The properties of an object (or struct) are a restricted form of public
\xcd`val` fields.\footnote{In many cases, a 
\xcd`val` field can be upgraded to a \xcd`property`, which 
entails no compile-time or runtime cost.  Some cannot be, \eg, in cases where
cyclic structures of \xcd`val` fields are required.} 
For example,  every array has a \xcd`rank` telling
how many subscripts it takes.  User-defined classes can have whatever
properties are desired. 

Properties differ from public \xcd`val` fields in a few ways: 
\begin{enumerate}
\item Property references are allowed on \xcd`self` in constraints:
      \xcd`self.prop`.  Field references are not.
\item Properties are in scope for all instance initialization expressions.
      \xcd`val` fields are not.
\item The graph of values reachable from a given object by following only
      property links is acyclic.  Conversely, it is possible (and routine) for
      two objects to point to each other with \xcd`val` fields.
\item Properties are declared in the class header; \xcd`val` fields are
      defined in the class body.
\item Properties are set in constructors by a \xcd`property` statement.
      \xcd`val` fields are set by assignment.
\end{enumerate}



Properties are defined in parentheses, after the name of the class.  They are
given values by the \xcd`property` command in constructors.

\begin{ex}
\xcd`Proper` has a single property, \xcd`t`.  \xcd`new Proper(4)` creates a
\xcd`Proper` object with \xcd`t==4`. 
%~~gen ^^^ Classes50
% package Classes.Toss.Freedom.Disk2;
%~~vis
\begin{xten}
class Proper(t:Int) {
  def this(t:Int) {property(t);}
}
\end{xten}
%~~siv
% class Hook{ def run() {
%   val p = new Proper(4);
%   return p.t == 4;
% } } 
%~~neg

\end{ex}


It is a static error for a class
defining a property \xcd"x: T" to have a subclass class that defines
a property or a field with the name \xcd"x".


A property \xcd`x:T` induces a field with the same name and type, 
as if defined with: 
%~~gen ^^^ Classes60
% package Classes.For.Masses.Of.NevermindTheRest;
% class Exampll[T] {
%~~vis
\begin{xten}
public val x : T;
\end{xten} 
%~~siv
% def this(y:T) { x=y; }
% }
%~~neg

\index{property!initialization}
Properties are initialized in a constructor by the invocation of a special \Xcd{property}
statement. The requirement to use the \xcd`property` statement means that all properties
must be given values at the same time: a container either has its properties
or it does not.
\begin{xten}
property(e1,..., en);
\end{xten}
The number and types of arguments to the \Xcd{property} statement must match
the number and types of the properties in the class declaration, in order.  
Every constructor of a class with properties must invoke \xcd`property(...)`
precisely once; it is a static error if X10 cannot prove that this holds.



By construction, the graph whose nodes are values and whose edges are
properties is acyclic.  \Eg, there cannot be values \xcd`a` and \xcd`b` with
properties \xcd`c` and \xcd`d` such that \xcd`a.c == b` and \xcd`b.d == a`.

\begin{ex}
%~~gen ^^^ Classes7h2f
% package Classes7h2f;
%~~vis
\begin{xten}
class Proper(a:Int, b:String) {
  def this(a:Int, b:String) {
      property(a, b);
  }
  def this(z:Int) {
      val theA = z+5;
      val theB = "X"+z;
      property(theA, theB);
  }
  static def example() {
      val p = new Proper(1, "one");
      assert p.a == 1 && p.b.equals("one");
      val q = new Proper(10);
      assert q.a == 15 && q.b.equals("X10");
  }
}
\end{xten}
%~~siv
% class Hook{ def run() {Proper.example(); return true;}}
%~~neg
\end{ex}

\subsection{Properties and Field Initialization}

Fields with explicit initializers are evaluated immediately after the
\xcd`property` command, and all properties are in scope when initializers are
evaluated.  

\begin{ex}
Class \xcd`Init` initializes the field \xcd`a` to be an array of \xcd`n`
elements, where \xcd`n` is a property.    
When \xcd`new Init(4)` is executed, the constructor first sets \xcd`n` to
\xcd`4` via the \xcd`property` statement, and then initializes \xcd`a` to a
4-element array.

However, \xcd`Outit` uses a field rather than a property for \xcd`n`.  
If the \xcd`ERROR` line were present, it would not compile.  \xcd`n` has not
been definitely assigned (\Sref{sect:DefiniteAssignment}) at this point, and
\xcd`n` has not been given its value, so \xcd`a` cannot be computed.  
(If one insisted that \xcd`n` be a property, \xcd`a` would have to be
initialized in the constructor, rather than by an initialization expression.)
%~~gen ^^^ Classes9c9r
% package Classes9c9r;
%~~vis
\begin{xten}
class Init(n:Int) {
  val a = new Rail[String](n, "");
  def this(n:Int) { property(n); }
}
class Outit {
  val n : Int;
  //ERROR: val a = new Rail[String](n, "");
  def this(m:Int) { this.n = m; }
}
\end{xten}
%~~siv
%
%~~neg


\end{ex}

\subsection{Properties and Fields}

A container with a property named \xcd`p`, or a nullary property method named
\xcd`p()`, cannot have a field named \xcd`p` --- either defined in that
container, or inherited from a superclass.

\subsection{Acyclicity of Properties}
\index{properties!acyclic}

X10 has certain restrictions that, ultimately, require that properties are
simpler than their containers.  For example, \xcd`class A(a:A){}` is not
allowed.  
Formally, this requirement is that there is  a total order $\preceq$ 
on all classes and
structs such that, if $A$ extends $B$, then $A \prec B$, and
if $A$ has a property of type $B$, then $A \prec B$, where $A \prec B$ means
$A \preceq B$ and $A \ne B$.   
For example, the preceding class \xcd`A` is ruled out because we would need
\xcd`A`$\prec$\xcd`A`, which violates the definition of $\prec$.
The programmer need not (and cannot) specify
$\preceq$, and rarely need worry about its existence.  

Similarly, 
the type of a property may not simply be a type parameter.  
For example, \xcd`class A[X](x:X){}` is illegal.





\section{Methods}
\label{sect:Methods}
\index{method}
\index{signature}
\index{method!signature}
\index{method!instance}
\index{method!static}

As is common in object-oriented languages, objects can have {\em methods}, of
two sorts.  {\em Static methods} are functions, conceptually associated with a
class and defined in its namespace.  {\em Instance methods} are parameterized
code bodies associated with an instance of the class, which execute with
convenient access to that instance's fields. 

Each method has a {\em signature}, telling what arguments it accepts, what
type it returns, and what precondition it requires. Method definitions may be
overridden by subclasses; the overriding definition may have a declared return
type that is a subtype of the return type of the definition being overridden.
Multiple methods with the same name but different signatures may be provided
\index{overloading}
\index{polymorphism}
on a class (called ``overloading'' or ``ad hoc polymorphism''). Methods may be
declared \Xcd{public}, \Xcd{private}, \Xcd{protected}, or given default package-level access
rights.

%##(MethMods MethodDeclaration TypeParams Formals FormalList HasResultType MethodBody BinOpDecln PrefixOpDecln ApplyOpDecln ConversionOpDecln
\begin{bbgrammar}
%(FROM #(prod:MethMods)#)
            MethMods \: Mods\opt & (\ref{prod:MethMods}) \\
                     \| MethMods \xcd"property"  \\
                     \| MethMods Mod \\
%(FROM #(prod:MethodDecln)#)
         MethodDecln \: MethMods \xcd"def" Id TypeParams\opt Formals Guard\opt Throws\opt HasResultType\opt MethodBody & (\ref{prod:MethodDecln}) \\
                     \| BinOpDecln \\
                     \| PrefixOpDecln \\
                     \| ApplyOpDecln \\
                     \| SetOpDecln \\
                     \| ConversionOpDecln \\
%(FROM #(prod:TypeParams)#)
          TypeParams \: \xcd"[" TypeParamList \xcd"]" & (\ref{prod:TypeParams}) \\
%(FROM #(prod:Formals)#)
             Formals \: \xcd"(" FormalList\opt \xcd")" & (\ref{prod:Formals}) \\
%(FROM #(prod:FormalList)#)
          FormalList \: Formal & (\ref{prod:FormalList}) \\
                     \| FormalList \xcd"," Formal \\
%(FROM #(prod:Throws)#)
             Throws \: \xcd"throws" ThrowList & (\ref{prod:Throws}) \\
%(FROM #(prod:ThrowsList)#)
          ThrowsList \: Type & (\ref{prod:ThrowsList}) \\
                     \| ThrowsList \xcd"," Type \\
%(FROM #(prod:HasResultType)#)
       HasResultType \: ResultType & (\ref{prod:HasResultType}) \\
                     \| \xcd"<:" Type \\
%(FROM #(prod:MethodBody)#)
          MethodBody \: \xcd"=" LastExp \xcd";" & (\ref{prod:MethodBody}) \\
                     \| \xcd"=" Annotations\opt \xcd"{" BlockStmts\opt LastExp \xcd"}" \\
                     \| \xcd"=" Annotations\opt Block \\
                     \| Annotations\opt Block \\
                     \| \xcd";" \\
%(FROM #(prod:BinOpDecln)#)
          BinOpDecln \: MethMods \xcd"operator" TypeParams\opt \xcd"(" Formal  \xcd")" BinOp \xcd"(" Formal  \xcd")" Guard\opt HasResultType\opt MethodBody & (\ref{prod:BinOpDecln}) \\
                     \| MethMods \xcd"operator" TypeParams\opt \xcd"this" BinOp \xcd"(" Formal  \xcd")" Guard\opt HasResultType\opt MethodBody \\
                     \| MethMods \xcd"operator" TypeParams\opt \xcd"(" Formal  \xcd")" BinOp \xcd"this" Guard\opt HasResultType\opt MethodBody \\
%(FROM #(prod:PrefixOpDecln)#)
       PrefixOpDecln \: MethMods \xcd"operator" TypeParams\opt PrefixOp \xcd"(" Formal  \xcd")" Guard\opt HasResultType\opt MethodBody & (\ref{prod:PrefixOpDecln}) \\
                     \| MethMods \xcd"operator" TypeParams\opt PrefixOp \xcd"this" Guard\opt HasResultType\opt MethodBody \\
%(FROM #(prod:ApplyOpDecln)#)
        ApplyOpDecln \: MethMods \xcd"operator" \xcd"this" TypeParams\opt Formals Guard\opt HasResultType\opt MethodBody & (\ref{prod:ApplyOpDecln}) \\
%(FROM #(prod:ConversionOpDecln)#)
   ConversionOpDecln \: ExplConvOpDecln & (\ref{prod:ConversionOpDecln}) \\
                     \| ImplConvOpDecln \\
\end{bbgrammar}
%##)


\index{parameter!var}
\index{parameter!val}
A formal parameter may have a \xcd"val" or \xcd"var"
% , or \Xcd{ref}
modifier; \xcd`val` is the default.
The body of the method is executed in an environment in which 
each formal parameter corresponds to a local variable (\xcd`var` iff the
formal parameter is \xcd`var`)
and is initialized with the value of the actual parameter.

\subsection{Forms of Method Definition}

There are several syntactic forms for definining methods.   The forms that
include a block, such as \xcd`def m(){S}`, allow an arbitrary block.  These
forms can define a \xcd`void` method, which does not return a value. 

The
forms that include an expression, such as \xcd`def m()=E`, require a
syntactically and semantically valid expression.   These forms cannot define a
\xcd`void` method, because expressions cannot be \xcd`void`.  

There are no other semantic differences between the two forms. 

\subsection{Method Return Types}

A method with an explicit return type returns that type.
A method without an
explicit return type is given a return type by type inference.
A {\em call} to a method has type given by substituting information about the
actual \xcd`val` parameters for the formals.

\begin{ex}

In the example below, \xcd`met1` has an explicit return type \xcd`Ret{n==a}`.
\xcd`met2` does not, so its return type is computed, also to be
\xcd`Ret{n==a}`, because that's what the implicitly-defined constructor 
returns.

\xcd`use3` requires that its argument have \xcd`n==3`.  
\xcd`example` shows that both \xcd`met1` and \xcd`met2` can be used to produce
such an object.  In both cases, the actual argument \xcd`3` is substituted for
the formal argument \xcd`a` in the return type expression for the method
\xcd`Ret{n==a}`, giving the type \xcd`Ret{n==3}` as required by \xcd`use3`.

%~~gen ^^^ Classes9q2w
% package Classes9q2w;
%~~vis
\begin{xten}
class Ret(n:Int) {
  static def met1(a:Int) : Ret{n==a} = new Ret(a);
  static def met2(a:Int)             = new Ret(a);
  static def use3(Ret{n==3}) {}
  static def example() {
     use3(met1(3));
     use3(met2(3));
  }  
}
\end{xten}
%~~siv
%
%~~neg


\end{ex}


\subsection{Throws Clause}
The \xcd`throws` clause indicates what checked exceptions may be
raised during the execution of the method and are not handled by
\xcd'catch' blocks within the method.  If a checked exception may
escape from the method, then it must be by a subtype of one of the
types listed in the \xcd`throws` clause of the method.   Checked
exceptions are defined to be any subclass of
\xcd{x10.lang.CheckedThrowable} that are not also subclasses of
either \xcd{x10.lang.Exception} or \xcd{x10.lang.Error}. 

If a method is implementing an interface or overriding a superclass
method the set of types represented by its \xcd'throws' clause must by
a (potentially improper) subset of the types of the \xcd'throws'
clause of the method it is overriding. 

\subsection{Final Methods}
\index{final}
\index{method!final}
An instance method may be given the \xcd`final` qualifier.  \xcd`final`
methods may not be overridden.

\subsection{Generic Instance Methods}
\index{method!generic instance}

\limitationx{}
In X10, an instance method may be generic: 
%~~gen ^^^ Classes1b7z
% package Classes1b7z;
% NOTEST
%~~vis
\begin{xten}
class Example {
  def example[T](t:T) = "I like " + t;
}
\end{xten}
%~~siv
%
%~~neg

However, the C++ back end does not currently support generic virtual instance
methods like \xcd`example`.  It does allow generic instance methods which are
\xcd`final` or \xcd`private`, and it does allow generic static methods.  


\subsection{Method Guards}
\label{MethodGuard}
\index{method!guard}
\index{guard!on method}

Often, a method will only make sense to invoke under certain
statically-determinable conditions.  These conditions may be expressed as a
guard on the method.

\begin{ex}
For example, \xcd`example(x)` is only
well-defined when \xcd`x != null`, as \xcd`null.toString()` throws a null
pointer exception, and returns nothing: 
%~~gen ^^^ Classes80
% package Classes.methodwithconstraintthingie;
% 
%~~vis
\begin{xten}
class Example {
   var f : String = "";
   def setF(x:Any){x != null} : void = {
      this.f = x.toString();
   }
}
\end{xten}
%~~siv
%
%~~neg
\noindent
(We could have used a constrained type \xcd`Any{self!=null}` for \xcd`x`
instead; in
most cases it is a matter of personal preference or convenience of expression
which one to use.) 
\end{ex}


The requirement of having a method guard 
is that callers must demonstrate to
the X10
compiler that the guard is satisfied.  
With the \xcd`STATIC_CHECKS` compiler option in force (\Sref{sect:Callstyle}), this is
checked at compile time. 
As usual with static constraint
checking, there is no runtime cost.  Indeed, this code can be more efficient
than usual, as it is statically provable that \xcd`x != null`.

When \xcd`STATIC_CHECKS` is not in force, dynamic checks are generated as
needed; method guards are checked at runtime. This is potentially more
expensive, but may be more convenient. 

\begin{ex}
The following code fragment contains a line which will not compile 
with \xcd`STATIC_CHECKS` on (assuming the guarded \xcd`example` method above).  (X10's type system does not attempt to propagate 
information from \xcd`if`s.)  It will compile with \xcd`STATIC_CHECKS` off,
but it may insert an extra \xcd`null`-test for \xcd`x`.  
%~~gen ^^^ Classes90
% package Classes.methodguardnadacastthingie;
%//OPTIONS: -STATIC_CHECKS
% class Example {var f : String = ""; def example(x:Any){x != null} = {this.f = x.toString();}}
% class Eyample {
%~~vis
\begin{xten}
  def exam(e:Example, x:Any) {
    if (x != null) 
       e.example(x as Any{x != null});
       // If STATIC_CHECKS is in force: 
       // ERROR: if (x != null) e.example(x); 
  }
\end{xten}
%~~siv
%}
%~~neg
\end{ex}


The guard \xcd`{c}` 
in a guarded method 
\xcd`def m(){c} = E;`
specifies a constraint \xcd"c" on the
properties of the class \xcd"C" on which the method is being defined. The
method, in effect, only exists  for those instances of \xcd"C" which satisfy
\xcd"c".  It is 
illegal for code to invoke the method on objects whose static type is
not a subtype of \xcd"C{c}".

Specifically: 
    the compiler checks that every method invocation
    \xcdmath"o.m(e$_1$, $\dots$, e$_n$)"
    is type correct. Each argument
    \xcdmath"e$_i$" must have a
    static type \xcdmath"S$_i$" that is a subtype of the declared type
    \xcdmath"T$_i$" for the $i$th
    argument of the method, and the conjunction of the constraints on the
    static types 
    of the arguments must entail the guard in the parameter list
    of the method.

    The compiler checks that in every method invocation
    \xcdmath"o.m(e$_1$, $\dots$, e$_n$)"
    the static type of \xcd"o", \xcd"S", is a subtype of \xcd"C{c}", where the method
    is defined in class \xcd"C" and the guard for \xcd"m" is equivalent to
    \xcd"c".

    Finally, if the declared return type of the method is
    \xcd"D{d}", the
    return type computed for the call is
    \xcdmath"D{a: S; x$_1$: S$_1$; $\dots$; x$_n$: S$_n$; d[a/this]}",
    where \xcd"a" is a new
    variable that does not occur in
    \xcdmath"d, S, S$_1$, $\dots$, S$_n$", and
    \xcdmath"x$_1$, $\dots$, x$_n$" are the formal
    parameters of the method.


\limitation{
Using a reference to an outer class, \xcd`Outer.this`, in a constraint, is not supported.
}


\subsection{Property methods}
\index{method!property}
\index{property method}

%##(PropertyMethodDeclaration
\begin{bbgrammar}
%(FROM #(prod:PropMethodDecln)#)
     PropMethodDecln \: MethMods Id TypeParams\opt Formals Guard\opt Throws\opt HasResultType\opt MethodBody & (\ref{prod:PropMethodDecln}) \\
                     \| MethMods Id Guard\opt HasResultType\opt MethodBody \\
\end{bbgrammar}
%##)

Property methods are methods that can be evaluated in constraints, as
properties can.   They provide a means of abstraction over properties; \eg,
interfaces can specify property methods that implementing containers must
provide, but, just as they cannot specify ordinary fields, they cannot specify
property fields.   Property methods are very limited in computing power: they
must obey the same restrictions as constraint expressions.  In particular,
they cannot have side effects, or even much code in their bodies.


\begin{ex}
The \xcd`eq()` method below tells if the \xcd`x` and \xcd`y`
properties are equal; the \xcd`is(z)` method tells if they are both equal to
\xcd`z`.  
The \xcd`eq` and \xcd`is` property methods are used in types in the
\xcd`example` method.
%~~gen ^^^ Classes100
%package Classes.PropertyMethods;
%~~vis
\begin{xten}
class Example(x:Int, y:Int) {
   def this(x:Int, y:Int) { property(x,y); }
   property eq() = (x==y);
   property is(z:Int) = x==z && y==z;
   def example( a : Example{eq()}, b : Example{is(3)} ) {}
}
\end{xten}
%~~siv
%
%~~neg
\end{ex}

A property method declared in a class must have
a body and must not be \xcd"void".  The body of the method must
consist of only a single \xcd"return" statement with an expression,  or a single
expression.  It is a static error if the expression cannot be
represented in the constraint system.   Property methods may be \xcd`abstract`
in \xcd`abstract` classes, and may be specified in interfaces, but are
implicitly \xcd`final` in 
non-\xcd`abstract` classes. 

The expression may contain invocations of other property methods.  The
compiler ensures that there are no circularities in property methods, so
property method evaluations always terminate.

Property methods in classes are implicitly \xcd"final"; they cannot be
overridden.  It is a static error if a superclass has a property method with a
given signature, and a subclass has a method or property method with the same
signature.   It is a static error if a superclass has a property with some
name \xcd`p`, and a subclass has a nullary method of any kind (instance,
static, or property) also named \xcd`p`. 



A nullary property method definition may omit 
the \xcd"def" keyword.  That is, the following are equivalent:

%~~gen ^^^ Classes110
% package classes.waifsome1;
% class Waif(rect:Boolean, onePlace:Place, zeroBased:Boolean) {
%~~vis
\begin{xten}
property def rail(): Boolean = 
   rect && onePlace == here && zeroBased;
\end{xten}
%~~siv
%}
%~~neg
and
%~~gen ^^^ Classes120
% package classes.waifsome2;
% class Waif(rect:Boolean, onePlace:Place, zeroBased:Boolean) {
%~~vis
\begin{xten}
property rail(): Boolean = 
   rect && onePlace == here && zeroBased;
\end{xten}
%~~siv
%}
%~~neg

Similarly, nullary property methods can be inspected in constraints without
\xcd`()`. If \xcd`ob`'s type has a property \xcd`p`, then \xcd`ob.p` is that
property. Otherwise, if it has a nullary property method \xcd`p()`, \xcd`ob.p`
is equivalent to \xcd`ob.p()`. As a consequence, if the type provides both a
property \xcd`p` and a nullary method \xcd`p()`, then the property can be
accessed as \xcd`ob.p` and the method as \xcd`ob.p()`.\footnote{This only
applies to nullary property methods, not nullary instance methods.  Nullary
property methods perform limited computations, have no side effects, and
always return the same value, since
they have to be expressed in the constraint sublanguage.  In this sense, a
nullary property method does not behave hugely different from a property.
Indeed, a compilation scheme which cached the value of the property method
would all but erase the distinction.  Other methods may
have more behavior, \eg, side effects, so we keep the \xcd`()` to make it
clear that a method call is potentially large.
}

%~~longexp~~`~~` ^^^ Classes130
% package classes.not.weasels;
% class Waif(rect:Boolean, onePlace:Place, zeroBased:Boolean) {
%   def this(rect:Boolean, onePlace:Place, zeroBased:Boolean) 
%          :Waif{self.rect==rect, self.onePlace==onePlace, self.zeroBased==zeroBased}
%          = {property(rect, onePlace, zeroBased);}
%   property rail(): Boolean = rect && onePlace == here && zeroBased;
%   static def zoink() {
%      val w : Waif{
%~~vis
\xcd`w.rail`, with either definition above, 
% }= new Waif(true, here, true);
% }}
%~~pxegnol
is equivalent to 
%~~longexp~~`~~` ^^^ Classes140
% package classes.not.ferrets;
% class Waif(rect:Boolean, onePlace:Place, zeroBased:Boolean) {
%   def this(rect:Boolean, onePlace:Place, zeroBased:Boolean) 
%          :Waif{self.rect==rect, self.onePlace==onePlace, self.zeroBased==zeroBased}
%          = {property(rect, onePlace, zeroBased);}
%   property rail(): Boolean = rect && onePlace == here && zeroBased;
%   static def zoink() {
%      val w : Waif{
%~~vis
\xcd`w.rail()`
% }= new Waif(true, here, true);
% }}
%~~pxegnol


\subsubsection{Limitation of Property Methods}

\limitationx{} 
Currently, X10 forbids the use of property methods which have all the
following features: 
\begin{itemize}
\item they are abstract, and
\item they have one or more arguments, and
\item they appear as subterms in constraints.
\end{itemize}
Any two of these features may be combined, but the three together may not be. 

\begin{ex} 
The constraint in \xcd`example1` is concrete, not abstract.  The constraint in
\xcd`example2` is nullary, and has no arguments.  The constraint in
\xcd`example3` appears at the top level, rather than as a subterm ({\em cf.}
the equality expressions \xcd`A==B` in the other examples).    However,
\xcd`example4` combines all three features, and is not allowed.
%~~gen ^^^ Classes7a5j
% package Classes7a5j;
% // If example4() compiles, then the limitation in Classes7a5j's section is
% // gone, so delete the whole subsection from the spec.
%~~vis
\begin{xten}
class Cls {
  property concrete(a:Int) = 7;
}
interface Inf {
  property nullary(): Int;
  property topLevel(z:Int):Boolean;
  property allThree(z:Int):Int;
}
class Example{
  def example1(Cls{self.concrete(3)==7}) = 1;
  def example2(Inf{self.nullary()==7})   = 2;
  def example3(Inf{self.topLevel(3)})    = 3;
  //ERROR: def example4(Inf{self.allThree(3)==7}) = "fails";
}
\end{xten}
%~~siv
%
%~~neg
\end{ex}


\subsection{Method overloading, overriding, hiding, shadowing and obscuring}
\label{MethodOverload}
\index{method!overloading}



The definitions of method overloading, overriding, hiding, shadowing and
obscuring in \Xten{} are familiar from languages such as Java, modulo the
following considerations motivated by type parameters and dependent types.



Two or more methods of a class or interface may have the same
name if they have a different number of type parameters, or
they have formal parameters of different constraint-erased types (in some instantiation of the
generic parameters). 



\begin{ex}
The following overloading of \xcd`m` is unproblematic.
%~~gen ^^^ Classes150
% package Classes.Mful;
%~~vis
\begin{xten}
class Mful{
   def m() = 1;
   def m[T]() = 2;
   def m(x:Int) = 3;
   def m[T](x:Int) = 4;
}
\end{xten}
%~~siv
%
%~~neg
\end{ex}


A class definition may include methods which are ambiguous in {\em some}
generic instantiation. (It is a compile-time error if the methods are
ambiguous in {\em every} generic instantiation, but excluding class
definitions which are are ambiguous in {\em some} instantiation would exclude
useful cases.)  It is a compile-time error to {\em use} an ambiguous method
call. 

\begin{ex}
The following class definition is acceptable.  However, the marked method
calls are ambiguous, and hence not acceptable.
%~~gen ^^^ Classes4d5e
% package Classes4d5e;
%~~vis
\begin{xten}
class Two[T,U]{
  def m(x:T)=1;
  def m(x:Int)=2;
  def m[X](x:X)=3;
  def m(x:U)=4;
  static def example() {
    val t12 = new Two[Int, Any]();
    // ERROR: t12.m(2);
    val t13  = new Two[String, Any]();
    t13.m("ferret");
    val t14 = new Two[Boolean,Boolean]();
    // ERROR: t14.m(true);
  }
}
\end{xten}
%~~siv
%~~neg
\noindent
The call \xcd`t12.m(2)` could refer to either the \xcd`1` or \xcd`2`
definition of \xcd`m`, so it is not allowed.   
The call \xcd`t14.m(true)` could refer to either the \xcd`1` or \xcd`4`
definition, so it, too, is not allowed.

The call \xcd`t13.m("ferret")` refers only to the \xcd`1` definition.  If
the \xcd`1` definition were absent, type argument inference would make it
refer to the \xcd`3` definition.  However, X10 will choose a fully-specified
call if there is one, before trying type inference, so this call unambiguously
refers to \xcd`1`.
\end{ex}


\XtenCurrVer{} does not permit overloading based on constraints. That is, the
following is {\em not} legal, although either method definition individually
is legal:
\begin{xten}
   def n(x:Int){x==1} = "one";
   def n(x:Int){x!=1} = "not";
\end{xten}




The definition of a method declaration \xcdmath"m$_1$" ``having the same signature
as'' a method declaration \xcdmath"m$_2$" involves identity of types. 



The {\em constraint erasure} of a type \xcdmath"T", 
\xcdmath"$ce$(T)",
is obtained by removing all the constraints outside of functions in \xcd`T`,
specificially: 
\begin{eqnarray}
ce({\tt T}) &=& {\tt T} \mbox{ if \xcd`T` is a container or interface}\\
ce({\tt T\{c\}}) &=& ce({\tt T})\\
ce({\tt T[S}_1{\tt,}\ldots{\tt,S}_n{\tt ]})
  &=&
ce({\tt T}){\tt [} ce({\tt S}_1){\tt,}\ldots{\tt,} ce({\tt S}_n){\tt ]}\\
ce({\tt (S}_1{\tt,}\ldots{\tt,S}_n{\tt ) => T})
  &=&
{\tt }{\tt (} ce({\tt S}_1){\tt,}\ldots{\tt,} ce({\tt S}_n){\tt ) => } 
ce({\tt T})
\end{eqnarray}



 Two methods are said to have {\em erasedly equivalent signatures} if (a) they have the
 same number of type parameters, 
(b) they have the same number of formal (value) parameters, and (c)
for each formal parameter the constraint erasure of its types are erasedly equivalent.
It is a 
compile-time error for there to be two methods with the same name and
erasedly equivalent signatures in a class (either defined in that class or in a
superclass), unless the signatures are identical (without erasures) and one of the methods is
defined in a superclass (in which case the superclass's method is overridden
by the subclass's, and the subclass's method's return type must be a subtype of
the superclass's method's return type).  

 



In addition, the guard of an overridden method
must entail
the guard of the overriding method.   This
ensures that any virtual call to the method
satisfies the guard of the callee.

\begin{ex}
In the following example, the call to \xcd`s.recip(3)` in \xcd`example()`
will invoke \xcd`Sub.recip(n)`.  The call is legitimate because
\xcd`Super.recip`'s guard, \xcd`n != 0`, is satisfied by \xcd`3`.  
The guard on \xcd`Sub.recip(n)` is simply
\xcd`true`, which is also satisfied.  However, if we had used the \xcd`ERROR`
line's definition, the guard on \xcd`Sub.recip(n)` would be \xcd`n != 0, n != 3`, which
is not satisfied by \xcd`3`, so -- despite the call statically type-checking
-- at runtime the call would violate its guard and (in this case) throw an exception.
%~~gen ^^^ Classes5l3r
% package Classes5l3r;
% // FOR-ERR-CASE-DELETE: def recip(m:Int){true} = 1.0/m;
%~~vis
\begin{xten}
class Super {
  def recip(n:Int){n != 0} = 1.0/n;
}
class Sub extends Super{
  //ERROR: def recip(n:Int){n != 0, n != 3} = 1.0/(n * (n-3));
  def recip(m:Int){true} = 1.0/m;
}
class Example{
  static def example() {
     val s : Super = new Sub();
     s.recip(3);
  }
}
\end{xten}
%~~siv
%
%~~neg

\end{ex}


  If a class \xcd"C" overrides a method of a class or interface
  \xcd"B", the guard of the method in \xcd"B" must entail
  the guard of the method in \xcd"C".


A class \xcd"C" inherits from its direct superclass and superinterfaces all
their methods visible according to the access modifiers
of the superclass/superinterfaces that are not hidden or overridden. A method \xcdmath"M$_1$" in a class
\xcd"C" overrides
a method \xcdmath"M$_2$" in a superclass \xcd"D" if
\xcdmath"M$_1$" and \xcdmath"M$_2$" have erasedly equivalent signatures.
Methods are overriden on a signature-by-signature basis.  It is a compile-time
error if an instance method overrides a static method.  (But is it permitted
for an instance {\em field} to hide a static {\em field}; that's hiding
(\Sref{sect:FieldHiding}), not 
overriding, and hence totally different.)

\section{Constructors}
\label{sect:constructors}
\index{constructor}

Instances of classes are created by the \xcd`new` expression: \\
%##(ClassInstCreationExp
\begin{bbgrammar}
%(FROM #(prod:ObCreationExp)#)
       ObCreationExp \: \xcd"new" TypeName TypeArgs\opt \xcd"(" ArgumentList\opt \xcd")" ClassBody\opt & (\ref{prod:ObCreationExp}) \\
                     \| Primary \xcd"." \xcd"new" Id TypeArgs\opt \xcd"(" ArgumentList\opt \xcd")" ClassBody\opt \\
                     \| FullyQualifiedName \xcd"." \xcd"new" Id TypeArgs\opt \xcd"(" ArgumentList\opt \xcd")" ClassBody\opt \\
\end{bbgrammar}
%##)

This constructs a new object, and calls some code, called a {\em constructor},
to initialize the newly-created object properly.

Constructors are defined like methods, except that they must be named \xcd`this`
and ordinary methods may not be.    The content of a constructor body has
certain capabilities (\eg, \xcd`val` fields of the object may be initialized)
and certain restrictions (\eg, most methods cannot be called); see
\Sref{ObjectInitialization} for the details.

\begin{ex}

The following class provides two constructors.  The unary constructor 
\xcd`def this(b : Int)` allows initialization of the \xcd`a` field to an 
arbitrary value.  The nullary constructor \xcd`def this()` gives it a default
value of 10.  The \xcd`example` method illustrates both of these calls.


%~~gen ^^^ ClassesCtor10
% package ClassesCtor10;
%~~vis
\begin{xten}
class C {
  public val a : Int;
  def this(b : Int) { a = b; } 
  def this()        { a = 10; }
  static def example() {
     val two = new C(2);
     assert two.a == 2;
     val ten = new C(); 
     assert ten.a == 10;
  }
}
\end{xten}
%~~siv
% class Hook{ def run() {C.example(); return true;}}
%~~neg
\end{ex}


\subsection{Automatic Generation of Constructors}
\index{constructor!generated}

Classes that have no constructors written in the class declaration are
automatically given a constructor which sets the class properties and does
nothing else. If this automatically-generated constructor is not valid (\eg,
if the class has \xcd`val` fields that need to be initialized in a
constructor), the class has no constructor, which is a static error.

\begin{ex}
The following class has no explicit constructor.
Its implicit constructor is 
\xcd`def this(x:Int){property(x);}`
This implicit constructor is valid, and so is the class. 
%~~gen ^^^ ClassesCtor20
% package ClassesCtor20;
%~~vis
\begin{xten}
class C(x:Int) {
  static def example() {
    val c : C = new C(4);
    assert c.x == 4;
  }
}
\end{xten}
%~~siv
% class Hook{ def run() {C.example(); return true;}}
%~~neg
\noindent 


The following class has the same default constructor.  However, that
constructor does not initialize \xcd`d`, and thus is invalid.  This 
class does not compile; it needs an explicit constructor.
%~~gen ^^^ ClassCtor30_MustFailCompile
% NOCOMPILE
%~~vis
\begin{xten}
// THIS CODE DOES NOT COMPILE
class Cfail(x:Int) {
  val d: Int;
  static def example() {
    val wrong = new Cfail(40);
  }
}
\end{xten}
%~~siv
%
%~~neg


\end{ex}

\subsection{Calling Other Constructors}
\label{sect:call-another-ctor}

The {\em first} statement of a constructor body may be a call of the form 
\xcd`this(a,b,c)` or \xcd`super(a,b,c)`.  The former will execute the body of
the matching constructor of the current class; the latter, of the superclass. 
This allows a measure of abstraction in constructor definitions; one may be
defined in terms of another.

\begin{ex}
The following class has two constructors.  \xcd`new Ctors(123)` constructs a
new \xcd`Ctors` object with parameter 123.  \xcd`new Ctors()` constructs one
whose parameter has a default value of 100: 
%~~gen ^^^ Classes5q6q
% package Classes5q6q;
%~~vis
\begin{xten}
class Ctors {
  public val a : Int;
  def this(a:Int) { this.a = a; }
  def this()      { this(100);  }
}
\end{xten}
%~~siv
%class Hook{ def run() {
% val x = new Ctors(10); assert x.a == 10;
% val y = new Ctors(); assert y.a == 100;
% return true;}}
%~~neg
\end{ex}

In the case of a class which implements \xcd`operator ()` 
--- or any other constructor and application with the same signature --- 
this can be ambiguous.  If \xcd`this()` appears as the first statement of a
constructor body, it could, in principle, mean either a constructor call or an
operator evaluation.   This ambiguity is resolved so that \xcd`this()` always
means the constructor invocation.  If, for some reason, it is necessary to
invoke an application operator as the first meaningful statement of a
constructor, write the target of the application as \xcd`(this)`, \eg,
\xcd`(this)(a,b);`. 

\subsection{Return Type of Constructor}

A constructor for class \xcd`C` may have a return type \xcd`C{c}`.  The return
type specifies a constraint on the kind of object returned.  It cannot change
its {\em class} --- a constructor for class \xcd`C` always returns an instance
of class \xcd`C`.  
If no explicit return type is specified, the constructor's return type is
inferred.

\begin{ex}
The constructor \xcd`(A)` below, having no explicit return type, 
has its return type inferred.  
\xcd`n` is set by the \xcd`property` statement to \xcd`1`, so the return type
is inferred as \xcd`Ret{self.n==1}.`
The constructor \xcd`(B)` has \xcd`Ret{n==self.n}` as an explicit return type.
The \xcd`example()` code shows both of these in action.

%~~gen ^^^ Classes1v9a
% package Classes1v9a;
%~~vis
\begin{xten}
class Ret(n:Int) {
   def this()    { property(1); }     // (A)
   def this(n:Int) : Ret{n==self.n} { // (B)
      property(n);
   }
   static def typeIs[T](x:T){}
   static def example() {
     typeIs[Ret{self.n==1}](new Ret());  // uses (A)
     typeIs[Ret{self.n==3}](new Ret(3)); // uses (B)
   }
}
\end{xten}
%~~siv
%
%~~neg


\end{ex}

\section{Static initialization}
\label{StaticInitialization}
\index{initialization!static} 
Static fields in \Xten{} are immutable and are guaranteed to be
initialized before they are accessed. Static fields are initialized on
a per-Place basis; thus an activity that reads a static field in two
different Places may read different values for the content of the
field in each Place.  Static fields are not eagerly initialized, thus
if a particular static field is not accessed in a given Place then the
initializer expression for that field may not be evaluated in that
Place.

When an activity running in a Place \Xcd{P} attempts to read a static
field \Xcd{F} that has not yet been initialized in \Xcd{P}, then the
activity will evaluate the initializer expression for \Xcd{F} and
store the resulting value in \Xcd{F}. It is guaranteed that at most
one activity in each Place will attempt to evaluate the initializer
expression for a given static field.  If a second activity attempts to
read \Xcd{F} while the first activity is still executing the
initializer expression the second activity will be suspended until the
first activity finishes evaluating the initializer and stores the
resulting value in \Xcd{F}.

The initializer expression may directly or indirectly read other
static fields in the program.  If there is a cycle in the field
initialization dependency graph for a set of static fields, then any
activities accessing those fields may deadlock, which in turn may
result in the program deadlocking.\footnote{The current \Xten{}
  runtime does not dynamically detect this situation. Future versions
  of \Xten may be able to detect this and convert such a deadlock into the
  throwing of an \Xcd{ExceptionInInitializer} exception.}.

If an exception is raised during the evaluation of a static field's
initializer expression, then the field is deemed uninitializable in
that Place and any subsequent attempt to access the static field's
value by another activity in the Place will also result in an
exception being raised.\footnote{The implementation will make a best
  effort attempt to present stack trace information about the original
  cause of the exception in all subsequent raised exceptions}.  Failure
to initialize a field in one Place does not impact the initialization
status of the field in other Places.

\subsection{Compatability with Prior Versions of \Xten{}}
Previous versions of \Xten{} eagerly initialized all static fields in
the program at Place 0 and serialized the resulting values to all
other Places before beginning execution of the user main function.  It
is possible to simulate these serialization semantics for specific
static fields under the lazy per-Place initialization semantics
by using the idiom below:

\begin{xten}
// Pre X10 2.3 code
// expr evaluated once in Place 0 and resulting value 
// serialized to all other places
public static val x:T = expr;

// X10 2.3 code when T haszero is false
private static val x_holder:Cell[T] = 
    (here == Place.FIRST_PLACE) ? new Cell[T](expr): null;
public static val x:T = at (Place.FIRST_PLACE) x_holder();

// simpler X10 2.3 code when T haszero is true
private static val x_holder:T = 
    (here == Place.FIRST_PLACE) ? expr : Zero.get[T]();
public static val x:T = at (Place.FIRST_PLACE) x_holder;

\end{xten}

A slightly more complex variant of the above idiom in which the
initializer expression for the public field conditionally does the \xcd{at}
only when not executed at \xcd{Place.FIRST_PLACE} can be used to
obtain exactly the same serialization behavior as the pre \Xten{} v2.3
semantics.  When necessary, eager initialization for specific static fields
can be simulated by reading the static fields in \xcd{main} before
executing the rest of the program.

\section{User-Defined Operators}
\label{sect:operators}
\index{operator}
\index{operator!user-defined}

%##(MethodDeclaration
\begin{bbgrammar}
%(FROM #(prod:MethodDecln)#)
         MethodDecln \: MethMods \xcd"def" Id TypeParams\opt Formals Guard\opt Throws\opt HasResultType\opt MethodBody & (\ref{prod:MethodDecln}) \\
                     \| BinOpDecln \\
                     \| PrefixOpDecln \\
                     \| ApplyOpDecln \\
                     \| SetOpDecln \\
                     \| ConversionOpDecln \\
\end{bbgrammar}
%##)


It is often convenient to have methods named by symbols rather than words.
For example, suppose that we wish to define a \xcd`Poly` class of
polynomials -- for the sake of illustration, single-variable polynomials with
\xcd`Int` coefficients.  It would be very nice to be able to manipulate these
polynomials by the usual operations: \xcd`+` to add, \xcd`*` to multiply,
\xcd`-` to subtract, and \xcd`p(x)` to compute the value of the polynomial at
argument \xcd`x`.  We would like to write code thus: 
%~~gen ^^^ Classes160
% package Classes.In.Poly101;
% // Integer-coefficient polynomials of one variable.
% class Poly {
%   public val coeff : Rail[Int];
%   public def this(coeff: Rail[Int]) { this.coeff = coeff;}
%   public def degree() = (coeff.size-1) as Int;
%   public def a(i:Int) = (i<0 || i>this.degree()) ? 0 : coeff(i);
%
%   public static operator (c : Int) as Poly = new Poly([c as Int]);
%
%   public operator this(x:Long) {
%     val d = this.degree();
%     var s : Long = this.a(d);
%     for( i in 1 .. this.degree() ) {
%        s = x * s + a(d-i);
%     }
%     return s;
%   }
%
%   public operator this + (p:Poly) =  new Poly(
%      new Rail[Int](
%         Math.max(this.coeff.size, p.coeff.size) as Int,
%         (i:Int) => this.a(i) + p.a(i)
%      ));
%   public operator this - (p:Poly) = this + (-1)*p;
%
%   public operator this * (p:Poly) = new Poly(
%      new Rail[Int](
%        this.degree() + p.degree() + 1,
%        (k:Int) => sumDeg(k, this, p)
%        )
%      );
%
%
%   public operator (n : Int) + this = (n as Poly) + this;
%   public operator this + (n : Int) = (n as Poly) + this;
%
%   public operator (n : Int) - this = (n as Poly) + (-1) * this;
%   public operator this - (n : Int) = ((-n) as Poly) + this;
%
%   public operator (n : Int) * this = new Poly(
%      new Rail[Int](
%        this.degree()+1,
%        (k:Int) => n * this.a(k)
%      ));
%   private static def sumDeg(k:Int, a:Poly, b:Poly) {
%      var s : Int = 0;
%      for( i in 0 .. k ) s += a.a(i) * b.a(k-i);
%        // x10.io.Console.OUT.println("sumdeg(" + k + "," + a + "," + b + ")=" + s);
%      return s;
%      };
%   public final def toString() = {
%      var allZeroSoFar : Boolean = true;
%      var s : String ="";
%      for( i in 0..this.degree() ) {
%        val ai = this.a(i);
%        if (ai == 0) continue;
%        if (allZeroSoFar) {
%           allZeroSoFar = false;
%           s = term(ai, i);
%        }
%        else
%           s +=
%              (ai > 0 ? " + " : " - ")
%             +term(ai, i);
%      }
%      if (allZeroSoFar) s = "0";
%      return s;
%   }
%   private final def term(ai: Int, n:Int) = {
%      val xpow = (n==0) ? "" : (n==1) ? "x" : "x^" + n ;
%      return (ai == 1) ? xpow : "" + Math.abs(ai) + xpow;
%   }
%
%   public static def Main(ss:Rail[String]):void {main(ss);};
%


%~~vis
\begin{xten}
  public static def main(Rail[String]):void {
     val X = new Poly([0,1]);
     val t <: Poly = 7 * X + 6 * X * X * X; 
     val u <: Poly = 3 + 5*X - 7*X*X;
     val v <: Poly = t * u - 1;
     for( i in -3 .. 3) {
       x10.io.Console.OUT.println(
         "" + i + "	X:" + X(i) + "	t:" + t(i) 
         + "	u:" + u(i) + "	v:" + v(i)
         );
     }
  }

\end{xten}
%~~siv
%}
%~~neg

Writing the same code with method calls, while possible, is far less elegant: 
%~~gen ^^^ Classes170

%package Classes.In.Remedial.Poly101;
% // Integer-coefficient polynomials of one variable.
% class UglyPoly {
%   public val coeff : Rail[Int];
%   public def this(coeff: Rail[Int]) { this.coeff = coeff;}
%   public def degree() = (coeff.size-1) as Int;
%   public  def  a(i:Int) = (i<0 || i>this.degree()) ? 0 : coeff(i);
%
%   public static operator (c : Int) as UglyPoly = new UglyPoly([c as Int]);
%
%   public def apply(x:Int) {
%     val d = this.degree();
%     var s : Int = this.a(d);
%     for( i in 1 .. this.degree() ) {
%        s = x * s + a(d-i);
%     }
%     return s;
%   }
%
%   public operator this + (p:UglyPoly) =  new UglyPoly(
%      new Rail[Int](
%         Math.max(this.coeff.size, p.coeff.size) as Int,
%         (i:Int) => this.a(i) + p.a(i)
%      ));
%   public operator this - (p:UglyPoly) = this + (-1)*p;
%
%   public operator this * (p:UglyPoly) = new UglyPoly(
%      new Rail[Int](
%        this.degree() + p.degree() + 1,
%        (k:Int) => sumDeg(k, this, p)
%        )
%      );
%
%
%   public operator (n : Int) + this = (n as UglyPoly) + this;
%   public operator this + (n : Int) = (n as UglyPoly) + this;
%
%   public operator (n : Int) - this = (n as UglyPoly) + (-1) * this;
%   public operator this - (n : Int) = ((-n) as UglyPoly) + this;
%
%   public operator (n : Int) * this = new UglyPoly(
%      new Rail[Int](
%        this.degree()+1,
%        (k:Int) => n * this.a(k)
%      ));
%   private static def sumDeg(k:Int, a:UglyPoly, b:UglyPoly) {
%      var s : Int = 0;
%      for( i in 0 .. k ) s += a.a(i) * b.a(k-i);
%        // x10.io.Console.OUT.println("sumdeg(" + k + "," + a + "," + b + ")=" + s);
%      return s;
%      };
%   public final def toString() = {
%      var allZeroSoFar : Boolean = true;
%      var s : String ="";
%      for( i in 0..this.degree() ) {
%        val ai = this.a(i);
%        if (ai == 0) continue;
%        if (allZeroSoFar) {
%           allZeroSoFar = false;
%           s = term(ai, i);
%        }
%        else
%           s +=
%              (ai > 0 ? " + " : " - ")
%             +term(ai, i);
%      }
%      if (allZeroSoFar) s = "0";
%      return s;
%   }
%   private final def term(ai: Int, n:Int) = {
%      val xpow = (n==0) ? "" : (n==1) ? "x" : "x^" + n ;
%      return (ai == 1) ? xpow : "" + Math.abs(ai) + xpow;
%   }
%
%   def mult(p:UglyPoly) : UglyPoly = this * p;
%   def mult(n:Int)      : UglyPoly = n * this;
%   def plus(p:UglyPoly) : UglyPoly = this + p;
%   def plus(n:Int)      : UglyPoly = n + this;
%   def minus(p:UglyPoly): UglyPoly = this - p;
%   def minus(n:Int)     : UglyPoly = this - n;
%   static def const(n:Int): UglyPoly = n as UglyPoly;
%
%~~vis
\begin{xten}
  public static def uglymain() {
     val X = new UglyPoly([0,1]);
     val t <: UglyPoly 
           = X.mult(7).plus(
               X.mult(X).mult(X).mult(6));  
     val u <: UglyPoly 
           = const(3).plus(
               X.mult(5)).minus(X.mult(X).mult(7));
     val v <: UglyPoly = t.mult(u).minus(1);
     for( i in -3 .. 3) {
       x10.io.Console.OUT.println(
         "" + i + "	X:" + X.apply(i) + "	t:" + t.apply(i) 
          + "	u:" + u.apply(i) + "	v:" + v.apply(i)
         );
     }
  }
\end{xten}
%~~siv
%}
%~~neg

The operator-using code can be written in X10, though a few variations are
necessary to handle such exotic cases as \xcd`1+X`.



Most X10 operators can be given definitions.\footnote{Indeed, even for the
standard types, these operators are defined in the library.  Not even as basic
an operation as integer addition is built into the language.  Conversely, if
you define a full-featured numeric type, it will have most of the privileges that
the standard ones enjoy.  The missing priveleges are (1) literals; (2) 
the \xcd`..` operator won't compute the \xcd`zeroBased` and \xcd`rail`
properties as it does for \xcd`Int` ranges; (3) \xcd`*` won't track ranks, as
it does for \xcd`Region`s; 
(4) \xcd`&&` and \xcd`||` won't short-circuit, as they do for \xcd`Boolean`s, 
(5) the built-in notion of equality \xcd`a==b` will only coincide with
the programmible notion \xcd`a.equals(b)`, as they do for most library types,
if coded that way; and (6) it is 
impossible to define an 
operation like \xcd`String.+` which converts both its left and right arguments
from any type.  For example, a \xcd`Polar` type might
have many representations for the origin, as radius 0 and any angle; these
will be \xcd`equals()`, but will not be \xcd`==`; whereas for the standard
\xcd`Complex` type, the two equalities are the same.}  (However, \xcd`&&` and
\xcd`||` 
are only short-circuiting for \xcd`Boolean` expressions; user-defined versions
of these operators have no special execution behavior.)

The user-definable operations are (in order of precedence): \\
\begin{tabular}{l}
implicit type coercions\\
postfix \xcd`()`\\
\xcd`as T`\\
these unary operators:  \xcd`- + ! ~ | & / ^ * %`\\
\xcd`..`\\
\xcd`*      /       %      **`\\
\xcd`+` \xcd`     -` \\
\xcd`<<     >>      >>>    ->     <-     >-      -<      !`\\
\xcd`>      ` \xcd`>=     ` \xcd`<     ` \xcd`<=     ` 
\xcd`~      !~`\\
\xcd`&` \\
\xcd`^` \\
\xcd`|` \\
\xcd`&&` \\
\xcd`||` \\
\end{tabular}

Several of these operators have no standard meaning on any library type, and
are included purely for programmer convenience.  


Many operators may be defined either in \xcd`static` or instance forms.  Those
defined in instance form are dynamically dispatched, just like an instance
method.  Those defined in static form are statically dispatched, just like a
static method.  Operators are scoped like methods; static operators are scoped
like static methods.

\begin{ex}
%~~gen ^^^ Classes6a1j
% package oifClasses6a1j;
% class Whatever {
% 
%~~vis
\begin{xten}
static class Trace(n:Int){
  public static operator !(f:Trace) 
      = new Trace(10 * f.n + 1);
  public operator -this = new Trace (10 * this.n + 2);
}
static class Brace extends Trace{
  def this(n:Int) { super(n); }
  public operator -this = new Brace (10 * this.n + 3);
  static def example() {
     val t = new Trace(1);
     assert (!t).n == 11;
     assert (-t).n == 12 && (-t instanceof Trace);
     val b = new Brace(1);
     assert (!b).n == 11;
     assert (-b).n == 13 && (-b instanceof Brace);
  }
}

\end{xten}
%~~siv
% // And checking the unambiguous syntax while I'm here...
% //static class Glook { def checky(t:Trace) { 
% //   Trace.operator !(t);
% //   t.operator -();
% //} }
% }
%~~neg
\end{ex}

%%OP%% Operators may be invoked by unambiguous syntax, loosely akin to a
%%OP%% fully-qualified name. For example, \xcd`!t` above may be invoked as
%%OP%% \xcd`Trace.operator !(t)`. This unambiguous syntax may be used even if there
%%OP%% are several \xcd`!` operators that could apply to \xcd`t`, rendering the
%%OP%% convenient short form \xcd`!t` unavailable in some context.




\subsection{Binary Operators}

Binary operators, illustrated by \xcd`+`, may be defined statically in a
container \xcd`A` as:
\begin{xten}
static operator (b:B) + (c:C) = ...;
\end{xten}
%%OP%% In this case it may be invoked as \xcd`A.operator +(b,c)`.
Or, it may be defined as  as an instance operator by one of the forms:
\begin{xten}
operator this + (b:B) = ...;
operator (b:B) + this = ...;
\end{xten}
%%OP%% and be invoked as 
%%OP%% \xcd`a.operator +(b)`
%%OP%% and as 
%%OP%% \xcd`a.operator ()+(b)` 
%%OP%% respectively.

\begin{ex}

Defining the sum \xcd`P+Q` of two polynomials looks much like a method
definition.  It uses the \xcd`operator` keyword instead of \xcd`def`, and
\xcd`this` appears in the definition in the place that a \xcd`Poly` would
appear in a use of the operator.  So, 
\xcd`operator this + (p:Poly)` explains how to add \xcd`this` to a
\xcd`Poly` value.
%~~gen ^^^ Classes180
% package Classes.In.Poly102;
%~~vis
\begin{xten}
class Poly {
  public val coeff : Rail[Int];
  public def this(coeff: Rail[Int]) { 
    this.coeff = coeff;}
  public def degree() = coeff.size-1 as Int;
  public def  a(i:Int) 
    = (i<0 || i>this.degree()) ? 0 : coeff(i);
  public operator this + (p:Poly) =  new Poly(
     new Rail[Int](
        Math.max(this.coeff.size, p.coeff.size) as Int,
        (i:Int) => this.a(i) + p.a(i)
     )); 
  // ... 
\end{xten}
%~~siv
%   public operator (n : Int) + this = new Poly([n as Int]) + this;
%   public operator this + (n : Int) = new Poly([n as Int]) + this;
% 
%   def makeSureItWorks() {
%      val x = new Poly([0,1]);
%      val p <: Poly = x+x+x;
%      val q <: Poly = 1+x;
%      val r <: Poly = x+1;
%   }
%     
% }
%~~neg


The sum of a polynomial and an integer, \xcd`P+3`, looks like
an overloaded method definition.  
%~~gen ^^^ Classes190
% package Classes.In.Poly103;
% class Poly {
%   public val coeff : Rail[Int];
%   public def this(coeff: Rail[Int]) { this.coeff = coeff;}
%   public def degree() = coeff.size-1 as Int;
%   public def  a(i:Int) = (i<0 || i>this.degree()) ? 0 : coeff(i);
% 
%   public operator this + (p:Poly) =  new Poly(
%      new Rail[Int](
%         Math.max(this.coeff.size, p.coeff.size) as Int,
%         (i:Int) => this.a(i) + p.a(i)
%      ));
%    public operator (n : Int) + this = new Poly([n as Int]) + this;
%~~vis
\begin{xten}
   public operator this + (n : Int) 
          = new Poly([n as Int]) + this;
\end{xten}
%~~siv
% 
%   def makeSureItWorks() {
%      val x = new Poly([0,1]);
%      val p <: Poly = x+x+x;
%      val q <: Poly = 1+x;
%      val r <: Poly = x+1;
%   }
%     
% }
%~~neg


However, we want to allow the sum of an integer and a polynomial as well:
\xcd`3+P`.  It would be quite inconvenient to have to define this as a method
on \xcd`Int`; changing \xcd`Int` is far outside of normal coding.  So, we
allow it as a method on \xcd`Poly` as well.


%~~gen ^^^ Classes200
% package Classes.In.Poly104o;
% class Poly {
%   public val coeff : Rail[Int];
%   public def this(coeff: Rail[Int]) { this.coeff = coeff;}
%   public def degree() = coeff.size-1 as Int;
%   public def  a(i:Int) = (i<0 || i>this.degree()) ? 0 : coeff(i);
% 
%   public operator this + (p:Poly) =  new Poly(
%      new Rail[Int](
%         Math.max(this.coeff.size, p.coeff.size) as Int,
%         (i:Int) => this.a(i) + p.a(i)
%      ));
%~~vis
\begin{xten}
   public operator (n : Int) + this 
          = new Poly([n as Int]) + this;
\end{xten}
%~~siv
% 
%   public operator this + (n : Int) = new Poly([n as Int]) + this;
%   def makeSureItWorks() {
%      val x = new Poly([0,1]);
%      val p <: Poly = x+x+x;
%      val q <: Poly = 1+x;
%      val r <: Poly = x+1;
%   }
%     
% }
%~~neg

Furthermore, it is sometimes convenient to express a binary operation as a
static method on a class. 
The definition for the sum of two
\xcd`Poly`s could have been written:
%~~gen ^^^ Classes210
% package Classes.In.Poly105;
% class Poly {
%   public val coeff : Rail[Int];
%   public def this(coeff: Rail[Int]) { this.coeff = coeff;}
%   public def degree() = coeff.size-1 as Int;
%   public def  a(i:Int) = (i<0 || i>this.degree()) ? 0 : coeff(i);
%~~vis
\begin{xten}
  public static operator (p:Poly) + (q:Poly) =  new Poly(
     new Rail[Int](
        Math.max(q.coeff.size, p.coeff.size) as Int,
        (i:Int) => q.a(i) + p.a(i)
     ));
\end{xten}
%~~siv
%
%   public operator (n : Int) + this = new Poly([n as Int]) + this;
%   public operator this + (n : Int) = new Poly([n as Int]) + this;
% 
%   def makeSureItWorks() {
%      val x = new Poly([0,1]);
%      val p <: Poly = x+x+x;
%      val q <: Poly = 1+x;
%      val r <: Poly = x+1;
%   }
%     
% }
%~~neg

\end{ex}

When X10 attempts to typecheck a binary operator expression like \xcd`P+Q`, it
first typechecks \xcd`P` and \xcd`Q`. Then, it looks for operator declarations
for \xcd`+` in the types of \xcd`P` and \xcd`Q`. If there are none, it is a
static error. If there is precisely one, that one will be used. If there are
several, X10 looks for a {\em best-matching} operation, \viz{} one which does
not require the operands to be converted to another type. For example,
\xcd`operator this + (n:Long)` and \xcd`operator this + (n:Int)` both apply to
\xcd`p+1`, because \xcd`1` can be converted from an \xcd`Int` to a \xcd`Long`.
However, the \xcd`Int` version will be chosen because it does not require a
conversion. If even the best-matching operation is not uniquely determined,
the compiler will report a static error.


\subsection{Unary Operators}

Unary operators,  illustrated by \xcd`!`, may be defined statically in
container 
\xcd`A` as 
\begin{xten}
static operator !(x:A) = ...;
\end{xten}
or as instance operators by: 
\begin{xten}
operator !this = ...;
\end{xten}

%%OP%% A statically-defined unary operator \xcd`!` may be invoked on \xcd`a:A` as 
%%OP%% \xcd`A.operator !(a)`.  An instance operator may be invoked as
%%OP%% \xcd`a.operator !()`.  

The rules for typechecking a unary operation are the same as for methods; the
complexities of binary operations are not needed.

\begin{ex}
The operator to negate a polynomial is: 

%~~gen ^^^ Classes220
% package Classes.In.Poly106;
% class Poly {
%   public val coeff : Rail[Int];
%   public def this(coeff: Rail[Int]) { this.coeff = coeff;}
%   public def degree() = coeff.size-1 as Int;
%   public def  a(i:Int) = (i<0 || i>this.degree()) ? 0 : coeff(i);
%~~vis
\begin{xten}
  public operator - this = new Poly(
    new Rail[Int](coeff.size as Int, (i:Int) => -coeff(i))
    );
\end{xten}
%~~siv
%   def makeSureItWorks() {
%      val x = new Poly([0,1]);
%      val p <: Poly = -x;
%   }
% }
%~~neg



\end{ex}


\subsection{Type Conversions}
\label{sect:type-conv}
\index{type conversion!user-defined}


Explicit type conversions, \xcd`e as A`, can be defined as operators on
class \xcd`A`, or on the container type of \xcd`e`.  These must be static
operators.  

To define an operator in \xcd`class A` (or \xcd`struct A)` converting values
of type \xcd`B` into type \xcd`A`, use the syntax: 
\begin{xten}
static operator (x:B) as ? {c} = ... 
\end{xten}
The \xcd`?` indicates the containing type \xcd`A`.  
The guard clause \xcd`{c}` may be omitted.



\begin{ex}
%~~gen ^^^ Classes230
% package Classes_explicit_type_conversions_a;
%~~vis
\begin{xten}
class Poly {
  public val coeff : Rail[Int];
  public def this(coeff: Rail[Int]) { this.coeff = coeff;}
  public static operator (a:Int) as ? = new Poly([a as Int]);
  public static def main(Rail[String]):void {
     val three : Poly = 3 as Poly;
  }
}
\end{xten}
%~~siv
%
%~~neg
\end{ex}
The \xcd`?` may be given a bound, such as \xcd`as ? <: Caster`, if desired.
  

There is little difference between an explicit conversion \xcd`e as T` and a
method call \xcd`e.asT()`.  The explicit conversion does say undeniably what
the result type will be.  However, as described in \Sref{sect:ambig-cast},
sometimes the built-in meaning of \xcd`as` as a cast overrides the
user-defined explicit conversion.  

Explicit casts are most suitable for cases
which resemble the use of explicit casts among the arithmetic types, where, 
for example, \xcd`1.0 as Int` is a way to turn a floating-point number into the
corresponding integer.  
While there is nothing in X10 which
requires it, \xcd`e as T` has the connotation that it gives a good
approximation of \xcd`e` in type \xcd`T`, just as \xcd`1` is a good
(indeed, perfect) approximation of \xcd`1.0` in type \xcd`Int`.  

\subsection{Implicit Type Coercions}
\label{sect:ImplicitCoercion}
\index{type conversion!implicit}

An implicit type conversion from \xcd`U`  to \xcd`T` may be specified in
container \xcd`T`.  
The syntax for it is: 
\begin{xten}
static operator (u:U) : T = e;
\end{xten}
%%OP%% which may be invoked by the unambiguous syntax 
%%OP%% \xcd`T.operator[T](u)` or \xcd`U.operator[T](u)`.
%%OP%% 



Implicit coercions are used automatically by the compiler on method calls 
(\Sref{sect:MethodResolution}) and assignments (\Sref{AssignmentStatement}).
Implicit coercions may be used when a value of type \xcd`T` appears in a
context expecting a value of type \xcd`U`.  If \xcd`T <: U`, no implicit
coercion is needed; \eg, a method \xcd`m` expecting an \xcd`Int` argument may 
be called as \xcd`m(3)`, with an argument of type \xcd`Int{self==3}`, which is
a subtype of \xcd`m`'s argument type \xcd`Int`. 
However, if it is not the case that \xcd`T <: U`, but there is an implicit
coercion from \xcd`T` to \xcd`U` defined in container \xcd`U`, then this
implicit coercion will be applied.

\begin{ex}
We can define an implicit coercion from \xcd`Int` to \xcd`Poly`,
and avoid having to define the sum of an integer and a polynomial
as many special cases.  In the following example, we only define \xcd`+` on
two polynomials.  The
calculation \xcd`1+x` coerces \xcd`1` to a polynomial and uses polynomial
addition to add it to \xcd`x`.

%~~gen ^^^ Classes240
% package Classes.And.Implicit.Coercions;
% class Poly {
%   public val coeff : Rail[Int];
%   public def this(coeff: Rail[Int]) { this.coeff = coeff;}
%   public def degree() = (coeff.size-1) as Int;
%   public def  a(i:Int) = (i<0 || i>this.degree()) ? 0 : coeff(i);
%   public final def toString() = {
%      var allZeroSoFar : Boolean = true;
%      var s : String ="";
%      for( i in 0..this.degree() ) {
%        val ai = this.a(i);
%        if (ai == 0) continue;
%        if (allZeroSoFar) {
%           allZeroSoFar = false;
%           s = term(ai, i);
%        }
%        else 
%           s += 
%              (ai > 0 ? " + " : " - ")
%             +term(ai, i);
%      }
%      if (allZeroSoFar) s = "0";
%      return s;
%   }
%   private final def term(ai: Int, n:Int) = {
%      val xpow = (n==0) ? "" : (n==1) ? "x" : "x^" + n ;
%      return (ai == 1) ? xpow : "" + Math.abs(ai) + xpow;
%   }

%~~vis
\begin{xten}
  public static operator (c : Int) : Poly 
     = new Poly([c as Int]);

  public static operator (p:Poly) + (q:Poly) = new Poly(
      new Rail[Int](
        Math.max(p.coeff.size, q.coeff.size) as Int,
        (i:Int) => p.a(i) + q.a(i)
     ));

  public static def main(Rail[String]):void {
     val x = new Poly([0,1]);
     x10.io.Console.OUT.println("1+x=" + (1+x));
  }
\end{xten}
%~~siv
%}
%~~neg
\end{ex}



\subsection{Assignment and Application Operators}
\index{assignment operator}
\index{application operator}
\index{()}
\index{()=}
\label{set-and-apply}
X10 allows types to implement the subscripting / function application
operator, and indexed assignment.  The \xcd`Array`-like classes take advantage
of both of these in \xcd`a(i) = a(i) + 1`.  

\xcd`a(b,c,d)`
is an operator call, to an operator defined with 
\xcd`public operator this(b:B, c:C, d:D)`.  It may be overloaded.
For
example, an ordered dictionary structure could allow subscripting by numbers
with \xcd`public operator this(i:Int)`, and by strings with 
\xcd`public operator this(s:String)`.  


\xcd`a(i,j)=b` is an \xcd`operator` as well, with zero or more indices
\xcd`i,j`.  It may also be overloaded. 

The update operations \xcd`a(i) += b` 
(for all binary operators in place of \xcd`+`)
are defined to be the same as the
corresponding \xcd`a(i) = a(i) + b`. This applies for all arities of
arguments, and all types, and all binary operations. Of course to use this,
the \xcd`+`, application and assignment \xcd`operator`s must be defined.


\begin{ex}

The \xcd`Oddvec` class of somewhat peculiar vectors illustrates this.

\xcd`a()` returns a string representation of the oddvec, which ordinarily
would 
0be done by \xcd`toString()` instead.  
\xcd`a(i)` sensibly picks out one of the three
coordinates of \xcd`a`.
\xcd`a()=b` sets all the coordinates of \xcd`a` to \xcd`b`.
\xcd`a(i)=b` assigns to one of the
coordinates.  \xcd`a(i,j)=b` assigns different values to \xcd`a(i)` and
\xcd`a(j)`.  

%~~gen ^^^ Classes250
% package Classes.Assignments1_oddvec;
%~~vis
\begin{xten}
class Oddvec {
  var v : Rail[Int] = new Rail[Int](3);
  public operator this () = 
      "(" + v(0) + "," + v(1) + "," + v(2) + ")";
  public operator this () = (newval: Int) { 
    for(p in v.range) v(p) = newval;
  }
  public operator this(i:Int) = v(i);
  public operator this(i:Int, j:Int) = [v(i),v(j)];
  public operator this(i:Int) = (newval:Int) 
      = {v(i) = newval;}
  public operator this(i:Int, j:Int) = (newval:Int) 
      = { v(i) = newval; v(j) = newval+1;} 
  public def example() {
    this(1) = 6;   assert this(1) == 6;
    this(1) += 7;  assert this(1) == 13;
  }
\end{xten}
%~~siv
% }
%  class Hook { def run() {
%     val a = new Oddvec();
%     assert a().equals("(0,0,0)");
%     a() = 1;
%     assert a().equals("(1,1,1)");
%     a(1) = 4;
%     assert a().equals("(1,4,1)");
%     a(0,2) = 5;
%     assert a().equals("(5,4,6)");
%     return true;
%   }
% }
%~~neg

\end{ex}

\section{Class Guards and Invariants}\label{DepType:ClassGuard}
\index{type invariants}
\index{class invariants}
\index{invariant!type}
\index{invariant!class}
\index{guard}


Classes (and structs and interfaces) may specify a {\em class guard}, a
constraint which must hold on all values of the class.    In the following
example, a \xcd`Line` is defined by two distinct \xcd`Pt`s\footnote{We use \xcd`Pt`
to avoid any possible confusion with the built-in class \xcd`Point`.}
%~~gen ^^^ Classes260
% package classes.guards.invariants.glurp;
%~~vis
\begin{xten}
class Pt(x:Int, y:Int){}
class Line(a:Pt, b:Pt){a != b} {}
\end{xten}
%~~siv
%
%~~neg

In most cases the class guard could be phrased as a type constraint on a property of
the class instead, if preferred.  Arguably, a symmetric constraint like two
points being different is better expressed as a class guard, rather than
asymmetrically as a constraint on one type: 
%~~gen ^^^ Classes270
% package classes.guards.invariants.glurp2;
% class Pt(x:Int, y:Int){}
%~~vis
\begin{xten}
class Line(a:Pt, b:Pt{a != b}) {}
\end{xten}
%~~siv
%
%~~neg



\label{DepType:TypeInvariant}
\index{class invariant}
\index{invariant!class}
\index{class!invariant}
\label{DepType:ClassGuardDef}



With every container  or interface \xcd"T" we associate a {\em type
invariant} $\mathit{inv}($\xcd"T"$)$, which describes the guarantees on the
properties of values of type \xcd`T`.  

Every value of \xcd`T` satisfies $\mathit{inv}($\xcd"T"$)$ at all times.  This
is somewhat stronger than the concept of type invariant in most languages
(which only requires that the invariant holds when no method calls are
active).  X10 invariants only concern properties, which are immutable; thus,
once established, they cannot be falsified.

The type
invariant associated with \xcd"x10.lang.Any"
is 
\xcd"true".

The type invariant associated with any interface or struct \xcd"I" that extends
interfaces \xcdmath"I$_1$, $\dots$, I$_k$" and defines properties
\xcdmath"x$_1$: P$_1$, $\dots$, x$_n$: P$_n$" and
specifies a guard \xcd"c" is given by:

\begin{xtenmath}
$\mathit{inv}$(I$_1$) && $\dots$ && $\mathit{inv}$(I$_k$) &&
self.x$_1$ instanceof P$_1$ &&  $\dots$ &&  self.x$_n$ instanceof P$_n$ 
&& c  
\end{xtenmath}

Similarly the type invariant associated with any class \xcd"C" that
implements interfaces \xcdmath"I$_1$, $\dots$, I$_k$",
extends class \xcd"D" and defines properties
\xcdmath"x$_1$: P$_1$, $\dots$, x$_n$: P$_n$" and
specifies a guard \xcd"c" is
given by the same thing with the invariant of the superclass \xcd`D` conjoined:
\begin{xtenmath}
$\mathit{inv}$(I$_1$) && $\dots$ && $\mathit{inv}$(I$_k$) 
&& self.x$_1$ instanceof P$_1$ &&  $\dots$ &&  self.x$_n$ instanceof P$_n$ 
&& c  
&& $\mathit{inv}$(D)
\end{xtenmath}


Note that the type invariant associated with a class entails the type
invariants of each interface that it implements (directly or indirectly), and
the type invariant of each ancestor class.
It is guaranteed that for any variable \xcd"v" of
type \xcd"T{c}" (where \xcd"T" is an interface name or a class name) the only
objects \xcd"o" that may be stored in \xcd"v" are such that \xcd"o" satisfies
$\mathit{inv}(\mbox{\xcd"T"}[\mbox{\xcd"o"}/\mbox{\xcd"this"}])
\wedge \mbox{\xcd"c"}[\mbox{\xcd"o"}/\mbox{\xcd"self"}]$.



\subsection{Invariants for {\tt implements} and {\tt extends} clauses}\label{DepType:Implements}
\label{DepType:Extends}
\index{type-checking!implements clause}
\index{type-checking!extends clause}
\index{implements}
\index{extends}
Consider a class definition
\begin{xtenmath}
$\mbox{\emph{ClassModifiers}}^{\mbox{?}}$
class C(x$_1$: P$_1$, $\dots$, x$_n$: P$_n$){c} extends D{d}
   implements I$_1${c$_1$}, $\dots$, I$_k${c$_k$}
$\mbox{\emph{ClassBody}}$
\end{xtenmath}

These two rules must be satisfied:


\begin{itemize}

\item 
The type invariant \xcdmath"$\mathit{inv}$(C)" of \xcd"C" must entail
\xcdmath"c$_i$[this/self]" for each $i$ in $\{1, \dots, k\}$


\item The return type \xcd"c" of each constructor in a class \xcd`C`
must entail the invariant \xcdmath"$\mathit{inv}$(C)".
\end{itemize}

\subsection{Timing of Invariant Checks}

\index{invariant!checked}

The invariants for a container are checked immediately after the
\xcd`property` statement in the container's constructor. 
This is the earliest that the invariant could possibly be checked. 
Recall that an invariant 
can mention the properties of the container (which are set, forever, at that
point in the code), but cannot mention the \xcd`val`
or \xcd`var` fields (which might not be set at that point), or \xcd`this`
(which might not have been fully initialized).  

If X10 can prove that the invariant always holds given the \xcd`property`
statement and other known information, it may omit the actual check.




\subsection{Invariants and constructor definitions}
\index{invariant!and constructor}
\index{constructor!and invariant}

A constructor for a class \xcd"C" is guaranteed to return an object of the
class on successful termination. This object must satisfy  \xcdmath"$\mathit{inv}$(C)", the
class invariant associated with \xcd"C" (\Sref{DepType:TypeInvariant}).
However,
often the objects returned by a constructor may satisfy {\em stronger}
properties than the class invariant. \Xten{}'s dependent type system
permits these extra properties to be asserted with the constructor in
the form of a constrained type (the ``return type'' of the constructor):

%##(CtorDeclaration
\begin{bbgrammar}
%(FROM #(prod:CtorDecln)#)
           CtorDecln \: Mods\opt \xcd"def" \xcd"this" TypeParams\opt Formals Guard\opt HasResultType\opt CtorBody & (\ref{prod:CtorDecln}) \\
\end{bbgrammar}
%##)

\label{ConstructorGuard}

The parameter list for the constructor
may specify a \emph{guard} that is to be satisfied by the parameters
to the list.

\begin{ex}
%%TODO--rewrite this
Here is another example, constructed as a simplified 
version of \Xcd{x10.array.Region}.  The \xcd`mockUnion` method 
has the type, though not the value, that a true \xcd`union` method would have.

%~~gen ^^^ Classes280
%package Classes.SimplifiedRegion;
%~~vis
\begin{xten}
class MyRegion(rank:Int) {
  static type MyRegion(n:Int)=MyRegion{rank==n};
  def this(r:Int):MyRegion(r) {
    property(r);
  }
  def this(diag:Rail[Int]):MyRegion(diag.size){ 
    property(diag.size);
  }
  def mockUnion(r:MyRegion(rank)):MyRegion(rank) = this;
  def example() {
    val R1 : MyRegion(3) = new MyRegion([4,4,4 as Int]); 
    val R2 : MyRegion(3) = new MyRegion([5,4,1]); 
    val R3 = R1.mockUnion(R2); // inferred type MyRegion(3)
  }
}
\end{xten}
%~~siv
%
%~~neg
The first constructor returns the empty region of rank \Xcd{r}.  The
second constructor takes a \Xcd{Array[Int](1)} of arbitrary length
\Xcd{n} and returns a \Xcd{MyRegion(n)} (intended to represent the set
of points in the rectangular parallelopiped between the origin and the
\Xcd{diag}.)

The code in \xcd`example` typechecks, and \xcd`R3`'s type is inferred as
\xcd`MyRegion(3)`.  


\end{ex}

   Let \xcd"C" be a class with properties
   \xcdmath"p$_1$: P$_1$, $\dots$, p$_n$: P$_n$", and invariant \xcd"c"
   extending the constrained type \xcd"D{d}" (where \xcd"D" is the name of a
   class).



   For every constructor in \xcd"C" the compiler checks that the call to
   super invokes a constructor for \xcd"D" whose return type is strong enough
   to entail \xcd"d". Specifically, if the call to super is of the form 
     \xcdmath"super(e$_1$, $\dots$, e$_k$)"
   and the static type of each expression \xcdmath"e$_i$" is
   \xcdmath"S$_i$", and the invocation
   is statically resolved to a constructor
\xcdmath"def this(x$_1$: T$_1$, $\dots$, x$_k$: T$_k$){c}: D{d$_1$}"
   then it must be the case that 
\begin{xtenmath}
x$_1$: S$_1$, $\dots$, x$_i$: S$_i$ entails x$_i$: T$_i$  (for $i \in \{1, \dots, k\}$)
x$_1$: S$_1$, $\dots$, x$_k$: S$_k$ entails c  
d$_1$[a/self], x$_1$: S$_1$, ..., x$_k$: S$_k$ entails d[a/self]      
\end{xtenmath}
\noindent where \xcd"a" is a constant that does not appear in 
\xcdmath"x$_1$: S$_1$ $\wedge$ ... $\wedge$ x$_k$: S$_k$".

   The compiler checks that every constructor for \xcd"C" ensures that
   the properties \xcdmath"p$_1$,..., p$_n$" are initialized with values which satisfy
   $\mathit{inv}($\xcd"T"$)$, and its own return type \xcd"c'" as follows.  In each constructor, the
   compiler checks that the static types \xcdmath"T$_i$" of the expressions \xcdmath"e$_i$"
   assigned to \xcdmath"p$_i$" are such that the following is
   true:
\begin{xtenmath}
p$_1$: T$_1$, $\dots$, p$_n$: T$_n$ entails $\mathit{inv}($T$)$ $\wedge$ c'     
\end{xtenmath}

(Note that for the assignment of \xcdmath"e$_i$" to \xcdmath"p$_i$"
to be type-correct it must be the
    case that \xcdmath"p$_i$: T$_i$ $\wedge$ p$_i$: P$_i$".) 



The compiler must check that every invocation \xcdmath"C(e$_1$, $\dots$, e$_n$)" to a
constructor is type correct: each argument \xcdmath"e$_i$" must have a static type
that is a subtype of the declared type \xcdmath"T$_i$" for the $i$th
argument of the
constructor, and the conjunction of static types of the argument must
entail the constraint in the parameter list of the constructor.

\section{Generic Classes}

Classes, like other units, can be generic.  They can be parameterized by
types.  The parameter types are used just like ordinary types inside the body
of the generic class -- with a few exceptions.  

\begin{ex}
A \xcd`Colorized[T]` holds a thing of type \xcd`T`, and a string which is intended to represent
its color.  Any type can be used for \xcd`T`; the \xcd`example` method shows
\xcd`Int` and \xcd`Boolean`.  The \xcd`thing()` method retrieves the thing;
note that its return type is the generic type variable \xcd`T`.  X10 is aware 
that \xcd`colInt.thing()` is an \xcd`Int` and \xcd`colTrue.thing()` is a
\xcd`Boolean`, and uses those typings in \xcd`example`. 
%~~gen ^^^ Classes6d9u
% package Classes6d9u;
%~~vis
\begin{xten}
class Colorized[T] {
  private var thing:T; 
  private var color:String; 
  def this(thing:T, color:String) {
     this.thing = thing;
     this.color = color;
  }
  public def thing():T = thing;
  public def color():String = color;  
  public static def example() {
    val colInt  : Colorized[Int] 
                = new Colorized[Int](3, "green");
    assert colInt.thing() == 3 
                && colInt.color().equals("green");
    val colTrue : Colorized[Boolean] 
                = new Colorized[Boolean](true, "blue");
    assert colTrue.thing() 
                && colTrue.color().equals("blue");
  }
}
\end{xten}
%~~siv
%class Hook{ def run() {Colorized.example(); return true;}}
%~~neg


\end{ex}



\subsection{Use of Generics}

An unconstrained type variable \Xcd{X} can be instantiated by any type. All
the operations of \Xcd{Any} are available on a 
variable of type \Xcd{X}. Additionally, variables of type
\Xcd{X} may be used with \Xcd{==, !=}, in \Xcd{instanceof}, and casts.  

If a type variable is constrained, the operations implied by its constraint
are available as well.

\begin{ex}
The interface \xcd`Named` describes entities which know their own name.  The
class \xcd`NameMap[T]` is a specialized map which stores and retrieves
\xcd`Named` entities by name.  The call \xcd`t.name()` in \xcd`put()` is only
valid because the constraint \xcd`{T <: Named}` implies that \xcd`T` is a
subtype of \xcd`Named`, and hence provides all the operations of \xcd`Named`. 
%~~gen ^^^ Types6e6x
% package Types6e6x;
% import x10.util.*;
%~~vis
\begin{xten}
interface Named { def name():String; }
class NameMap[T]{T <: Named} {
   val m = new HashMap[String, T]();
   def put(t:T) { m.put(t.name(), t); }
   def get(s:String):T = m.getOrThrow(s);
}
\end{xten}
%~~siv
%
%~~neg


\end{ex}





\section{Object Initialization}
\label{ObjectInitialization}
\index{initialization}
\index{constructor}
\index{object!constructor}
\index{struct!constructor}

% \noo{Confirm this chapter with the paper}

X10 does object initialization safely.  It avoids certain bad things which
trouble some other languages:
\begin{enumerate}
\item Use of a field before the field has been initialized.
\item A program reading two different values from a \xcd`val` field of a
      container. 
\item \Xcd{this} escaping from a constructor, which can cause problems as
      noted below. 

\end{enumerate}

It should be unsurprising that fields must not be used before they are
initialized. At best, it is uncertain what value will be in them, as in
\Xcd{x} below. Worse, the value might not even be an allowable value; \Xcd{y},
declared to be nonzero in the following example, might be zero before it is
initialized.
\begin{xten}
// Not correct X10
class ThisIsWrong {
  val x : Int;
  val y : Int{y != 0};
  def this() {
    x10.io.Console.OUT.println("x=" + x + "; y=" + y);
    x = 1; y = 2;
  }
}
\end{xten}

One particularly insidious way to read uninitialized fields is to allow
\Xcd{this} to escape from a constructor. For example, the constructor could
put \Xcd{this} into a data structure before initializing it, and another
activity could read it from the data structure and look at its fields:
\begin{xten}
class Wrong {
  val shouldBe8 : Int;
  static Cell[Wrong] wrongCell = new Cell[Wrong]();
  static def doItWrong() {
     finish {
       async { new Wrong(); } // (A)
       assert( wrongCell().shouldBe8 == 8); // (B)
     }
  }
  def this() {
     wrongCell.set(this); // (C) - ILLEGAL
     this.shouldBe8 = 8; // (D)
  }
}
\end{xten}
\noindent
In this example, the underconstructed \Xcd{Wrong} object is leaked into a
storage cell at line \Xcd{(C)}, and then initialized.  The \Xcd{doItWrong}
method constructs a new \Xcd{Wrong} object, and looks at the \Xcd{Wrong}
object in the storage cell to check on its \Xcd{shouldBe8} field.  One
possible order of events is the following:
\begin{enumerate}
\item \Xcd{doItWrong()} is called.
\item \Xcd{(A)} is started.  Space for a new \Xcd{Wrong} object is allocated.
      Its \Xcd{shouldBe8} field, not yet initialized, contains some garbage
      value.
\item \Xcd{(C)} is executed, as part of the process of constructing a new
      \Xcd{Wrong} object.  The new, uninitialized object is stored in
      \Xcd{wrongCell}.
\item Now, the initialization activity is paused, and execution of the main activity
      proceeds from \Xcd{(B)}.
\item The value in \Xcd{wrongCell} is retrieved, and is \Xcd{shouldBe8} field
      is read.  This field contains garbage, and the assertion fails.
\item Now let the initialization activity proceed with \Xcd{(D)},
      initializing \Xcd{shouldBe8} --- too late.
\end{enumerate}

The \xcd`at` statement (\Sref{AtStatement}) introduces the potential for
escape as well. The following class prints an uninitialized value: 
%~~gen ^^^ ThisEscapingViaAt_MustFailCompile
% package ObjInit_at;
% NOCOMPILE
%~~vis
\begin{xten}
// This code violates this chapter's constraints
// and thus will not compile in X10.
class Example {
  val a: Int;
  def this() { 
    at(here.next()) {
      // Recall that 'this' is a copy of 'this' outside 'at'.
      Console.OUT.println("this.a = " + this.a);
    }
    this.a = 1;
  }
}
\end{xten}
%~~siv
%
%~~neg


X10 must protect against such possibilities.  The rules explaining how
constructors can be written are somewhat intricate; they are designed to allow
as much programming as possible without leading to potential problems.
Ultimately, they simply are elaborations of the fundamental principles that
uninitialized fields must never be read, and \Xcd{this} must never be leaked.

%%RAW%% \subsection{Raw and Cooked Objects}
%%RAW%% \index{raw}
%%RAW%% \index{cooked}
%%RAW%% 
%%RAW%% An object is {\em raw} before its constructor ends, and {\em cooked} after its
%%RAW%% constructor ends. Note that, when an object is cooked, all its subobjects are
%%RAW%% cooked.  
%%RAW%% 



\subsection{Constructors and Non-Escaping Methods}
\index{non-escaping}
\label{sect:nonescaping}

In general, constructors must not be allowed to call methods with \Xcd{this} as
an argument or receiver. Such calls could leak references to \Xcd{this},
either directly from a call to \Xcd{cell.set(this)}, or indirectly because
\Xcd{toString} leaks \Xcd{this}, and the concatenation
\Xcd`"Escaper = "+this` calls \Xcd{toString}.\footnote{This is abominable behavior for
\Xcd{toString}, but it cannot be prevented -- save by a scheme such as we
present in this section.}
%~WRONG~gen
%package ObjectInit.CtorAndNonEscaping.One;
%~WRONG~vis
\begin{xten}
// This code violates this chapter's constraints
// and thus will not compile in X10.
class Escaper {
  static val Cell[Escaper] cell = new Cell[Escaper]();
  def toString() {
    cell.set(this);
    return "Evil!";
  }
  def this() {
    cell.set(this);
    x10.io.Console.OUT.println("Escaper = " + this);
  }
}
\end{xten}
%~WRONG~siv
%
%~WRONG~neg

However, it is convenient to be able to call methods from constructors; {\em
e.g.}, a class might have eleven constructors whose common behavior is best
described by three methods.
Under certain stringent conditions, it {\em is}
safe to call a method: the method called must not leak references to
\Xcd{this}, and must not read \Xcd{val}s or \Xcd{var}s which might not have
been assigned.

So, X10 performs a static dataflow analysis, sufficient to guarantee that
method calls in constructors are safe.  This analysis requires having access
to or guarantees about all the code that could possibly be called.  This can
be accomplished in two ways:
\begin{enumerate}
\item Ensuring that only code from the class itself can be called, by
      forbidding overriding of
      methods called from the constructor: they can be marked \Xcd{final} or
      \Xcd{private}, or the whole class can be \Xcd{final}.
\item Marking the methods called from the constructor by
      \xcd`@NonEscaping` or \xcd`@NoThisAccess`
\end{enumerate}

\subsubsection{Non-Escaping Methods}
\index{method!non-escaping}
\index{method!implicitly non-escaping}
\index{method!NonEscaping}
\index{implicitly non-escaping}
\index{non-escaping}
\index{non-escaping!implicitly}
\index{NonEscaping}


A method may be annotated with \xcd`@NonEscaping`.  This
imposes several restrictions on the method body, and on all methods overriding
it.  However, it is the only way that a method can be called from
constructors.  The
\Xcd{@NonEscaping} annotation makes explicit all the X10 compiler's needs for
constructor-safety.

A method can, however, be safe to call from constructors without being marked
\Xcd{@NonEscaping}. We call such methods {\em implicitly non-escaping}.
Implicitly non-escaping methods need to obey the same constraints on
\Xcd{this}, \Xcd{super}, and variable usage as \Xcd{@NonEscaping} methods. An
implicitly non-escaping method {\em could} be marked as
\xcd`@NonEscaping`; the compiler, in
effect, infers the annotation. In addition, all non-escaping methods
must be \Xcd{private} or \Xcd{final} or members of a \Xcd{final} class; this
corresponds to the hereditary nature of \Xcd{@NonEscaping} (by forbidding
inheritance of implicitly non-escaping methods).

We say that a method is {\em non-escaping} if it is either implicitly
non-escaping, or annotated \Xcd{@NonEscaping}.

The first requirement on non-escaping methods is that they do not allow
\Xcd{this} to escape. Inside of their bodies, \Xcd{this} and \Xcd{super} may
only be used for field access and assignment, and as the receiver of
non-escaping methods.


The following example uses the possible variations.  \Xcd{aplomb()} 
explicitly forbids reading any field but
\Xcd{a}. \Xcd{boric()} is called after \Xcd{a} and \Xcd{b} are set, but
\Xcd{c} is not.
The \xcd`@NonEscaping` annotation on \xcd`boric()` is optional, but the
compiler will print a warning if it is left out.
\Xcd{cajoled()} is only called after all fields are set, so it
can read anything; its annotation, too, is not required.   \Xcd{SeeAlso} is able to override \Xcd{aplomb()}, because
\Xcd{aplomb()} is \xcd`@NonEscaping`; it cannot override the final method
\Xcd{boric()} or the private one \Xcd{cajoled()}.  
%~~gen ^^^ ObjectInitialization10
%package ObjInit.C2;
%~~vis
\begin{xten}
import x10.compiler.*;

final class C2 {
  protected val a:Int; protected val b:Int; protected val c:Int;
  protected var x:Int; protected var y:Int; protected var z:Int;
  def this() {
    a = 1;
    this.aplomb();
    b = 2;
    this.boric();
    c = 3;
    this.cajoled();
  }
  @NonEscaping def aplomb() {
    x = a;
    // this.boric(); // not allowed; boric reads b.
    // z = b; // not allowed -- only 'a' can be read here
  }
  @NonEscaping final def boric() {
    y = b;
    this.aplomb(); // allowed; 
       // a is definitely set before boric is called
    // z = c; // not allowed; c is not definitely written
  }
  @NonEscaping private def cajoled() {
    z = c;
  }
}

\end{xten}
%~~siv
%
%~~neg

\subsubsection{NoThisAccess Methods}

A method may be annotated \xcd`@NoThisAccess`.  \xcd`@NoThisAccess` methods
may be called from constructors, and they may be overridden in subclasses.
However, they may not refer to \xcd`this` in any way -- in particular, they
cannot refer to fields of \xcd`this`, nor to \xcd`super`.

\begin{ex}

The class \xcd`IDed` has an \xcd`Float`-valued \xcd`id` field.  The method
\xcd`count()` is used to initialize the \xcd`id`.  For \xcd`IDed` objects,
the \xcd`id` is the count of \xcd`IDed`s created with the same parity of its
\xcd`kind`.   Note that \xcd`count()` does not refer to \xcd`this`, though
it does refer to a \xcd`static` field \xcd`counts`. 

The subclass \xcd`SubIDed` has \xcd`id`s that depend on \xcd`kind%3`
as well as the parity of \xcd`kind`.  It overrides the \xcd`count()`
method.  The body of \xcd`count()` still cannot refer to \xcd`this`.
Nor can it refer to \xcd`super` (which is \xcd`self` under another name).
This precludes the use of a \xcd`super` call.  This is why we have separated
the body of \xcd`count` out as the static method \xcd`kind2count` -- without
that, we would have had to duplicate its body, as we could not call 
\xcd`super.count(kind)` in a \xcd`NoThisAccess` method, as is shown by 
the \xcd`ERROR` line \xcd`(A)`. 

Note that \xcd`NoThisAccess` is in \xcd`x10.compiler` and must be imported,
and that the overriding method \xcd`SubIDed.count` must be declared
\xcd`@NoThisAccess` as well as the overridden method.
Line \xcd`(B)` is not allowed because \xcd`code` is a field of \xcd`this`, 
and field accesses are forbidden.   Line \xcd`(C)` references \xcd`this`
directly, which, of course, is forbidden by \xcd`@NoThisAccess`.  


%~~gen ^^^ ObjectInitialization7p2v
% package ObjectInitialization7p2v;
%~~vis
\begin{xten}
import x10.compiler.*;
class UseNoThisAccess {
  static class IDed {
    protected static val counts = [0 as Int,0];
    protected var code : Int;
    val id: Float;
    public def this(kind:Int) { 
      code = kind;
      this.id = this.count(kind); 
    }
    protected static def kind2count(kind:Int) = ++counts(kind % 2);
    @NoThisAccess def count(kind:Int) : Float = kind2count(kind);
  }
  static class SubIDed extends IDed {
    protected static val subcounts = [0 as Int, 0, 0];
    public static val all = new x10.util.ArrayList[SubIDed]();
    public def this(kind:Int) { 
       super(kind); 
    }
    @NoThisAccess
    def count(kind:Int) : Float {
       val subcount <: Int = ++subcounts(kind % 3);
       val supercount <: Float = kind2count(kind);
       //ERROR: val badSuperCount = super.count(kind); //(A)
       //ERROR: code = kind;                           //(B)
       //ERROR: all.add(this);                         //(C)
       return  supercount + 1.0f / subcount;
    }
  }
}
\end{xten}
%~~siv
%
%~~neg


\end{ex}

\subsection{Fine Structure of Constructors}
\label{SFineStructCtors}

The code of a constructor consists of four segments, three of them optional
and one of them implicit.
\begin{enumerate}
\item The first segment is an optional call to \Xcd{this(...)} or
      \Xcd{super(...)}.  If this is supplied, it must be the first statement
      of the constructor.  If it is not supplied, the compiler treats it as a
      nullary super-call \Xcd{super()};
\item If the class or struct has properties, there must be a single
      \Xcd{property(...)} command in the constructor, or a \xcd`this(...)`
      constructor call.  Every execution path
      through the constructor must go through this \Xcd{property(...)} command
      precisely once.   The second segment of the constructor is the code
      following the first segment, up to and including the \Xcd{property()}
      statement.

      If the class or struct has no properties, the \Xcd{property()} call must
      be omitted. If it is present, the second segment is defined as before.  If
      it is absent, the second segment is empty.
\item The third segment is automatically generated.  Fields with initializers
      are initialized immediately after the \Xcd{property} statement.
      In the following example, \Xcd{b} is initialized to \Xcd{y*9000} in
      segment three.  The initialization makes sense and does the right
      thing; \Xcd{b} will be \Xcd{y*9000} for every \Xcd{Overdone} object.
      (This would not be possible if field initializers were processed
      earlier, before properties were set.)
\item The fourth segment is the remainder of the constructor body.
\end{enumerate}

The segments in the following code are shown in the comments.
%~~gen ^^^ ObjectInitialization20
% package ObjectInitialization.ShowingSegments;
%~~vis
\begin{xten}
class Overlord(x:Int) {
  def this(x:Int) { property(x); }
}//Overlord
class Overdone(y:Int) extends Overlord  {
  val a : Int;
  val b =  y * 9000;
  def this(r:Int) {
    super(r);                      // (1)
    x10.io.Console.OUT.println(r); // (2)
    val rp1 = r+1;
    property(rp1);                 // (2)
    // field initializations here  // (3)
    a = r + 2 + b;                 // (4)
  }
  def this() {
    this(10);                      // (1), (2), (3)
    val x = a + b;                 // (4)
  }
}//Overdone
\end{xten}
%~~siv
%
%~~neg

The rules of what is allowed in the three segments are different, though
unsurprising.  For example, properties of the current class can only be read
in segment 3 or 4---naturally, because they are set at the end of segment 2.

\subsubsection{Initialization and Inner Classses}
\index{constructor!inner classes in}

Constructors of inner classes are tantamount to method calls on \Xcd{this}.
For example, the constructor for Inner {\bf is} acceptable.  It does not leak
\Xcd{this}.  It leaks \Xcd{Outer.this}, which is an utterly different object.
So, the call to \Xcd{this.new Inner()} in the \Xcd{Outer} constructor {\em
is} illegal.  It would leak \Xcd{this}.  There is no special rule in effect
preventing this; a constructor call of an inner class is no
different from a method as far as leaking is concerned.
%~~gen ^^^ ObjectInitialization30
% package ObjInit.InnerClass; 
% // NOTEST-packaging-issue
%~~vis
\begin{xten}
class Outer {
  static val leak : Cell[Outer] = new Cell[Outer](null);
  class Inner {
     def this() {Outer.leak.set(Outer.this);}
  }
  def /*Outer*/this() {
     //ERROR: val inner = this.new Inner();
  }
}
\end{xten}
%~~siv
%
%~~neg



\subsubsection{Initialization and Closures}
\index{constructor!closure in}

Closures in constructors may not refer to \xcd`this`.  They may not even refer
to fields of \xcd`this` that have been initialized.   For example, the
closure \xcd`bad1` is not allowed because it refers to \xcd`this`; \xcd`bad2`
is not allowed because it mentions \xcd`a` --- which is, of course, identical
to \xcd`this.a`. 

%%-deleted-%% valid if they were invoked (or inlined) at the
%%-deleted-%%place of creation. For example, \Xcd{closure} below is acceptable because it
%%-deleted-%%only refers to fields defined at the point it was written.  \Xcd{badClosure}
%%-deleted-%%would not be acceptable, because it refers to fields that were not defined at
%%-deleted-%%that point, although they were defined later.
%~~gen ^^^ ObjectInitialization40
% package ObjectInitialization.Closures; 
%~~vis
\begin{xten}
class C {
  val a:Int;
  def this() {
    this.a = 1;
    //ERROR: val bad1 = () => this; 
    //ERROR: val bad2 = () => a*10;
  }
}
\end{xten}
%~~siv
%
%~~neg


\subsection{Definite Initialization in Constructors}


An instance field \Xcd{var x:T}, when \Xcd{T} has a default value, need not be
explicitly initialized.  In this case, \Xcd{x} will be initialized to the
default value of type \Xcd{T}.  For example, a \Xcd{Score} object will have
its \Xcd{currently} field initialized to zero, below:
%~~gen ^^^ ObjectInitialization50
% package ObjectInit.DefaultInit;
%~~vis
\begin{xten}
class Score {
  public var currently : Int;
}
\end{xten}
%~~siv
%
%~~neg

All other sorts of instance fields do need to be initialized before they can
be used.  \Xcd{val} fields must be initialized in the constructor, even if
their type has a 
default value.  It would be silly to have a field \Xcd{val z : Int} that was
always given default value of \Xcd{0} and, since it is \Xcd{val}, can never be
changed.  \Xcd{var} fields whose type has no default value must be initialized
as well, such as \xcd`var y : Int{y != 0}`, since it cannot be assigned a
sensible initial value.

The fundamental principles are:
\begin{enumerate}
\item \Xcd{val} fields must be assigned precisely once in each constructor on every
possible execution path.
\item \Xcd{var} fields of defaultless type must be
assigned at least once on every possible execution path, but may be assigned
more than once.
\item No variable may be read before it is guaranteed to have been
assigned.
\item Initialization may be by field initialization expressions (\Xcd{val x :
      Int = y+z}), or by uninitialized fields \Xcd{val x : Int;} plus
an initializing assignment \Xcd{x = y+z}.  Recall that field initialization
expressions are performed after the \Xcd{property} statement, in segment 3 in
the terminology of \Sref{SFineStructCtors}.
\end{enumerate}



\subsection{Summary of Restrictions on Classes and Constructors}

The following table tells whether a given feature is (yes), is not (no) or is
with some conditions (note) allowed in a given context.   For example, a
property method is allowed with the type of another property, as long as it
only mentions the preceding properties. The first column of the table gives
examples, by line of the following code body.

\begin{tabular}{||l|l|c|c|c|c|c|c||}
\hline
~
  & {\bf Example}
  & {\bf Prop.}
  & {\bf {\tt \small self==this}(1)}
  & {\bf Prop.Meth.}
  & {\bf {\tt this}}
  & {\bf {fields}}
\\\hline
Type of property
  & (A)
  & %?properties
    yes (2)
  & no %? self==this
  & no %? property methods
  & no %? this may be used
  & no %? fields may be used
\\\hline
Class Invariant
  & (B)
  & yes %?properties
  & yes %? self==this
  & yes %? property methods
  & yes %? this may be used
  & no %? fields may be used
\\\hline
Supertype (3)
  & (C), (D)
  & yes%?properties
  & yes%? self==this
  & yes%? property methods
  & no%? this may be used
  & no%? fields may be used
\\\hline
Property Method Body
  & (E)
  & yes %?properties
  & yes %? self==this
  & yes %? property methods
  & yes %? this may be used
  & no %? fields may be used
\\\hline

Static field (4)
  & (F) (G)
  & no %?properties
  & no %? self==this
  & no %? property methods
  & no %? this may be used
  & no %? fields may be used
\\\hline

Instance field (5)
  & (H), (I)
  & yes %?properties
  & yes %? self==this
  & yes %? property methods
  & yes %? this may be used
  & yes %? fields may be used
\\\hline

Constructor arg. type
  & (J)
  & no %?properties
  & no %? self==this
  & no  %? property methods
  & no %? this may be used
  & no %? fields may be used
\\\hline

Constructor guard
  & (K)
  & no %?properties
  & no %? self==this
  & no %? property methods
  & no %? this may be used
  & no %? fields may be used
\\\hline

Constructor ret. type
  & (L)
  & yes %?properties
  & yes %? self==this
  & yes %? property methods
  & yes %? this may be used
  & yes %? fields may be used
\\\hline

Constructor segment 1
  & (M)
  & no%?properties
  & yes%? self==this
  & no%? property methods
  & no%? this may be used
  & no%? fields may be used
\\\hline


Constructor segment 2
  & (N)
  & no%?properties
  & yes%? self==this
  & no%? property methods
  & no%? this may be used
  & no%? fields may be used
\\\hline

Constructor segment 4
  & (O)
  & yes%?properties
  & yes%? self==this
  & yes%? property methods
  & yes%? this may be used
  & yes%? fields may be used
\\\hline

Methods
  & (P)
  & yes %?properties
  & yes %? self==this
  & yes %? property methods
  & yes %? this may be used
  & yes %? fields may be used
\\\hline



\iffalse
place
  & (pos)
  & %?properties
  & %? self==this
  & %? property methods
  & %? this may be used
  & %? fields may be used
\\\hline
\fi
\end{tabular}

Details:

\begin{itemize}
\item (1) {Top-level {\tt self} only.}
\item (2) {The type of the {$i^{th}$} property may only mention
                 properties {$1$} through {$i$}.}
\item (3) Super-interfaces follow the same rules as supertypes.
\item (4) The same rules apply to types and initializers.
\end{itemize}



The example indices refer to the following code:
%~~gen ^^^ ObjectInitialization60
% package ObjectInit.GrandExample;
% class Supertype[T]{}
% interface SuperInterface[T]{}
%~~vis
\begin{xten}
class Example (
   prop : Int,
   proq : Int{prop != proq},                    // (A)
   pror : Int
   )
   {prop != 0}                                  // (B)
   extends Supertype[Int{self != prop}]         // (C)
   implements SuperInterface[Int{self != prop}] // (D)
{
   property def propmeth() = (prop == pror);    // (E)
   static staticField
      : Cell[Int{self != 0}]                    // (F)
      = new Cell[Int{self != 0}](1);            // (G)
   var instanceField
      : Int {self != prop}                      // (H)
      = (prop + 1) as Int{self != prop};        // (I)
   def this(
      a : Int{a != 0},
      b : Int{b != a}                           // (J)
      )
      {a != b}                                  // (K)
      : Example{self.prop == a && self.proq==b} // (L)
   {
      super();                                  // (M)
      property(a,b,a);                          // (N)
      // fields initialized here
      instanceField = b as Int{self != prop};   // (O)
   }

   def someMethod() =
        prop + staticField() + instanceField;   // (P)
}
\end{xten}
%~~siv
%
%~~neg

\section{Method Resolution}
\index{method!resolution}
\index{method!which one will get called}
\label{sect:MethodResolution}

Method resolution is the problem of determining, statically, which method (or
constructor or operator)
should be invoked, when there are several choices that could be invoked.  For
example, the following class has two overloaded \xcd`zap` methods, one taking
an \Xcd{Any}, and the other a \Xcd{Resolve}.  Method resolution will figure
out that the call \Xcd{zap(1..4)} should call \xcd`zap(Any)`, and
\Xcd{zap(new Resolve())} should call \xcd`zap(Resolve)`.  

\begin{ex}
%~~gen ^^^ MethodResolution10
%package MethodResolution.yousayyouwantaresolution;
% // This depends on https://jira.codehaus.org/browse/XTENLANG-2696
%~~vis
\begin{xten}
class Res {
  public static interface Surface {}
  public static interface Deface {}

  public static class Ace implements Surface {
    public static operator (Boolean) : Ace = new Ace();
    public static operator (Place) : Ace = new Ace();
  }
  public static class Face implements Surface, Deface{}

  public static class A {}
  public static class B extends A {}
  public static class C extends B {}

  def m(x:A) = 0;
  def m(x:Int) = 1;
  def m(x:Boolean) = 2;
  def m(x:Surface) = 3;
  def m(x:Deface) = 4; 

  def example() {
     assert m(100) == 1 : "Int"; 
     assert m(new C()) == 0 : "C";
     // An Ace is a Surface, unambiguous best choice
     assert m(new Ace()) == 3 : "Ace";
     // ERROR: m(new Face());

     // The match must be exact.
     // ERROR: assert m(here) == 3 : "Place";

     // Boolean could be handled directly, or by 
     // implicit coercion Boolean -> Ace.
     // Direct matches always win.
     assert m(true) == 2 : "Boolean"; 
  }
\end{xten}
%~~siv
%  public static def main(argv:Rail[String]) {(new Res()).example(); Console.OUT.println("That's all!");}
% public def claim() { val ace : Ace = here; assert m(ace)==3; }
% }
% class Hook{ def run(){ (new Res()).example(); return true;} }
%~~neg

In the \xcd`"Int"` line, there is a very close match.  \xcd`100` is an
\xcd`Int`.  In fact, \xcd`100` is an \xcd`Int{self==100}`, so even in this
case the type of the actual parameter is not {\em precisely} equal to the type
of the method.

In the \xcd`"C"` line of the example, \xcd`new C()` is an instance of \xcd`C`,
which is a subtype of \xcd`A`, so the \xcd`A` method applies.  No other method
does, and so the \xcd`A` method will be invoked.

Similarly, in the \xcd`"Ace"` line, the \xcd`Ace` class implements
\xcd`Surface`, and so \xcd`new Ace()` matches the \xcd`Surface` method. 

However, a \xcd`Face` is both a \xcd`Surface` and a \xcd`Deface`, so there is
no unique best match for the invocation \xcd`m(new Face())`.  This invocation
would be forbidden, and a compile-time error issued.


The match must be exact.  There is an implicit coercion 
from \xcd`Place` to \xcd`Ace`, and \xcd`Ace` implements \xcd`Surface`, so the
code
\begin{xten}
val ace : Ace = here;
assert m(ace) == 3;
\end{xten}
works, by using the \xcd`Surface` form of \xcd`m`.  But doing it in one step
requires a deeper search than X10 performs\footnote{In general this search is
unbounded, so X10 can't perform it.}, and is not allowed.


For \xcd`m(true)`, both the \xcd`Boolean` and, with the implicit coercion,
\xcd`Ace` methods could apply.  Since the \xcd`Boolean` method applies
directly, and the \xcd`Ace` method requires an implicit coercion, this call
resolves to the \xcd`Boolean` method, without an error.

\end{ex}


The basic concept of method resolution is:
\begin{enumerate}
\item List all the methods that could possibly be used, inferring generic
      types but not performing implicit coercions.    
\item If one possible method is more specific than all the others, that one 
      is the desired method.
\item If there are two or more methods neither of which is more specific than
      the others, then the method invocation is ambiguous.  Method resolution
      fails and reports an error.
\item Otherwise, no possible methods were found without implicit coercions.
      Try the preceding steps again, but with coercions allowed: zero or one
      implicit coercion for each argument.  If a single
      most specific method is found with coercions, it is the desired method.
      If there are several, the invocation is ambiguous and erronious.
\item If no methods were found even with coercions, then the method invocation
      is undetermined.  Method resolution fails and reports an error.
\end{enumerate}

After method resolution is done, there is a validation phase that checks the
legality of the call, based on the \xcd`STATIC_CHECKS` compiler flag.  
With \xcd`STATIC_CHECKS`, the method's constraints must be satisfied; that is,
they must be entailed (\Sref{SemanticsOfConstraints}) by the information in
force at the point of the call.  With \xcd`DYNAMIC_CHECKS`, if the constraint
is not entailed at that point, a dynamic check is inserted to make sure that
it is true at runtime.

\noindent
In the presence of X10's highly-detailed type system, some subtleties arise. 
One point, at least, is {\em not} subtle. The same procedure is used, {\em
mutatis mutandis} for method, constructor, and operator resolution.  



\subsection{Space of Methods}

X10 allows some constructs, particularly \xcd`operator`s, to be defined in a
number of ways, and invoked in a number of ways. This section specifies which
forms of definition could correspond to a given definiendum.
%%OP%% , and (redundantly)
%%OP%% the syntax for invoking that definition unambiguously.  

Method invocations \xcd`a.m(b)`, where \xcd`a` is an expression, can be either
of the following forms.  There may be any number of arguments.
\begin{itemize}
\item An instance method on \xcd`a`, of the form \xcd`def m(B)`.
%%OP%% , so that the   invocation is \xcd`a.m(b)`;
\item A static method on \xcd`a`'s class, of the form \xcd`static def m(B)`.
%%OP%%       so that the invocation is \xcd`A.m(b)`.
\end{itemize}

The meaning of an invocation of the form \xcd`m(b)`, with any number of
arguments, depends slightly on its context.  Inside of a constraint, it might
mean \xcd`self.m(b)`.  Outside of a constraint, there is no \xcd`self`
defined, so it can't mean that.  The first of these that applies will be
chosen. 
\begin{enumerate}
\item Invoke a method on \xcd`this`, \viz{} \xcd`this.m(b)`.  Inside a
      constraint, it may also invoke a property method on \xcd`self`, \viz.
      \xcd`self.m(b)`.  It is an error if both \xcd`this.m(b)` and
      \xcd`self.m(b)` are possible.
\item Invoke a function named \xcd`m` in a local or field.
\item Construct a structure named \xcd`m`.
\end{enumerate}

Static method invocations, \xcd`A.m(b)`, where \xcd`A` is a container name,
can only be static.  There may be any number of arguments.
\begin{itemize}
\item A static method on \xcd`A`, of the form \xcd`static def m(B)`.
%%OP%%       the invocation is \xcd`A.m(b)`; 
\end{itemize}


Constructor invocations, \xcd`new A(b)`, must invoke constructors. There may
be any number of arguments. 
\begin{itemize}
\item A constructor on \xcd`A`, of the form \xcd`def this(B)`.
%%OP%% , so that the
%%OP%%       invocation is \xcd`new A(b)`.
\end{itemize}


A unary operator \xcdmath"$\star$ a" may be defined as: 
\begin{itemize}
\item An instance operator on \xcd`A`, defined as 
      \xcdmath"operator $\star$ this()".
%%OP%%       so that the invocation is 
%%OP%%       \xcdmath"a.operator $\star$()"; or
\item A static operator on \xcd`A`, defined as 
      \xcdmath"operator $\star$(a:A)".
%%OP%%       so that the invocation is 
%%OP%%       \xcdmath"A.operator $\star$(a)"
\end{itemize}

A binary operator \xcdmath"a $\star$ b" may be defined as: 
\begin{itemize}
\item An instance operator on \xcd`A`, defined as 
      \xcdmath"operator this $\star$(b:B)";
%%OP%%       so that the invocation is \xcdmath"a.operator $\star$(b)", 
or
\item A right-hand operator on \xcd`B`, defined as
      \xcdmath"operator (a:A) $\star$ this"; or
%%OP%%       so that the invocation is \xcdmath"b.operator ()$\star$(b)"

\item A static operator on \xcd`A`, defined as
      \xcdmath"operator (a:A) $\star$ (b:B)", 
%%OP%%       so that the invocation is \xcdmath"A.operator $\star$(a,b)"
; or
\item A static operator on \xcd`B`, if \xcd`A` and \xcd`B` are different
      classes, defined as
      \xcdmath"operator (a:A) $\star$ (b:B)"
%%OP%% , so that the invocation is 
%%OP%%       \xcdmath"B.operator $\star$(a,b)".
\end{itemize}
\noindent
If none of those resolve to a method, then either operand may be implicitly
coerced to the
other.  If one of the following two situations obtains, it will be done; if
both, the expression causes a static error.
\begin{itemize}
\item An implicit coercion from \xcd`A` to \xcd`B`, and 
      an operator \xcdmath"B $\star$ B" can be used, by 
      coercing \xcd`a` to be of type \xcd`B`, and then using \xcd`B`'s
      $\star$.  
\item An implicit coercion from \xcd`B` to \xcd`A`, and 
      an operator \xcdmath"A $\star$ A" can be used,
      coercing \xcd`b` to be of type \xcd`A`, and then using \xcd`A`'s
      $\star$.  
\end{itemize}

An application \xcd`a(b)`, for any number of arguments, can come from a number
of things. 
\begin{itemize}
\item an application operator on \xcd`a`, defined as \xcd`operator this(b:B)`;
%%OP%% , so that the 
%%OP%% invocation is \xcd`a.operator()(b)`
\item If \xcd`a` is an identifier, \xcd`a(b)` can also be a method invocation
      equivalent to \xcd`this.a(b)`, which  invokes \xcd`a` as
      either an instance or static method on \xcd`this`
\item If \xcd`a` is a qualified identifier, \xcd`a(b)` can also be an
      invocation of a struct constructor.
\end{itemize}


An indexed assignment, \xcd`a(b)=c`, for any number of \xcd`b`'s, can only
come from an indexed assignment definition: 
\begin{itemize}
\item \xcd`operator this(b:B)=(c:C) {...}`
%%OP%%       so that the invocation is \xcd`a.operator()=(b,c)`.
\end{itemize}

An implicit coercion, in 
which a value \xcd`a:A` is used in a context which requires a value of some
other non-subtype \xcd`B`, 
can only come from implicit coercion operation defined on
\xcd`B`: 
\begin{itemize}
\item an implicit coercion in \xcd`B`:
      \xcd`static operator (a:A):B`;
%%OP%%       so that the coercion is \xcd`B.operator[B](a)`;
\end{itemize}

An explicit conversion \xcd`a as B` can come from an explicit conversion
operator, or an implicit coercion operator.  X10 tries two things, in order,
only checking 2 if 1 fails: 
\begin{enumerate}
\item An \xcd`as` operator in \xcd`B`: 
      \xcdmath"static operator (a:A) as ?";
%%OP%%       so that the conversion is \xcd`B.operator as[B](a)`

\item or, failing that, an implicit coercion in \xcd`B`:
      \xcd`static operator (a:A):B`.
%%OP%% , so that the conversion is 
%%OP%%       \xcd`B.operator[B](a)`;

\end{enumerate}



\subsection{Possible Methods}

This section describes what it means for a method to be a {\em possible}
resolution of a method invocation.  



Generics introduce several subtleties, especially with the inference of
generic types. 
For the purposes of method resolution, all that matters about a method,
constructor, or operator \xcd`M` --- we use the word ``method'' to include all
three choices for this section --- is its signature, plus which method it is.
So, a typical \xcd`M` might look like 
\xcdmath"def m[G$_1$,$\ldots$, G$_g$](x$_1$:T$_1$,$\ldots$, x$_f$:T$_f$){c} =...".  The code body \xcd`...` is irrelevant for the purpose of whether a
given method call means \xcd`M` or not, so we ignore it for this section.

All that matters about a method definition, for the purposes of method
resolution, is: 
\begin{enumerate}
\item The method name \xcd`m`;
\item The generic type parameters of the method \xcd`m`,  \xcdmath"G$_1$,$\ldots$, G$_g$".  If there
      are no generic type parameters, {$g=0$}.  
\item The types \xcdmath"x$_1$:T$_1$,$\ldots$, x$_f$:T$_f$" of the formal parameters.  If
      there are no formal parameters, {$f=0$}. In the case of an instance
      method, the receiver will be the first formal parameter.\footnote{The
      variable names are relevant because one formal can be mentioned in a
      later type, or even a constraint: {\tt def f(a:Int, b:Point\{rank==a\})=...}.}
\item A {\em unique identifier} \xcd`id`, sufficient to tell the compiler
      which method body is intended.  A file name and position in that file
      would suffice.  The details of the identifier are not relevant.
\end{enumerate}

For the purposes of understanding method resolution, we assume that all the
actual parameters of an invocation are simply variables: \xcd`x1.meth(x2,x3)`.
This is done routinely by the compiler in any case; the code 
\xcd`tbl(i).meth(true, a+1)` would be treated roughly as 
\begin{xten}
val x1 = tbl(i);
val x2 = true;
val x3 = a+1;
x1.meth(x2,x3);
\end{xten}

All that matters about an invocation \xcd`I` is: 
\begin{enumerate}
\item The method name \xcdmath"m$'$";
\item The generic type parameters \xcdmath"G$'_1$,$\ldots$, G$'_g$".  If there
      are no generic type parameters, {$g=0$}.  
\item The names and types \xcdmath"x$_1$:T$'_1$,$\ldots$, x$_f$:T$'_f$" of the
      actual parameters.
      If
      there are no actual parameters, {$f=0$}. In the case of an instance
      method, the receiver is the first actual parameter.
\end{enumerate}

The signature of the method resolution procedure is: 
\xcd`resolve(invo : Invocation, context: Set[Method]) : MethodID`.  
Given a particular invocation and the set \xcd`context` of all methods
which could be called at that point of code, method resolution either returns
the unique identifier of the method that should be called, or (conceptually)
throws an exception if the call cannot be resolved.

The procedure for computing \xcd`resolve(invo, context)` is: 
\begin{enumerate}
\item Eliminate from \xcd`context` those methods which are not {\em
      acceptable}; \viz, those whose name, type parameters, and formal parameters
      do not suitably match \xcd`invo`.  In more detail:
      \begin{itemize}
      \item The method name \xcd`m` must simply equal the invocation name \xcdmath"m$'$";
      \item X10 infers type parameters, by an algorithm given in \Sref{TypeParamInfer}.
      \item The method's type parameters are bound to the invocation's for the
            remainder of the acceptability test.
      \item The actual parameter types must be subtypes of the formal
            parameter types, or be coercible to such subtypes.  Parameter $i$
            is a subtype if \xcdmath"T$'_i$ <: T$_i$".  It is implicitly
            coercible to a subtype if either it is a subtype, or if there is
            an implicit coercion operator 
            defined from \xcdmath"T$'_i$" to some type \xcd`U`, and 
            \xcdmath"U <: T$_i$". \index{method resolution!implicit coercions
            and} \index{implicit coercion}\index{coercion}.  If coercions are
            used to resolve the method, they will be called on the arguments
            before the method is invoked.
            
      \end{itemize}
\item Eliminate from \xcd`context` those methods which are not {\em
      available}; \viz, those which cannot be called due to visibility
      constraints, such as methods from other classes marked \xcd`private`.
      The remaining methods are both acceptable and available; they might be
      the one that is intended.
\item If the method invocation is a \xcd`super` invocation appearing in class
      \xcd`Cl`, methods of \xcd`Cl` and its subclasses are considered
      unavailable as well.
      
\item From the remaining methods, find the unique \xcd`ms` which is more specific than all the
      others, \viz, for which \xcd`specific(ms,mo) = true` for all other
      methods \xcd`mo`.
      The specificity test \xcd`specific` is given next.
      \begin{itemize}
      \item If there is a unique such \xcd`ms`, then
            \xcd`resolve(invo,context)` returns the \xcd`id` of \xcd`ms`.  
      \item If there is not a unique such \xcd`ms`, then \xcd`resolve` reports
            an error.
      \end{itemize}

\end{enumerate}

The subsidiary procedure \xcd`specific(m1, m2)` determines whether method
\xcd`m1` is equally or more specific than \xcd`m2`.  \xcd`specific` is not a
total order: is is possible for each one to be considered more specific than
the other, or either to be more specific.  \xcd`specific` is computed as: 
\begin{enumerate}
\item Construct an invocation \xcd`invo1` based on \xcd`m1`: 
      \begin{itemize}
      \item \xcd`invo1`'s method name is \xcd`m1`'s method name;
      \item \xcd`invo1`'s generic parameters are those of \xcd`m1`--- simply
            some type variables.
      \item \xcd`invo1`'s parameters are those of \xcd`m1`.
      \end{itemize}
\item If \xcd`m2` is acceptable for the invocation \xcd`invo1`,
      \xcd`specific(m1,m2)` returns true; 
\item Construct an invocation \xcd`invo2p`, which is \xcd`invo1` with the
      generic parameters erased.  Let \xcd`invo2` be \xcd`invo2p` with generic
      parameters as inferred by X10's type inference algorithm.  If type
      inference fails, \xcd`specific(m1,m2)` returns false.
\item If \xcd`m2` is acceptable for the invocation \xcd`invo2`,
      \xcd`specific(m1,m2)` returns true; 
\item Otherwise, \xcd`specific(m1,m2)` returns false.
\end{enumerate}

\subsection{Field Resolution}

An identifier \xcd`p` can refer to a number of things.  The rules are somewhat
different inside and outside of a constraint.

Outside of a constraint, the compiler chooses
the first one from the following list which applies: 
\begin{enumerate}
\item A local variable named \xcd`p`.
\item A field of \xcd`this`, \viz{} \xcd`this.p`.
\item A nullary property method, \xcd`this.p()`
\item A member type named \xcd`p`.
\item A package named \xcd`p`.
\end{enumerate}

Inside of a constraint, the rules are slightly different, because \xcd`self`
is available, and packages cannot be used per se.
\begin{enumerate}
\item A local variable named \xcd`p`.
\item A property of \xcd`this` or of \xcd`self`, \viz{} \xcd`this.p` or
      \xcd`self.p`.  If both are available, report an error.
\item A nullary property method, \xcd`this.p()`
\item A member type named \xcd`p`.
\end{enumerate}

\subsection{Other Disambiguations}
\label{sect:disambiguations}

It is possible to have a field of the same name as a method.
Indeed, it is a common pattern to have private field and a public
method of the same name to access it:
\begin{ex}
%~~gen ^^^ MethodResolution_disamb_a
%package MethodResolution_disamb_a;
%~~vis
\begin{xten}
class Xhaver {
  private var x: Int = 0;
  public def x() = x;
  public def bumpX() { x ++; }
}
\end{xten}
%~~siv
%
%~~neg
\end{ex}

\begin{ex}
However, this can lead to syntactic ambiguity in the case where the field
\Xcd{f} of object \xcd`a` is a
function, array, list, or the like, and where \xcd`a` has a method also named
\xcd`f`.  The term \Xcd{a.f(b)} could either mean ``call method \xcd`f` of \xcd`a` upon
\xcd`b`'', or ``apply the function \xcd`a.f` to argument \xcd`b`''.  

%~~gen  ^^^ MethodResolution_disamb_b
%package MethodResolution_disamb_b;
%NOCOMPILE
%~~vis
\begin{xten}
class Ambig {
  public val f : (Int)=>Int =  (x:Int) => x*x;
  public def f(y:int) = y+1;
  public def example() {
      val v = this.f(10);
      // is v 100, or 11?
  }
}
\end{xten}
%~~siv
%
%~~neg
\end{ex}

In the case where a syntactic form \xcdmath"E.m(F$_1$, $\ldots$, F$_n$)" could
be resolved as either a method call, or the application of a field \xcd`E.m`
to some arguments, it will be treated as a method call.  
The application of \xcd`E.m` to some arguments can be specified by adding
parentheses:  \xcdmath"(E.m)(F$_1$, $\ldots$, F$_n$)".

\begin{ex}

%~~gen ^^^ MethodResolution_disamb_c
%package MethodResolution_disamb_c;
%NOCOMPILE
%~~vis
\begin{xten}
class Disambig {
  public val f : (Int)=>Int =  (x:Int) => x*x;
  public def f(y:int) = y+1;
  public def example() {
      assert(  this.f(10)  == 11  );
      assert( (this.f)(10) == 100 );
  }
}
\end{xten}
%~~siv
%
%~~neg

\end{ex}

Similarly, it is possible to have a method with the same name as a struct, say
\xcd`ambig`, giving an ambiguity as to whether \xcd`ambig()` is a struct
constructor invocation or a method invocation.  This ambiguity is resolved by
treating it as a method invocation.  If the constructor invocation is desired,
it can be achieved by including the optional \xcd`new`.  That is, 
\xcd`new ambig()` is struct constructor invocation; \xcd`ambig()` is a 
method invocation.

\section{Static Nested Classes}
\label{StaticNestedClasses}
\index{class!static nested}
\index{class!nested}
\index{static nested class}

One class (or struct or interface) may be nested within another.  The simplest
way to do this is as a \xcd`static` nested class, written by putting one class
definition at top level inside another, with the inner one having a
\xcd`static` modifier.  
For most purposes, a static nested class behaves like a top-level class.
However, a static nested class has access to private static
fields and methods of its containing class.  

Nested interfaces and static structs are permitted as well.

%~~gen ^^^ InnerClasses10
% package Classes.StaticNested; 
% NOTEST
%~~vis
\begin{xten}
class Outer {
  private static val priv = 1;
  private static def special(n:Int) = n*n;
  public static class StaticNested {
     static def reveal(n:Int) = special(n) + priv;
  }
}
\end{xten}
%~~siv
%
%~~neg

\section{Inner Classes}
\label{InnerClasses}
\index{class!inner}
\index{inner class}


Non-static nested classes are called {\em inner classes}. An inner class
instance can be thought of as a very elaborate member of an object --- one
with a full class structure of its own.   The crucial characteristic of an
inner class instance is that it has an implicit reference to an instance of
its containing class.  

\begin{ex}
This feature is particularly useful when an instance of the inner class makes
no sense without reference to an instance of the outer, and is closely tied to
it.  For example, consider a range class, describing a span of integers {$m$}
to {$n$}, and an iterator over the range.  The iterator might as well have
access to the range object, and there is little point to discussing
iterators-over-ranges without discussing ranges as well.
In the following example, the inner class \xcd`RangeIter` iterates over the
enclosing \xcd`Range`.  

It has its own private cursor field \xcd`n`, telling
where it is in the iteration; different iterations over the same \xcd`Range`
can exist, and will each have their own cursor.
It is perhaps unwise to use the name \xcd`n` for a field of the inner class,
since it is also a field of the outer class, but it is legal.  (It can happen
by accident as well -- \eg, if a programmer were to add a field \xcd`n` to a
superclass of the  outer class, the inner class would still work.)
It does not even
interfere with the inner class's ability to refer to the outer class's \xcd`n`
field: the cursor initialization 
refers to the \xcd`Range`'s lower bound through a fully qualified name
\xcd`Range.this.n`.
The initialization of its \xcd`n` field refers to the outer class's \xcd`m` field, which is
not shadowed and can be referred to directly, as \xcd`m`.


%~~gen ^^^ InnerClasses20
% package Classes.InnerClasses_a; 
% NOTEST
%~~vis
\begin{xten}
class Range(m:Int, n:Int) implements Iterable[Int]{
  public def iterator ()  = new RangeIter();
  private class RangeIter implements Iterator[Int] {
     private var n : Int = m;
     public def hasNext() = n <= Range.this.n;
     public def next() = n++;
  }
  public static def main(argv:Rail[String]) {
    val r = new Range(3,5);
    for(i in r) Console.OUT.println("i=" + i);
  }
}
\end{xten}
%~~siv
%
%~~neg
\end{ex}

An inner class has full access to the members of its enclosing class, both
static and instance.  In particular, it can access \xcd`private` information,
just as methods of the enclosing class can.  

An inner class can have its own members.  
Inside instance methods of an inner class, \xcd`this` refers to the instance
of the {\em inner} class.  The instance of the outer class can be accessed as
{\em Outer}\xcd`.this` (where {\em Outer} is the name of the outer class).
If, for some dire reason, it is necessary to have an inner class within an inner
class, the innermost class can refer to the \xcd`this` of either outer class
by using its name.

An inner class can inherit from any class in scope,
with no special restrictions. \xcd`super` inside an inner class refers to the
inner class's superclass. If it is necessary to refer to the outer classes's
superclass, use a qualified name of the form {\em Outer}\xcd`.super`.

The members of inner classes must be instance members.  They cannot be static
members.  Classes, interfaces, static methods, static fields, and typedefs are
not allowed as members of inner classes. 
The same restriction applies to local classes (\Sref{sect:LocalClasses}).

\index{inner class!extending}
Consider
an inner class \xcd`IC1` of some outer class \xcd`OC1`, being extended by 
another class \xcd`IC2`. However, since an \xcd`IC1` only exists as a
dependent of an \xcd`OC1`, each \xcd`IC2` must be associated with an \xcd`OC1`
--- or a subtype thereof --- as well.   So, \xcd`IC2` must be an inner class
of either \xcd`OC1` or some subclass \xcd`OC2 <: OC1`.

\begin{ex}For example, one often extends an
inner class when one extends its outer class: 
%~~gen ^^^ InnerClasses30
% package Classes.Innerclasses.Are.For.Innermasses;
%~~vis
\begin{xten}
class OC1 {
   class IC1 {}
}
class OC2 extends OC1 {
   class IC2 extends IC1 {} 
}
\end{xten}
%~~siv
%
%~~neg
\end{ex}


The hiding of method names has one fine point.  If an inner class defines a
method named \xcd`doit`, then {\em all} methods named \xcd`doit` from the
outer class are hidden --- even if they have different argument types than the
one defined in the inner class.
They are still accessible via
\xcd`Outer.this.doit()`, but not simply via \xcd`doit()`.  The following code
is correct, but would not be correct if the ERROR line were uncommented.

%~~gen ^^^ InnerClasses40
% package Classes.Innerclasses.StupidOverloading; 
% NOTEST
%~~vis
\begin{xten}
class Outer {
  def doit() {}
  def doit(String) {}
  class Inner { 
     def doit(Boolean, Outer) {}
     def example() {
        doit(true, Outer.this);
        Outer.this.doit();
        //ERROR: doit("fails");
     }
  }
}
\end{xten}
%~~siv
%
%~~neg


\subsection{Constructors and Inner Classes}
\label{sect:InnerClassCtor}
\index{inner class!constructor}

If \xcd`IC` is an inner class of \xcd`OC`, then instance code in the body of
\xcd`OC` can create instances of \xcd`IC` simply by calling a constructor
\xcd`new IC(...)`: 
%~~gen ^^^ InnerClasses50
% package Classes.Innerclasses.Constructors.Easy;
%~~vis
\begin{xten}
class OC {
  class IC {}
  def method(){
    val ic = new IC();
  }
}
\end{xten}
%~~siv
%
%~~neg

Instances of \xcd`IC` can be constructed from elsewhere as well.  Since every
instance of \xcd`IC` is associated with an instance of \xcd`OC`, an \xcd`OC`
must be supplied to the \xcd`IC` constructor.  The syntax for doing so is: 
\xcd`oc.new IC()`.  For example: 
%~~gen ^^^ InnerClasses60
% package Classes.Inner_a; 
% NOTEST
% /*NONSTATIC*/
%~~vis
\begin{xten}
class OC {
  class IC {}
  static val oc1 = new OC();
  static val oc2 = new OC();
  static val ic1 = oc1.new IC();
  static val ic2 = oc2.new IC();
}
class Elsewhere{
  def method(oc : OC) {
    val ic = oc.new IC();
  }
}
\end{xten}
%~~siv
%
%~~neg


\section{Local Classes}
\label{sect:LocalClasses}

Classes can be defined and instantiated in the middle of methods and other
code blocks.
A local class in a static method is a static class; a local class in an
instance method is an inner class.
 Local classes are local to the block in which they are defined.
They have access to almost everything defined at that point in the method; the
one exception is that they cannot use \xcd`var` variables. Local classes
cannot be \xcd`public`, \xcd`protected`, or \xcd`private`, because they are
only visible from within the block of declaration. They cannot be
\xcd`static`.

\begin{ex}
The following example illustrates the use of a local class \xcd`Local`, 
defined inside the body of method \xcd`m()`. 
%~~gen ^^^ InnerClasses5p9v
% package InnerClasses5p9v;
% NOTEST
%~~vis
\begin{xten}
class Outer {
  val a = 1;
  def m() {
    val a = -2; 
    val b = 2;
    class Local {
      val a = 3;
      def m() = 100*Outer.this.a + 10*b + a; 
    }
    val l : Local = new Local();
    assert l.m() == 123;
  }//end of m()
}
\end{xten}
%~~siv
% class Hook{ def run() {
%   val o <: Outer = new Outer();
%   o.m();
%   return true;
% } }
%~~neg
Note that the middle \xcd`a`,
whose value is \xcd`-2`, is not accessible inside of \xcd`Local`; it is
shadowed by \xcd`Local`'s \xcd`a` field.  \xcd`Outer`'s \xcd`a` is also
shadowed, but the notation \xcd`Outer.this` gives a reference to the enclosing
\xcd`Outer` object.  There is no corresponding notation to access shadowed local
variables from the enclosing block; if you need to get them, rename the fields
of \xcd`Local`.    
\end{ex}


The members of inner classes must be instance members.  They cannot be static
members.  Classes, interfaces, static methods, static fields, and typedefs are
not allowed as members of local classes. 
The same restriction applies to inner classes (\Sref{InnerClasses}). 





\section{Anonymous Classes}
\index{class!anonymous}
\index{anonymous class}

It is possible to define a new local class and instantiate it as part of an
expression.  The new class can extend an existing class or interface.  Its body
can include all of the usual members of a local class. It can refer to any
identifiers available at that point in the expression --- except for \xcd`var`
variables.  An anonymous class in a static context is a static inner class.

Anonymous classes are useful when you want to package several pieces of
behavior together (a single piece of behavior can often be expressed as a
function, which is syntactically lighter-weight), or if you want to extend and
vary an extant class without going through the trouble of actually defining a
whole new class.

The syntax for an anonymous class is a constructor call followed immediately
by a braced class body: \xcd`new C(1){def foo()=2;}`.

\begin{ex}In the following minimalist example, the abstract class \xcd`Choice`
encapsulates a decision.   A \xcd`Choice` has a \xcd`yes()` and a \xcd`no()`
method.  The \xcd`choose(b)` method will invoke one of the two.  \xcd`Choice`s
also have names.

The \xcd`main()` method creates a specific \xcd`Choice`.  \xcd`c` is not a
immediate instance of \xcd`Choice` --- as an abstract class, \xcd`Choice` has
no immediate instances. \xcd`c` is an instance of an anonymous class which
inherits from \xcd`Choice`, but supplies \xcd`yes()` and \xcd`no()` methods.
These methods modify the contents of the \xcd`Cell[Int]` \xcd`n`.  (Note that,
as \xcd`n` is a local variable, it would take a few lines more coding to
extract \xcd`c`'s class, name it, and make it an inner class.)  The call to
\xcd`c.choose(true)`  will call \xcd`c.yes()`, incrementing \xcd`n()`, in a
rather roundabout manner.

%~~gen ^^^ InnerClasses70
% package ClassInnnerclassAnonclassOw; 
%~~vis
\begin{xten}
abstract class Choice(name: String) {
  def this(name:String) {property(name);}
  def choose(b:Boolean) { 
     if (b) this.yes(); else this.no(); }
  abstract def yes():void;
  abstract def no():void;
}

class Example {
  static def main(Rail[String]) {
    val n = new Cell[Int](0);
    val c = new Choice("Inc Or Dec") {
      def yes() { n() += 1; }
      def no()  { n() -= 1; }
      };
    c.choose(true);
    Console.OUT.println("n=" + n());
  }
}

\end{xten}
%~~siv
%
%~~neg
\end{ex}

Anonymous classes have many of the features of classes in general.  A few
features are unavailable because they don't make sense.

\begin{itemize}

\item Anonymous classes don't have constructors.  Since they don't have names,
      there's no way a constructor could get called in the ordinary way.
      Instead, the \xcd`new C(...)` expression must match a constructor of the
      parent class \xcd`C`, which will be called to initialize the
      newly-created object of the anonymous class.

\item The \xcd`public`,
      \xcd`private`, and \xcd`protected`  modifiers don't make sense for
      anonymous classes:  
      Anonymous classes, being anonymous,
      cannot be referenced at all, so references to them can't be public,
      private, or protected.

\item Anonymous classes cannot be \xcd`abstract`.  Since they only exist in
      combination with a constructor call, they must be constructable.  The
      parent class of the anonymous class may be abstract, or may be an
      interface; in this case, the anonymous class must provide all the
      methods that the parent demands.

\item Anonymous classes cannot have explicit \xcd`extends` or \xcd`implements`
      clauses; there's no place in the syntax for them. They have a single
      parent and that is that. 
\end{itemize}

\chapter{Structs}
\label{XtenStructs}
\label{StructClasses}
\label{Structs}
\index{struct}

X10 objects are a powerful general-purpose programming tool. However, the
power must be paid for in space and time. In space, a typical object
implementation requires some extra memory for run-time class information, as
well as a pointer for each reference to the object.  In time, a typical object
requires an extra indirection to read or write data, and some computation to
figure out which method body to call.  

For high-performance computing, this overhead may not be acceptable for all
objects. X10 provides structs, which are stripped-down objects. They are less
powerful than objects; in particular they lack inheritance and mutable fields.
Without inheritance, method calls do not need to do any lookup; they can be
implemented directly. Accordingly, structs can be implemented and used more
cheaply than objects, potentially avoiding the space and time overhead.
(Currently, the C++ back end avoids the overhead, but the Java back end
implements structs as Java objects and does not avoid it.)

Structs and classes are interoperable. Both can implement interfaces (in
particular, like all X10 values they implement \xcd`Any`), and subprocedures
whose arguments are defined by interfaces can take both structs and classes.
(Some caution is necessary here: referring to a struct through an interface
requires overhead similar to that required for an object.)

They are also interconvertable, within the constraints of structs. If you
start off defining a struct and decide you need a class instead, the code
change required is simply changing the keyword \xcd`struct` to \xcd`class`. If
you have a class that does not use inheritance or mutable fields, it can be
converted to a struct by changing its keyword. Client code using the struct
that was a class will need certain changes: the \xcd`new` keyword must be
added in constructor calls, and structs (unlike classes) do not have default values.



\section{Struct declaration}
\index{struct!declaration}
A struct declaration has the structure: 
\begin{xtenmath}
$\mbox{\emph{StructModifiers}}^{\mbox{?}}$
struct C[X$_1$, $\ldots$, X$_n$](p$_1$:T$_1$, $\ldots$, p$_n$:T$_n$){c} 
   implements I$_1$, $\ldots$, I$_k$ {
$\mbox{\emph{StructBody}}$
}
\end{xtenmath}

All fields of a struct must be \xcd`val`.

A struct \Xcd{S} cannot contain a field of type \Xcd{S}, or a field of struct
type \Xcd{T} which, recursively, contains a field of type \Xcd{S}.  This
restriction is necessary to permit \xcd`S` to be implemented as a contiguous
block of memory of size equal to the sum of the sizes of its fields.  


Values of a struct \Xcd{C} type can be created by invoking a constructor
defined in \Xcd{C}, but without prefixing it with \Xcd{new}: 
%~~gen
% package Structs.For.Muckts;
%~~vis
\begin{xten}
struct Polar(r:Double, theta:Double){
  def this(r:Double, theta:Double) {property(r,theta);}
  static val Origin = Polar(0,0);
  static val x0y1 = Polar(1, 3.14159/2);
}
\end{xten}
%~~siv
%
%~~neg

Structs support the same notions of generics, properties, and constrained
types that classes do.  For example, the \xcd`Pair` type below provides pairs
of values; the \xcd`diag()` method applies only when the two elements of the
pair are equal, and returns that common value: 
%~~gen
% package Structs.For.Muckts;
%~~vis
\begin{xten}
struct Pair[T,U](t:T, u:U) {
  def this(t:T, u:U) { property(t,u); }
  def diag(){T==U && t==u} = t;
}
\end{xten}
%~~siv
%
%~~neg


\section{Boxing of structs}
\index{auto-boxing!struct to interface}
\index{struct!auto-boxing}
\index{struct!casting to interface}
\label{auto-boxing} 
If a struct \Xcd{S} implements an interface \Xcd{I} (\eg, \Xcd{Any}),
a value \Xcd{v} of type \Xcd{S} can be assigned to a variable of type
\Xcd{I}. The implementation creates an object \Xcd{o} that is an
instance of an anonymous class implementing \Xcd{I} and containing
\Xcd{v}.  The result of invoking a method of \Xcd{I} on \Xcd{o} is the
same as invoking it on \Xcd{v}. This operation is termed {\em auto-boxing}.
It allows full interoperability of structs and objects---at the cost of losing
the extra efficiency of the structs when they are boxed.

In a generic class or struct obtained by instantiating a type parameter
\Xcd{T} with a struct \Xcd{S}, variables declared at type \Xcd{T} in the body
of the class are not boxed. They are implemented as if they were declared at
type \Xcd{S}.

\section{Optional Implementation of \Xcd{Any} methods}
\label{StructAnyMethods}
\index{Any!structs}

Two
structs are equal (\Xcd{==}) if and only if their corresponding fields
are equal (\Xcd{==}). 

All structs implement \Xcd{x10.lang.Any}. 
Structs are required to implement the following methods from \xcd`Any`.  
Programmers need not provide them; X10 will produce them automatically if 
the program does not include them. 
\begin{xten}
  public def equals(Any):Boolean;
  public def hashCode():Int;
  public def typeName():String;
  public def toString():String;  
\end{xten}


A programmer who provides an explicit implementation
of \Xcd{equals(Any)} for a struct \Xcd{S} should also consider
supplying a definition for \Xcd{equals(S):Boolean}. This will often
yield better performance since the cost of an upcast to \Xcd{Any} and
then a downcast to \Xcd{S} can be avoided.

\section{Primitive Types}
\index{types!primitive}
\index{primitive types}
\index{Int}
\index{UInt}
\index{Long}
\index{ULong}
\index{Char}
\index{Byte}
\index{UByte}
\index{Boolean}
\index{Short}
\index{UShort}
\index{Float}
\index{Double}

Certain types that might be built in to other languages are in fact
implemented as structs in package \xcd`x10.lang` in X10. Their methods and
operations are often provided with \xcd`@Native` (\Sref{NativeCode}) rather
than X10 code, however. These types are:
\begin{xten}
Boolean, Char, Byte, Short, Int, Long
Float, Double, UByte, UShort, UInt, ULong
\end{xten}
 
\section{Generic programming with structs}
\section{struct!generic}
\section{generics!struct}

An unconstrained type variable \Xcd{X} can be instantiated with \Xcd{Object} or
its subclasses or structs or functions.

Within a generic struct, all the operations of \Xcd{Any} are available
on a variable of type \Xcd{X}. Additionally, variables of type \Xcd{X} may
be used with \Xcd{==, !=}, in \Xcd{instanceof}, and casts.

\bard{The rest of this section is under discussion.  The example is wrong; it
ignores the fact that values can be functions.}
The programmer must be aware of the different interpretations of
equality for structs and classes and ensure that the code is correctly
written for both cases. If necessary the programmer can write code
that distinguishes between the two cases (a type parameter \Xcd{X} is
instantiated to a struct or not) as follows:

\begin{xten}
val x:X = ...;
if (x instanceof Object) { // x is a real object
   val x2 = x as Object; // this cast will always succeed.
   ...
} else { // x is a struct
   ...
}
\end{xten}
 
  
\section{Example structs}

\xcd`x10.lang.Complex` provides a detailed example of a practical struct,
suitable for use in a library.  For a shorter example, we define the
\xcd`Pair` struct---available in \xcd`x10.util.Pair`.  A \xcd`Pair` packages
two values of possibly unrelated type together in a single value, \eg, to
return two values from a function.

%~~gen
% package Structs.Pairs.Are.For.Squares;
%~~vis
\begin{xten}
struct Pair[T,U] {
    public val first:T;
    public val second:U;
    public def this(first:T, second:U):Pair[T,U] {
        this.first = first;
        this.second = second;
    }
    public def toString():String {
        return "(" + first + ", " + second + ")";
    }
}
\end{xten}
%~~siv
%
%~~neg

\section{Nested Structs}
\index{struct!static nested}
\index{static nested struct}

Static nested structs may be defined, essentially as static nested classes
except for making them structs
(\Sref{StaticNestedClasses}).  Inner structs may be defined, essentially as
inner classes except making them structs (\Sref{InnerClasses}).



\chapter{Functions}
\label{Functions}
\label{functions}
\index{functions}
\label{Closures}

\section{Overview}
Functions, the last of the three kinds of values in X10, encapsulate pieces of
code which can be applied to a vector of arguments to produce a value.
Functions, when applied, can do nearly anything that any other code could do:
fail to terminate, throw an exception, modify variables, spawn activities,
execute in several places, and so on. X10 functions are not mathematical
functions: the \xcd`f(1)` may return \xcd`true` on one call and \xcd`false` on
an immediately following call.

It is a limitation of \XtenCurrVer{} that functions do not support
type arguments. This limitation may be removed in future versions of
the language.

A \emph{function literal} \xcd"(x1:T1,..,xn:Tn){c}:T=>e" creates a function of
type\\ \xcd"(x1:T1,...,xn:Tn){c}=>T" (\Sref{FunctionType}).  For example, 
\xcd`(x:Int) => x*x` is a function literal describing the squaring function on
integers.   
\xcd`null` is also a function value.

Function application is written \xcd`f(a,b,c)`, following common mathematical
usage. 
\index{Exception!unchecked}
Function invocation may throw unchecked exceptions. 

The function body may be a block.  To compute integer squares by repeated
addition (inefficiently), one may write: 
%~~gen
% package Functions.Are.For.Spunctions;
% class Examplllll {
% static 
%~~vis
\begin{xten}
val sq: (Int) => Int 
      = (n:Int) => {
           var s : Int = 0;
           val abs_n = n < 0 ? -n : n;
           for ([i] in 1..abs_n) s += abs_n;
           s
        };
\end{xten}
%~~siv
%}
%~~neg




A function literal evaluates to a function entity {$\phi$}. When {$\phi$} is
applied to a suitable list of actual parameters \xcd`a1`-\xcd`an`, it
evaluates \xcd`e` with the formal parameters bound to the actual parameters.
So, the following are equivalent, where \xcd`e` is an expression involving
\xcd`x1` and \xcd`x2`\footnote{Strictly, there are a few other requirements;
  \eg, \xcd`result` must be a \xcd`var` of type \xcd`T` defined outside the
  outer block, the variables \xcd`a1` and \xcd`a2` had better not appear in
  \xcd`e`, and everything in sight had better typecheck properly.}

%~~gen
% package functions2.why.is.there.a.two;
% abstract class FunctionsTooManyFlippingFunctions[T, T1, T2]{
% abstract def arg1():T1;
% abstract def arg2():T2;
% def thing1(e:T) {var result:T;
%~~vis
\begin{xten}
{
  val f = (x1:T1,x2:T2){true}:T => e;
  val a1 : T1 = arg1();
  val a2 : T2 = arg2();
  result = f(a1,a2);
}
\end{xten}
%~~siv
%}}
%~~neg
and 
%~~gen
% package functions2.why.is.there.a.two.but.here.is.the.other.one;
% abstract class FunctionsTooManyFlippingFunctions[T, T1, T2]{
% abstract def arg1():T1;
% abstract def arg2():T2;
% def thing1(e:T) {var result:T;
%~~vis
\begin{xten}
{
  val a1 : T1 = arg1();
  val a2 : T2 = arg2();
  {
     val x1 : T1 = a1;
     val x2 : T2 = a2;
     result = e;
  }  
}
\end{xten}
%~~siv
%}}
%~~neg
\noindent
This doesn't quite work if the body is a statement rather than an expression.
A few language features are forbidden (\xcd`break` or \xcd`continue` of a loop
that surrounds the function literal) or mean something different (\xcd`return`
inside a function returns from the function). 





The \emph{method selector expression} \Xcd{e.m.(x1:T1,...,xn:Tn)} (\Sref{MethodSelectors})
permits the specification of the function underlying
the method \Xcd{m}, which takes arguments of type \Xcd{(x1:T1,..., xn:Tn)}.
Within this function, \Xcd{this} is bound to the result of evaluating \Xcd{e}.

Function types may be used in \Xcd{implements} clauses of class
definitions. Instances of such classes may be used as functions of the
given type.  Indeed, an object may behave like any (fixed) number of
functions, since the class it is an instance of may implement any
(fixed) number of function types.

%\section{Implementation Notes}
%\begin{itemize}
%
%\item Note that e.m.(T1,...,Tn) will evaluate e to create a
%  function. This function will be applied later to given
%  arguments. Thus this syntax can be used to evaluate the receiver of
%  a method call ahead of the actual invocation. The resulting function
%  can be used multiple times, of course.
%\end{itemize}


\section{Function Literals}
\index{literal!function}
\label{FunctionLiteral}

\Xten{} provides first-class, typed functions, including
\emph{closures}, \emph{operator functions}, and \emph{method
  selectors}.

\begin{grammar}
ClosureExpression \:
        \xcd"("
        Formals\opt
        \xcd")"
\\ &&
        Guard\opt
        ReturnType\opt
        Throws\opt
        Offers\opt
        \xcd"=>" ClosureBody \\
ClosureBody \:
        Expression \\
        \| \xcd"{" Statement\star \xcd"}" \\
        \| \xcd"{" Statement\star Expression \xcd"}" \\
\end{grammar}

Functions have zero or more formal parameters and an optional return type.
The body has the 
same syntax as a method body; it may be either an expression, a block
of statements, or a block terminated by an expression to return. In
particular, a value may be returned from the body of the function
using a return statement (\Sref{ReturnStatement}). 

The type of a
function is a function type (\Sref{FunctionType}).  In some cases the
return type \Xcd{T} is also optional and defaults to the type of the
body. If a formal \Xcd{xi} does not occur in any
\Xcd{Tj}, \Xcd{c}, \Xcd{T} or \Xcd{e}, the declaration \Xcd{xi:Ti} may
be replaced by just \Xcd{Ti}: \xcd`(Int)=>7` is the integer function returning
7 for all inputs.

\label{ClosureGuard}

As with methods, a function may declare a guard to
constrain the actual parameters with which it may be invoked.
The guard may refer to the type parameters, formal parameters,
and any \xcd`val`s in scope at the function expression.

The body of the function is evaluated when the function is
invoked by a call expression (\Sref{Call}), not at the function's
place in the program text.

As with methods, a function with return type \xcd"Void" cannot
have a terminating expression. 
If the return type is omitted, it is inferred, as described in
\Sref{TypeInference}.
It is a static error if the return type cannot be inferred.  \Eg,
\xcd`(Int)=>null` is not well-defined; X10 does not know which type of
\xcd`null` is intended.  
%~~exp~~`~~`~~ ~~
But \xcd`(Int):Rail[Double] => null` is legal.


\begin{example}
The following method takes a function parameter and uses it to
test each element of the list, returning the first matching
element.  It returns \xcd`otherwise` if no element matches.

%~~gen
% package functions2.oh.no;
% import x10.util.*;
% class Finder {
% static 
%~~vis
\begin{xten}

def find[T](f: (T) => Boolean, xs: List[T], absent:T): T = {
  for (x: T in xs)
    if (f(x)) return x;
  absent
  }
\end{xten}
%~~siv
% }
%~~neg

The method may be invoked thus:
%~~gen
% package functions2.oh.no.my.ears;
% import x10.util.*;
% class Finderator {
% static def find[T](f: (T) => Boolean, xs: x10.util.List[T], absent:T): T = {
%  for (x: T in xs)
%    if (f(x)) return x;
%  absent
%}
% static def checkery() {
%~~vis
\begin{xten}
xs: List[Int] = new ArrayList[Int]();
x: Int = find((x: Int) => x>0, xs, 0);
\end{xten}
%~~siv
%}}
%~~neg

\end{example}

As with a normal method, the function may have a \xcd"throws"
clause. It is a static error if the body of the function throws a
checked exception that is not declared in the function's \xcd"throws"
clause.


\subsection{Outer variable access}

In a function
\xcdmath"(x$_1$: T$_1$, $\dots$, x$_n$: T$_n$){c} => { s }"
the types \xcdmath"T$_i$", the guard \xcd"c" and the body \xcd"s"
may access many, though not all, sorts of variables from outer scopes.  
Specifically, they can access: 
\begin{itemize}
\item All fields of the enclosing object and class;
\item All type parameters;
\item All \xcd`val` variables;
\item \xcd`var` variables with the \xcd`shared` annotation. 
\end{itemize}


\limitation{\xcd`shared` is not currently supported.}

The function body may refer to instances of enclosing classes using
the syntax \xcd"C.this", where \xcd"C" is the name of the
enclosing class.  \xcd`this` refers to the instance of the immediately
enclosing class, as usual.

For example, the following is legal.  However, it would not be legal to add
\xcd`e` or \xcd`h` to the sum; they are non-\xcd`shared` \xcd`var`s from the
surrounding scope.

%%TODO -- this example uses 'shared', which is not currently available.
\begin{xten}
class Lambda {
   var a : Int = 0;
   val b = 0;
   def m(var c : Int, shared var d : Int,  val e : Int) {
      var f : Int = 0;
      shared var g : Int = 0;
      val h : Int = 0;
      val closure = (var i: Int, val j: Int) => {
    	  return a + b + d + g + i + j + this.a + Lambda.this.a;
      };
      return closure;
   }
}
\end{xten}


{\bf Rationale:} Non-\xcd`shared` \xcd`var`s like \xcd`e` and \xcd`h` are
excluded in X10, as in many other languages, for practical implementation
reasons. They are allocated on the stack, which is desirable for efficiency.
However, the closure may exist for long after the stack frame containing
\xcd`e` and \xcd`h` has been freed, so those storage locations are no longer
valid for those variables. \xcd`shared var`s are heap-allocated, which is less
efficient but allows them to exist after \xcd`m` returns. 


\xcd`shared` does not guarantee {\bf atomic} access to the shared variable. As
with any code that might mutate shared data concurrently, be sure to protect
references to mutable shared state with \xcd`atomic`. For example, the
following code returns a pair of closures which operate on the same shared
variable \xcd`a`, which are concurrency-safe---even if invoked many times
simultaneously. Without \xcd`atomic`, it would no longer be concurrency-safe.


%~fails~gen
% package Functions2.Are.All.Too.Much;
% class Fun2Frivols {
%~fails~vis
\begin{xten}
  def counters() {
      shared var a : Int = 0;
       return [
          () => {atomic a ++;},
          () => {atomic return a;}
          ];
   }
\end{xten}
%~fails~siv
%}
%
%~fails~neg


%SHARED% \begin{note}
%SHARED% The main activity may run in parallel with any
%SHARED% functions it creates. Hence even the read of an outer variable by the
%SHARED% body of a function may result in a race condition. Since functions are
%SHARED% first-class, the analysis of whether a function may execute in parallel
%SHARED% with the activity that created it may be difficult.
%SHARED% \end{note}

%% vj: This should be verified.
%\begin{note}
%The rule for accessing outer variables from function bodies
%should be the same as the rule for accessing outer variables from local
%or anonymous classes.
%\end{note}

\section{Method selectors}
\label{MethodSelectors}
\index{function!method selector}
\index{method!underlying function}

A method selector expression allows a method to be used as a
first-class function, without writing a function expression for it.
For example, consider a class \xcd`Span` defining ranges of integers.  

%~~gen
% package Functions2.Span;
%~~vis
\begin{xten}
class Span(low:Int, high:Int) {
  def this(low:Int, high:Int) {property(low,high);}
  def between(n:Int) = low <= n && n <= high;
  def example() {
    val digit = new Span(0,9);
    val isDigit : (Int) => Boolean = digit.between.(Int);
    if (isDigit(8)) Console.OUT.println("8 is!");
  }
}
\end{xten}
%~~siv
%
%~~neg
\noindent


In \xcd`example()`, 
%~~exp~~`~~`~~ digit:Span~~class Span(low:Int, high:Int) {def this(low:Int, high:Int) {property(low,high);} def between(n:Int) = low <= n && n <= high;}
\xcd`digit.between.(Int)` 
is a unary function testing whether its argument is between zero
and nine.  It could also be written 
%~~exp~~`~~`~~ digit:Span~~class Span(low:Int, high:Int) {def this(low:Int, high:Int) {property(low,high);} def between(n:Int) = low <= n && n <= high;}
\xcd`(n:Int) => digit.between(n)`.

This is formalized thus:

\begin{grammar}
MethodSelector \:
        Primary \xcd"."
        MethodName \xcd"."
                TypeParameters\opt \xcd"(" Formals\opt \xcd")" \\
      \|
        TypeName \xcd"."
        MethodName \xcd"."
                TypeParameters\opt \xcd"(" Formals\opt \xcd")" \\
\end{grammar}

The \emph{method selector expression} \Xcd{e.m.(T1,...,Tn)} is type
correct only if  the static type of \Xcd{e} is a
class or struct or interface \xcd`V` with a method
\Xcd{m(x1:T1,...xn:Tn)\{c\}:T} defined on it (for some
\Xcd{x1,...,xn,c,T)}. At runtime the evaluation of this expression
evaluates \Xcd{e} to a value \Xcd{v} and creates a function \Xcd{f}
which, when applied to an argument list \Xcd{(a1,...,an)} (of the right
type) yields the value obtained by evaluating \Xcd{v.m(a1,...,an)}.

Thus, the method selector

\begin{xtenmath}
e.m.[X$_1$, $\dots$, X$_m$](T$_1$, $\dots$, T$_n$)
\end{xtenmath}
\noindent behaves as if it were the function
\begin{xtenmath}
((v:V)=>
  [X$_1$, $\dots$, X$_m$](x$_1$: T$_1$, $\dots$, x$_n$: T$_n$){c} 
  => v.m[X$_1$, $\dots$, X$_m$](x$_1$, $\dots$, x$_n$))
(e)
\end{xtenmath}


\limitation{X10 functions, including method selectors, do not currently accept
generic arguments.}

Because of overloading, a method name is not sufficient to
uniquely identify a function for a given class (in Java-like languages).
One needs the argument type information as well.
The selector syntax (dot) is used to distinguish \xcd"e.m()" (a
method invocation on \xcd"e" of method named \xcd"m" with no arguments)
from \xcd"e.m.()"
(the function bound to the method). 

A static method provides a binding from a name to a function that is
independent of any instance of a class; rather it is associated with the
class itself. The static function selector
\xcdmath"T.m.(T$_1$, $\dots$, T$_n$)" denotes the
function bound to the static method named \xcd"m", with argument types
\xcdmath"(T$_1$, $\dots$, T$_n$)" for the type \xcd"T". The return type
of the function is specified by the declaration of \xcd"T.m".

There is no difference between using a function defined directly 
directly using the function syntax, or obtained via static or
instance function selectors.


\section{Operator functions}
\label{OperatorFunction}
\index{function!operator}
Every operator (e.g.,
\xcd"+",
\xcd"-",
\xcd"*",
\xcd"/",
\dots) has a family of functions, one for
each type on which the operator is defined. The function can be
selected using the ``\xcd`.`'' syntax:

\begin{grammar}
OperatorFunction
        \: TypeName \xcd"." Operator \xcd"(" Formals\opt \xcd")" \\
        \| TypeName \xcd"." Operator \\
\end{grammar}

If an operator has more than one arity (\eg, unary and binary
\xcd"-"), the unary version may be selected by giving the
formal parameter types.  The binary version is selected by
default, or the types may be specified for clarity.
For example, the following equivalences hold:

\begin{xtenmath}
String.+             $\equiv$ (x: String, y: String): String => x + y
Long.-               $\equiv$ (x: Long, y: Long): Long => x - y
Float.-(Float,Float) $\equiv$ (x: Float, y: Float): Float => x - y
Int.-(Int)           $\equiv$ (x: Int): Int => -x
Boolean.&            $\equiv$ (x: Boolean, y: Boolean): Boolean => x & y
Boolean.!            $\equiv$ (x: Boolean): Boolean => !x
Int.<(Int,Int)       $\equiv$ (x: Int, y: Int): Boolean => x < y
Dist.|(Place)        $\equiv$ (d: Dist, p: Place): Dist => d | p
\end{xtenmath}


%%TODO -- fix commented-out lines!

%~~gen
% package Functions2.For.The.Lose;
% class TypecheckThatSillyExample {
%   def checker() {
%    val l1 : (String, String) => String = String.+;
%    val r1 : (String, String) => String = (x: String, y: String): String => x + y;
%    val l2 : (Long,Long) => Long = Long.-;
%    val r2 : (Long,Long) => Long = (x: Long, y: Long): Long => x - y;
%//var v1 : (Float,Float) => Float = Float.-(Float,Float) ;
%var v2 : (Float,Float) => Float = (x: Float, y: Float): Float => x - y;
%//var v3 : (Int) => Int =  Int.-(Int)     ;      ;
%var v4  : (Int) => Int  =  (x: Int): Int => -x;
%var v5 : (Boolean,Boolean) => Boolean = Boolean.&            ;
%var v6 : (Boolean,Boolean) => Boolean =  (x: Boolean, y: Boolean): Boolean => x & y;
%//var v7 : (Boolean) => Boolean = Boolean.!            ;
%var v8 : (Boolean) => Boolean =  (x: Boolean): Boolean => !x;
%//var v9 : (Int,Int) => Boolean = Int.<(Int,Int)       ;
%var v10: (Int,Int) => Boolean =  (x: Int, y: Int): Boolean => x < y;
%//var v11 : (Dist,Place)=>Dist = Dist.|(Place)        ;
%var v12 : (Dist,Place)=>Dist=  (d: Dist, p: Place): Dist => d | p;
%}
% }
%~~vis
%~~siv
%
%~~neg

Unary and binary promotion (\Sref{XtenPromotions}) is not performed
when invoking these
operations; instead, the operands are coerced individually via implicit
coercions (\Sref{XtenConversions}), as appropriate.


%%WE-NEVER-GOT-TO-IT%%  \begin{planned}
%%WE-NEVER-GOT-TO-IT%%  
%%WE-NEVER-GOT-TO-IT%%  {\bf The following is not implemented in version 2.0.3:}
%%WE-NEVER-GOT-TO-IT%%  
%%WE-NEVER-GOT-TO-IT%%  Additionally, for every expression \xcd"e" of a type \xcd"T" at which a binary
%%WE-NEVER-GOT-TO-IT%%  operator \xcd"OP" is defined, the expression \xcd"e.OP" or
%%WE-NEVER-GOT-TO-IT%%  \xcd"e.OP(T)" represents the function
%%WE-NEVER-GOT-TO-IT%%  defined by:
%%WE-NEVER-GOT-TO-IT%%  
%%WE-NEVER-GOT-TO-IT%%  \begin{xten}
%%WE-NEVER-GOT-TO-IT%%  (x: T): T => { e OP x }
%%WE-NEVER-GOT-TO-IT%%  \end{xten}
%%WE-NEVER-GOT-TO-IT%%  
%%WE-NEVER-GOT-TO-IT%%  \begin{grammar}
%%WE-NEVER-GOT-TO-IT%%  Primary \: Expr \xcd"." Operator \xcd"(" Formals\opt \xcd")" \\
%%WE-NEVER-GOT-TO-IT%%          \| Expr \xcd"." Operator \\
%%WE-NEVER-GOT-TO-IT%%  \end{grammar}
%%WE-NEVER-GOT-TO-IT%%  
%%WE-NEVER-GOT-TO-IT%%  %% For every expression \xcd"e" of a type \xcd"T" at which a unary
%%WE-NEVER-GOT-TO-IT%%  %%operator \xcd"OP" is defined, the expression \xcd"e.OP()"
%%WE-NEVER-GOT-TO-IT%%  %% represents the function defined by:
%%WE-NEVER-GOT-TO-IT%%  
%%WE-NEVER-GOT-TO-IT%%  %% \begin{xten}
%%WE-NEVER-GOT-TO-IT%%  %% (): T => { OP e }
%%WE-NEVER-GOT-TO-IT%%  %% \end{xten}
%%WE-NEVER-GOT-TO-IT%%  
%%WE-NEVER-GOT-TO-IT%%  For example,
%%WE-NEVER-GOT-TO-IT%%  one may write an expression that adds one to each member of a
%%WE-NEVER-GOT-TO-IT%%  list \xcd"xs" by:
%%WE-NEVER-GOT-TO-IT%%  
%%WE-NEVER-GOT-TO-IT%%  %%TODO -- when this topic works, make the example wwork too.
%%WE-NEVER-GOT-TO-IT%%  %~x~gen
%%WE-NEVER-GOT-TO-IT%%  % package Functions2.Wants.A.Dinner.Reservation;
%%WE-NEVER-GOT-TO-IT%%  % import x10.util.*;
%%WE-NEVER-GOT-TO-IT%%  % class Reservation {
%%WE-NEVER-GOT-TO-IT%%  % def smerp() {
%%WE-NEVER-GOT-TO-IT%%  %   val xs = new ArrayList[Int]();
%%WE-NEVER-GOT-TO-IT%%  %~x~vis
%%WE-NEVER-GOT-TO-IT%%  \begin{xten}
%%WE-NEVER-GOT-TO-IT%%  xs.map(1.+);
%%WE-NEVER-GOT-TO-IT%%  \end{xten}
%%WE-NEVER-GOT-TO-IT%%  %~x~siv
%%WE-NEVER-GOT-TO-IT%%  % }
%%WE-NEVER-GOT-TO-IT%%  % }
%%WE-NEVER-GOT-TO-IT%%  %
%%WE-NEVER-GOT-TO-IT%%  %~x~neg
%%WE-NEVER-GOT-TO-IT%%  \end{planned}
%%WE-NEVER-GOT-TO-IT%%  
%%WE-NEVER-GOT-TO-IT%%  
\section{Functions as objects of type \Xcd{Any}}
\label{FunctionAnyMethods}

\label{FunctionEquality}
\index{function!equality} \index{equality!function} Two functions \Xcd{f} and
\Xcd{g} are equal, \xcd`f==g` if both are instances of classes and the same
object, or if both were obtained by the same evaluation of a function
literal.\footnote{A literal may occur in program text within a loop, and hence
  may be evaluated multiple times.} Further, it is guaranteed that if two
functions are equal then they refer to the same locations in the environment
and represent the same code, so their executions in an identical situation are
indistinguishable. (Specifically, if \xcd`f == g`, then \xcd`f(1)` can be
substituted for \xcd`g(1)` and the result will be identical. However, there is
no guarantee that \xcd`f(1)==g(1)` will evaluate to true. Indeed, there is no
guarantee that \xcd`f(1)==f(1)` will evaluate to true either, as \xcd`f` might
be a function which returns {$n$} on its {$n^{th}$} invocation. However,
\xcd`f(1)==f(1)` and \xcd`f(1)==g(1)` are interchangeable.)
\index{function!==}


Every function type implements all the methods of \Xcd{Any}.
\xcd`f.equals(g)` is equivalent to \xcd`f==g`.  \xcd`f.hashCode()`, 
\xcd`f.toString()`, and \xcd`f.typeName()` are implementation-dependent, but
respect \xcd`equals` and the basic contracts of \xcd`Any`. 

\index{function!equals}
\index{function!hashCode}
\index{function!toString}
\index{function!typeName}
\index{function!home}
\index{function!at(Place)}
\index{function!at(Object)}



\chapter{Expressions}\label{XtenExpressions}\index{expression}

\Xten{} has a rich expression language.
Evaluating an expression produces a value, or, in a few cases, no value. 
Expression evaluation may have side effects, such as change of the value of a 
\xcd`var` variable or a data structure, allocation of new values, or throwing
an exception. 



\section{Literals}
\index{literal}

Literals denote fixed values of built-in types. 
The syntax for literals is given in \Sref{Literals}. 

The type that \Xten{} gives a literal often includes its value. \Eg, \xcd`1`
is of type \xcd`Int{self==1}`, and \xcd`true` is of type
\xcd`Boolean{self==true}`.

\section{{\tt this}}
\index{this}
\index{\Xcd{this}}

\begin{bbgrammar}
%(FROM #(prod:Primary)#)
             Primary \: \xcd"this" (\ref{prod:Primary}) \\
                    \| \xcd"this" \\
                    \| ClassName \xcd"." \xcd"this" \\
\end{bbgrammar}


The expression \xcd"this" is a  local \xcd`val` containing a reference
to an instance of the lexically enclosing class.
It may be used only within the body of an instance method, a
constructor, or in the initializer of a instance field -- that is, the places
where there is an instance of the class under consideration.

Within an inner class, \xcd"this" may be qualified with the
name of a lexically enclosing class.  In this case, it
represents an instance of that enclosing class.  


\begin{ex}
\xcd`Outer` is a class containing \xcd`Inner`.  Each instance of
\xcd`Inner` has a reference \xcd`Outer.this` to the \xcd`Outer` involved in its
creation.  \xcd`Inner` has access to the fields of \xcd`Outer.this`, as seen
in the \xcd`outerThree` and \xcd`alwaysTrue` methods.  Note that \xcd`Inner`
has its own \xcd`three` field, which is different from and not even the same
type as \xcd`Outer.this.three`. 
%~~gen ^^^ Expressions10
% package exp.vexp.pexp.lexp.shexp; 
% NOTEST
%~~vis 
\begin{xten}
class Outer {
  val three = 3;
  class Inner {
     val three = "THREE";
     def outerThree() = Outer.this.three;
     def alwaysTrue() = outerThree() == 3;
  }
}
\end{xten}
%~~siv
%
%~~neg
\end{ex}

The type of a \xcd"this" expression is the
innermost enclosing class, or the qualifying class,
constrained by the class invariant and the
method guard, if any.

The \xcd"this" expression may also be used within constraints in
a class or interface header (the class invariant and
\xcd"extends" and \xcd"implements" clauses).  Here, the type of
\xcd"this" is restricted so that only properties declared in the
class header itself, and specifically not any members declared in the class
body or in supertypes, are accessible through \xcd"this".

\section{Local variables}

%##(Id
\begin{bbgrammar}
%(FROM #(prod:Id)#)
                  Id \: identifier & (\ref{prod:Id}) \\
\end{bbgrammar}
%##)

A local variable expression consists simply of the name of the local variable,
field of the current object, formal parameter in scope, etc. It evaluates to
the value of the local variable. 


\begin{ex}
\xcd`n` in the second line below is a local
variable expression.  The \xcd`n` in the first line is not; it is part of a
local variable declaration.
%~~gen  ^^^ Expressions20
% package exp.loc.al.varia.ble; 
% class Example {
% def example() { 
%~~vis
\begin{xten}
val n = 22;
val m = n + 56;
\end{xten}
%~~siv
%} }
%~~neg

\end{ex}

\section{Field access}
\label{FieldAccess}
\index{field!access to}

%##(FieldAccess
\begin{bbgrammar}
%(FROM #(prod:FieldAccess)#)
         FieldAccess \: Primary \xcd"." Id & (\ref{prod:FieldAccess}) \\
                    \| \xcd"super" \xcd"." Id \\
                    \| ClassName \xcd"." \xcd"super"  \xcd"." Id \\
                    \| Primary \xcd"." \xcd"class"  \\
                    \| \xcd"super" \xcd"." \xcd"class"  \\
                    \| ClassName \xcd"." \xcd"super"  \xcd"." \xcd"class"  \\
\end{bbgrammar}
%##)

A field of an object instance may be  accessed
with a field access expression.

The type of the access is the declared type of the field with the
actual target substituted for \xcd"this" in the type. 

\begin{ex}
The declaration of \xcd`b` below has a constraint involving \xcd`this`.  
The use of an instance of it, \xcd`f.b`, has the same constraint involving
\xcd`f` instead of \xcd`this`, as required.
%~~gen ^^^ Expressions5s7v
% package Expressions5s7v;
%~~vis
\begin{xten}
class Fielded {
  public val a : Int = 1;
  public val b : Int{this.a == b} = this.a;
  static def example() {
    val f : Fielded = new Fielded();
    val fb : Int{fb == f.a} = f.b;
  }
}
\end{xten}
%~~siv
%
%~~neg

\end{ex}
% If the actual
%target is not a final access path (\Sref{FinalAccessPath}),
%an anonymous path is substituted for \xcd"this".

The field accessed is selected from the fields and value properties
of the static type of the target and its superclasses.

If the field target is given by the keyword \xcd"super", the target's type is
the superclass of the enclosing class.  This form is used to access fields of
the parent class shadowed by same-named fields of the current class.

If the field target is \xcd`Cls.super`, then the target's type is \xcd`Cls`,
which must be an  enclosing class.  This (admittedly
obscure) form is used to access fields of an ancestor class which are shadowed
by same-named fields of some more recent ancestor.  

\begin{ex}
This illustrates all four cases of field access.
%~~gen ^^^ Expressions30
% package exp.re.ssio.ns.fiel.dacc.ess;
% NOTEST
%~~vis
\begin{xten}
class Uncle {
  public static val f = 1;
}
class Parent {
  public val f = 2;
}
class Ego extends Parent {
  public val f = 3;
  class Child extends Ego {
     public val f = 4;
     def classNameDotId() =  Uncle.f;     // 1
     def cnDotSuperDotId() = Ego.super.f; // 2
     def superDotId() =      super.f;     // 3
     def expDotId() =        this.f;      // 4
  }
}
\end{xten}
%~~siv
%
%~~neg
\end{ex}

If the field target is \xcd"null", a \xcd"NullPointerException"
is thrown.
If the field target is a class name, a static field is selected.
It is illegal to access  a field that is not visible from
the current context.
It is illegal to access a non-static field
through a static field access expression.  However, it is legal to access a
static field through a non-static reference.

\section{Function Literals}
Function literals are described in \Sref{Functions}.

\section{Calls}
\label{Call}
\label{MethodInvocation}
\label{MethodInvocationSubstitution}
\index{invocation}
\index{call}
\index{invocation!method}
\index{call!method}
\index{invocation!function}
\index{call!function}
\index{method!calling}
\index{method!invoking}


%##(MethodInvocation ArgumentList
\begin{bbgrammar}
%(FROM #(prod:MethodInvocation)#)
    MethodInvocation \: MethodPrimaryPrefix \xcd"(" ArgumentList\opt \xcd")" & (\ref{prod:MethodInvocation}) \\
                    \| MethodSuperPrefix \xcd"(" ArgumentList\opt \xcd")" \\
                    \| MethodClassNameSuperPrefix \xcd"(" ArgumentList\opt \xcd")" \\
                    \| MethodName TypeArguments\opt \xcd"(" ArgumentList\opt \xcd")" \\
                    \| Primary \xcd"." Id TypeArguments\opt \xcd"(" ArgumentList\opt \xcd")" \\
                    \| \xcd"super" \xcd"." Id TypeArguments\opt \xcd"(" ArgumentList\opt \xcd")" \\
                    \| ClassName \xcd"." \xcd"super"  \xcd"." Id TypeArguments\opt \xcd"(" ArgumentList\opt \xcd")" \\
                    \| Primary TypeArguments\opt \xcd"(" ArgumentList\opt \xcd")" \\
%(FROM #(prod:ArgumentList)#)
        ArgumentList \: Exp & (\ref{prod:ArgumentList}) \\
                    \| ArgumentList \xcd"," Exp \\
\end{bbgrammar}
%##)


A \grammarrule{MethodInvocation} may be to either a \xcd"static" method, an
instance method, or a closure.

The syntax for method invocations is ambiguous. \xcd`ob.m()` could either be
the invocation of a method named \xcd`m` on object \xcd`ob`, or the
application of a function held in a field \xcd`ob.m`. The target \xcd`ob` must
be type-checked to determine which of these it is.  It is a static error if
both cases are possible after type checking.

\begin{ex}
%~~gen ^^^ Expressions40
% package expres.sio.nsca.lls;
%~~vis
\begin{xten}
class Callsome {
  static val closure : () => Int = () => 1;
  static def method()            = 2;
  static def example() {
     assert Callsome.closure() == 1;
     assert Callsome.method()  == 2;
  } 
}
\end{xten}
%~~siv
% class Hook{ def run() { Callsome.example(); return true; } }
%~~neg
However, adding a static method [mis]named \xcd`closure` makes
\xcd`Callsome.closure()` 
ambiguous: it could be a call to the closure, or to the static method: 
%~~gen ^^^ Expressions50
% package expres.sio.nsca.lls.twoooo;
% class Callsome {static val closure = () => 1; static def method () = 2; static val methodEvaluated = Callsome.method();
%~~vis
\begin{xten}
  static def closure () = 3;
  // ERROR: static errory = Callsome.closure();
\end{xten}
%~~siv
% }
%~~neg
\end{ex}

The application form \xcd`e(f,g)`, when \xcd`e` evaluates to an object or
struct, invokes the application \xcd`operator`, 
defined in the form 
%~~gen ^^^ Expressions2x1f
% package Expressions2x1f;
% class Example[F,G] {
%~~vis
\begin{xten}
public operator this(f:F, g:G) = "value";
\end{xten}
%~~siv
%  }
%~~neg


Method selection rules are given in \Sref{sect:MethodResolution}.

It is a static error if a method's \grammarrule{Guard} is not statically
satisfied by the 
caller.  

\begin{ex}
In this example, a \xcd`DivideBy` object provides the service of dividing
numbers by \xcd`denom` --- so long as \xcd`denom` is not zero. 
In the \xcd`example` method, \xcd`this.div(100)`  is not allowed; there is no
guarantee that \xcd`denom != 0`.  Casting \xcd`this` to a type 
whose constraint implies \xcd`denom != 0` permits the method call.
%~~gen ^^^ Expressions60
%package Expressions.Calls.Guarded.By.Walls;
% KNOWNFAIL
%~~vis
\begin{xten}
class DivideBy(denom:Int) {
  def div(numer:Int){denom != 0} = numer / denom;
  def example() {
     val thisCast = (this as DivideBy{self.denom != 0});
     thisCast.div(100);
     //ERROR: this.div(100); 
  }
}
\end{xten}
%~~siv
% class Hook{ def run() { (new DivideBy(1)).example(); return true; } }
%~~neg
\end{ex}

\section{Assignment}\index{assignment}\label{AssignmentStatement}

%##(Assignment LeftHandSide AssignmentOperator
\begin{bbgrammar}
%(FROM #(prod:Assignment)#)
          Assignment \: LeftHandSide AssignmentOperator AssignmentExp & (\ref{prod:Assignment}) \\
                    \| ExpName  \xcd"(" ArgumentList\opt \xcd")" AssignmentOperator AssignmentExp \\
                    \| Primary  \xcd"(" ArgumentList\opt \xcd")" AssignmentOperator AssignmentExp \\
%(FROM #(prod:LeftHandSide)#)
        LeftHandSide \: ExpName & (\ref{prod:LeftHandSide}) \\
                    \| FieldAccess \\
%(FROM #(prod:AssignmentOperator)#)
  AssignmentOperator \: \xcd"=" & (\ref{prod:AssignmentOperator}) \\
                    \| \xcd"*=" \\
                    \| \xcd"/=" \\
                    \| \xcd"%=" \\
                    \| \xcd"+=" \\
                    \| \xcd"-=" \\
                    \| \xcd"<<=" \\
                    \| \xcd">>=" \\
                    \| \xcd">>>=" \\
                    \| \xcd"&=" \\
                    \| \xcd"^=" \\
                    \| \xcd"|=" \\
\end{bbgrammar}
%##)



The assignment expression \xcd"x = e" assigns a value given by
expression \xcd"e"
to a variable \xcd"x".  
Most often, \xcd`x` is mutable, a \xcd`var` variable.  The same syntax is
used for delayed initialization of a \xcd`val`, but \xcd`val`s can only be
initialized once.
%~~gen ^^^ Expressions70
% package express.ions.ass.ignment;
% class Example {
% static def exasmple() {
%~~vis
\begin{xten}
  var x : Int;
  val y : Int;
  x = 1;
  y = 2; // Correct; initializes y
  x = 3; 
  // ERROR: y = 4;
\end{xten}
%~~siv
% } } 
%~~neg


There are three syntactic forms of
assignment: 
\begin{enumerate}
\item \xcd`x = e;`, assigning to a local variable, formal parameter, field of
      \xcd`this`, etc. 
\item \xcd`x.f = e;`, assigning to a field of an object.
\item \xcdmath`a(i$_1$,$\ldots$,i$_n$) = v;`, where {$n \ge 0$}, assigning to
      an element of an array or some other such structure. This is an operator
      call (\Sref{sect:operators}).  For well-behaved classes it works like
      array assignment, mutatis mutandis, but there is no actual guarantee,
      and the compiler makes no assumptions about how this works for arbitrary \xcd`a`.
      Naturally, it is a static error if no suitable assignment operator
      for \xcd`a`.
\end{enumerate}

For a binary operator $\diamond$, the $\diamond$-assignment expression
\xcdmath"x $\diamond$= e" combines the current value of \xcd`x` with the value
of \xcd`e` by {$\diamond$}, and stores the result back into \xcd`x`.  
\xcd`i += 2`, for example, adds 2 to \xcd`i`. For variables and fields, 
\xcdmath"x $\diamond$= e" behaves just like \xcdmath"x = x $\diamond$ e". 

The subscripting forms of \xcdmath"a(i) $\diamond$= b" are slightly subtle.
Subexpressions of \xcd`a` and \xcd`i` are only evaluated once.  However,
\xcd`a(i)` and \xcd`a(i)=c` are each executed once---in particular, there is
one call to the application operator, and one to the assignment operator.
If subscripting is implemented strangely for
the class of \xcd`a`, the behavior is {\em not} necessarily updating a single
storage location. Specifically, \xcd`A()(I()) += B()` is tantamount to: 
%~~gen ^^^ Expressions80
% package expressions.stupid.addab;
% class Example {
% def example(A:()=>Rail[Int], I: () => Int, B: () => Int ) {
%~~vis
\begin{xten}
{
  val aa = A();  // Evaluate A() once
  val ii = I();  // Evaluate I() once
  val bb = B();  // Evaluate B() once
  val tmp = aa(ii) + bb; // read aa(ii)
  aa(ii) = tmp;  // write sum back to aa(ii)
}
\end{xten}
%~~siv
%}}
%~~neg

\limitation{+= does not currently meet this specification.}




\section{Increment and decrement}
\index{increment}
\index{decrement}
\index{\Xcd{++}}
\index{\Xcd{--}}


The operators \xcd"++" and \xcd"--" increment and decrement
a variable, respectively.  
\xcd`x++` and \xcd`++x` both increment \xcd`x`, just as the statement 
\xcd`x += 1` would, and similarly for \xcd`--`.  

The difference between the two is the return value.  
\xcd`++x` and \xcd`--x` return the {\em new} value of \xcd`x`, after
incrementing or decrementing.
\xcd`x++` and \xcd`x--` return the {\em old} value of \xcd`x`, before
incrementing or decrementing.


\limitation{This currently only works for numeric types.}

\section{Numeric Operations}
\label{XtenPromotions}
\index{promotion}
\index{numeric promotion}
\index{numeric operations}
\index{operation!numeric}

Numeric types (\xcd`Byte`, \xcd`Short`, \xcd`Int`, \xcd`Long`, \xcd`Float`,
\xcd`Double`, \xcd`Complex`, and unsigned variants of fixed-point types) are normal X10
structs, though most of their methods are implemented via native code. They
obey the same general rules as other X10 structs. For example, numeric
operations, coercions, and conversions are defined by \xcd`operator` definitions, the same way you could
for any struct.

Promoting a numeric value to a longer numeric type preserves the sign of the
value.  For example, \xcd`(255 as UByte) as UInt` is 255. 

Most of these operations can be defined on user-defined types as well.  While
it is good practice to keep such operations consistent with the numeric
operations whenever possible, the compiler neither enforces nor assumes any
particular semantics of user-defined operations. 

\subsection{Conversions and coercions}

Specifically, each numeric type can be converted or coerced into each other
numeric type, perhaps with loss of accuracy.
%~~gen ^^^ Expressions90
% package exp.ress.io.ns.numeric.conversions;
% class ExampleOfConversionAndStuff {
% def example() {
%~~vis
\begin{xten}
val n : Byte = 123 as Byte; // explicit 
val f : (Int)=>Boolean = (Int) => true; 
val ok = f(n); // implicit
\end{xten}
%~~siv
% } }
%~~neg



\subsection{Unary plus and unary minus}

The unary \xcd`+` operation on numbers is an identity function.
The unary \xcd`-` operation on numbers is a negation function.
On unsigned numbers, these are two's-complement.  For example, 
\xcd`-(0x0F as UByte)` is 
\xcd`(0xF1 as UByte)`.
\bard{UInts and such are closed under negation -- the negative of a UInt is
done binarily.  }



\section{Bitwise complement}

The unary \xcd"~" operator, only defined on integral types, complements each
bit in its operand.  

\section{Binary arithmetic operations} 

The binary arithmetic operators perform the familiar binary arithmetic
operations: \xcd`+` adds, \xcd`-` subtracts, \xcd`*` multiplies, 
\xcd`/` divides, and \xcd`%`
computes remainder.

On integers, the operands are coerced to the longer of their two types, and
then operated upon.  
Floating point operations are determined by the IEEE 754
standard. 
The integer \xcd"/" and \xcd"%" throw an exception 
if the right operand is zero.



\section{Binary shift operations}

The operands of the binary shift operations must be of integral type.
The type of the result is the type of the left operand.
The right operand, describing a number of bits, must be unsigned: 
%~~exp~~`~~`~~ x:Int ~~ ^^^Expressions1l4m
\xcd`x << 1U`.  


If the promoted type of the left operand is \xcd"Int",
the right operand is masked with \xcd"0x1f" using the bitwise
AND (\xcd"&") operator, giving a number at most the number of bits in an
\xcd`Int`. 
If the promoted type of the left operand is \xcd"Long",
the right operand is masked with \xcd"0x3f" using the bitwise
AND (\xcd"&") operator, giving a number at most the number of bits in a
\xcd`Long`. 

The \xcd"<<" operator left-shifts the left operand by the number of
bits given by the right operand.
The \xcd">>" operator right-shifts the left operand by the number of
bits given by the right operand.  The result is sign extended;
that is, if the right operand is $k$,
the most significant $k$ bits of the result are set to the most
significant bit of the operand.

The \xcd">>>" operator right-shifts the left operand by the number of
bits given by the right operand.  The result is not sign extended;
that is, if the right operand is $k$,
the most significant $k$ bits of the result are set to \xcd"0".
This operation is deprecated, and may be removed in a later version of the
language. 


\section{Binary bitwise operations}

The binary bitwise operations operate on integral types, which are promoted to
the longer of the two types.
The \xcd"&" operator  performs the bitwise AND of the promoted operands.
The \xcd"|" operator  performs the bitwise inclusive OR of the promoted operands.
The \xcd"^" operator  performs the bitwise exclusive OR of the promoted operands.

\section{String concatenation}
\index{string!concatenation}

The \xcd"+"  operator is used for string concatenation 
 as well as addition.
If either operand is of static type \xcd"x10.lang.String",
 the other operand is converted to a \xcd"String" , if needed,
  and  the two strings  are concatenated.
 String conversion of a non-\xcd"null" value is  performed by invoking the
 \xcd"toString()" method of the value.
  If the value is \xcd"null", the value is converted to 
  \xcd'"null"'.

The type of the result is \xcd"String".

 For example, 
%~~exp~~`~~`~~ ~~ ^^^ Expressions100
      \xcd`"one " + 2 + here` 
      evaluates to  \xcd`one 2(Place 0)`.  

\section{Logical negation}

The operand of the  unary \xcd"!" operator 
must be of type \xcd"x10.lang.Boolean".
The type of the result is \xcd"Boolean".
If the value of the operand is \xcd"true", the result is \xcd"false"; if
if the value of the operand  is \xcd"false", the result is \xcd"true".

\section{Boolean logical operations}

Operands of the binary boolean logical operators must be of type \xcd"Boolean".
The type of the result is \xcd"Boolean"

The \xcd"&" operator  evaluates to \xcd"true" if both of its
operands evaluate to \xcd"true"; otherwise, the operator
evaluates to \xcd"false".

The \xcd"|" operator  evaluates to \xcd"false" if both of its
operands evaluate to \xcd"false"; otherwise, the operator
evaluates to \xcd"true".

\section{Boolean conditional operations}

Operands of the binary boolean conditional operators must be of type
\xcd"Boolean". 
The type of the result is \xcd"Boolean"

The \xcd"&&" operator  evaluates to \xcd"true" if both of its
operands evaluate to \xcd"true"; otherwise, the operator
evaluates to \xcd"false".
Unlike the logical operator \xcd"&",
if the first operand is \xcd"false",
the second operand is not evaluated.

The \xcd"||" operator  evaluates to \xcd"false" if both of its
operands evaluate to \xcd"false"; otherwise, the operator
evaluates to \xcd"true".
Unlike the logical operator \xcd"||",
if the first operand is \xcd"true",
the second operand is not evaluated.

\section{Relational operations} 

The relational operations on numeric types compare numbers, producing
\xcd`Boolean` results.

The \xcd"<" operator evaluates to \xcd"true" if the left operand is
less than the right.
The \xcd"<=" operator evaluates to \xcd"true" if the left operand is
less than or equal to the right.
The \xcd">" operator evaluates to \xcd"true" if the left operand is
greater than the right.
The \xcd">=" operator evaluates to \xcd"true" if the left operand is
greater than or equal to the right.

Floating point comparison is determined by the IEEE 754
standard.  Thus,
if either operand is NaN, the result is \xcd"false".
Negative zero and positive zero are considered to be equal.
All finite values are less than positive infinity and greater
than negative infinity.



\section{Conditional expressions}
\index{\Xcd{? :}}
\index{conditional expression}
\index{expression!conditional}
\label{Conditional}

%##(ConditionalExp
\begin{bbgrammar}
%(FROM #(prod:ConditionalExp)#)
      ConditionalExp \: ConditionalOrExp & (\ref{prod:ConditionalExp}) \\
                    \| ClosureExp \\
                    \| AtExp \\
                    \| FinishExp \\
                    \| ConditionalOrExp \xcd"?" Exp \xcd":" ConditionalExp \\
\end{bbgrammar}
%##)

A conditional expression evaluates its first subexpression (the
condition); if \xcd"true"
the second subexpression (the consequent) is evaluated; otherwise,
the third subexpression (the alternative) is evaluated.

The type of the condition must be \xcd"Boolean".
The type of the conditional expression is some common 
ancestor (as constrained by \Sref{LCA}) of the types of the consequent and the
alternative. 

\begin{ex}
%~~exp~~`~~`~~a:Int,b:Int ~~ ^^^ Expressions110
\xcd`a == b ? 1 : 2`
evaluates to \xcd`1` if \xcd`a` and \xcd`b` are the same, and \xcd`2` if they
are different.   As the type of \xcd`1` is \xcd`Int{self==1}` and of \xcd`2`
is \xcd`Int{self==2}`, the type of the conditional expression has the form
\xcd`Int{c}`, where \xcd`self==1` and \xcd`self==2` both imply \xcd`c`.  For
example, it might be \xcd`Int{true}` -- or perhaps it might be 
\xcd`Int{self != 8}`. Note that this term has no most accurate type in the X10
type system.
\end{ex}

The subexpression not selected is not evaluated.

\begin{ex}
The following use of the conditional expression prevents division by zero.  If
\xcd`den==0`, the division is not performed at all.
%~~gen ^^^ Expressions4t3m
% package Expressions4t3m;
% class Hook {
% static def example(num:Int, den:Int ) =
%~~vis
\begin{xten}
(den == 0) ? 0 : num/den
\end{xten}
%~~siv
%; 
% def run() { 
%   return example(1,0) == 0 && example(6,3) == 2;
% } }
%~~neg

Similarly, the following code performs a method call if \xcd`op` is non-null,
and avoids the null pointer error if it is null.  Defensive coding like this
is quite common when working with possibly-null objects.
%~~gen ^^^ Expressions6o2b
% package Expressions6o2b;
% class Hook { 
% static def example(ob:Object) = 
%~~vis
\begin{xten}
(ob == null) ? null : ob.toString();
\end{xten}
%~~siv
%def run() {
%  return example(null) == null && example("yes").equals("yes"); 
% } } 
%~~neg



\end{ex}

\section{Stable equality}
\label{StableEquality}
\index{\Xcd{==}}
\index{equality}

\begin{bbgrammar}
 EqualityExp    \: RelationalExp & (\ref{prod:EqualityExp})\\
%<FROM #(prod:EqualityExp)#
    \| EqualityExp \xcd"==" RelationalExp\\
    \| EqualityExp \xcd"!=" RelationalExp\\
    \| Type  \xcd"==" Type \\
\end{bbgrammar}


The \xcd"==" and \xcd"!=" operators provide a fundamental, though
non-abstract, notion of equality.  \xcd`a==b` is true if the values of \xcd`a`
and \xcd`b` are extremely identical.

\begin{itemize}
\item If \xcd`a` and \xcd`b` are values of object type, then \xcd`a==b` holds
      if \xcd`a` and \xcd`b` are the same object.
\item If one operand is \xcd`null`, then \xcd`a==b` holds iff the other is
      also \xcd`null`.
\item If the operands both have struct type, then they must be structurally equal;
that is, they must be instances of the same struct
and all their fields or components must be \xcd"==". 
\item The definition of equality for function types is specified in
      \Sref{FunctionEquality}.
\item No implicit coercions are performed by \xcd`==`.  
\item It is a static error to have an expression \xcd`a == b` if the types of
      \xcd`a` and \xcd`b` are disjoint.  
\end{itemize}

\xcd`a != b`
is true iff \xcd`a==b` is false.

The predicates \xcd"==" and \xcd"!=" may not be overridden by the programmer.

\xcd`==` provides a {\em stable} notion of equality.  If two values are
\xcd`==` at any time, they remain \xcd`==` forevermore, regardless of what
happens to the mutable state of the program. 

\begin{ex}
Regardless of the values and types of \xcd`a` and \xcd`b`, 
or the behavior of \xcd`any_code_at_all` (which may, indeed, be
any code at all---not just a method call), the value of 
\xcd`a==b` does not change: 
%~~gen ^^^ Expressions1i5k
% package Expressions1i5k;
% class Example{ 
% def example( something: ()=>Int, something_else: ()=>Int,
%   any_code_at_all: () => Int) {
%~~vis
\begin{xten}
val a = something();
val b = something_else();
val eq1 = (a == b);
any_code_at_all();
val eq2 = (a == b);
assert eq1 == eq2;
\end{xten}
%~~siv
%} } 
%~~neg
\end{ex}

\subsection{No Implicit Coercions}
\label{sect:eqeq-no-coerce}

\xcd`==` is a primitive operation in X10 -- one of very few. Most operations,
like \xcd`+` and \xcd`<=`, are defined as \xcd`operator`s. \xcd`==` and
\xcd`!=` are not. As non-\xcd`operator`s, they need not and do not follow the
general method resolution procedure of \Sref{sect:MethodResolution}. In
particular, while \xcd`operator`s perform implicit conversions on their
arguments, \xcd`==` and \xcd`!=` do not.

The advantage of this restriction is that \xcd`==`'s behavior is as simple and
efficient as possible.  It never runs user-defined code, and the compiler can
analyze and understand it in detail -- and guarantee that it is efficient.

The disadvantage is that certain straightforward-looking idioms do not work.
One may not, for example, write
\begin{xten}
// NOT ALLOWED
for(var i : Long = 0; i != 100; i++) 
\end{xten}
A \xcd`Long` like \xcd`i` can never \xcd`==` an \xcd`Int` like \xcd`100`.

We can write \xcd`i = i + 1;`, adding an \xcd`Int` to \xcd`i`. This works 
because the expression uses \xcd`+`,  an ordinary \xcd`operator`.
There is an implicit coercion from \xcd`Int` to \xcd`Long`, so the
\xcd`1` can be converted to \xcd`1L`, which can be added to \xcd`i`.  

However, \xcd`==` does not permit implicit coercions, and so the \xcd`100`
stays an \xcd`Int`.  The loop must be written with a comparison of two
\xcd`Long`s: 
\begin{xten}
for(var i : Long = 0; i != 100L; i++) 
\end{xten}

Incidentally, it could also be written 
\begin{xten}
for(var i : Long = 0; i <= 100; i++) 
\end{xten}
The operation \xcd`<=` is a regular operator, and thus uses coercions in its
arguments, so \xcd`100` gets coerced to \xcd`100L`.  

\subsection{Non-Disjointness Requirement}

It is a static error to have an expression \xcd`a==b` where \xcd`a` and
\xcd`b` could not possibly be equal, based on their types.  This is a
practical codicil to \Sref{sect:eqeq-no-coerce}.  Consider the illegal code 
\begin{xten}
// NOT ALLOWED
for(var i : Long = 0; i != 100; i++) 
\end{xten}

\xcd`100` and \xcd`100L` are different values; they are not \xcd`==`. A
coercion could make them equal, but \xcd`==` does not allow coercions. So, if
\xcd`100 == 100L` were going to return anything, it would have to return
\xcd`false`. This would have the unfortunate effect of making the \xcd`for`
loop diverge.

Since this and related idioms are so common, and since so many programmers are
used to languages which are less precise about their numeric types, X10 avoids
the mistake by declaring it a static error in most cases.  Specifically,
\xcd`a==b` is not allowed if, by inspection of the types, \xcd`a` and \xcd`b`
could not possibly be equal.


\begin{itemize}

\item Numbers of different base types cannot be compared for equality.  
\xcd`100==100L` is a static error.  To compare numbers, explicitly cast them
%~~exp~~`~~`~~ ~~ ^^^Expressions2g6f
to the same type: \xcd`100 as Long == 100L`.

\item Indeed, structs of different types cannot be equal, and so they cannot be
compared for equality.  

\item For objects, the story is different. Unconstrained object types can
      always be compared for equality. Given objects \xcd`a:Person` and
      \xcd`b:Theory`, \xcd`a==b` could be true if \xcd`a==null` and
      \xcd`b==null`. 

\item Constrained object types may or may not be comparable.  For example,  
      if \xcd`Person` and \xcd`Theory` are both direct subclasses of
      \xcd`Object`, and \xcd`a:Person{self!=null}` and \xcd`b:Theory`, then
      \xcd`a==b` is not allowed, since the two could not possibly be equal.

\item Explicit casts erase type information.  If you wanted
      to have a comparison \xcd`a==b` for \xcd`a:Person{self!=null}` and
      \xcd`b:Theory`, you could write it as \xcd`a as Object == b as Object`.
      It would, of course, return \xcd`false`, but it would not be a compiler
      error.\footnote{Code generators often find this trick too be useful.}
      A struct and an object may both be cast to \xcd`Any` and compared for
      equality, though they, too, will always be different.

\end{itemize}





\section{Allocation}
\label{ClassCreation}
\index{new}
\index{allocation}
\index{class!instantation}
\index{class!construction}
\index{struct!instantation}
\index{struct!construction}
\index{instantation}

%##(ClassInstCreationExp
\begin{bbgrammar}
%(FROM #(prod:ClassInstCreationExp)#)
ClassInstCreationExp \: \xcd"new" TypeName TypeArguments\opt \xcd"(" ArgumentList\opt \xcd")" ClassBody\opt & (\ref{prod:ClassInstCreationExp}) \\
                    \| \xcd"new" TypeName \xcd"[" Type \xcd"]" \xcd"[" ArgumentList\opt \xcd"]" \\
                    \| Primary \xcd"." \xcd"new" Id TypeArguments\opt \xcd"(" ArgumentList\opt \xcd")" ClassBody\opt \\
                    \| AmbiguousName \xcd"." \xcd"new" Id TypeArguments\opt \xcd"(" ArgumentList\opt \xcd")" ClassBody\opt \\
\end{bbgrammar}
%##)

An allocation expression creates a new instance of a class and
invokes a constructor of the class.
The expression designates the class name and passes
type and value arguments to the constructor.

The allocation expression may have an optional class body.
In this case, an anonymous subclass of the given class is
allocated.   An anonymous class allocation may also specify a
single super-interface rather than a superclass; the superclass
of the anonymous class is \xcd"x10.lang.Object".

If the class is anonymous---that is, if a class body is
provided---then the constructor is selected from the superclass.
The constructor to invoke is selected using the same rules as
for method invocation (\Sref{MethodInvocation}).

The type of an allocation expression
is the return type of the constructor invoked, with appropriate
substitutions  of actual arguments for formal parameters, as
specified in \Sref{MethodInvocationSubstitution}.

It is illegal to allocate an instance of an \xcd"abstract" class.
The usual visibility rules apply to allocations: 
it is illegal to allocate an instance of a class or to invoke a
constructor that is not visible at
the allocation expression.

Note that instantiating a struct type can use function application syntax; 
\xcd`new` is optional.  As structs do not have subclassing, there is no need or
possibility of a {\em ClassBody}.


\section{Casts}\label{ClassCast}\index{cast}
\index{type conversion}

The cast operation may be used to cast an expression to a given type:

%##(CastExp
\begin{bbgrammar}
%(FROM #(prod:CastExp)#)
             CastExp \: Primary & (\ref{prod:CastExp}) \\
                    \| ExpName \\
                    \| CastExp \xcd"as" Type \\
\end{bbgrammar}
%##)

The result of this operation is a value of the given type if the cast
is permissible at run time, and either a compile-time error or a runtime
exception 
(\xcd`x10.lang.TypeCastException`) if it is not.  

When evaluating \xcd`E as T{c}`, first the value of \xcd`E` is converted to
type \xcd`T` (which may fail), and then the constraint \xcd`{c}` is checked. 



\begin{itemize}
\item If \xcd`T` is a primitive type, then \xcd`E`'s value is converted to type
      \xcd`T` according to the rules of
      \Sref{sec:effects-of-explicit-numeric-coercions}. 
      
\item If \xcd`T` is a class, then the first half of the cast succeeds if the
      run-time value of \xcd`E` is an instance of class \xcd`T`, or of a
      subclass. 

\item If \xcd`T` is an interface, then the first half of the cast succeeds if
      the run-time value of \xcd`E` is an instance of a class implementing
      \xcd`T`. 

\item If \xcd`T` is a struct type, then the first half of the cast succeeds if
      the run-time value of \xcd`E` is an instance of \xcd`T`.  

\item If \xcd`T` is a function type, then the first half of the cast succeeds
      if the run-time value of \xcd`X` is a function of that type, or a
      subtype of it.
\end{itemize}

If the first half of the cast succeeds, the second half -- the constraint
\xcd`{c}` -- must be checked.  In general this will be done at runtime, though
in special cases it can be checked at compile time.   For example, 
\xcd`n as Int{self != w}` succeeds if \xcd`n != w` --- even if \xcd`w` is a value
read from input, and thus not determined at compile time.

The compiler may forbid casts that it knows cannot possibly work. If there is
no way for the value of \xcd`E` to be of type \xcd`T{c}`, then 
\xcd`E as T{c}` can result in a static error, rather than a runtime error.  
For example, \xcd`1 as Int{self==2}` may fail to compile, because the compiler
knows that \xcd`1`, which has type \xcd`Int{self==1}`, cannot possibly be of
type \xcd`Int{self==2}`. 


%BB% \bard{This section need serious whomping.  The Java mention needs to go.  The
%BB% rules for coercions are given in \Sref{sec:effects-of-explicit-numeric-coercions}.
%BB% If the \xcd`Type` has a constraint, the constraint will be checked at runtime. 
%BB% We need to give examples. 
%BB% }
%BB% 
%BB% Type conversion is checked according to the
%BB% rules of the \java{} language (e.g., \cite[\S 5.5]{jls2}).
%BB% For constrained types, both the base
%BB% type and the constraint are checked.
%BB% If the
%BB% value cannot be cast to the appropriate type, a
%BB% \xcd"ClassCastException"
%BB% is thrown. 



% {\bf Conversions of numeric values}
% {\bf Can't do (a as T) if a can't be a T.}


%If the value cannot be cast to the
%appropriate place type a \xcd"BadPlaceException" is thrown. 

% Any attempt to cast an expression of a reference type to a value type
% (or vice versa) results in a compile-time error. Some casts---such as
% those that seek to cast a value of a subtype to a supertype---are
% known to succeed at compile-time. Such casts should not cause extra
% computational overhead at run time.

\section{\Xcd{instanceof}}
\label{instanceOf}
\index{\Xcd{instanceof}}
\index{instanceof}

\Xten{} permits types to be used in an in instanceof expression
to determine whether an object is an instance of the given type:

%##(RelationalExp
\begin{bbgrammar}
%(FROM #(prod:RelationalExp)#)
       RelationalExp \: RangeExp & (\ref{prod:RelationalExp}) \\
                    \| SubtypeConstraint \\
                    \| RelationalExp \xcd"<" RangeExp \\
                    \| RelationalExp \xcd">" RangeExp \\
                    \| RelationalExp \xcd"<=" RangeExp \\
                    \| RelationalExp \xcd">=" RangeExp \\
                    \| RelationalExp \xcd"instanceof" Type \\
                    \| RelationalExp \xcd"in" ShiftExp \\
\end{bbgrammar}
%##)

In the above expression, \grammarrule{Type} is any type. At run time, the
result of \xcd`e instanceof T`
is \xcd"true" if the
value of \xcd`e` is an instance of type \xcd`T`.
Otherwise the result is \xcd"false". This determination may involve checking
that the constraint, if any, associated with the type is true for the given
expression.

%~~exp~~`~~`~~x:Int~~ ^^^ Expressions120
For example, \xcd`3 instanceof Int{self==x}` is an overly-complicated way of
saying \xcd`3==x`.


However, it is a static error if \xcd`e` cannot possibly be an instance of
\xcd`C{c}`; the compiler will reject \xcd`1 instanceof Int{self == 2}` because
\xcd`1` can never satisfy \xcd`Int{self == 2}`. Similarly, \Xcd{1 instanceof
String} is a static error, rather than an expression always returning false. 

\limitationx
X10 does not currently handle \xcd`instanceof` of generics in the way you
%~NO~exp~~`~~`~~r:Array[Int](1) ~~
might expect.  For example, \xcd`r instanceof Array[Int{self != 0}]` does
not test that every element of \xcd`r` is non-zero; instead, the compiler
rejects it.


\section{Subtyping expressions}
\index{\Xcd{<:}}
\index{\Xcd{:>}}
\index{subtype!test}


%##(SubtypeConstraint
\begin{bbgrammar}
%(FROM #(prod:SubtypeConstraint)#)
   SubtypeConstraint \: Type  \xcd"<:" Type  & (\ref{prod:SubtypeConstraint}) \\
                    \| Type  \xcd":>" Type  \\
\end{bbgrammar}
%##)

The subtyping expression \xcdmath"T$_1$ <: T$_2$" evaluates to \xcd"true" if
\xcdmath"T$_1$" is a subtype of \xcdmath"T$_2$".

The expression \xcdmath"T$_1$ :> T$_2$" evaluates to \xcd"true" if
\xcdmath"T$_2$" is a subtype of \xcdmath"T$_1$".

The expression \xcdmath"T$_1$ == T$_2$"
evaluates to  \xcd"true" if 
\xcdmath"T$_1$" is a subtype of \xcdmath"T$_2$" and
if \xcdmath"T$_2$" is a subtype of \xcdmath"T$_1$".

\begin{ex}
Subtyping expressions are particularly useful in giving constraints on generic
types.  \xcd`x10.util.Ordered[T]` is an interface whose values can be compared
with values of type \xcd`T`. 
In particular, \xcd`T <: x10.util.Ordered[T]` is
true if values of type \xcd`T` can be compared to other values of type
\xcd`T`.  So, if we wish to define a generic class \xcd`OrderedList[T]`, of
lists whose elements are kept in the right order, we need the elements to be
ordered.  This is phrased as a constraint on \xcd`T`: 
%~~gen ^^^ Expressions130
% package expre.ssi.onsfgua.rde.dq.uantification;
%~~vis
\begin{xten}
class OrderedList[T]{T <: x10.util.Ordered[T]} {
  // ...
}
\end{xten}
%~~siv
%
%~~neg
\end{ex}


\section{Contains expressions}
\index{in}

\begin{bbgrammar}
       RelationalExp \:RelationalExp \xcd"in" ShiftExp & (\ref{prod:RelationalExp}) \\
\end{bbgrammar}

\xcd`in` is a binary operator, definable in \Sref{sect:operators}.  It is
conventionally used for checking containment.

\begin{ex}
The built-in type \xcd`Region` provides \xcd`in`, testing whether a
\xcd`Point` is in the region: 
%~~gen ^^^ Expressions6d2z
% package Expressions6d2z;
% class Hook { def run() {
%~~vis
\begin{xten}
assert 3 in 1..10;
assert !(10 in 1..3);
\end{xten}
%~~siv
% return true;
%}}
%~~neg

Other types can provide them as well:
%~~gen ^^^ Expressions3c4m
% package Expressions3c4m;
%~~vis
\begin{xten}
class Cont {
   operator this in (Int) = true;
   operator (String) in this = false;
   static operator (Cont) in (b:Boolean) = b;
   static def example() {
      val c:Cont = new Cont();
      assert c in 4 && !("odd" in c) && (c in true);
   }
}
\end{xten}
%~~siv
%class Hook{ def run() { Cont.example(); return true; } }
%~~neg


\end{ex}

\section{Array Constructors}
\label{sect:ArrayCtors}
\index{array!construction}
\index{array!literal}


\begin{bbgrammar}
             Primary \: 
                    \xcd"[" ArgumentList\opt \xcd"]" 
\end{bbgrammar}
%##)

X10 includes short syntactic forms for constructing one-dimensional arrays.
The shortest form is to enclose some expressions in brackets: 
%~~gen ^^^ Expressions140
% package Expressions.ArrayCtor.Primo;
% class Example {
% def example() {
%~~vis
\begin{xten}
val ints <: Array[Int](1) = [1,3,7,21];
\end{xten}
%~~siv
%}}
%~~neg

The expression \xcdmath"[e$_1$, $\ldots$, e$_n$]" produces an \Xcd{n}-element
\xcd`Array[T](1)`, where \xcd`T` is the computed common supertype (\Sref{LCA}) of the {\bf
base types} of the expressions  \xcdmath"e$_i$". 

\begin{ex}
The type of
\xcd`[0,1,2]` is \Xcd{Array[Int](1)}.    
More importantly, the type of 
\xcd`[0]` is also \xcd`Array[Int](1)`.  It is {\em not} 
\xcd`Array[Int{self==0}](1)`, even though all the elements are all 
of type \xcd`Int{self==0}`.  This is subtle but important. There are many
functions that take \xcd`Array[Int](1)`s, such as conversions to \xcd`Point`.
These functions do {\em not} take
\xcd`Array[Int{self==0}]`'s.

(Suppose that the function took \xcd`a:Array[Int](1)` and did 
the operation \xcd`a(i)=100`.   This operation is perfectly fine for
an \xcd`Array[Int](1)`, which is all the compiler knows about \xcd`a`.  
However, it is invalid for an \xcd`Array[Int{self==0}](1)`, because it assigns
a non-zero value to an element of the array, violating the type constraint
which says that all the elements are zero.  So, \xcd`Array[Int{self==0}](1)`
is not and must not be a subtype of \xcd`Array[Int](1)` --- the two types are simply unrelated.
%~~type~~`~~`~~ ~~ ^^^ Expressions150
Since there are far more uses for \xcd`Array[Int](1)` than
%~~type~~`~~`~~ ~~ ^^^ Expressions160
\xcd`Array[Int{self==0}](1)`, the compiler produces the former.)
\end{ex}



\begin{ex}
Occasionally one does actually need \xcd`Array[Int{self==0}](1)`, 
or, say, \xcd`Array[Eel{self != null}](1)`, an array of non-null \xcd`Eel`s.  
For these cases, cast one of the elements of the array to the desired type,
and the array constructor will do the right thing.  
%~~gen ^^^ Expressions170
%package Expressions.ArrayCtor.Details;
%class Eel{}
%class Example{
%def example(){
%~~vis
\begin{xten}
val zero <: Array[Int{self == 0}](1) 
          = [0];
val non1 <: Array[Int{self != 1}](1) 
          = [0 as Int{self != 1}];
val eels <: Array[Eel{self != null}](1) 
          = [ new Eel() as Eel{self != null}, new Eel(), new Eel()];
\end{xten}
%~~siv
%}}
%~~neg
\end{ex}


\section{Coercions and conversions}
\label{XtenConversions}
\label{User-definedCoercions}
\index{conversion}\index{coercion}
\index{type!conversion}\index{type!coercion}

\XtenCurrVer{} supports the following coercions and conversions.

\subsection{Coercions}

%##(CastExp
\begin{bbgrammar}
%(FROM #(prod:CastExp)#)
             CastExp \: CastExp \xcd"as" Type \\
\end{bbgrammar}
%##)


A {\em coercion} does not change object identity; a coerced object may
be explicitly coerced back to its original type through a cast. A {\em
  conversion} may change object identity if the type being converted
to is not the same as the type converted from. \Xten{} permits
user-defined conversions (\Sref{sec:user-defined-conversions}).

\paragraph{Subsumption coercion.}
A value of a subtype may be implicitly coerced to any supertype.  
\index{coercion!subsumption}

\begin{ex}
If \xcd`Child <: Person` and \xcd`val rhys:Child`, then \xcd`rhys` may be used
in any context that expects a \xcd`Person`.  For example, 
%~~gen ^^^ Expressions7f1h
% package Expressions7f1h;
% class Person{}
% class Child extends Person {}
%~~vis
\begin{xten}
class Example {
  def greet(Person) = "Hi!";
  def example(rhys: Child) {
     greet(rhys);
  }
}
\end{xten}
%~~siv
%
%~~neg

Similarly, \xcd`2` (whose innate type is \xcd`Int{self==2}`)
is usable in a context requiring a non-zero integer
(\xcd`Int{self != 0}`).  
\end{ex}

\paragraph{Explicit Coercion (Casting with \xcd"as")}

All classes and interfaces allow the use of the \xcd`as` operator for explicit
type coercion.  
Any class or
interface may be cast to any interface.  
Any interface may be cast to
any class.  Also, any interface can be cast to a struct that implements
(directly or indirectly) that interface.

\begin{ex}
In the following code, a \xcd`Person` is cast to \xcd`Childlike`.  There is
nothing in the class definition of \xcd`Person` that suggests that a
\xcd`Person` can be \xcd`Childlike`.  However, the \xcd`Person` in question,
\xcd`p`, is actually a \xcd`HappyChild` --- a subclass of \xcd`Person` --- and
is, in fact, \xcd`Childlike`.  

Similarly, the \xcd`Childlike` value \xcd`cl` is cast to \xcd`Happy`.  Though
these two interfaces are unrelated, the value of \xcd`cl` is, in fact,
\xcd`Happy`.  And the \xcd`Happy` value \xcd`hc` is cast to the class
\xcd`Child`, though there is no relationship between the two, but the actual
value is a \xcd`HappyChild`, and thus the cast is correct at runtime.

\xcd`Cyborg` is a struct rather than a class.  So, it cannot have substructs,
and all the interfaces of all \xcd`Cyborg`s are known: a \xcd`Cyborg` is
\xcd`Personable`, but not \xcd`Childlike` or \xcd`Happy`.  So, it is correct
and meaningful to cast \xcd`r` to \xcd`Personable`.  There is no way that a
cast to \xcd`Childlike` could succeed, so \xcd`r as Childlike` is a static error.

%~~gen ^^^ Expressions180
% package Types.Coercions;
%~~vis
\begin{xten}
interface Personable {}
class Person implements Personable {}
interface Childlike extends Personable {}
class Child extends Person implements Childlike {}
struct Cyborg implements Personable {}
interface Happy {}
class HappyChild extends Child implements Happy {}
class Example {
  static def example() {
    var p : Person = new HappyChild();
    val cl : Childlike = p as Childlike; // class -> interface
    val hc : Happy = cl as Happy; //        interface -> interface
    val ch : Child = hc as Child; //        interface -> class
    var r : Cyborg = Cyborg();
    val rl : Personable = r as Personable; 
    // ERROR: r as Childlike
  }
}
\end{xten}
%~~siv
% class Hook{ def run(){ Example.example(); return true; } }
%~~neg




\end{ex}


If the value coerced is not an instance of the target type,
and no coercion operators that can convert it to that type are defined, 
a \xcd"ClassCastException" is thrown.  Casting to a constrained
type may require a run-time check that the constraint is
satisfied.
\index{coercion!explicit}
\index{cast}
\index{\Xcd{as}}

\limitation{It is currently a static error, rather than the specified
\xcd`ClassCastException`, when the cast is statically determinable to be
impossible.}



\paragraph{Effects of explicit numeric coercion}
\label{sec:effects-of-explicit-numeric-coercions}

Coercing a number of one type to another type gives the best approximation of
the number in the result type, or a suitable disaster value if no
approximation is good enough.  

\begin{itemize}
\item Casting a number to a {\em wider} numeric type is safe and effective,
      and can be done by an implicit conversion as well as an explicit
%~~exp~~`~~`~~ ~~ ^^^ Expressions190
      coercion.  For example, \xcd`4 as Long` produces the \xcd`Long` value of
      4. 
\item Casting a floating-point value to an integer value truncates the digits
      after the decimal point, thereby rounding the number towards zero.  
%~~exp~~`~~`~~ ^^^ Expressions200
      \xcd`54.321 as Int` is \xcd`54`, and 
%~~exp~~`~~`~~ ~~ ^^^ Expressions210
      \xcd`-54.321 as Int` is \xcd`-54`.
      If the floating-point value is too large to represent as that kind of
      integer, the coercion returns the largest or smallest value of that type
      instead: \xcd`1e110 as Int` is 
      \xcd`Int.MAX_VALUE`, \viz{} \xcd`2147483647`. 

\item Casting a \xcd`Double` to a \xcd`Float` normally truncates binary digits: 
%~~exp~~`~~`~~ ~~ ^^^ Expressions220
      \xcd`0.12345678901234567890 as Float` is approximately \xcd`0.12345679f`.  This can
      turn a nonzero \xcd`Double` into \xcd`0.0f`, the zero of type
      \xcd`Float`: 
%~~exp~~`~~`~~ ~~ ^^^ Expressions230
      \xcd`1e-100 as Float` is \xcd`0.0f`.  Since 
      \xcd`Double`s can be as large as about \xcd`1.79E308` and \xcd`Float`s
      can only be as large as about \xcd`3.4E38f`, a large \xcd`Double` will
      be converted to the special \xcd`Float` value of \xcd`Infinity`: 
%~~exp~~`~~`~~ ~~ ^^^ Expressions240
      \xcd`1e100 as Float` is \xcd`Infinity`.
\item Integers are coerced to smaller integer types by truncating the
      high-order bits. If the value of the large integer fits into the smaller
      integer's range, this gives the same number in the smaller type: 
%~~exp~~`~~`~~ ~~ ^^^ Expressions250
      \xcd`12 as Byte` is the \xcd`Byte`-sized 12, 
%~~exp~~`~~`~~ ~~ ^^^ Expressions260
      \xcd`-12 as Byte` is -12. 
      However, if the larger integer {\em doesn't} fit in the smaller type,
%~~exp~~`~~`~~ ~~ ^^^ Expressions270
      the numeric value and even the sign can change: \xcd`254 as Byte` is
      the \xcd`Byte`sized \xcd`-2y`.  

\item Casting an unsigned integer type to a signed integer type of the same
      size (\eg, \xcd`UInt` to \xcd`Int`) preserves 2's-complement bit pattern
      (\eg,  
      \xcd`UInt.MAX_VALUE as Int == -1`.   Casting an unsigned integer type to
      a signed integer type of a different size is equivalent to first casting
      to an unsigned integer type of the target size, and then casting to a
      signed integer type.

\item Casting a signed integer type to an unsigned one is similar.  

\end{itemize}

\subsubsection{User-defined Coercions}
\index{coercion!user-defined}

Users may define coercions from arbitrary types into the container type
\xcd`B`, and coercions from \xcd`B` to arbitrary types, by providing
\xcd`static operator` definitions for the \xcd`as` operator in the definition of
\xcd`B`.  

\begin{ex}

%~~gen ^^^ Expressions2j7z
% package Expressions2j7z;
% KNOWNFAIL
%~~vis
\begin{xten}
class Bee {
  public static operator (x:Bee) as Int = 1;
  public static operator (x:Int) as Bee = new Bee();
  def example() {
    val b:Bee = 2 as Bee; 
    assert (b as Int) == 1;
  }
}
\end{xten}
%~~siv
%
%~~neg


\end{ex}



\subsection{Conversions}
\index{conversion}
\index{type!conversion}

\paragraph{Widening numeric conversion.}
\label{WideningConversions}
A numeric type may be implicitly converted to a wider numeric type. In
particular, an implicit conversion may be performed between a numeric
type and a type to its right, below:

\begin{xten}
Byte < Short < Int < Long < Float < Double
UByte < UShort < UInt < ULong
\end{xten}

Furthermore, an unsigned integer type may be implicitly coerced a signed type
large 
enough to hold any value of the type: \xcd`UByte` to \xcd`Short`, \xcd`UShort`
to \xcd`Int`, \xcd`UInt` to \xcd`Long`.  There are no implicit conversions
from signed to unsigned numbers, since they cannot treat negatives properly.

There are no implicit conversions in cases when overflow is possible.  For
example, there is no implicit conversion between \xcd`Int` and \xcd`UInt`.  If
it is necessary to convert between these types, use \xcd`n as Int` or 
\xcd`n as UInt`, generally with a test to ensure that the value will fit and
code to handle the case in which it does not.  


\index{conversion!widening}
\index{conversion!numeric}

\paragraph{String conversion.}
Any value that is an operand of the binary
\xcd"+" operator may
be converted to \xcd"String" if the other operand is a \xcd"String".
A conversion to \xcd"String" is performed by invoking the \xcd"toString()"
method.

\index{conversion!string}

\paragraph{User defined conversions.}\label{sec:user-defined-conversions}
\index{conversion!user-defined}

The user may define implicit conversion operators from type \Xcd{A} {\em to} a
container type \Xcd{B} by specifying an operator in \Xcd{B}'s definition of the form:

\begin{xten}
  public static operator (r: A): T = ... 
\end{xten}

The return type \Xcd{T} should be a subtype of \Xcd{B}. The return
type need not be specified explicitly; it will be computed in the
usual fashion if it is not. However, it is good practice for the
programmer to specify the return type for such operators explicitly.
The return type can be more specific than simply \xcd`B`, for cases when there
is more information available.


\begin{ex}
The code for \Xcd{x10.lang.Point} contains a conversion from 
one-dimensional \xcd`Array`s of integers to \xcd`Point`s of the same length: 
\begin{xten}
  public operator (r: Array[Int](1)): Point(r.length) = make(r);
\end{xten}
This conversion is used whenever an array of integers appears in a 
context that requires a \xcd`Point`, such as subscripting. Note 
that \xcd`a` requires a \xcd`Point` of rank 2 as a subscript, and that 
a two-element \xcd`Array` (like \xcd`[2,4]`) is converted to a 
\xcd`Point(2)`.
%~~gen ^^^ Expressions4f4y
% package Expressions4f4y;
% class Example { def example() {
%~~vis
\begin{xten}
val a = new Array[String]((2..3) * (4..5), "hi!");
a([2,4]) = "converted!";
\end{xten}
%~~siv
%} } 
%~~neg


\end{ex}

\section{Parenthesized Expressions}

If \xcd`E` is any expression, \xcd`(E)` is an expression which, when
evaluated, produces the same result as \xcd`E`.   

\begin{ex}
The main use of parentheses is to write complex expressions for which the 
standard precedence order of operations is not appropriate: \xcd`1+2*3` is 7,
but \xcd`(1+2)*3` is 9.  

Similarly, but perhaps less familiarly, 
parentheses can disambiguate other expressions.  In the following code, 
\xcd`funny.f` is a field-selection expression, and so \xcd`(funny.f)()` means
``select the \xcd`f` field from \xcd`funny`, and evaluate it''.  However, 
\xcd`funny.f()` means ``evaluate the \xcd`f` method on object \xcd`funny`.''  
%~~gen ^^^ Expressions3f6f
% package Expressions3f6f;
%~~vis
\begin{xten}
class Funny {
  def f () = 1;
  val f = () => 2;
  static def example() {
    val funny = new Funny();
    assert funny.f() == 1;
    assert (funny.f)() == 2;
  }
}
\end{xten}
%~~siv
% class Hook{ def run() { Funny.example(); return true; }}
%~~neg


\end{ex}

Note that this does {\em
not} mean that \xcd`E` and \xcd`(E)` are identical in all respects; for
example, if \xcd`i` is an \xcd`Int` variable, \xcd`i++` increments \xcd`i`,
but \xcd`(i)++` is not allowed.    \xcd`++` is an assignment; it operates on
variables, not merely values, and \xcd`(i)` is simply an expression whose {\em
value} is the same as that of \xcd`i`. 
	
\chapter{Statements}\label{XtenStatements}\index{statement}

This chapter describes the statements in the sequential core of
\Xten{}.  Statements involving concurrency and distribution
are described in \Sref{XtenActivities}.

\section{Empty statement}

The empty statement \xcd";" does nothing.  

\begin{ex}
Sometimes, the syntax of X10 requires a statement in some position, but you do
not actually want to do any computation there.   
The following code searches the array \xcd`a` for the value \xcd`v`, assumed
to appear somewhere in \xcd`a`, and returns the index at which it was found.  
There is no computation to do in the loop body, so we use an empty statement
there. 
%~~gen ^^^ Statements10
% package statements.emptystatement;
% class EmptyStatementExample {
%~~vis
\begin{xten}
static def search[T](a: Array[T](1), v: T):Int {
  var i : Int;
  for(i = a.region.min(0); a(i) != v; i++)
     ;
  return i;
}
\end{xten}
%~~siv
%}
%~~neg

\end{ex}

\section{Local variable declaration}
\label{sect:LocalVarDecl}
\index{variable!declaration}
\index{var}
\index{val}

%##(LocVarDecl LocVarDeclStmt VarDeclWType VarDeclsWType VariableDeclarators VariableInitializer FormalDeclarators
\begin{bbgrammar}
%(FROM #(prod:LocVarDecl)#)
          LocVarDecl \: Mods\opt VarKeyword VariableDeclarators & (\ref{prod:LocVarDecl}) \\
                     \| Mods\opt VarDeclsWType \\
                     \| Mods\opt VarKeyword FormalDeclarators \\
%(FROM #(prod:LocVarDeclStmt)#)
      LocVarDeclStmt \: LocVarDecl \xcd";" & (\ref{prod:LocVarDeclStmt}) \\
%(FROM #(prod:VarDeclWType)#)
        VarDeclWType \: Id HasResultType \xcd"=" VariableInitializer & (\ref{prod:VarDeclWType}) \\
                     \| \xcd"[" IdList \xcd"]" HasResultType \xcd"=" VariableInitializer \\
                     \| Id \xcd"[" IdList \xcd"]" HasResultType \xcd"=" VariableInitializer \\
%(FROM #(prod:VarDeclsWType)#)
       VarDeclsWType \: VarDeclWType & (\ref{prod:VarDeclsWType}) \\
                     \| VarDeclsWType \xcd"," VarDeclWType \\
%(FROM #(prod:VariableDeclarators)#)
 VariableDeclarators \: VariableDeclarator & (\ref{prod:VariableDeclarators}) \\
                     \| VariableDeclarators \xcd"," VariableDeclarator \\
%(FROM #(prod:VariableInitializer)#)
 VariableInitializer \: Exp & (\ref{prod:VariableInitializer}) \\
%(FROM #(prod:FormalDeclarators)#)
   FormalDeclarators \: FormalDeclarator & (\ref{prod:FormalDeclarators}) \\
                     \| FormalDeclarators \xcd"," FormalDeclarator \\
\end{bbgrammar}
%##)

Short-lived variables are introduced by local variables declarations, as
described in \Sref{VariableDeclarations}. Local variables may be declared only
within a block statement (\Sref{Blocks}). The scope of a local variable
declaration is the subsequent statements in the
block.   
%~~gen ^^^ Statements20
% package statements.should.have.locals;
% class LocalExample {
% def example(a:Int) {
%~~vis
\begin{xten}
  if (a > 1) {
     val b = a/2;
     var c : Int = 0;
     // b and c are defined here
  }
  // b and c are not defined here.
\end{xten}
%~~siv
%} }
%~~neg

Variables declared in such statements shadow variables of the same
name declared elsewhere.
A local variable of a given name, say \xcd`x`, cannot shadow another local
variable or parameter named \xcd`x` unless there is an intervening method,
constructor, initializer, or
closure declaration.
%%, or unless the inner \xcd`x` is declared inside an
%%\xcd`async` or \xcd`at` statement and the outer variable is declared outside
%% of that.   Strictly, \xcd`at` introduces a new scope that does not share the
%%variables of the external scope, so its variables do not actually shadow those
%%outside.  

\begin{ex}
The following code illustrates both legal and illegal uses of shadowing.
Note that a shadowed {\em field} name \xcd`x` can still be accessed 
as \xcd`this.x`. 

%%AT-COPY%% %~~gen ^^^ Statements4h6p
%%AT-COPY%% % package Statements4h6p;
%%AT-COPY%% %~~vis
%%AT-COPY%% \begin{xten}
%%AT-COPY%% class Shadow{
%%AT-COPY%%   var x : Int; 
%%AT-COPY%%   def this(x:Int) { 
%%AT-COPY%%      // Parameter can shadow field
%%AT-COPY%%      this.x = x; 
%%AT-COPY%%   }
%%AT-COPY%%   def example(y:Int) {
%%AT-COPY%%      val x = "shadows a field";
%%AT-COPY%%      // ERROR: val y = "shadows a param";
%%AT-COPY%%      val z = "local";
%%AT-COPY%%      for (a in [1,2,3]) {
%%AT-COPY%%         // ERROR: val x = "can't shadow local var";
%%AT-COPY%%      }
%%AT-COPY%%      async {
%%AT-COPY%%         val x = "can shadow through async";
%%AT-COPY%%      }        
%%AT-COPY%%      at (here;) {
%%AT-COPY%%         val x = "at gives a whole new namespace";
%%AT-COPY%%      }        
%%AT-COPY%%      val f = () => { 
%%AT-COPY%%        val x = "can shadow through closure";
%%AT-COPY%%        x
%%AT-COPY%%      };
%%AT-COPY%%   }
%%AT-COPY%% }
%%AT-COPY%% \end{xten}
%%AT-COPY%% %~~siv
%%AT-COPY%% %
%%AT-COPY%% %~~neg
%%AT-COPY%% 
%~~gen ^^^ Statements4h6p
% package Statements4h6p;
% // NOTEST-stupid-packaging-issue
%~~vis
\begin{xten}
class Shadow{
  var x : Int; 
  def this(x:Int) { 
     // Parameter can shadow field
     this.x = x; 
  }
  def example(y:Int) {
     val x = "shadows a field";
     // ERROR: val y = "shadows a param";
     val z = "local";
     for (a in [1,2,3]) {
        // ERROR: val x = "can't shadow local var";
     }
     async {
        // ERROR: val x = "can't shadow through async";
     }        
     val f = () => { 
       val x = "can shadow through closure";
       x
     };
     class Local {
        val f = at(here.next()){ val x = "can here"; x };
        def this() { val x = "can here, too"; }
     }
  }
}
\end{xten}
%~~siv
%
%~~neg



\end{ex}

\begin{ex}
Note that recursive definitions of local variables is not allowed.  There are
few useful recursive declarations of objects and structs; \xcd`x`, in the
following example, has no meaningful definition.  Recursive declarations of
local functions is forbidden, even though (like \xcd`f` below) there are
meaningful uses of it.  
\begin{xten}
val x : Int = x + 1; // ERROR: recursive local declaration
val f : (Int)=>Int 
      = (n:Int) => (n <= 2) ? 1 : f(n-1) + f(n-2);
      // ERROR: recursive local declaration
\end{xten}

\end{ex}



\section{Block statement}
\index{block}
\label{Blocks}

%##(Block BlockStatements BlockStatement
\begin{bbgrammar}
%(FROM #(prod:Block)#)
               Block \: \xcd"{" BlockStatements\opt \xcd"}" & (\ref{prod:Block}) \\
%(FROM #(prod:BlockStatements)#)
     BlockStatements \: BlockStatement & (\ref{prod:BlockStatements}) \\
                     \| BlockStatements BlockStatement \\
%(FROM #(prod:BlockStatement)#)
      BlockStatement \: LocVarDeclStmt & (\ref{prod:BlockStatement}) \\
                     \| ClassDecl \\
                     \| TypeDefDecl \\
                     \| Statement \\
\end{bbgrammar}
%##)


A block statement consists of a sequence of statements delimited by
``\xcd"{"'' and ``\xcd"}"''. When a block is evaluated, the statements inside
of it are evaluated in order.  Blocks are useful for putting several
statements in a place where X10 asks for a single one, such as the consequent
of an \xcd`if`, and for limiting the scope of local variables.
%~~gen ^^^ Statements30
% package statements.FOR.block.heads;
% class Example {
% def example(b:Boolean, S1:(Int)=>void, S2:(Int)=>void ) {
%~~vis
\begin{xten}
if (b) {
  // This is a block
  val v = 1;
  S1(v); 
  S2(v);
}
\end{xten}
%~~siv
%  } } 
%~~neg



\section{Expression statement}

Any expression may be used as a statement.

%##(ExpStatement StatementExp
\begin{bbgrammar}
%(FROM #(prod:ExpStatement)#)
        ExpStatement \: StatementExp \xcd";" & (\ref{prod:ExpStatement}) \\
%(FROM #(prod:StatementExp)#)
        StatementExp \: Assignment & (\ref{prod:StatementExp}) \\
                     \| PreIncrementExp \\
                     \| PreDecrementExp \\
                     \| PostIncrementExp \\
                     \| PostDecrementExp \\
                     \| MethodInvocation \\
                     \| ClassInstCreationExp \\
\end{bbgrammar}
%##)

The expression statement evaluates an expression. The value of the expression
is not used. Side effects of the expression occur, and may produce results
used by following statements. Indeed, statement expressions which terminate
without side effects cannot have any visible effect on the results of the
computation. 


\begin{ex}
%~~gen ^^^ Statements40
% package Sta.tem.ent.s.expressions;
% import x10.util.*;
%~~vis
\begin{xten}
class StmtEx {
  def this() { 
     x10.io.Console.OUT.println("New StmtEx made");  }
  static def call() { 
     x10.io.Console.OUT.println("call!");}
  def example() {
     var a : Int = 0;
     a = 1; // assignment
     new StmtEx(); // allocation
     call(); // call
  }
}
\end{xten}
%~~siv
%
%~~neg
\end{ex}



\section{Labeled statement}
\index{label}
\index{statement label}


\begin{bbgrammar}
    LabeledStatement \: Id \xcd":" Statement 
\end{bbgrammar}


Statements may be labeled. The label may be used to describe the target of a
\xcd`break` statement appearing within a substatement (which, when executed,
ends the labeled statement), or, in the case of a loop, a \xcd`continue` as
well (which, when executed, proceeds to the next iteration of the loop). The
scope of a label is the statement labeled.

\begin{ex}
The label on the outer \xcd`for` statement allows \xcd`continue` and
\xcd`break` statements to continue or break it.  Without the label,
\xcd`continue` or \xcd`break` would only continue or break the inner \xcd`for`
loop. 
%~~gen ^^^ Statements50
% package state.meant.labe.L;
% class Example {
% def example(a:(Int,Int) => Int, do_things_to:(Int)=>Int) {
%~~vis
\begin{xten}
lbl : for (i in 1..10) {
   for (j in i..10) {  
      if (a(i,j) == 0) break lbl;
      if (a(i,j) == 1) continue lbl;
      if (a(i,j) == a(j,i)) break lbl;
   }
}
\end{xten}
%~~siv
%} } 
%~~neg
\end{ex}

In particular, a block statement may be labeled: \xcd` L:{S}`.  This allows
the use of \xcd`break L` within \xcd`S` to leave \xcd`S`, which can, if
carefully used, avoid deeply-nested \xcd`if`s. 

\begin{ex}
%~~gen ^^^ Statements51
% package statements51;
% abstract class Example {
% abstract def phase1(String):void;
% abstract def phase2(String):void;
% abstract def phase3(String):void;
% abstract def suitable_for_phase_2(String):Boolean;
% abstract def suitable_for_phase_3(String):Boolean;
% def example(filename: String) {
% KNOWNFAIL-labelled-blocks
%~~vis
\begin{xten}
multiphase: {
  if (!exists(filename)) break multiphase;
  phase1(filename);
  if (!suitable_for_phase_2(filename)) break multiphase;
  phase2(filename);
  if (!suitable_for_phase_3(filename)) break multiphase;
  phase3(filename);
}
// Now the file has been phased as much as possible
\end{xten}
%~~siv
%}
%~~neg
\end{ex}

\limitation{Blocks cannot currently be labeled.}

\section{Break statement}
\index{break}

%##(BreakStatement
\begin{bbgrammar}
%(FROM #(prod:BreakStatement)#)
      BreakStatement \: \xcd"break" Id\opt \xcd";" & (\ref{prod:BreakStatement}) \\
\end{bbgrammar}
%##)


An unlabeled break statement exits the currently enclosing loop or switch
statement. A labeled break statement exits the enclosing 
statement with the given label.
It is illegal to break out of a statement not defined in the current
method, constructor, initializer, or closure.  
\xcd`break` is only allowed in sequential code.

\begin{ex}
The following code searches for an element of a two-dimensional
array and breaks out of the loop when it is found:
%~~gen ^^^ Statements60
% package statements.come.from.banks.and.cranks;
% class LabelledBreakeyBreakyHeart {
% def findy(a:Array[Array[Int](1)](1), v:Int): Boolean {
%~~vis
\begin{xten}
var found: Boolean = false;
outer: for (var i: Int = 0; i < a.size; i++)
    for (var j: Int = 0; j < a(i).size; j++)
        if (a(i)(j) == v) {
            found = true;
            break outer;
        }
\end{xten}
%~~siv
% return found;
%}}
%~~neg
\end{ex}

\section{Continue statement}
\index{continue}

%##(ContinueStatement
\begin{bbgrammar}
%(FROM #(prod:ContinueStatement)#)
   ContinueStatement \: \xcd"continue" Id\opt \xcd";" & (\ref{prod:ContinueStatement}) \\
\end{bbgrammar}
%##)
An unlabeled \xcd`continue` skips the rest of the current iteration of the
innermost enclosing loop, and proceeds on to the next.  A labeled
\xcd`continue` does the same to the enclosing loop with that label.
It is illegal to continue a loop not defined in the current
method, constructor, initializer, or closure.
\xcd`continue` is only allowed in sequential code.



\section{If statement}
\index{if}

%##(IfThenStatement IfThenElseStatement
\begin{bbgrammar}
%(FROM #(prod:IfThenStatement)#)
     IfThenStatement \: \xcd"if" \xcd"(" Exp \xcd")" Statement & (\ref{prod:IfThenStatement}) \\
%(FROM #(prod:IfThenElseStatement)#)
 IfThenElseStatement \: \xcd"if" \xcd"(" Exp \xcd")" Statement  \xcd"else" Statement  & (\ref{prod:IfThenElseStatement}) \\
\end{bbgrammar}
%##)

An if statement comes in two forms: with and without an else
clause.

The if-then statement evaluates a condition expression, which must be of type
\xcd`Boolean`. If the condition is \xcd`true`, it evaluates the then-clause.
If the condition is \xcd"false", the if-then statement completes normally.

The if-then-else statement evaluates a \xcd`Boolean` expression and 
evaluates the then-clause if the condition is
\xcd"true"; otherwise, the \xcd`else`-clause is evaluated.

As is traditional in languages derived from Algol, the if-statement is syntactically
ambiguous.  That is, 
\begin{xten}
if (B1) if (B2) S1 else S2
\end{xten}
could be intended to mean either 
\begin{xten}
if (B1) { if (B2) S1 else S2 }
\end{xten} 
or 
\begin{xten}
if (B1) {if (B2) S1} else S2
\end{xten}
X10, as is traditional, attaches an \xcd`else` clause to the most recent
\xcd`if` that doesn't have one.
This example is interpreted as 
\xcd`if (B1) { if (B2) S1 else S2 }`. 



\section{Switch statement}
\index{switch}

%##(SwitchStatement SwitchBlock SwitchBlockGroups SwitchBlockGroup SwitchLabels SwitchLabel
\begin{bbgrammar}
%(FROM #(prod:SwitchStatement)#)
     SwitchStatement \: \xcd"switch" \xcd"(" Exp \xcd")" SwitchBlock & (\ref{prod:SwitchStatement}) \\
%(FROM #(prod:SwitchBlock)#)
         SwitchBlock \: \xcd"{" SwitchBlockGroups\opt SwitchLabels\opt \xcd"}" & (\ref{prod:SwitchBlock}) \\
%(FROM #(prod:SwitchBlockGroups)#)
   SwitchBlockGroups \: SwitchBlockGroup & (\ref{prod:SwitchBlockGroups}) \\
                     \| SwitchBlockGroups SwitchBlockGroup \\
%(FROM #(prod:SwitchBlockGroup)#)
    SwitchBlockGroup \: SwitchLabels BlockStatements & (\ref{prod:SwitchBlockGroup}) \\
%(FROM #(prod:SwitchLabels)#)
        SwitchLabels \: SwitchLabel & (\ref{prod:SwitchLabels}) \\
                     \| SwitchLabels SwitchLabel \\
%(FROM #(prod:SwitchLabel)#)
         SwitchLabel \: \xcd"case" ConstantExp \xcd":" & (\ref{prod:SwitchLabel}) \\
                     \| \xcd"default" \xcd":" \\
\end{bbgrammar}
%##)

A switch statement evaluates an index expression and then branches to
a case whose value is equal to the value of the index expression.
If no such case exists, the switch branches to the 
\xcd"default" case, if any.

Statements in each case branch are evaluated in sequence.  At the
end of the branch, normal control-flow falls through to the next case, if
any.  To prevent fall-through, a case branch may be exited using
a \xcd"break" statement.

The index expression must be of type \xcd"Int".
Case labels must be of type \xcd"Int", \xcd`Byte`, or \xcd`Short`, 
and must be compile-time 
constants.  Case labels cannot be duplicated within the
\xcd"switch" statement.

\begin{ex}
In this \xcd`switch`, case \xcd`1` falls through to case \xcd`2`.  The
other cases are separated by \xcd`break`s.
%~~gen ^^^ Statements70
% package Statement.Case;
% class Example {
% def example(i : Int, println: (String)=>void) {
%~~vis
\begin{xten}
switch (i) {
  case 1: println("one, and ");
  case 2: println("two"); 
          break;
  case 3: println("three");
          break;
  default: println("Something else");
           break;
}
\end{xten}
%~~siv
% } } 
%~~neg
\end{ex}

\section{While statement}
\index{while}

%##(WhileStatement
\begin{bbgrammar}
%(FROM #(prod:WhileStatement)#)
      WhileStatement \: \xcd"while" \xcd"(" Exp \xcd")" Statement & (\ref{prod:WhileStatement}) \\
\end{bbgrammar}
%##)

A while statement evaluates a \xcd`Boolean`-valued condition and executes a
loop body if \xcd"true". If the loop body completes normally (either by
reaching the end or via a \xcd"continue" statement with the loop header as
target), the condition is reevaluated and the loop repeats if \xcd"true". If
the condition is \xcd"false", the loop exits.

\begin{ex}
A loop to execute the process in the Collatz conjecture (a.k.a. 3n+1 problem,
Ulam conjecture, Kakutani's problem, Thwaites conjecture, Hasse's algorithm,
and Syracuse problem) can be written as follows:
%~~gen ^^^ Statements80
% package Statements.AreFor.Flatements;
% class Example {
% def example(var n:Int) {
%~~vis
\begin{xten}
  while (n > 1) {
     n = (n % 2 == 1) ? 3*n+1 : n/2;
  }
\end{xten}
%~~siv
% } } 
%~~neg
\end{ex}
\section{Do--while statement}
\index{do}

%##(DoStatement
\begin{bbgrammar}
%(FROM #(prod:DoStatement)#)
         DoStatement \: \xcd"do" Statement \xcd"while" \xcd"(" Exp \xcd")" \xcd";" & (\ref{prod:DoStatement}) \\
\end{bbgrammar}
%##)


A \Xcd{do-while} statement executes the loop body, and then evaluates a
\xcd`Boolean`-valued condition expression. If \xcd"true", the loop repeats.
Otherwise, the loop exits.


\section{For statement}
\index{for}

%##(ForStatement BasicForStatement ForInit ForUpdate StatementExpList EnhancedForStatement
\begin{bbgrammar}
%(FROM #(prod:ForStatement)#)
        ForStatement \: BasicForStatement & (\ref{prod:ForStatement}) \\
                     \| EnhancedForStatement \\
%(FROM #(prod:BasicForStatement)#)
   BasicForStatement \: \xcd"for" \xcd"(" ForInit\opt \xcd";" Exp\opt \xcd";" ForUpdate\opt \xcd")" Statement & (\ref{prod:BasicForStatement}) \\
%(FROM #(prod:ForInit)#)
             ForInit \: StatementExpList & (\ref{prod:ForInit}) \\
                     \| LocVarDecl \\
%(FROM #(prod:ForUpdate)#)
           ForUpdate \: StatementExpList & (\ref{prod:ForUpdate}) \\
%(FROM #(prod:StatementExpList)#)
    StatementExpList \: StatementExp & (\ref{prod:StatementExpList}) \\
                     \| StatementExpList \xcd"," StatementExp \\
%(FROM #(prod:EnhancedForStatement)#)
EnhancedForStatement \: \xcd"for" \xcd"(" LoopIndex \xcd"in" Exp \xcd")" Statement & (\ref{prod:EnhancedForStatement}) \\
                     \| \xcd"for" \xcd"(" Exp \xcd")" Statement \\
\end{bbgrammar}
%##)

\xcd`for` statements provide bounded iteration, such as looping over a list.
It has two forms: a basic form allowing near-arbitrary iteration, {\em a la}
C, and an enhanced form designed to iterate over a collection.

A basic \xcd`for` statement provides for arbitrary iteration in a somewhat
more organized fashion than a \xcd`while`.  The loop 
\xcd`for(init; test; step)body` is
similar to: 
\begin{xten}
{
   init;
   while(test) {
      body;
      step;
   }
}
\end{xten}
\noindent
except that \xcd`continue` statements which continue the \xcd`for` loop will
perform the \xcd`step`, which, in the \xcd`while` loop, they will not do. 

\xcd`init` is performed before the loop, and is traditionally used to declare
and/or initialize the loop variables. It may be a single variable binding
statement, such as \xcd`var i:Int = 0` or \xcd`var i:Int=0, j:Int=100`. (Note
that a single variable binding statement may bind multiple variables.)
Variables introduced by \xcd`init` may appear anywhere in the \xcd`for`
statement, but not outside of it.  Or, it may be a sequence of expression
statements, such as \xcd`i=0, j=100`, operating on already-defined variables.
If omitted, \xcd`init` does nothing.

\xcd`test` is a Boolean-valued expression; an iteration of the loop will only
proceed if \xcd`test` is true at the beginning of the loop, after \xcd`init`
on the first iteration or after \xcd`step` on later ones. If omitted, \xcd`test`
defaults to \xcd`true`, giving a loop that will run until stopped by some
other means such as \xcd`break`, \xcd`return`, or \xcd`throw`.

\xcd`step` is performed after the loop body, between one iteration and the
next. It traditionally updates the loop variables from one iteration to the
next: \eg, \xcd`i++` and \xcd`i++,j--`.  If omitted, \xcd`step` does nothing.

\xcd`body` is a statement, often a code block, which is performed whenever
\xcd`test` is true.  If omitted, \xcd`body` does nothing.




\label{ForAllLoop}


An enhanced for statement is used to iterate over a collection, or other
structure designed to support iteration by implementing the interface
\xcd`Iterable[T]`.    The loop variable must be of type \xcd`T`, 
or destructurable from a value of type \xcd`T`
(\Sref{exploded-syntax}).  
Each iteration of the loop
binds the iteration variable to another element of the collection.
The loop \xcd`for(x in c)S` behaves like: 
%~~gen ^^^ Statements5e4u
% package Statements5e4u;
% class ForAll {
% def forall[T](c:Iterable[T], S: () => void) {
%~~vis
\begin{xten}
val iterator: Iterator[T] = c.iterator();
while (iterator.hasNext()) {
  val x : T = iterator.next();
  S();
}
\end{xten}
%~~siv
%} }
%~~neg

A number of library classes implement \xcd`Iterable`, and thus can be iterated
over.  For example, iterating over a \xcd`Region` iterates the \xcd`Point`s in
the region, and iterating over an \xcd`Array` iterates over the
\xcd`Point`s at which the  array is defined.

The type of the loop variable may be supplied as \xcd`x <: T`.  In this case
the iterable \xcd`c` must have type \xcd`Iterable[U}` for some \xcd`U <: T`,
and \xcd`x` will be given the type \xcd`U`.

\begin{ex}
This loop adds up the elements of a \xcd`List[Int]`.
Note that iterating over a list yields the elements of the list, as specified
in the \xcd`List` API. 
%~~gen ^^^ Statements3d9l
% package Statements3d9l;
% class Example {
%~~vis
\begin{xten}
static def sum(a:x10.util.List[Int]):Int {
  var s : Int = 0;
  for(x in a) s += x;
  return s;
}
\end{xten}
%~~siv
%}
%~~neg

The following code sums the elements of an integer array.  Note that the
\xcd`for` loop iterates over the indices of the array, not the elements, as
specified in the \xcd`Array` API.  
%~~gen ^^^ Statements2d4h
% package Statements2d4h;
% class Example { 
%~~vis
\begin{xten}
static def sum(a: Array[Int]): Int {
  var s : Int = 0;
  for(p in a) s += a(p);
  return s;
}
\end{xten}
%~~siv
%}
%~~neg

Iteration over an \xcd`IntRange` (\Sref{sect:intrange}) is quite common. This
allows looping while varying an integer index: 
%~~gen ^^^ Statements3o9s
% package Statements3o9s;
% class Example { static def example() {
%~~vis
\begin{xten}
var sum : Int = 0;
for(i in 1..10) sum += i;
assert sum == 55;
\end{xten}
%~~siv
%} } 
% class Hook { def run() { Example.example(); return true; } }
%~~neg


\end{ex}

Iteration variables have the \xcd`for` statement as scope.  They shadow other
variables of the same names.


\section{Return statement}
\label{ReturnStatement}
\index{return}

%##(ReturnStatement
\begin{bbgrammar}
%(FROM #(prod:ReturnStatement)#)
     ReturnStatement \: \xcd"return" Exp\opt \xcd";" & (\ref{prod:ReturnStatement}) \\
\end{bbgrammar}
%##)

Methods and closures may return values using a return statement.
If the method's return type is explicitely declared \xcd"void",
the method must return without a value; otherwise, it must return
a value of the appropriate type.

\begin{ex}
The following code illustrates returning values from a closure and a method.
The \xcd`return` inside of \xcd`closure` returns from \xcd`closure`, not from
\xcd`method`.  
%~~gen ^^^ Statements2j1d
% package Statements2j1d;
% class Example {
%~~vis
\begin{xten}
def method(x:Int) {
  val closure = (y:Int) => {return x+y;}; 
  val res = closure(0);
  assert res == x;
  return res == x;
}
\end{xten}
%~~siv
%}
%~~neg


\end{ex}


\section{Assert statement} 
\index{assert}

%##(AssertStatement
\begin{bbgrammar}
%(FROM #(prod:AssertStatement)#)
     AssertStatement \: \xcd"assert" Exp \xcd";" & (\ref{prod:AssertStatement}) \\
                     \| \xcd"assert" Exp  \xcd":" Exp  \xcd";" \\
\end{bbgrammar}
%##)

The statement \xcd`assert E` checks that the Boolean expression \xcd`E`
evaluates to true, and, if not, throws an \xcd`x10.lang.Error`  exception.  
The annotated assertion statement \xcd`assert E : F;` checks \xcd`E`, and, if
it is 
false, throws an \xcd`x10.lang.Error` exception with \xcd`F`'s value attached
to it. 

\begin{ex}
The following code compiles properly.  
%~~gen ^^^ Statements100
% package Statements.Assert;
% 
%~~vis
\begin{xten}
class Example {
  public static def main(argv:Array[String](1)) {
    val a = 1;
    assert a != 1 : "Changed my mind about a.";
  }
}
\end{xten}
%~~siv
%~~neg
\noindent
However, when run, it 
prints a stack trace starting with 
\begin{xten}
x10.lang.Error: Changed my mind about a.
\end{xten}
\end{ex}

\section{Exceptions in X10}
\index{exception}
\index{termination!abrupt}
\index{termination!normal}

X10 programs can throw {\em Exceptions} to indicate unusual or problematic
situations; this is {\em abrupt termination}.  Exceptions, as data values, are
objects which which inherit from 
\xcd`x10.lang.Throwable`.    Exceptions may be thrown intentionally with the
\xcd`throw` statement. Many primitives and library functions throw exceptions
if they encounter problems; \eg, dividing by zero throws an instance of
\xcd`x10.lang.ArithmeticException`. 

When an exception is thrown, statically and dynamically enclosing
\xcd`try`-\xcd`catch` blocks in the same activity can attempt to handle it.   If the throwing
statement in inside some \xcd`try` clause, and some matching \xcd`catch`
clause catches that type of exception, the corresponding \xcd`catch` body will
be executed, and the process of throwing is finished.  
If no statically-enclosing \xcd`try`-\xcd`catch` block can handle the
exception, the current method call returns (abnormally), throwing the same
exception from the point at which the method was called.  

This process continues until the exception is handled or there are no more
calling methods in the activity. In the latter case, the activity will
terminate abnormally, and the exception will propagate to the activity's root;
see \Sref{ExceptionModel} for details.

Unlike some statically-typed languages with exceptions, X10's exceptions are
all {\em unchecked}. Methods do not declare which exceptions they might throw;
any method can, potentially, throw any exception.


\section{Throw statement}
\index{throw}

%##(ThrowStatement
\begin{bbgrammar}
%(FROM #(prod:ThrowStatement)#)
      ThrowStatement \: \xcd"throw" Exp \xcd";" & (\ref{prod:ThrowStatement}) \\
\end{bbgrammar}
%##)

\index{Exception}
\xcd"throw E" throws an exception whose value is \xcd`E`, which must be an
instance of a subtype of \xcd`x10.lang.Throwable`. 

\begin{ex}
The following code checks if an index is in range and
throws an exception if not.

%~~gen ^^^ Statements110
% package Statements_index_check;
% class ThrowStatementExample {
% def thingie(i:Int, x:Array[Boolean](1))  {
%~~vis
\begin{xten}
if (i < 0 || i >= x.size)
    throw new MyIndexOutOfBoundsException();
\end{xten}
%~~siv
%} }
% class MyIndexOutOfBoundsException extends Exception {}
%~~neg
\end{ex}

\section{Try--catch statement}
\index{try}
\index{catch}
\index{finally}
\index{exception}

%##(TryStatement Catches CatchClause Finally
\begin{bbgrammar}
%(FROM #(prod:TryStatement)#)
        TryStatement \: \xcd"try" Block Catches & (\ref{prod:TryStatement}) \\
                     \| \xcd"try" Block Catches\opt Finally \\
%(FROM #(prod:Catches)#)
             Catches \: CatchClause & (\ref{prod:Catches}) \\
                     \| Catches CatchClause \\
%(FROM #(prod:CatchClause)#)
         CatchClause \: \xcd"catch" \xcd"(" Formal \xcd")" Block & (\ref{prod:CatchClause}) \\
%(FROM #(prod:Finally)#)
             Finally \: \xcd"finally" Block & (\ref{prod:Finally}) \\
\end{bbgrammar}
%##)

Exceptions are handled with a \xcd"try" statement.
A \xcd"try" statement consists of a \xcd"try" block, zero or more
\xcd"catch" blocks, and an optional \xcd"finally" block.

First, the \xcd"try" block is evaluated.  If the block throws an
exception, control transfers to the first matching \xcd"catch"
block, if any.  A \xcd"catch" matches if the value of the
exception thrown is a subclass of the \xcd"catch" block's formal
parameter type.

The \xcd"finally" block, if present, is evaluated on all normal
and exceptional control-flow paths from the \xcd"try" block.
If the \xcd"try" block completes normally
or via a \xcd"return", a \xcd"break", or a
\xcd"continue" statement, 
the \xcd"finally"
block is evaluated, and then control resumes at
the statement following the \xcd"try" statement, at the branch target, or at
the caller as appropriate.
If the \xcd"try" block completes
exceptionally, the \xcd"finally" block is evaluated after the
matching \xcd"catch" block, if any, and when and if the \xcd`finally` block
finishs normally, the
exception is rethrown.


The parameter of a \xcd`catch` block has the block as scope.  It shadows other
variables of the same name.

\begin{ex}
The \xcd`example()` method below executes without any assertion errors
%~~gen ^^^ Statements9x3m
% package Statements9x3m;
% 
%~~vis
\begin{xten}
class Example {
  class SeriousExn extends Throwable {}
  class SillyExn   extends Throwable {}
  var didFinally : Boolean = false;
  def example() {
    try {
       throw new SillyExn();
       assert false; // This cannot happen
    }
    catch(SillyExn)   {return true;}
    catch(SeriousExn) {return false;}
    finally {
       this.didFinally = true;
    }
    return false;
  }
  static def doExample() {
    val e = new Example();
    assert e.example();
    assert e.didFinally == true;
  }
}
\end{xten}
%~~siv
% 
% class Hook { def run() { Example.doExample(); return true; } }
%~~neg

\end{ex}

\limitation{Constraints on exception types in \xcd`catch` blocks are not
currently supported. 
}

\section{Assert}

The \xcd`assert` statement 
%~~stmt~~`~~`~~B:Boolean ~~
\xcd`assert B;` 
checks that the Boolean expression \xcd`B` evaluates to true.  If so,
computation proceeds.  If not, it throws \xcd`x10.lang.AssertionError`.

The extended form 
%~~stmt~~`~~`~~B:Boolean, A:Any ~~ 
\xcd`assert B:A;`
is similar, but provides more debugging information.  The value of the
expression \xcd`A` is available as part of the \xcd`AssertionError`, \eg, to
be printed on the console.

\begin{ex}
\xcd`assert` is useful for confirming properties that you believe to be true
and wish to rely on.  In particular, well-chosen \xcd`assert`s make a program
robust in the face of code changes and unexpected uses of methods.
For example, the following method compute percent differences, but asserts
that it is not dividing by zero.  If the mean is zero, it throws an exception,
including the values of the numbers as potentially useful debugging
information. 
%~~gen ^^^ StmtAssert10
%package StmtAssert10;
% class Example {
%~~vis
\begin{xten}
static def percentDiff(x:Double, y:Double) {
  val diff = x-y;
  val mean = (x+y)/2;
  assert mean != 0.0  : [x,y]; 
  return Math.abs(100 * (diff / mean));
}
\end{xten}
%~~siv
% }
%~~neg

\end{ex}


At times it may be considered important not to check \xcd`assert` statements;
\eg, if the test is expensive and the code is sufficiently well-tested.  The
\xcd`-noassert` command line option causes the compiler to ignore all
\xcd`assert` statements. 
	

\chapter{Places}
\label{XtenPlaces}
\index{place}

An \Xten{} place is a repository for data and activities, corresponding
loosely to a process or a processor. Places induce a concept of ``local''. The
activities running in a place may access data items located at that place with
the efficiency of on-chip access. Accesses to remote places may take orders of
magnitude longer. X10's system of places is designed to make this obvious.
Programmers are aware of the places of their data, and know when they are
incurring communication costs, but the actual operation to do so is easy. It's
not hard to use non-local data; it's simply hard to to do so accidentally.

The set of places available to a computation is determined at the time that
the program is started, and remains fixed through the run of the program. See
the {\tt README} documentation on how to set command line and configuration
options to set the number of places.

Places are first-class values in X10, as instances 
\xcd"x10.lang.Place".   \xcd`Place` provides a number of useful ways to
query places, such as \xcd`Place.places`, which is a  \xcd`Sequence[Place]` of 
the places
available to the current run of the program.

Objects and structs (with one exception) are created in a single place -- the
place that the constructor call was running in. They cannot change places.
They can be {\em copied} to other places, and the special library struct
\Xcd{GlobalRef} allows values at one place to point to values at another.  

\section{The Structure of Places}
\index{place!MAX\_PLACES}
\index{place!FIRST\_PLACES}
\index{MAX\_PLACES}
\index{FIRST\_PLACE}

%~~exp~~`~~`~~ ~~ ^^^ Places10
Places are numbered 0 through \xcd`Place.MAX_PLACES-1`; the number is stored
in the field 
\xcd`pl.id`.  The \xcd`Sequence[Place]` \xcd`Place.places()` contains the places of the
program, in numeric order. 
The program starts by executing a \xcd`main` method at
%~~exp~~`~~`~~ ~~ ^^^ Places20
\xcd`Place.FIRST_PLACE`, which is 
%~~exp~~`~~`~~ ~~ ^^^ Placesoik
\xcd`Place.places()(0)`; see
\Sref{initial-computation}. 

Operations on places include \xcd`pl.next()`, which gives the next entry
(looping around) in \xcd`Place.places` and its opposite \xcd`pl.prev()`. 
In multi-place executions, 
\xcd`here.next()` is a convenient way to express ``a place other than \xcd`here`''.
There are also tests, like  
%~~exp~~`~~`~~pl:Place ~~ ^^^ Placesoid
\xcd`pl.isCUDA()`, which test for particular kinds of processors.




\section{{\tt here}}\index{here}\label{Here}

The variable \xcd"here" is always bound to the place at which the current
computation is running, in the same way that \xcd`this` is always bound to the
instance of the current class (for non-static code), or \xcd`self` is bound to
the instance of the type currently being constrained.  
\xcd`here` may denote different places in the same method body or even the
same expression, due to
place-shifting operations.


This is not unusual for automatic variables:  \Xcd{self} denotes 
two different values (one \xcd`List`, one \xcd`Int`) 
when one describes a non-null list of non-zero numbers as
\xcd`List[Int{self!=0}]{self!=null}`. In the following 
code, \xcd`here` has one value at 
\xcd`h0`, and a different one at \xcd`h1` (unless there is only one place).
%~~gen ^^^ Placesoijo
% package places.are.For.Graces;
% class Example {
% def example() {
%~~vis
\begin{xten}
val h0 = here;
at (here.next()) {
  val h1 = here; 
  assert (h0 != h1);
}
\end{xten}
%~~siv
%} } 
% 
%~~neg
\noindent
(Similar examples show that \xcd`self` and \xcd`this` have the same behavior:
\xcd`self` can be shadowed by constrained types appearing inside of type
constraints, and \xcd`this` by inner classes.)



The following example looks through a list of references to \Xcd{Thing}s.  
It finds those references to things that are \Xcd{here}, and deals with them.  
%~~gen ^^^ Places70
%package Places.Are.For.Graces.2;
%import x10.util.*;
%abstract class Thing {}
%class DoMine {
%  static def dealWith(Thing) {}	
%~~vis
\begin{xten}
  public static def deal(things: List[GlobalRef[Thing]]) {
     for(gr in things) {
        if (gr.home == here) {
           val grHere = 
               gr as GlobalRef[Thing]{gr.home == here};
           val thing <: Thing = grHere();
           dealWith(thing);
        }
     }
  }
\end{xten}
%~~siv
%}
% 
%~~neg

\section{ {\tt at}: Place Changing}\label{AtStatement}
\index{at}
\index{place!changing}

An activity may change place synchronously using the \xcd"at" statement or
\xcd"at" expression. Like any distributed operation, it is 
potentially expensive, as it requires, at a minimum, two messages
and the copying of all data used in the operation, and must be used with care
-- but it provides the basis for multicore programming in X10.

%##(AtStatement AtExp
\begin{bbgrammar}
%(FROM #(prod:AtStmt)#)
              AtStmt \: \xcd"at" \xcd"(" Exp \xcd")" Stmt & (\ref{prod:AtStmt}) \\
%(FROM #(prod:AtExp)#)
               AtExp \: \xcd"at" \xcd"(" Exp \xcd")" ClosureBody & (\ref{prod:AtExp}) \\
\end{bbgrammar}
%##)

The {\it PlaceExp} must be an expression of type \xcd`Place` or some
subtype. For programming convenience, if {\it PlaceExp} is of type
\xcd`GlobalRef[T]` then the \xcd'home' property of \xcd'GlobalRef' is
used as the value of {\it PlaceExp}.

%%AT-COPY%% The \xcd`at`-statment \xcd`at(p;F)S` first evaluates \xcd`p` to a place, then
%%AT-COPY%% copies information to that place as determined by \xcd`F`, and then executes
%%AT-COPY%% \xcd`S` using the resulting copies.  The \xcd`at`-{\em expression}
%%AT-COPY%% \xcd`at(p;F)E` is similar, but it copies the result of the expression \xcd`E`
%%AT-COPY%% and returns the copy as its result.
%%AT-COPY%% 
%%AT-COPY%% The clause \xcd`F` in \xcd`at(p;F)S` is a list of zero or more {\em copy
%%AT-COPY%% specifiers}, explaining what values are to be copied to the place \xcd`p`, and
%%AT-COPY%% how they are to be referred to at \xcd`p`.  
%%AT-COPY%% 

%%AT-COPY%% \begin{ex}
%%AT-COPY%% The following example creates a rail \xcd`a` located \xcd`here`, and copies
%%AT-COPY%% it to another place, giving the copy the name \xcd`a2` there.  The copy is
%%AT-COPY%% modified and examined.  After the \xcd`at` finishes, the original is also
%%AT-COPY%% examined, and (since only the copy, not the original, was modified) is observed
%%AT-COPY%% to be unchanged. 
%%AT-COPY%% %~x~gen ^^^ Places6e1o
%%AT-COPY%% % package Places6e1o;
%%AT-COPY%% % KNOWNFAIL-at
%%AT-COPY%% % class Example { static def example() { 
%%AT-COPY%% %~x~vis
%%AT-COPY%% \begin{xten}
%%AT-COPY%% val a = [1,2,3];
%%AT-COPY%% at(here.next(); a2 = a) {
%%AT-COPY%%   a2(1) = 4;
%%AT-COPY%%   assert a2(0)==1 && a2(1)==4 && a2(2)==3; 
%%AT-COPY%%   // 'a' is not accessible here
%%AT-COPY%% }
%%AT-COPY%% assert  a(0)==1 && a(1)==2 && a(2)==3; 
%%AT-COPY%% \end{xten}
%%AT-COPY%% %~x~siv
%%AT-COPY%% %} } 
%%AT-COPY%% % class Hook { def run() { Example.example(); return true; }}
%%AT-COPY%% %~x~neg
%%AT-COPY%% \end{ex}
%%AT-COPY%% 

\begin{ex}
The following example creates a rail \xcd`a` located \xcd`here`, and copies
it to another place.  \xcd`a` in the second place (\xcd`here.next()`) refers
to the copy.  The copy is
modified and examined.  After the \xcd`at` finishes, the original is also
examined, and (since only the copy, not the original, was modified) is observed
to be unchanged. 
%~~gen ^^^ Places6e1o
% package Places6e1o;
% KNOWNFAIL-at
% class Example { static def example() { 
%~~vis
\begin{xten}
val a = [1,2,3];
at(here.next()) {
  a(1) = 4;
  assert a(0)==1 && a(1)==4 && a(2)==3; 
}
assert  a(0)==1 && a(1)==2 && a(2)==3; 
\end{xten}
%~~siv
%} } 
% class Hook { def run() { Example.example(); return true; }}
%~~neg
\end{ex}

%%AT-COPY%% \subsection{Copy Specifiers}
%%AT-COPY%% \label{sect:copy-spec}
%%AT-COPY%% \index{copy specifier}
%%AT-COPY%% \index{at!copy specifier}
%%AT-COPY%% 
%%AT-COPY%% A single copy specifier can be one of the following forms.   
%%AT-COPY%% Each copy specifier determines an {\em original-expression}, saying what value
%%AT-COPY%% will be copied, and a {\em target variable}, saying what it will be called.
%%AT-COPY%% 
%%AT-COPY%% \begin{itemize}
%%AT-COPY%% 
%%AT-COPY%% \item \xcd`val x = E`, and its usual variants \xcd`val x:T = E`, 
%%AT-COPY%%       \xcd`x : T = E`, and 
%%AT-COPY%%       \xcd`val x <: T = E`, evaluate the expression \xcd`E` at the initial
%%AT-COPY%%       place, copy it to \xcd`p`, and bind \xcd`x` to the copy, as normal for a
%%AT-COPY%%       local \xcd`val` binding.  If a type is supplied, it is checked
%%AT-COPY%%       statically in the usual way.  
%%AT-COPY%%       The original-expression is \xcd`E`, and the target variable is \xcd`x`.
%%AT-COPY%% 
%%AT-COPY%% \begin{ex}
%%AT-COPY%% The following code copies a variable \xcd`a` located \xcd`here` to a variable
%%AT-COPY%% \xcd`d` located \xcd`there`.  
%%AT-COPY%% Note that, while the copy \xcd`d` is available \xcd`there` inside of the \xcd`at`-block,
%%AT-COPY%% the original \xcd`a` is not.  (\xcd`a` could not be available in the block in
%%AT-COPY%% any case; it is not located \xcd`there`.)
%%AT-COPY%% %~~gen ^^^ Places9v2e1
%%AT-COPY%% % package Places9v2e1;
%%AT-COPY%% % KNOWNFAIL-at
%%AT-COPY%% % class Example{ 
%%AT-COPY%% % static def use(Any) = 1;
%%AT-COPY%% % static def example() { 
%%AT-COPY%% %  val there = here.next();
%%AT-COPY%% %~~vis
%%AT-COPY%% \begin{xten}
%%AT-COPY%% var a : Int = 1;
%%AT-COPY%% at(there; val d = a) {
%%AT-COPY%%    assert d == 1;
%%AT-COPY%%    // ERROR: assert a == 1;
%%AT-COPY%% }
%%AT-COPY%% \end{xten}
%%AT-COPY%% %~~siv
%%AT-COPY%% % } } 
%%AT-COPY%% % class Hook{ def run() {Example.example(); return true;}}
%%AT-COPY%% %~~neg
%%AT-COPY%% \end{ex}
%%AT-COPY%% 
%%AT-COPY%% \item \xcd`var x : T = E` evaluates \xcd`E` at the initial place, copies it to
%%AT-COPY%%       \xcd`p`, and binds \xcd`x` to a new \xcd`var` whose initial value is the
%%AT-COPY%%       copy, as normal for a local \xcd`var` binding.
%%AT-COPY%%       If a type is supplied, it is checked
%%AT-COPY%%       statically in the usual way.
%%AT-COPY%%       The original-expression is \xcd`E`, and the target variable is \xcd`x`.
%%AT-COPY%%       Note that, like a \xcd`var` parameter to a method, \xcd`x` is a local
%%AT-COPY%%       variable.  Changes to \xcd`x` will not change anything else. In
%%AT-COPY%%       particular, even if \xcd`x` has the same name as a \xcd`var` variable
%%AT-COPY%%       outside, the two \xcd`var`s are unconnected.  
%%AT-COPY%%       See \Sref{sect:athome} for the way to modify a variable from the
%%AT-COPY%%       surrounding scope.
%%AT-COPY%% 
%%AT-COPY%% \begin{ex}
%%AT-COPY%% The following code copies \xcd`a` to a \xcd`var` named \xcd`e`.  Changing
%%AT-COPY%% \xcd`e` does not change \xcd`a`; the two \xcd`var`s have no ongoing relationship.
%%AT-COPY%% %~~gen ^^^ Places9v2e2
%%AT-COPY%% % package Places9v2e2;
%%AT-COPY%% % KNOWNFAIL-at
%%AT-COPY%% % class Example{ 
%%AT-COPY%% % static def use(Any) = 1;
%%AT-COPY%% % static def example() { 
%%AT-COPY%% %  val there = here.next();
%%AT-COPY%% %~~vis
%%AT-COPY%% \begin{xten}
%%AT-COPY%% var a : Int = 1;
%%AT-COPY%% assert a == 1;
%%AT-COPY%% at(there; var e = a) { 
%%AT-COPY%%    assert e == 1;
%%AT-COPY%%    e += 1;
%%AT-COPY%%    assert e == 2;
%%AT-COPY%% }
%%AT-COPY%% assert a == 1; 
%%AT-COPY%% \end{xten}
%%AT-COPY%% %~~siv
%%AT-COPY%% % 
%%AT-COPY%% % }  } 
%%AT-COPY%% % class Hook{ def run() {Example.example(); return true;}}
%%AT-COPY%% %~~neg
%%AT-COPY%% \end{ex}
%%AT-COPY%% 
%%AT-COPY%% \item \xcd`x = E`, as a copy specifier, is equivalent to \xcd`val x = E`.
%%AT-COPY%%       Note that this abbreviated form is not available as a local variable
%%AT-COPY%%       definition, (because it is used as an assignment statement), but in a
%%AT-COPY%%       copy specifier there are no assignment statements and so the
%%AT-COPY%%       abbreviation is allowed.
%%AT-COPY%%       The original-expression is \xcd`E`, and the target variable is \xcd`x`.
%%AT-COPY%% 
%%AT-COPY%% \begin{ex}
%%AT-COPY%% The following code evaluates an expression \xcd`a+b(0)`.  The result of this
%%AT-COPY%% expression is stored \xcd`there`, in the \xcd`val` variable \xcd`f`, but is
%%AT-COPY%% not stored \xcd`here`. 
%%AT-COPY%% %~~gen ^^^ Places9v2e3
%%AT-COPY%% % package Places9v2e3;
%%AT-COPY%% % KNOWNFAIL-at
%%AT-COPY%% % class Example{ 
%%AT-COPY%% % static def use(Any) = 1;
%%AT-COPY%% % static def example() { 
%%AT-COPY%% %  val there = here.next();
%%AT-COPY%% %~~vis
%%AT-COPY%% \begin{xten}
%%AT-COPY%% var a : Int = 1;
%%AT-COPY%% var b : Rail[Int] = [2,3,4];
%%AT-COPY%% at(there; f = a + b(0)) {
%%AT-COPY%%    assert f == 3;
%%AT-COPY%% }
%%AT-COPY%% \end{xten}
%%AT-COPY%% %~~siv
%%AT-COPY%% % }  } 
%%AT-COPY%% % class Hook{ def run() {Example.example(); return true;}}
%%AT-COPY%% % 
%%AT-COPY%% %~~neg
%%AT-COPY%% 
%%AT-COPY%% 
%%AT-COPY%% \end{ex}
%%AT-COPY%% 
%%AT-COPY%% \item \xcd`x` alone, as a copy specifier, is equivalent to \xcd`val x = x`.
%%AT-COPY%%       It says that the variable \xcd`x` will be copied, and the copy will also
%%AT-COPY%%       be named \xcd`x`.  
%%AT-COPY%%       The original-expression is \xcd`x`, and the target variable is \xcd`x`.
%%AT-COPY%% 
%%AT-COPY%% \begin{ex}
%%AT-COPY%% The following code copies \xcd`b` to \xcd`there`.  The copy is also called
%%AT-COPY%% \xcd`b`.  The two \xcd`b`'s are not connected; \eg, changing one does not
%%AT-COPY%% change the other.
%%AT-COPY%% %~~gen ^^^ Places9v2e4
%%AT-COPY%% % package Places9v2e4;
%%AT-COPY%% % KNOWNFAIL-at
%%AT-COPY%% % class Example{ 
%%AT-COPY%% % static def use(Any) = 1;
%%AT-COPY%% % static def example() { 
%%AT-COPY%% %  val there = here.next();
%%AT-COPY%% %~~vis
%%AT-COPY%% \begin{xten}
%%AT-COPY%% var b : Rail[Int] = [2,3,4];
%%AT-COPY%% assert b(0) == 2;
%%AT-COPY%% at(there; b) {
%%AT-COPY%%   b(0) = 200;  // Modify copy of b.
%%AT-COPY%%   assert b(0) == 200;
%%AT-COPY%% }
%%AT-COPY%% assert b(0) == 2; 
%%AT-COPY%% \end{xten}
%%AT-COPY%% %~~siv
%%AT-COPY%% % 
%%AT-COPY%% % }  } 
%%AT-COPY%% % class Hook{ def run() {Example.example(); return true;}}
%%AT-COPY%% %~~neg
%%AT-COPY%% \end{ex}
%%AT-COPY%% 
%%AT-COPY%% \item A field assignment statements \xcdmath"a.fld = $E_2$", evaluates 
%%AT-COPY%%       \xcd`a` and $E_2$ on the sending side to values $v_1$ and {$v_2$}.  
%%AT-COPY%%       {$v_1$} must be an object with a mutable field \xcd`fld`.  {$v_1$} and
%%AT-COPY%%       {$v_2$} are sent to place \xcd`p`, and the field assignment is performed
%%AT-COPY%%       there.  The modified version of {$v_1$} is available as a \xcd`val`
%%AT-COPY%%       variable \xcd`a`.   The compiler may optimize this, \eg, by neglecting to
%%AT-COPY%%       deserialize \xcdmath"$v_1$.fld", and deserializing {$v_2$} directly into
%%AT-COPY%%       that field rather than into a separate buffer.
%%AT-COPY%% 
%%AT-COPY%% \begin{ex}
%%AT-COPY%% %~~gen ^^^ Places9v2e5
%%AT-COPY%% % package Places9v2e5;
%%AT-COPY%% % KNOWNFAIL
%%AT-COPY%% % class Example {
%%AT-COPY%% % static def use(Any) = 1;
%%AT-COPY%% % static def example() { 
%%AT-COPY%% %  val there = here.next();
%%AT-COPY%% %~~vis
%%AT-COPY%% \begin{xten}
%%AT-COPY%% class Example{ 
%%AT-COPY%%    var f : Int = 1;
%%AT-COPY%%    var g : Int = 2;
%%AT-COPY%%    static def example() { 
%%AT-COPY%%       val there = here.next();
%%AT-COPY%%       val e : Example = new Example();
%%AT-COPY%%       assert e.f == 1 && e.g == 2;
%%AT-COPY%%       at(there; e.f = 3) {
%%AT-COPY%%           assert e.f == 3; && e.g == 2;
%%AT-COPY%%       }
%%AT-COPY%%       assert e.f == 1 && e.g == 2;
%%AT-COPY%%    }
%%AT-COPY%% }
%%AT-COPY%% \end{xten}
%%AT-COPY%% %~~siv
%%AT-COPY%% % class Hook{ def run() {Example.example(); return true;}}
%%AT-COPY%% %~~neg
%%AT-COPY%% %
%%AT-COPY%% \end{ex}
%%AT-COPY%% 
%%AT-COPY%% \item A rail-element assignment 
%%AT-COPY%%       \xcdmath"a($E_1$, $\ldots$, $E_n$) = $E_+$".
%%AT-COPY%%       This copies and transmits \xcd`a` as normal for a rail.  In addition,
%%AT-COPY%%       and 
%%AT-COPY%%       much like a field assignment, it also evaluates all the expressions $E_i$
%%AT-COPY%%       at the sending side to values $v_i$, and transmits them.  \xcd`a`'s value must
%%AT-COPY%%       admit a suitably-typed $n$-ary subscripting operation.  That operation
%%AT-COPY%%       is applied after the values are deserialized at \xcd`p`.  The compiler
%%AT-COPY%%       may optimize this, \eg, by neglecting to deserialize one element of the
%%AT-COPY%%       rail $v_0$, and deserializing $v_+$ directly into that location.  
%%AT-COPY%% 
%%AT-COPY%% 
%%AT-COPY%% \begin{ex}
%%AT-COPY%% The following code sends a modified \xcd`b` to \xcd`there`, while (as always)
%%AT-COPY%% keeping an unmodified version \xcd`here`.   X10 may perform optimizations to
%%AT-COPY%% avoid transmitting the original value of \xcd`b(1)`, since it will be
%%AT-COPY%% overwritten immediately in any case.
%%AT-COPY%% %~~gen ^^^ Places9v2e6
%%AT-COPY%% % package Places9v2e6;
%%AT-COPY%% % KNOWNFAIL
%%AT-COPY%% % class Example{ 
%%AT-COPY%% % static def use(Any) = 1;
%%AT-COPY%% % static def example() { 
%%AT-COPY%% %  val there = here.next();
%%AT-COPY%% %~~vis
%%AT-COPY%% \begin{xten}
%%AT-COPY%% var b = [2,3,4];
%%AT-COPY%% assert b(0) == 2 && b(1) == 3;
%%AT-COPY%% at(there; b(1) = 300) {
%%AT-COPY%%   assert b(0) == 2 && b(1) == 300;
%%AT-COPY%% }
%%AT-COPY%% assert b(0) == 2 && b(1) == 3;
%%AT-COPY%% \end{xten}
%%AT-COPY%% %~~siv
%%AT-COPY%% % 
%%AT-COPY%% %~~neg
%%AT-COPY%% % }  }
%%AT-COPY%% % class Hook{ def run() {Example.example(); return true;}}
%%AT-COPY%% \end{ex}
%%AT-COPY%% 
%%AT-COPY%% \item \xcd`*` may appear as the last copy specifier in the list, indicating
%%AT-COPY%%       that all \xcd`val` variables from outside \xcd`S` which are used in
%%AT-COPY%%       \xcd`S` should be copied. Specifically, let 
%%AT-COPY%%       \xcdmath"x$_1, \ldots, $x$_n$" be all the \xcd`val` variables defined
%%AT-COPY%%       outside of \xcd`S` 
%%AT-COPY%%       mentioned in \xcd`S`. The \xcd`*` copy specifier is equivalent to 
%%AT-COPY%%       the list of variables 
%%AT-COPY%%       \xcdmath"x$_1, \ldots, $x$_n$".
%%AT-COPY%% 
%%AT-COPY%% \begin{ex}
%%AT-COPY%% %~~gen ^^^ Places9v2e7
%%AT-COPY%% % package Places9v2e7;
%%AT-COPY%% % KNOWNFAIL-at
%%AT-COPY%% % class Example{ 
%%AT-COPY%% % static def use(Any) = 1;
%%AT-COPY%% % static def example() { 
%%AT-COPY%% %  val there = here.next();
%%AT-COPY%% %~~vis
%%AT-COPY%% \begin{xten}
%%AT-COPY%% var a : Int = 1;
%%AT-COPY%% val b = [2,3,4];
%%AT-COPY%% at(there; *) {
%%AT-COPY%%   assert a + b(0) == b(1);
%%AT-COPY%% }
%%AT-COPY%% \end{xten}
%%AT-COPY%% %~~siv
%%AT-COPY%% % }  }
%%AT-COPY%% % class Hook{ def run() {Example.example(); return true;}}
%%AT-COPY%% %~~neg
%%AT-COPY%% 
%%AT-COPY%% \end{ex}
%%AT-COPY%% 
%%AT-COPY%% \end{itemize}
%%AT-COPY%% 
%%AT-COPY%% As an important special case, \xcd`at(p;)S` copies {\em nothing} to \xcd`S`.
%%AT-COPY%% This must not be confused with \xcd`at(p)S`, which copies {\em everything}.
%%AT-COPY%% 
%%AT-COPY%% 
%%AT-COPY%% 
%%AT-COPY%% Note that \xcd`at(p;x,*)use(x,y);` is equivalent to \xcd`at(p;*)use(x,y);`.
%%AT-COPY%% In both statements, the \xcd`*` indicates that all variables used in the body
%%AT-COPY%% are to be copied in.  The former makes clear that \xcd`x` is one of the things
%%AT-COPY%% being copied, but, from the \xcd`*`, there may be others. 
%%AT-COPY%% 
%%AT-COPY%% However, other copy specifiers may be used to compute
%%AT-COPY%% values in \xcd`S` which are not available (and thus need not be stored)
%%AT-COPY%% outside of it.  
%%AT-COPY%% 
%%AT-COPY%% \begin{ex}The following code may end up with a large object \xcd`c` in
%%AT-COPY%% memory at \xcd`p` but not at the initial place: 
%%AT-COPY%% %~~gen ^^^ Places3q9u
%%AT-COPY%% % package Places3q9u;
%%AT-COPY%% % KNOWNFAIL-at
%%AT-COPY%% % class Example { 
%%AT-COPY%% % def use(Example, Example, Example) = 1;
%%AT-COPY%% % def Elephant(Example) = 1;
%%AT-COPY%% % static def example(a: Example, b:Example, p:Place) { 
%%AT-COPY%% %~~vis
%%AT-COPY%% \begin{xten}
%%AT-COPY%% at(p; c = a.Elephant(b), *) {
%%AT-COPY%%   use(a,b,c);
%%AT-COPY%% }
%%AT-COPY%% \end{xten}
%%AT-COPY%% %~~siv
%%AT-COPY%% %} } 
%%AT-COPY%% %~~neg
%%AT-COPY%% \end{ex}
%%AT-COPY%% 
%%AT-COPY%% The blanket \xcd`at`-statement \xcd`at(p)S` copies everything.  It is an
%%AT-COPY%% abbreviation for \xcd`at(p;*)S`.  
%%AT-COPY%% When this manual refers to a generic \xcd`at`-statement as \xcd`at(p;F)S`, it
%%AT-COPY%% should be understood as including the blanket \xcd`at` statement \xcd`at(p)S`
%%AT-COPY%% with this interpretation.
%%AT-COPY%% 


\subsection{Copying Values}
%%AT-COPY%% An activity executing statement \xcd"at (q;F) S" at a place \xcd`p`
%%AT-COPY%% evaluates \xcd`q` at \xcd`p` and then moves to \xcd`q` to execute
%%AT-COPY%% \xcd`S`.  
%%AT-COPY%% The original-expressions of \xcd`F` are evaluated at \xcd`p`.
%%AT-COPY%% Their values are copied (\Sref{sect:at-init-val}) to \xcd`q`, and bound to 
%%AT-COPY%% names there, as specified by \xcd`F`.  
%%AT-COPY%% \xcd`S` is evaluated in an environment containing the target variables of
%%AT-COPY%% \xcd`F`, and \xcd`here` and {\em no} other variables.  (In particular, if this
%%AT-COPY%% statement appears in an instance method body and \xcd`this` is not copied,
%%AT-COPY%% \xcd`this` is not accessible.  This fact is important: it allows the
%%AT-COPY%% programmer to control when \xcd`this` is copied, which may be expensive for
%%AT-COPY%% large containers.)

An activity executing \xcd`at(q)S` at a place \xcd`p` evaluates \xcd`q` at
place \xcd`p`, which should be a \xcd`Place`.  It then moves to place \xcd`q`
to execute \xcd`S`.  The values variables that \xcd`S` refers to are copied
(\Sref{sect:at-init-val}) to \xcd`q`, and bound to the variables of the same
name.   If the \xcd`at` is inside of an instance method and \xcd`S` uses
\xcd`this`, \xcd`this` is copied as well.  Note that a field reference
\xcd`this.fld` or a method call \xcd`this.meth()` will cause \xcd`this` to be
copied --- as will their abbreviated forms \xcd`fld` and \xcd`meth()`, despite
the lack of a visible \xcd`this`.  


Note that the value obtained by evaluating \xcd`q`
is not necessarily distinct from \xcd`p` (\eg, \xcd`q` may be
\xcd`here`). 
This does not alter the behavior of \xcd`at`.  
%%AT-COPY%%  \xcd`at(here;F)S` will copy all the values specified by \xcd`F`, 
%%AT-COPY%% even though there is no actual change of place, and even though the original
%%AT-COPY%% values already exist there.
\xcd`at(here)S` will copy all the values mentioned in \xcd`S`, even though
there is no actual change of place, and even though the original values
already exist there. 

On normal termination of \xcd`S` control returns to \xcd`p` and
execution is continued with the statement following 
%%AT-COPY%% \xcd`at (q;F) S`. 
\xcd`at (q) S`. 
If
\xcd`S` terminates abruptly with exception \xcd`E`, \xcd`E` is
serialized into a buffer, the buffer is communicated to \xcd`p` where
it is deserialized into an exception \xcd`E1` and \xcd`at (p) S`
throws \xcd`E1`.

Since 
%%AT-COPY%% \xcd`at(p;F) S` 
\xcd`at(p) S` 
is a synchronous construct, usual control-flow
constructs such as \xcd`break`, \xcd`continue`, \xcd`return` and 
\xcd`throw` are permitted in \xcd`S`.  All concurrency related
constructs -- \xcd`async`, \xcd`finish`, \xcd`atomic`, \xcd`when` are
also permitted.

The \xcd`at`-expression 
%%AT-COPY%% \xcd`at(p;F)E` 
\xcd`at(p)E` 
is similar, except that, in the case of
normal termination of \xcd`E`, the value that \xcd`E` produces is serialized
into a buffer, transported to the starting place, and deserialized, and the
value of the \xcd`at`-expression is the result of deserialization.

\limitation{
X10 does not currently allow {\tt break}, {\tt continue}, or {\tt return}
to exit from an {\tt at}.
}



\subsection{How {\tt at} Copies Values}
\label{sect:at-init-val}

%%AT-COPY%% The values of the original-expressions  specified by \xcd`F` in 
%%AT-COPY%% \xcd`at (p;F)S` are copied to \xcd`p`, as follows.

The values mentioned in \xcd`S` are copied to place \xcd`p` by \xcd`at(p)S` as follows.

First, the original-expressions are evaluated to give a vector of X10 values.
Consider the graph of all values reachable from these values (except for 
\xcd`transient` fields 
(\Sref{sect:transient}, \xcd`GlobalRef`s (\Sref{GlobalRef}); also custom
serialization (\Sref{sect:ser+deser} may alter this behavior)). 

Second this graph is {\em
serialized} into a buffer and transmitted to place \xcd`q`.  Third,
the vector of X10 values is 
re-created at \xcd`q` 
by deserializing the buffer at
\xcd`q`. Fourth, \xcd`S` is executed at \xcd`q`, in an environment in
which each variable \xcd`v` declared in \xcd`F` 
refers to the corresponding deserialized value.  

Note that since values accessed across an \xcd`at` boundary are
copied, the programmer may wish to adopt the discipline that either
variables accessed across an \xcd`at` boundary  contain only structs 
or stateless objects, or the methods invoked on them do not access any
mutable state on the objects. Otherwise the programmer has to ensure
that side effects are made to the correct copy of the object. For this
the struct \xcd`x10.lang.GlobalRef[T]` is often useful.


\subsubsection{Serialization and deserialization.}
\label{sect:ser+deser}
\index{transient}
\index{field!transient}
The X10 runtime provides a default mechanism for
serializing/deserializing an object graph with a given set of roots.
This mechanism may be overridden by the programmer on a per class or
struct basis as described in the API documentation for
\xcd`x10.io.CustomSerialization`.  
The default mechanism performs a
deep copy of the object graph (that is, it copies the object or struct
and, recursively, the values contained in its fields), but does not
traverse or copy \xcd`transient` fields. \xcd`transient` fields are omitted from the
serialized data.   On deserialization, \xcd`transient` fields are initialized
with their default values (\Sref{DefaultValues}).    The types of
\xcd`transient` fields must therefore have default values.



A struct \xcd`s` of type \xcd`x10.lang.GlobalRef[T]` \ref{GlobalRef}
is serialized as a unique global reference to its contained object
\xcd`o` (of type \xcd`T`).  Please see the documentation
of \xcd`x10.lang.GlobalRef[T]` for more details.



\subsection{{\tt at} and Activities}
%%AT-COPY%% \xcd`at(p;F)S` 
\xcd`at(p)S` 
does {\em not} start a new activity.  It should be thought of as
transporting the current activity to \xcd`p`, running \xcd`S` there, and then
transporting it back.  \xcd`async` is the only construct in the
language that starts a new activity. In different contexts, each one
of the following makes sense:
%%AT-COPY%% (1)~\xcd`async at(p;F) S` 
(1)~\xcd`async at(p) S` 
(spawn an activity locally to execute \xcd`S` at
\xcd`p`; here \xcd`p` is evaluated by the spawned activity) , 
%%AT-COPY%% (2)~\xcd`at(p;F) async S` 
(2)~\xcd`at(p) async S` 
(evaluate \xcd`p` and then at \xcd`p` spawn an
activity to execute \xcd`S`), and,
%%AT-COPY%% (3)~\xcd`async at(p;F) async S`. 
(3)~\xcd`async at(p) async S`. 
%%AT-COPY%% In most cases, \xcd`at(p;F) async S` is preferred to
%%\xcd`async at(p;F)`, since In most cases, \xcd`at(p) async S` is
preferred to \xcd`async at(p) S`, since the former form enables a more
efficient runtime implementation.  In the first case, the expression
\xcd`p` is evaluated synchronously by the current activity and then a
single remote async is spawned.  In the second case, \xcd`p` is
semantically required to be evaluated asynchronously with the parent
async as it is contained in the body of an async.  Therefore, if the
compiler cannot prove that "async at (p)" can be safely rewritten into
"at (p) async", a first local async is spawned to evaluate \xcd`p`
then a remote async is spawned to evaluate \xcd`S`.

Since 
%%AT-COPY%% \Xcd{at(p;F) S} 
\Xcd{at(p) S} 
does not start a new activity, 
\xcd`S` may contain constructs which only make sense
within a single activity.  
For example, 
\begin{xten}
    for(x in globalRefsToThings) 
      if (at(x.home) x().isNice()) 
        return x();
\end{xten}
returns the first nice thing in a collection.   If we had used 
\xcd`async at(x.home)`, this would not be allowed; 
you can't \xcd`return` from an
\xcd`async`. 

\limitation{
X10 does not currently allow {\tt break}, {\tt continue}, or {\tt return}
to exit from an {\tt at}.
}



\subsection{Copying from {\tt at} }
\index{at!copying}

%%AT-COPY%% \xcd`at(p;F)S` copies data as specified by \xcd`F`, and sends it
\xcd`at(p)S` copies data required in \xcd`S`, and sends it
to place \xcd`p`, before executing \xcd`S` there. The only things that are not
copied are values only reachable through \xcd`GlobalRef`s and \xcd`transient`
fields, and data omitted by custom serialization.    
%%AT-COPY%% Several choices of copy specifier use the same identifier for the original
%%AT-COPY%% variable outside of 
%%AT-COPY%% \xcd`at(p)S` 
%%AT-COPY%% and its copy inside of \xcd`S`.  
%%AT-COPY%% 

\begin{ex}
%%AT-COPY%% 
%%AT-COPY%% %~~gen ^^^ Places_implicit_copy_from_at_example_1
%%AT-COPY%% % package Places.implicitcopyfromat;
%%AT-COPY%% % class Example {
%%AT-COPY%% % static def example() {
%%AT-COPY%% % 
%%AT-COPY%% %~~vis
%%AT-COPY%% \begin{xten}
%%AT-COPY%% val c = new Cell[Int](9); // (1)
%%AT-COPY%% at (here;c) {             // (2)
%%AT-COPY%%    assert(c() == 9);      // (3)
%%AT-COPY%%    c.set(8);              // (4)
%%AT-COPY%%    assert(c() == 8);      // (5)
%%AT-COPY%% }
%%AT-COPY%% assert(c() == 9);         // (6)
%%AT-COPY%% \end{xten}
%%AT-COPY%% %~~siv
%%AT-COPY%% %}}
%%AT-COPY%% % class Hook{ def run() { Example.example(); return true; } }
%%AT-COPY%% %~~neg
%%AT-COPY%% 

%~~gen ^^^ Places_implicit_copy_from_at_example_1
% package Places.implicitcopyfromat;
% class Example {
% static def example() {
% 
%~~vis
\begin{xten}
val c = new Cell[Int](9); // (1)
at (here) {               // (2) 
   assert(c() == 9);      // (3)
   c.set(8);              // (4)
   assert(c() == 8);      // (5)
}
assert(c() == 9);         // (6)
\end{xten}
%~~siv
%}}
% class Hook{ def run() { Example.example(); return true; } }
%~~neg


The \xcd`at` statement copies the \xcd`Cell` and its contents.  
After \xcd`(1)`, \xcd`c` is a \xcd`Cell` containing 9; call that cell {$c_1$}
At \xcd`(2)`, that cell is copied, resulting in another cell {$c_2$} whose
contents are also 9, as tested at \xcd`(3)`.
(Note that the copying behavior of \xcd`at` happens {\em even when the
destination place is the same as the starting place}--- even with
\xcd`at(here)`.)
At \xcd`(4)`, the contents of {$c_2$} are changed to 8, as confirmed at \xcd`(5)`; the contents of
{$c_1$} are of course untouched.    Finally, at \xcd`(c)`, outside the scope
of the \xcd`at` started at line \xcd`(2)`, \xcd`c` refers to its original
value {$c_1$} rather than the copy {$c_2$}.  
\end{ex}

The \xcd`at` statement induces a {\em deep copy}.  Not only does it copy the
values of variables, it copies values that they refer to through zero or more
levels of reference.  Structures are preserved as well: if two fields
\xcd`x.f` and \xcd`x.g` refer to the same object {$o_1$} in the original, then
\xcd`x.f` and \xcd`x.g` will both refer to the same object {$o_2$} in the
copy.  

\begin{ex}
In the following variation of the preceding example,
\xcd`a`'s original value {$a_1$} is a rail with two references to the same
\xcd`Cell[Int]` {$c_1$}.  The fact that {$a_1(0)$} and {$a_1(1)$} are both
identical to {$c_1$} is demonstrated in \xcd`(A)`-\xcd`(C)`, as {$a_1(0)$} is modified
and {$a_1(1)$} is observed to change.  In \xcd`(D)`-\xcd`(F)`, the copy
{$a_2$} is tested in the same way, showing that {$a_2(0)$} and {$a_2(1)$} both
refer to the same \xcd`Cell[Int]` {$c_2$}.  However, the test at \xcd`(G)`
shows that {$c_2$} is a different cell from {$c_1$}, because changes to
{$c_2$} did not propagate to {$c_1$}.  

%%AT-COPY%% %~~gen ^^^ PlacesAtCopy
%%AT-COPY%% %package Places.AtCopy2;
%%AT-COPY%% %class example {
%%AT-COPY%% %static def Example() {
%%AT-COPY%% %
%%AT-COPY%% %~~vis
%%AT-COPY%% \begin{xten}
%%AT-COPY%% val c = new Cell[Int](5);
%%AT-COPY%% val a : Rail[Cell[Int]] = [c,c as Cell[Int]];
%%AT-COPY%% assert(a(0)() == 5 && a(1)() == 5);     // (A)
%%AT-COPY%% c.set(6);                               // (B)
%%AT-COPY%% assert(a(0)() == 6 && a(1)() == 6);     // (C)
%%AT-COPY%% at(here;a) {
%%AT-COPY%%   assert(a(0)() == 6 && a(1)() == 6);   // (D)
%%AT-COPY%%   c.set(7);                             // (E)
%%AT-COPY%%   assert(a(0)() == 7 && a(1)() == 7);   // (F)
%%AT-COPY%% }
%%AT-COPY%% assert(a(0)() == 6 && a(1)() == 6);     // (G)
%%AT-COPY%% \end{xten}
%%AT-COPY%% %~~siv
%%AT-COPY%% %}}
%%AT-COPY%% %class Hook{ def run() { example.Example(); return true; } }
%%AT-COPY%% %~~neg

%~~gen ^^^ PlacesAtCopy
%package Places.AtCopy2;
%class example {
%static def Example() {
%
%~~vis
\begin{xten}
val c = new Cell[Int](5);
val a : Rail[Cell[Int]] = [c,c as Cell[Int]];
assert(a(0)() == 5 && a(1)() == 5);     // (A)
c.set(6);                               // (B)
assert(a(0)() == 6 && a(1)() == 6);     // (C)
at(here) {
  assert(a(0)() == 6 && a(1)() == 6);   // (D)
  c.set(7);                             // (E)
  assert(a(0)() == 7 && a(1)() == 7);   // (F)
}
assert(a(0)() == 6 && a(1)() == 6);     // (G)
\end{xten}
%~~siv
%}}
%class Hook{ def run() { example.Example(); return true; } }
%~~neg


\end{ex}

\subsection{Copying and Transient Fields}
\label{sect:transient}
\index{at!transient fields and}
\index{transient}
\index{field!transient}

Recall that fields of classes and structs marked \xcd`transient` are not copied by
\xcd`at`.  Instead, they are set to the default values for their types. Types
that do not have default values cannot be used in \xcd`transient` fields.

\begin{ex}
Every \xcd`Trans` object has an \xcd`a`-field equal
to 1.  However, despite the initializer on the \xcd`b` field, it is not the
case that every \xcd`Trans` has \xcd`b==2`.  Since \xcd`b` is \xcd`transient`,
when the \xcd`Trans` value \xcd`this` is copied at \xcd`at(here){...}` in
\xcd`example()`, its \xcd`b` field is not copied, and the default value for an
\xcd`Int`, 0, is used instead.  
Note that we could not make a transient field \xcd`c : Int{c != 0}`, since the
type has no default value, and copying would in fact set it to zero.

%%AT-COPY%% %~~gen ^^^ Places40
%%AT-COPY%% %package Places_transient_a;
%%AT-COPY%% % 
%%AT-COPY%% %~~vis
%%AT-COPY%% \begin{xten}
%%AT-COPY%% class Trans {
%%AT-COPY%%    val a : Int = 1;
%%AT-COPY%%    transient val b : Int = 2;
%%AT-COPY%%    //ERROR transient val c : Int{c != 0} = 3;
%%AT-COPY%%    def example() {
%%AT-COPY%%      assert(a == 1 && b == 2);
%%AT-COPY%%      at(here;a) {
%%AT-COPY%%         assert(a == 1 && b == 0);
%%AT-COPY%%      }
%%AT-COPY%%    }
%%AT-COPY%% }
%%AT-COPY%% \end{xten}
%%AT-COPY%% %~~siv
%%AT-COPY%% %class Hook{ def run() { (new Trans()).example(); return true; } }
%%AT-COPY%% %~~neg

%~~gen ^^^ Places40
%package Places_transient_a;
% 
%~~vis
\begin{xten}
class Trans {
   val a : Int = 1;
   transient val b : Int = 2;
   //ERROR: transient val c : Int{c != 0} = 3;
   def example() {
     assert(a == 1 && b == 2);
     at(here) {
        assert(a == 1 && b == 0);
     }
   }
}
\end{xten}
%~~siv
%class Hook{ def run() { (new Trans()).example(); return true; } }
%~~neg



\end{ex}

\subsection{Copying and GlobalRef}
\label{GlobalRef}
\index{at!GlobalRef}
\index{at!blocking copying}

%%The other barrier to the potentially copious copying behavior of \xcd`at`
%%is the \xcd`GlobalRef` struct.  
A \xcd`GlobalRef[T]` (say \xcd`g`) contains a reference to
a value \xcd`v` of type \xcd`T`, in a form which can be transmitted, and a \xcd`Place`
\xcd`g.home` indicating where the value lives. When a 
\xcd`GlobalRef` is serialized an opaque, globally unique handle to
\xcd`v` is created.  

\begin{ex}The following example does not copy the value \xcd`huge`.  However, \xcd`huge`
would have been copied if it had been put into a \xcd`Cell`, or simply used
directly. 

%%AT-COPY%% %~~gen ^^^ Places50
%%AT-COPY%% %package Places.copyingblockingwithglobref;
%%AT-COPY%% % class GR {
%%AT-COPY%% %  static def use(Any){}
%%AT-COPY%% %  static def example() {
%%AT-COPY%% % 
%%AT-COPY%% %~~vis
%%AT-COPY%% \begin{xten}
%%AT-COPY%% val huge = "A potentially big thing";
%%AT-COPY%% val href = GlobalRef(huge);
%%AT-COPY%% at (here;href) {
%%AT-COPY%%    use(href);
%%AT-COPY%%   }
%%AT-COPY%% }
%%AT-COPY%% \end{xten}
%%AT-COPY%% %~~siv
%%AT-COPY%% %}
%%AT-COPY%% % class Hook{ def run() { GR.example(); return true; } }
%%AT-COPY%% %~~neg

%~~gen ^^^ Places50
%package Places.copyingblockingwithglobref;
% class GR {
%  static def use(Any){}
%  static def example() {
% 
%~~vis
\begin{xten}
val huge = "A potentially big thing";
val href = GlobalRef(huge);
at (here) {
   use(href);
  }
}
\end{xten}
%~~siv
%}
% class Hook{ def run() { GR.example(); return true; } }
%~~neg


\end{ex}

Values protected in \xcd`GlobalRef`s can be retrieved by the application
%~~exp~~`~~`~~ g:GlobalRef[Any]{here == g.home}~~ ^^^Places4e7q
operation \xcd`g()`.  \xcd`g()` is guarded; it can 
only be called when \xcd`g.home == here`.  If you  want to do anything other
than pass a global reference around or compare two of them for equality, you
need to placeshift back to the home place of the reference, often with
\xcd`at(g.home)`.   

\begin{ex}The following program, for reasons best known to the programmer,
modifies the 
command-line argument array.

%%AT-COPY%% 
%%AT-COPY%% %~~gen ^^^ Places60
%%AT-COPY%% % package Places.Atsome.Globref2;
%%AT-COPY%% % class GR2 {
%%AT-COPY%% % 
%%AT-COPY%% %~~vis
%%AT-COPY%% \begin{xten}
%%AT-COPY%%   public static def main(argv:Rail[String]) {
%%AT-COPY%%     val argref = GlobalRef[Rail[String]](argv);
%%AT-COPY%%     at(here.next(); argref) 
%%AT-COPY%%         use(argref);
%%AT-COPY%%   }
%%AT-COPY%%   static def use(argref : GlobalRef[Rail[String]]) {
%%AT-COPY%%     at(argref.home; argref) {
%%AT-COPY%%       val argv = argref();
%%AT-COPY%%       argv(0) = "Hi!";
%%AT-COPY%%     }
%%AT-COPY%%   }
%%AT-COPY%% \end{xten}
%%AT-COPY%% %~~siv
%%AT-COPY%% %} 
%%AT-COPY%% % class Hook{ def run() { GR2.main(["what, me weasel?" as String]); return true; }}
%%AT-COPY%% %~~neg
%%AT-COPY%% 

%~~gen ^^^ Places60
% package Places.Atsome.Globref2;
% class GR2 {
% 
%~~vis
\begin{xten}
  public static def main(argv:Rail[String]) {
    val argref = GlobalRef[Rail[String]](argv);
    at(here.next()) 
        use(argref);
  }
  static def use(argref : GlobalRef[Rail[String]]) {
    at(argref) {
      val argv = argref();
      argv(0) = "Hi!";
    }
  }
\end{xten}
%~~siv
%} 
% class Hook{ def run() { GR2.main(["what, me weasel?" as String]); return true; }}
%~~neg

\end{ex}

There is an implicit coercion from \xcd`GlobalRef[T]` to \xcd`Place`, so
\xcd`at(argref)S` goes to \xcd`argref.home`.  


\subsection{Warnings about \xcd`at`}
There are two dangers involved with \xcd`at`: 
\begin{itemize}
\item Careless use of \xcd`at` can result in copying and transmission
of very large data structures.  
%%AT-COPY%% This is particularly an issue with the blanket
%%AT-COPY%% \xcd`at` statement, \xcd`at(p)S`, where everything used in \xcd`S` is copied.  
In particular, it is very easy to capture
\xcd`this` -- a field reference will do it -- and accidentally copy everything
that \xcd`this` refers to, which can be very large.  A disciplined use of copy
specifiers to make explicit just what gets copied can ameliorate this issue.

\item As seen in the examples above, a local variable reference
  \xcd`x` may refer to different objects in different nested \xcd`at`
  scopes. The programmer must either ensure that a variable accessed
  across an \xcd`at` boundary has no mutable state or be prepared to
  reason about which copy gets modified.   A disciplined use of copy specifiers to give
  different names to variables can ameliorate this concern.
\end{itemize}


%%AT-COPY%% \section{{\tt athome}: Returning Values from {\tt at}-Blocks}
%%AT-COPY%% \label{sect:athome}
%%AT-COPY%% \index{athome}
%%AT-COPY%% 
%%AT-COPY%% The 
%%AT-COPY%% \xcd`at(p;F)S` 
%%AT-COPY%% construct renders external variables unavailable within
%%AT-COPY%% \xcd`S`.  However, it is often useful to transmit values back from \xcd`S`,
%%AT-COPY%% and store them in external variables. 
%%AT-COPY%% 
%%AT-COPY%% The \xcd`athome(V;F)S` construct provides
%%AT-COPY%% this ability.  \xcd`V` is a list of variables, which must all be defined at
%%AT-COPY%% the same place.  \xcd`athome(V;F)S` goes to the place where the variables are
%%AT-COPY%% defined, copying \xcd`F` as for \xcd`at(p;F)S`, and executes \xcd`S` ---
%%AT-COPY%% allowing reading, assignment and initialization of the listed variables in
%%AT-COPY%% \xcd`V`. 
%%AT-COPY%% 
%%AT-COPY%% \xcd`V`, the list of variables, may include one or more variables.  It is a
%%AT-COPY%% static error if X10 cannot determine that all the variables in the list are
%%AT-COPY%% defined at the same place.
%%AT-COPY%% 
%%AT-COPY%% 
%%AT-COPY%% 
%%AT-COPY%% 
%%AT-COPY%% \begin{ex}
%%AT-COPY%% \xcd`athome` allows returning multiple pieces of information from an
%%AT-COPY%% \xcd`at`-statement.  In the following example, we return two data: 
%%AT-COPY%% one as a \xcd`val` named \xcd`square`, and the other as an addition in to a
%%AT-COPY%% partially-computed polynomial named \xcd`poly`.  
%%AT-COPY%% %~~gen ^^^ Places5f9g
%%AT-COPY%% % package Places5f9g;
%%AT-COPY%% % % KNOWNFAIL-at
%%AT-COPY%% % class Example { 
%%AT-COPY%% %~~vis
%%AT-COPY%% \begin{xten}
%%AT-COPY%% static def example(a: Int, mathProc: Place) { 
%%AT-COPY%%   val square : Int;
%%AT-COPY%%   var poly : Int = 1 + a; // will be 1+a+a*a
%%AT-COPY%%   at(mathProc; a) {
%%AT-COPY%%     val sq = a*a; 
%%AT-COPY%%     athome(square, poly; sq) {
%%AT-COPY%%        square = sq;  // initialization
%%AT-COPY%%        poly += sq;   // read and update
%%AT-COPY%%     }
%%AT-COPY%%   return [square, poly];
%%AT-COPY%%   }
%%AT-COPY%% \end{xten}
%%AT-COPY%% %~~siv
%%AT-COPY%% %}}
%%AT-COPY%% % class Hook { def run() { 
%%AT-COPY%% %   val e = example(2, here);
%%AT-COPY%% %   assert e(0) == 4 && e(1) == 7;
%%AT-COPY%% %   return true;
%%AT-COPY%% % }} 
%%AT-COPY%% %~~neg
%%AT-COPY%% \end{ex}
%%AT-COPY%% 
%%AT-COPY%% The abbreviated forms 
%%AT-COPY%% \xcd`athome (*) S` and 
%%AT-COPY%% \xcd`athome S` 
%%AT-COPY%% allow a block of assignments without specifying the variables being assigned
%%AT-COPY%% to, which is convenient for a small set of assignments. 
%%AT-COPY%% They 
%%AT-COPY%% are both equivalent to \xcd`athome(V;F)S`,
%%AT-COPY%% where: 
%%AT-COPY%% \begin{itemize}
%%AT-COPY%% \item \xcd`V` is the list of all variables appearing on the left-hand side of
%%AT-COPY%%       an assignment or update statement in \xcd`S`, excluding those which
%%AT-COPY%%       appear inside the body of an \xcd`at` or \xcd`athome` statement in \xcd`S`;
%%AT-COPY%% \item \xcd`F` is the same as for \xcd`at(p)S` (\Sref{sect:copy-spec})
%%AT-COPY%% \end{itemize}
%%AT-COPY%% 
%%AT-COPY%% 
%%AT-COPY%% \begin{ex}
%%AT-COPY%% 
%%AT-COPY%% Much as the blanket \xcd`at` construct \xcd`at(p)S` is convenient for
%%AT-COPY%% executing a small code body at another place, the blanket \xcd`athome`
%%AT-COPY%% construct \xcd`athome(*) S` 
%%AT-COPY%% (which may be written as simply \xcd`athome S`)
%%AT-COPY%% is convenient for returning a result or two.   The
%%AT-COPY%% preceding example could have been written using blanket statements.
%%AT-COPY%% 
%%AT-COPY%% %~~gen ^^^ Places5f9gblanket
%%AT-COPY%% % package Places5f9gblanket;
%%AT-COPY%% % class Example { 
%%AT-COPY%% % KNOWNFAIL-at
%%AT-COPY%% %~~vis
%%AT-COPY%% \begin{xten}
%%AT-COPY%% static def example(a: Int, mathProc: Place) { 
%%AT-COPY%%   val square : Int;
%%AT-COPY%%   var poly : Int = 1 + a; // will be 1+a+a*a
%%AT-COPY%%   at(mathProc) {
%%AT-COPY%%     val sq = a*a; 
%%AT-COPY%%     athome {
%%AT-COPY%%        square = sq;  // initialization
%%AT-COPY%%        poly += sq;   // read and update
%%AT-COPY%%     }
%%AT-COPY%%   return [square, poly];
%%AT-COPY%%   }
%%AT-COPY%% \end{xten}
%%AT-COPY%% %~~siv
%%AT-COPY%% %}}
%%AT-COPY%% % class Hook { def run() { 
%%AT-COPY%% %   val e = example(2, here);
%%AT-COPY%% %   assert e(0) == 4 && e(1) == 7;
%%AT-COPY%% %   return true;
%%AT-COPY%% % }} 
%%AT-COPY%% %~~neg
%%AT-COPY%% \end{ex}
%%AT-COPY%% 
%%AT-COPY%% {\bf Design:} It is not fundamentally essential to distinguish \xcd`at` from
%%AT-COPY%% \xcd`athome`.  \xcd`at(p;F)S` could allow writing to variables whose homes are
%%AT-COPY%% known at compile-time to be equal to \xcd`p`.  Indeed, in earlier versions of
%%AT-COPY%% X10, it did so.    This required an idiom in which programmers had to manage
%%AT-COPY%% the home locations of variables directly, and keep track of which home
%%AT-COPY%% location corresponded to which variable.  The \xcd`athome` construct makes
%%AT-COPY%% this idiom more convenient. 
	
\chapter{Activities}\label{XtenActivities}

An {\em activity} is a statement being executed, independently, with its own
local variables; it may be thought of as a very light-weight thread. An
\Xten{} computation may have many concurrent {\em activities} executing at any
give time.  All X10 code runs as part of an activity; when an X10 program is
started, the \xcd`main` method is invoked in an activity, called the {\em root
activity}.\index{root
activity}


Activities coordinate their execution by various control and data structures.
For example, `
%~~stmt~~`~~`~~x:Int, var y:Int ~~
\xcd`when(x==0);` blocks the current activity until some other activity
sets \xcd`x` to zero.  However, activities determine the places at which they
may be blocked and resumed, by \xcd`when` and similar constructs.  There are
no means by which one activity can arbitrarily interrupt, block, or resume
another, no method  \xcd`activity.interrupt()`.

An activity may be {\em running}, {\em blocked} on some condition or {\em
terminated}. If terminated, it is terminated in the same way that its
statement is: in particular, if the statement terminates abruptly, the
activity terminates abruptly for the same reason.
(\Sref{ExceptionModel}).

Activities can be long-running entities with a good deal of local state.  In
particular they can involve recursive method calls (and therefore have runtime
stacks).  However, activities can also be short-running light-weight entities,
\eg, it is reasonable to have an activity that simply increments a variable.

An activity may asynchronously and in parallel launch activities at
other places.  Every activity save the initial \xcd`main` activity is spawned
by another.  Thus, at any instant, the activities in a program form a tree.

X10 uses this tree in crucial ways.  
First is the distinction 
between {\em local} termination and {\em global}
termination of a statement. The execution of a statement by an
activity is said to terminate locally when the activity has finished
all its computation. (For instance the
creation of an asynchronous activity terminates locally when the
activity has been created.)  It is said to terminate globally when it
has terminated locally and all activities that it may have spawned at
any place have, recursively, terminated globally.
For example, consider: 
%~~gen
% package Activites.Are.For.Whacktivities;
% class Example {
% def example( s1:() => Void, s2 : () => Void ) {
%~~vis
\begin{xten}
async {s1();}
async {s2();}
\end{xten}
%~~siv
% } } 
%~~neg
The primary activity spawns two child activities and then terminates locally,
very quickly.  The child activities may take arbitrary amounts of time to
terminate (and may spawn grandchildren).  When \xcd`s1()`, \xcd`s2()`, and
all their descendants terminate locally, then the primary activity terminates
globally. 

The program as a whole terminates when the root activity terminates globally.
In particular, X10 does not permit the creation of 
daemon threads---threads that outlive the lifetime of the root
activity.  We say that an \Xten{} computation is {\em rooted}
(\Sref{initial-computation}).

\futureext{ We may permit the initial activity to be a daemon activity
to permit reactive computations, such as webservers, that may not
terminate.}

\section{The \Xten{} rooted exception model}
\label{ExceptionModel}
\index{Exception!model}

The rooted nature of \Xten{} computations permits the definition of a
{\em rooted exception model.} In multi-threaded programming languages
there is a natural parent-child relationship between a thread and a
thread that it spawns. Typically the parent thread continues execution
in parallel with the child thread. Therefore the parent thread cannot
serve to catch any exceptions thrown by the child thread. 

The presence of a root activity and the concept of global termination permits
\Xten{} to adopt a more powerful exception model. In any state of the
computation, say that an activity $A$ is {\em a root of} an activity $B$ if
$A$ is an ancestor of $B$ and $A$ is blocked at a statement (such as the
\xcd"finish" statement \Sref{finish}) awaiting the termination of $B$ (and
possibly other activities). For every \Xten{} computation, the \emph{root-of}
relation is guaranteed to be a tree. The root of the tree is the root activity
of the entire computation. If $A$ is the nearest root of $B$, the path from
$A$ to $B$ is called the {\em activation path} for the activity.\footnote{Note
  that depending on the state of the computation the activation path may
  traverse activities that are running, blocked or terminated.}

We may now state the exception model for \Xten.  An uncaught exception
propagates up the activation path to its nearest root activity, where
it may be handled locally or propagated up the \emph{root-of} tree when
the activity terminates (based on the semantics of the statement being
executed by the activity).\footnote{In \XtenCurrVer{} the \xcd"finish"
statement is the only statement that marks its activity as a root
activity. Future versions of the language may introduce more such
statements.}  In Java, exceptions may be overlooked because there is no good
place to put a \xcd`try`-\xcd`catch` block; this is never the case in X10.

\section{\xcd`at`: Place changing}\label{AtStatement}

An activity may change place using the \xcd"at" statement or
\xcd"at" expression:

\begin{grammar}
Statement \: AtStatement \\
AtStatement \: \xcd"at" PlaceExpressionSingleList Statement \\
Expression \: AtExpression \\
AtExpression \: \xcd"at" PlaceExpressionSingleList ClosureBody 
\end{grammar}

The statement \xcd"at (p) S" executes the statement \xcd"S"
synchronously at a place described by \xcd"p".
The expression \xcd"at (p) E" executes the statement \xcd"E"
synchronously at place \xcd"p", returning the result to the
originating place.  




\xcd`p` may be an expression of type \xcd`Place`, in which case its value is
used as the place to execute the body: 
%~~gen
% package Activities.At.A.Standstill;
% class Example {
% def example(ob: Object, S: ()=>Void) {
%~~vis
\begin{xten}
   at (here.next()) S();
\end{xten}
%~~siv
% } } 
%~~neg
\noindent



\xcd`at(p)S` does {\em not} start a new activity.  It should be thought of as
transporting the current activity to \xcd`p`, running \xcd`S` there, and then
transporting it back.    If you want to start a new activity, use \xcd`async`;
if you want to start a new activity at \xcd`p`, use 
\xcd`at(p) async S`.  

As a consequence of this, \xcd`S` may contain constructs which only make sense
within a single activity.  
For example, 
\begin{xten}
    for(x in globalRefsToThings) 
    if (at(x.home) x().nice()) 
        return x();
\end{xten}
returns the first nice thing in a collection.   If we had used 
\xcd`async at(x.home)`, this would not be allowed; 
you can't \xcd`return` from an
\xcd`async`. 



\section{\xcd`async`: Spawning an activity}\label{AsynchronousActivity}\label{AsyncActivity}

Asynchronous activities serve as a single abstraction for supporting a
wide range of concurrency constructs such as message passing, threads,
DMA, streaming, data prefetching. (In general, asynchronous operations
are better suited for supporting scalability than synchronous
operations.)

An activity is created by executing the \xcd`async` statement: 

\begin{grammar}
Statement \: AsyncStatement \\
AsyncStatement \: \xcd"async"  Statement \\
PlaceExpressionSingleList \: \xcd"(" PlaceExpression \xcd")" \\
PlaceExpression \: Expression 
\end{grammar} 


The basic form of \xcd`async` is \xcd`async S`, which starts a new activity
located \xcd`here` executing \xcd`S`.   


\bard{The followingin para is under investigation:}
In many cases the compiler may infer the unique place at which the
statement is to be executed by an analysis of the types of the
variables occurring in the statement. (The place must be such that the
statement can be executed safely, without generating a
\xcd"BadPlaceException".) In such cases the programmer may omit the
place designator; the compiler will throw an error if it cannot
determine the unique designated place.\footnote{\XtenCurrVer{} does
not specify a particular algorithm; this will be fixed in future
versions.}

An activity $A$ executes the statement \xcd"async (P) S" by launching
a new activity $B$ at place \xcd`P` (or \xcd`P.home` if \xcd`P` is of an
object type), to execute \xcd`S`. The statement terminates locally as soon as $B$ is
launched.  The activation path for $B$ is that of $A$ augmented by the
information that {$A$} is the parent of {$B$}. 
$B$
terminates normally when $S$ terminates normally.  It terminates
abruptly if $S$ throws an uncaught exception. The exception is
propagated to $A$ if $A$ is a root activity (see \Sref{finish}),
otherwise it is propagated through $A$ to $A$'s root activity. Note that while
{$A$} is running, exceptions thrown by activities it has already
spawned may propagate through it up to its root activity, without {$A$} noticing.

Multiple activities launched by a single activity at another place are not
ordered in any way. They are added to the set of activities at the target
place and will be executed based on the local scheduler's decisions.
If some particular sequencing of events is needed, \xcd`when`, \xcd`atomic`,
\xcd`finish`, clocks, and other X10 constructs can be used.
\Xten{} implementations are not required to have fair schedulers,
though every implementation should make a best faith effort to ensure
that every activity eventually gets a chance to make forward progress.

\begin{staticrule*}
The statement in the body of an \xcd"async" is subject to the
restriction that it must be acceptable as the body of a \xcd"void"
method for an anonymous inner class declared at that point in the code,
which throws no checked exceptions. As such, it may reference
variables in lexically enclosing scopes (including \xcd"clock"
variables, \Sref{XtenClocks}) provided that such variables are
(implicitly or explicitly) \xcd"val".
\end{staticrule*}

\section{Finish}\index{finish}\label{finish}
The statement \xcd"finish S" converts global termination to local
termination and introduces a root activity.   It executes \xcd`S`, and then
waits for all activities spawned by \xcd`S`, directly or indirectly, to
finish. It also collects exceptions thrown by those activities.

\begin{grammar}
Statement \: FinishStatement \\
FinishStatement \: \xcd"finish" Statement 
\end{grammar}

An activity $A$ executes \xcd"finish S" by executing \xcd"S".  The
execution of \xcd"S" may spawn other asynchronous activities (here or
at other places).  Uncaught exceptions thrown or propagated by any
activity spawned by \xcd"S" are accumulated at \xcd"finish S".
\xcd"finish S" terminates locally when all activities spawned by
\xcd"S" terminate globally (either abruptly or normally). If \xcd"S"
terminates normally, then \xcd"finish S" terminates normally and $A$
continues execution with the next statement after \xcd"finish S".  If
\xcd"S" terminates abruptly, then \xcd"finish S" terminates abruptly
and throws a single exception, \Xcd{x10.lang.MultipleExceptions}
formed from the collection of exceptions accumulated at \xcd"finish S".

Thus a \xcd"finish S" statement serves as a collection point for
uncaught exceptions generated during the execution of \xcd"S".

Note that repeatedly \xcd"finish"ing a statement has little effect after
the first \xcd"finish": \xcd"finish finish S" is indistinguishable
from \xcd"finish S" if \xcd`S` throws no exceptions.  (If \xcd`S` throws
exceptions, \xcd`finish S` wraps them in one layer of 
\xcd`MultipleExceptions` and \xcd`finish finish S` in two layers.)

%%OLIVIER-DENIES%% \paragraph{Interaction with clocks.}\label{sec:finish:clock-rule}
%%OLIVIER-DENIES%% 
%%OLIVIER-DENIES%% \xcd"finish S" interacts with clocks (\Sref{XtenClocks}). 
%%OLIVIER-DENIES%% While executing \xcd"S", an activity must not spawn any \xcd"clocked"
%%OLIVIER-DENIES%% asyncs. (Asyncs spawned during the execution of \xcd"S" may spawn
%%OLIVIER-DENIES%% clocked asyncs.) A
%%OLIVIER-DENIES%% \xcd"ClockUseException"\index{clock!ClockUseException} is thrown
%%OLIVIER-DENIES%% if (and when) this condition is violated.
%%OLIVIER-DENIES%% 
%%OLIVIER-DENIES%% This is necessary to prevent deadlocks.  In the following invalid code 
%%OLIVIER-DENIES%% %~s~gen
%%OLIVIER-DENIES%% % package Activities.Finish.Hates.Clocks;
%%OLIVIER-DENIES%% % class Example{
%%OLIVIER-DENIES%% % def example() {
%%OLIVIER-DENIES%% %~s~vis
%%OLIVIER-DENIES%% \begin{xten}
%%OLIVIER-DENIES%% val c:Clock = Clock.make();
%%OLIVIER-DENIES%% async clocked(c) {                // (A) 
%%OLIVIER-DENIES%%       finish async clocked(c) {   // (B) INVALID
%%OLIVIER-DENIES%%             next;                 // (Bnext)
%%OLIVIER-DENIES%%       }
%%OLIVIER-DENIES%%       next;                       // (Anext)
%%OLIVIER-DENIES%% }
%%OLIVIER-DENIES%% \end{xten}
%%OLIVIER-DENIES%% %~s~siv
%%OLIVIER-DENIES%% % } } 
%%OLIVIER-DENIES%% %~s~neg
%%OLIVIER-DENIES%% \xcd`(A)`, first of all, waits for the \xcd`finish` containing \xcd`(B)` to
%%OLIVIER-DENIES%% finish.  
%%OLIVIER-DENIES%% \xcd`(B)` will execute its \xcd`next` at \xcd`(Bnext)`, and then wait for all
%%OLIVIER-DENIES%% other activities registered on \xcd`c` to execute their \xcd`next`s.
%%OLIVIER-DENIES%% However, \xcd`(A)` is registered on \xcd`c`.  So, \xcd`(B)` cannot finish
%%OLIVIER-DENIES%% until \xcd`(A)` has proceeded to \xcd`(Anext)`, and \xcd`(A)` cannot proceed
%%OLIVIER-DENIES%% until \xcd`(B)` finishes. Thus, this causes deadlock.
%%OLIVIER-DENIES%% 
%%OLIVIER-DENIES%% 
%%OLIVIER-DENIES%% 
%%OLIVIER-DENIES%% In \XtenCurrVer{} this condition is checked dynamically; future
%%OLIVIER-DENIES%% versions of the language will introduce type qualifiers which permit
%%OLIVIER-DENIES%% this condition to be checked statically.
%%OLIVIER-DENIES%% 
%%OLIVIER-DENIES%% \futureext{
%%OLIVIER-DENIES%% The semantics of \xcd"finish S" is conjunctive; it terminates when all
%%OLIVIER-DENIES%% the activities created during the execution of \xcd"S" (recursively)
%%OLIVIER-DENIES%% terminate. In many situations (e.g., nondeterministic search) it is
%%OLIVIER-DENIES%% natural to require a statement to terminate when any {\em one} of the
%%OLIVIER-DENIES%% activities it has spawned succeeds. The other activities may then be
%%OLIVIER-DENIES%% safely aborted. Future versions of the language may introduce a
%%OLIVIER-DENIES%% \xcd"finishone S" construct to support such speculative or nondeterministic
%%OLIVIER-DENIES%% computation.
%%OLIVIER-DENIES%% }
%%OLIVIER-DENIES%% 



\section{Initial activity}\label{initial-computation}\index{initial activity}

An \Xten{} computation is initiated from the command line on the
presentation of a classname \xcd"C". The class must have a
\xcd"public static def main(a: Rail[String]):Void" method, otherwise an
exception is thrown
and the computation terminates.  The single statement
\begin{xten}
finish async (Place.FIRST_PLACE) {
  C.main(s);
}
\end{xten} 
\noindent is executed where \xcd"s" is an Rail of strings created
from the command line arguments. This single activity is the root activity
for the entire computation. (See \Sref{XtenPlaces} for a discussion of
places.)

%% Say something about configuration information? 




\section{Ateach statements}\index{\Xcd{ateach}}\label{ateach-section}

\begin{grammar}
Statement \: AtEachStatement \\
AtEachStatement \:
      \xcd"ateach" \xcd"(" Formal \xcd"in" Expression \xcd")"
         Statement \\
AtEachStatement \:
      \xcd"ateach" \xcd"(" Expression \xcd")"
         Statement 
\end{grammar}

The \xcd"ateach" statement \xcd`ateach (p in D) S`
spawns an activity \xcd`S` at each place \xcd`p` of a distribution \xcd`D`. 
In \xcd`ateach(p in D) S`, 
\xcd`D` must be either of type \xcd"Dist" 
(see \Sref{XtenDistributions})
or of type
\xcd`DistArray[T]` (see \Sref{XtenArrays}), 
and \xcd`p` will be of type \xcd"Point" (see \Sref{point-syntax}).

\xcd`ateach(p in D)S` is equivalent to 
\xcd`for(p in D) at(D(p)) async S`.  That is, the elements of \xcd`D` are all
points \xcd`p`.  \xcd`D(p)` is a \xcd`Place`.  \xcd`ateach(p in D)S` executes
the body \xcd`S` at the place \xcd`D(p)` (and may use the point \xcd`p`
there). 


However, the compiler may implement it more efficiently to avoid extraneous
communications.  In particular, it is recommended that \xcd`ateach(p in D)S`
be implemented as the following code, which coordinates with each place of
\xcd`D` just once, rather than once per element of \xcd`D` at that place: 

%~~gen
% package Activities.Activities.Activities;
% class EquivCode {
% static def S(pt:Point) {}
% static def example(D:Dist) {
%~~vis
\begin{xten}
for (p in D.places()) async at (p) {
    for (pt in D|here) async {
        S(pt);
    }
}
\end{xten}
%~~siv
%}} 
%~~neg

If \xcd`e` is an \xcd`DistArray[T]`, then \xcd`ateach (p in e)S` is identical to
\xcd`ateach(p in e.dist)S`; the iteration is over the array's underlying
distribution.   
The code below is a common and generally efficient way to work with the
elements of a distributed array:
%~~gen
%package Activities.For.Fnu.And.Pforit;
%class Example[T]{
%  def dealWith(T):Void = {}
% def idiom(A:DistArray[T]){
%~~vis
\begin{xten}
ateach(p in A) 
  dealWith(A(p));
\end{xten}
%~~siv
%}}
%~~neg








\section{At expressions}

\begin{grammar}
Expression \: \xcd"at" \xcd"(" Expression \xcd")" Expression
\end{grammar}

An \Xcd{at} expression evaluates an expression synchronously at the
given place and returns its value.  For instance a copy of the
value pointed to by a \Xcd{GlobalRef} may be obtained using
the \Xcd{fetch} method:
%~~gen
% package Activities.AtExpressions.Fetching;
% class Example[T] {
%~~vis
\begin{xten}
  def fetch(g:GlobalRef[T]):T = at (g) g();
\end{xten}
%~~siv
% } 
%~~neg

The expression evaluation may spawn asynchronous activities. The \Xcd{at}
expression will return without waiting for those activities to terminate. That
is, \Xcd{at} does not have built-in \Xcd{finish} semantics.

\section{Atomic blocks}\label{AtomicBlocks}\index{atomic blocks}
Languages such as \java{} use low-level synchronization locks to allow
multiple interacting threads to coordinate the mutation of shared
data. \Xten{} eschews locks in favor of a very simple high-level
construct, the {\em atomic block}.

A programmer may use atomic blocks to guarantee that invariants of
shared data-structures are maintained even as they are being accessed
simultaneously by multiple activities running in the same place.  

For example, consider a class \xcd`Redund[T]`, which encapsulates a list
\xcd`list` and, (redundantly) keeps the size of the list in a second field
\xcd`size`.  Then \xcd`r:Redund[T]` has the invariant 
\xcd`r.list.size() == r.size`, which must be true at any point that there are
no method calls on \xcd`r` active.

If the \xcd`add` method on \xcd`Redund` (which adds an element to the list) 
were defined as: 
%~~gen
% package Activities.Atomic.Redund.One;
% import x10.util.*;
% class Redund[T] {
%   val list = new ArrayList[T]();
%   var size : Int = 0;
%~~vis
\begin{xten}
def add(x:T) { // Incorrect
  this.list.add(x);
  this.size = this.size + 1;
}
\end{xten}
%~~siv
%}
%~~neg
Then two activities simultaneously adding elements to the same \xcd`r` could break the
invariant.  Suppose that \xcd`r` starts out empty.  Let the first activity
perform the \xcd`list.add`, and compute \xcd`this.size+1`, which is 1, but not store it
back into \xcd`this.size` yet.  
(At this point, \xcd`r.list.size()==1` and \xcd`r.size==0`; the invariant
expression is false, but, as the first call to \xcd`r.add()` is active, the
invariant does not need to be true -- it only needs to be true when the
call finishes.)
Now, let the second activity do its call to
\xcd`add` to completion, which finishes with \xcd`r.size==1`.  
(As before, the invariant expression is false, but a call to \xcd`r.add()` is
still active, so the invariant need not be true.)
Finally, let
the first activity finish, which assigns the \xcd`1` computed before back into
\xcd`this.size`.  At the end, there are two elements in \xcd`r.list`, but
\xcd`r.size==1`. Since there are no calls to \xcd`r.add()` active, the
invariant must be true, but it is not.

In this case, the invariant can be maintained by making the increment atomic.
Doing so forbids that sequence of events; the \xcd`atomic` block cannot be
stopped partway.  
%~~gen
% package Activities.Atomic.Redund.Two;
% import x10.util.*;
% class Redund[T] {
%   val list = new ArrayList[T]();
%   var size : Int = 0;
%~~vis
\begin{xten}
def add(x:T) { 
  this.list.add(x);
  atomic { this.size = this.size + 1; }
}
\end{xten}
%~~siv
%}
%~~neg



\subsection{Unconditional atomic blocks}
The simplest form of an atomic block is the {\em unconditional
atomic block}:

\begin{grammar}
Statement \: AtomicStatement \\
AtomicStatement \: \xcd"atomic"  Statement \\
MethodModifier \: \xcd"atomic" \\
\end{grammar}

For the sake of efficient implementation \XtenCurrVer{} requires
that the atomic block be {\em analyzable}, that is, the set of
locations that are read and written by the \grammarrule{BlockStatement} are
bounded and determined statically.\footnote{A static bound is a constant
that depends only on the program text, and is independent 
of any runtime parameters. }
The exact algorithm to be used by
the compiler to perform this analysis will be specified in future
versions of the language.
\tbd{}

Such a statement is executed by an activity as if in a single step
during which all other concurrent activities in the same place are
blocked. If execution of the statement may throw an exception, it is
the programmer's responsibility to wrap the atomic block within a
\xcd"try"/\xcd"finally" clause and include undo code in the finally
clause. Thus the \xcd"atomic" statement only guarantees atomicity on
successful execution, not on a faulty execution.


We allow methods of an object to be annotated with \xcd"atomic". Such
a method is taken to stand for a method whose body is wrapped within an
\xcd"atomic" statement.

Atomic blocks are closely related to non-blocking synchronization
constructs \cite{herlihy91waitfree}, and can be used to implement 
non-blocking concurrent algorithms.

\begin{staticrule*}
In \xcd"atomic S", \xcd"S" may include calls to \xcd`safe` methods, and use of
sequential control structures.

It may {\em not} include an \xcd"async" activity (such as creation
of a \Xcd{future}).

It may {\em not} include any statement that may potentially block at
runtime (\eg, \xcd"when", \xcd"force" operations, \xcd"next"
operations on clocks, \xcd"finish"). 

It may {\em not} include any \xcd`at` expressions or
statements. (Hence all locations accessed in the atomic block must
belong to the current place.)
\index{locality condition}\label{LocalityCondition} 

\end{staticrule*}

The compiler checks for this condition by checking whether the statement
could be the body of a \xcd"void" method annotated with \xcd"safe" at
that point in the code (\Sref{SafeAnnotation}).

\paragraph{Consequences.}
Note an important property of an (unconditional) atomic block:

\begin{eqnarray}
 \mbox{\xcd"atomic \{s1; atomic s2\}"} &=& \mbox{\xcd"atomic \{s1; s2\}"}
\end{eqnarray}

Atomic blocks do not introduce deadlocks.    They may exhibit all the bad
behavior of sequential programs, including throwing exceptions and running
forever, but they are guaranteed not to deadlock.


\subsubsection{Example}

The following class method implements a (generic) compare and swap (CAS) operation:


%~~gen
% package Activities.And.Protein;
% class CASSizer{
%~~vis
\begin{xten}
var target:Object = null;
public atomic def CAS(old1: Object, new1: Object): Boolean {
   if (target.equals(old1)) {
     target = new1;
     return true;
   }
   return false;
}
\end{xten}
%~~siv
%}
%~~neg

\subsection{Conditional atomic blocks}

Conditional atomic blocks allow the activity to wait for some condition to be
satisfied before executing an atomic block. For example, consider a
\xcd`Redund` class holding a list \xcd`r.list` and, redundantly, its length
\xcd`r.size`.  A \xcd`pop` operation will delay until the \xcd`Redund` is
nonempty, and then remove an element and update the length.  
%~~gen
% package Activities.Condato.Example.Not.A.Tree;
% import x10.util.*;
% class Redund[T] {
% val list = new ArrayList[T]();
% var size : Int = 0;
%~~vis
\begin{xten}
def pop():T {
  var ret : T;
  when(size>0) {
    ret = list.removeAt(0);
    size --;
    }
  return ret;
}
\end{xten}
%~~siv
% }
%~~neg


The execution of the test is atomic with the execution of the block.  This is
important; it means that no other activity can sneak in and make the condition
be false before the block is executed.  In this example, two \xcd`pop`s
executing on a list with one element would work properly. Without the
conditional atomic block -- even doing the decrement atomically -- one call to
\xcd`pop` could pass the \xcd`size>0` guard; then the other call could run to
completion (removing the only element of the list); then, when the first call
proceeds, its \xcd`removeAt` will fail.  

Note that \xcd`if` would not work here.  
\xcd`if(size>0) atomic{size--; return list.removeAt(0);}` allows another
activity to act between the test and the atomic block.  
And 
\xcd`atomic{ if(size>0) {size--; ret = list.removeAt(0);}}` 
does not wait for \xcd`size>0` to become true.


Conditional atomic blocks are of the form:

\begin{grammar}
Statement \:  WhenStatement \\
WhenStatement \:  \xcd"when" \xcd"(" Expression \xcd")" Statement \\
            \| WhenStatement \xcd"or" \xcd"(" Expression \xcd")" Statement 
\end{grammar}

In such a statement the one or more expressions are called {\em
guards} and must be \xcd"Boolean" expressions. The statements are the
corresponding {\em guarded statements}.  

An activity executing such a statement suspends until such time as any
one of the guards is true in the current state. In that state, the
statement corresponding to the first guard that is true is executed.
The checking of the guards and the execution of the corresponding
guarded statement is done atomically. 

\Xten{} does not guarantee that a conditional atomic block
will execute if its condition holds only intermittently. For, based on
the vagaries of the scheduler, the precise instant at which a
condition holds may be missed. Therefore the programmer is advised to
ensure that conditions being tested by conditional atomic blocks are
eventually stable, \ie, they will continue to hold until the block
executes (the action in the body of the block may cause the condition
to not hold any more).

%%Fourth, \Xten{} does not guarantees only {\em weak fairness} when executing
%%conditional atomic blocks. Let $c$ be the guard of some conditional
%%atomic block $A$. $A$ is required to make forward progress only if
%%$c$ is {\em eventually stable}. That is, any execution $s_1, s_2,
%%\ldots$ of the program is considered illegal only if there is a $j$
%%such that $c$ holds in all states $s_k$ for $k > j$ and in which $A$
%%does not execute. Specifically, if the system executes in such a way
%%that $c$ holds only intermmitently (that is, for some state in which
%%$c$ holds there is always a later state in which $c$ does not hold),
%%$A$ is not required to be executed (though it may be executed).

\begin{rationale}
The guarantee provided by \xcd"wait"/\xcd"notify" in \java{} is no
stronger. Indeed conditional atomic blocks may be thought of as a
replacement for \java's wait/notify functionality.
\end{rationale} 


The statement \xcd"when (true) S" is
behaviorally identical to \xcd"atomic S": it never suspends.

\begin{staticrule*}
For the sake of efficient implementation certain restrictions are
placed on the guards and statements in a conditional atomic
block. 
\end{staticrule*}

Guards are statically required not to have side-effects, not to spawn
asynchronous activities (as for the \xcd`sequential` qualifier on methods) and
to have a statically determinable upper bound on their execution (as for the
\xcd`nonblocking` qualifier on methods).

The body of a \xcd"when" statement must satisfy the conditions
for the body of an \xcd"atomic" block.

Note that this implies that guarded statements are required to be {\em
flat}, that is, they may not contain conditional atomic blocks. (The
implementation of nested conditional atomic blocks may require
sophisticated operational techniques such as rollbacks.)


\begin{example}
The following class shows how to implement a bounded buffer of size
$1$ in \Xten{} for repeated communication between a sender and a
receiver.  The call \xcd`buf.send(ob)` waits until the buffer has space, and
then puts \xcd`ob` into it.  Dually, \xcd`buf.receive()` waits until the
buffer has something in it, and then returns that thing.


%~~gen
% package Activities;
%~~vis
\begin{xten}
class OneBuffer[T] {
  var datum: T;
  def this(t:T) { this.datum = t; this.filled = true; }
  var filled: Boolean;
  public def send(v: T) {
    when (!filled) {
      this.datum = v;
      this.filled = true;
    }
  }
  public def receive(): T {
    when (filled) {
      v: T = datum;
      filled = false;
      return v;
    }
  }
}
\end{xten}
%~~siv
%
%~~neg
\end{example}

	
\chapter{Clocks}\label{XtenClocks}\index{clocks}

Many concurrent algorithms proceed in phases: in phase {$k$}, several
activities work independently, but synchronize together before proceeding on
to phase {$k+1$}. X10 supports this communication structure (and many
variations on it) with a generalization of barriers 
called {\em clocks}. Clocks are designed so that programs which follow a
simple syntactic discipline will not have either deadlocks or race conditions.


The following minimalist example of clocked code has two worker activities A
and B, and three phases. In the first phase, each worker activity says its
name followed by 1; in the second phase, by a 2, and in the third, by a 3.  
So, if \xcd`say` prints its argument, 
\xcd`A-1 B-1 A-2 B-2 B-3 A-3`
would be a legitimate run of the program, but
\xcd`A-1 A-2 B-1 B-2 A-3 B-3`
(with \xcd`A-2` before \xcd`B-1`) would not.

The program creates a clock \xcd`cl` to manage the phases.  Each participating
activity does
the work of its first phase, and then executes \xcd`next;` to signal that it
is finished with that work. \xcd`next;` is blocking, and causes the participant to
wait until all participant have finished with the phase -- as measured by the
clock \xcd`cl` to which they are both registered.  
Then they do the second phase, and another \xcd`next;` to make sure that
neither proceeds to the third phase until both are ready.  This example uses
\xcd`finish` to wait for both particiants to finish.  The parent thread is also
registered on the clock just as the particiants are, and executes \xcd`next;next;`
to run through the phases.


%%TODO -- put the 'atomic' back in when that's legal.

%~~gen
%package Clocks.For.Spock;
%class ClockEx {
%  static def say(s:String) = 
% { /*atomic{x10.io.Console.OUT.println(s);}*/ }
%  public static def main(argv:Rail[String]) {
%~~vis
\begin{xten}
    finish async{
      val cl = Clock.make();
      async clocked(cl) {// Activity A
        say("A-1");
        next;
        say("A-2");
        next;
        say("A-3"); 
      }// Activity A

      async clocked(cl) {// Activity B
        say("B-1");
        next;
        say("B-2");
        next;
        say("B-3"); 
      }// Activity B
    }
\end{xten}
%~~siv
%  }
% }
%~~neg

This chapter describes the syntax and semantics of clocks and
statements in the language that have parameters of type \xcd"Clock". 

The key invariants associated with clocks are as follows.  At any
stage of the computation, a clock has zero or more {\em registered}
activities. An activity may perform operations only on those clocks it
is registered with (these clocks constitute its {\em clock set}). 
An attempt by an activity to operate on a clock it is not registered with
will cause a 
\xcd"ClockUseException"\index{clock!ClockUseException}. 
to be thrown.  
An activity is registered with zero or more clocks when it is created.
During its lifetime the only additional clocks it is registered with
are exactly those that it creates. In particular it is not possible
for an activity to register itself with a clock it discovers by
reading a data structure.

The primary operations that an activity \xcd`a` may perform on a clock \xcd`c`
that it is registered upon are: 
\begin{itemize}
\item It may spawn and simultaneously  {\em register} a new activity on
      \xcd`c`, with the statement       \xcd`async clocked(c){S}`.
\item It may {\em unregister} itself from \xcd`c`, with \xcd`c.drop()`.  After
      doing so, it can no longer use most primary operations on \xcd`c`.
\item It may {\em resume} the clock, with \xcd`c.resume()`, indicating that it
      has finished with the current phase associated with \xcd`c` and is ready
      to move on to the next one.
\item It may {\em wait} on the clock, with \xcd`c.next()`.  This first does
      \xcd`c.resume()`, and then blocks the current activity until the start
      of the next phase, \viz, until all other activities registered on that
      clock have called \xcd`c.resume()`.
\item It may {\em block} on all the clocks it is registered with
      simultaneously, by the command \xcd`next;`.  This, in effect, calls
      \xcd`c.next()` simultaneously 
      on all clocks \xcd`c` that the current activity is registered with.
\item Other miscellaneous operations are available as well; see the
      \xcd`Clock` API.
\end{itemize}

%%CLOCK%% An activity may perform the following operations on a clock \xcd"c".
%%CLOCK%% It may {\em unregister} with \xcd"c" by executing \xcd"c.drop();".
%%CLOCK%% After this, it may perform no further actions on \xcd"c"
%%CLOCK%% for its lifetime. It may {\em check} to see if it is unregistered on a
%%CLOCK%% clock. It may {\em register} a newly forked activity with \xcd"c".
%%CLOCK%% %% It may {\em post} a statement \xcd"S" for completion in the current phase
%%CLOCK%% %% of \xcd"c" by executing the statement \xcd"now(c) S". 
%%CLOCK%% Once registered and "active" (see below), it may also perform the following operations.
%%CLOCK%% It may {\em resume} the clock by executing \xcd"c.resume();". This
%%CLOCK%% indicates to \xcd"c" that it has finished posting all statements it
%%CLOCK%% wishes to perform in the current phase. Finally, it may {\em block}
%%CLOCK%% (by executing \xcd"next;") on all the clocks that it is registered
%%CLOCK%% with. (This operation implicitly \xcd"resume"'s all clocks for the
%%CLOCK%% activity.) It will resume from this statement only when all these
%%CLOCK%% clocks are ready to advance to the next phase.

%%CLOCK%% A clock becomes ready to advance to the next phase when every activity
%%CLOCK%% registered with the clock has executed at least one \xcd"resume"
%%CLOCK%% operation on that clock and all statements posted for completion in
%%CLOCK%% the current phase have been completed.

%%OLIVIER-DENIES%% Though clocks introduce a blocking statement (\xcd"next") an important
%%OLIVIER-DENIES%% property of \Xten{} is that clocks -- when used with the \xcd`next;` {\em
%%OLIVIER-DENIES%%   statement} only, without the \xcd`c.next()` method call -- cannot introduce
%%OLIVIER-DENIES%% deadlocks. That is, the system cannot reach a quiescent state (in which no
%%OLIVIER-DENIES%% activity is progressing) from which it is unable to progress. For, before
%%OLIVIER-DENIES%% blocking each activity resumes all clocks it is registered with. Thus if a
%%OLIVIER-DENIES%% configuration were to be stuck (that is, no activity can progress) all clocks
%%OLIVIER-DENIES%% will have been resumed. But this implies that all activities blocked on
%%OLIVIER-DENIES%% \xcd"next" may continue and the configuration is not stuck. The only other
%%OLIVIER-DENIES%% possibility is that an activity may be stuck on \xcd"finish". But the
%%OLIVIER-DENIES%% interaction rule between \xcd"finish" and clocks
%%OLIVIER-DENIES%% (\Sref{sec:finish:clock-rule}) guarantees that this cannot cause a cycle in
%%OLIVIER-DENIES%% the wait-for graph. A more rigorous proof may be found in \cite{X10-concur05}.

\section{Clock operations}\label{sec:clock}
There are two language constructs for working with clocks. 
\xcd`async clocked(cl) S` starts a new activity registered on one or more
clocks.  \xcd`next;` blocks the current activity until all the activities
sharing clocks with it are ready to proceed to the next clock phase. 
Clocks are objects, and have a number of useful methods on them as well.

\subsection{Creating new clocks}\index{clock!creation}\label{sec:clock:create}

Clocks are created using a factory method on \xcd"x10.lang.Clock":


%~~gen
% package Clocks.For.Spocks;
%class Clockuser {
% def example() {
%~~vis
\begin{xten}
val c: Clock = Clock.make();
\end{xten}
%~~siv
%}}
%~~neg

%%CLOCKVAR%% \eat{All clocked variables are implicitly \xcd`val`. The initializer for a
%%CLOCKVAR%% local variable declaration of type \xcd"Clock" must be a new clock
%%CLOCKVAR%% expression. Thus \Xten{} does not permit aliasing of clocks.
%%CLOCKVAR%% Clocks are created in the place global heap and hence outlive the
%%CLOCKVAR%% lifetime of the creating activity.  Clocks are structs, hence may be freely
%%CLOCKVAR%% copied from place to 
%%CLOCKVAR%% place. (Clock instances typically contain references to mutable state
%%CLOCKVAR%% that maintains the current state of the clock.)
%%CLOCKVAR%% }

The current activity is automatically registered with the newly
created clock.  It may deregister using the \xcd"drop" method on
clocks (see the documentation of \xcd"x10.lang.Clock"). All activities
are automatically deregistered from all clocks they are registered
with on termination (normal or abrupt).

\subsection{Registering new activities on clocks}
\index{clock!clocked statements}\label{sec:clock:register}

The statement 

%~~gen
%package Clocks.For.Jocks;
%class Qlocked{
%static def S():void{}
%static def flock() { 
% val c1 = Clock.make(), c2 = Clock.make(), c3 = Clock.make();
%~~vis
\begin{xten}
  async clocked (c1, c2, c3) S
\end{xten}
%~~siv
%();
%}}
%~~neg
starts a new activity, initially registered with
clocks \xcd`c1`, \xcd`c2`, and \xcd`c3`, and  running \xcd`S`. The activity running this code must
be registered on those clocks. 
Violations of these conditions are punished by the throwing of a
\xcd"ClockUseException"\index{clock!ClockUseException}. 

% An activity may transmit only those clocks that are registered with and
% has not quiesced on (\Sref{resume}). 
% A \xcd"ClockUseException"\index{clock!ClockUseException} is
%thrown if (and when) this condition is violated.

If an activity {$a$} that has executed \xcd`c.resume()` then starts a
new activity {$b$} also registered on \xcd`c` (\eg, via \Xcd{async
clocked(c) S}), the new activity {$b$} starts out having also resumed
\xcd`c`, as if it too had executed \xcd`c.resume()`.  
%~~gen
% package Clocks.For.Jocks.In.Clicky.Smocks;
%class Example{
%static def S():void{}
%static def a_phase_two():void{}
%static def b_phase_two():void{}
%static def example() {
%~~vis
\begin{xten}
//a
val c = Clock.make();
c.resume();
async clocked(c) {
  // b
  c.next();
  b_phase_two();
}
c.next();
a_phase_two();
\end{xten}
%~~siv
%} }
%~~neg
In the proper execution, {$a$} and {$b$} both perform
\xcd`c.next()` and then their phase-2 actions.  
However, if {$b$} were not
initially in the resume state for \xcd`c`, there would be a race condition;
{$b$} could perform \xcd`c.next()` and proceed to \xcd`b_phase_two`
before {$a$} performed \xcd`c.next()`.


An activity may check that it is registered on a clock \xcd"c" by
%~~exp~~`~~`~~c:Clock ~~
the predicate \xcd`c.registered()`


\begin{note}
\Xten{} does not contain a ``register'' operation that would allow an activity
to discover a clock in a datastructure and register itself on it. Therefore,
while a clock \xcd`c` may be stored in a data structure by one activity
\xcd`a` and read from it by another activity \xcd`b`, \xcd`b` cannot do much
with \xcd`c` unless it is already registered with it.  In particular, it
cannot register itself on \xcd`c`, and, lacking that registration, cannot
register a sub-activity on it with \xcd`async clocked(c) S`.
\end{note}


\subsection{Resuming clocks}\index{clock!resume}\label{resume}\label{sec:clock:resume}
\Xten{} permits {\em split phase} clocks. An activity may wish
to indicate that it has completed whatever work it wishes to perform
in the current phase of a  clock \xcd"c" it is registered with, without
suspending altogether. It may do so  by executing 
%~~exp~~`~~`~~c:Clock ~~
\xcd`c.resume()`.



An activity may invoke \xcd`resume()` only on a clock it is registered with,
and has not yet dropped (\Sref{sec:clock:drop}). A
\xcd"ClockUseException"\index{clock!ClockUseException} is thrown if this
condition is violated. Nothing happens if the activity has already invoked a
\xcd"resume" on this clock in the current phase.
%%OLIVIER-DENIES%%  Otherwise, \xcd`c.resume()`
%%OLIVIER-DENIES%% indicates that the activity will not transmit \xcd"c" to an 
%%OLIVIER-DENIES%% \xcd"async" (through a \xcd"clocked" clause), 
%%OLIVIER-DENIES%% until it terminates, drops \xcd"c" or executes a \xcd"next".

%%OLIVIER-DENIES%% \bard{The following is under investigation}
%%OLIVIER-DENIES%% \begin{staticrule*}
%%OLIVIER-DENIES%% It is a static error if any activity has a potentially
%%OLIVIER-DENIES%% live execution path from a \xcd"resume" statement on a clock \xcd"c"
%%OLIVIER-DENIES%% to a
%%OLIVIER-DENIES%% %\xcd"now" or
%%OLIVIER-DENIES%% async spawn statement (which registers the new
%%OLIVIER-DENIES%% activity on \xcd"c") unless the path goes through a \xcd"next"
%%OLIVIER-DENIES%% statement. (A \xcd"c.drop()" following a \xcd"c.resume()" is legal,
%%OLIVIER-DENIES%% as is \xcd"c.resume()" following a \xcd"c.resume()".)
%%OLIVIER-DENIES%% \end{staticrule*}

\subsection{Advancing clocks}\index{clock!next}\label{sec:clock:next}
An activity may execute the statement
\begin{xten}
next;
\end{xten}

\noindent 
Execution of this statement blocks until all the clocks that the
activity is registered with (if any) have advanced. (The activity
implicitly issues a \xcd"resume" on all clocks it is registered
with before suspending.)

\xcd`next;` may be thought of as calling \xcd`c.next()` in parallel for all
clocks that the current activity is registered with.  (The parallelism is
conceptually important: if activities {$a$} and {$b$} are both
registered on clocks \xcd`c` and \xcd`d`, and {$a$} executes
\xcd`c.wait(); d.wait()` while {$b$} executes \xcd`d.wait(); c.wait()`,
then the two will deadlock.  However, if the two clocks are waited on in
parallel, as \xcd`next;` does, {$a$} and {$b$} will not deadlock.)

Equivalently, \xcd`next;` sequentially calls \xcd`c.resume()` for each
registered clock \xcd`c`, in arbitrary order, and then \xcd`c.wait()` for each
clock, again in arbitrary order.  


%%OLIVIER-DENIES%% An \Xten{} computation is said to be {\em quiescent} on a clock
%%OLIVIER-DENIES%% \xcd"c" if each activity registered with \xcd"c" has resumed \xcd"c".
%%OLIVIER-DENIES%% Note that once a computation is quiescent on \xcd"c", it will remain
%%OLIVIER-DENIES%% quiescent on \xcd"c" forever (unless the system takes some action),
%%OLIVIER-DENIES%% since no other activity can become registered with \xcd"c".  That is,
%%OLIVIER-DENIES%% quiescence on a clock is a {\em stable property}.

%%OLIVIER-DENIES%% Once the implementation has detected quiescence on \xcd"c", the system
%%OLIVIER-DENIES%% marks all activities registered with \xcd"c" as being able to progress
%%OLIVIER-DENIES%% on \xcd"c". 
%%OLIVIER-DENIES%% 
An activity blocked on \xcd"next" resumes execution once
it is marked for progress by all the clocks it is registered with.

\subsection{Dropping clocks}\index{clock!drop}\label{sec:clock:drop}
%~~exp~~`~~`~~ c:Clock~~
An activity may drop a clock by executing \xcd`c.drop()`.



\noindent{} The activity is no longer considered registered with this
clock.  A \xcd"ClockUseException" is thrown if the activity has
already dropped \xcd"c".

\section{Deadlock Freedom}

In general, programs using clocks can deadlock, just as programs using loops
can fail to terminate.  However, programs written with a particular syntactic
discipline {\em are} guaranteed to be deadlock-free, just as programs which
use only bounded loops are guaranteed to terminate.  The syntactic discipline
is: 
\begin{itemize}
\item The \xcd`next()` {\bf method} may not be called on any clock. (The
      \xcd`next;` statement is allowed.)
\item Inside of \xcd`finish{S}`, all clocked \xcd`async`s must be in the scope
      an unclocked \xcd`async`.  
\end{itemize}


The second clause prevents the following deadlock.  
%~~gen
% package Clocks.Finish.Hates.Clocks;
% class Example{
% def example() {
%~~vis
\begin{xten}
val c:Clock = Clock.make();
async clocked(c) {                // (A) 
      finish async clocked(c) {   // (B) Violates clause 2
            next;                 // (Bnext)
      }
      next;                       // (Anext)
}
\end{xten}
%~~siv
% } } 
%~~neg
\xcd`(A)`, first of all, waits for the \xcd`finish` containing \xcd`(B)` to
finish.  
\xcd`(B)` will execute its \xcd`next` at \xcd`(Bnext)`, and then wait for all
other activities registered on \xcd`c` to execute their \xcd`next`s.
However, \xcd`(A)` is registered on \xcd`c`.  So, \xcd`(B)` cannot finish
until \xcd`(A)` has proceeded to \xcd`(Anext)`, and \xcd`(A)` cannot proceed
until \xcd`(B)` finishes. Thus, this causes deadlock.


\section{Program equivalences}
From the discussion above it should be clear that the following
equivalences hold:

\begin{eqnarray}
 \mbox{\xcd"c.resume(); next;"}       &=& \mbox{\xcd"next;"}\\
 \mbox{\xcd"c.resume(); d.resume();"} &=& \mbox{\xcd"d.resume(); c.resume();"}\\
 \mbox{\xcd"c.resume(); c.resume();"} &=& \mbox{\xcd"c.resume();"}
\end{eqnarray}

Note that \xcd"next; next;" is not the same as \xcd"next;". The
first will wait for clocks to advance twice, and the second
once.  

%\notinfouro{\input{clock/imp-notes.tex}}
%\notinfouro{\input{clock/clocked-types.tex}}
%\notinfouro{\input{clock/examples.tex}}

\section{Clocked Finish}
\index{finish!clocked}
\index{async!clocked}
\index{clocked!finish}
\index{clocked!async}
\label{ClockedFinish}

In the most common case of a single clock coordinating a few behaviors, X10
allows coding with an implicit clock.  \xcd`finish` and \xcd`async` statements
may be qualified with \xcd`clocked`.  

A \xcd`clocked finish` introduces a new clock.  It executes its body in the
usual way that a \xcd`finish` does--- except that, when its body completes,
the activity executing the \xcd`clocked finish` drops the clock, while it
waits for asynchronous spawned \xcd`async`s to terminate.  

A \xcd`clocked async` registers its async with the implicit clock of
the surrounding \xcd`clocked finish`.   

Both the \xcd`clocked finish` and \xcd`clocked async` may use the \xcd`next`
statement to advance implicit clock.  Since the implicit clock is not
available in a variable, it cannot be manipulated directly. (If you want to
manipulate the clock directly, use an explicit clock.)

The following code starts two activities, each of which perform their first
phase, wait for the other to finish phase 1, and then perform their second
phase.  
%~~gen
%package Clocks.ClockedFinish;
%class Example{
%static def phase(String, Int) {}
%def example() {
%~~vis
\begin{xten}
clocked finish {
  clocked async {
     phase("A", 1);
     next;
     phase("A", 2);
  }
  clocked async {
     phase("B", 1);
     next;
     phase("B", 2);
  }
}
\end{xten}
%~~siv
%}}
%~~neg


\index{finish!nested clocked}
\index{clocked finish!nested}

Clocked finishes may be nested.  The inner \xcd`clocked finish` operates in a
single phase of the outer one.  
	
\chapter{Local and Distributed Arrays}\label{XtenArrays}\index{array}

\Xcd{Array}s provide indexed access to data at a single \Xcd{Place}, {\em via}
\Xcd{Point}s---indices of any dimensionality. \Xcd{DistArray}s is similar, but
spreads the data across multiple \xcd`Place`s, {\em via} \Xcd{Dist}s.  
We refer to arrays either sort as ``general arrays''.  


This chapter provides an overview of the \Xcd{x10.array} classes \Xcd{Array}
and \Xcd{DistArray}, and their supporting classes \Xcd{Point}, \Xcd{Region}
and \Xcd{Dist}.  


\section{Points}\label{point-syntax}
\index{point}
\index{point!syntax}


General arrays are indexed by \xcd`Point`s, which are $n$-dimensional tuples of
integers.  The \xcd`rank`
property of a point gives its dimensionality.  Points can be constructed from
integers or \xcd`Array[Int](1)`s by
the \xcd`Point.make` factory methods:
%~~gen
% package Arrays.Points.Example1;
% class Example1 {
% def example1() {
%~~vis
\begin{xten}
val origin_1 : Point{rank==1} = Point.make(0);
val origin_2 : Point{rank==2} = Point.make(0,0);
val origin_5 : Point{rank==5} = Point.make([0,0,0,0,0]);
\end{xten}
%~~siv
% } } 
%~~neg

%~~type~~`~~`~~ ~~
There is an implicit conversion from \xcd`Array[Int](1)` to 
%~~type~~`~~`~~ ~~
\xcd`Point`, giving
a convenient syntax for constructing points: 

%~~gen
% package Arrays.Points.Example2;
% class Example{
% def example() {
%~~vis
\begin{xten}
val p : Point = [1,2,3];
val q : Point{rank==5} = [1,2,3,4,5];
val r : Point(3) = [11,22,33];
\end{xten}
%~~siv
% } } 
%~~neg

The coordinates of a point are available by subscripting; \xcd`p(i)` is the
\xcd`i`th coordinate of the point \xcd`p`. 
\xcdmath`Point($n$)` is a \Xcd{type}-defined shorthand  for 
\xcdmath`Point{rank==$n$}`.


\section{Regions}\label{XtenRegions}\index{region}
\index{region!syntax}

A region is a set of points of the same rank.  {}\Xten{}
provides a built-in class, \xcd`x10.array.Region`, to allow the
creation of new regions and to perform operations on regions. 
Each region \xcd`R` has a property \xcd`R.rank`, giving the dimensionality of
all the points in it.

%~~gen
% package Arrays.Some.Examples.Fidget.Fidget;
% class Example {
% static def example() {
%~~vis
\begin{xten}
val MAX_HEIGHT=20;
val Null = Region.makeUnit();  // Empty 0-dimensional region
val R1 = 1..100; // 1-dim region with extent 1..100
val R2 = (1..100) as Region(1); // same as R1
val R3 = (0..99) * (-1..MAX_HEIGHT);
val R4 = Region.makeUpperTriangular(10);
val R5 = R4 && R3; // intersection of two regions
\end{xten}
%~~siv
% } } 
%~~neg

The expression \xcdmath`m..n`, for integer expressions \Xcd{m} and \Xcd{n},
evaluates to the rectangular, rank-1 region consisting of the points
$\{$\xcdmath`[m]`, \dots, \xcdmath`[n]`$\}$. If \xcdmath`m` is greater than
\xcdmath`n`, the region \Xcd{m..n} is empty.

%%MAYBE%% A region may be constructed by converting from a rail of
%%MAYBE%% regions (\eg, \xcd`R4` above).
%%MAYBE%% The region constructed from a rail of regions represents the Cartesian product
%%MAYBE%% of the arguments. \Eg, \Xcd{R8} is a region of {$100 \times 16 \times 78$}
%%MAYBE%% points, in
%%MAYBE%% %~s~gen
%%MAYBE%% %package Arrays.Region.RailOfRegions;
%%MAYBE%% %class Example{
%%MAYBE%% %def example(){
%%MAYBE%% %~s~vis
%%MAYBE%% \begin{xten}
%%MAYBE%%   val R8 = [1..100, 3..18, 1..78] as Region(3);
%%MAYBE%% \end{xten}
%%MAYBE%% %~s~siv
%%MAYBE%% %}}
%%MAYBE%% %~s~neg


\index{region!upperTriangular}
\index{region!lowerTriangular}\index{region!banded}

Various built-in regions are provided through  factory
methods on \xcd`Region`.  
\begin{itemize}
\item \Xcd{Region.makeEmpty(n)} returns an empty region of rank \Xcd{n}.
\item \Xcd{Region.makeFull(n)} returns the region containing all points of
      rank \Xcd{n}.  
\item \Xcd{Region.makeUnit()} returns the region of rank 0 containing the
      unique point of rank 0.  It is useful as the identity for Cartesian
      product of regions.
\item \Xcd{Region.makeHalfspace(normal:Point, k:Int)} returns the unbounded
      half-space of rank \Xcd{normal.rank}, consisting of all points \Xcd{p}
      satisfying \xcdmath`p$\cdot$normal $\le$ k`.
\item \Xcd{Region.makeRectangular(min, max)}, where \Xcd{min} and \Xcd{max}
      are \Xcd{Int} rails or valrails of length \Xcd{n}, returns a
      \Xcd{Region(n)} equal to: 
      \xcdmath`[min(0) .. max(0), $\ldots$, min(n-1)..max(n-1)]`.
\item \Xcd{Region.make(regions)} constructs the Cartesian product of the
      \Xcd{Region(1)}s in \Xcd{regions}.
\item \Xcd{Region.makeBanded(size, upper, lower)} constructs the
      banded \Xcd{Region(2)} of size \Xcd{size}, with \Xcd{upper} bands above
      and \Xcd{lower} bands below the diagonal.
\item \Xcd{Region.makeBanded(size)} constructs the banded \Xcd{Region(2)} with
      just the main diagonal.
\item \xcd`Region.makeUpperTriangular(N)` returns a region corresponding
to the non-zero indices in an upper-triangular \xcd`N x N` matrix.
\item \xcd`Region.makeLowerTriangular(N)` returns a region corresponding
to the non-zero indices in a lower-triangular \xcd`N x N` matrix.
\item 
  If \xcd`R` is a region, and \xcd`p` a Point of the same rank, then 
%~~exp~~`~~`~~R:Region, p:Point(R.rank) ~~
  \xcd`R+p` is \xcd`R` translated forwards by 
  \xcd`p` -- the region whose
%~~exp~~`~~`~~r:Point, p:Point(r.rank) ~~
  points are \xcd`r+p` 
  for each \xcd`r` in \xcd`R`.
\item 
  If \xcd`R` is a region, and \xcd`p` a Point of the same rank, then 
%~~exp~~`~~`~~R:Region, p:Point(R.rank) ~~
  \xcd`R-p` is \xcd`R` translated backwards by 
  \xcd`p` -- the region whose
%~~exp~~`~~`~~r:Point, p:Point(r.rank) ~~
  points are \xcd`r-p` 
  for each \xcd`r` in \xcd`R`.

\end{itemize}

All the points in a region are ordered canonically by the
lexicographic total order. Thus the points of the region \xcd`(1..2)*(1..2)`
are ordered as 
\begin{xten}
(1,1), (1,2), (2,1), (2,2)
\end{xten}
Sequential iteration statements such as \xcd`for` (\Sref{ForAllLoop})
iterate over the points in a region in the canonical order.

A region is said to be {\em rectangular}\index{region!convex} if it is of
the form \xcdmath`(T$_1$ * $\cdots$ * T$_k$)` for some set of intervals
\xcdmath`T$_i = $ l$_i$ .. h$_i$ `. Such a
region satisfies the property that if two points $p_1$ and $p_3$ are
in the region, then so is every point $p_2$ between them (that is, it is {\em convex}). 
(Banded and triangular regions are not rectangular.)
The operation
%~~exp~~`~~`~~R:Region ~~
\xcd`R.boundingBox()` gives the smallest rectangular region containing
\xcd`R`.

\subsection{Operations on regions}
\index{region!operations}

Let \xcd`R` be a region. A {\em sub-region} is a subset of \Xcd{R}.
\index{region!sub-region}

Let \xcdmath`R1` and \xcdmath`R2` be two regions whose types establish that
they are of the same rank. Let \xcdmath`S` be another region; its rank is
irrelevant. 

\xcdmath`R1 && R2` is the intersection of \xcdmath`R1` and
\xcdmath`R2`, \viz, the region containing all points which are in both
\Xcd{R1} and \Xcd{R2}.  \index{region!intersection}
%~~exp~~`~~`~~ ~~
For example, \xcd`1..10 && 2..20` is \Xcd{2..10}.


%%NO-DIFF%% \xcdmath`R1 - R2` is the set difference of \xcdmath`R1` and
%%NO-DIFF%% \xcdmath`R2`; \viz, the points in \xcdmath`R1` which are not in
%%NO-DIFF%% \xcdmath`R2`.\index{region!set difference}
%%NO-DIFF%% For example, 
%%NO-DIFF%% ~~exp~~`~~`~~ ~~
%%NO-DIFF%% \xcd`(1..10) - (1..3)` 
%%NO-DIFF%% is 
%%NO-DIFF%% \Xcd{4..10}.

\xcdmath`R1 * S` is the Cartesian product of \xcdmath`R1` and
\xcdmath`S`,  formed by pairing each point in \xcdmath`R1` with every  point in \xcdmath`S`.
\index{region!product}
%~~exp~~`~~`~~ ~~
Thus, \xcd`(1..2)*(3..4)*(5..6)`
is the region of rank \Xcd{3} containing the eight points with coordinates
\xcd`[1,3,5]`, \xcd`[1,3,6]`, \xcd`[1,4,5]`, \xcd`[1,4,6]`,
\xcd`[2,3,5]`, \xcd`[2,3,6]`, \xcd`[2,4,5]`, \xcd`[2,4,6]`.


For a region \xcdmath`R` and point \xcdmath`p` of the same rank,
%~~exp~~`~~`~~R:Region, p:Point{p.rank==R.rank} ~~
\xcd`R+p` 
and
%~~exp~~`~~`~~R:Region, p:Point{p.rank==R.rank} ~~
\xcd`R-p` 
represent the translation of the region
forward 
and backward 
by \xcdmath`p`. That is, \Xcd{R+p} is the set of points
\Xcd{p+q} for all \Xcd{q} in \Xcd{R}, and \Xcd{R-p} is the set of \Xcd{q-p}.

More \Xcd{Region} methods are described in the API documentation.

\section{Arrays}
\index{array}

Arrays are organized data, arranged so that it can be accessed by subscript.
An \xcd`Array[T]` \Xcd{A} has a \Xcd{Region} \Xcd{A.region}, telling which
\Xcd{Point}s are in \Xcd{A}.  For each point \Xcd{p} in \Xcd{A.region},
\Xcd{A(p)} is the datum of type \Xcd{T} associated with \Xcd{p}.  X10
implementations should 
attempt to store \xcd`Array`s efficiently, and to make array element accesses
quick---\eg, avoiding constructing \Xcd{Point}s when unnecessary.

This generalizes the concepts of arrays appearing in many other programming
languages.  A \Xcd{Point} may have any number of coordinates, so an
\xcd`Array` can have, in effect, any number of integer subscripts.  

Indeed, it is possible to write code that works on \Xcd{Array}s regardless 
of dimension.  For example, to add one \Xcd{Array[Int]} \Xcd{src} into another
\Xcd{dest}, 
%~~gen
%package Arrays.Arrays.Arrays.Example;
% class Example{
%~~vis
\begin{xten}
static def addInto(src: Array[Int], dest:Array[Int])
  {src.region == dest.region}
  = {
    for (p in src.region) 
       dest(p) += src(p);
  }
\end{xten}
%~~siv
%}
%~~neg
\noindent
Since \Xcd{p} is a \Xcd{Point}, it can hold as many coordinates as are
necessary for the arrays \Xcd{src} and \Xcd{dest}.

The basic operation on arrays is subscripting: if \Xcd{A} is an \Xcd{Array[T]}
and \Xcd{p} a point with the same rank as \xcd`A.region`, then
%~~exp~~`~~`~~A:Array[Int], p:Point{self.rank == A.region.rank} ~~
\xcd`A(p)`
is the value of type \Xcd{T} associated with point \Xcd{p}.

Array elements can be changed by assignment. If \Xcd{t:T}, 
%~~gen
%package Arrays.Arrays.Subscripting.Is.From.Mars;
%class Example{
%def example[T](A:Array[T], p: Point{rank == A.region.rank}, t:T){
%~~vis
\begin{xten}
A(p) = t;
\end{xten}
%~~siv
%} } 
%~~neg
modifies the value associated with \Xcd{p} to be \Xcd{t}, and leaves all other
values in \Xcd{A} unchanged.

An \Xcd{Array[T]} \Xcd{A} has: 
\begin{itemize}
%~~exp~~`~~`~~A:Array[Int] ~~
\item \xcd`A.region`: the \Xcd{Region} upon which \Xcd{A} is defined.
%~~exp~~`~~`~~A:Array[Int] ~~
\item \xcd`A.size`: the number of elements in \Xcd{A}.
%~~exp~~`~~`~~A:Array[Int] ~~
\item \xcd`A.rank`, the rank of the points usable to subscript \Xcd{A}.
      Identical to \Xcd{A.region.rank}.
\end{itemize}

\subsection{Array Constructors}
\index{array!constructor}

To construct an array whose elements all have the same value \Xcd{init}, call
\Xcd{new Array[T](R, init)}. 
For example, an array of a thousand \xcd`"oh!"`s can be made by:
%~~exp~~`~~`~~ ~~
\xcd`new Array[String](1..1000, "oh!")`.


To construct and initialize an array, call the two-argument constructor. 
\Xcd{new Array[T](R, f)} constructs an array of elements of type \Xcd{T} on
region \Xcd{R}, with \Xcd{A(p)} initialized to \Xcd{f(p)} for each point
\Xcd{p} in \Xcd{R}.  \Xcd{f} must be a function taking a point of rank
\Xcd{R.rank} to a value of type \Xcd{T}.  \Eg, to construct an array of a
hundred zero values, call
%~~exp~~`~~`~~ ~~
\xcd`new Array[Int](1..100, (Point(1))=>0)`. 
To construct a multiplication table, call
%~~exp~~`~~`~~ ~~
\xcd`new Array[Int]((0..9)*(0..9), (p:Point(2)) => p(0)*p(1))`.

Other constructors are available; see the API documentation and
\Sref{sect:ArrayCtors}. 

\subsection{Array Operations}
\index{array!operations on}

The basic operation on \Xcd{Array}s is subscripting.  If \Xcd{A:Array[T]} and 
\xcd`p:Point{rank == A.rank}`, then \Xcd{a(p)} is the value of type \Xcd{T}
appearing at position \Xcd{p} in \Xcd{A}.    The syntax is identical to
function application, and, indeed, arrays may be used as functions.
\Xcd{A(p)} may be assigned to, as well, by the usual assignment syntax
%~~exp~~`~~`~~A:Array[Int], p:Point{rank == A.rank}, t:Int ~~
\xcd`A(p)=t`.
(This uses the application and setting syntactic sugar, as given in \Sref{set-and-apply}.)

Sometimes it is more convenient to subscript by integers.  Arrays of rank 1-4
can, in fact, be accessed by integers: 
%~~gen
%package Arrays.Arrays.wombatsfromlemuria;
%class Example{
%def example(){
%~~vis
\begin{xten}
val A1 = new Array[Int](1..10, 0);
A1(4) = A1(4) + 1;
val A4 = new Array[Int]((1..2)*(1..3)*(1..4)*(1..5), 0);
A4(2,3,4,5) = A4(1,1,1,1)+1;
\end{xten}
%~~siv
%}}
%~~neg



Iteration over an \Xcd{Array} is defined, and produces the \Xcd{Point}s of the
array's region.  If you want to use the values in the array, you have to
subscript it.  For example, you could double every element of an
\Xcd{Array[Int]} by: 
%~~gen
%package Arrays.Arrays.mostly_dire_dreams_tonight;
%class Example{
%def example(A:Array[Int]) {
%~~vis
\begin{xten}
for (p in A) A(p) = 2*A(p);
\end{xten}
%~~siv
%}}
%~~neg



\section{Distributions}\label{XtenDistributions}
\index{distribution}

Distributed arrays are spread across multiple \xcd`Place`s.  
A {\em distribution}, a mapping from a region to a set of places, 
describes where each element of a distributed array is kept.
Distributions are embodied by the class \Xcd{x10.array.Dist}.
This class is \xcd`final` in
{}\XtenCurrVer; future versions of the language may permit
user-definable distributions. 
The {\em rank} of a distribution is the rank of the underlying region, and
thus the rank of every point that the distribution applies to.



%~~gen
%package Arrays.Dists.Examples.Examples.EXAMPLESDAMMIT;
% class Example{
% def example() {
%~~vis
\begin{xten}
val R  <: Region = 1..100;
val D1 <: Dist = Dist.makeBlock(R);
val D2 <: Dist = R -> here;
\end{xten}
%~~siv
% } } 
%~~neg

Let \xcd`D` be a distribution. 
%~~exp~~`~~`~~D:Dist ~~
\xcd`D.region` 
denotes the underlying
region. 
Given a point \xcd`p`, the expression
%~~exp~~`~~`~~ D:Dist, p:Point{p.rank == D.rank}~~
\xcd`D(p)` represents the application of \xcd`D` to \xcd`p`, that is,
the place that \xcd`p` is mapped to by \xcd`D`. The evaluation of the
expression \xcd`D(p)` throws an \xcd`ArrayIndexOutofBoundsException`
if \xcd`p` does not lie in the underlying region.
%%NO-R2D2-CONV%% 
%%NO-R2D2-CONV%% When operated on as a distribution, a region \xcd`R` implicitly
%%NO-R2D2-CONV%% behaves as the distribution mapping each item in \xcd`R` to \xcd`here`
%%NO-R2D2-CONV%% (\ie, \xcd`R->here`, see below). Conversely, when used in a context
%%NO-R2D2-CONV%% expecting a region, a distribution \xcd`D` should be thought of as
%%NO-R2D2-CONV%% standing for \xcd`D.region`.


\subsection{Operations returning distributions}
\index{distribution!operations}

Let \xcd`R` be a region, \xcd`Q` a Sequence of places \{\xcd`p1`, \dots,
\xcd`pk`\} (enumerated in canonical order), and \xcd`P` a place.

\paragraph{Unique distribution} \index{distribution!unique}
%~~exp~~`~~`~~Q:Sequence[Place] ~~
The distribution \xcd`Dist.makeUnique(Q)` is the unique distribution from the
region \xcd`1..k` to \xcd`Q` mapping each point \xcd`i` to \xcd`pi`.

\paragraph{Constant distributions.} \index{distribution!constant}
%~~exp~~`~~`~~R:Region, P:Place ~~
The distribution \xcd`R->P` maps every point in region \xcd`R` to place \xcd`P`, as does
%~~exp~~`~~`~~R:Region, P:Place ~~
\xcd`Dist.makeConstant(R,P)`. 

\paragraph{Block distributions.}\index{distribution!block}
%~~exp~~`~~`~~R:Region ~~
The distribution \xcd`Dist.makeBlock(R)` distributes the elements of \xcd`R`,
in order, over all the places available to the program. 
Let $p$ equal \xcd`|R| div N` and $q$ equal \xcd`|R| mod N`,
where \xcd`N` is the size of \xcd`Q`, and 
\xcd`|R|` is the size of \xcd`R`.  The first $q$ places get
successive blocks of size $(p+1)$ and the remaining places get blocks of
size $p$.

There are other \xcd`Dist.makeBlock` methods capable of controlling the
distribution and the set of places used; see the API documentation.


%%NO-CYCLIC-DIST%%  \paragraph{Cyclic distributions.} \index{distribution!cyclic}
%%NO-CYCLIC-DIST%%  The distribution \xcd`Dist.makeCyclic(R, Q)` distributes the points in \xcd`R`
%%NO-CYCLIC-DIST%%  cyclically across places in \xcd`Q` in order.
%%NO-CYCLIC-DIST%%  
%%NO-CYCLIC-DIST%%  The distribution \xcd`Dist.makeCyclic(R)` is the same distribution as
%%NO-CYCLIC-DIST%%  \xcd`Dist.makeCyclic(R, Place.places)`. 
%%NO-CYCLIC-DIST%%  
%%NO-CYCLIC-DIST%%  Thus the distribution \xcd`Dist.makeCyclic(Place.MAX_PLACES)` provides a 1--1
%%NO-CYCLIC-DIST%%  mapping from the region \xcd`Place.MAX_PLACES` to the set of all
%%NO-CYCLIC-DIST%%  places and is the same as the distribution \xcd`Dist.makeCyclic(Place.places)`.
%%NO-CYCLIC-DIST%%  
%%NO-CYCLIC-DIST%%  \paragraph{Block cyclic distributions.}\index{distribution!block cyclic}
%%NO-CYCLIC-DIST%%  The distribution \xcd`Dist.makeBlockCyclic(R, N, Q)` distributes the elements
%%NO-CYCLIC-DIST%%  of \xcd`R` cyclically over the set of places \xcd`Q` in blocks of size
%%NO-CYCLIC-DIST%%  \xcd`N`.
%%NO-ARB-DIST%%  
%%NO-ARB-DIST%%  \paragraph{Arbitrary distributions.} \index{distribution!arbitrary}
%%NO-ARB-DIST%%  The distribution \xcd`Dist.makeArbitrary(R,Q)` arbitrarily allocates points in {\cf
%%NO-ARB-DIST%%  R} to \xcd`Q`. As above, \xcd`Dist.makeArbitrary(R)` is the same distribution as
%%NO-ARB-DIST%%  \xcd`Dist.makeArbitrary(R, Place.places)`.
%%NO-ARB-DIST%%  
%%NO-ARB-DIST%%  \oldtodo{Determine which other built-in distributions to provide.}
%%NO-ARB-DIST%%  
\paragraph{Domain Restriction.} \index{distribution!restriction!region}

If \xcd`D` is a distribution and \xcd`R` is a sub-region of {\cf
%~~exp~~`~~`~~D:Dist,R :Region{R.rank==D.rank} ~~
D.region}, then \xcd`D | R` represents the restriction of \xcd`D` to
\xcd`R`---that is, the distribution that takes each point \xcd`p` in \xcd`R`
to 
%~~exp~~`~~`~~D:Dist, p:Point{p.rank==D.rank} ~~
\xcd`D(p)`, 
but doesn't apply to any points but those in \xcd`R`.

\paragraph{Range Restriction.}\index{distribution!restriction!range}

If \xcd`D` is a distribution and \xcd`P` a place expression, the term
%~~exp~~`~~`~~ D:Dist, P:Place~~
\xcd`D | P` 
denotes the sub-distribution of \xcd`D` defined over all the
points in the region of \xcd`D` mapped to \xcd`P`.

Note that \xcd`D | here` does not necessarily contain adjacent points
in \xcd`D.region`. For instance, if \xcd`D` is a cyclic distribution,
\xcd`D | here` will typically contain points that differ by the number of
places. 
An implementation may find a
way to still represent them in contiguous memory, \eg, using a
complex arithmetic function to map from the region index to an index
into the array.

%%NO-USER-DIST%%  \subsection{User-defined distributions}\index{distribution!user-defined}
%%NO-USER-DIST%%  
%%NO-USER-DIST%%  Future versions of \Xten{} may provide user-defined distributions, in
%%NO-USER-DIST%%  a way that supports static reasoning.

%%NO-flinking-operations-on-DIST%%  \subsection{Operations on distributions}
%%NO-flinking-operations-on-DIST%%  
%%NO-flinking-operations-on-DIST%%  A {\em sub-distribution}\index{sub-distribution} of \xcd`D` is
%%NO-flinking-operations-on-DIST%%  any distribution \xcd`E` defined on some subset of the region of
%%NO-flinking-operations-on-DIST%%  \xcd`D`, which agrees with \xcd`D` on all points in its region.
%%NO-flinking-operations-on-DIST%%  We also say that \xcd`D` is a {\em super-distribution} of
%%NO-flinking-operations-on-DIST%%  \xcd`E`. A distribution \xcdmath`D1` {\em is larger than}
%%NO-flinking-operations-on-DIST%%  \xcdmath`D2` if \xcdmath`D1` is a super-distribution of
%%NO-flinking-operations-on-DIST%%  \xcdmath`D2`.
%%NO-flinking-operations-on-DIST%%  
%%NO-flinking-operations-on-DIST%%  Let \xcdmath`D1` and \xcdmath`D2` be two distributions with the same rank.  
%%NO-flinking-operations-on-DIST%%  

%%NO-&&-DIST%%  \paragraph{Intersection of distributions.}\index{distribution!intersection}
%%NO-&&-DIST%%  ~~exp~~`~~`~~D1:Dist, D2:Dist{D1.rank==D2.rank} ~~
%%NO-&&-DIST%%  \xcdmath`D1 && D2`, the intersection 
%%NO-&&-DIST%%  of \xcdmath`D1`
%%NO-&&-DIST%%  and \xcdmath`D2`, is the largest common sub-distribution of
%%NO-&&-DIST%%  \xcdmath`D1` and \xcdmath`D2`.

%%NO-overlay-DIST%%  \paragraph{Asymmetric union of distributions.}\index{distribution!union!asymmetric}
%%NO-overlay-DIST%%  ~~exp~~`~~`~~D1:Dist, D2:Dist{D1.rank==D2.rank} ~~
%%NO-overlay-DIST%%  \xcdmath`D1.overlay(D2)`, the asymmetric union of
%%NO-overlay-DIST%%  \xcdmath`D1` and \xcdmath`D2`, is the distribution whose
%%NO-overlay-DIST%%  region is the union of the regions of \xcdmath`D1` and
%%NO-overlay-DIST%%  \xcdmath`D2`, and whose value at each point \xcd`p` in its
%%NO-overlay-DIST%%  region is \xcdmath`D2(p)` if \xcdmath`p` lies in
%%NO-overlay-DIST%%  \xcdmath`D2.region` otherwise it is \xcdmath`D1(p)`.
%%NO-overlay-DIST%%  (\xcdmath`D1` provides the defaults.)

%%NO-flinking-operations-on-DIST%%  \paragraph{Disjoint union of distributions.}\index{distribution!union!disjoint}
%%NO-flinking-operations-on-DIST%%  ~~exp~~`~~`~~D1:Dist, D2:Dist{D1.rank==D2.rank} ~~
%%NO-flinking-operations-on-DIST%%  \xcdmath`D1 || D2`, the disjoint union of
%%NO-flinking-operations-on-DIST%%  \xcdmath`D1`
%%NO-flinking-operations-on-DIST%%  and \xcdmath`D2`, is defined only if the regions of
%%NO-flinking-operations-on-DIST%%  \xcdmath`D1` and \xcdmath`D2` are disjoint. Its value is
%%NO-flinking-operations-on-DIST%%  \xcdmath`D1.overlay(D2)` (or equivalently
%%NO-flinking-operations-on-DIST%%  \xcdmath`D2.overlay(D1)`.  (It is the least
%%NO-flinking-operations-on-DIST%%  super-distribution of \xcdmath`D1` and \xcdmath`D2`.)
%%NO-flinking-operations-on-DIST%%  
%%NO-flinking-operations-on-DIST%%  \paragraph{Difference of distributions.}\index{distribution!difference}
%%NO-flinking-operations-on-DIST%%  \xcdmath`D1 - D2` is the largest sub-distribution of
%%NO-flinking-operations-on-DIST%%  \xcdmath`D1` whose region is disjoint from that of
%%NO-flinking-operations-on-DIST%%  \xcdmath`D2`.
%%NO-flinking-operations-on-DIST%%  
%%NO-flinking-operations-on-DIST%%  
%%What-Is-This-Example%% \subsection{Example}
%%What-Is-This-Example%% \begin{xten}
%%What-Is-This-Example%% def dotProduct(a: Array[T](D), b: Array[T](D)): Array[Double](D) =
%%What-Is-This-Example%%   (new Array[T]([1:D.places],
%%What-Is-This-Example%%       (Point) => (new Array[T](D | here,
%%What-Is-This-Example%%                     (i): Point) => a(i)*b(i)).sum())).sum();
%%What-Is-This-Example%% \end{xten}
%%What-Is-This-Example%% 
%%What-Is-This-Example%% This code returns the inner product of two \xcd`T` vectors defined
%%What-Is-This-Example%% over the same (otherwise unknown) distribution. The result is the sum
%%What-Is-This-Example%% reduction of an array of \xcd`T` with one element at each place in the
%%What-Is-This-Example%% range of \xcd`D`. The value of this array at each point is the sum
%%What-Is-This-Example%% reduction of the array formed by multiplying the corresponding
%%What-Is-This-Example%% elements of \xcd`a` and \xcd`b` in the local sub-array at the current
%%What-Is-This-Example%% place.
%%What-Is-This-Example%% 

\section{Distributed Arrays}
\index{array!distributed}
\index{distributed array}
\index{\Xcd{DistArray}}
\index{DistArray}

Distributed arrays, instances of \xcd`DistArray[T]`, are very much like
\xcd`Array`s, except that they distribute information among multiple
\xcd`Place`s according to a \xcd`Dist` value passed in as a constructor
argument.  For example, the following code creates a distributed array holding
a thousand cells, each initialized to 0.0, distributed via a block
distribution over all places.
%~~gen
% package Arrays.Distarrays.basic.example;
% class Example {
% def example() {
%~~vis
\begin{xten}
val R <: Region = 1..1000;
val D <: Dist = Dist.makeBlock(R);
val da <: DistArray[Float] = DistArray.make[Float](D, (Point(1))=>0.0f);
\end{xten}
%~~siv
%}}
%~~neg




\section{Distributed Array Construction}\label{ArrayInitializer}
\index{distributed array!creation}
\index{\Xcd{DistArray}!creation}
\index{DistArray!creation}

\xcd`DistArray`s are instantiated by invoking one of the \xcd`make` factory
methods of the \xcd`DistArray` class.
A \xcd`DistArray` creation 
must take either an \xcd`Int` as an argument or a \xcd`Dist`. In the first
case,  a distributed array is created over the distribution \xcd`[0:N-1]->here`;
in the second over the given distribution. 

A distributed array creation operation may also specify an initializer
function.
The function is applied in parallel
at all points in the domain of the distribution. The
construction operation terminates locally only when the \xcd`DistArray` has been
fully created and initialized (at all places in the range of the
distribution).

For instance:
%~~gen
% package Arrays.DistArray.Construction.FeralWolf;
% class Example {
% def example() {
%~~vis
\begin{xten}
val data : DistArray[Int]
    = DistArray.make[Int](1..1000->here, ([i]:Point(1)) => i);
val blocked = Dist.makeBlock((1..1000)*(1..1000));
val data2 : DistArray[Int]
    = DistArray.make[Int](blocked, ([i,j]:Point(2)) => i*j);
\end{xten}
%~~siv
% }  }
%~~neg


{}\noindent 
The first declaration stores in \xcd`data` a reference to a mutable
distributed array with \xcd`1000` elements each of which is located in the
same place as the array. The element at \Xcd{[i]} is initialized to its index
\xcd`i`. 

The second declaration stores in \xcd`data2` a reference to a mutable
two-dimensional distributed array, whose coordinates both range from 1 to
1000, distributed in blocks over all \xcd`Place`s, 
initialized with \xcd`i*j`
at point \xcd`[i,j]`.

%%WHY-THIS-EXAMPLE%% In the following 
%%WHY-THIS-EXAMPLE%% %~x~gen
%%WHY-THIS-EXAMPLE%% % package Arrays.DistArrays.FlistArrays.GlistArrays;
%%WHY-THIS-EXAMPLE%% %~x~vis
%%WHY-THIS-EXAMPLE%% \begin{xten}
%%WHY-THIS-EXAMPLE%% val D1:Dist(1) = Dist.makeBlock(1..100);
%%WHY-THIS-EXAMPLE%% val D2:Dist(2) = Dist.makeBlock((1..100)*(-1..1));
%%WHY-THIS-EXAMPLE%% val ints : Array[Int]
%%WHY-THIS-EXAMPLE%%     = Array.make[Int](1000, ((i):Point) => i*i);
%%WHY-THIS-EXAMPLE%% val floats1 : Array[Float]
%%WHY-THIS-EXAMPLE%%     = Array.make[Float](D1, ((i):Point) => i*i as Float);
%%WHY-THIS-EXAMPLE%% val floats2 : Array[Float]
%%WHY-THIS-EXAMPLE%%    = Array.make[Float](D2, ((i,j):Point) => i+j as Float);;
%%WHY-THIS-EXAMPLE%% \end{xten}
%%WHY-THIS-EXAMPLE%% %~x~siv
%%WHY-THIS-EXAMPLE%% %
%%WHY-THIS-EXAMPLE%% %~x~neg


\section{Operations on Arrays and Distributed Arrays}

Arrays and distributed arrays share many operations.
In the following, let \xcd`a` be an array with base type T, and \xcd`da` be an
array with distribution \xcd`D` and base type \xcd`T`.




\subsection{Element operations}\index{array!access}
The value of \xcd`a` at a point \xcd`p` in its region of definition is
%~~exp~~`~~`~~a:Array[Int](3), p:Point(3) ~~
obtained by using the indexing operation \xcd`a(p)`. 
The value of \xcd`da` at \xcd`p` is similarly
%~~exp~~`~~`~~da:DistArray[Int](3), p:Point(3) ~~
\xcd`da(p)`
This operation
may be used on the left hand side of an assignment operation to update
the value: 
%~~stmt~~`~~`~~a:Array[Int](3), p:Point(3), t:Int ~~
\xcd`a(p)=t;`
and 
%~~stmt~~`~~`~~da:DistArray[Int](3), p:Point(3), t:Int ~~
\xcd`da(p)=t;`
The operator assignments, \xcd`a(i) += e` and so on,  are also
available. 

It is a runtime error to use either \xcd`da(p)` or \xcd`da(p)=v` at a place
other than \xcd`da.dist(p)`, \viz{} at the place that the element exists. 

%%HUH%%  For distributed array variables, the right-hand-side of an assignment must
%%HUH%%  have the same distribution \xcd`D` as an array being assigned. This
%%HUH%%  assignment involves
%%HUH%%  control communication between the sites hosting \xcd`D`. Each
%%HUH%%  site performs the assignment(s) of array components locally. The
%%HUH%%  assignment terminates when assignment has terminated at all
%%HUH%%  sites hosting \xcd`D`.

\subsection{Constant promotion}\label{ConstantArray}
\index{array!constant promotion}

For a region \xcd`R` and a value \xcd`v` of type \xcd`T`, the expression 
%~~genexp~~`~~`~~T~~R:Region, v:T ~~
\xcd`new Array[T](R, v)` 
produces an array on region \xcd`R` initialized with value \xcd`v`
Similarly, 
for a distribution \xcd`D` and a value \xcd`v` of
type \xcd`T` the expression 
%~~genexp~~`~~`~~T ~~D:Dist, v:T ~~
\xcd`DistArray.make[T](D, (Point(D.rank))=>v)`
constructs a distributed array with
distribution \xcd`D` and base type \xcd`T` initialized with \xcd`v`
at every point.

Note that \xcd`Array`s are constructed by constructor calls, but
\xcd`DistArrays` are constructed by calls to the factory methods
\xcd`DistArray.make`. This is because \xcd`Array`s are fairly simple objects,
but \xcd`DistArray`s may be implemented by different classes for different
distributions. The use of the factory method gives the library writer the
freedom to select appropriate implementations.


\subsection{Restriction of an array}\index{array!restriction}

Let \xcd`R` be a sub-region of \xcd`da.region`. Then 
%~~exp~~`~~`~~da:DistArray[Int](3), p:Point(3), R: Region(da.rank) ~~
\xcd`da | R`
represents the sub-\xcd`DistArray` of \xcd`da` on the region \xcd`R`.
That is, \xcd`da | R` has the same values as \xcd`da` when subscripted by a
%~~exp~~`~~`~~R:Region, da: DistArray[Int]{da.region.rank == R.rank} ~~
point in region \xcd`R && da.region`, and is undefined elsewhere.
`
Recall that a rich set of operators are available on distributions
(\Sref{XtenDistributions}) to obtain sub-distributions
(e.g. restricting to a sub-region, to a specific place etc).

%%GONE-AWAY%%  \subsection{Assembling an array}
%%GONE-AWAY%%  Let \xcd`da1,da2` be distributed arrays of the same base type \xcd`T` defined over
%%GONE-AWAY%%  distributions \xcd`D1` and \xcd`D2` respectively. 
%%GONE-AWAY%%  \paragraph{Assembling arrays over disjoint regions}\index{array!union!disjoint}
%%GONE-AWAY%%  
%%GONE-AWAY%%  
%%GONE-AWAY%%  If \xcd`D1` and \xcd`D2` are disjoint then the expression 
%%GONE-AWAY%%  %~x~genexp~~`~~`~~T ~~ da1: Array[T], da2: Array[T](da1.rank)~~
%%GONE-AWAY%%  \xcd`da1 || da2` denotes the unique array of base type \xcd`T` defined over the
%%GONE-AWAY%%  distribution \xcd`D1 || D2` such that its value at point \xcd`p` is
%%GONE-AWAY%%  \xcd`a1(p)` if \xcd`p` lies in \xcd`D1` and \xcd`a2(p)`
%%GONE-AWAY%%  otherwise. This array is a reference (value) array if \xcd`a1` is.
%%GONE-AWAY%%  
%%GONE-AWAY%%  \paragraph{Overlaying an array on another}\index{array!union!asymmetric}
%%GONE-AWAY%%  The expression
%%GONE-AWAY%%  \xcd`a1.overlay(a2)` (read: the array \xcd`a1` {\em overlaid with} \xcd`a2`)
%%GONE-AWAY%%  represents an array whose underlying region is the union of that of
%%GONE-AWAY%%  \xcd`a1` and \xcd`a2` and whose distribution maps each point \xcd`p`
%%GONE-AWAY%%  in this region to \xcd`D2(p)` if that is defined and to \xcd`D1(p)`
%%GONE-AWAY%%  otherwise. The value \xcd`a1.overlay(a2)(p)` is \xcd`a2(p)` if it is defined and \xcd`a1(p)` otherwise.
%%GONE-AWAY%%  
%%GONE-AWAY%%  This array is a reference (value) array if \xcd`a1` is.
%%GONE-AWAY%%  
%%GONE-AWAY%%  The expression \xcd`a1.update(a2)` updates the array \xcd`a1` in place
%%GONE-AWAY%%  with the result of \xcd`a1.overlay(a2)`.
%%GONE-AWAY%%  
%%GONE-AWAY%%  \oldtodo{Define Flooding of arrays}
%%GONE-AWAY%%  
%%GONE-AWAY%%  \oldtodo{Wrapping an array}
%%GONE-AWAY%%  
%%GONE-AWAY%%  \oldtodo{Extending an array in a given direction.}
%%GONE-AWAY%%  
\subsection{Operations on Whole Arrays}

\paragraph{Pointwise operations}\label{ArrayPointwise}\index{array!pointwise operations}
The unary \xcd`map` operation applies a function to each element of
a distributed or non-distributed array, returning a new distributed array with
the same distribution, or a non-distributed array with the same region.
For example, the following produces an array of cubes: 
%~~gen
%package Arrays.arrays.ginungagap.bakery.treats;
%class Example{
%def example() {
%~~vis
\begin{xten}
val A = new Array[Int](1..10, (p:Point(1))=>p(0) );
// A = 1,2,3,4,5,6,7,8,9,10
val cube = (i:Int) => i*i*i;
val B = A.map(cube);
// B = 1,8,27,64,216,343,512,729,1000
\end{xten}
%~~siv
%} } 
%~~neg

A variant operation lets you specify the array \Xcd{B} into which the result
will be stored.   
%~~gen
%package Arrays.arrays.ginungagap.bakery.treats.doomed;
%class Example{
%def example() {
%~~vis
\begin{xten}
val A = new Array[Int](1..10, (p:Point(1))=>p(0) );
// A = 1,2,3,4,5,6,7,8,9,10
val cube = (i:Int) => i*i*i;
val B = new Array[Int](A.region); // B = 0,0,0,0,0,0,0,0,0,0
A.map(B, cube);
// B = 1,8,27,64,216,343,512,729,1000
\end{xten}
%~~siv
%} } 
%~~neg
\noindent
This is convenient if you have an already-allocated array lying around unused.
In particular, it can be used if you don't need \Xcd{A} afterwards and want to
reuse its space:
%~~gen
%package Arrays.arrays.ginungagap.bakery.treats.doomed.spackled;
%class Example{
%def example() {
%~~vis
\begin{xten}
val A = new Array[Int](1..10, (p:Point(1))=>p(0) );
// A = 1,2,3,4,5,6,7,8,9,10
val cube = (i:Int) => i*i*i;
A.map(A, cube);
// A = 1,8,27,64,216,343,512,729,1000
\end{xten}
%~~siv
%} } 
%~~neg


The binary \xcd`map` operation takes a binary function and
another
array over the same region or distributed array over the same  distribution,
and applies the function 
pointwise to corresponding elements of the two arrays, returning
a new array or distributed array of the same shape.
The following code adds two distributed arrays: 
%~~gen
% package Arrays.Pointwise.Pointless.Map2;
% class Example{
%~~vis
\begin{xten}
static def add(da:DistArray[Int], db: DistArray[Int]{da.dist==db.dist})
    = da.map(db, Int.+);
\end{xten}
%~~siv
%}
%~~neg



\paragraph{Reductions}\label{ArrayReductions}\index{array!reductions}

Let \xcd`f` be a function of type \xcd`(T,T)=>T`.  Let
\xcd`a` be an array over base type \xcd`T`.
Let \xcd`unit` be a value of type \xcd`T`.
Then the
%~~genexp~~`~~`~~ T ~~ f:(T,T)=>T, a : Array[T], unit:T ~~
operation \xcd`a.reduce(f, unit)` returns a value of type \xcd`T` obtained
by combining all the elements of \xcd`a` by use of  \xcd`f` in some unspecified order
(perhaps in parallel).   
The following code gives one method which 
meets the definition of \Xcd{reduce},
having a running total \Xcd{r}, and accumulating each value \xcd`a(p)` into it
using \Xcd{f} in turn.  (This code is simply given as an example; \Xcd{Array}
has this operation defined already.)
%~~gen
%package Arrays.Reductions.And.Eliminations;
% class Example {
%~~vis
\begin{xten}
def oneWayToReduce[T](a:Array[T], f:(T,T)=>T, unit:T):T {
  var r : T = unit;
  for(p in a.region) r = f(r, a(p));
  return r;
}
\end{xten}
%~~siv
%}
%~~neg


For example,  the following sums an array of integers.  \Xcd{f} is addition,
and \Xcd{unit} is zero.  
%~~gen
% package Arrays.Reductions.And.Emulsions;
% class Example {
% def example() {
%~~vis
\begin{xten}
val a = [1,2,3,4];
val sum = a.reduce(Int.+, 0); 
assert(sum == 10); // 10 == 1+2+3+4
\end{xten}
%~~siv
%}}
%~~neg

Other orders of evaluation, degrees of parallelism, and applications of
\Xcd{f(x,unit)} and \xcd`f(unit,x)`are also correct.
In order to guarantee that the result is precisely
determined, the  function \xcd`f` should be associative and
commutative, and the value \xcd`unit` should satisfy
\xcd`f(unit,x)` \xcd`==` \xcd`x` \xcd`==` \xcd`f(x,unit)`
for all \Xcd{x:T}.  




\xcd`DistArray`s have the same operation.
This operation involves communication between the places over which
the \xcd`DistArray` is distributed. The \Xten{} implementation guarantees that
only one value of type \xcd`T` is communicated from a place as part of
this reduction process.

\paragraph{Scans}\label{ArrayScans}\index{array!scans}


Let \xcd`f:(T,T)=>T`, \xcd`unit:T`, and \xcd`a` be an \xcd`Array[T]` or
\xcd`DistArray[T]`.  Then \xcd`a.scan(f,unit)` is the array or distributed
array of type \xcd`T` whose {$i$}th element in canonical order is the
reduction by \xcd`f` with unit \xcd`unit` of the first {$i$} elements of
\xcd`a`. 


This operation involves communication between the places over which the
distributed array is distributed. The \Xten{} implementation will endeavour to
minimize the communication between places to implement this operation.

Other operations on arrays, distributed arrays, and the related classes may be
found in the \xcd`x10.array` package.
	
\chapter{Annotations}\label{XtenAnnotations}\index{annotations}


\Xten{} provides an 
an annotation system  system for to allow the
compiler to be extended with new static analyses and new
transformations.

Annotations are constraint-free interface types that decorate the abstract syntax tree
of an \Xten{} program.  The \Xten{} type-checker ensures that an annotation
is a legal interface type.
In \Xten{}, interfaces may declare
both methods and properties.  Therefore, like any interface type, an
annotation may instantiate
one or more of its interface's properties.
%%PLUGINNERY%%  Unlike with Java
%%PLUGINNERY%%  annotations,
%%PLUGINNERY%%  property initializers need not be
%%PLUGINNERY%%  compile-time constants;
%%PLUGINNERY%%  however, a given compiler plugin
%%PLUGINNERY%%  may do additional checks to constrain the allowable
%%PLUGINNERY%%  initializer expressions.
%%PLUGINNERY%%  The \Xten{} type-checker does not check that
%%PLUGINNERY%%  all properties of an annotation are initialized,
%%PLUGINNERY%%  although this could be enforced by
%%PLUGINNERY%%  a compiler plugin.

\section{Annotation syntax}

The annotation syntax consists of an ``\texttt{@}'' followed by an interface type.

%##(Annotations Annotation
\begin{bbgrammar}
%(FROM #(prod:Annotations)#)
         Annotations \: Annotation & (\ref{prod:Annotations}) \\
                     \| Annotations Annotation \\
%(FROM #(prod:Annotation)#)
          Annotation \: \xcd"@" NamedTypeNoConstraints & (\ref{prod:Annotation}) \\
\end{bbgrammar}
%##)

Annotations can be applied to most syntactic constructs in the language
including class declarations, constructors, methods, field declarations,
local variable declarations and formal parameters, statements,
expressions, and types.
Multiple occurrences of the same annotation (i.e., multiple
annotations with the same interface type) on the same entity are permitted.

%%OBSOLETE%% \begin{grammar}
%%OBSOLETE%% ClassModifier \: Annotation \\
%%OBSOLETE%% InterfaceModifier \: Annotation \\
%%OBSOLETE%% FieldModifier \: Annotation \\
%%OBSOLETE%% MethodModifier \: Annotation \\
%%OBSOLETE%% VariableModifier \: Annotation \\
%%OBSOLETE%% ConstructorModifier \: Annotation \\
%%OBSOLETE%% AbstractMethodModifier \: Annotation \\
%%OBSOLETE%% ConstantModifier \: Annotation \\
%%OBSOLETE%% Type \: AnnotatedType \\
%%OBSOLETE%% AnnotatedType \: Annotation\plus Type \\
%%OBSOLETE%% Statement \: AnnotatedStatement \\
%%OBSOLETE%% AnnotatedStatement \: Annotation\plus Statement \\
%%OBSOLETE%% Expression \: AnnotatedExpression \\
%%OBSOLETE%% AnnotatedExpression \: Annotation\plus Expression \\
%%OBSOLETE%% \end{grammar}

\noindent
Recall that interface types may have dependent parameters.

\noindent
The following examples illustrate the syntax:

\begin{itemize}
\item Declaration annotations:
\begin{xtennoindent}
  // class annotation
  @Value
  class Cons { ... }

  // method annotation
  @PreCondition(0 <= i && i < this.size)
  public def get(i: Int): Object { ... }

  // constructor annotation
  @Where(x != null)
  def this(x: T) { ... }

  // constructor return type annotation
  def this(x: T): C@Initialized { ... }

  // variable annotation
  @Unique x: A;
\end{xtennoindent}
\item Type annotations:
\begin{xtennoindent}
  List@Nonempty

  Int@Range(1,4)

  Array[Array[Double]]@Size(n * n)
\end{xtennoindent}
\item Expression annotations:
\begin{xtennoindent}
  m()  @RemoteCall
\end{xtennoindent}
\item Statement annotations:
\begin{xtennoindent}
  @Atomic { ... }

  @MinIterations(0)
  @MaxIterations(n)
  for (var i: Int = 0; i < n; i++) { ... }

  // An annotated empty statement ;
  @Assert(x < y);
\end{xtennoindent}
\end{itemize}

\section{Annotation declarations}

Annotations are declared as interfaces.  They must be
subtypes of the interface \texttt{x10.lang.annotation.Annotation}.
Annotations on particular static entities must extend the corresponding
\xcd`Annotation` subclasses, as follows: 
\begin{itemize}
\item Expressions---\xcd`ExpressionAnnotation`
\item Statements---\xcd`StatementAnnotation`
\item Classes---\xcd`ClassAnnotation`
\item Fields---\xcd`FieldAnnotation`
\item Methods---\xcd`MethodAnnotation`
\item Imports---\xcd`ImportAnnotation`
\item Packages---\xcd`PackageAnnotation`
\end{itemize}

%% vj Thu Sep 19 21:39:41 EDT 2013
% updated for v2.4 -- no change.
\chapter{Interoperability with Other Languages}
\label{NativeCode}
\index{native code}
\label{Interoperability}
\index{interoperability}

The ability to interoperate with other programming languages is an
essential feature of the \Xten{} implementation.  Cross-language
interoperability enables both the incremental adoption of \Xten{} in
existing applications and the usage of existing libraries and
frameworks by newly developed \Xten{} programs. 

There are two primary interoperability scenarios that are supported by
\XtenCurrVer{}: inline substitution of fragments of foreign code for
\Xten program fragments (expressions, statements) and external linkage
to foreign code.

\section{Embedded Native Code Fragments}

The
\xcd`@Native(lang,code) Construct` annotation from \xcd`x10.compiler.Native` in
\Xten{} can be used to tell the \Xten{} compiler to substitute \xcd`code` for
whatever it would have generated when compiling \xcd`Construct`
with the \xcd`lang` back end.

The compiler cannot analyze native code the same way it analyzes \Xten{} code.  In
particular, \xcd`@Native` fields and methods must be explicitly typed; the
compiler will not infer types.


\subsection{Native {\tt static} Methods}

\xcd`static` methods can be given native implementations.  Note that these
implementations are syntactically {\em expressions}, not statements, in C++ or
\Java{}.   Also, it is possible (and common) to provide native implementations
into both \Java{} and C++ for the same method.
%~~gen ^^^ extern10
% package Extern.or_current_turn;
%~~vis
\begin{xten}
import x10.compiler.Native;
class Son {
  @Native("c++", "printf(\"Hi!\")")
  @Native("java", "System.out.println(\"Hi!\")")
  static def printNatively():void = {};
}
\end{xten}
%~~siv
%
%~~neg

If only some back-end languages are given, the \Xten{} code will be used for the
remaining back ends: 
%~~gen ^^^ extern20
% package Extern.or.burn;
%~~vis
\begin{xten}
import x10.compiler.Native;
class Land {
  @Native("c++", "printf(\"Hi from C++!\")")
  static def example():void = {
    x10.io.Console.OUT.println("Hi from X10!");
  };
}
\end{xten}
%~~siv
%
%~~neg

The \xcd`native` modifier on methods indicates that the method must not have
an \Xten{} code body, and \xcd`@Native` implementations must be given for all back
ends:
%~~gen ^^^ extern30
% package Extern.or_maybe_getting_back_together;
%~~vis
\begin{xten}
import x10.compiler.Native;
class Plants {
  @Native("c++", "printf(\"Hi!\")")
  @Native("java", "System.out.println(\"Hi!\")")
  static native def printNatively():void;
}
\end{xten}
%~~siv
%
%~~neg


Values may be returned from external code to \Xten{}.  Scalar types in \Java{} and
C++ correspond directly to the analogous types in \Xten{}.  
%~~gen ^^^ extern40
% package Extern.hardy;
%~~vis
\begin{xten}
import x10.compiler.Native;
class Return {
  @Native("c++", "1")
  @Native("java", "1")
  static native def one():Int;
}
\end{xten}
%~~siv
%
%~~neg
Types are {\em not} inferred for methods marked as \xcd`@Native`.

Parameters may be passed to external code.  \xcd`(#1)`  is the first parameter,
\xcd`(#2)` the second, and so forth.  (\xcd`(#0)` is the name of the enclosing
class, or the \xcd`this` variable.)  Be aware that this is macro substitution
rather than normal parameter 
passing; \eg, if the first actual parameter is \xcd`i++`, and \xcd`(#1)`
appears twice in the external code, \xcd`i` will be incremented twice.
For example, a (ridiculous) way to print the sum of two numbers is: 
%~~gen ^^^ extern50
% package Extern.or_turnabout_is_fair_play;
%~~vis
\begin{xten}
import x10.compiler.Native;
class Species {
  @Native("c++","printf(\"Sum=%d\", ((#1)+(#2)) )")
  @Native("java","System.out.println(\"\" + ((#1)+(#2)))")
  static native def printNatively(x:Int, y:Int):void;
}
\end{xten}
%~~siv
%
%~~neg


Static variables in the class are available in the external code.  For \Java{},
the static variables are used with their \Xten{} names.  For C++, the names
must be mangled, by use of the \xcd`FMGL` macro.  

%~~gen ^^^ extern60
%package Extern.or.Die;
%~~vis
\begin{xten}
import x10.compiler.Native;
class Ability {
  static val A : Int = 1n;
  @Native("java", "A+2")
  @Native("c++", "Ability::FMGL(A)+2")
  static native def fromStatic():Int;
}
\end{xten}
%~~siv
%
%~~neg




\subsection{Native Blocks}

Any block may be annotated with \xcd`@Native(lang,stmt)`, indicating that, in
the given back end, it should be implemented as \xcd`stmt`. All 
variables  from the surrounding context are available inside \xcd`stmt`. For
example, the method call \xcd`born.example(10n)`, if compiled to \Java{}, changes
the field \xcd`y` of a \xcd`Born` object to 10. If compiled to C++ (for which
there is no \xcd`@Native`), it sets it to 3. 
%~~gen ^^^ extern70
%package Extern.me.plz; 
%~~vis
\begin{xten}
import x10.compiler.Native;
class Born {
  var y : Int = 1n; 
  public def example(x:Int):Int{
    @Native("java", "y=x;") 
    {y = 3n;}
    return y;
  }
}
\end{xten}
%~~siv
%
%~~neg

Note that the code being replaced is a statement -- the block \xcd`{y = 3n;}`
in this case -- so the replacement should also be a statement. 


Other \Xten{} constructs may or may not be available in \Java{} and/or C++ code.  For
example, type variables do not correspond exactly to type variables in either
language, and may not be available there.  The exact compilation scheme is
{\em not} fully specified.  You may inspect the generated \Java{} or C++ code and
see how to do specific things, but there is no guarantee that fancy external
coding will continue to work in later versions of \Xten{}.



The full facilities of C++ or \Java{} are available in native code blocks.
However, there is no guarantee that advanced features behave sensibly. You
must follow the exact conventions that the code generator does, or you will
get unpredictable results.  Furthermore, the code generator's conventions may
change without notice or documentation from version to version.  In most cases
the  code should either be a very simple expression, or a method
or function call to external code.


\section{Interoperability with External Java Code}

With Managed \Xten{}, we can seamlessly call existing \Java{} code from \Xten{},
and call \Xten{} code from \Java{}.  We call this the 
\emph{Java interoperability}~\cite{TakeuchiX1013} feature.

By combining \Java{} interoperability with \Xten{}'s distributed
execution features, we can even make existing \Java{} applications, which
are originally designed to run on a single \Java{} VM, scale-out with
minor modifications.

\subsection{How Java program is seen in X10}

Managed \Xten{} does not pre-process the existing \Java{} code to make it
accessible from \Xten{}.  \Xten{} programs compiled with Managed \Xten{} will call
existing \Java{} code as is.

\paragraph{Types}

In \Xten{}, both at compile time and run time, there is no way to
distinguish \Java{} types from \Xten{} types.  \Java{} types can be referred to
with regular \xcd{import} statement, or their qualified names.  The
package \xcd{java.lang} is not auto-imported into \Xten.  In Managed
x10, the resolver is enhanced to resolve types with \Xten{} source files
in the source path first, then resolve them with \Java{} class files in
the class path. Note that the resolver does not resolve types with
\Java{} source files, therefore \Java{} source files must be compiled in
advance.  To refer to \Java{} types listed in
Tables~\ref{tab:specialtypes}, and \ref{tab:otherspecialtypes}, which
include all \Java{} primitive types, use the corresponding \Xten{} type
(e.g. use \xcd{x10.lang.Int} (or in short, \xcd{Int}) instead of
\xcd{int}).

\paragraph{Fields}

Fields of \Java{} types are seen as fields of \Xten{} types.

Managed \Xten{} does not change the static initialization semantics of
\Java{} types, which is per-class, at load time, and per-place (\Java{} VM),
therefore, it is subtly different than the per-field lazy
initialization semantics of \Xten{} static fields.

\paragraph{Methods}

Methods of \Java{} types are seen as methods of \Xten{} types.

\paragraph{Generic types}

Generic \Java{} types are seen as their raw types 
(\S 4.8 in~\cite{java-lang-spec2005}).  Raw type is a mechanism to handle generic
\Java{} types as non-generic types, where the type parameters are assumed
as \verb|java.lang.Object| or their upperbound if they have it.  \Java{}
introduced generics and raw type at the same time to facilitate
generic \Java{} code interfacing with non-generic legacy \Java{} code.
Managed \Xten{} uses this mechanism for a slightly different purpose.
\Java{} erases type parameters at compile time, whereas \Xten{} preserves
their values at run time.  To manifest this semantic gap in generics,
Managed \Xten{} represents \Java{} generic types as raw types and eliminates
type parameters at source code level.  For more detailed discussions,
please refer to~\cite{TakeuchiX1011,TakeuchiX1012}.

\paragraph{Arrays}

\Xten{} rail and array types are generic types whose representation is different
from \Java{} array types.

Managed \Xten{} provides a special \Xten{} type
\xcd{x10.interop.Java.array[T]} which represents \Java{} array type
\xcd{T[]}.  Note that for \Xten{} types in Table~\ref{tab:specialtypes},
this type means the \Java{} array type of their primary type.  For
example, \xcd{array[Int]} and \xcd{array[String]} mean
\xcd{int[]} and \xcd{java.lang.String[]}, respectively.  Managed \Xten{}
also provides conversion methods between \Xten{} \xcd`Rail`s and \Java{}
arrays (\xcd{Java.convert[T](a:Rail[T]):array[T]} and
\xcd{Java.convert[T](a:array[T]):Rail[T]}),
and creation methods for \Java{} arrays 
(\xcd{Java.newArray[T](d0:Int):array[T]}
etc.).

\paragraph{Exceptions}

The \XtenCurrVer{} exception hierarchy has been designed so that there is a
natural correspondence with the \Java{} exception hierarchy. As shown in
Table~\ref{tab:otherspecialtypes}, many commonly used \Java{}
exception types are directly mapped to \Xten{} exception types. 
Exception types that are thus aliased may be caught (and thrown) using
either their \Java{} or \Xten types.  In \Xten code, it is stylistically
preferable to use the \Xten type to refer to the exception as shown in 
Figure~\ref{fig:javaexceptions}.

%----------------
\begin{figure}
\begin{xten}
import x10.interop.Java;
public class XClass {   
  public static def main(args:Rail[String]):void {
    try {
      val a = Java.newArray[Int](2n);
      a(0n) = 0n;
      a(1n) = 1n;
      a(2n) = 2n;
    } catch (e:x10.lang.ArrayIndexOutOfBoundsException) {
      Console.OUT.println(e);
    }
  }
}
\end{xten}
\begin{verbatim}
> x10c -d bin src/XClass.x10
> x10 -cp bin XClass
x10.lang.ArrayIndexOutOfBoundsException: Array index out of range: 2
\end{verbatim}
\caption{Java exceptions in X10}
\label{fig:javaexceptions}
\end{figure}
%----------------

\paragraph{Compiling and executing X10 programs}

We can compile and run \Xten{} programs that call existing \Java{} code with
the same \verb|x10c| and \verb|x10| command by specifying the location
of \Java{} class files or jar files that your \Xten{} programs refer to, with
\verb|-classpath| (or in short, \verb|-cp|) option.

\subsection{How X10 program is translated to Java}

Managed \Xten{} translates \Xten{} programs to \Java{} class files. 

\Xten{} does not provide a \Java{} reflection-like mechanism to resolve \Xten{}
types, methods, and fields with their names at runtime, nor a code
generation tool, such as \verb|javah|, to generate stub code to access
them from other languages.  \Java{} programmers, therefore, need to
access \Xten{} types, methods, and fields in the generated \Java{} code
directly as they access \Java{} types, methods, and fields.  To make it
possible, \Java{} programmers need to understand how \Xten{} programs are
translated to \Java{}.

Some aspects of the \Xten{} to \Java{} translation scheme may change in
future version of \Xten{}; therefore in this document only a stable
subset of translation scheme will be explained.  Although it is a
subset, it has been extensively used by \Xten{} core team and proved to be
useful to develop \Java{} Hadoop interop layer for a Main-memory Map
Reduce (M3R) engine~\cite{Shinnar12M3R} in \Xten{}.

In the following discussions, we deliberately ignore generic \Xten{}
types because the translation of generics is an area of active
development and will undergo some changes in future versions of \Xten{}.
For those who are interested in the implementation of generics
in Managed \Xten{}, please consult~\cite{TakeuchiX1012}.  We also don't
cover function types, function values, and all non-static methods.
Although slightly outdated, another paper~\cite{TakeuchiX1011}, which
describes translation scheme in \Xten{} 2.1.2, is still useful to
understand the overview of \Java{} code generation in Managed \Xten{}.


\paragraph{Types}

\Xten{} classes and structs are translated to \Java{} classes with the same
names.  \Xten{} interfaces are translated to \Java{} interfaces with the same
names.

Table~\ref{tab:specialtypes} shows the list of special types that are
mapped to \Java{} primitives.  Primitives are their primary
representations that are useful for good performance.  Wrapper classes
are used when the reference types are needed.  Each wrapper class has
two static methods \verb|$box()| and \verb|$unbox()| to convert its
value from primary representation to wrapper class, and vice versa,
and \Java{} backend inserts their calls as needed.  An you notice, every
unsigned type uses the same \Java{} primitive as its corresponding signed
type for its representation.

Table~\ref{tab:otherspecialtypes} shows a non-exhaustive list of
another kind of special types that are mapped (not translated) to \Java{}
types.  As you notice, since an interface \verb|Any| is mapped to a
class |java.lang.Object| and \verb|Object| is hidden from the
language, there is no direct way to create an instance of
\verb|Object|. As a workaround, \verb|Java.newObject()| is provided.

As you also notice, \verb|x10.lang.Comparable[T]| is mapped to \verb|java.lang.Comparable|.
This is needed to map \verb|x10.lang.String|, which implements \verb|x10.lang.Compatable[String]|, to \verb|java.lang.String| for performance, but as a trade off, this mapping results in the lost of runtime type information for \verb|Comparable[T]| in Managed \Xten{}.
The runtime of Managed \Xten{} has built-in knowledge for \verb|String|, but for other \Java{} classes that implement \verb|java.lang.Comparable|, \verb|instanceof Comparable[Int]| etc. may return incorrect results.
In principle, it is impossible to map \Xten{} generic type to the existing \Java{} generic type without losing runtime type information.

%----------------
\begin{table}
%\scriptsize
\small
\centering
\mbox{
\begin{tabular}{|lr|lr|l|}												   \hline
\multicolumn{2}{|c|}{\textbf{X10}}	& \multicolumn{2}{|c|}{\textbf{Java (primary)}}	& \textbf{Java (wrapper class)}	\\ \hline
															   \hline
{\tt x10.lang.Byte}	& {\tt 1y}	& {\tt byte}		& {\tt (byte)1}		& {\tt x10.core.Byte}		\\ \hline
{\tt x10.lang.UByte}	& {\tt 1uy}	& {\tt byte}		& {\tt (byte)1}		& {\tt x10.core.UByte}		\\ \hline
{\tt x10.lang.Short}	& {\tt 1s}	& {\tt short}		& {\tt (short)1}	& {\tt x10.core.Short}		\\ \hline
{\tt x10.lang.UShort}	& {\tt 1us}	& {\tt short}		& {\tt (short)1}	& {\tt x10.core.UShort} 	\\ \hline
{\tt x10.lang.Int}	& {\tt 1n}	& {\tt int}		& {\tt 1}		& {\tt x10.core.Int}		\\ \hline
{\tt x10.lang.UInt}	& {\tt 1un}	& {\tt int}		& {\tt 1}		& {\tt x10.core.UInt}		\\ \hline
{\tt x10.lang.Long}	& {\tt 1}	& {\tt long}		& {\tt 1l}		& {\tt x10.core.Long}	 	\\ \hline
{\tt x10.lang.ULong}	& {\tt 1u}	& {\tt long}		& {\tt 1l}		& {\tt x10.core.ULong}	 	\\ \hline
{\tt x10.lang.Float}	& {\tt 1.0f}	& {\tt float}		& {\tt 1.0f}		& {\tt x10.core.Float}	 	\\ \hline
{\tt x10.lang.Double}	& {\tt 1.0}	& {\tt double}		& {\tt 1.0}		& {\tt x10.core.Double} 	\\ \hline
{\tt x10.lang.Char}	& {\tt 'c'}	& {\tt char}		& {\tt 'c'}		& {\tt x10.core.Char}		\\ \hline
{\tt x10.lang.Boolean}	& {\tt true}	& {\tt boolean}		& {\tt true}		& {\tt x10.core.Boolean}	\\ \hline
%{\tt x10.lang.String} 	& {\tt "abc"}	& {\tt java.lang.String}& {\tt "abc"}		& {\tt x10.core.String}		\\ \hline
\end{tabular}
}
\caption{X10 types that are mapped to Java primitives}
\label{tab:specialtypes}
\end{table}
%----------------


%----------------
\begin{table}
%\scriptsize
\small
\centering
\mbox{
\begin{tabular}{|l|l|}										   \hline
\multicolumn{1}{|c|}{\textbf{X10}}		& \multicolumn{1}{|c|}{\textbf{Java}}		\\ \hline
												   \hline
{\tt x10.lang.Any} 				& {\tt java.lang.Object}			\\ \hline
{\tt x10.lang.Comparable[T]} 			& {\tt java.lang.Comparable}			\\ \hline
{\tt x10.lang.String}		 		& {\tt java.lang.String}			\\ \hline
{\tt x10.lang.CheckedThrowable}		 	& {\tt java.lang.Throwable}			\\ \hline
{\tt x10.lang.CheckedException}		 	& {\tt java.lang.Exception}			\\ \hline
{\tt x10.lang.Exception} 			& {\tt java.lang.RuntimeException}		\\ \hline
{\tt x10.lang.ArithmeticException} 		& {\tt java.lang.ArithmeticException}		\\ \hline
{\tt x10.lang.ClassCastException} 		& {\tt java.lang.ClassCastException}		\\ \hline
{\tt x10.lang.IllegalArgumentException} 	& {\tt java.lang.IllegalArgumentException}	\\ \hline
{\tt x10.util.NoSuchElementException}	 	& {\tt java.util.NoSuchElementException}	\\ \hline
{\tt x10.lang.NullPointerException} 		& {\tt java.lang.NullPointerException}		\\ \hline
{\tt x10.lang.NumberFormatException} 		& {\tt java.lang.NumberFormatException}		\\ \hline
{\tt x10.lang.UnsupportedOperationException} 	& {\tt java.lang.UnsupportedOperationException}	\\ \hline
{\tt x10.lang.IndexOutOfBoundsException} 	& {\tt java.lang.IndexOutOfBoundsException}	\\ \hline
{\tt x10.lang.ArrayIndexOutOfBoundsException} 	& {\tt java.lang.ArrayIndexOutOfBoundsException}\\ \hline
{\tt x10.lang.StringIndexOutOfBoundsException} 	& {\tt java.lang.StringIndexOutOfBoundsException}\\ \hline
{\tt x10.lang.Error} 				& {\tt java.lang.Error}				\\ \hline
{\tt x10.lang.AssertionError} 			& {\tt java.lang.AssertionError}		\\ \hline
{\tt x10.lang.OutOfMemoryError} 		& {\tt java.lang.OutOfMemoryError}		\\ \hline
{\tt x10.lang.StackOverflowError} 		& {\tt java.lang.StackOverflowError}		\\ \hline
{\tt void} 					& {\tt void}					\\ \hline
\end{tabular}
}
\caption{X10 types that are mapped (not translated) to Java types}
\label{tab:otherspecialtypes}
\end{table}
%----------------


\paragraph{Fields}

As shown in Figure~\ref{fig:fields}, instance fields of \Xten{} classes and structs are translated to the instance fields of the same names of the generated \Java{} classes.
Static fields of \Xten{} classes and structs are translated to the static methods of the generated \Java{} classes, whose name has \verb|get$| prefix.
Static fields of \Xten{} interfaces are translated to the static methods of the special nested class named \verb|$Shadow| of the generated \Java{} interfaces.

%----------------
\begin{figure}
\begin{xten}
class C {
  static val a:Int = ...;
  var b:Int;
}
interface I {
  val x:Int = ...;
}
\end{xten}
\begin{xten}
class C {
  static int get$a() { return ...; }
  int b;
}
interface I {
  abstract static class $Shadow {
    static int get$x() { return ...; }
  }
}
\end{xten}
\caption{X10 fields in Java}
\label{fig:fields}
\end{figure}
%----------------


\paragraph{Methods}

As shown in Figure~\ref{fig:methods}, methods of \Xten{} classes or structs are translated to the methods of the same names of the generated \Java{} classes.
Methods of \Xten{} interfaces are translated to the methods of the same names of the generated \Java{} interfaces.

Every method whose return type has two representations, such as the types in Table~\ref{tab:specialtypes}, will have \verb|$O| suffix with its name.
For example, \verb|def f():Int| in \Xten{} will be compiled as \verb|int f$O()| in \Java{}.
For those who are interested in the reason, please look at our paper~\cite{TakeuchiX1012}.

%----------------
\begin{figure}
\begin{xten}
interface I {
  def f():Int;
  def g():Any;
}
class C implements I {
  static def s():Int = 0n;
  static def t():Any = null;
  public def f():Int = 1n;
  public def g():Any = null;
}
\end{xten}
\begin{xten}
interface I {
  int f$O();
  java.lang.Object g();
}
class C implements I {
  static int s$O() { return 0; }
  static java.lang.Object t() { return null; }
  public int f$O() { return 1; }
  public java.lang.Object g() { return null; }
}
\end{xten}
\caption{X10 methods in Java}
\label{fig:methods}
\end{figure}
%----------------


\paragraph{Compiling Java programs}

To compile \Java{} program that calls \Xten{} code, you should use
\verb|x10cj| command instead of javac (or whatever your \Java{}
compiler). It invokes the post \Java{}-compiler of \verb|x10c| with the
appropriate options. You need to specify the location of \Xten{}-generated
class files or jar files that your \Java{} program refers to.

\verb|x10cj -cp MyX10Lib.jar MyJavaProg.java|


\paragraph{Executing Java programs}

Before executing any \Xten{}-generated \Java{} code, the runtime of Managed
\Xten{} needs to be set up properly at each place.  To set up the runtime,
a special launcher named \verb|runjava| is used to run \Java{} programs.
All \Java{} programs that call \Xten{} code should be launched with it, and
no other mechanisms, including direct execution with \verb|java| command, are
supported.

\begin{verbatim}
Usage: runjava <Java-main-class> [arg0 arg1 ...]
\end{verbatim}


\section{Interoperability with External C and C++ Code}

C and C++ code can be linked to \Xten{} code, either by writing auxiliary C++ files and
adding them with suitable annotations, or by linking libraries.

\subsection{Auxiliary C++ Files}

Auxiliary C++ code can be written in \xcd`.h` and \xcd`.cc` files, which
should be put in the same directory as the the \Xten{} file using them.
Connecting with the library uses the \xcd`@NativeCPPInclude(dot_h_file_name)`
annotation to include the header file, and the 
\xcd`@NativeCPPCompilationUnit(dot_cc_file_name)` annotation to include the
C++ code proper.  For example: 

{\bf MyCppCode.h}: 
\begin{xten}
void foo();
\end{xten}


{\bf MyCppCode.cc}:
\begin{xten}
#include <cstdlib>
#include <cstdio>
void foo() {
    printf("Hello World!\n");
}
\end{xten}

{\bf Test.x10}:
\begin{xten}
import x10.compiler.Native;
import x10.compiler.NativeCPPInclude;
import x10.compiler.NativeCPPCompilationUnit;

@NativeCPPInclude("MyCPPCode.h")
@NativeCPPCompilationUnit("MyCPPCode.cc")
public class Test {
    public static def main (args:Rail[String]) {
        { @Native("c++","foo();") {} }
    }
}
\end{xten}

\subsection{C++ System Libraries}

If we want to additionally link to more libraries in \xcd`/usr/lib` for
example, it is necessary to adjust the post-compilation directly.  The
post-compilation is the compilation of the C++ which the \Xten{}-to-C++ compiler
\xcd`x10c++` produces.  

The primary mechanism used for this is the \xcd`-cxx-prearg` and
\xcd`-cxx-postarg` command line arguments to
\xcd`x10c++`. The values of any \xcd`-cxx-prearg` commands are placed
in the post compiler command before the list of .cc files to compile.
The values of any \xcd`-cxx-postarg` commands are placed in the post
compiler command after the list of .cc files to compile. Typically
pre-args are arguments intended for the C++ compiler itself, while
post-args are arguments intended for the linker. 

The following example shows how to compile \xcd`blas` into the
executable via these commands. The command must be issued on one line.

\begin{xten}
x10c++ Test.x10 -cxx-prearg -I/usr/local/blas 
  -cxx-postarg -L/usr/local/blas -cxx-postarg -lblas'
\end{xten}


\chapter{Definite Assignment}
\label{sect:DefiniteAssignment}
\index{definite assignment}
\index{assignment!definite}
\index{definitely assigned}
\index{definitely not assigned}

X10 requires, reasonably enough, that every variable be set before it is read.
Sometimes this is easy, as when a variable is declared and assigned together: 
%~~gen ^^^ DefiniteAssignment4x1u
% package DefiniteAssignment4x1u;
% class Example {
% def example() {
%~~vis
\begin{xten}
  var x : Int = 0;
  assert x == 0;
\end{xten}
%~~siv
%}}
%~~neg
However, it is convenient to allow programs to make decisions before
initializing variables.
%~~gen ^^^ DefiniteAssignment4u7z
% package DefiniteAssignment4u7z;
% class Example {
%~~vis
\begin{xten}
def example(a:Int, b:Int) {
  val max:Int;
  //ERROR: assert max==max; // can't read 'max'
  if (a > b) max = a;
  else max = b;
  assert max >= a && max >= b;
}
\end{xten}
%~~siv
%}
%~~neg
This is particularly useful for \xcd`val` variables.  \xcd`var`s could be
initialized to a default value and then reassigned with the right value.
\xcd`val`s must be initialized once and cannot be changed, so they must be
initialized with the correct value. 

However, one must be careful -- and the X10 compiler enforces this care.
Without the \xcd`else` clause, the preceding code might not give \xcd`max` a
value by the \xcd`assert`.  

This leads to the concept of {\em definite assignment} \cite{Javasomething}.
A variable is definitely assigned at a point in code if, no matter how that
point in code is reached, the variable has been assigned to.  In X10,
variables must be definitely assigned before they can be read.


As X10 requires that \xcd`val` variables {\em not} be initialized
twice,  we need the dual concept as well.  A variable is {\em definitely
unassigned} at a point in code if it cannot have been assigned there.  For
example, immediately after \xcd`val x:Int`, \xcd`x` is definitely unassigned. 

Finally, we need the concept of {\em singly} and {\em multiply assigned}.
A variable is singly assigned in a block if it is assigned precisely
once; it is multiply assigned if it could possibly be assigned more than once.  
\xcd`var`s can  multiply assigned as desired. \xcd`val`s must be singly
assigned.  For example, the code \xcd`x = 1; x = 2;` is perfectly fine if
\xcd`x` is a \xcd`var`, but incorrect (even in a constructor) if \xcd`x` is a
\xcd`val`.  

At some points in code, a variable might be neither definitely assigned nor
definitely unassigned.    Such states are not always useful.  
%~~gen ^^^ DefiniteAssignment4f5z
% package DefiniteAssignment4f5z;
% class Example {
% 
%~~vis
\begin{xten}
def example(flag : Boolean) {
  var x : Int;
  if (flag) x = 1;
  // x is neither def. assigned nor unassigned.
  x = 2; 
  // x is def. assigned.
\end{xten}
%~~siv
% } } 
%~~neg
This shows that we cannot simply define ``definitely unassigned'' as ``not
definitely assigned''.   If \xcd`x` had been a \xcd`val` rather than a
\xcd`var`, the previous example would not be allowed.    

Unfortunately, a completely accurate definition of ``definitely assigned''
or ``definitely unassigned'' is undecidable -- impossible for the compiler to
determine.  So, X10 takes a {\em conservative approximation} of these
concepts.  If X10's definition says that \xcd`x` is definitely assigned (or
definitely unassigned), then it will be assigned (or not assigned) in every
execution of the program.  

However, there are programs which X10's algorithm says are incorrect, but
which actually would behave properly if they were executed.   In the following
example, \xcd`flag` is either \xcd`true` or \xcd`false`, and in either case
\xcd`x` will be initialized.  However, X10's analysis does not understand this
--- thought it {\em would} understand if the example were coded with an
\xcd`if-else` rather than a pair of \xcd`if`s.  So, after the two \xcd`if`
statements, \xcd`x` is not definitely assigned, and thus the \xcd`assert`
statement, which reads it, is forbidden.  
%~~gen ^^^ DefiniteAssignment3x6i
% package DefiniteAssignment3x6i;
% class Example{ 
%~~vis
\begin{xten}
def example(flag:Boolean) {
  var x : Int;
  if (flag) x = 1;
  if (!flag) x = 2;
  // ERROR: assert x < 3;
}
\end{xten}
%~~siv
%}
%~~neg

\section{Asynchronous Definite Assignment}


Local variables (but not fields) allow {\em asynchronous assignment}. A local
variable can be assigned in an \xcd`async` statement, and, when the
\xcd`async` is \xcd`finish`ed, the variable is definitely assigned.  

\begin{ex}
%~~gen ^^^ DefiniteAssignment4a6n
% package DefiniteAssignment4a6n;
% class Example {
% def example() {
%~~vis
\begin{xten}
val a : Int;
finish {
  async {
    a = 1;
  } 
  // a is not definitely assigned here
}
// a is definitely assigned after 'finish'
assert a==1; 
\end{xten}
%~~siv
%} } 
%~~neg
\end{ex}

This concept supports a core X10 programming idiom.  A \xcd`val` variable may
be initialized asynchronously, thereby providing a means for returning a value
from an \xcd`async` to be used after the enclosing \xcd`finish`.  

\section{Characteristics of Definite Assignment}

The properties ``definitely assigned'', ``singly assigned'', and
``definitely unassigned'' are computed by a conservative approximation of
X10's evaluation rules.

The precise details are up to the implementation. 
Many basic cases must be handled accurately; \eg, \xcd`x=1;` definitely and
singly assigns \xcd`x`.  

However, in more complicated cases, a conforming X10 may mark as invalid 
some code which, when executed, would actually be correct.  
For example, the following
program fragment will always result in \xcd`x` being definitely and singly
assigned:  
\begin{xten}
val x : Int;
var b : Boolean = mysterious();
if (b) {
   x = cryptic();
}
if (!b) { 
   x = unknown();
}
\end{xten}
However, most conservative approximations of program execution won't mark
\xcd`x` as properly initialized. For this to be correct, precisely one of the
two assignments to \xcd`x` must be executed. If \xcd`b` were true initially,
it would still be true after the call to \xcd`cryptic()` --- since methods
cannot modify their caller's local variables -- and so the first but not the
second assignment would happen. If \xcd`b` were false initially, it would
still be false when \xcd`!b` is tested, and so the second but not the first
assignment would happen.  Either way, \xcd`x` is definitely and singly assigned.

However, for a slightly different program, this analysis would be wrong. \Eg,
if  \xcd`b` were a field of \xcd`this` rather than a local variable,
\xcd`cryptic()` could change \xcd`b`; if \xcd`b` were true initially, both
assignments might happen, which is incorrect for a \xcd`val`.  

This sort of reasoning is beyond  most conservative approximation algorithms.
(Indeed, many do not bother checking that \xcd`!b` late in the program is the
opposite of \xcd`b` earlier.)
Algorithms that pay attention to such details and subtleties tend to be
fairly expensive, which would lead to very slow compilation for X10 -- for the
sake of obscure cases.

X10's analysis provides at least the following guarantees. We describe them in
terms of a statement \xcd`S` performing some collection of possible numbers of
assignments to variables --- on a scale of ``0'', ``1'', and ``many''. For
example, \xcd`if(b) x=1; else {x=1;x=2;y=2;}` might assign to \xcd`x` one or
many times, and might assign to \xcd`y` zero or one time. Hence, after it,
\xcd`x` is definitely assigned and may be multiply assigned, and \xcd`y` is
neither definitely assigned nor definitely unassigned.  

These descriptions are combined in natural ways.  For example, if \xcd`R` says
that \xcd`x` will be assigned 0 or 1 times, and \xcd`S` says it will be
assigned precisely once, then \xcd`R;S` will assign it one or many times.  If
only one or \xcd`R` or \xcd`S` will occur, as from \xcd`if(b)R; else S;`, 
then \xcd`x` may be assigned 0 or 1 times. 

This information is sufficient for the tests X10 makes.  If \xcd`x` can is
assigned one or many times in \xcd`S`, it is definitely assigned.  It is an
error if 
\xcd`x` is ever read at a point where it have been assigned zero times.  It is
an error if a \xcd`val` may be assigned many times.


We do not guarantee that any particular X10 compiler uses this algorithm;
indeed, as of the time of writing, the X10 compiler uses a somewhat more
precise one.  However, any conformant X10 compiler must provide results which
are at least as accurate as this analysis.




\subsubsection{Assignment: {\tt x = e}}   

\xcd`x = e` assigns to \xcd`x`, in addition to whatever assignments
\xcd`e` makes.   For example, if \xcd`this.setX(y)` sets a field \xcd`x` to
\xcd`y` and returns \xcd`y`, then \xcd`x = this.setX(y)` definitely and
multiply assigns \xcd`x`.  

\subsubsection{{\tt async} and {\tt finish}}

By itself, \xcd`async S` provides few guarantees.  After \xcd`async{x=1;}`
finishes, we know that there is a separate activity which will, when the
scheduler lets it, set \xcd`x` to \xcd`1`.  We do not know that anything has
happened yet.

However, if there is a \xcd`finish` around the \xcd`async`, the situation is
clearer.  After \xcd`finish{ async{ x=1; }}`, \xcd`x` has definitely been
assigned.  

In general, if an \xcd`async S` appears in the body of a \xcd`finish` in a way
that guarantees that it will be executed, then, after the \xcd`finish`, the
assignments made by \xcd`S` will have occurred.  For example, if \xcd`S`
definitely assigns to \xcd`x`, and the body of the \xcd`finish` guarantees
that \xcd`async S` will be executed, then \xcd`finish{...async S...}`
definitely assigns \xcd`x`.



\subsubsection{{\tt if} and {\tt switch}}

When \xcd`if(E) S else T` finishes, it will have performed the assignments of
\xcd`E`, together with those of either \xcd`S` or \xcd`T` but not both.  For
example, \xcd`if(b) x=1; else x=2;` definitely assigns \xcd`x`,
but \xcd`if(b) x=1;` does not.

{\tt switch} is more complex, but follows the same principles as \xcd`if`.
For example, \xcd`switch(E){case 1: A; break; case 2: B; default: C;}`  
performs the assignments of \xcd`E`, and those of precisely one of \xcd`A`, or
\xcd`B;C`, or \xcd`C`.  Note that case \xcd`2` falls through to the default
case, so it performs the same assignments as \xcd`B;C`.

\subsubsection{Sequencing}

When \xcd`R;S` finishes, it will have performed the assignments of \xcd`R` and
those of \xcd`S`.  For example, \xcd`x=1;y=2;` definitely assigns \xcd`x` and
\xcd`y`, and \xcd`x=1;x=2;` multiply assigns \xcd`x`. 


\subsubsection{Loops}

\xcd`while(E)S` performs the assignments of \xcd`E` one or more times, and
those of \xcd`S` zero or more times.  For example, if \xcd`while(b()){x=1;}`
might assign to \xcd`x` zero, one, or many times.  
\xcd`do S while(E)` performs the assignments of \xcd`E` one or more times, and
those of \xcd`S` one or more times. 

\xcd`for(A;B;C)D` performs the assignments of \xcd`A` once, those of \xcd`B`
one or more times, and those of \xcd`C` and \xcd`D` one or more times.
\xcd`for(x in E)S` performs the assignments of \xcd`E` once and those of
\xcd`S` zero or more times.  

Loops are of very little value for providing definite assignments, since X10
does not in general know how many times they will be executed. 

\xcd`continue` and \xcd`break` inside of a loop are hard to describe in simple
terms.  They may be conservatively assumed to cause the loop give no
information about the variables assigned inside of it.
For example, the analysis may conservatively conclude that 
\xcd`do{ x = 1; if (true) break; } while(true)` may 
assign to \xcd`x` zero, one, or many times, overlooking the more precise fact
that it is assigned once.  




\subsubsection{Method Calls}

A method call \xcd`E.m(A,B)` performs the assignments of \xcd`E`, \xcd`A`, and
\xcd`B` once each, and also those of \xcd`m`.  This implies that X10 must be
aware of the possible assignments performed by each method.


If X10 has complete information about \xcd`m` (as when \xcd`m` is a
\xcd`private` or \xcd`final` method), this is straightforward.  When such
information is fundamentally impossible to acquire, as when \xcd`m` is a
non-final method invocation, X10 has no choice but to assume that \xcd`m`
might do anything that a method can do.    (For this reason, the only methods
that can be called from within a constructor on a raw --
incompletely-constructed -- object) are the \xcd`private` and
\xcd`final` ones.)  
\begin{itemize}
\item \xcd`m` cannot assign to local fields of the caller; methods have no
      such power.
\item \xcd`m` can assign to \xcd`var` fields of \xcd`this` freely, unless this
      is prohibited by an annotation; see \Sref{somewhere}
\item \xcd`m` cannot initialize \xcd`val` fields of \xcd`this`.  (With one
      exception; see \Sref{sect:call-another-constructor}.) 
\end{itemize}

Recall that every container must be fully initialized (``cooked'') upon exit
from its constructor.  
X10 places certain restrictions on which methods can be called from a
constructor; see \Sref{sect:nonescaping}.  One of these restrictions is that
methods called before object initialization is complete must be \xcd`final` or
\xcd`private` --- and hence, available for static analysis.  So, when checking
field initialization, X10 will ensure: 
\begin{enumerate}
\item Each \xcd`val` field is initialized before it is read.   
      A method that does not read a \xcd`val` field \xcd`f` {\em may} be
      called before \xcd`f` is initialized; a method that reads \xcd`f` must
      not be called until \xcd`f` is initialized.        
      For example, 
      a constructor may have the form:
%~~gen ^^^ DefiniteAssignment4x6k
% package DefiniteAssignment4x6k;
%~~vis
\begin{xten}
class C {
  val f : Int;
  val g : String;
  def this() {
     f = fless();
     g = useF();
  }
  private def fless() = "f not used here".length();
  private def useF() = "f=" + this.f;
}
\end{xten}
%~~siv
%
%~~neg

\item \xcd`var` fields require a deeper analysis.  Consider a \xcd`var`
      field \xcd`var x:T`  without initializer.  If \xcd`T` has a default
      value, \xcd`x` may be read inside of a constructor before it is
      otherwise written, and it will 
      have its default value.   

      If \xcd`T` has no default value, an analysis
      like that used for \xcd`val`s must be performed to determine that
      \xcd`x` is initialized before it is used.  The situation is 
      more complex than for \xcd`val`s, however, because a method can assign to
      \xcd`x` as well read from it.  The X10 compiler computes a conservative
      approximation of which methods
      read and write which \xcd`var` fields. (Doing this carefully 
      requires finding a solution of a set of equations over sets of
      variables, with each callable method having equations describing what it
      reads and writes.)    

\end{enumerate}


\subsubsection{{\tt at} and \xcd`athome`}

%%AT-COPY%% \xcd`at(E)S` performs the assignments of \xcd`E`. Within \xcd`S`, only those
%%AT-COPY%% assignments to variables \xcd`x` from the surrounding environment which take
%%AT-COPY%% place within a suitable \xcd`athome(x)R` are counted. 
%%AT-COPY%% 
%%AT-COPY%% \begin{ex}
%%AT-COPY%% In the following code, the outer variable named \xcd`a` is definitely assigned
%%AT-COPY%% once, by the assignment \xcd`a = 3;`.  The inner variable (also named \xcd`a`)
%%AT-COPY%% is definitely multiply assigned 
%%AT-COPY%% by the two assignments \xcd`a = 1;` and \xcd`a = 2;` 
%%AT-COPY%% between the \xcd`at` and the \xcd`athome`.  
%%AT-COPY%% 
%%AT-COPY%% %~~gen ^^^ DefiniteAssignment3n5q
%%AT-COPY%% % package DefiniteAssignment3n5q;
%%AT-COPY%% % KNOWNFAIL-at
%%AT-COPY%% % class DefAss { def defass() { 
%%AT-COPY%% %~~vis
%%AT-COPY%% \begin{xten}
%%AT-COPY%% var a : Int;
%%AT-COPY%% at(here.next(); var a : Int = a) {
%%AT-COPY%%   a = 1;
%%AT-COPY%%   a = 2; 
%%AT-COPY%%   athome(a) a = 3;
%%AT-COPY%% }
%%AT-COPY%% \end{xten}
%%AT-COPY%% %~~siv
%%AT-COPY%% % } } 
%%AT-COPY%% %~~neg
%%AT-COPY%% 
%%AT-COPY%% 
%%AT-COPY%% \end{ex}
%%AT-COPY%% 

\xcd`at(p)S` cannot perform any assignments.

\subsubsection{{\tt atomic}}

\xcd`atomic S` performs the assignments of \xcd`S`, 
and \xcd`when(E)S` performs those of \xcd`E` and \xcd`S`.  

\subsubsection{{\tt try}}

\xcd`try S catch(x:T1) E1 ... catch(x:Tn) En  finally F` 
performs some or all of the assignments of \xcd`S`, plus all the assignments
of zero or one of the \xcd`E`'s, plus those of \xcd`F`.  
For example,
\begin{xten}
try {
  x = boomy();
  x = 0;
}
catch(e:Boom) { y = 1; }
finally { z = 1; }
\end{xten}
\noindent 
assigns \xcd`x` zero, one, or many times\footnote{A more precise
analysis could discover that \xcd`x` cannot be initialized only once.}, 
assigns \xcd`y` zero or one time, and assigns \xcd`z` exactly once.

\subsubsection{Expression Statements}

Expression statements \xcd`E;`, and other statements that execute an
expression and do something innocuous with it (local variable declaration and
\xcd`assert`) have the same effects as \xcd`E`. 

\subsubsection{{\tt return}, {\tt throw}}

Statements that do not finish normally, such as \xcd`return` and \xcd`throw`,
don't initialize anything (though the computation of the return or thrown
value may).    They also terminate a line of computation.  For example, 
\xcd`if(b) {x=1; return;}  x=2;` definitely and singly assigns \xcd`x`.  

\chapter{Grammar}


\begin{bbgrammar}

 MethodInvocation  \refstepcounter{equation}\label{prod:MethodInvocation}  \: MethodPrimaryPrefix \xcd"(" ArgumentList\opt \xcd")" & (\arabic{equation})\\
    \| MethodSuperPrefix \xcd"(" ArgumentList\opt \xcd")"\\
    \| MethodClassNameSuperPrefix \xcd"(" ArgumentList\opt \xcd")"\\
 Mod  \refstepcounter{equation}\label{prod:Mod}  \: \xcd"abstract" & (\arabic{equation})\\
    \| Annotation\\
    \| atomic\\
    \| \xcd"final"\\
    \| \xcd"native"\\
    \| \xcd"private"\\
    \| \xcd"protected"\\
    \| \xcd"public"\\
    \| \xcd"static"\\
    \| \xcd"transient"\\
    \| clocked\\
 TypeDefDecl  \refstepcounter{equation}\label{prod:TypeDefDecl}  \: Mods\opt type Id TypeParams\opt FormalParams\opt WhereClause\opt \xcd"=" Type \xcd";" & (\arabic{equation})\\
 Properties  \refstepcounter{equation}\label{prod:Properties}  \: \xcd"(" PropertyList \xcd")" & (\arabic{equation})\\
\end{bbgrammar}

\begin{bbgrammar}

 PropertyList  \refstepcounter{equation}\label{prod:PropertyList}  \: Property & (\arabic{equation})\\
    \| PropertyList \xcd"," Property\\
 Property  \refstepcounter{equation}\label{prod:Property}  \: Annotations\opt Id ResultType & (\arabic{equation})\\
 MethodDecl  \refstepcounter{equation}\label{prod:MethodDecl}  \: MethodMods\opt \xcd"def" Id TypeParams\opt FormalParams WhereClause\opt HasResultType\opt Offers\opt MethodBody & (\arabic{equation})\\
    \| MethodMods\opt \xcd"operator" TypeParams\opt \xcd"(" FormalParam  \xcd")" BinOp \xcd"(" FormalParam  \xcd")" WhereClause\opt HasResultType\opt Offers\opt MethodBody\\
    \| MethodMods\opt \xcd"operator" TypeParams\opt PrefixOp \xcd"(" FormalParam  \xcd")" WhereClause\opt HasResultType\opt Offers\opt MethodBody\\
    \| MethodMods\opt \xcd"operator" TypeParams\opt \xcd"this" BinOp \xcd"(" FormalParam  \xcd")" WhereClause\opt HasResultType\opt Offers\opt MethodBody\\
    \| MethodMods\opt \xcd"operator" TypeParams\opt \xcd"(" FormalParam  \xcd")" BinOp \xcd"this" WhereClause\opt HasResultType\opt Offers\opt MethodBody\\
    \| MethodMods\opt \xcd"operator" TypeParams\opt PrefixOp \xcd"this" WhereClause\opt HasResultType\opt Offers\opt MethodBody\\
    \| MethodMods\opt \xcd"operator" \xcd"this" TypeParams\opt FormalParams WhereClause\opt HasResultType\opt Offers\opt MethodBody\\
    \| MethodMods\opt \xcd"operator" \xcd"this" TypeParams\opt FormalParams \xcd"=" \xcd"(" FormalParam  \xcd")" WhereClause\opt HasResultType\opt Offers\opt MethodBody\\
    \| MethodMods\opt \xcd"operator" TypeParams\opt \xcd"(" FormalParam  \xcd")" \xcd"as" Type WhereClause\opt Offers\opt MethodBody\\
    \| MethodMods\opt \xcd"operator" TypeParams\opt \xcd"(" FormalParam  \xcd")" \xcd"as" \xcd"?" WhereClause\opt HasResultType\opt Offers\opt MethodBody\\
    \| MethodMods\opt \xcd"operator" TypeParams\opt \xcd"(" FormalParam  \xcd")" WhereClause\opt HasResultType\opt Offers\opt MethodBody\\
 PropertyMethodDecl  \refstepcounter{equation}\label{prod:PropertyMethodDecl}  \: MethodMods\opt Id TypeParams\opt FormalParams WhereClause\opt HasResultType\opt MethodBody & (\arabic{equation})\\
    \| MethodMods\opt Id WhereClause\opt HasResultType\opt MethodBody\\
 ExplicitCtorInvocation  \refstepcounter{equation}\label{prod:ExplicitCtorInvocation}  \: \xcd"this" TypeArguments\opt \xcd"(" ArgumentList\opt \xcd")" \xcd";" & (\arabic{equation})\\
    \| \xcd"super" TypeArguments\opt \xcd"(" ArgumentList\opt \xcd")" \xcd";"\\
    \| Primary \xcd"." \xcd"this" TypeArguments\opt \xcd"(" ArgumentList\opt \xcd")" \xcd";"\\
    \| Primary \xcd"." \xcd"super" TypeArguments\opt \xcd"(" ArgumentList\opt \xcd")" \xcd";"\\
 NormalInterfaceDecl  \refstepcounter{equation}\label{prod:NormalInterfaceDecl}  \: Mods\opt \xcd"interface" Id TypeParamsWithVariance\opt Properties\opt WhereClause\opt ExtendsInterfaces\opt InterfaceBody & (\arabic{equation})\\
 ClassInstCreationExp  \refstepcounter{equation}\label{prod:ClassInstCreationExp}  \: \xcd"new" TypeName TypeArguments\opt \xcd"(" ArgumentList\opt \xcd")" ClassBody\opt & (\arabic{equation})\\
    \| \xcd"new" TypeName \xcd"[" Type \xcd"]" \xcd"[" ArgumentList\opt \xcd"]"\\
    \| Primary \xcd"." \xcd"new" Id TypeArguments\opt \xcd"(" ArgumentList\opt \xcd")" ClassBody\opt\\
    \| AmbiguousName \xcd"." \xcd"new" Id TypeArguments\opt \xcd"(" ArgumentList\opt \xcd")" ClassBody\opt\\
 AssignPropertyCall  \refstepcounter{equation}\label{prod:AssignPropertyCall}  \: \xcd"property" TypeArguments\opt \xcd"(" ArgumentList\opt \xcd")" \xcd";" & (\arabic{equation})\\
 Type  \refstepcounter{equation}\label{prod:Type}  \: FunctionType & (\arabic{equation})\\
    \| ConstrainedType\\
 FunctionType  \refstepcounter{equation}\label{prod:FunctionType}  \: TypeParams\opt \xcd"(" FormalParamList\opt \xcd")" WhereClause\opt Offers\opt \xcd"=>" Type & (\arabic{equation})\\
 ClassType  \refstepcounter{equation}\label{prod:ClassType}  \: NamedType & (\arabic{equation})\\
 AnnotatedType  \refstepcounter{equation}\label{prod:AnnotatedType}  \: Type Annotations & (\arabic{equation})\\
 ConstrainedType  \refstepcounter{equation}\label{prod:ConstrainedType}  \: NamedType & (\arabic{equation})\\
    \| AnnotatedType\\
    \| \xcd"(" Type \xcd")"\\
 PlaceType  \refstepcounter{equation}\label{prod:PlaceType}  \: PlaceExp & (\arabic{equation})\\
 SimpleNamedType  \refstepcounter{equation}\label{prod:SimpleNamedType}  \: TypeName & (\arabic{equation})\\
    \| Primary \xcd"." Id\\
    \| DepNamedType \xcd"." Id\\
 DepNamedType  \refstepcounter{equation}\label{prod:DepNamedType}  \: SimpleNamedType DepParams & (\arabic{equation})\\
    \| SimpleNamedType Arguments\\
    \| SimpleNamedType Arguments DepParams\\
\end{bbgrammar}

\begin{bbgrammar}

    \| SimpleNamedType TypeArguments\\
    \| SimpleNamedType TypeArguments DepParams\\
    \| SimpleNamedType TypeArguments Arguments\\
    \| SimpleNamedType TypeArguments Arguments DepParams\\
 NamedType  \refstepcounter{equation}\label{prod:NamedType}  \: SimpleNamedType & (\arabic{equation})\\
    \| DepNamedType\\
 DepParams  \refstepcounter{equation}\label{prod:DepParams}  \: \xcd"{" ExistentialList\opt Conjunction\opt \xcd"}" & (\arabic{equation})\\
 TypeParamsWithVariance  \refstepcounter{equation}\label{prod:TypeParamsWithVariance}  \: \xcd"[" TypeParamWithVarianceList \xcd"]" & (\arabic{equation})\\
 TypeParams  \refstepcounter{equation}\label{prod:TypeParams}  \: \xcd"[" TypeParamList \xcd"]" & (\arabic{equation})\\
 FormalParams  \refstepcounter{equation}\label{prod:FormalParams}  \: \xcd"(" FormalParamList\opt \xcd")" & (\arabic{equation})\\
 Conjunction  \refstepcounter{equation}\label{prod:Conjunction}  \: Exp & (\arabic{equation})\\
    \| Conjunction \xcd"," Exp\\
 SubtypeConstraint  \refstepcounter{equation}\label{prod:SubtypeConstraint}  \: Type  \xcd"<:" Type  & (\arabic{equation})\\
    \| Type  \xcd":>" Type \\
 WhereClause  \refstepcounter{equation}\label{prod:WhereClause}  \: DepParams & (\arabic{equation})\\
 ExistentialList  \refstepcounter{equation}\label{prod:ExistentialList}  \: FormalParam & (\arabic{equation})\\
    \| ExistentialList \xcd";" FormalParam\\
 ClassDecl  \refstepcounter{equation}\label{prod:ClassDecl}  \: StructDecl & (\arabic{equation})\\
    \| NormalClassDecl\\
 NormalClassDecl  \refstepcounter{equation}\label{prod:NormalClassDecl}  \: Mods\opt \xcd"class" Id TypeParamsWithVariance\opt Properties\opt WhereClause\opt Super\opt Interfaces\opt ClassBody & (\arabic{equation})\\
 StructDecl  \refstepcounter{equation}\label{prod:StructDecl}  \: Mods\opt \xcd"struct" Id TypeParamsWithVariance\opt Properties\opt WhereClause\opt Interfaces\opt ClassBody & (\arabic{equation})\\
 CtorDecl  \refstepcounter{equation}\label{prod:CtorDecl}  \: Mods\opt \xcd"def" \xcd"this" TypeParams\opt FormalParams WhereClause\opt HasResultType\opt Offers\opt CtorBody & (\arabic{equation})\\
 Super  \refstepcounter{equation}\label{prod:Super}  \: \xcd"extends" ClassType & (\arabic{equation})\\
 FieldKeyword  \refstepcounter{equation}\label{prod:FieldKeyword}  \: val & (\arabic{equation})\\
    \| \xcd"var"\\
 VarKeyword  \refstepcounter{equation}\label{prod:VarKeyword}  \: val & (\arabic{equation})\\
    \| \xcd"var"\\
 FieldDecl  \refstepcounter{equation}\label{prod:FieldDecl}  \: Mods\opt FieldKeyword FieldDeclarators \xcd";" & (\arabic{equation})\\
    \| Mods\opt FieldDeclarators \xcd";"\\
 Statement  \refstepcounter{equation}\label{prod:Statement}  \: AnnotationStatement & (\arabic{equation})\\
    \| ExpStatement\\
 AnnotationStatement  \refstepcounter{equation}\label{prod:AnnotationStatement}  \: Annotations\opt NonExpStatement & (\arabic{equation})\\
 NonExpStatement  \refstepcounter{equation}\label{prod:NonExpStatement}  \: Block & (\arabic{equation})\\
    \| EmptyStatement\\
    \| AssertStatement\\
    \| SwitchStatement\\
    \| DoStatement\\
    \| BreakStatement\\
    \| ContinueStatement\\
    \| ReturnStatement\\
    \| ThrowStatement\\
\end{bbgrammar}

\begin{bbgrammar}

    \| TryStatement\\
    \| LabeledStatement\\
    \| IfThenStatement\\
    \| IfThenElseStatement\\
    \| WhileStatement\\
    \| ForStatement\\
    \| AsyncStatement\\
    \| AtStatement\\
    \| AtomicStatement\\
    \| WhenStatement\\
    \| AtEachStatement\\
    \| FinishStatement\\
    \| NextStatement\\
    \| ResumeStatement\\
    \| AssignPropertyCall\\
    \| OfferStatement\\
 OfferStatement  \refstepcounter{equation}\label{prod:OfferStatement}  \: offer Exp \xcd";" & (\arabic{equation})\\
 IfThenStatement  \refstepcounter{equation}\label{prod:IfThenStatement}  \: \xcd"if" \xcd"(" Exp \xcd")" Statement & (\arabic{equation})\\
 IfThenElseStatement  \refstepcounter{equation}\label{prod:IfThenElseStatement}  \: \xcd"if" \xcd"(" Exp \xcd")" Statement  \xcd"else" Statement  & (\arabic{equation})\\
 EmptyStatement  \refstepcounter{equation}\label{prod:EmptyStatement}  \: \xcd";" & (\arabic{equation})\\
 LabeledStatement  \refstepcounter{equation}\label{prod:LabeledStatement}  \: Id \xcd":" LoopStatement & (\arabic{equation})\\
 LoopStatement  \refstepcounter{equation}\label{prod:LoopStatement}  \: ForStatement & (\arabic{equation})\\
    \| WhileStatement\\
    \| DoStatement\\
    \| AtEachStatement\\
 ExpStatement  \refstepcounter{equation}\label{prod:ExpStatement}  \: StatementExp \xcd";" & (\arabic{equation})\\
 StatementExp  \refstepcounter{equation}\label{prod:StatementExp}  \: Assignment & (\arabic{equation})\\
    \| PreIncrementExp\\
    \| PreDecrementExp\\
    \| PostIncrementExp\\
    \| PostDecrementExp\\
    \| MethodInvocation\\
    \| ClassInstCreationExp\\
 AssertStatement  \refstepcounter{equation}\label{prod:AssertStatement}  \: \xcd"assert" Exp \xcd";" & (\arabic{equation})\\
    \| \xcd"assert" Exp  \xcd":" Exp  \xcd";"\\
 SwitchStatement  \refstepcounter{equation}\label{prod:SwitchStatement}  \: \xcd"switch" \xcd"(" Exp \xcd")" SwitchBlock & (\arabic{equation})\\
 SwitchBlock  \refstepcounter{equation}\label{prod:SwitchBlock}  \: \xcd"{" SwitchBlockStatementGroups\opt SwitchLabels\opt \xcd"}" & (\arabic{equation})\\
 SwitchBlockStatementGroups  \refstepcounter{equation}\label{prod:SwitchBlockStatementGroups}  \: SwitchBlockStatementGroup & (\arabic{equation})\\
    \| SwitchBlockStatementGroups SwitchBlockStatementGroup\\
 SwitchBlockStatementGroup  \refstepcounter{equation}\label{prod:SwitchBlockStatementGroup}  \: SwitchLabels BlockStatements & (\arabic{equation})\\
 SwitchLabels  \refstepcounter{equation}\label{prod:SwitchLabels}  \: SwitchLabel & (\arabic{equation})\\
\end{bbgrammar}

\begin{bbgrammar}

    \| SwitchLabels SwitchLabel\\
 SwitchLabel  \refstepcounter{equation}\label{prod:SwitchLabel}  \: \xcd"case" ConstantExp \xcd":" & (\arabic{equation})\\
    \| \xcd"default" \xcd":"\\
 WhileStatement  \refstepcounter{equation}\label{prod:WhileStatement}  \: \xcd"while" \xcd"(" Exp \xcd")" Statement & (\arabic{equation})\\
 DoStatement  \refstepcounter{equation}\label{prod:DoStatement}  \: \xcd"do" Statement \xcd"while" \xcd"(" Exp \xcd")" \xcd";" & (\arabic{equation})\\
 ForStatement  \refstepcounter{equation}\label{prod:ForStatement}  \: BasicForStatement & (\arabic{equation})\\
    \| EnhancedForStatement\\
 BasicForStatement  \refstepcounter{equation}\label{prod:BasicForStatement}  \: \xcd"for" \xcd"(" ForInit\opt \xcd";" Exp\opt \xcd";" ForUpdate\opt \xcd")" Statement & (\arabic{equation})\\
 ForInit  \refstepcounter{equation}\label{prod:ForInit}  \: StatementExpList & (\arabic{equation})\\
    \| LocalVariableDecl\\
 ForUpdate  \refstepcounter{equation}\label{prod:ForUpdate}  \: StatementExpList & (\arabic{equation})\\
 StatementExpList  \refstepcounter{equation}\label{prod:StatementExpList}  \: StatementExp & (\arabic{equation})\\
    \| StatementExpList \xcd"," StatementExp\\
 BreakStatement  \refstepcounter{equation}\label{prod:BreakStatement}  \: \xcd"break" Id\opt \xcd";" & (\arabic{equation})\\
 ContinueStatement  \refstepcounter{equation}\label{prod:ContinueStatement}  \: \xcd"continue" Id\opt \xcd";" & (\arabic{equation})\\
 ReturnStatement  \refstepcounter{equation}\label{prod:ReturnStatement}  \: \xcd"return" Exp\opt \xcd";" & (\arabic{equation})\\
 ThrowStatement  \refstepcounter{equation}\label{prod:ThrowStatement}  \: \xcd"throw" Exp \xcd";" & (\arabic{equation})\\
 TryStatement  \refstepcounter{equation}\label{prod:TryStatement}  \: \xcd"try" Block Catches & (\arabic{equation})\\
    \| \xcd"try" Block Catches\opt Finally\\
 Catches  \refstepcounter{equation}\label{prod:Catches}  \: CatchClause & (\arabic{equation})\\
    \| Catches CatchClause\\
 CatchClause  \refstepcounter{equation}\label{prod:CatchClause}  \: \xcd"catch" \xcd"(" FormalParam \xcd")" Block & (\arabic{equation})\\
 Finally  \refstepcounter{equation}\label{prod:Finally}  \: \xcd"finally" Block & (\arabic{equation})\\
 ClockedClause  \refstepcounter{equation}\label{prod:ClockedClause}  \: clocked \xcd"(" ClockList \xcd")" & (\arabic{equation})\\
 AsyncStatement  \refstepcounter{equation}\label{prod:AsyncStatement}  \: \xcd"async" ClockedClause\opt Statement & (\arabic{equation})\\
    \| clocked \xcd"async" Statement\\
 AtStatement  \refstepcounter{equation}\label{prod:AtStatement}  \: at PlaceExpSingleList Statement & (\arabic{equation})\\
 AtomicStatement  \refstepcounter{equation}\label{prod:AtomicStatement}  \: atomic Statement & (\arabic{equation})\\
 WhenStatement  \refstepcounter{equation}\label{prod:WhenStatement}  \: \xcd"when" \xcd"(" Exp \xcd")" Statement & (\arabic{equation})\\
 AtEachStatement  \refstepcounter{equation}\label{prod:AtEachStatement}  \: \xcd"ateach" \xcd"(" LoopIndex \xcd"in" Exp \xcd")" ClockedClause\opt Statement & (\arabic{equation})\\
    \| \xcd"ateach" \xcd"(" Exp \xcd")" Statement\\
 EnhancedForStatement  \refstepcounter{equation}\label{prod:EnhancedForStatement}  \: \xcd"for" \xcd"(" LoopIndex \xcd"in" Exp \xcd")" Statement & (\arabic{equation})\\
    \| \xcd"for" \xcd"(" Exp \xcd")" Statement\\
 FinishStatement  \refstepcounter{equation}\label{prod:FinishStatement}  \: \xcd"finish" Statement & (\arabic{equation})\\
    \| clocked \xcd"finish" Statement\\
 PlaceExpSingleList  \refstepcounter{equation}\label{prod:PlaceExpSingleList}  \: \xcd"(" PlaceExp \xcd")" & (\arabic{equation})\\
 PlaceExp  \refstepcounter{equation}\label{prod:PlaceExp}  \: Exp & (\arabic{equation})\\
 NextStatement  \refstepcounter{equation}\label{prod:NextStatement}  \: next \xcd";" & (\arabic{equation})\\
 ResumeStatement  \refstepcounter{equation}\label{prod:ResumeStatement}  \: resume \xcd";" & (\arabic{equation})\\
 ClockList  \refstepcounter{equation}\label{prod:ClockList}  \: Clock & (\arabic{equation})\\
    \| ClockList \xcd"," Clock\\
\end{bbgrammar}

\begin{bbgrammar}

 Clock  \refstepcounter{equation}\label{prod:Clock}  \: Exp & (\arabic{equation})\\
 CastExp  \refstepcounter{equation}\label{prod:CastExp}  \: Primary & (\arabic{equation})\\
    \| ExpName\\
    \| CastExp \xcd"as" Type\\
 TypeParamWithVarianceList  \refstepcounter{equation}\label{prod:TypeParamWithVarianceList}  \: TypeParamWithVariance & (\arabic{equation})\\
    \| TypeParamWithVarianceList \xcd"," TypeParamWithVariance\\
 TypeParamList  \refstepcounter{equation}\label{prod:TypeParamList}  \: TypeParam & (\arabic{equation})\\
    \| TypeParamList \xcd"," TypeParam\\
 TypeParamWithVariance  \refstepcounter{equation}\label{prod:TypeParamWithVariance}  \: Id & (\arabic{equation})\\
    \| \xcd"+" Id\\
    \| \xcd"-" Id\\
 TypeParam  \refstepcounter{equation}\label{prod:TypeParam}  \: Id & (\arabic{equation})\\
 AssignmentExp  \refstepcounter{equation}\label{prod:AssignmentExp}  \: Exp  \xcd"->" Exp  & (\arabic{equation})\\
 ClosureExp  \refstepcounter{equation}\label{prod:ClosureExp}  \: FormalParams WhereClause\opt HasResultType\opt Offers\opt \xcd"=>" ClosureBody & (\arabic{equation})\\
 LastExp  \refstepcounter{equation}\label{prod:LastExp}  \: Exp & (\arabic{equation})\\
 ClosureBody  \refstepcounter{equation}\label{prod:ClosureBody}  \: ConditionalExp & (\arabic{equation})\\
    \| Annotations\opt \xcd"{" BlockStatements\opt LastExp \xcd"}"\\
    \| Annotations\opt Block\\
 AtExp  \refstepcounter{equation}\label{prod:AtExp}  \: at PlaceExpSingleList ClosureBody & (\arabic{equation})\\
 FinishExp  \refstepcounter{equation}\label{prod:FinishExp}  \: \xcd"finish" \xcd"(" Exp \xcd")" Block & (\arabic{equation})\\
 identifier  \refstepcounter{equation}\label{prod:identifier}  \: \xcd"IDENTIFIER"  & (\arabic{equation})\\
 TypeName  \refstepcounter{equation}\label{prod:TypeName}  \: Id & (\arabic{equation})\\
    \| TypeName \xcd"." Id\\
 ClassName  \refstepcounter{equation}\label{prod:ClassName}  \: TypeName & (\arabic{equation})\\
 TypeArguments  \refstepcounter{equation}\label{prod:TypeArguments}  \: \xcd"[" TypeArgumentList \xcd"]" & (\arabic{equation})\\
 TypeArgumentList  \refstepcounter{equation}\label{prod:TypeArgumentList}  \: Type & (\arabic{equation})\\
    \| TypeArgumentList \xcd"," Type\\
 PackageName  \refstepcounter{equation}\label{prod:PackageName}  \: Id & (\arabic{equation})\\
    \| PackageName \xcd"." Id\\
 ExpName  \refstepcounter{equation}\label{prod:ExpName}  \: Id & (\arabic{equation})\\
    \| AmbiguousName \xcd"." Id\\
 MethodName  \refstepcounter{equation}\label{prod:MethodName}  \: Id & (\arabic{equation})\\
    \| AmbiguousName \xcd"." Id\\
 PackageOrTypeName  \refstepcounter{equation}\label{prod:PackageOrTypeName}  \: Id & (\arabic{equation})\\
    \| PackageOrTypeName \xcd"." Id\\
 AmbiguousName  \refstepcounter{equation}\label{prod:AmbiguousName}  \: Id & (\arabic{equation})\\
    \| AmbiguousName \xcd"." Id\\
 CompilationUnit  \refstepcounter{equation}\label{prod:CompilationUnit}  \: PackageDecl\opt TypeDecls\opt & (\arabic{equation})\\
    \| PackageDecl\opt ImportDecls TypeDecls\opt\\
    \| ImportDecls PackageDecl  ImportDecls\opt  TypeDecls\opt\\
    \| PackageDecl ImportDecls PackageDecl  ImportDecls\opt  TypeDecls\opt\\
\end{bbgrammar}

\begin{bbgrammar}

 ImportDecls  \refstepcounter{equation}\label{prod:ImportDecls}  \: ImportDecl & (\arabic{equation})\\
    \| ImportDecls ImportDecl\\
 TypeDecls  \refstepcounter{equation}\label{prod:TypeDecls}  \: TypeDecl & (\arabic{equation})\\
    \| TypeDecls TypeDecl\\
 PackageDecl  \refstepcounter{equation}\label{prod:PackageDecl}  \: Annotations\opt \xcd"package" PackageName \xcd";" & (\arabic{equation})\\
 ImportDecl  \refstepcounter{equation}\label{prod:ImportDecl}  \: SingleTypeImportDecl & (\arabic{equation})\\
    \| TypeImportOnDemandDecl\\
 SingleTypeImportDecl  \refstepcounter{equation}\label{prod:SingleTypeImportDecl}  \: \xcd"import" TypeName \xcd";" & (\arabic{equation})\\
 TypeImportOnDemandDecl  \refstepcounter{equation}\label{prod:TypeImportOnDemandDecl}  \: \xcd"import" PackageOrTypeName \xcd"." \xcd"*" \xcd";" & (\arabic{equation})\\
 TypeDecl  \refstepcounter{equation}\label{prod:TypeDecl}  \: ClassDecl & (\arabic{equation})\\
    \| InterfaceDecl\\
    \| TypeDefDecl\\
    \| \xcd";"\\
 Interfaces  \refstepcounter{equation}\label{prod:Interfaces}  \: \xcd"implements" InterfaceTypeList & (\arabic{equation})\\
 InterfaceTypeList  \refstepcounter{equation}\label{prod:InterfaceTypeList}  \: Type & (\arabic{equation})\\
    \| InterfaceTypeList \xcd"," Type\\
 ClassBody  \refstepcounter{equation}\label{prod:ClassBody}  \: \xcd"{" ClassBodyDecls\opt \xcd"}" & (\arabic{equation})\\
 ClassBodyDecls  \refstepcounter{equation}\label{prod:ClassBodyDecls}  \: ClassBodyDecl & (\arabic{equation})\\
    \| ClassBodyDecls ClassBodyDecl\\
 ClassBodyDecl  \refstepcounter{equation}\label{prod:ClassBodyDecl}  \: ClassMemberDecl & (\arabic{equation})\\
    \| CtorDecl\\
 ClassMemberDecl  \refstepcounter{equation}\label{prod:ClassMemberDecl}  \: FieldDecl & (\arabic{equation})\\
    \| MethodDecl\\
    \| PropertyMethodDecl\\
    \| TypeDefDecl\\
    \| ClassDecl\\
    \| InterfaceDecl\\
    \| \xcd";"\\
 FormalDeclarators  \refstepcounter{equation}\label{prod:FormalDeclarators}  \: FormalDeclarator & (\arabic{equation})\\
    \| FormalDeclarators \xcd"," FormalDeclarator\\
 FieldDeclarators  \refstepcounter{equation}\label{prod:FieldDeclarators}  \: FieldDeclarator & (\arabic{equation})\\
    \| FieldDeclarators \xcd"," FieldDeclarator\\
 VariableDeclaratorsWithType  \refstepcounter{equation}\label{prod:VariableDeclaratorsWithType}  \: VariableDeclaratorWithType & (\arabic{equation})\\
    \| VariableDeclaratorsWithType \xcd"," VariableDeclaratorWithType\\
 VariableDeclarators  \refstepcounter{equation}\label{prod:VariableDeclarators}  \: VariableDeclarator & (\arabic{equation})\\
    \| VariableDeclarators \xcd"," VariableDeclarator\\
 VariableInitializer  \refstepcounter{equation}\label{prod:VariableInitializer}  \: Exp & (\arabic{equation})\\
 ResultType  \refstepcounter{equation}\label{prod:ResultType}  \: \xcd":" Type & (\arabic{equation})\\
 HasResultType  \refstepcounter{equation}\label{prod:HasResultType}  \: \xcd":" Type & (\arabic{equation})\\
    \| \xcd"<:" Type\\
 FormalParamList  \refstepcounter{equation}\label{prod:FormalParamList}  \: FormalParam & (\arabic{equation})\\
\end{bbgrammar}

\begin{bbgrammar}

    \| FormalParamList \xcd"," FormalParam\\
 LoopIndexDeclarator  \refstepcounter{equation}\label{prod:LoopIndexDeclarator}  \: Id HasResultType\opt & (\arabic{equation})\\
    \| \xcd"[" IdList \xcd"]" HasResultType\opt\\
    \| Id \xcd"[" IdList \xcd"]" HasResultType\opt\\
 LoopIndex  \refstepcounter{equation}\label{prod:LoopIndex}  \: Mods\opt LoopIndexDeclarator & (\arabic{equation})\\
    \| Mods\opt VarKeyword LoopIndexDeclarator\\
 FormalParam  \refstepcounter{equation}\label{prod:FormalParam}  \: Mods\opt FormalDeclarator & (\arabic{equation})\\
    \| Mods\opt VarKeyword FormalDeclarator\\
    \| Type\\
 Offers  \refstepcounter{equation}\label{prod:Offers}  \: \xcd"offers" Type & (\arabic{equation})\\
 ExceptionTypeList  \refstepcounter{equation}\label{prod:ExceptionTypeList}  \: ExceptionType & (\arabic{equation})\\
    \| ExceptionTypeList \xcd"," ExceptionType\\
 ExceptionType  \refstepcounter{equation}\label{prod:ExceptionType}  \: ClassType & (\arabic{equation})\\
 MethodBody  \refstepcounter{equation}\label{prod:MethodBody}  \: \xcd"=" LastExp \xcd";" & (\arabic{equation})\\
    \| \xcd"=" Annotations\opt \xcd"{" BlockStatements\opt LastExp \xcd"}"\\
    \| \xcd"=" Annotations\opt Block\\
    \| Annotations\opt Block\\
    \| \xcd";"\\
 CtorBody  \refstepcounter{equation}\label{prod:CtorBody}  \: \xcd"=" CtorBlock & (\arabic{equation})\\
    \| CtorBlock\\
    \| \xcd"=" ExplicitCtorInvocation\\
    \| \xcd"=" AssignPropertyCall\\
    \| \xcd";"\\
 CtorBlock  \refstepcounter{equation}\label{prod:CtorBlock}  \: \xcd"{" ExplicitCtorInvocation\opt BlockStatements\opt \xcd"}" & (\arabic{equation})\\
 Arguments  \refstepcounter{equation}\label{prod:Arguments}  \: \xcd"(" ArgumentList\opt \xcd")" & (\arabic{equation})\\
 InterfaceDecl  \refstepcounter{equation}\label{prod:InterfaceDecl}  \: NormalInterfaceDecl & (\arabic{equation})\\
 ExtendsInterfaces  \refstepcounter{equation}\label{prod:ExtendsInterfaces}  \: \xcd"extends" Type & (\arabic{equation})\\
    \| ExtendsInterfaces \xcd"," Type\\
 InterfaceBody  \refstepcounter{equation}\label{prod:InterfaceBody}  \: \xcd"{" InterfaceMemberDecls\opt \xcd"}" & (\arabic{equation})\\
 InterfaceMemberDecls  \refstepcounter{equation}\label{prod:InterfaceMemberDecls}  \: InterfaceMemberDecl & (\arabic{equation})\\
    \| InterfaceMemberDecls InterfaceMemberDecl\\
 InterfaceMemberDecl  \refstepcounter{equation}\label{prod:InterfaceMemberDecl}  \: MethodDecl & (\arabic{equation})\\
    \| PropertyMethodDecl\\
    \| FieldDecl\\
    \| ClassDecl\\
    \| InterfaceDecl\\
    \| TypeDefDecl\\
    \| \xcd";"\\
 Annotations  \refstepcounter{equation}\label{prod:Annotations}  \: Annotation & (\arabic{equation})\\
    \| Annotations Annotation\\
 Annotation  \refstepcounter{equation}\label{prod:Annotation}  \: \xcd"@" NamedType & (\arabic{equation})\\
\end{bbgrammar}

\begin{bbgrammar}

 Id  \refstepcounter{equation}\label{prod:Id}  \: identifier & (\arabic{equation})\\
 Block  \refstepcounter{equation}\label{prod:Block}  \: \xcd"{" BlockStatements\opt \xcd"}" & (\arabic{equation})\\
 BlockStatements  \refstepcounter{equation}\label{prod:BlockStatements}  \: BlockStatement & (\arabic{equation})\\
    \| BlockStatements BlockStatement\\
 BlockStatement  \refstepcounter{equation}\label{prod:BlockStatement}  \: LocalVariableDeclStatement & (\arabic{equation})\\
    \| ClassDecl\\
    \| TypeDefDecl\\
    \| Statement\\
 IdList  \refstepcounter{equation}\label{prod:IdList}  \: Id & (\arabic{equation})\\
    \| IdList \xcd"," Id\\
 FormalDeclarator  \refstepcounter{equation}\label{prod:FormalDeclarator}  \: Id ResultType & (\arabic{equation})\\
    \| \xcd"[" IdList \xcd"]" ResultType\\
    \| Id \xcd"[" IdList \xcd"]" ResultType\\
 FieldDeclarator  \refstepcounter{equation}\label{prod:FieldDeclarator}  \: Id HasResultType & (\arabic{equation})\\
    \| Id HasResultType\opt \xcd"=" VariableInitializer\\
 VariableDeclarator  \refstepcounter{equation}\label{prod:VariableDeclarator}  \: Id HasResultType\opt \xcd"=" VariableInitializer & (\arabic{equation})\\
    \| \xcd"[" IdList \xcd"]" HasResultType\opt \xcd"=" VariableInitializer\\
    \| Id \xcd"[" IdList \xcd"]" HasResultType\opt \xcd"=" VariableInitializer\\
 VariableDeclaratorWithType  \refstepcounter{equation}\label{prod:VariableDeclaratorWithType}  \: Id HasResultType \xcd"=" VariableInitializer & (\arabic{equation})\\
    \| \xcd"[" IdList \xcd"]" HasResultType \xcd"=" VariableInitializer\\
    \| Id \xcd"[" IdList \xcd"]" HasResultType \xcd"=" VariableInitializer\\
 LocalVariableDeclStatement  \refstepcounter{equation}\label{prod:LocalVariableDeclStatement}  \: LocalVariableDecl \xcd";" & (\arabic{equation})\\
 LocalVariableDecl  \refstepcounter{equation}\label{prod:LocalVariableDecl}  \: Mods\opt VarKeyword VariableDeclarators & (\arabic{equation})\\
    \| Mods\opt VariableDeclaratorsWithType\\
    \| Mods\opt VarKeyword FormalDeclarators\\
 Primary  \refstepcounter{equation}\label{prod:Primary}  \: here & (\arabic{equation})\\
    \| \xcd"[" ArgumentList\opt \xcd"]"\\
    \| Literal\\
    \| \xcd"self"\\
    \| \xcd"this"\\
    \| ClassName \xcd"." \xcd"this"\\
    \| \xcd"(" Exp \xcd")"\\
    \| ClassInstCreationExp\\
    \| FieldAccess\\
    \| MethodInvocation\\
    \| MethodSelection\\
    \| OperatorFunction\\
 OperatorFunction  \refstepcounter{equation}\label{prod:OperatorFunction}  \: TypeName \xcd"." \xcd"+" & (\arabic{equation})\\
    \| TypeName \xcd"." \xcd"-"\\
    \| TypeName \xcd"." \xcd"*"\\
    \| TypeName \xcd"." \xcd"/"\\
\end{bbgrammar}

\begin{bbgrammar}

    \| TypeName \xcd"." \xcd"%"\\
    \| TypeName \xcd"." \xcd"&"\\
    \| TypeName \xcd"." \xcd"|"\\
    \| TypeName \xcd"." \xcd"^"\\
    \| TypeName \xcd"." \xcd"<<"\\
    \| TypeName \xcd"." \xcd">>"\\
    \| TypeName \xcd"." \xcd">>>"\\
    \| TypeName \xcd"." \xcd"<"\\
    \| TypeName \xcd"." \xcd"<="\\
    \| TypeName \xcd"." \xcd">="\\
    \| TypeName \xcd"." \xcd">"\\
    \| TypeName \xcd"." \xcd"=="\\
    \| TypeName \xcd"." \xcd"!="\\
 Literal  \refstepcounter{equation}\label{prod:Literal}  \: \xcd"IntegerLiteral"  & (\arabic{equation})\\
    \| \xcd"LongLiteral" \\
    \| \xcd"UnsignedIntegerLiteral" \\
    \| \xcd"UnsignedLongLiteral" \\
    \| \xcd"FloatingPointLiteral" \\
    \| \xcd"DoubleLiteral" \\
    \| BooleanLiteral\\
    \| \xcd"CharacterLiteral" \\
    \| \xcd"StringLiteral" \\
    \| \xcd"null"\\
 BooleanLiteral  \refstepcounter{equation}\label{prod:BooleanLiteral}  \: \xcd"true"  & (\arabic{equation})\\
    \| \xcd"false" \\
 ArgumentList  \refstepcounter{equation}\label{prod:ArgumentList}  \: Exp & (\arabic{equation})\\
    \| ArgumentList \xcd"," Exp\\
 FieldAccess  \refstepcounter{equation}\label{prod:FieldAccess}  \: Primary \xcd"." Id & (\arabic{equation})\\
    \| \xcd"super" \xcd"." Id\\
    \| ClassName \xcd"." \xcd"super"  \xcd"." Id\\
    \| Primary \xcd"." \xcd"class" \\
    \| \xcd"super" \xcd"." \xcd"class" \\
    \| ClassName \xcd"." \xcd"super"  \xcd"." \xcd"class" \\
 MethodInvocation  \refstepcounter{equation}\label{prod:MethodInvocation}  \: MethodName TypeArguments\opt \xcd"(" ArgumentList\opt \xcd")" & (\arabic{equation})\\
    \| Primary \xcd"." Id TypeArguments\opt \xcd"(" ArgumentList\opt \xcd")"\\
    \| \xcd"super" \xcd"." Id TypeArguments\opt \xcd"(" ArgumentList\opt \xcd")"\\
    \| ClassName \xcd"." \xcd"super"  \xcd"." Id TypeArguments\opt \xcd"(" ArgumentList\opt \xcd")"\\
    \| Primary TypeArguments\opt \xcd"(" ArgumentList\opt \xcd")"\\
 MethodSelection  \refstepcounter{equation}\label{prod:MethodSelection}  \: MethodName \xcd"." \xcd"(" FormalParamList\opt \xcd")" & (\arabic{equation})\\
    \| Primary \xcd"." Id \xcd"." \xcd"(" FormalParamList\opt \xcd")"\\
    \| \xcd"super" \xcd"." Id \xcd"." \xcd"(" FormalParamList\opt \xcd")"\\
\end{bbgrammar}

\begin{bbgrammar}

    \| ClassName \xcd"." \xcd"super"  \xcd"." Id \xcd"." \xcd"(" FormalParamList\opt \xcd")"\\
 PostfixExp  \refstepcounter{equation}\label{prod:PostfixExp}  \: CastExp & (\arabic{equation})\\
    \| PostIncrementExp\\
    \| PostDecrementExp\\
 PostIncrementExp  \refstepcounter{equation}\label{prod:PostIncrementExp}  \: PostfixExp \xcd"++" & (\arabic{equation})\\
 PostDecrementExp  \refstepcounter{equation}\label{prod:PostDecrementExp}  \: PostfixExp \xcd"--" & (\arabic{equation})\\
 UnannotatedUnaryExp  \refstepcounter{equation}\label{prod:UnannotatedUnaryExp}  \: PreIncrementExp & (\arabic{equation})\\
    \| PreDecrementExp\\
    \| \xcd"+" UnaryExpNotPlusMinus\\
    \| \xcd"-" UnaryExpNotPlusMinus\\
    \| UnaryExpNotPlusMinus\\
 UnaryExp  \refstepcounter{equation}\label{prod:UnaryExp}  \: UnannotatedUnaryExp & (\arabic{equation})\\
    \| Annotations UnannotatedUnaryExp\\
 PreIncrementExp  \refstepcounter{equation}\label{prod:PreIncrementExp}  \: \xcd"++" UnaryExpNotPlusMinus & (\arabic{equation})\\
 PreDecrementExp  \refstepcounter{equation}\label{prod:PreDecrementExp}  \: \xcd"--" UnaryExpNotPlusMinus & (\arabic{equation})\\
 UnaryExpNotPlusMinus  \refstepcounter{equation}\label{prod:UnaryExpNotPlusMinus}  \: PostfixExp & (\arabic{equation})\\
    \| \xcd"~" UnaryExp\\
    \| \xcd"!" UnaryExp\\
 MultiplicativeExp  \refstepcounter{equation}\label{prod:MultiplicativeExp}  \: UnaryExp & (\arabic{equation})\\
    \| MultiplicativeExp \xcd"*" UnaryExp\\
    \| MultiplicativeExp \xcd"/" UnaryExp\\
    \| MultiplicativeExp \xcd"%" UnaryExp\\
 AdditiveExp  \refstepcounter{equation}\label{prod:AdditiveExp}  \: MultiplicativeExp & (\arabic{equation})\\
    \| AdditiveExp \xcd"+" MultiplicativeExp\\
    \| AdditiveExp \xcd"-" MultiplicativeExp\\
 ShiftExp  \refstepcounter{equation}\label{prod:ShiftExp}  \: AdditiveExp & (\arabic{equation})\\
    \| ShiftExp \xcd"<<" AdditiveExp\\
    \| ShiftExp \xcd">>" AdditiveExp\\
    \| ShiftExp \xcd">>>" AdditiveExp\\
 RangeExp  \refstepcounter{equation}\label{prod:RangeExp}  \: ShiftExp & (\arabic{equation})\\
    \| ShiftExp  \xcd".." ShiftExp \\
 RelationalExp  \refstepcounter{equation}\label{prod:RelationalExp}  \: RangeExp & (\arabic{equation})\\
    \| SubtypeConstraint\\
    \| RelationalExp \xcd"<" RangeExp\\
    \| RelationalExp \xcd">" RangeExp\\
    \| RelationalExp \xcd"<=" RangeExp\\
    \| RelationalExp \xcd">=" RangeExp\\
    \| RelationalExp \xcd"instanceof" Type\\
    \| RelationalExp \xcd"in" ShiftExp\\
 EqualityExp  \refstepcounter{equation}\label{prod:EqualityExp}  \: RelationalExp & (\arabic{equation})\\
    \| EqualityExp \xcd"==" RelationalExp\\
\end{bbgrammar}

\begin{bbgrammar}

    \| EqualityExp \xcd"!=" RelationalExp\\
    \| Type  \xcd"==" Type \\
 AndExp  \refstepcounter{equation}\label{prod:AndExp}  \: EqualityExp & (\arabic{equation})\\
    \| AndExp \xcd"&" EqualityExp\\
 ExclusiveOrExp  \refstepcounter{equation}\label{prod:ExclusiveOrExp}  \: AndExp & (\arabic{equation})\\
    \| ExclusiveOrExp \xcd"^" AndExp\\
 InclusiveOrExp  \refstepcounter{equation}\label{prod:InclusiveOrExp}  \: ExclusiveOrExp & (\arabic{equation})\\
    \| InclusiveOrExp \xcd"|" ExclusiveOrExp\\
 ConditionalAndExp  \refstepcounter{equation}\label{prod:ConditionalAndExp}  \: InclusiveOrExp & (\arabic{equation})\\
    \| ConditionalAndExp \xcd"&&" InclusiveOrExp\\
 ConditionalOrExp  \refstepcounter{equation}\label{prod:ConditionalOrExp}  \: ConditionalAndExp & (\arabic{equation})\\
    \| ConditionalOrExp \xcd"||" ConditionalAndExp\\
 ConditionalExp  \refstepcounter{equation}\label{prod:ConditionalExp}  \: ConditionalOrExp & (\arabic{equation})\\
    \| ClosureExp\\
    \| AtExp\\
    \| FinishExp\\
    \| ConditionalOrExp \xcd"?" Exp \xcd":" ConditionalExp\\
 AssignmentExp  \refstepcounter{equation}\label{prod:AssignmentExp}  \: Assignment & (\arabic{equation})\\
    \| ConditionalExp\\
 Assignment  \refstepcounter{equation}\label{prod:Assignment}  \: LeftHandSide AssignmentOperator AssignmentExp & (\arabic{equation})\\
    \| ExpName  \xcd"(" ArgumentList\opt \xcd")" AssignmentOperator AssignmentExp\\
    \| Primary  \xcd"(" ArgumentList\opt \xcd")" AssignmentOperator AssignmentExp\\
 LeftHandSide  \refstepcounter{equation}\label{prod:LeftHandSide}  \: ExpName & (\arabic{equation})\\
    \| FieldAccess\\
 AssignmentOperator  \refstepcounter{equation}\label{prod:AssignmentOperator}  \: \xcd"=" & (\arabic{equation})\\
    \| \xcd"*="\\
    \| \xcd"/="\\
    \| \xcd"%="\\
    \| \xcd"+="\\
    \| \xcd"-="\\
    \| \xcd"<<="\\
    \| \xcd">>="\\
    \| \xcd">>>="\\
    \| \xcd"&="\\
    \| \xcd"^="\\
    \| \xcd"|="\\
 Exp  \refstepcounter{equation}\label{prod:Exp}  \: AssignmentExp & (\arabic{equation})\\
 ConstantExp  \refstepcounter{equation}\label{prod:ConstantExp}  \: Exp & (\arabic{equation})\\
 PrefixOp  \refstepcounter{equation}\label{prod:PrefixOp}  \: \xcd"+" & (\arabic{equation})\\
    \| \xcd"-"\\
    \| \xcd"!"\\
\end{bbgrammar}

\begin{bbgrammar}

    \| \xcd"~"\\
 BinOp  \refstepcounter{equation}\label{prod:BinOp}  \: \xcd"+" & (\arabic{equation})\\
    \| \xcd"-"\\
    \| \xcd"*"\\
    \| \xcd"/"\\
    \| \xcd"%"\\
    \| \xcd"&"\\
    \| \xcd"|"\\
    \| \xcd"^"\\
    \| \xcd"&&"\\
    \| \xcd"||"\\
    \| \xcd"<<"\\
    \| \xcd">>"\\
    \| \xcd">>>"\\
    \| \xcd">="\\
    \| \xcd"<="\\
    \| \xcd">"\\
    \| \xcd"<"\\
    \| \xcd"=="\\
    \| \xcd"!="\\
\end{bbgrammar}




\clearpage
\addcontentsline{toc}{chapter}{References}
\renewcommand{\bibname}{References}
\bibliographystyle{plain}
\bibliography{master}

%\documentclass[10pt,twoside,twocolumn,notitlepage]{report}
%\documentclass[12pt,twoside,notitlepage]{report}
\documentclass[10pt,twoside,notitlepage]{report}
\usepackage{tex/x10}
\usepackage{tex/tenv}
\def\Hat{{\tt \char`\^}}
\usepackage{url}
\usepackage{times}
\usepackage{tex/txtt}
\usepackage{ifpdf}
\usepackage{tocloft}
\usepackage{tex/bcprules}
\usepackage{xspace}
\usepackage{makeidx}

\newif\ifdraft
%\drafttrue
\draftfalse

\pagestyle{headings}
\showboxdepth=0
\makeindex

\usepackage{tex/commands}

\usepackage[
pdfauthor={Vijay Saraswat, Bard Bloom, Igor Peshansky, Olivier Tardieu, and David Grove},
pdftitle={X10 Language Specification},
pdfcreator={pdftex},
pdfkeywords={X10},
linkcolor=blue,
citecolor=blue,
urlcolor=blue
]{hyperref}

\ifpdf
          \pdfinfo {
              /Author   (Vijay Saraswat, Bard Bloom, Igor Peshansky, Olivier Tardieu, and David Grove)
              /Title    (X10 Language Specification)
              /Keywords (X10)
              /Subject  ()
              /Creator  (TeX)
              /Producer (PDFLaTeX)
          }
\fi

\def\headertitle{The \XtenCurrVer{} Report}
\def\integerversion{2.4}

% Sizes and dimensions

%\topmargin -.375in       %    Nominal distance from top of page to top of
                         %    box containing running head.
%\headsep 15pt            %    Space between running head and text.

%\textheight 9.0in        % Height of text (including footnotes and figures, 
                         % excluding running head and foot).

%\textwidth 5.5in         % Width of text line.
\columnsep 15pt          % Space between columns 
\columnseprule 0pt       % Width of rule between columns.

\parskip 5pt plus 2pt minus 2pt % Extra vertical space between paragraphs.
\parindent 0pt                  % Width of paragraph indentation.
%\topsep 0pt plus 2pt            % Extra vertical space, in addition to 
                                % \parskip, added above and below list and
                                % paragraphing environments.


\newif\iftwocolumn

\makeatletter
\twocolumnfalse
\if@twocolumn
\twocolumntrue
\fi
\makeatother

\iftwocolumn

\oddsidemargin  0in    % Left margin on odd-numbered pages.
\evensidemargin 0in    % Left margin on even-numbered pages.

\else

\oddsidemargin  .5in    % Left margin on odd-numbered pages.
\evensidemargin .5in    % Left margin on even-numbered pages.

\fi


\newtenv{example}{Example}[section]
\newtenv{planned}{Planned}[section]

\begin{document}

% \section{Work In Progress}
% \begin{itemize}
%     \item Rewrite first chapter
%     \item Describe library classes, including such fundamentals as Any and String
%     \item Examples for covariant/contravariant generics are wrong -- use Nate's examples
%     \item Describe local classes.
%     \item Reduce the use of \xcd`self` in constraints.
%     \item Copy sections of grammar to relevant sections of text.
%     \item Do something about 4.12.3
% \end{itemize}
% 
% {\bf Feedback:} 
% To help us the most, we would appreciate comments in one of these formats: 
% \begin{itemize}
% \item An annotated copy of the PDF document, if it's convenient.  Acrobat
%       Writer can produce helpful highlighting and sticky notes.  If you don't
%       use Acrobat Writer, don't fuss.
% \item Text comments.  Since the document is still being edited, page numbers
%       are going to be useless as pointers to the text.  If possible, we'd like
%       pointers to sections by number and title: {\em In 12.1, ``Empty
%       Statement'', please discuss side effects and performance implications
%       for this construct''}  If it's a long section, giving us a couple words
%       we can grep for would help too.
% \end{itemize}
% 
% Thank you very much!




% \parindent 0pt %!! 15pt                    % Width of paragraph indentation.

%\hfil {\bf 7 Feb 2005}
%\hfil \today{}

\input{first} 

\clearpage

{\parskip 0pt
\addtolength{\cftsecnumwidth}{0.5em}
\addtolength{\cftsubsecnumwidth}{0.5em}
%\addtolength{\cftsecindent}{0.5em}
\addtolength{\cftsubsecindent}{0.5em}
\tableofcontents
}


\input{Intro}
\input{Overview}
\input{Lex}
\input{Types}	
\input{Vars}
\input{Packages}
\input{Interfaces}
\input{Classes}
\input{Structs}
\input{Functions}
\input{Expressions}	
\input{Statements}	
\input{Places}	
\input{Activities}	
\input{Clocks}	
\input{Arrays}	
\input{Annotations}
\input{NativeCode}
\input{DefiniteAssignment}
\input{Grammar}


\clearpage
\addcontentsline{toc}{chapter}{References}
\renewcommand{\bibname}{References}
\bibliographystyle{plain}
\bibliography{master}

\input{x10.ind}

\appendix

\input{NotCovered.tex}
\input{ChangeLog.tex}
\input{Options.tex}

\chapter{Acknowledgments and Trademarks}

{\em The \Xten{} language has been developed as part of the IBM PERCS
Project, which is supported in part by the Defense Advanced Research
Projects Agency (DARPA) under contract No. NBCH30390004.}

{\em Java and all Java-based trademarks are trademarks of Sun Microsystems,
Inc. in the United States, other countries, or both.}
\end{document}


\appendix

\chapter{Deprecations}

X10 version 2.2 has a few relics of previous versions, code that is being used
by libraries but is not intended for general programming.    They should be
ignored.

These are: 

\begin{itemize}

\item \xcd`acc` variables. \index{acc}

\item The \xcd`offers` clause, as seen in the {\it Offers} nonterminal in the
      grammar (\ref{prod:Offers}).\index{offers}\index{Offers}

\item The grammar allows covariant and contravariant type parameters, marked
      by \xcd`+` and \xcd`-`: 
\begin{xtenmath}
class Variant[X, +Y, -Z] {}
\end{xtenmath}
      X10 does not support these in any other way.  

\end{itemize}

\chapter{Change Log}

\section{Changes from \Xten{} v2.3}
\Xten{} v2.4 is not backwards compatible with \Xten{} v2.3. The
motivation for making backwards incompatible language changes with
this release of \Xten{} is to significantly improve the ability of the
\Xten{} programmer to exploit the expanded memory capabilities of
modern computer systems.  In particular, \Xten{} v2.4 includes an
extensive redesign of arrays and a change of the default type of
unqualified integral literals (\eg \xcd`2`) from \xcd{Int} to
\xcd{Long}. Taken together these two changes enable natural
exploitation of large memories via 64-bit addressing and
\xcd{Long}-based indexing of arrays and similar data structures. 

\subsection{Integral Literals}

The default type of unqualified integral literals was change from
\xcd{Int} to \xcd{Long}. 

The qualifying suffix \xcd`n` and \xcd`un` are used to indicate
\xcd{Int} and \xcd{UInt} literals respectively.  The suffix xcd\`u` is
now interpreted as indicating a \xcd{ULong} literal.

\subsection{Arrays}

An extensive redesign of the \Xten{} abstractions for arrays is the
major new feature of the \Xten{} v2.4 release.  Although this redesign
only involved very minor changes to the actual \Xten{} language
specification, the core class libraries did change significantly.  As
mentioned above, the driving motivation for
the change was a long-contemplated strategic decision to shift to from
\xcd{Int}-based (32-bit) to \xcd{Long}-based (64-bit) indexing for all
\Xten{} arrays. This change enables \Xten{} to better utilize the rapidly
expanding memory capacity and 64-bit address space found on modern
machines. For consistency, the \xcd{id} field of \xcd{x10.lang.Place} and
the \xcd{size} and indexing-related APIs of the \xcd{x10.util}
collection hierarchy were also changed from \xcd{Int} to \xcd{Long}.

Once this inherently backwards-incompatible decision was made, the
\Xten{} team decided to do a larger rethinking of all of \Xten{}'s
array implementations to introduce a new time and space optimal
implementation of zero-based, dense, rectangular multi-dimensional
arrays.  This new implementation, in the \xcd{x10.array} package, is
intended to provide the best possible performance for the common-case
it supports.  The previous, more general array implementation is still
available, but has been relocated to a new package
\xcd{x10.array.regionarray}.  In addition, the \xcd{x10.lang.Rail}
class was re-introduced as a separate class in its own right and
provides the intrinsic indexed storage abstraction on which both array
packages are built.  The intent is that the combination of \xcd{Rail},
\xcd{x10.array} and \xcd{x10.regionarray} provide a spectrum of array
abstractions that capture common usage patterns and enable appropriate
trade-offs between performance and flexibility. 

In more detail the major array-related changes made in the \Xten{}
v2.4 release are 
\begin{enumerate}
\item The class \xcd{x10.lang.Rail} was introduced.  It provides
  an efficient one-dimensional, zero-based, densely indexed array
  implementation. \xcd{Rail} will provide the best performance and is
  the preferred implementation of this basic abstraction. 
\item The array literal syntax \xcd{[1,2,3]} is now defined to create
  a \xcd{Rail} instead of an \xcd{Array}. 
\item The main method signature is changed from \xcd{Array[String]} to
  \xcd{Rail[String]}. 
\item \xcd{x10.util.IndexedMemoryChunk} has removed from the \Xten{}
  standard library.
\item To enable usage of classes from both \xcd{x10.array} and
  \xcd{x10.regionarray}, the package \xcd{x10.array} is no longer
  auto-imported by the \Xten{} compiler.
\item Most classes in the \xcd{x10.array} package in the \Xten{} v2.3
  release were relocated to the \xcd{x10.regionarray} package in
  v2.4. A few classes like \xcd{Point} and \xcd{PlaceGroup} were moved
  to the \xcd{x10.lang} package instead.
\item \xcd{Point}, \xcd{Region}, \xcd{Dist}, etc. were all updated to
  support long-based indexing by consistently changing indexing
  related fields and methods from \xcd{Int} to \xcd{Long}. 
\end{enumerate}

\subsection{Other Changes from \Xten{} v2.3}

\begin{enumerate}
\item The custom serialization protocol was changed to operate in
  terms of new user-level classes \xcd{x10.io.Serializer} and
  \xcd{x10.io.Deserializer}.  The \xcd{serialize} method of
  the xcd{x10.io.CustomSerialization} interface now takes a
  \xcd{Serializer} as an argument.  The custom deserialization
  constructor for a class takes a \xcd{Deserializer}. The
  \xcd{x10.io.SerialData} class used by the \Xten{} v2.3 custom
  serialization protocol has been removed from the class library. 

\item A constraint was added to \xcd{PlaceLocalHandle} that types used
  to instantiate a \xcd{PlaceLocalHandle} must satisfy both the
  \xcd`isref` and \xcd`haszero` constraints.

\item The \xcd{x10.util.Team} API was revised by (a) removing the
  endpoint argument from all API calls and (b) to operate on \xcd`Rail` and
  xcd`Long` where appropriate.
\end{enumerate}

\section{Changes from \Xten{} v2.2}

\begin{enumerate}

\item In previous versions of \Xten{} static fields were
  eagerly initialized in \xcd{Place 0} and the resulting values were
  serialized to all other places before execution of the user main
  function was started. Starting with \Xten{} v2.2.3, static fields are
  lazily initialized on a per-Place basis when the field is first read
  by an activity executing in a given Place.

\item The new syntax \xcd{T isref} for some type \xcd{T} will hold if
  \xcd{T} is represented by a pointer at runtime.  This is similar to
  the type constraint \xcd{T haszero}.  \xcd{T isref} is true for
  \xcd{T} that are function types, classes, and all values that have
  been cast to interfaces (including boxed structs).  \xcd{T isref} is used
  in the standard library, e.g. for the \xcd{GlobalRef[T]} and
  \xcd{PlaceLocalHandle[T]} APIs.

\item \xcd{x10.lang.Object is gone}, there is now no single class that
  is the root of the X10 class hierarchy.
\begin{itemize}
  \item If, for some reason, you were explicitly extending \xcd{Object}, don't do
    that anymore.
  \item If you were doing \xcd{new Object()} to get a fresh value, use
    \xcd{new Empty()} instead.
  \item If you were using \xcd{Object} as a supertype, use \xcd{Any}
    (the one true supertype).  
  \item If you were using the type constraint \xcd{T <: Object} to
    disallow structs, use \xcd{T isref} instead.
\end{itemize}

\item The exception hierarchy has changed, and checked exceptions have
  been reintroduced.  The 'throws' annotation is required on methods,
  as in Java.  It is not supported on closures, so checked exceptions
  cannot be thrown from a closure.  The exception hierarchy has been
  chosen to exist in a 1:1 relationship with Java's.  However, unlike
  Java, we prefer using unchecked exceptions wherever possible, and
  this is reflected in the naming of the X10 classes.  The following
  classes are all in the \xcd{x10.lang} package.
  \begin{itemize}
  \item \xcd{CheckedThrowable} (mapped to \xcd{java.lang.Throwable}) 
  \item \xcd{CheckedException extends CheckedThrowable} (mapped to \xcd{java.lang.Exception}) 
  \item \xcd{Exception extends CheckedException}  (mapped to \xcd{java.lang.RuntimeException}) 
  \item \xcd{Error extends CheckedThrowable}  (mapped to \xcd{java.lang.Error}) 
  \end{itemize}

  Anything under \xcd{CheckedThrowable} can be thrown using the \xcd{throw}
  statement. But anything that is not under \xcd{Exception} or \xcd{Error} can
  only be thrown if it is caught by an enclosing \xcd{try/catch}, or it is
  thrown from a method with an appropriate throws annotation, as in
  Java.

  \xcd{RuntimeException} is gone from \Xten{}.  Use \xcd{Exception} instead.

  All the exceptions in the standard library are under \xcd{Exception},
  except \xcd{AssertionError} and \xcd{OutOfMemoryException}, which are under
  \xcd{Error} (as in Java).  This means all exceptions in the standard
  library remain unchecked.

\end{enumerate}

\section{Changes from \Xten{} v2.1}

\begin{enumerate}

\item Covariance and contravariance are gone.

\item Operator definitions are regularized.  A number of new operator symbols
      are available.

\item The operator \xcd`in` is gone.  \xcd`in` is now only a keyword.

\item Method functions and operator functions are gone.

\item \xcd`m..n` is now a type of struct called \xcd`IntRange`.  

\item \xcd`for(i in m..n)` now works.  The old forms, \xcd`for((i) in m..n)`
      and \xcd`for([i] in m..n)`, are no longer needed.

\item \xcd`(e as T)` now has type \xcd`T`.  (It used to have an identity
      constraint conjoined in.)

\item \xcd`var`s can no longer be assigned in their place of origin.  Use a
      \xcd`GlobalRef[Cell[T]]` instead.  We'll have a new idiom for this in 2.3.

\item The \xcd`-STATIC_CALLS` command-line flag is now \xcd`-STATIC_CHECKS`.

\item Any string may be written in backquotes to make an identifier: {\tt
      `while`}.

\item The \xcd`next` and \xcd`resume` keywords are gone; they have been
      replaced by static methods on \xcd`Clock`.

\item The typed array construction syntax \xcd`new Array[T][t1,t2]` is gone.
      Use \xcd`[t1 as T, t2]` (if just plain \xcd`[t1,t2]` doesn't work).

\end{enumerate}


\section{Changes from \Xten{} v2.0.6}

This document summarizes the main changes between X10 2.0.6 and X10 2.1.  The
descriptions are intended to be suggestive rather than definitive; see the
language specification for full details.

\subsection{Object Model}

\begin{enumerate}
\item Objects are now local rather than global.
   
    \begin{enumerate}
    \item The \Xcd{home} property is gone.
    \item \Xcd{at(P)S} produces deep copies of all objects reachable from
          lexically exposed variables in \xcd`S` 
          when it executes \Xcd{S}.  ({\bf Warning:} They are copied even in  
          \Xcd{at(here)S}.)
    \end{enumerate}

\item The \Xcd{GlobalRef[T]} struct is the only way to produce or manipulate
      cross-place references.
    \begin{enumerate}
    \item \Xcd{GlobalRef}'s have a \Xcd{home} property.
    \item Use \Xcd{GlobalRef[Foo](foo)} to make a new global reference.
    \item Use \Xcd{myGlobalRef()} to access the object referenced; this
          requires \Xcd{here == myGlobalRef.home}. 
    \end{enumerate}


\item  The \xcd`!` type modifier is no longer needed or present.

\item \Xcd{global} modifiers are now gone:
    
    \begin{enumerate}
    \item \Xcd{global} methods in {\em interfaces} are now the default. 
    \item \Xcd{global} {\em fields} are gone.  In some cases object copying
          will produce the same effect as global fields.  In other cases code
          must be rewritten.  It may be desirable to mark nonglobal fields
          \Xcd{transient} in many cases.
    \item \Xcd{global} {\em methods} are now marked \Xcd{@Global} instead.  
          Methods intended to be non-global may be marked \Xcd{@Pinned}.
    \end{enumerate}


\end{enumerate}

\subsection{Constructors}


\begin{enumerate}
\item \Xcd{proto} types are gone.
\item Constructors and the methods they call must satisfy a number of static
      checks.  
    
    \begin{enumerate}
    \item Constructors can only invoke \Xcd{private} or \Xcd{final} methods, 
          or methods annotated \xcd`@NonEscaping`.  
    \item Methods invoked by constructors cannot read fields before they are
          written. 
    \item The compiler ensures this with a detailed protocol. 
    \end{enumerate}

\item It is still impossible for X10 constructors to leak references to
      \Xcd{this} or observe uninitialized fields of an object.  Now, however,
      the mechanisms enforcing this are less obtrusive than in 2.0.6; the
      burden is largely on the compiler, not the programmer.
\end{enumerate}




%REF> \subsection{Call by Reference}
%REF> 
%REF> A very limited form of call-by-reference is now available.
%REF> 
%REF> 
%REF> \begin{enumerate}
%REF> 
%REF> \item Formal parameters to functions and methods may be \Xcd{ref} rather than
%REF>       \Xcd{var} or \Xcd{val}.  
%REF> \item Assignment to a \Xcd{ref} parameter \Xcd{x} changes the original
%REF>       location that the \Xcd{ref} refers to.  \eg, 
%REF>       \xcd`def inc(ref x:Int) { x ++; }`
%REF>       allows a call \Xcd{inc(n)} to increment a local \Xcd{var} \Xcd{n}.
%REF> \item Only local variables or \Xcd{ref} parameters can be passed as actual
%REF>       \Xcd{ref} parameters.  Fields, array elements, and other variable-like
%REF>       items cannot be. 
%REF> \item External \Xcd{ref} variables cannot be captured in closures. However,
%REF>       closures may have \Xcd{ref} parameters.
%REF> \item \Xcd{ref}s are {\em not} first-class objects in X10. They cannot be
%REF>       returned from functions, stored in data structures, etc.
%REF> \item These restrictions limit the possibilities of aliasing and the need for
%REF>       boxing of \Xcd{ref} parameters.  \Xcd{ref}s to stack locations cannot,
%REF>       with these restrictions, live past the death of the location's
%REF>       containing stack frame.      
%REF> \item This allows the implementation of many core constructs as syntactic
%REF>       sugar on library calls.   Programmers may use it, but mutability should
%REF>       generally be encapsulated inside objects rather than \Xcd{ref}
%REF>       parameters. 
%REF> \end{enumerate}
%REF> 

%ACC> \subsection{Accumulator Variables}
%ACC> 
%ACC> Accumulator variables generalize and make explicit collecting \Xcd{finish} in
%ACC> X10 2.0.6.  An \Xcd{acc} variable is declared: 
%ACC> \begin{xten}
%ACC> acc(r) A;
%ACC> \end{xten}
%ACC> where \Xcd{r} is a {\em reducer} (much as in 2.0.6): 
%ACC> \begin{xten}
%ACC> struct Reducer[T](zero:T, apply:global (T,T)=>T){}
%ACC> \end{xten}
%ACC> 
%ACC> Usage of \Xcd{A} is restricted in ways that make it determinate in the
%ACC> intended case of a pure, associative, commutative \Xcd{apply} with unit
%ACC> element \Xcd{zero}.  
%ACC> 
%ACC> \begin{enumerate}
%ACC> \item \Xcd{A} is initialized to \Xcd{r.zero}.  
%ACC> \item Multiple activities can {\em write} into \Xcd{A}.  In particular, the
%ACC>       ``assignment'' \Xcd{A = v} is approximately interpreted as 
%ACC>       \xcd`atomic{A = r.apply(A, v)}` --- that is, it accumulates \Xcd{v} into
%ACC>       \Xcd{A} using \Xcd{r.apply.}
%ACC> \item {\em Reading} of \Xcd{A} is restricted to situations where it makes
%ACC>       sense.  Specifically, only the activity in which \Xcd{A} is declared can
%ACC>       read from it, and it can only do so when all asyncs which it has spawned
%ACC>       have terminated -- \eg, outside of the scope of all \Xcd{async}s and
%ACC>       \Xcd{finish}es.  
%ACC> \item Formal parameters of functions may be marked \Xcd{acc x:T}.  The reducer
%ACC>       \Xcd{r} must not be specified; it is passed as an implicit parameter
%ACC>       going with the actual \Xcd{acc} variable.  
%ACC> \item X10 provides protocols for indexed collections of \Xcd{acc} variables,
%ACC>       presented as objects.
%ACC> \end{enumerate}
%ACC> 


\subsection{Implicit clocks for each finish}


Most clock operations can be accomplished using the new implicit clocks.

\begin{enumerate}
\item A \Xcd{finish} may be qualified with \Xcd{clocked}, which gives it a
      clock.
\item An \Xcd{async} in a \Xcd{clocked finish} may be marked \Xcd{clocked}.
      This registers it on the same clock as the enclosing \Xcd{finish}.  
\item \xcd`clocked async S` and \xcd`clocked finish S` may use \xcd`next` in
      the body of \Xcd{S} to advance the clock.
\item When the body of a \Xcd{clocked finish} completes, the \Xcd{clocked
      finish} is dropped form the clock.  It will still wait for spawned
      asyncs to terminate, but such asyncs need to wait for it.
\end{enumerate}


%CLOCAL>\subsection{Clocked local variables}
%CLOCAL>
%CLOCAL>Local \Xcd{val} and \Xcd{acc} variables may be \Xcd{clocked}.  They are
%CLOCAL>associated with the clock of the surrounding \Xcd{clocked finish}.  
%CLOCAL>Clocked variables have a {\em current} value and an {\em upcoming} value.  The
%CLOCAL>current value may be read at suitable times; the upcoming value may be
%CLOCAL>updated.  The \Xcd{next} phase makes the upcoming value current.

\subsection{Asynchronous initialization of val}

\Xcd{val}s can be initialized asynchronously.   As always with \Xcd{val}s,
they can only be read after it is guaranteed that they have been initialized.
For example, both of the \Xcd{print}s below are good.  However, the
commented-out \Xcd{print} in the \Xcd{async} is bad, since it is possible that
it will be executed before the initialization of \Xcd{a}. 
\begin{xten}
val a:Int;
finish {
  async {
     a = 1; 
     print("a=" + a);
  }
  // WRONG: print("a=" + a);
}
print("a=" + a);
\end{xten}



\subsection{Main Method}

The signature for the \Xcd{main} method is now: 
\begin{xten}
           def main(Array[String]) {..}
\end{xten}
or, if the arguments are actually used, 
\begin{xten}
           def main(argv: Array[String](1)) {..}
\end{xten}

\subsection{Assorted Changes}


\begin{enumerate}
\item The syntax for destructuring a point now uses brackets rather than
      braces: \Xcd{for( [i] in 1..10 )}, rather than the prior \Xcd{(i)}.  
\end{enumerate}

\subsection{Safety of atomic and when blocks}


\begin{enumerate}
\item Static effect annotations (\Xcd{safe}, \Xcd{sequential},
      \Xcd{nonblocking}, \Xcd{pinned}) are no longer used. They have been
      replaced by dynamic checks.
\item Using an inappropriate operation in the scope of an \Xcd{atomic} or
      \Xcd{when} construct will throw \Xcd{IllegalOperationException}.  
      The following are inappropriate:      
      \begin{itemize}
      \item \Xcd{when}
      \item \Xcd{resume()} or \Xcd{next} on clocks
      \item async
      \item \Xcd{Future.make()}, or \Xcd{Future.force()}.
      \item \Xcd{at}
      \end{itemize}

\end{enumerate}


\subsection{Removed Topics}

The following are gone: 

\begin{enumerate}
\item \Xcd{foreach} is gone.
\item All \Xcd{var}s are effectively \Xcd{shared}, so \Xcd{shared} is gone.
\item The place clause on \Xcd{async} is gone.  \Xcd{async (P) S} should be
      written \Xcd{at(P) async S}.
\item Checked exceptions are gone.
\item \Xcd{future} is gone.
\item \Xcd{await ... or ... } is gone.
\item \Xcd{const} is gone.

\end{enumerate}

\subsection{Deprecated}

The following constructs are still available, but are likely to be replaced in
a future version: 


\begin{enumerate}
\item \Xcd{ValRail}.
\item \Xcd{Rail}.
\item \xcd`ateach`
\item \xcd`offers`.  \index{offers}  The \xcd`offers` concept was experimental
      in 2.1, but was determined inadequate.  It has not been removed from the
      compiler yet, but it will be soon.  In the meantime, traces of it are
      still visible in the grammar.  They should not be used and can safely be ignored.
\end{enumerate}

\section{Changes from \Xten{} v2.0}
Some of these changes have been made obsolete in X10 2.2.

\begin{itemize}
\item \Xcd{Any} is now the top of the type hierarchy (every object,
  struct and function has a type that is a subtype of
  \Xcd{Any}). \Xcd{Any} defines \Xcd{home}, \Xcd{at}, \Xcd{toString},
  \Xcd{typeName}, \Xcd{equals} and \Xcd{hashCode}. \Xcd{Any} also defines the methods
  of \Xcd{Equals}, so \Xcd{Equals} is not needed any more.
\item Revised discussion of incomplete types.
\item The manual has been revised and brought into line with the current implementation. 
\end{itemize}
\section{Changes from \Xten{} v1.7}

The language has changed in the following ways.  
Some of these changes have been made obsolete in X10 2.2.

\begin{itemize}

\item {\bf Type system changes}: There are now three kinds of entities
  in an \Xten{} computation: objects, structs and functions. Their
  associated types are class types, struct types and function
  types. 

  Class and struct types are called {\em container types} in that they
  specify a collection of fields and methods. Container types have a
  name and a signature (the collection of members accessible on that
  type). Collection types support primitive equality \Xcd{==} and may
  support user-defined equality if they implement the {\tt
    x10.lang.Equals} interface. 

  Container types (and interface types) may be further qualified with
  constraints.

  A function type specifies a set of arguments and their type, the
  result type, and (optionally) a guard. A function application
  type-checks if the arguments are of the given type and the guard is
  satisfied, and the return value is of the given type.  A function
  type does not permit \Xcd{==} checks. Closure literals create
  instances of the corresponding function type.

  Container types may implement interfaces and zero or more function
  types.

  All types support a basic set of operations that return a string
  representation, a type name, and specify the home place of the entity.

  The type system is not unitary. However, any type may be used to
  instantiate a generic type. 

  There is no longer any notion of \Xcd{value} classes. \Xcd{value}
  classes must be re-written into structs or (reference) classes. 

\item {\bf Global object model}: Objects are instances of
  classes. Each object is associated with a globally unique
  identifier. Two objects are considered identical \Xcd{==} if their
  ids are identical. Classes may specify \Xcd{global} fields and
  methods. These can be accessed at any place. (\Xcd{global} fields
  must be immutable.)

\item{\bf Proto types.} For the decidability of dependent type
  checking it is necessary that the property graph is acyclic. This is
  ensured by enforcing rules on the leakage of \Xcd{this} in
  constructors. The rules are flexible enough to permit cycles to be
  created with normal fields, but not with properties.

\item{Place types.} Place types are now implemented. This means that
  non-global methods can be invoked on a variable, only if the
  variable's type is either a struct type or a function type, or a
  class type whose constraint specifies that the object is located in
  the current place.

  There is still no support for statically checking array access
  bounds, or performing place checks on array accesses.

\end{itemize}

\chapter{Options}

\subsection{Compiler Options}

The X10 compilers have many useful options. 

% -CHECK_INVARIANTS seems to check some internal compiler invariants.

\subsection{Optimization: {\tt -O} or {\tt -optimize}}

This flag causes the compiler to generate optimized code.


\subsection{Debugging: {\tt -DEBUG=boolean}}

This flag, if true, causes the compiler to generate debugging information.  It
is false by default.

\subsection{Call Style: {\tt -STATIC\_CHECKS, -VERBOSE\_CHECKS}}
\label{sect:Callstyle}
\index{STATIC\_CHECKS}
\index{VERBOSE\_CHECKS}
\index{dynamic checks}

By default, if a method call {\em could} be correct but is not {\em
necessarily} correct, the X10 compiler generates a dynamic check to ensure
that it is correct before it is performed.  For example, the following code: 
\begin{xten}
def use(n:Int{self == 0}) {}
def test(x:Int) { 
   use(x); // creates a dynamic cast
}
\end{xten}
compiles with \xcd`-STATIC_CHECKS`, even though it is possible that 
\xcd`x!=0` when \xcd`use(x)` is called.  In this case, the compiler inserts a
cast, which has the effect of checking that the call is correct before it
happens: 
\begin{xten}
def use(n:Int{self == 0}) {}
def test(x:Int) { 
   use(x as Int{self == 0}); 
}
\end{xten}
The compiler produces a warning that it inserted some dynamic casts.
If you then want to see what it did, use \xcd`-VERBOSE_CHECKS`.

You may also turn on static checking, with the \xcd`-STATIC_CHECKS` flag.  With
static checking, calls that cannot be proved correct statically will be
marked as errors.  





\subsection{Help: {\tt -help} and {\tt -- -help}}

These options cause the compiler to print a list of all command-line options.


\subsection{Source Path: {\tt -sourcepath {\em path}}}

This option tells the compiler where to look for X10 source code.  


\subsection{(Deprecated) Class Path: {\tt -classpath {\em path}}}

This option is accepted for backward compatibility, but ignored.

\subsection{Output Directory: {\tt -d {\em directory}}}

This option tells the compiler to produce its output files in the specified directory.

\subsection{Runtime {\tt -x10rt {\em impl}}}

This option tells which runtime implementation to use.  The choices are
\xcd`lapi`, \xcd`pgp`, \xcd`sockets`, \xcd`mpi`, and \xcd`standalone`.

\subsection{Executable File {\tt -o {\em path}}}

This option tells the compiler what path to use for the executable file. 

\section{Execution Options: Java}

The Java execution command \xcd`x10` has a number of options as well. 

\subsection{Class Path: {\tt -classpath {\em path}}}

This option specifies the search path for class files. 

\subsection{Library Path: {\tt -libpath {\em path}}}

This option specifies the search path for native libraries.

\subsection{Heap Size: {\tt -mx {\em size}}}

Sets the maximum size of the heap. 

\subsection{Help: {\tt -h}}

Prints a listing of all execution options.



%\subsection{{\tt }}


\chapter{Running X10}

An X10 application is launched either by a direct invocation of the generated
executable or using a launcher command. The specification of the number of
places and the mapping from places to hosts is transport specific and
discussed in \Sref{sect:RunningManaged} for Managed X10 (Java back end) and
\Sref{sect:RunningNative} for Native X10 (C++ back end). For distributed runs,
the x10 distribution (libraries) and the compiled application code (binary or
bytecode) are expected to be available at the same paths on all the nodes.  

Detailed, up-to-date documentation may be found at
\begin{xten}
http://xj.watson.ibm.com/twiki/bin/view/Main/LaunchingX10Applications
\end{xten}


\section{Managed X10}
\label{sect:RunningManaged}


Managed X10 applications are launched using the x10 script followed by the qualified name of the main class.

\begin{xten}
x10c HelloWholeWorld.x10
x10 HelloWholeWorld
\end{xten}

The main purpose of the x10 script is to set the jvm classpath and the
\xcd`java.library.path` system property to ensure the x10 libraries are on the
path.  


\section{Native X10}
\label{sect:RunningNative}

On most platforms and for most transports, X10 applications can be launched by invoking the generated executable.

\begin{xten}
x10c++ -o HelloWholeWorld HelloWholeWorld.x10
./HelloWholeWorld
\end{xten}

On cygwin, X10 applications must be launched using the runx10 script followed by the name of the generated executable.

\begin{xten}
x10c++ -o HelloWholeWorld HelloWholeWorld.x10
runx10 HelloWholeWorld
\end{xten}
The purpose of the runx10 script is to ensure the x10 libraries are on the path. 



Detailed, up-to-date documentation may be found at
\begin{xten}
http://xj.watson.ibm.com/twiki/bin/view/Main/X10NativeImplementation
\end{xten}


\chapter{Acknowledgments and Trademarks}

{\em The \Xten{} language has been developed as part of the IBM PERCS
Project, which is supported in part by the Defense Advanced Research
Projects Agency (DARPA) under contract No. NBCH30390004.}

{\em Java and all Java-based trademarks are trademarks of Sun Microsystems,
Inc. in the United States, other countries, or both.}
\end{document}


\appendix

\chapter{Deprecations}

X10 version 2.2 has a few relics of previous versions, code that is being used
by libraries but is not intended for general programming.    They should be
ignored.

These are: 

\begin{itemize}

\item \xcd`acc` variables. \index{acc}

\item The \xcd`offers` clause, as seen in the {\it Offers} nonterminal in the
      grammar (\ref{prod:Offers}).\index{offers}\index{Offers}

\item The grammar allows covariant and contravariant type parameters, marked
      by \xcd`+` and \xcd`-`: 
\begin{xtenmath}
class Variant[X, +Y, -Z] {}
\end{xtenmath}
      X10 does not support these in any other way.  

\end{itemize}

\chapter{Change Log}

\section{Changes from \Xten{} v2.3}
\Xten{} v2.4 is not backwards compatible with \Xten{} v2.3. The
motivation for making backwards incompatible language changes with
this release of \Xten{} is to significantly improve the ability of the
\Xten{} programmer to exploit the expanded memory capabilities of
modern computer systems.  In particular, \Xten{} v2.4 includes an
extensive redesign of arrays and a change of the default type of
unqualified integral literals (\eg \xcd`2`) from \xcd{Int} to
\xcd{Long}. Taken together these two changes enable natural
exploitation of large memories via 64-bit addressing and
\xcd{Long}-based indexing of arrays and similar data structures. 

\subsection{Integral Literals}

The default type of unqualified integral literals was change from
\xcd{Int} to \xcd{Long}. 

The qualifying suffix \xcd`n` and \xcd`un` are used to indicate
\xcd{Int} and \xcd{UInt} literals respectively.  The suffix xcd\`u` is
now interpreted as indicating a \xcd{ULong} literal.

\subsection{Arrays}

An extensive redesign of the \Xten{} abstractions for arrays is the
major new feature of the \Xten{} v2.4 release.  Although this redesign
only involved very minor changes to the actual \Xten{} language
specification, the core class libraries did change significantly.  As
mentioned above, the driving motivation for
the change was a long-contemplated strategic decision to shift to from
\xcd{Int}-based (32-bit) to \xcd{Long}-based (64-bit) indexing for all
\Xten{} arrays. This change enables \Xten{} to better utilize the rapidly
expanding memory capacity and 64-bit address space found on modern
machines. For consistency, the \xcd{id} field of \xcd{x10.lang.Place} and
the \xcd{size} and indexing-related APIs of the \xcd{x10.util}
collection hierarchy were also changed from \xcd{Int} to \xcd{Long}.

Once this inherently backwards-incompatible decision was made, the
\Xten{} team decided to do a larger rethinking of all of \Xten{}'s
array implementations to introduce a new time and space optimal
implementation of zero-based, dense, rectangular multi-dimensional
arrays.  This new implementation, in the \xcd{x10.array} package, is
intended to provide the best possible performance for the common-case
it supports.  The previous, more general array implementation is still
available, but has been relocated to a new package
\xcd{x10.array.regionarray}.  In addition, the \xcd{x10.lang.Rail}
class was re-introduced as a separate class in its own right and
provides the intrinsic indexed storage abstraction on which both array
packages are built.  The intent is that the combination of \xcd{Rail},
\xcd{x10.array} and \xcd{x10.regionarray} provide a spectrum of array
abstractions that capture common usage patterns and enable appropriate
trade-offs between performance and flexibility. 

In more detail the major array-related changes made in the \Xten{}
v2.4 release are 
\begin{enumerate}
\item The class \xcd{x10.lang.Rail} was introduced.  It provides
  an efficient one-dimensional, zero-based, densely indexed array
  implementation. \xcd{Rail} will provide the best performance and is
  the preferred implementation of this basic abstraction. 
\item The array literal syntax \xcd{[1,2,3]} is now defined to create
  a \xcd{Rail} instead of an \xcd{Array}. 
\item The main method signature is changed from \xcd{Array[String]} to
  \xcd{Rail[String]}. 
\item \xcd{x10.util.IndexedMemoryChunk} has removed from the \Xten{}
  standard library.
\item To enable usage of classes from both \xcd{x10.array} and
  \xcd{x10.regionarray}, the package \xcd{x10.array} is no longer
  auto-imported by the \Xten{} compiler.
\item Most classes in the \xcd{x10.array} package in the \Xten{} v2.3
  release were relocated to the \xcd{x10.regionarray} package in
  v2.4. A few classes like \xcd{Point} and \xcd{PlaceGroup} were moved
  to the \xcd{x10.lang} package instead.
\item \xcd{Point}, \xcd{Region}, \xcd{Dist}, etc. were all updated to
  support long-based indexing by consistently changing indexing
  related fields and methods from \xcd{Int} to \xcd{Long}. 
\end{enumerate}

\subsection{Other Changes from \Xten{} v2.3}

\begin{enumerate}
\item The custom serialization protocol was changed to operate in
  terms of new user-level classes \xcd{x10.io.Serializer} and
  \xcd{x10.io.Deserializer}.  The \xcd{serialize} method of
  the xcd{x10.io.CustomSerialization} interface now takes a
  \xcd{Serializer} as an argument.  The custom deserialization
  constructor for a class takes a \xcd{Deserializer}. The
  \xcd{x10.io.SerialData} class used by the \Xten{} v2.3 custom
  serialization protocol has been removed from the class library. 

\item A constraint was added to \xcd{PlaceLocalHandle} that types used
  to instantiate a \xcd{PlaceLocalHandle} must satisfy both the
  \xcd`isref` and \xcd`haszero` constraints.

\item The \xcd{x10.util.Team} API was revised by (a) removing the
  endpoint argument from all API calls and (b) to operate on \xcd`Rail` and
  xcd`Long` where appropriate.
\end{enumerate}

\section{Changes from \Xten{} v2.2}

\begin{enumerate}

\item In previous versions of \Xten{} static fields were
  eagerly initialized in \xcd{Place 0} and the resulting values were
  serialized to all other places before execution of the user main
  function was started. Starting with \Xten{} v2.2.3, static fields are
  lazily initialized on a per-Place basis when the field is first read
  by an activity executing in a given Place.

\item The new syntax \xcd{T isref} for some type \xcd{T} will hold if
  \xcd{T} is represented by a pointer at runtime.  This is similar to
  the type constraint \xcd{T haszero}.  \xcd{T isref} is true for
  \xcd{T} that are function types, classes, and all values that have
  been cast to interfaces (including boxed structs).  \xcd{T isref} is used
  in the standard library, e.g. for the \xcd{GlobalRef[T]} and
  \xcd{PlaceLocalHandle[T]} APIs.

\item \xcd{x10.lang.Object is gone}, there is now no single class that
  is the root of the X10 class hierarchy.
\begin{itemize}
  \item If, for some reason, you were explicitly extending \xcd{Object}, don't do
    that anymore.
  \item If you were doing \xcd{new Object()} to get a fresh value, use
    \xcd{new Empty()} instead.
  \item If you were using \xcd{Object} as a supertype, use \xcd{Any}
    (the one true supertype).  
  \item If you were using the type constraint \xcd{T <: Object} to
    disallow structs, use \xcd{T isref} instead.
\end{itemize}

\item The exception hierarchy has changed, and checked exceptions have
  been reintroduced.  The 'throws' annotation is required on methods,
  as in Java.  It is not supported on closures, so checked exceptions
  cannot be thrown from a closure.  The exception hierarchy has been
  chosen to exist in a 1:1 relationship with Java's.  However, unlike
  Java, we prefer using unchecked exceptions wherever possible, and
  this is reflected in the naming of the X10 classes.  The following
  classes are all in the \xcd{x10.lang} package.
  \begin{itemize}
  \item \xcd{CheckedThrowable} (mapped to \xcd{java.lang.Throwable}) 
  \item \xcd{CheckedException extends CheckedThrowable} (mapped to \xcd{java.lang.Exception}) 
  \item \xcd{Exception extends CheckedException}  (mapped to \xcd{java.lang.RuntimeException}) 
  \item \xcd{Error extends CheckedThrowable}  (mapped to \xcd{java.lang.Error}) 
  \end{itemize}

  Anything under \xcd{CheckedThrowable} can be thrown using the \xcd{throw}
  statement. But anything that is not under \xcd{Exception} or \xcd{Error} can
  only be thrown if it is caught by an enclosing \xcd{try/catch}, or it is
  thrown from a method with an appropriate throws annotation, as in
  Java.

  \xcd{RuntimeException} is gone from \Xten{}.  Use \xcd{Exception} instead.

  All the exceptions in the standard library are under \xcd{Exception},
  except \xcd{AssertionError} and \xcd{OutOfMemoryException}, which are under
  \xcd{Error} (as in Java).  This means all exceptions in the standard
  library remain unchecked.

\end{enumerate}

\section{Changes from \Xten{} v2.1}

\begin{enumerate}

\item Covariance and contravariance are gone.

\item Operator definitions are regularized.  A number of new operator symbols
      are available.

\item The operator \xcd`in` is gone.  \xcd`in` is now only a keyword.

\item Method functions and operator functions are gone.

\item \xcd`m..n` is now a type of struct called \xcd`IntRange`.  

\item \xcd`for(i in m..n)` now works.  The old forms, \xcd`for((i) in m..n)`
      and \xcd`for([i] in m..n)`, are no longer needed.

\item \xcd`(e as T)` now has type \xcd`T`.  (It used to have an identity
      constraint conjoined in.)

\item \xcd`var`s can no longer be assigned in their place of origin.  Use a
      \xcd`GlobalRef[Cell[T]]` instead.  We'll have a new idiom for this in 2.3.

\item The \xcd`-STATIC_CALLS` command-line flag is now \xcd`-STATIC_CHECKS`.

\item Any string may be written in backquotes to make an identifier: {\tt
      `while`}.

\item The \xcd`next` and \xcd`resume` keywords are gone; they have been
      replaced by static methods on \xcd`Clock`.

\item The typed array construction syntax \xcd`new Array[T][t1,t2]` is gone.
      Use \xcd`[t1 as T, t2]` (if just plain \xcd`[t1,t2]` doesn't work).

\end{enumerate}


\section{Changes from \Xten{} v2.0.6}

This document summarizes the main changes between X10 2.0.6 and X10 2.1.  The
descriptions are intended to be suggestive rather than definitive; see the
language specification for full details.

\subsection{Object Model}

\begin{enumerate}
\item Objects are now local rather than global.
   
    \begin{enumerate}
    \item The \Xcd{home} property is gone.
    \item \Xcd{at(P)S} produces deep copies of all objects reachable from
          lexically exposed variables in \xcd`S` 
          when it executes \Xcd{S}.  ({\bf Warning:} They are copied even in  
          \Xcd{at(here)S}.)
    \end{enumerate}

\item The \Xcd{GlobalRef[T]} struct is the only way to produce or manipulate
      cross-place references.
    \begin{enumerate}
    \item \Xcd{GlobalRef}'s have a \Xcd{home} property.
    \item Use \Xcd{GlobalRef[Foo](foo)} to make a new global reference.
    \item Use \Xcd{myGlobalRef()} to access the object referenced; this
          requires \Xcd{here == myGlobalRef.home}. 
    \end{enumerate}


\item  The \xcd`!` type modifier is no longer needed or present.

\item \Xcd{global} modifiers are now gone:
    
    \begin{enumerate}
    \item \Xcd{global} methods in {\em interfaces} are now the default. 
    \item \Xcd{global} {\em fields} are gone.  In some cases object copying
          will produce the same effect as global fields.  In other cases code
          must be rewritten.  It may be desirable to mark nonglobal fields
          \Xcd{transient} in many cases.
    \item \Xcd{global} {\em methods} are now marked \Xcd{@Global} instead.  
          Methods intended to be non-global may be marked \Xcd{@Pinned}.
    \end{enumerate}


\end{enumerate}

\subsection{Constructors}


\begin{enumerate}
\item \Xcd{proto} types are gone.
\item Constructors and the methods they call must satisfy a number of static
      checks.  
    
    \begin{enumerate}
    \item Constructors can only invoke \Xcd{private} or \Xcd{final} methods, 
          or methods annotated \xcd`@NonEscaping`.  
    \item Methods invoked by constructors cannot read fields before they are
          written. 
    \item The compiler ensures this with a detailed protocol. 
    \end{enumerate}

\item It is still impossible for X10 constructors to leak references to
      \Xcd{this} or observe uninitialized fields of an object.  Now, however,
      the mechanisms enforcing this are less obtrusive than in 2.0.6; the
      burden is largely on the compiler, not the programmer.
\end{enumerate}




%REF> \subsection{Call by Reference}
%REF> 
%REF> A very limited form of call-by-reference is now available.
%REF> 
%REF> 
%REF> \begin{enumerate}
%REF> 
%REF> \item Formal parameters to functions and methods may be \Xcd{ref} rather than
%REF>       \Xcd{var} or \Xcd{val}.  
%REF> \item Assignment to a \Xcd{ref} parameter \Xcd{x} changes the original
%REF>       location that the \Xcd{ref} refers to.  \eg, 
%REF>       \xcd`def inc(ref x:Int) { x ++; }`
%REF>       allows a call \Xcd{inc(n)} to increment a local \Xcd{var} \Xcd{n}.
%REF> \item Only local variables or \Xcd{ref} parameters can be passed as actual
%REF>       \Xcd{ref} parameters.  Fields, array elements, and other variable-like
%REF>       items cannot be. 
%REF> \item External \Xcd{ref} variables cannot be captured in closures. However,
%REF>       closures may have \Xcd{ref} parameters.
%REF> \item \Xcd{ref}s are {\em not} first-class objects in X10. They cannot be
%REF>       returned from functions, stored in data structures, etc.
%REF> \item These restrictions limit the possibilities of aliasing and the need for
%REF>       boxing of \Xcd{ref} parameters.  \Xcd{ref}s to stack locations cannot,
%REF>       with these restrictions, live past the death of the location's
%REF>       containing stack frame.      
%REF> \item This allows the implementation of many core constructs as syntactic
%REF>       sugar on library calls.   Programmers may use it, but mutability should
%REF>       generally be encapsulated inside objects rather than \Xcd{ref}
%REF>       parameters. 
%REF> \end{enumerate}
%REF> 

%ACC> \subsection{Accumulator Variables}
%ACC> 
%ACC> Accumulator variables generalize and make explicit collecting \Xcd{finish} in
%ACC> X10 2.0.6.  An \Xcd{acc} variable is declared: 
%ACC> \begin{xten}
%ACC> acc(r) A;
%ACC> \end{xten}
%ACC> where \Xcd{r} is a {\em reducer} (much as in 2.0.6): 
%ACC> \begin{xten}
%ACC> struct Reducer[T](zero:T, apply:global (T,T)=>T){}
%ACC> \end{xten}
%ACC> 
%ACC> Usage of \Xcd{A} is restricted in ways that make it determinate in the
%ACC> intended case of a pure, associative, commutative \Xcd{apply} with unit
%ACC> element \Xcd{zero}.  
%ACC> 
%ACC> \begin{enumerate}
%ACC> \item \Xcd{A} is initialized to \Xcd{r.zero}.  
%ACC> \item Multiple activities can {\em write} into \Xcd{A}.  In particular, the
%ACC>       ``assignment'' \Xcd{A = v} is approximately interpreted as 
%ACC>       \xcd`atomic{A = r.apply(A, v)}` --- that is, it accumulates \Xcd{v} into
%ACC>       \Xcd{A} using \Xcd{r.apply.}
%ACC> \item {\em Reading} of \Xcd{A} is restricted to situations where it makes
%ACC>       sense.  Specifically, only the activity in which \Xcd{A} is declared can
%ACC>       read from it, and it can only do so when all asyncs which it has spawned
%ACC>       have terminated -- \eg, outside of the scope of all \Xcd{async}s and
%ACC>       \Xcd{finish}es.  
%ACC> \item Formal parameters of functions may be marked \Xcd{acc x:T}.  The reducer
%ACC>       \Xcd{r} must not be specified; it is passed as an implicit parameter
%ACC>       going with the actual \Xcd{acc} variable.  
%ACC> \item X10 provides protocols for indexed collections of \Xcd{acc} variables,
%ACC>       presented as objects.
%ACC> \end{enumerate}
%ACC> 


\subsection{Implicit clocks for each finish}


Most clock operations can be accomplished using the new implicit clocks.

\begin{enumerate}
\item A \Xcd{finish} may be qualified with \Xcd{clocked}, which gives it a
      clock.
\item An \Xcd{async} in a \Xcd{clocked finish} may be marked \Xcd{clocked}.
      This registers it on the same clock as the enclosing \Xcd{finish}.  
\item \xcd`clocked async S` and \xcd`clocked finish S` may use \xcd`next` in
      the body of \Xcd{S} to advance the clock.
\item When the body of a \Xcd{clocked finish} completes, the \Xcd{clocked
      finish} is dropped form the clock.  It will still wait for spawned
      asyncs to terminate, but such asyncs need to wait for it.
\end{enumerate}


%CLOCAL>\subsection{Clocked local variables}
%CLOCAL>
%CLOCAL>Local \Xcd{val} and \Xcd{acc} variables may be \Xcd{clocked}.  They are
%CLOCAL>associated with the clock of the surrounding \Xcd{clocked finish}.  
%CLOCAL>Clocked variables have a {\em current} value and an {\em upcoming} value.  The
%CLOCAL>current value may be read at suitable times; the upcoming value may be
%CLOCAL>updated.  The \Xcd{next} phase makes the upcoming value current.

\subsection{Asynchronous initialization of val}

\Xcd{val}s can be initialized asynchronously.   As always with \Xcd{val}s,
they can only be read after it is guaranteed that they have been initialized.
For example, both of the \Xcd{print}s below are good.  However, the
commented-out \Xcd{print} in the \Xcd{async} is bad, since it is possible that
it will be executed before the initialization of \Xcd{a}. 
\begin{xten}
val a:Int;
finish {
  async {
     a = 1; 
     print("a=" + a);
  }
  // WRONG: print("a=" + a);
}
print("a=" + a);
\end{xten}



\subsection{Main Method}

The signature for the \Xcd{main} method is now: 
\begin{xten}
           def main(Array[String]) {..}
\end{xten}
or, if the arguments are actually used, 
\begin{xten}
           def main(argv: Array[String](1)) {..}
\end{xten}

\subsection{Assorted Changes}


\begin{enumerate}
\item The syntax for destructuring a point now uses brackets rather than
      braces: \Xcd{for( [i] in 1..10 )}, rather than the prior \Xcd{(i)}.  
\end{enumerate}

\subsection{Safety of atomic and when blocks}


\begin{enumerate}
\item Static effect annotations (\Xcd{safe}, \Xcd{sequential},
      \Xcd{nonblocking}, \Xcd{pinned}) are no longer used. They have been
      replaced by dynamic checks.
\item Using an inappropriate operation in the scope of an \Xcd{atomic} or
      \Xcd{when} construct will throw \Xcd{IllegalOperationException}.  
      The following are inappropriate:      
      \begin{itemize}
      \item \Xcd{when}
      \item \Xcd{resume()} or \Xcd{next} on clocks
      \item async
      \item \Xcd{Future.make()}, or \Xcd{Future.force()}.
      \item \Xcd{at}
      \end{itemize}

\end{enumerate}


\subsection{Removed Topics}

The following are gone: 

\begin{enumerate}
\item \Xcd{foreach} is gone.
\item All \Xcd{var}s are effectively \Xcd{shared}, so \Xcd{shared} is gone.
\item The place clause on \Xcd{async} is gone.  \Xcd{async (P) S} should be
      written \Xcd{at(P) async S}.
\item Checked exceptions are gone.
\item \Xcd{future} is gone.
\item \Xcd{await ... or ... } is gone.
\item \Xcd{const} is gone.

\end{enumerate}

\subsection{Deprecated}

The following constructs are still available, but are likely to be replaced in
a future version: 


\begin{enumerate}
\item \Xcd{ValRail}.
\item \Xcd{Rail}.
\item \xcd`ateach`
\item \xcd`offers`.  \index{offers}  The \xcd`offers` concept was experimental
      in 2.1, but was determined inadequate.  It has not been removed from the
      compiler yet, but it will be soon.  In the meantime, traces of it are
      still visible in the grammar.  They should not be used and can safely be ignored.
\end{enumerate}

\section{Changes from \Xten{} v2.0}
Some of these changes have been made obsolete in X10 2.2.

\begin{itemize}
\item \Xcd{Any} is now the top of the type hierarchy (every object,
  struct and function has a type that is a subtype of
  \Xcd{Any}). \Xcd{Any} defines \Xcd{home}, \Xcd{at}, \Xcd{toString},
  \Xcd{typeName}, \Xcd{equals} and \Xcd{hashCode}. \Xcd{Any} also defines the methods
  of \Xcd{Equals}, so \Xcd{Equals} is not needed any more.
\item Revised discussion of incomplete types.
\item The manual has been revised and brought into line with the current implementation. 
\end{itemize}
\section{Changes from \Xten{} v1.7}

The language has changed in the following ways.  
Some of these changes have been made obsolete in X10 2.2.

\begin{itemize}

\item {\bf Type system changes}: There are now three kinds of entities
  in an \Xten{} computation: objects, structs and functions. Their
  associated types are class types, struct types and function
  types. 

  Class and struct types are called {\em container types} in that they
  specify a collection of fields and methods. Container types have a
  name and a signature (the collection of members accessible on that
  type). Collection types support primitive equality \Xcd{==} and may
  support user-defined equality if they implement the {\tt
    x10.lang.Equals} interface. 

  Container types (and interface types) may be further qualified with
  constraints.

  A function type specifies a set of arguments and their type, the
  result type, and (optionally) a guard. A function application
  type-checks if the arguments are of the given type and the guard is
  satisfied, and the return value is of the given type.  A function
  type does not permit \Xcd{==} checks. Closure literals create
  instances of the corresponding function type.

  Container types may implement interfaces and zero or more function
  types.

  All types support a basic set of operations that return a string
  representation, a type name, and specify the home place of the entity.

  The type system is not unitary. However, any type may be used to
  instantiate a generic type. 

  There is no longer any notion of \Xcd{value} classes. \Xcd{value}
  classes must be re-written into structs or (reference) classes. 

\item {\bf Global object model}: Objects are instances of
  classes. Each object is associated with a globally unique
  identifier. Two objects are considered identical \Xcd{==} if their
  ids are identical. Classes may specify \Xcd{global} fields and
  methods. These can be accessed at any place. (\Xcd{global} fields
  must be immutable.)

\item{\bf Proto types.} For the decidability of dependent type
  checking it is necessary that the property graph is acyclic. This is
  ensured by enforcing rules on the leakage of \Xcd{this} in
  constructors. The rules are flexible enough to permit cycles to be
  created with normal fields, but not with properties.

\item{Place types.} Place types are now implemented. This means that
  non-global methods can be invoked on a variable, only if the
  variable's type is either a struct type or a function type, or a
  class type whose constraint specifies that the object is located in
  the current place.

  There is still no support for statically checking array access
  bounds, or performing place checks on array accesses.

\end{itemize}

\chapter{Options}

\subsection{Compiler Options}

The X10 compilers have many useful options. 

% -CHECK_INVARIANTS seems to check some internal compiler invariants.

\subsection{Optimization: {\tt -O} or {\tt -optimize}}

This flag causes the compiler to generate optimized code.


\subsection{Debugging: {\tt -DEBUG=boolean}}

This flag, if true, causes the compiler to generate debugging information.  It
is false by default.

\subsection{Call Style: {\tt -STATIC\_CHECKS, -VERBOSE\_CHECKS}}
\label{sect:Callstyle}
\index{STATIC\_CHECKS}
\index{VERBOSE\_CHECKS}
\index{dynamic checks}

By default, if a method call {\em could} be correct but is not {\em
necessarily} correct, the X10 compiler generates a dynamic check to ensure
that it is correct before it is performed.  For example, the following code: 
\begin{xten}
def use(n:Int{self == 0}) {}
def test(x:Int) { 
   use(x); // creates a dynamic cast
}
\end{xten}
compiles with \xcd`-STATIC_CHECKS`, even though it is possible that 
\xcd`x!=0` when \xcd`use(x)` is called.  In this case, the compiler inserts a
cast, which has the effect of checking that the call is correct before it
happens: 
\begin{xten}
def use(n:Int{self == 0}) {}
def test(x:Int) { 
   use(x as Int{self == 0}); 
}
\end{xten}
The compiler produces a warning that it inserted some dynamic casts.
If you then want to see what it did, use \xcd`-VERBOSE_CHECKS`.

You may also turn on static checking, with the \xcd`-STATIC_CHECKS` flag.  With
static checking, calls that cannot be proved correct statically will be
marked as errors.  





\subsection{Help: {\tt -help} and {\tt -- -help}}

These options cause the compiler to print a list of all command-line options.


\subsection{Source Path: {\tt -sourcepath {\em path}}}

This option tells the compiler where to look for X10 source code.  


\subsection{(Deprecated) Class Path: {\tt -classpath {\em path}}}

This option is accepted for backward compatibility, but ignored.

\subsection{Output Directory: {\tt -d {\em directory}}}

This option tells the compiler to produce its output files in the specified directory.

\subsection{Runtime {\tt -x10rt {\em impl}}}

This option tells which runtime implementation to use.  The choices are
\xcd`lapi`, \xcd`pgp`, \xcd`sockets`, \xcd`mpi`, and \xcd`standalone`.

\subsection{Executable File {\tt -o {\em path}}}

This option tells the compiler what path to use for the executable file. 

\section{Execution Options: Java}

The Java execution command \xcd`x10` has a number of options as well. 

\subsection{Class Path: {\tt -classpath {\em path}}}

This option specifies the search path for class files. 

\subsection{Library Path: {\tt -libpath {\em path}}}

This option specifies the search path for native libraries.

\subsection{Heap Size: {\tt -mx {\em size}}}

Sets the maximum size of the heap. 

\subsection{Help: {\tt -h}}

Prints a listing of all execution options.



%\subsection{{\tt }}


\chapter{Running X10}

An X10 application is launched either by a direct invocation of the generated
executable or using a launcher command. The specification of the number of
places and the mapping from places to hosts is transport specific and
discussed in \Sref{sect:RunningManaged} for Managed X10 (Java back end) and
\Sref{sect:RunningNative} for Native X10 (C++ back end). For distributed runs,
the x10 distribution (libraries) and the compiled application code (binary or
bytecode) are expected to be available at the same paths on all the nodes.  

Detailed, up-to-date documentation may be found at
\begin{xten}
http://xj.watson.ibm.com/twiki/bin/view/Main/LaunchingX10Applications
\end{xten}


\section{Managed X10}
\label{sect:RunningManaged}


Managed X10 applications are launched using the x10 script followed by the qualified name of the main class.

\begin{xten}
x10c HelloWholeWorld.x10
x10 HelloWholeWorld
\end{xten}

The main purpose of the x10 script is to set the jvm classpath and the
\xcd`java.library.path` system property to ensure the x10 libraries are on the
path.  


\section{Native X10}
\label{sect:RunningNative}

On most platforms and for most transports, X10 applications can be launched by invoking the generated executable.

\begin{xten}
x10c++ -o HelloWholeWorld HelloWholeWorld.x10
./HelloWholeWorld
\end{xten}

On cygwin, X10 applications must be launched using the runx10 script followed by the name of the generated executable.

\begin{xten}
x10c++ -o HelloWholeWorld HelloWholeWorld.x10
runx10 HelloWholeWorld
\end{xten}
The purpose of the runx10 script is to ensure the x10 libraries are on the path. 



Detailed, up-to-date documentation may be found at
\begin{xten}
http://xj.watson.ibm.com/twiki/bin/view/Main/X10NativeImplementation
\end{xten}


\chapter{Acknowledgments and Trademarks}

{\em The \Xten{} language has been developed as part of the IBM PERCS
Project, which is supported in part by the Defense Advanced Research
Projects Agency (DARPA) under contract No. NBCH30390004.}

{\em Java and all Java-based trademarks are trademarks of Sun Microsystems,
Inc. in the United States, other countries, or both.}
\end{document}


\appendix

\chapter{Change Log}

\section{Changes from \Xten{} v2.0}

\begin{itemize}
\item \Xcd{Any} is now the top of the type hierarchy (every object,
  struct and function has a type that is a subtype of
  \Xcd{Any}). \Xcd{Any} defines \Xcd{home}, \Xcd{at}, \Xcd{toString},
  \Xcd{typeName}, \Xcd{equals} and \Xcd{hashCode}. \Xcd{Any} also defines the methods
  of \Xcd{Equals}, so \Xcd{Equals} is not needed any more.
\item Revised discussion of incomplete types (\Sref{ProtoRules}).
\item The manual has been revised and brought into line with the current implementation. 
\end{itemize}
\section{Changes from \Xten{} v1.7}

The language has changed in the following way:
\begin{itemize}

\item {\bf Type system changes}: There are now three kinds of entities
  in an \Xten{} computation: objects, structs and functions. Their
  associated types are class types, struct types and function
  types. 

  Class and struct types are called {\em container types} in that they
  specify a collection of fields and methods. Container types have a
  name and a signature (the collection of members accessible on that
  type). Collection types support primitive equality \Xcd{==} and may
  support user-defined equality if they implement the {\tt
    x10.lang.Equals} interface. 

  Container types (and interface types) may be further qualified with
  constraints.

  A function type specifies a set of arguments and their type, the
  result type, and (optionally) a guard. A function application
  type-checks if the arguments are of the given type and the guard is
  satisfied, and the return value is of the given type.  A function
  type does not permit \Xcd{==} checks. Closure literals create
  instances of the corresponding function type.

  Container types may implement interfaces and zero or more function
  types.

  All types support a basic set of operations that return a string
  representation, a type name, and specify the home place of the entity.

  The type system is not unitary. However, any type may be used to
  instantiate a generic type. 

  There is no longer any notion of \Xcd{value} classes. \Xcd{value}
  classes must be re-written into structs or (reference) classes. 

\item {\bf Global object model}: Objects are instances of
  classes. Each object is associated with a globally unique
  identifier. Two objects are considered identical \Xcd{==} if their
  ids are identical. Classes may specify \Xcd{global} fields and
  methods. These can be accessed at any place. (\Xcd{global} fields
  must be immutable.)

\item{\bf Proto types.} For the decidability of dependent type
  checking it is necessary that the property graph is acyclic. This is
  ensured by enforcing rules on the leakage of \Xcd{this} in
  constructors. The rules are flexible enough to permit cycles to be
  created with normal fields, but not with properties.

\item{Place types.} Place types are now implemented. This means that
  non-global methods can be invoked on a variable, only if the
  variable's type is either a struct type or a function type, or a
  class type whose constraint specifies that the object is located in
  the current place.

  There is still no support for statically checking array access
  bounds, or performing place checks on array accesses.

\end{itemize}
{\em The \Xten{} language has been developed as part of the IBM PERCS
Project, which is supported in part by the Defense Advanced Research
Projects Agency (DARPA) under contract No. NBCH30390004.}

{\em Java and all Java-based trademarks are trademarks of Sun Microsystems,
Inc. in the United States, other countries, or both.}
\end{document}

