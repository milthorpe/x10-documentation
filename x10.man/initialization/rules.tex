X10 is an advanced object-oriented language with a complex type-system
    and concurrency constructs.
This section describes how object initialization interacts with X10 features.
We begin with object-oriented features found in mainstream languages,
    such as constructors, inheritance, dynamic dispatching, exceptions, inner classes,
    and static fields.
We then proceed to X10 type-system features,
    such as constraints, properties, class invariants, closures, (non-erased) generics, and structs.
The parallel features of X10 allow writing concurrent code (using \code{finish} and \code{async}),
    and distributed code (using \code{at} and global references).
Finally, we conclude with an inter-procedural data-flow analysis that ensures that
    a field is read only after it was written.


\subsection{Constructors and inheritance}
Inheritance is the first feature that interacts with initialization:
    when class \code{B} inherits from \code{A}
    then every instance of \code{B} has a sub-object that is like an instance of \code{A}.
When we initialize an instance of \code{B}, we must first initialize its \code{A} sub-object.
We do this in X10 by forcing the constructors of \code{B} to make a super call,
    i.e., call a constructor of \code{A}
    (either explicitly or implicitly).



\begin{figure}
\begin{lstlisting}
class A {
  val a:Int;
  def this() {
    LeakIt.foo(this); //err
    this.a = 1;
    val me = this; //err
    LeakIt.foo(me);
    this.m2(); // so m2 is implicitly non-escaping
  }
  final def m1() {
    LeakIt.foo(this);
  }
  // implicitly non-escaping
  final def m2() {
    LeakIt.foo(this); //err
  }
  // explicitly non-escaping
  @NonEscaping final def m3() {
    LeakIt.foo(this); //err
  }
}
class B extends A {
  val b:Int;
  def this() {
    super();
    this.b = 2;
    super.m3();
  }
}
\end{lstlisting}
\definerule{Rule3}
\definerule{Rule4}
\caption{Escaping \this example.
    \textbf{Definition of \emph{raw}:}
    {\this and \code{super} are \emph{raw} in {non-escaping} methods and in field initializers}.
    \textbf{Definition of \emph{non-escaping}:}
        {A method is \emph{non-escaping} if it is a constructor,
            or annotated with \code{@NonEscaping} or \code{@NoThisAccess},
            or a method that is called on a raw \this receiver}.
    \myrule{\arabic{Rule3}}{A raw \this or \code{super} cannot escape or be aliased}
    \myrule{\arabic{Rule4}}{A call on a raw \code{super} is allowed only for a \code{@NonEscaping} method}
    (\code{\textbf{final}} and \code{@NoThisAccess} are related
        to dynamic-dispatching as shown in \Ref{Figure}{Dynamic-dispatch}.)}
\label{Figure:Escaping-this}
\end{figure}

\Ref{Figure}{Escaping-this} shows X10 code that demonstrates the interaction
    between inheritance and initialization,
    and explains by example why leaking \this during construction can cause bugs.
In all the examples, errors issued by the X10 compiler are marked with \code{//err}.

We say that an object is \emph{raw} before its constructor ends,
    and afterward it is \emph{cooked}.
Note that when an object is cooked, all its sub-objects must be cooked as well.
X10 prohibits any aliasing or leaking of \this during construction,
    therefore only \this or \code{super} can be raw (any other variable is definitely cooked).

Object initialization begins by invoking a constructor,
    denoted by the method definition \code{def this()}.
The first leak would cause a problem because field \code{a} was not assigned yet.
However, even after all the fields of \code{A} have been assigned,
    leaking is still a problem
    because fields in a subclass (field \code{b}) have not yet been initialized.
Note that leaking is not a problem if \this is not raw, e.g., in \code{m1()}.

We begin with two definitions:
    (i)~when an object is \emph{raw}, and
    (ii)~when a method is \emph{non-escaping}.
(i)~Variables \this and \code{super} are raw
    during the object's construction,
    i.e., in field initializers and in \emph{non-escaping} methods
    (methods that cannot escape or leak \this).
(ii)~Obviously constructors are non-escaping,
    but you can also annotate methods explicitly as \code{@NonEscaping},
    or they can be inferred to be implicitly non-escaping
    if they are called on a raw \this receiver.

For example, \code{m2} is \emph{implicitly} non-escaping (and therefore cannot leak \this)
    because of the call to \code{m2}
    in the constructor.
The user could also mark \code{m2} \emph{explicitly} as non-escaping by using the annotation
    \code{@NonEscaping}.
(Like in Java, \code{@} is used for annotations in X10.)
We recommend to explicitly mark public methods as \code{@NonEscaping} to show intent,
    as done on method \code{m3}.
Without this annotation the call \code{super.m3()} in \code{B} would be illegal,
    due to rule~\userule{Rule4}.
(We could infer that \code{m3} must be non-escaping,
    but that would cause a dependency from a subclass to a superclass,
    which is not natural for people used to separate compilation.)
Finally, we note that all errors in this example are due to rule~\userule{Rule3}
    that prevents leaking a raw \this or \code{super}.




\subsection{Dynamic dispatch}
Dynamic dispatching interacts with initialization by transferring control to the subclass
    before the superclass completed its initialization.
\Ref{Figure}{Dynamic-dispatch} demonstrates why dynamic dispatching is error-prone during construction:
    calling \code{m1} in \code{A} would dynamically dispatch and
    call the implementation in \code{B}
    that would read the default value.



\begin{figure}
\begin{lstlisting}
abstract class A {
  val a:Int;
  def this() {
    this.a = m1(); //err1
    this.a = m2();
  }
  abstract def m1():Int;
  @NoThisAccess abstract def m2():Int;
}
class B extends A {
  var b:Int = 3;
  def this(i:Int) {
    super(i);
  }
  def m1() {
    val x = super.a; // returns 0 in Java
    val y = this.b; // returns 0 in Java
    return 1;
  }
  @NoThisAccess def m2() {
    val x = super.a; //err2
    val y = this.b; //err3
    return 2;
  }
}
\end{lstlisting}
\definerule{Rule5}
\definerule{Rule6}
\caption{Dynamic dispatching example.
    \myrule{\arabic{Rule5}}{A non-escaping method must be private or final, unless it has \code{@NoThisAccess}}
    \myrule{\arabic{Rule6}}{A method with \code{@NoThisAccess} cannot access \this or \code{super} (neither read nor write its fields)}
    }
\label{Figure:Dynamic-dispatch}
\end{figure}


X10 prevents dynamic dispatching by requiring that non-escaping methods
    must be private or final
    (so overriding is impossible).
For example, \code{err1} is caused by rule~\userule{Rule5}
    because \code{m1} is neither private nor final nor \code{@NoThisAccess}.

However, sometimes dynamic dispatching is required during construction.
For example, if a subclass needs to refine initialization
    of the superclass's fields.
Such refinement cannot have any access to \this, and therefore
    such methods are marked with \code{@NoThisAccess}.
For example, \code{err2} and \code{err3} are caused by rule~\userule{Rule6} that prohibits access \this or \code{super}
    when using \code{@NoThisAccess}.
\code{@NoThisAccess} prohibits any access to \this,
    however, one could still access the method parameters.


In C++, the call to \code{m1} is legal,
    but at runtime
    methods are statically bound,
    so you will get an error for calling a pure virtual function.
In Java, the call to \code{m1} is also legal,
    but at runtime
    methods are dynamically bound,
    so the implementation of \code{m1} in \code{B}
    will read the default value of \code{b}.
%This behavior is undesired in Java,
%    and Java discourages it by trying to catch statically most of these cases.
%For example, Java prohibits calls to member functions before the super object was initialized,
%    as this example shows (which is also illegal in X10):
%\begin{lstlisting}
%class B extends A { B() {super(f()); }}
%\end{lstlisting}

Finally, classes \code{C} and \code{D} show why it is sometimes required to mark
    methods explicitly as \code{@NonEscaping}:
    if a subclass (\code{D}) needs to call a method a method (\code{m3})
    during construction, then it must be marked as non-escaping.




\subsection{Exceptions}
Constructing an object may not always end normally,
    e.g., building a date object from an illegal date string should throw an exception.
Exceptions combined with inheritance interact with initialization in the following way:
    a cooked object must have cooked sub-objects,
    therefore if a constructor ends normally (thus returning a cooked object)
    then all preceding constructor calls (either \code{super(\ldots)} or \code{this(\ldots)})
    must end normally as well.
Phrased differently, in a constructor it should not be possible to
    recover from an exception thrown by a constructor call.
This is one of the reason why a constructor call must be the first statement in Java;
    failure to verify this led to a famous security attack~\cite{Dean:1996}.

\begin{figure}
\begin{lstlisting}
class B extends A {
  def this() {
    try { super(); } catch(Throwable e){} //err
  }
}
\end{lstlisting}
\definerule{Rule7}
\caption{Exceptions example:
    if a constructor ends normally (without throwing an exception),
        then all preceding constructor calls ended normally as well.
    \myrule{\arabic{Rule7}}{If a constructor does not call \code{super(\ldots)} or \code{this(\ldots)},
        then an implicit \code{super()} is added at the beginning of the constructor;
        the first statement in a constructor is a constructor call (either \code{super(\ldots)} or \code{this(\ldots)});
        a constructor call may only appear as the first statement in a constructor
        }
    }
\label{Figure:Exceptions}
\end{figure}


\Ref{Figure}{Exceptions} shows that it is an error to try to recover from an exception thrown
    by a constructor call, because from rule~\userule{Rule7} the first statement must be \code{super()}.


\Subsection[Inner]{Inner classes}
Inner classes usually read the outer instance's fields during construction,
    e.g., a list iterator would read the list's header node.
Therefore, X10 requires that the outer instance is cooked,
    and prohibits creating an inner instance when the receiver is a raw \this.


\begin{figure}
\begin{lstlisting}
class Outer {
  val a:Int;
  def this() {
    // Outer.this is raw
    Outer.this. new Inner(); //err
    new Nested(); // ok
    a = 3;
  }
  class Inner {
    def this() {
      // Inner.this is raw, but
      // Outer.this is cooked
      val x = Outer.this.a;
    }
  }
  static class Nested {}
}
\end{lstlisting}
\definerule{Rule8}
\caption{Inner class example: the outer instance is always cooked.
    \myrule{\arabic{Rule8}}{a raw \this cannot be the receiver of \code{new}}
    }
\label{Figure:InnerClass}
\end{figure}

\Ref{Figure}{InnerClass} shows it is an error in X10 to create an inner instance
    if the outer is raw (from rule~\userule{Rule8}),
    but it is ok to create an instance of a static nested class,
    because it has no outer instance.

In fact, it is possible to view this rule as a special case to the rule that
    prohibits leaking a raw \this
    (because when you create an inner instance on a raw \this receiver,
    you created an alias,
    and now you have two raw objects: \code{Inner.this} and \code{Outer.this}).
We wish to keep the invariant that only \this might be raw.


assume a transformation that makes them static by having additional arguments for all outer instances (and making them properties of the static nested class).
Because X10 requires that the outer instances are cooked, then this transformation maintains the legality of the code.

Same for anonymous and local classes. \todo ...


\subsection{Static initialization}
X10 does not support dynamic class loading as opposed to Java,
    and all static fields in X10 are final.
Thus, initialization of static fields is a one-time phase, denoted the static-init phase,
    that is done before the \code{main} method is executed.

During the static-init phase we must finish writing to all static fields,
    and reading a static field \emph{waits} until the field is initialized
    (i.e., the current activity/thread blocks if the field was not written to,
    and it resumes after another activity writes to it).
Obviously, this may lead to deadlock as demonstrated by \Ref{Figure}{Static-init}.
However, in practice, deadlock is rare,
    and usually found quickly the first time a program is executed.

\begin{figure}
\begin{lstlisting}
class A {
  static val a:Int = B.b;
}
class B {
  static val b:Int = A.a;
}
\end{lstlisting}
\caption{Static initialization example:
    the program will deadlock at run-time
    during the static-init phase (before the \code{main} method is executed).
    }
\label{Figure:Static-init}
\end{figure}





We now turn our attention to X10's sophisticated type-system
    that has features that are not found in main-stream languages:
    constraints, properties, class invariants, closures, (non-erased) generics, and structs.

\subsection{Constraints and default/zero values}
Constraints and default values.
The following types do not have a default value:
\code{Int\lb self!=0\rb}
\code{String\lb self!=null\rb}

Therefore the fields of an object cannot be zero-initialized in X10.

\Ref{Figure}{Constraints} todo ...

\begin{figure}
\begin{lstlisting}
class A {
  var a:Int{self!=0}; //err
}
\end{lstlisting}
\definerule{Rule9}
\definerule{Rule10}
\caption{No default value example.
    \myrule{\arabic{Rule9}}{A type \emph{has-zero} if a it contains the zero value
        (which is either \code{null}, \code{false}, 0, or
            zero in all fields for user-defined structs, see \Ref{Section}{Generics-and-Structs})}
    \myrule{\arabic{Rule10}}{A \code{var} field whose type has-zero that lacks a field initializer,
        is implicitly added an zero initializer}
    }
\label{Figure:Constraints}
\end{figure}


\subsection{Properties and the class invariant}
Properties are final values that can be used in constraints,
    e.g., \code{Array} has a \code{size} property,
    so an array of size 2 can be expressed as: \code{Array\lb self.size==2\rb}.

Properties are final fields that are initialized before all other fields.
% Should I talk about interface and abstract property methods? Doesn't relate to initialization...
The \emph{class invariant} may refer only to properties of the class,
    and it must be satisfied after the property call in every constructor.


\Ref{Figure}{Properties} todo ...

\begin{figure}
\begin{lstlisting}
class A(a:Int) {
  def this() {
    property(42);
  }
}
class B(b:Int) {b!=0} extends A {
  val f1 = a+b;
  val f2:Int;
  def this() {
    super();
    f2 = f1; //err
    property(3);
  }
}
\end{lstlisting}
\definerule{Rule16}
\definerule{Rule17}
\definerule{Rule18}
\definerule{Rule19}
\caption{Properties and class invariant example:
        properties (e.g., \code{a} and \code{b})
        are final fields that are initialized before all other fields
        using a property call (\code{property(\ldots);} statement).
    \myrule{\arabic{Rule16}}{If a class does not define any properties, then
        an implicit \code{property()} is added
        after (the implicit or explicit) \code{super(\ldots)}}
    \myrule{\arabic{Rule17}}{If a constructor does not call \code{this(\ldots)},
        then it must have exactly one
        property call, and it must be unconditionally executed
        (unless the constructor throws an exception);
        the class invariant must hold after the property call}
    \myrule{\arabic{Rule18}}{Field initializers are executed in their declaration order
        after (the implicit or explicit) the property call}
    \myrule{\arabic{Rule19}}{The super fields can only be accessed after \code{super(\ldots)},
        and the fields of \this can only be accessed after \code{property(\ldots)}}
    }
\label{Figure:Properties}
\end{figure}




\subsection{Closures}



\Ref{Figure}{Closures} todo ...

\begin{figure}
\begin{lstlisting}
class A {
  val a = 3;
  def this() {
    val local = this.a;
    val closure1 = ()=>local;
    val closure2 = ()=>this.a; //err
    at (here.next())
      closure2();
  }
}
\end{lstlisting}
\definerule{Rule20}
\caption{Closure capture \this example.
    \myrule{\arabic{Rule20}}{A closure cannot capture a raw \this}
    }
\label{Figure:Closures}
\end{figure}



\subsection{Generics and Structs}
\label{Section:Generics-and-Structs}
Structs in X10 ...
Does not contain \code{null}, so \code{haszero} ...

Generics are not unique to X10,
    however the combination of generics and the lack of default values for all types
    lead to new pitfalls.
In addition, generics in X10 are \emph{not} erased as in Java
    in order to make instantiations over structs efficient.

for example, \code{Box[Byte]} and \code{Box[Int]}
    would have the same size in Java but different sizes in X10.

\Ref{Figure}{Constraints} showed that a \code{var} must be assigned if
    it does not contain the zero value.
For generics, we added a \code{haszero} type condition that requires a type parameter to have the zero value.


\Ref{Figure}{Generics} todo ...

\begin{figure}
\begin{lstlisting}
class A[T] {
  var a:T; //err
}
class B[T] {T haszero} {
  var a:T;
}
class Usage {
  var b1:B[Int];
  var b1:B[String];
  var b1:B[Int{self!=0}]; //err
}
\end{lstlisting}
\definerule{Rule21}
\caption{\code{haszero} type condition.
    \myrule{\arabic{Rule21}}{statically checks a type has the zero value}
    }
\label{Figure:Generics}
\end{figure}


examples for \code{Array}

 \code{Zero.get[T]()}







We now turn our attention to the parallel features of X10:
    concurrent programming (\code{finish} and \code{async})
    and distributed programming (\code{at} and global references).


\subsection{Concurrency}
Explain async+finish.
If the finish was removed, then it would be an error ...


\Ref{Figure}{Asynchronously-init} todo ...

\begin{figure}
\begin{lstlisting}
class A {
  val a:Int;
  val b:Int;
  def this() {
    finish {
      async a = 1;
    }
    async b = 2; //err
  }
}
\end{lstlisting}
\definerule{Rule13}
\caption{Asynchronously assigned fields example.
    \myrule{\arabic{Rule13}}{All field assignments must finish when the constructor ends}
    }
\label{Figure:Asynchronously-init}
\end{figure}


\subsection{Multiple Places}
Places require serialization and deserialization (both custom and automatic) across "at".

\Ref{Figure}{Multi-place} todo ...

\begin{figure}
\begin{lstlisting}
class A {
  val a:Int;
  def this() {
    // Execute at another place
    at (here.next())
      this.a = 1; //err
  }
}
\end{lstlisting}
\definerule{Rule14}
\caption{Multi-place initialization example.
    \myrule{\arabic{Rule14}}{a raw \this cannot cross to another place}
    }
\label{Figure:Multi-place}
\end{figure}


\subsection{Global references}
Special GlobalRef class: let \this escape and it creates a cycle.
We allow a raw \this to escape iff
* field is private with a field initializer.
* cannot be used with a raw \this receiver.


\Ref{Figure}{GlobalRef} todo ...

\begin{figure}
\begin{lstlisting}
class A {
  private val root =
   new GlobalRef(this);
  def me() = root();
}
class B extends A {
  def this() {
    val alias = me(); //err
  }
}
\end{lstlisting}
\definerule{Rule15}
\caption{\code{GlobalRef} example.
    \myrule{\arabic{Rule15}}{A raw \this can only escape to a global ref constructor in a field initializer,
        provided the field is private and is not read via a raw \this receiver}
    }
\label{Figure:GlobalRef}
\end{figure}

If \code{me()} was prefixed with
\code{@NonEscaping public final}
then accessing \code{root} would be an error.
%Cannot use 'root' because a GlobalRef[\ldots](this) cannot be used in a field initializer, constructor, or methods called from a constructor.



The last subsection is dedicated to a inter-procedural data-flow analysis for guaranteeing
    that a field is read only after it was written.

\subsection{Read and write of fields}
Java performs intra-procedural data flow in constructors ...
(also discuss locals?)

Inter-procedural data flow

final vs.\ non-final fields:
(fixed point)
todo

- Dataflow for definite-assignment for locals (similarity to Java, with the extension for async-initialization)
- Dataflow for definite-assignment for fields:
  + @NoThisAccess and @NonEscaping
  + The fixed-point algorithm to infer the sets of fields that are written, async-written, and read, on each private method that is transitively called from a ctor or field-init. Maybe also add a nice proof :)


In Java it is legal...

\Ref{Figure}{Read-Write-Order} todo ...

\begin{figure}
\begin{lstlisting}
class A {
  val a:Int;
  def this() {
    m2();
    a = 1;
  }
  final def m2() {
    val x = a; //err
  }
}
class B {
  var i:Int; // implicitly initialized to 0
  var j:Int{self!=0};
  def this() {
    i++;
    j++; //err
  }
}
class C {
  var j:Int{self!=0};
  def this() {
    write();
    read();
  }
  final def write() {
    this.j = 1;
  }
  final def read() {
    val x = this.j;
  }
}
\end{lstlisting}
\definerule{Rule11}
\definerule{Rule12}
\caption{Read-Write order for fields examples.
    \myrule{\arabic{Rule11}}{A field can be read only after it was definitely written}
    \myrule{\arabic{Rule12}}{A constructor must write to all fields}
    }
\label{Figure:Read-Write-Order}
\end{figure}



\subsection{Claims about initialization in X10}
We assume we transformed all anonymous, local and inner classes into static nested classes
    as described in \Ref{Section}{Inner}.
Therefore we have only have a single \this variable.


* \this and \code{super} is the only accessible raw object (there could be several raw objects in the heap but only \this is accessible). Reason: \this cannot be aliased or leaked.
* only cooked objects cross places. Reason: \this is the only non-cooked object and it cannot have any aliases and it cannot cross an \code{at}.
* Only \code{@NoThisAccess} methods can be dynamically dispatched during construction. Reason: can only call private or final methods.
* all field writes finish by the time the ctor ends. Reason: data flow ensures that any write within an async has an enclosing finish.
* one cannot read an uninitialized field. Reason: reading from \this is ok by data-flow, any other read is from a cooked object.


\subsection{Memory model and constructor barrier}

todo: discuss final in Java and String class.
(people think that removing final may only hurt performance, but it may be semantic changing.)

type safety and the weak memory model:
* in Java if you don't use final fields correctly or leak this, you will simply see default values (you don't lose type-safety)
* in X10 it could break type safety (if we don't put a barrier at the end of a ctor).

class Box[T] {T haszero} {
  var value:T;
}
class A {
  static val box = new Box[A]();

  var f:Int{self!=0} = 1;
}

var a = new A();
a.f = 2;
A.box.value = a;

Suppose another activity reads A.box.value.
Should the writes to
a.f
and
A.box.value
be ordered? (I don't think we should order them without losing performance)

Therefore, X10 needs a synchronization barrier at the end of a ctor that guarantees that all writes to the fields (both VAL and VAR) of the object has finished before the handle is returned.
(This is different from Java that only promises this for final fields. And the barrier also happens again after deserialization - requiring this weird freeze operation and you could freeze a final field again even after the ctor ended due to deserialization.)


Bowen proposes this transformation in order to inline ctors:
class A {
  static val box = new Box[A]();

  var f:Int{self!=0} = 0;
}


var a = new A();
a.f = 1;
// INSERT BARRIER EXPLICITLY HERE
a.f = 2;
A.box.value = a;
