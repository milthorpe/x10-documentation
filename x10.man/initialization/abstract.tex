X10 is an object oriented programming language with a sophisticated
    type system (constraints, class invariants, non-erased generics, closures)
    and concurrency constructs (asynchronous activities, multiple places\removeGlobalRef{, global references}).
Object initialization is a cross-cutting concern that interacts with all of these features
    in delicate ways that may cause type, runtime, and security errors.
This paper discusses possible designs for object initialization,
    and the ``hardhat'' design chosen and implemented in X10 version 2.2.
Our implementation includes a
    fixed-point inter-procedural (intra-class)
    data-flow analysis
    that infers, for each method called during initialization,
    the set of fields that are read, and
    those that are asynchronously and synchronously assigned.
Our codebase of more than 200K lines of code only had 104 annotations.
%	and Java code was converted to X10 with minor changes.
Finally, we formalize the essence of initialization checking with an
effect system intended to complement a standard FJ style
formalization of the type system for X10. This system is substantially
simpler than the masked types of \cite{XinQi:2009}. To our knowledge, this is
the first formalization of a type and (flow-sensitive) effect system
for safe initialization in the presence of concurrency
constructs. This formalization can be extended to cover all the
features discussed in the first part of the paper.
%Finally, we present a case study in which a large collection of java classes
%    were converted to X10,
%    and discuss the consequences of having a hardhat design.
