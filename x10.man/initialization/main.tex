\documentclass[9pt]{sigplanconf}

%\conferenceinfo{X10'11,} {June 4, 2011, San Jose, California, USA.}
%\CopyrightYear{2011}
%\crdata{978-1-4503-0770-3}
%\copyrightdata{}

% \stepcounter Doesn't work in a caption!
\newcounter{RuleCounter}
\stepcounter{RuleCounter}
\newcommand{\userule}[1]{\arabic{#1}}
\newcommand{\definerule}[1]{\newcounter{#1}\addtocounter{#1}{\arabic{RuleCounter}}\stepcounter{RuleCounter}}
\newcommand{\removeGlobalRef}[1]{}

\usepackage{relsize}
\usepackage{amsmath}
\usepackage{url}


% http://en.wikibooks.org/wiki/LaTeX/Packages/Listings
\usepackage{listings}


\usepackage[ruled]{algorithm} % [plain]
\usepackage[noend]{algorithmic} % [noend]
\renewcommand\algorithmiccomment[1]{// \textit{#1}} %


% For fancy end of line formatting.
\usepackage{microtype}


% For smaller font.
\usepackage{pslatex}


\usepackage{xspace}
\usepackage{yglabels}
\usepackage{yglang}
\usepackage{ygequation}
\usepackage{graphicx}
%\usepackage{epstopdf}

\newcommand{\formalrule}[1]{\mbox{\textsc{\scriptsize #1}}}
\newcommand{\myrule}[1]{\textsc{\codesmaller #1 Rule}}
\newcommand{\umyrule}[1]{\textbf{\underline{\textsc{\codesmaller #1 Rule}}}}


% \small \footnotesize \scriptsize \tiny
% \codesize and \scriptsize seem to do the same thing.
% \newcommand{\code}[1]{\texttt{\textup{\footnotesize #1}}}
% \newcommand{\code}[1]{\texttt{\textup{\codesize #1}}}
\newcommand{\normalcode}[1]{\texttt{\textup{#1}}}
\def\codesmaller{\small}
\newcommand{\myCOMMENT}[1]{\COMMENT{\small #1}}
\newcommand{\code}[1]{\texttt{\textup{\codesmaller #1}}}
% \newcommand{\code}[1]{\ifmmode{\mbox{\smaller\ttfamily{#1}}}
%                       \else{\smaller\ttfamily #1}\fi}
\newcommand{\smallcode}[1]{\texttt{\textup{\scriptsize #1}}}
%\newcommand{\myparagraph}[1]{\noindent\textit{\textbf{#1}}~} %\vspace{-1mm}\paragraph{#1}}

% See: \usepackage{bold-extra} if you want to do \textbf
\newcommand{\keyword}[1]{\code{#1}}

% For new, method invocation, and cast:
\newcommand{\hparen}[1]{\code{(}#1\code{)}}

\newcommand{\hgn}[1]{\lt#1\gt} % type parameters and generic method parameters

\newcommand{\Own}{{\it O}}
\newcommand{\Ifn}[1]{\ensuremath{I(#1)}}
\newcommand{\Ofn}[1]{\ensuremath{O(#1)}}
\newcommand{\Cooker}[1]{\ensuremath{{\kappa}(#1)}}
\newcommand{\Owner}[1]{\ensuremath{{\theta}(#1)}}
\newcommand{\Oprec}[0]{\ensuremath{\preceq_{\theta}}}
\newcommand{\Tprec}[0]{\ensuremath{\preceq^T}}
\newcommand{\TprecNotEqual}[0]{\ensuremath{\prec^T}}
\newcommand{\OprecNotEqual}[0]{\ensuremath{\prec_{\theta}}}
\newcommand{\IfnDelta}[1]{\ensuremath{I_\Delta(#1)}}
\newcommand\Abs[1]{\ensuremath{\left\lvert#1\right\rvert}}
\newcommand{\erase}[1]{\ensuremath{\Abs{#1}}}

\newcommand{\CookerH}[1]{\ensuremath{{\kappa}_H(#1)}}
\newcommand{\IfnH}[1]{\ensuremath{I_H(#1)}}
\newcommand{\OwnerH}[1]{\ensuremath{{\theta}_H(#1)}}
\newcommand{\Gdash}[0]{\ensuremath{\Gamma \vdash }}
\newcommand{\reducesto}[0]{\rightsquigarrow}
\newcommand{\reduce}[0]{\rightsquigarrow}

\usepackage{color}
\definecolor{light}{gray}{.75}


\newcommand{\todo}[1]{\textbf{[[#1]]}}
%% To disable, just uncomment this line:
%\renewcommand{\todo}[1]{\relax}

%% Additional todo commands:
\newcommand{\TODO}[1]{\todo{TODO: #1}}

\newcommand\xX[1]{$\textsuperscript{\textit{\text{#1}}}$}


\newcommand{\ol}[1]{\overline{#1}}
\newcommand{\nounderline}[1]{{#1}}


%% Commands used to typeset the FIGJ type system.
\newcommand{\typerule}[2]{
\begin{array}{c}
  #1 \\
\hline
  #2
\end{array}}


%% Commands used to typeset the FOIGJ type system.
\newcommand{\inside}{\prec}
\newcommand{\visible}{{\it visible}}
\newcommand{\placeholderowners}{{\it placeholderowners}}
\newcommand{\nullexpression}{{\tt null}}
\newcommand{\errorexpression}{{\tt error}}
\newcommand{\locations}{{\it locations}} %% \mathop{\mathit{locations}}}
\newcommand{\xo}{{\tt X^O}}
\newcommand{\no}{{\tt N^O}}
\newcommand{\co}{{\tt C^O}}
\newcommand{\I}{\it I}


\newcommand\mynewcommand[2]{\newcommand{#1}{#2\xspace}}


\mynewcommand{\hI}{\code{I}} % iparam

% In the syntax: \hI or ReadOnly or Mutable or Immut
\mynewcommand{\hJ}{\code{J}}
\mynewcommand{\hO}{\code{O}}
\mynewcommand{\ho}{\code{o}}
\mynewcommand{\hnull}{\code{null}}
\mynewcommand{\htrue}{\code{true}}
\mynewcommand{\hfalse}{\code{false}}

\mynewcommand{\hX}{\code{X}} % vars
\mynewcommand{\hY}{\code{Y}} % vars
\mynewcommand{\hB}{\code{B}} % class
\mynewcommand{\hC}{\code{C}} % class
\mynewcommand{\hD}{\code{D}} % class
\mynewcommand{\hL}{\code{L}} % class decl
\mynewcommand{\hM}{\code{M}} % Method decl
\mynewcommand{\hm}{\code{m}} % method
\mynewcommand{\he}{\code{e}} % expression
\mynewcommand{\hl}{\code{l}} % location in the store
\mynewcommand{\hx}{\code{x}} % method parameter
\mynewcommand{\hf}{\code{f}} % field name
\mynewcommand{\hF}{\code{F}} % field
\mynewcommand{\hH}{\code{H}} % Heap
\mynewcommand{\hS}{\code{S}}
\mynewcommand{\hsub}{\code{/}} % substitute (reduction rules)

\mynewcommand{\hand}{\code{~and~}}
\mynewcommand{\hor}{\code{~or~}}
\mynewcommand{\hthis}{\code{this}} % this
\mynewcommand{\super}{\code{super}} % this
\mynewcommand{\hsuper}{\code{super}} % this
\mynewcommand{\hclass}{\code{class}}
\mynewcommand{\hreturn}{\code{return}}
\mynewcommand{\hnew}{\code{new}}
\newcommand{\lt}{\code{<}}%{\mathop{\textrm{\tt <}}}
\newcommand{\gt}{\code{>}}%{\mathop{\textrm{\tt >}}}

\mynewcommand{\this}{\keyword{this}}
\mynewcommand{\ctor}{\keyword{ctor}}
\mynewcommand{\Object}{\code{Object}}
\mynewcommand{\const}{\keyword{const}} %C++ keyword
\mynewcommand{\mutable}{\keyword{mutable}} %C++ keyword
\mynewcommand{\romaybe}{\keyword{romaybe}} %Javari keyword

%% Define the behaviour of the theorem package.
%% Use http://math.ucsd.edu/~jeggers/latex/amsthdoc.pdf for reference.

\newtheorem{theorem}{Theorem}[section]
\newtheorem{definition}[theorem]{Definition}
\newtheorem{lemma}[theorem]{Lemma}
\newtheorem{corollary}[theorem]{Corollary}
\newtheorem{fact}[theorem]{Fact}
\newtheorem{example}[theorem]{Example}
\newtheorem{remark}[theorem]{Remark}


\mynewcommand{\IP}{\code{I}}   % formal type parameter
\mynewcommand{\JP}{\code{J}}   % formal type parameter (for soundness proofs)


\mynewcommand{\Iparam}{Immutability parameter}
\mynewcommand{\iparam}{immutability parameter}
\mynewcommand{\iparams}{immutability parameters}
\mynewcommand{\Iparams}{Immutability parameters}
\mynewcommand{\Iarg}{Immutability argument}
\mynewcommand{\iarg}{immutability argument}
\mynewcommand{\iargs}{immutability arguments}
\mynewcommand{\Iargs}{Immutability arguments}
\mynewcommand{\ReadOnly}{\code{ReadOnly}}
\mynewcommand{\WriteOnly}{\code{WriteOnly}}
\mynewcommand{\None}{\code{None}}
\mynewcommand{\Mutable}{\code{Mutable}}
\mynewcommand{\Immut}{\code{Immut}}
\mynewcommand{\Raw}{\code{Raw}}


\mynewcommand{\This}{\code{This}}
\mynewcommand{\World}{\code{World}}


% Our annotations
\mynewcommand{\OMutable}{\code{@OMutable}}
\mynewcommand{\OI}{\code{@OI}}


\mynewcommand{\InVariantAnnot}{\code{@InVariant}}


\newcommand{\func}[1]{\text{\textnormal{\textit{\codesmaller #1}}}}


\mynewcommand{\st}{\ensuremath{\mathrel{{\leq}}}} %{\mathop{\textrm{\tt <:}}}
\mynewcommand{\notst}{\mathrel{\st\hspace{-1.5ex}\rule[-.25em]{.4pt}{1em}~}}
\mynewcommand{\tl}{\ensuremath{\triangleleft}}
\mynewcommand{\gap}{~ ~ ~ ~ ~ ~}


\newcommand{\RULE}[1]{\textsc{\scriptsize{}#1}} %\RULEhape\scriptsize}


\mynewcommand{\DA}{\texttt{DA}}
\mynewcommand{\ok}{\texttt{OK}}
\mynewcommand{\OK}{\texttt{OK}}
\mynewcommand{\IN}{\texttt{IN}}
\mynewcommand{\subterm}{\func{subterm}}
\mynewcommand{\TP}{\func{TP}} % function that returns type parameters in a type
\mynewcommand{\CT}{\func{CT}} % class table
\mynewcommand{\mtype}{\func{mtype}}
\mynewcommand{\mmodifier}{\func{mmodifier}}
\mynewcommand{\fmodifier}{\func{fmodifier}}
\mynewcommand{\ctortype}{\func{ctortype}}
\mynewcommand{\ctorbody}{\func{ctorbody}}
\mynewcommand{\mbody}{\func{mbody}}
\mynewcommand{\ftype}{\func{ftype}}
\mynewcommand{\fields}{\func{fields}}
\DeclareMathOperator{\dom}{dom}
\mynewcommand{\methodVal}{\func{methodVal}_\Gamma} % val not assigned
\mynewcommand{\ctorVal}{\func{ctorVal}_\Gamma} % val assigned at most once
\mynewcommand{\reductionVal}{\func{reductionVal}} % val assigned at most once



\mynewcommand\xth{\xX{th}}
\mynewcommand\xrd{\xX{rd}}
\mynewcommand\xnd{\xX{nd}}
\mynewcommand\xst{\xX{st}}
\mynewcommand\ith{$i$\xth}
\mynewcommand\jth{$j$\xth}


%\mynewcommand{\emptyline}{\vspace{\baselineskip}}
\mynewcommand{\myindent}{~~}


% Add line between figure and text
\makeatletter
\def\topfigrule{\kern3\p@ \hrule \kern -3.4\p@} % the \hrule is .4pt high
\def\botfigrule{\kern-3\p@ \hrule \kern 2.6\p@} % the \hrule is .4pt high
\def\dblfigrule{\kern3\p@ \hrule \kern -3.4\p@} % the \hrule is .4pt high
\makeatother

\setlength{\textfloatsep}{.75\textfloatsep}


% Remove line between figure and its caption.  (The line is prettier, and
% it also saves a couple column-inches.)
\makeatletter
%\@setflag \@caprule = \@false
\makeatother


% http://www.tex.ac.uk/cgi-bin/texfaq2html?label=bold-extras
\usepackage{bold-extra}


% Left and right curly braces in tt font
\newcommand{\ttlcb}{\texttt{\char "7B}}
\newcommand{\ttrcb}{\texttt{\char "7D}}
\newcommand{\lb}{\ttlcb}
\newcommand{\rb}{\ttrcb}


\setlength{\leftmargini}{.75\leftmargini}


\begin{document}




\title{Object Initialization in X10}

\authorinfo{}{}{}
%\authorinfo{Yoav Zibin \and David Cunningham \and Igor Peshansky \and Vijay Saraswat}
%           {IBM research in TJ Watson}
%           {yzibin$~|~$dcunnin$~|~$igorp$~|~$vsaraswa@us.ibm.com}


\maketitle



\begin{abstract}
\Xten{} is a modern object-oriented language designed for productivity and
performance in concurrent and distributed systems.  In this setting, dependent
types offer significant opportunities for detecting design errors statically,
documenting design decisions, eliminating costly run-time checks (e.g., for
array bounds, null values), and improving the quality of generated code.

We present the design and implementation of {\em constrained types}, a natural,
simple, clean, and expressive extension to object-oriented programming: A type
\xcd{C\{c\}} names a class or interface \xcd{C} and a {\em constraint} \xcd{c}
on the immutable state of \xcd{C} and in-scope final variables.  Constraints
may also be associated with class definitions (representing class invariants)
and with method and constructor definitions (representing preconditions).
Dynamic casting is permitted.  The system is parametric on the underlying
constraint system: the compiler supports a simple equality-based constraint
system but, in addition, supports extension with new constraint systems using
compiler plugins.


\end{abstract}

%\category{D.3.3}{Programming Languages}{Language Constructs and Features}
%\category{D.1.5}{Programming Techniques}{Object-oriented Programming}

%\terms
%Asynchronous, Initialization, Types, X10

% Keywords are not required in the paper itself - only in the submission system's meta data.
% \keywords
% Immutability, Ownership, Java

\Section[introduction]{Introduction}
A {\em serial} schedule for a parallel program is one which always
executes the first enabled step in program order. A {\em safe}
parallel program is one that can be executed with a serial schedule
$S$ and for which for every input every schedule produces the same
result (error, correct termination, divergence) as $S$.  Such a
program is semantically a sequential program, hence it is
scheduler-determinate and deadlock-free.

A {\em safe parallel programming language} is an imperative parallel
programming language in which every legal program is a safe
program. Programmers can write code in such a language secure in the
knowledge that they will not encounter a large class of parallel
programming problems. Such a language is particularly useful for
parallelizing sequential (imperative) programs. In such cases ({\em
  contra} reactive programming) the desired application semantics are
sequential, and parallelism is needed solely for efficient
implementation.

The characteristic property of a safe programming language is that the
{\em same} program has a sequential reading and a parallel reading,
and both are compatible with each other. Hence the program can be
developed and debugged as a sequential program, using the serial
scheduler, and then run the unchanged program in parallel.  Parallel
execution is guaranteed to effect only performance, not
correctness. Safety is a very strong property.

There are many challenges in designing an efficient, usable, powerful,
explicitly parallel imperative language that is safe. One central
challenge is that 

One way to get safety is through implicitly parallel languages (e.g.{}
Jade~\cite{Rinard98thedesign},\cite{vonPraun:2007:IPO:1229428.1229443}). One
starts with a sequential programming language, and adds constructs
(e.g.{} tasks) that permit speculative execution while guaranteeing
that the only observable write to a shared variable is the write by
the last task to execute in program order.  While this work is
promising, extracting usable parallelism from a wide variety of
sequential programs remains very hard.

Explicitly parallel programming languages provide a variety of
constructs for spawning tasks in parallel and coordinating between
them. Here the programmer can typically directly control the
granularity of concurrency, and locality of access (e.g. placement of
data-structures in a multi-node computation) and use efficient
concurrent primitives (atomic reads/writes, test and sets, locks etc)
to control their execution. 

For such languages proving that a program is safe -- much less that
the programming language is safe -- now becomes very hard,
particularly for modern object-oriented languages which allow the
programmer to create arbitrarily complicated data-structures in the
shared heap. It becomes very difficult to show that any possible
schedule will produce the same result as the serial schedule.

Our starting point is the language X10 \cite{x10} since it offers a
simple and elegant treatment of concurrency and distribution, with
some nice properties.  In brief, X10
introduces the constructs \code{async S} to spawn an activity to
execute \code{S}; \code{finish S} to execute \code{S} and wait until
such time as all activities spawned during its execution have
terminated; \code{at (p) S} to execute \code{S} at place
\code{p}. These constructs can be nested arbitrarily -- this is a
source of significant elegance and power. Additionally, X10 v 2.2
introduces a simplified version of X10 clocks (adequate for many
practical usages) -- \code{clocked finish S} and \code{clocked async
  S}. Briefly, a clocked finish introduces a new barrier that can be
used by this activity and its children activities for
synchronization.\footnote{X10 also has a conditional atomic construct,
  \code{when (c) S} which permits data-dependent synchronization and
  can introduce deadlocks. We do not consider this construct in this paper.}

\cite{vj-clock} establishes that a large class of programs
in X10, namely those that use \code{finish}, \code{async} and
\code{clock}s are deadlock-free. The central intuition is that a clock
can only be used by the activity that created it and by its children,
and hence the spawn tree structure can be used to avoid depends-for
cycles. 

To obtain determinacy, another idea is required. The central problem
is to ensure that in the statement \code{async \{S\} S1} it is not the
case that \code{S} can write to a location that \code{S1} can read
from or read from a location that \code{S} can read from. Otherwise
the behavior is not determinate. One line of attack has been the pursuit
of {\em effect systems} \cite{Lucassen:1988:PES:73560.73564},
\cite{Leino:2002:UDG:543552.512559},
\cite{boyland:01interdependence},
\cite{DPJ}. At a broad conceptual level, static effects systems call
for a user-specified partitioning of heap into {\em regions} in a
fine-grained enough way to show that operations that may occur
simultaneously work on different regions. For instance \cite{DPJ}
introduces separate syntax for regions, introduces the ability to
specify an arbitrary tree of regions, and new syntax for specifying
which region mutable locations belong to. Methods must be associated
with their read and write effects which capture the set of regions
that can be operated upon during the execution of this
method. Now determinacy can be established if it can be determined
that that in \code{async \{S\} S1} it is the case that \code{S} does
not write any region that \code{S1} reads or writes from, and vice
versa ({\em disjoint parallelism}). 

\cite{DPJ} develops these ideas in the context of a language with
\code{cobegin/coend} and \code{forall} parallelism, and does not
address arbitrary nested or clocked parallelism.  In particular it is
not clear how to adapt these ideas to support some form of pipelined
communication between multiple parallel activities. Once can think of
these computations as requiring ``disjointness across time'' rather
than disjointness across space. A producer is going to write into
locations \code{w(t),w(t+1),w(t+2), \ldots}, but this does not
conflict with a consumer reading from the same locations as long as
the consumer can arrange to read the values in time-staggered order,
i.e. read \code{w(t)} once the produce is writing \code{w(t+1)}.  

We believe it is possible to significantly simplify this approach
(e.g.{} using the dependent-type system of X10) and extend it to cover
all of X10's concurrency constructs (see
Section~\ref{future-work}). Nevertheless the cost of developing these
region assertions is not trivial. Their viability for developing large
commercial-strength software systems is yet to be establishd.

In this paper we chose a different approach, a {\em lightweight}
approach to safety. We introduce two simple ideas -- {\em
  accumulators} and {\em clocked types} -- which require very modest
compiler support and can be implemented efficiently at run-time.

Accumulators arise very naturally in concurrent programming: an
accumulator is a mutable location associated with a commutative and
associative {\em reduction} operator that can be operated on
simultaneously by multiple activities. Multiple values offered by
multiple activities are combined by the reduction operator. We show
that the concrete rules for accumulators can be defined in such a way
that they do not compromise safety: a serial execution precisely
captures results generable from any execution.

Similarly clocked computations (barrier-based synchronization) is
quite common in parallel programming e.g. in the SPMD model, BSP model
etc. We observe that many clocked programs can be written in such a
way that shared variables take on a single value in one phase of the
clock. Further, clocked computations are often iterative and operate
on large aggregate data-structures (e.g. arrays, hash-maps) in a
data-parallel fashion, reading one version of the data-structure (the
``red'' version) while simultaneously writing another version (the
 ``black'' version). To support this widely used idiom, we introduce
the notion of {\em clocked types}. An instance \code{a} of a  clocked type
\code{Clocked[T]} keeps two instances of type \code{T}, the \code{now} and
the \code{next} instance. \code{a} can be operated upon only by
activities registered on the current clock.  All read operations
during the current clock phase are directed to the \code{now} version,
and write operation to the \code{next} version. This ensures that
there are no read-write conflicts. There may be write-write conflicts
-- these must be managed by either using accumulators or an effects
system. Once computation in the current phase has quiesced -- and
before activities start in the next phase -- the \code{now} and
\code{next} versions are switched; \code{now} becomes \code{next} and
\code{next} becomes \code{now}.\footnote{Clearly this idea can be
  extended to $k$-buffered clocked types where each clock tick rotates
  the buffer. This idea is related to the K-bounded Kahn networks of \cite{k-bounded-kahn}.}

We show that clocked types can be defined in a safe way, provided that
accumulators are used to resolve write-write conflicts. The only
dynamic check needed is that a value of this type is being operated
upon only by the activity that created the value or its
descendants. 

In the following by Safe X10 we shall mean the language X10 restricted
to use (\code{clocked}) \code{finish}/\code{async}, \code{at} with
(clocked) accumulators.  All programs in Safe X10 are safe -- they can
be run with a serial schedule and their I/O behavior is identical
under any schedule.  We show that Safe X10 is surprisingly
powerful. Many concurrent idioms can be expressed in this language --
histograms, all-reduce, SpecJBB-style communication Indeed, even some
form of pipelined/systolic communication is expressible.

Since the dynamic conditions introduced on accumulators and clocked
types are not straightforward, we formalize the concurrent and serial
semantics of an abstraction of Safe X10 using Plotkin's structural
operational style. We are able to do this in such a way that the two
proof systems share most of the proof rules, simplifying the proof. We
establish that the language is safe -- for any program and any input,
any execution sequence for the concurrent proof rules can be
transformed into an execution sequence for the sequential proof rules
with the same result. 

In summary the contributions of this paper are as follows.

\begin{itemize}
\item We identify the notion of a {\em safe} program -- one which can
  be executed with a serial schedule and for which 
  every schedule produces the same result. Such a program is
  simultaneously a sequential program and a parallel program with
  identical I/O behavior. 
\item We introduce accumulators and clocked types in the X10
  programming model. These are introduced in such a way that arbitrary
  programs using (\code{clocked}) \code{finish}, \code{async} and
  \code{at} and in which the only variables shared between
  concurrently executing activities are accumulators or clocked
  accumulators are guaranteed to be safe.  
\item We show that many common programming idioms can be expressed in
  this language.
\item We formalize a fairly rich subset of X10 -- including (clocked)
  finish, async, accumulators and clocked accumulators.  This is the
  first formalization of the nested clock design of X10 2.2, and is
  substantially simpler than \cite{vj-clock}. We establish that this
  language is safe. 
\end{itemize}
In companion work we show how these ideas can be extended to support
modularly defined effects analyses, using X10's dependent type system.

The rest of this paper is as follows. In \Ref{related-work} we discuss
related work. In \Ref{constructs} we present the constructs in detail,
followed by examples of their use. \Ref{semantics} presents the
semantics of these constructs. We discuss implementation in
\Ref{implementation} and finally conclude with future work.

%%\cite{Gifford:1986:IFI:319838.319848}
%%
%%
%%Figure~\ref{fig:1} shows  the famous ``parallel Or'' program of Plotkin
%% (in X10 syntax, \cite{x10}). This program can be executed with a
%%depth-first schedule, is partially determinate and deadlock-free, but {\em not}
%%safe. The result of running the sequential schedule is not the same as
%%the result that can be obtained with other schedules. Specifically
%%\code{parallelOr(()=> CONT, ()=>TRUE)} will diverge (exhibit an
%%infinite exection sequence) under the depth-first schedule, but will
%%return \code{true} under any fair schedule that permits the second
%%async to progress.
%%
%%\begin{figure}
%%  \begin{lstlisting}
%%static val CONT=1, TRUE=2, FALSE=3;
%%def run(done:Cell[Boolean], a:()=>Int) {
%% var aa:Int=a();
%% var cont:Boolean=true;
%% for (; aa==CONT && cont;aa=a()) {
%%  atomic cont = !done();
%% }
%% if (aa==TRUE)
%%  atomic done()=true;
%%}
%%def parallelOr(a:()=>Int, b:()=>Int):Boolean {
%% val done=new Cell[Boolean](false);
%% finish {
%%  async run(done, a);
%%  async run(done, b);
%% }
%% return done();
%%}
%%  \end{lstlisting}
%%  \caption{A program that is not sequential}\label{fig:1}
%%\end{figure}



%%{\em
%%\begin{enumerate}
%%\item Use activity registration as a mechanism to tame object graphs.
%%\item Focus on structured concurrency. Using scoping and block-structure
%%    to delimit regions of code that may execute in parallel and affect
%%    the data structure.
%%
%%\item Accumulation can be defined safely by delaying. However, the delay
%%    operation is guaranteed to be deadlock-free.
%%
%%\item Clocked types support phased computation, another common idiom
%%    particularly for stencil computations.
%%\end{enumerate}
%%}
%%
%%Key contributions:
%%{\em
%%\begin{enumerate}
%%\item Identification of determinate, deadlock-free data-structures.
%%\item Discussion of design alternatives which points out the
%%  difficulty of integrating these ideas in a modern OO language.
%%\item Discussion of various idioms expressible using these data-structures.
%%\item Proof of determinacy and deadlock-freedom in an abstract version
%%  of the language.
%%\end{enumerate}
%%These constructs are implemented in \Xten, available as open source from
%%SVN head and will be in the next release of \Xten.
%%}


%Semantics and theorems for an abstract version of the language.





\vspace{-0.3cm}
\Section[rules]{X10 Initialization Rules}
X10 is an advanced object-oriented language with a complex type-system
    and concurrency constructs.
This section describes how object initialization interacts with X10 features.
We begin with object-oriented features found in mainstream languages,
    such as constructors, inheritance, dynamic dispatching, exceptions, and inner classes.
We then proceed to X10's type-system features,
    such as constraints, properties, class invariants, closures, (non-erased) generics, and structs.
The parallel features of X10 allow writing concurrent code (using \code{finish} and \code{async}),
    and distributed code (using \code{at} and global references).
Next we describe the inter-procedural data-flow analysis that ensures that
    a field is read only after it was assigned.
Finally, we summarize the virtues and attributes of initialization in X10.


\subsection{Constructors and inheritance}
Inheritance is the first feature that interacts with initialization:
    when class \code{B} inherits from \code{A}
    then every instance of \code{B} has a sub-object that is like an instance of \code{A}.
When we initialize an instance of \code{B}, we must first initialize its \code{A} sub-object.
We do this in X10 by forcing the constructors of \code{B} to make a super call,
    i.e., call a constructor of \code{A}
    (either explicitly or implicitly).



\begin{figure}
\begin{lstlisting}
class A {
  val a:Int;
  def this() {
    LeakIt.foo(this); //err
    this.a = 1;
    val me = this; //err
    LeakIt.foo(me);
    this.m2(); // so m2 is implicitly non-escaping
  }
  // permitted to escape
  final def m1() {
    LeakIt.foo(this);
  }
  // implicitly non-escaping because of this.m2()
  final def m2() {
    LeakIt.foo(this); //err
  }
  // explicitly non-escaping
  @NonEscaping final def m3() {
    LeakIt.foo(this); //err
  }
}
class B extends A {
  val b:Int;
  def this() {
    super(); this.b = 2; super.m3();
  }
}
\end{lstlisting}
\definerule{Rule3}
\definerule{Rule4}
\caption{Escaping \this example.
    \textbf{Definition of \emph{raw}:}
    {\this and \code{super} are \emph{raw} in {non-escaping} methods and in field initializers}.
    \textbf{Definition of \emph{non-escaping}:}
        {A method is \emph{non-escaping} if it is a constructor,
            or annotated with \code{@NonEscaping} or \code{@NoThisAccess},
            or a method that is called on a raw \this receiver}.
    \myrule{\arabic{Rule3}}{A raw \this or \code{super} cannot escape or be aliased}
    \myrule{\arabic{Rule4}}{A call on a raw \code{super} is allowed only for a \code{@NonEscaping} method}
    (\code{\textbf{final}} and \code{@NoThisAccess} are related
        to dynamic-dispatching as shown in \Ref{Figure}{Dynamic-dispatch}.)}
\label{Figure:Escaping-this}
\end{figure}

\Ref{Figure}{Escaping-this} shows X10 code that demonstrates the interaction
    between inheritance and initialization,
    and explains by example why leaking \this during construction can cause bugs.
In all the examples, errors issued by the X10 compiler are marked with \code{//err}.

We say that an object is \emph{raw} (also called partially initialized) before its constructor ends,
    and afterward it is \emph{cooked} (also called fully initialized).
Note that when an object is cooked, all its sub-objects must be cooked as well.
X10 prohibits any aliasing or leaking of \this during construction,
    therefore only \this or \code{super} can be raw (any other variable is definitely cooked).

Object initialization begins by invoking a constructor,
    denoted by the method definition \code{def this()}.
The first leak would cause a problem because field \code{a} was not assigned yet.
However, even after all the fields of \code{A} have been assigned,
    leaking is still a problem
    because fields in a subclass (field \code{b}) have not yet been initialized.
Note that leaking is not a problem if \this is not raw, e.g., in \code{m1()}.

We begin with two definitions:
    (i)~when an object is \emph{raw}, and
    (ii)~when a method is \emph{non-escaping}.
(i)~Variables \this and \code{super} are raw
    during the object's construction,
    i.e., in field initializers and in {non-escaping} methods
    (methods that cannot escape or leak \this).
(ii)~Obviously constructors are non-escaping,
    but you can also annotate methods \emph{explicitly} as \code{@NonEscaping},
    or they can be inferred to be \emph{implicitly} non-escaping
    if they are called on a raw \this receiver.

For example, \code{m2} is \emph{implicitly} non-escaping (and therefore cannot leak \this)
    because of the call to \code{m2}
    in the constructor.
The user could also mark \code{m2} \emph{explicitly} as non-escaping by using the annotation
    \code{@NonEscaping}.
(Like in Java, \code{@} is used for annotations in X10.)
We recommend to explicitly mark public methods as \code{@NonEscaping} to show intent,
    as done on method \code{m3}.
Without this annotation the call \code{super.m3()} in \code{B} would be illegal,
    due to rule~\userule{Rule4}.
(We could infer that \code{m3} must be non-escaping,
    but that would cause a dependency from a subclass to a superclass,
    which is not natural for people used to separate compilation.)
Finally, we note that all errors in this example are due to rule~\userule{Rule3}
    that prevents leaking a raw \this or \code{super}.




\subsection{Dynamic dispatch}
Dynamic dispatching interacts with initialization by transferring control to the subclass
    before the superclass completed its initialization.
\Ref{Figure}{Dynamic-dispatch} demonstrates why dynamic dispatching is error-prone during construction:
    calling \code{m1} in \code{A} would dynamically dispatch and
    call the implementation in \code{B}
    that would read the default value.



\begin{figure}
\begin{lstlisting}
abstract class A {
  val a1:Int;
  val a2:Int;
  def this() {
    this.a1 = m1(); //err1
    this.a2 = m2();
  }
  abstract def m1():Int;
  @NoThisAccess abstract def m2():Int;
}
class B extends A {
  var b:Int = 3; // non-final field
  def this() {
    super();
  }
  def m1() {
    val x = super.a1;
    val y = this.b;
    return 1;
  }
  @NoThisAccess def m2() {
    val x = super.a1; //err2
    val y = this.b; //err3
    return 2;
  }
}
\end{lstlisting}
\definerule{Rule5}
\definerule{Rule6}
\caption{Dynamic dispatching example.
    \myrule{\arabic{Rule5}}{A non-escaping method must be private or final, unless it has \code{@NoThisAccess}}
    \myrule{\arabic{Rule6}}{A method with \code{@NoThisAccess} cannot access \this or \code{super} (neither read nor write its fields)}
    }
\label{Figure:Dynamic-dispatch}
\end{figure}


X10 prevents dynamic dispatching by requiring that non-escaping methods
    must be private or final
    (so overriding is impossible).
For example, \code{\itshape err1} is caused by rule~\userule{Rule5}
    because \code{m1} is neither private nor final nor \code{@NoThisAccess}.

However, sometimes dynamic dispatching is required during construction.
For example, if a subclass needs to refine initialization
    of the superclass's fields.
Such refinement cannot have any access to \this, and therefore
    such methods are marked with \code{@NoThisAccess}.
For example, \code{\itshape err2} and \code{\itshape err3} are caused by rule~\userule{Rule6} that prohibits access \this or \code{super}
    when using \code{@NoThisAccess}.
\code{@NoThisAccess} prohibits any access to \this,
    however, one could still access the method parameters.
(If the subclass needs to read a certain field of the superclass that was previously assigned,
    then that field can be passed as an argument.) % to the \code{@NoThisAccess} method


In C++, the call to \code{m1} is legal,
    but at runtime
    methods are statically bound,
    so you will get an error for calling a pure virtual function.
In Java, the call to \code{m1} is also legal,
    but at runtime
    methods are dynamically bound,
    so the implementation of \code{m1} in \code{B}
    will read the default value of \code{a} and \code{b}.
%This behavior is undesired in Java,
%    and Java discourages it by trying to catch statically most of these cases.
%For example, Java prohibits calls to member functions before the super object was initialized,
%    as this example shows (which is also illegal in X10):
%\begin{lstlisting}
%class B extends A { B() {super(f()); }}
%\end{lstlisting}




\subsection{Exceptions}
Constructing an object may not always end normally,
    e.g., building a date object from an illegal date string should throw an exception.
Exceptions combined with inheritance interact with initialization in the following way:
    a cooked object must have cooked sub-objects,
    therefore if a constructor ends normally (thus returning a cooked object)
    then all preceding constructor calls (either \code{super(\ldots)} or \code{this(\ldots)})
    must end normally as well.
Phrased differently, in a constructor it should not be possible to
    recover from an exception thrown by a constructor call.
This is one of the reason why a constructor call must be the first statement in Java;
    failure to verify this led to a famous security attack~\cite{Dean:1996}.

\begin{figure}
\begin{lstlisting}
class B extends A {
  def this() {
    try { super(); } catch(e:Throwable){} //err
  }
}
\end{lstlisting}
\definerule{Rule7}
\caption{Exceptions example:
    if a constructor ends normally (without throwing an exception),
        then all preceding constructor calls ended normally as well.
    \myrule{\arabic{Rule7}}{If a constructor does not call \code{super(\ldots)} or \code{this(\ldots)},
        then an implicit \code{super()} is added at the beginning of the constructor;
        the first statement in a constructor is a constructor call (either \code{super(\ldots)} or \code{this(\ldots)});
        a constructor call may only appear as the first statement in a constructor
        }
    }
\label{Figure:Exceptions}
\end{figure}


\Ref{Figure}{Exceptions} shows that it is an error to try to recover from an exception thrown
    by a constructor call;
    the reason is for the error is rule~\userule{Rule7} that requires the first statement to be \code{super()}.


\Subsection[Inner]{Inner classes}
Inner classes usually read the outer instance's fields during construction,
    e.g., a list iterator would read the list's header node.
Therefore, X10 requires that the outer instance is cooked,
    and prohibits creating an inner instance when the receiver is a raw \this.


\begin{figure}
\begin{lstlisting}
class Outer {
  val a:Int;
  def this() {
    // Outer.this is raw
    Outer.this. new Inner(); //err
    new Nested(); // ok
    a = 3;
  }
  class Inner {
    def this() {
      // Inner.this is raw, but
      // Outer.this is cooked
      val x = Outer.this.a;
    }
  }
  static class Nested {}
}
\end{lstlisting}
\definerule{Rule8}
\caption{Inner class example: the outer instance is always cooked.
    \myrule{\arabic{Rule8}}{a raw \this cannot be the receiver of \code{new}}
    }
\label{Figure:InnerClass}
\end{figure}

\Ref{Figure}{InnerClass} shows it is an error in X10 to create an inner instance
    if the outer is raw (from rule~\userule{Rule8}),
    but it is ok to create an instance of a static nested class,
    because it has no outer instance.

In fact, it is possible to view this rule as a special case to the rule that
    prohibits leaking a raw \this
    (because when you create an inner instance on a raw \this receiver,
    you created an alias,
    and now you have two raw objects: \code{Inner.this} and \code{Outer.this}).
We wish to keep the invariant that only one \this can be raw.

%To reduce complexity in the following subsections,
In our rules, we assume that there is a single \this reference,
    because we convert all inner, anonymous and local classes into
    static nested classes
    by passing the outer instance and all other captured variables
    explicitly as arguments to the constructor.






We now turn our attention to X10's sophisticated type-system
    that has features that are not found in main-stream languages:
    constraints, properties, class invariants, closures, (non-erased) generics, and structs.

\subsection{Constraints and default/zero values}
X10 supports constrained types using the syntax \code{T\lb{}c\rb},
    where \code{c} is a boolean expression that can use final variables in scope,
    literals, properties (described below),
    the special keyword \code{self}
    that denotes the type itself,
    field access, equality (\code{==}) and disequality (\code{!=}).
There are plans to add arithmetic inequality (\code{<}, \code{<=})
    to X10 in the future,
    and it is possible to plugin any constraint solver into the X10 compiler.

As a consequence of constrained types,
    some types do not have a default value, e.g., \code{Int\lb self!=0\rb}.
Therefore, in X10, the fields of an object cannot be zero-initialized as done in Java.
Furthermore, in Java, a non-final field does not have to be assigned in a constructor
    because it is assumed to have an implicit zero initializer.
X10 follows the same line, and a non-final field is implicitly added a zero initializer
    \emph{if its type has-zero}.
\Ref{Figure}{Constraints} defines when a type \emph{has-zero},
    and gives examples of types without zero.
Note that \code{i0} has to be assigned because it is a final field (\code{val}),
    as opposed to \code{i1} which is non-final (\code{var}).

\begin{figure}
\begin{lstlisting}
class A {
  val i0:Int; //err
  var i1:Int;
  var i2:Int{self!=0}; //err
  var i3:Int{self!=0} = 3;
  var i4:Int{self==42}; //err
  var s1:String;
  var s2:String{self!=null}; //err
  var b1:Boolean;
  var b2:Boolean{self==true}; //err
}
\end{lstlisting}
\definerule{Rule10}
\caption{Default value example.
    \textbf{Definition of \emph{has-zero}:}
        {A type \emph{has-zero} if it contains the zero value
            (which is either \code{null}, \code{false}, 0, or
                zero in all fields for user-defined structs)
            or if it is a type parameter guarded with \code{haszero} (see \Ref{Section}{Generics-and-Structs}).}
    \myrule{\arabic{Rule10}}{A \code{var} field that lacks a field initializer and whose type has-zero,
        is implicitly added a zero initializer}
    }
\label{Figure:Constraints}
\end{figure}


\subsection{Properties and the class invariant}
Properties are final fields that can be used in constraints,
    e.g., \code{Array} has a \code{size} property,
    so an array of size 2 can be expressed as: \code{Array\lb self.size==2\rb}.
The differences between a property and a final field are both syntactically and semantically,
    as seen in class \code{A} of \Ref{Figure}{Properties}.
Syntactically, properties are defined after the class name,
    they must have a type and cannot have an initializer,
    and they must be initialized in a constructor using a property call statement written as \code{property(\ldots)}.
Semantically, properties are initialized before all other fields,
    and they can be used in constraints with the prefix \code{self}.


\begin{figure}
\begin{lstlisting}
class A(a:Int) {
  def this(x:Int) {
    property(x);
  }
}
class B(b:Int) {b==a} extends A {
  val f1 = a+b;
  val f2:Int;
  val f3:A{this.a==self.a};
  def this(x:Int) {
    super(x);
    val i1 = super.a;
    val i2 = this.b; //err
    val i3 = this.f1; //err
    f2 = 2; //err
    property(x);
    f3 = new A(this.a);
  }
}
\end{lstlisting}
\definerule{Rule17}
\definerule{Rule18}
\definerule{Rule19}
\caption{Properties and class invariant example:
        properties (\code{a} and \code{b})
        are final fields that are initialized before all other fields
        using a property call (\code{property(\ldots);} statement).
    {If a class does not define any properties, then
        an implicit \code{property()} is added
        after (the implicit or explicit) \code{super(\ldots)}.}
    {Field initializers are executed in their declaration order
        after (the implicit or explicit) the property call.}
    \myrule{\arabic{Rule17}}{If a constructor does not call \code{this(\ldots)},
        then it must have exactly one
        property call, and it must be unconditionally executed
        (unless the constructor throws an exception)}
    \myrule{\arabic{Rule18}}{The class invariant must be satisfied after the property call}
    \myrule{\arabic{Rule19}}{The super fields can only be accessed after \code{super(\ldots)},
        and the fields of \this can only be accessed after \code{property(\ldots)}}
    }
\label{Figure:Properties}
\end{figure}


When using the prefix \this, you can access both properties and other final fields.
The difference between \this and \code{self} is
    shown in field \code{f3} in \Ref{Figure}{Properties}:
    \code{this.a} refers to the property \code{a} stored in \this,
    whereas \code{self.a} refers to \code{a} stored in the object to which \code{f3} refers.
(In the constructor, we indeed see that we assign to \code{f3} a new instance of \code{A}
    whose \code{a} property is equal to \code{this.a}.)

% Should I talk about interface and abstract property methods? Doesn't relate to initialization...

Properties must be initialized before other fields because
    field initializers and field types can refer to properties (see initializer for \code{f1} and the type of \code{f3}).
The super's fields can be accessed after the super call,
    and the other fields after the property call;
    field initializers are executed after the property call.

The \emph{class invariant} (\code{\lb{}b==a\rb} in \Ref{Figure}{Properties})
    may refer only to properties,
    and it must be satisfied after the property call (rule~\userule{Rule18}).
%For example, \code{new B(1,1)} is ok,
%    but \code{new B(1,2)} is rejected.



\subsection{Closures}
Closures are functions that can refer to final variables in the enclosing scope,
    e.g., they can refer to final method parameters, locals, and \this.
When a closure refer to a variable, we say that it is captured by the closure,
    and the variable is thus stored in the closure object.
Closures interact with initialization when they capture \this during construction.

\begin{figure}
%    LeakIt.foo(closure1);
\begin{lstlisting}
class A {
  var a:Int = 3;
  def this() {
    val closure1 = ()=>this.a; //err
    at(here.next()) closure1();
    val local = this.a;
    val closure2 = ()=>local;
  }
}
\end{lstlisting}
\definerule{Rule20}
\caption{Closures example.
    \myrule{\arabic{Rule20}}{A closure cannot capture a raw \this}
    }
\label{Figure:Closures}
\end{figure}


\Ref{Figure}{Closures} shows why it is prohibited to capture a raw \this in a closure:
    that closure can later escape to another place which will serialize all captured variables
    (including the raw \this, that should not be serialized, see \Ref{Section}{Multiple-Places}).
The work-around for using a field in a closure is to define a local that will refer only to the field (which is definitely cooked)
    and capture the local instead of the field as done in \code{closure2}.



\subsection{Generics and Structs}
\label{Section:Generics-and-Structs}
\emph{Structs} in X10 are header-less inlinable objects
    that cannot inherit code (i.e., they can \emph{implement} interfaces, but cannot \emph{extend} anything).
Therefore an instance of a struct type has a known size and it can be inlined in a containing object.
Java's primitive types (\code{int}, \code{byte}, etc) are represented as structs in X10.
Structs, as opposed to classes, do not contain the value \code{null}.

\emph{Generics} in X10 are reified, i.e, not erased as in Java.
For example, instances of \code{Box[Byte]} and \code{Box[Double]}
    would have the same size in Java but different sizes in X10.
%an instance of type \code{Box[Byte]} would have a different size than one of type \code{Box[Byte]}
Although generics are not a new concept,
    the combination of generics and the lack of default values
    leads to new pitfalls.
For example, in Java and C\#, it is possible to define an equivalent to

~~~~~~~\code{class A[T] \lb{} var a:T; \rb}\\
However, this is illegal in X10 because we cannot be sure that \code{T} has-zero (see \Ref{Figure}{Constraints}),
    e.g., if the user instantiate \code{A[Int\lb{}self!=0\rb]} then field \code{a} cannot be assigned a zero value
    without violating type-safety.
Therefore X10 has a type predicate written \code{X haszero} that returns true if type \code{X} has-zero.
Using \code{haszero} in a constraint (e.g., in a class invariant or a method guard),
    it is possible to guarantee a type-parameter will be instantiated by a type that has-zero.

\begin{figure}
\begin{lstlisting}
class B[T] {T haszero} {
  var f1:T;
  val f2 = Zero.get[T]();
}
struct WithZeroValue(x:Int,y:Int) {}
struct WithoutZeroValue(x:Int{self!=0}) {}
class Usage {
  var b1:B[Int];
  var b2:B[Int{self!=0}]; //err
  var b3:B[WithZeroValue];
  var b4:B[WithoutZeroValue]; //err
}
\end{lstlisting}
\definerule{Rule21}
    \caption{\code{haszero} type predicate example.
    \myrule{\arabic{Rule21}}{A type must be consistent, i.e., it cannot contradict the environment; the environment includes final variables in scope, method guards, and class invariants.}
    }
\label{Figure:Generics}
\end{figure}



\Ref{Figure}{Generics} shows an example of a generic class \code{B[T]}
    that constrains the type-parameter \code{T} to always have a zero value.
According to rule~\userule{Rule10}, field \code{f1} has an implicit zero field initializer.
It is also possible to write the initializer explicitly (as done in field \code{f2}) by using the static method \code{Zero.get[X]()}
    (that is guarded by \code{X haszero}).
Next we see two struct definitions:
    the first has two properties that has-zero,
    and the second has a property that does not have zero.
According to the definition of has-zero in \Ref{Figure}{Constraints},
    a struct has-zero if all its fields has-zero,
    therefore \code{WithZeroValue haszero} is true, but
    \code{WithoutZeroValue haszero} is false.
Finally, we see an example of usages of \code{B[T]},
    where two usages are legal and two are illegal
    (see rule~\userule{Rule21}).







We now turn our attention to the parallel features of X10:
    concurrent programming (\code{finish} and \code{async})
    and distributed programming (\code{at} and global references).
\Ref{Section}{Parallelism} already explained how parallel code is written in X10,
    and what are the common pitfalls of initialization in parallel code.
Next we present the rules that prevent these pitfalls.

\subsection{Concurrent programming and Initialization}


\begin{figure}
\begin{lstlisting}
class A {
  var f1:Int;
  val f2:Int;
  val f3:Int;
  def this() {//err: f2 was not definitely assigned
    async f1 = 1;
    async f2 = 2;
    finish {
      async f3 = 3;
    }
  }
}
\end{lstlisting}
\definerule{Rule13}
\definerule{Rule13a}
\caption{Concurrency in initialization example: asynchronously assign to a field.
    \myrule{\arabic{Rule13}}{A constructor must finish assigning to all fields at least once}
    \myrule{\arabic{Rule13a}}{A final field can be assigned at most once}
    %\myrule{\arabic{Rule13}}{All field assignments must finish when the constructor ends}
    }
\label{Figure:Asynchronously-init}
\end{figure}


\Ref{Figure}{Asynchronously-init} shows how to asynchronously assign to fields.
Recall that we wish to guarantee that one can never read an uninitialized field,
    therefore rule~\userule{Rule13} ensures that all fields are assigned at least once.

All three fields in \code{A} are asynchronously assigned,
    however, only \code{f2} is not definitely assigned.
Assigning to \code{f3} has an enclosing \code{finish}, so
    it is definitely assigned.
Field \code{f1} is also definitely assigned, because from rule~\userule{Rule10} it has an implicit zero field initializer.
However, field \code{f2} is only asynchronously assigned,
    and the constructor does not wait for this assignment to finish,
    thus violating rule~\userule{Rule13}.
(The exact data-flow analysis to enforce rule~\userule{Rule13} is described in
    \Ref{Section}{Read-write-rules}.)
Rule~\userule{Rule13a} is the same as in Java: a final field is assigned \emph{at most} once
    (and combined with rule~\userule{Rule13} we know it is assigned \emph{exactly} once).


\subsection{Distributed programming and Initialization}
\label{Section:Multiple-Places}
X10 programs can be executed on a distributed system with multiple places
    that have no shared-memory.
Therefore, objects are copied from one place to another when captured by an \code{at}.
Copying is done by first serializing the object into a buffer,
    sending the buffer to the other place, and then de-serializing the buffer at the other place.
As in Java, one can write custom serialization code in X10 by implementing \code{CustomSerialization},
    and defining the method \code{serialize():SerialData} and the constructor \code{this(data:SerialData)}.


\begin{figure}
\begin{lstlisting}
class A {
  val f:Int;
  def this() {//err: f was not definitely assigned
    // Execute at another place
    at (here.next())
      this.f = 1; //err: this escaped
  }
}
\end{lstlisting}
\definerule{Rule14}
% todo: Support remote-initialization (at and at back to init a field of this)
\caption{Distributed initialization example. %only cooked objects can be copied.
    \myrule{\arabic{Rule14}}{a raw \this cannot be captured by an \code{at}}
    }
\label{Figure:Multi-place}
\end{figure}


\Ref{Figure}{Multi-place} shows a common pitfall
    where a raw \this escapes to another place,
    and the field assignment would have been done on a copy of \this.
We wish to de-serialize only cooked objects,
    and therefore rule~\userule{Rule14} prohibits \this to be captured by an \code{at}.
Consequently, we also report that field \code{f} was not definitely assigned.


%\begin{lstlisting}
%class A {
%  var i:Int;
%  val distArray = DistArray.make( ..., (Point)=>this.i); // "this" is serialized to another place before it is cooked
%}
%\end{lstlisting}

\subsection{Global references}
\label{Section:Global-references}
A distributed data-structures is dispersed over multiple places,
    and it is convenient to have pointers from one place to an object in another place.
These cross-places pointers are called \emph{global references}.
A global reference has two fields: \code{object} that points to some object,
    and \code{home} which is the place where the global reference was created.
When a global reference is serialized, we serialize its \code{home} and the value of the \emph{pointer} to the \code{object}
    (we do not serialize the \code{object}).
%Phrased differently, serializing an object will recursively serialize all its fields;
%    the recursion ends when there are no fields or with global references.
%Retrieving the object is only allowed at place \code{home}.
%    i.e., the apply method returns the root and is guarded by \code{home==here}.
For example, suppose that \code{o} is some object.
Then, when a box pointing to \code{o} is serialized, then \code{o} is recursively serialized.
However, when a global reference pointing to \code{o} is serialized, then only the pointer to \code{o} is serialized (not \code{o} itself).
%  assert r()==o;
%def copyExample(o:Any, p:Place) {
%\begin{lstlisting}
%  val box = new Box(o);
%  at (here.next()) { // copies box and o
%    val x = box;
%  }
%  val r = new GlobalRef(o);
%  at (here.next()) { // copies r but not o
%    val x = r;
%  }
%\end{lstlisting}

% No room to talk about custom serialization:
% Places require serialization and deserialization (both custom and automatic) across "at".

A \emph{global object} has mutable state in a single place~\code{p}
    and methods that can be called from any place that mutate state in~\code{p}.
The common \emph{global object idiom}
    uses a global reference
    to point to a single mutable object.
\Ref{Figure}{GlobalRef} shows a global counter object
    that has mutable state (\code{count}) in a single place,
    but any place can increment the counter by incrementing \code{count} at that single place,
    which is \code{root.home}.
Note how \code{root()} returns the referent of the global reference.
(The call \code{root()} is guarded by \code{root.home==here}, which is statically verified in this code.)


%  def me() = root();
%}
%class B extends A {
%  def this() {
%    val alias = me(); //err
%  }
%}
%If \code{me()} was prefixed with
%\code{@NonEscaping public final}
%then accessing \code{root} would be an error.
%Cannot use 'root' because a GlobalRef[\ldots](this) cannot be used in a field initializer, constructor, or methods called from a constructor.

\begin{figure}
\begin{lstlisting}
class GlobalCounter {
  private val root = new GlobalRef(this);
  transient var count:Int;
  def inc() {
    return at(root.home) root().count++;
  }
}
\end{lstlisting}
\definerule{TransientRule}
\caption{A global counter example.
    \textbf{Revision of~\userule{Rule3}:}
        A raw \this can only escape to a global ref constructor in a private field initializer;
            that field cannot be read via a raw \this receiver.
    \myrule{\arabic{TransientRule}}{The type of a \code{transient} field must satisfy \code{haszero}}
    }
\label{Figure:GlobalRef}
\end{figure}

Note that a raw \this leaked into the constructor of \code{GlobalRef}
    which is a violation of rule~\userule{Rule3}.
Because this idiom is common in distributed programming,
    we relaxed this rule and allow \this to escape into a global ref,
    but only in a private field initializer that is not read during construction.

As in Java, the \code{transient} keyword marks a field that should not be serialized.
Upon de-serialization, such a field has the zero value.
Therefore, the type of a transient field must have the zero value (rule~\userule{TransientRule}).

We finally note that the global object idiom is error-prone because
    it is easy to forget to use \code{root()} before accessing a mutable field.
There is an RFC that suggests an annotation that will automatically convert a class to a global class
    using this pattern.





\subsection{Read and write of fields}
\label{Section:Read-write-rules}


\begin{figure}
\begin{lstlisting}
class A {
  val a:Int;
  def this() {
    readA(); //err1
    finish {
      async {
        a = 1; // assigned={a}
        readA();
      } // asyncAssigned={a}
      readA(); //err2
    } // assigned={a}
    readA();
  }
  // reads={a}
  private def readA() {
    val x = a;
  }
}
class B {
  var i:Int{self!=0}, j:Int{self!=0};
  def this() {
    finish {
     asyncWriteI(); // asyncAssigned={i}
    } // assigned={i}
    writeJ();// assigned={i,j}
    readIJ();
  }
  // asyncAssigned={i}
  private def asyncWriteI() {
    async i=1;
  }
  // reads={i} assigned={j}
  private def writeJ() {
    if (i==1) writeJ(); else this.j = 1;
  }
  // reads={i,j}
  private def readIJ() {
    val x = this.i+this.j;
  }
}
\end{lstlisting}
\definerule{Rule11}
\definerule{Rule11b}
\caption{Read-Write order for fields.
    We infer for each method three sets:
        (i)~fields it reads (i.e., these fields must be assigned before the method is called),
        (ii)~fields it assigned,
        (iii)~fields it asynchronously assigned.
    The flow maintain similar three sets before and after each statement;
        \code{\itshape assigned} becomes \code{\itshape asyncAssigned} after an \code{async},
        and \code{\itshape asyncAssigned} becomes \code{\itshape assigned} after a \code{finish}.
    In the example, we omitted empty sets.
    \myrule{\arabic{Rule11}}{A field can be read only if it is definitely-assigned}
    \myrule{\arabic{Rule11b}}{A final field can be written only if it is definitely-unassigned}
    }
\label{Figure:Read-Write-Order}
\end{figure}

We now present a data-flow analysis for guaranteeing
    that a field is read only after it was written,
    and that a final field is assigned exactly once.
Java performs an \emph{intra}-procedural data-flow analysis in \emph{constructors} to calculate
    when a \emph{final} field is definitely-assigned and definitely-unassigned.
In contrast, X10 performs an \emph{inter}-procedural (fixed-point) data-flow analysis
    in all \emph{non-escaping methods} to calculate
    when {a} field (\emph{both final and non-final}) is
    definitely-assigned, \emph{definitely-asynchronously-assigned}, and definitely-unassigned.
The details are explained using examples (\Ref{Figure}{Read-Write-Order}) by comparison with Java;
    the full analysis is described in X10's language specification.

X10 and Java allows \emph{writing} to a final field only when it is definitely-\emph{unassigned},
    and it allows \emph{reading} from a final field only when it is definitely-\emph{assigned}.
X10 also has the same read restriction on non-final fields
    (recall that rule~\userule{Rule10} adds a field initializer if the field's type has-zero).


Consider first only \emph{final} fields.
    They are easier to type-check because they can only be assigned in constructors.
X10 extends Java rules,
    by calculating for each non-escaping method \code{m} the set of final fields it reads,
    and calling \code{m} is legal only if these fields have been definitely assigned.
For example, in class \code{A}, method \code{readA} reads field \code{a}
    and therefore cannot be called before \code{a} is assigned (e.g., \code{\itshape err1}).
Note that Java does not perform this check, and it is legal to call \code{readA}
    which will return the zero value of \code{a}.
X10 also adds the notion of \emph{definitely-asynchronously-assigned}
    which means a field was definitely-assigned within an \code{async}
    (so it cannot be read, e.g., \code{\itshape err2}),
    but after an enclosing \code{finish} it will become definitely-assigned
    (so it can be read).
The flow maintains three sets:
    \code{\itshape reads}, \code{\itshape assigned}, and \code{\itshape asyncAssigned}.
If a constructor
    reads an uninitialized field, then it is an error;
    however, if a method reads an uninitialized field, then we add it to its \code{\itshape reads} set.
Phrased differently, the \code{\itshape reads} set of a constructor must be empty.

Now consider non-final fields.
    They can be assigned and read in methods,
        thus requiring a fixed-point algorithm.
For example, consider method \code{writeJ}.
Initially, \code{\itshape reads} is empty,
    while \code{\itshape assigned} and \code{\itshape asyncAssigned} are the entire set of fields.
In the first iteration, we add \code{i} to \code{\itshape reads},
    and when we join the two branches of the \code{if},
    \code{\itshape assigned} is decreased to only \code{j}.
The fixed-point calculation, in every iteration, increases \code{\itshape reads}
    and decreases \code{\itshape assigned} and \code{\itshape asyncAssigned},
    until a fixed-point is reached.



\subsection{Virtues and attributes of initialization in X10}
We assume there is a single \this variable, because all nested classes were converted to static,
    as described in \Ref{Section}{Inner}.
Therefore, initialization in X10 has the following attributes:
(i)~\this (and its alias \code{super}) is the only accessible raw object in scope (rule~\userule{Rule3}),
(ii)~only cooked objects cross places (rule~\userule{Rule14}),
(iii)~only \code{@NoThisAccess} methods can be dynamically dispatched during construction (rule~\userule{Rule5}),
(iv)~all final field assignments finish by the time the constructor ends (rule~\userule{Rule13}),
(v)~it is not possible to read an uninitialized field (rule~\userule{Rule11}), and
(vi)~reading a final field always return the same value (rule~\userule{Rule11b} combined with attribute~(v)).


\vspace{-0.3cm}
\Section[designs]{Alternative Initialization Designs}


\subsection{Default values design}

Java first initializes fields with either $0$, \texttt{false}, or \texttt{null}
(depending on the field type) and then running the constructor to initialize
the fields according to the programmer's wishes.  If every X10 type had a
default value that was statically known, then Java's object initialization
scheme could be used in X10.  This would have the advantage of familiarity for
Java programmers that are learning X10.  The disadvantages are that that it is
nonintuitive that final fields can be observed to change value, and that it is
prone to undetectable errors where the field is read before initialization.

Unfortunately it is hard to reconcile the notion of a default value with X10's
type system, because a programmer can define a type which does not contain the
default value.  In the X10 type system, one can define a type with no values at
all, by using a constraint that yields contradiction.

This could be addressed by extending the X10 type to require that the
programmer define a new constant value for any type that has been constrained
enough that the original default value is no longer a member of the type.  This
means every field can be initialized to the value defined in its type.  The
disadvantage of this is that the type system becomes more complex and more type
annotations are required.  We decided that this, in combination with the
disadvantages given above, was too problematic to justify the advantages of
Java-style object initialization.

\subsection{Proto Design}

If we want to allow some of the programs that the Hardhat design rejects, such
as immutable cycles in the object graph, but we do not want to burden the type
system with default value annotations, then one solution is to allow
\texttt{this} to escape in certain cases while still preventing reads from
uninitialized fields.  This can be achieved by annotating reference types with
a keyword \texttt{proto} to indicate that the referenced object is partially
constructed.  Reads of fields where the target object is \texttt{proto}
are not allowed because a partially constructed object may not yet have
initialized its fields.  The advantage of this approach is that it allows a set
of partially constructed objects to establish themselves as a mutually
referential cycle of objects in the heap, which would not otherwise be possible.
The disadvantage is that it requires an additional type annotations, although this
annotation is only required in code that creates immutable cyclic heap
structures.  Also note that there are no additional space or runtime overheads
since these extra type system mechanisms are for static checking only.

An example of an immutable cycle of two nodes is given in
fig.\ref{Figure:Cyclic}.  A more practical but less concise example would be an
immutable doubly-linked list.  Let us assume that we would like to optimize
away any null pointer checks, so we constrain all references to exclude the
null value.  The commented out lines indicate code that would be rejected by
the type system.

\begin{figure}
\begin{lstlisting}
class C {
  val next : C {self!=null};
  var fld : C;
  def this(n:proto C{self!=null}) {
    //Console.OUT.println(n.next); //err1
    //n.f(); //err2
    this.next = n;
  }
  def this() {
    //Console.OUT.println(this.next); //err3
    //this.f(); //err4
    val c = new C(this);
    //Console.OUT.println(c.next.next); //err5
    this.next = c;
  }
  def f() {
    Console.OUT.println(this.next);
  }
  def this(c:C, Int) {
    //c.m(this); //err6
    Console.OUT.println(c.fld.next);
    this.next = c;
  }
  void m (n : proto C) proto {
    this.fld = n;
  }
  static def test() {
    val c:C{self!=null} = new C();
    assert c.next.next==c;
  }
}
\end{lstlisting}
\caption{An immutable cycle of heap references, using \texttt{proto}.}
\label{Figure:Cyclic}
\end{figure}

In the public constructor, \texttt{this} is a pointer to a partially
constructed object.  If the type of \texttt{this} were to be explicit, it would
be \texttt{proto C \{self!=null\}}.  The \texttt{proto} element of the type
forbids any field reads.  It also prevents the reference being leaked (e.g.
into \texttt{f()}), except into variables of \texttt{proto} type where it
follows that there is protection from uninitialized field reads.

The private constructor's \texttt{n} parameter takes a \texttt{proto} pointer
to the original \texttt{C} instance.  It is limited in what it can do with
\texttt{n}, e.g. it cannot read \texttt{n.next}, but it can initialize its own
\texttt{next} field with the passed-in value.

When the public constructor returns, both objects are fully constructed with
all fields initialized.  Thus, the type of the variable \texttt{c} does not
have a proto annotation and the field read \texttt{c.next} is allowed.

If a type has the \texttt{proto} keyword, then its fields (both var and val)
may have partially constructed objects assigned to them, but fields may not be
read.  Conversely, the absence of \texttt{proto} means that the fields may be
read but var fields may not have partially constructed objects assigned to them.
This means that \texttt{proto C} and \texttt{C} are not related by sub-typing.
In other words, \texttt{proto C} means definitely partially constructed and
\texttt{C} means definitely fully constructed.  Consequently it makes no sense
to allow casting between the two types, and one may not extend or implement a
proto type.  The only sources of \texttt{proto} typed objects are via the
\texttt{this} keyword in a constructor and via method parameters of
\texttt{proto} type.  The only way a type can lose its \texttt{proto} is by
becoming fully constructed.

Consider \code{\itshape err5} in fig.~\ref{Figure:Cyclic}.
If we had inferred the type of \code{c} to be non-proto,
    then we could have read the uninitialized field \code{this.next}.
To solve this problem, we must ensure that the whole cycle becomes fully
constructed together.
This can be arranged by changing the type of \texttt{new
C(...)} to be \texttt{proto C} if one of its arguments is of \texttt{proto}
type.  This does not affect the assignment \texttt{this.next = c} because
\texttt{this} is \texttt{proto}.

We do not allow fields to have \texttt{proto} type.  This is because the
referenced object will eventually be fully constructed and then there would be
a variable of \texttt{proto} type pointing to a fully constructed instance.
This admits the possibility of someone assigning a partially constructed object
to a field of the fully constructed object, just as was done in the private
constructor in fig.~\ref{Figure:Cyclic}.  Then, one could accidentally read an
uninitialized field from the partially constructed object by going through the
fully constructed object.  Disallowing \texttt{proto} in fields avoids this
problem.  However local variables are safe because of lexical scoping, they
will go out of scope before the constructor returns and the object becomes
fully constructed.

If one examines the state of the heap as these examples execute, there can be
seen to be a subgraph of 2 partially-constructed objects, which is completely
isolated except for references from the stack of the thread which is executing
the constructors.  The type rules described hitherto ensure the heap has the
property that partially constructed objects are isolated from the rest of the
heap until their construction is completed.

We are not aware of any utility in throwing or catching \texttt{proto} types so
we avoid issues relating to partially constructed objects escaping via the
exception mechanism, by simply forbidding the throwing and catching of
\texttt{proto} exceptions.

There would be an issue calling other instance methods on \texttt{this} from a
constructor, because the type of \texttt{this} in those methods would need to
be \texttt{proto} since the target is still partially constructed.  We support
this by allowing the \texttt{proto} keyword to also be used on a method as an
effect annotation, i.e. it must be preserved by inheritance.  Such methods are
called \texttt{proto} methods and can be called on partially constructed
targets.  The type of \texttt{this} then subjects the body of the method to the
same restrictions as we have already seen in constructor bodies.

However in some cases we would like to avoid code duplication by allowing some
methods to be callable on both \texttt{proto} and non-\texttt{proto} targets.
This violates our principle that the two kinds of objects enjoy different
privileges and are completely distinct.
The error \code{\itshape err6} in fig.~\ref{Figure:Cyclic}
shows how we could potentially read an uninitialized field if we allowed this
relaxation.

To address this, we only allow the method to be called on non-\texttt{proto}
targets if there are no \texttt{proto} parameters to the method.  No such
parameters means the only partially constructed object in scope is
\texttt{this}.  In the case where the method is called on a non-\texttt{proto}
target there is therefore no partially constructed object in scope, and no harm
can be done.

While we believe this type system is correct and usable for writing real
programs in the X10 language, we had to decide whether the additional type
system complexity and annotations were a reasonable price to pay for the
additional expressiveness (i.e. the ability to construct immutable heap
cycles).  We ultimately decided that immutable heap cycles are too rare in
practice to justify including these extra mechanisms in the language.


%\Section[implementation]{Implementation}
%implementation design, overheads, some measurements, etc.

outlines our implementation within the X10 compiler using the polyglot framework,
    the compilation time overhead of checking these initialization rules,
    and the annotation overhead in our X10 code base.


Due to page limitation, we mainly focused on the formal effect system for POPL,
but we can easily add an empirical evaluation section that describes some test cases (where minor code refactoring was needed) and shows the annotation burden.
X10 has only two possible method annotations: @NonEscaping, @NoThisAccess.
Methods transitively called from a constructor are implicitly non-escaping (but the compiler issues a warning that they should be marked as @NonEscaping).
SPECjbb and M3R are closed-source whereas the rest is open-source and publicly available at x10-lang.org

------------------------------------------------------------------------------
Programs:           XRX SPECjbb     M3R UTS Other
# of lines          27153   14603       71682   2765    155345
# of files          257 63      294 14  2283
# of constructors       276 267     401 23  1297
# of methods            2216    2475        2831    124 8273
# of non-escaping methods   8   38      34  3   83
# of @NonEscaping       7   7       13  1   62
# of @NoThisAccess      1   0       1   0   12
------------------------------------------------------------------------------
XRX: X10 Runtime (and libraries)
SPECjbb: SPECjbb from 2005 converted to X10
M3R: Map-reduce in X10
UTS: Global load balancing
Other: Programmer guide examples, test suite, issues, samples
------------------------------------------------------------------------------

As can be seen, the annotations burden is minor.

Asynchronous initialization was not used in our applications because they pre-date this feature.
(It is used in our examples and tests 50 times.)
However, it is a useful pattern, especially for local variables.
More importantly, the analysis prevents bugs such as:
val res:Int;
finish {
  async {
    res = doCalculation();
  }
  // WRONG to use res here
}
// OK to use res here

Here are two examples for the use of annotations:
1) In Any.x10 we have:
@NoThisAccess def typeName():String
Method typeName is overridden in subclasses to return a constant string (all structs automatically override this method).
This annotation allows typeName() to be called even during construction.
2) In HashMap.x10, after we added the strict initializations rules, we had to refactor put and rehash methods into:
public def put(k: K, v: V) = putInternal(k,v);
@NonEscaping protected final def putInternal(k: K, v: V) {...}
(Similarly, we have rehash() and rehashInternal())
The reason is that putInternal is called from the deserialization constructor:
def this(x:SerialData) { ... putInternal(...) ... }
And we still want subclasses to be able to override the "put" method.


%\Section[case-study]{Case Study}
%a lot of Java code was recently translated to X10, and Java is less strict regarding initialization. How did affect the translation?



\Section[formalism]{Formalism: FX10}
Featherweight X10 (FX10) is a formal calculus for X10 intended to  complement Featherweight Java
(FJ).  It models imperative aspects of X10 including the concurrency
constructs \hfinish{} and \hasync{}.
FX10 models the heart of the field initialization problem:
    a field can be read only after it  is definitely assigned.

%    and that final (\code{val}) fields can be assigned exactly once (in \Ref{Section}{val}).

The basic idea behind the formalization is very straightforward. We
break up the formalization into two distinct but interacting
subsystems, a {\em type system}~(\Ref{Section}{Type system}) and an {\em
  effect} system~(\Ref{Section}{Effect system}). The type
system is completely standard -- think the system of FJ, adapted to
the richer constructs of FX10.

The effect system is built on a very simple {\em logic of
  initialization assertions}.  The primitive formula $+\hx$
($+\hp.\hf$) asserts that the variable $\hx$ (the field $\hf$ of
$\hp$) is definitely initialized with a cooked object, and the formula $-\hx$
($-\hp.\hf$) asserts that it is being initialized by a concurrent
activity (and hence it will be definitely initialized once an
enclosing $\hfinish$ is crossed).
An {\em initialization formula}
$\phi$ or $\psi$ is simply a conjunction of such formulas $\phi \wedge
\psi$. % or an existential quantification $\exists{}\hx.\phi$.
An {\em
  effects assertion} $\phi~\hS~\psi$ (for a statement $\hS$) is read as a
partial correctness assertion: when executed in a heap~$H$ that satisfies
the constraint $\phi$, $\hS$ will on termination result in a heap~$H'$ that
satisfies $\psi$.  Since we do not model \code{null}, our
formalization can be particularly simple: variables, once initialized,
stay initialized, hence $H'$ will also satisfy $\phi$
    (see \Ref{Section}{Reduction} for a definition of heaps and when a heap satisfy~$\phi$).

Another feature of our approach is that, unlike Masked
Types~\cite{XinQi:2009}, the source program syntax does not permit the
specification of initialization assertions. Instead we use a standard
least fixed point computation to automatically decorate each method
$\hdef~\hm(\ol{\hx}:\ol{\hC})\{\hS\}$ with pairs $(\phi, \psi)$ (in the free
variables $\hthis, \ol{\hx}$) such that under the assumption that all
methods satisfy their corresponding assertion we can show that
$\phi~\hS~\psi$.\footnote{Note that this approach permits a formal $\hx$ to a
  method to be completely raw ($\phi$ does not entail $+\hx$ or
  $+\hx.\hf$ for any field $\hf$) or partially raw ($\phi$ does not
  entail $+\hx$ but may entail $+\hx.\hf$ for some fields $\hf$). As a
  result of the method invocation the formal may become more cooked.
  In X10, in order for the inference to be intra-class, we require that all method parameters~$\ol{\hx}$ (except \this) are cooked, i.e.,
    $+\hx_i$. In FX10 we are more relaxed and allow methods to receive and initialize raw parameters.
  }
This computation must be sensitive to the semantics
of method overriding, that is a method with decoration $(\phi,\psi)$
can only be overridden by a method with decoration $(\phi',\psi')$
that is ``at least as strong as'' $(\phi,\psi)$ (viz, it must be the
case that ($\phi \vdash \phi'$ and $\psi' \vdash \psi$). Further, if
the method is not marked \code{@NonEscaping}, then $\phi$ is required
to entail $+\hthis$ (that is, \code{this} is cooked), and if it marked
\code{@NoThisAccess} then $\phi,\psi$ cannot have \code{this} free.


\comment{{\sc We also have to check that if it is marked
    \code{@NonEscaping} then it does not escape.
    Yoav notes: if \this is not cooked then it cannot escape to the heap - only as a parameter to other methods.}}

By not permitting the user to specify initialization assertions we
make the source language much simpler than \cite{XinQi:2009} and
usable by most programmers. The down side is that some initialization
idioms, such as cyclic initialization, are not expressible.

For reasons of space we do not include the (standard) details behind
the decoration of methods with initialization assertions. We also omit
many extensions (such as generics, interfaces, constraints, casting,
inner classes, overloading, co-variant return types, final, field initializers etc.) necessary to establish the soundness
of initialization for the full language discussed in the first half of
the paper. FX10 also does not model places because the language design
decision to only permit cooked objects to cross places means that the
rules for \code{at} are routine.

We use the usual notation of~$\ol{\hx}$ to represent a vector or set of~$\hx_1, \ldots,\hx_n$.
A program~\hP is a pair of class declarations~$\ol{\hL}$ (that is assumed to be global information)
    and a statement \hS.
%Like in masked-types, there is no \hnull value in the language, because none is needed for object initialization.
%Phrased differently,
%    FX10 guarantees that a field is read only after assigned, so no need to initialize it with \hnull.
%In the formalism, an object is represented as a mapping from initialized fields to their values
%    (so initially the mapping is empty because no field is initialized).

\paragraph{Overview of formalism}
\Ref{Section}{Syntax} presents the syntax of FX10.
% todo
\Ref{Section}{Type system} briefly describes the type
system. \Ref{Section}{Effect system} defines the flow-sensitive rules for
    ($\phi~\hS~\psi$), while
    \Ref{Section}{Reduction} gives the reduction rules for statements
    and expressions ($\hS,H \reducesto \hS',H' ~|~H'$
    and~$\he,H \reduce \hl,H'$). Finally \Ref{Section}{Results}
    presents the main formal result (soundness theorem).
%Finally, \Ref{Section}{val} extends the formalism with \hval and \hvar fields.


\Subsection{Syntax}

\begin{figure}[htpb!]
\begin{center}
\begin{tabular}{|l|l|}
\hline

$\hP ::= \ol{\hL},\hS$ & Program. \\

$\hL ::= \hclass ~ \hC~\hextends~\hD~\lb~\ol{\hF};~\ol{\hM}~\rb$
& cLass declaration. \\

$\hF ::= \hvar\,\hf:\hC$
& Field declaration. \\

$\hM ::= \hG\ \hdef\ \hm(\ol{\hx}:\ol{\hC}):\hC\{\hS\}$
& Method declaration. \\

$\hG ::= \code{@NonEscaping} ~~|~~ \code{@NoThisAccess}$
& Method modifier. \\

$\hp ::= \hl ~~|~~ \hx$
& Path. \\ %(location or parameter)

$\he ::=  \hp.\hf  ~|~ \hnew{\hC}$
& Expressions. \\ %: locations, parameters, field access\&assignment,  %invocation, \code{new}

$\hS ::=  \hp.\hf = \hp; ~|~ \hp.\hm(\ol{\hp});  ~|~ \hval{\hx}{\he}{\hS}$ &\\
$~~~~|~ \finish{\hS}~|~ \async{\hS} ~|~ \hS~\hS$
& Statements. \\ %: locations, parameters, field access\&assignment, invocation, \code{new}

\hline
\end{tabular}
\end{center}
\caption{FX10 Syntax.
    The terminals are locations (\hl), parameters and \hthis (\hx), field name (\hf), method name (\hm), class name (\hB,\hC,\hD,\hObject),
        and keywords (\hhnew, \hfinish, \hasync, \code{val}).
    The program source code cannot contain locations (\hl), because locations are only created during execution/reduction in \RULE{R-New} of \Ref{Figure}{reduction}.
    }
\label{Figure:syntax}
\end{figure}

\Ref{Figure}{syntax} shows the syntax of FX10.
%(\Ref{Section}{val} will later add the \hval and \hvar field modifiers.)
Expression~$\hval{\hx}{\he}{\hS}$ evaluates $\he$, assigns it to a
new variable $\hx$, and then evaluates \hS. The scope of \hx{} is \hS.

The syntax is similar to the real X10 syntax with the following difference:
%Non-escaping methods are marked with \code{@NonEscaping}, such methods
%can be invoked on raw objects (and can be used to initialize them).
FX10 does not have constructors; instead, an object is initialized by assigning to its fields or
    by calling
    non-escaping methods.

%To simplify presentation, we allow only paths~$\hp$
%    (variables \hx at
%    compile time, or heap locations \hl at run time) to appear in field access, assignment and method call.
% Phrased differently, the user must flatten the program.
%This does not restrict expressiveness, because of \code{val} expressions.

\Subsection{Type system}
%\vspace{-0.2cm}\begin{lstlisting}
%class D extends Object { f:Object; }
%val d:D = new D; d.f = new Object
%\end{lstlisting}\vspace{-0.2cm}
%Then, the type of \code{d} is \code{D} where field \code{f} is uninitialized
%    (which we formally write below as $\code{D}\myinit{\hf}{\hf}$).
%Therefore, reading \code{d.f} would be illegal, but writing to \code{d.f} would update the type of \code{d}
%    to be fully initialized.

The type system for FX10 checks that every parameter and variable has
a type (a type is the name of a class), and that a variable of type
\hC can be assigned only expressions whose type is a subclass of \hC,
and can only be the receiver of invocations of methods defined in
\hC. The type system is formalized along the lines of FJ. No
complications are introduced by the extra features of FX10 --
assignable fields, local variable declarations, \hfinish{} and
\hasync{}. We omit details for lack of space and because they are
completely routine.

In the rest of this section we shall assume that the program being
considered $\ol{\hL},\hS$ is well-typed.

\Subsection{Effect system}
\begin{figure*}[t]
\begin{center}
\begin{tabular}{|c|}
\hline

$\typerule{
 \phi \vdash +\hp.\hf \gap \phi, +\hx~\hS~\psi
}{
 \phi~\hval{\hx}{\hp.\hf}{\hS}~\psi
}$~\RULE{(T-Access)}
\quad
$\typerule{
  \phi~\hS~\psi
}{
 \phi~ \hval{\hx}{\hnew{\hC}}{\hS}~\psi
}$~\RULE{(T-New)}
\quad
$\typerule{
  \phi \vdash +\hq
}{
 \phi ~ \hp.\hf=\hq ~ + \hp.\hf
}$~\RULE{(T-Assign)}
\\\\

$\typerule{
    \phi~\hS~\psi
}{
  \begin{array}{l}
    \phi~\finish{\hS}~ +\psi \\
    \phi~\async{\hS}~ -\psi
  \end{array}
}$~\RULE{(T-Finish,Async)}

\quad
$\typerule{
  \phi~\hS_1~\psi_1
        \gap
    \phi,\psi_1~\hS_2~\psi_2
}{
  \phi~\hS_1~\hS_2~\psi_1,\psi_2
}$~\RULE{(T-Seq)}
\quad

$\typerule{
\hm(\ol{\hx}):: \phi' \Rightarrow \psi' \gap \phi \vdash \phi'[\hp/\hthis,\ol{\hp}/\ol{\hx}]
}{
\phi~\hp.\hm(\ol{\hp})~\psi'[\hp/\hthis,\ol{\hp}/\ol{\hx}]
}$~\RULE{(T-Invoke)}\\

\hline
\end{tabular}
\end{center}
\caption{FX10 Effect System ($\phi~\hS~\psi$)}
\label{Figure:effects}
\end{figure*}

We use a simple logic of initialization for our basic assertions.
This is an intuitionistic logic over the
primitive formulas $+\hp$ (the variable or parameter $\hp$ is
initialized), $+\hp.\hf$ (the field $\hp.\hf$ is initialized), and
$-\hp,-\hp.\hf$ (it is being \emph{concurrently}
initialized). We are only concerned with conjunctions and existential
quantifications over these formulas:
$
 \phi,\psi {::=}  \htrue ~|~ +\hx ~|~ +\hp.\hf ~|~ -\hp.\hf ~|~ \phi \wedge \psi
$

The notion of substitution on formulas $\phi\char`[\ol{\hx}/\ol{\hz}\char`]$ is
specified in a standard fashion.

The inference relation is the usual intuitionistic implication over
these formulas, and the following additional proof rules:
(1)~if $\phi\vdash +\hp$ then~$\phi\vdash -\hp$;
(2)~if $\phi\vdash +\hp.\hf$ then~$\phi\vdash -\hp.\hf$;
(3)~if $\phi\vdash +\hp$ ($\phi\vdash -\hp$) then~$\phi\vdash +\hp.\hf$ ($\phi\vdash -\hp.\hf$);
and (4)~if
the \emph{exact} class of $\hx$ is $\hC$, and $\hC$ has the fields $\ol{\hf}$,
then $\phi\vdash +\hp$ ($\phi\vdash -\hp$) if $\phi\vdash +\hp.\hf_i$ ($\phi\vdash -\hp.\hf_i$), for each $i$.
(We only know the \emph{exact} class for a local~\hx when~$\hval{\hx}{\hnew \hC}{\hS}$.)


The proof rules for the judgement $\phi~\hS~\psi$ are given in
Figure~\ref{Figure:effects}. They use two syntactic operations on initialization
formulas defined as follows.
$+\psi$ is defined inductively as follows:
$+\htrue=\htrue$,
$+\pm\hx=+\hx$,
$+\pm\hp.\hf=+\hp.\hf$,
$+(\phi \wedge\psi)=(+\phi)\wedge(+\psi)$.
$-\psi$ is defined similarly:
$-\htrue=\htrue$,
$-\pm\hx=-\hx$,
$-\pm\hp.\hf=-\hp.\hf$,
$-(\phi \wedge\psi)=(-\phi)\wedge(-\psi)$.

The rule~\RULE{(T-Access)} can be read as asserting: if $\phi$ entails
the field $\hp.\hf$ is initialized
(together with $+\hx$ which states that $\hx$ is initialized to a
cooked object), we can establish that execution of $\hS$ satisfies the assertion
$\psi$ then we can establish that execution of
$\hval{\hx}{\hp.\hf}{\hS}$ in
(a heap satisfying) $\phi$ establishes $\psi$. Here we
must take care to project $\hx$ out of $\psi$ since $\hx$ is not
accessible outside its scope $\hS$; similarly we must take care to
project $\hx$ out of $\phi$ when checking \hS.
The rule ~\RULE{(T-New)} can be read in a similar way except that when
executing $\hS$ we can make no assumption that $\hx$ is initialized,
since it has been initialized with a raw object (none of its fields
are initialized). Subsequent assignments to the fields of $\hx$ will
introduce effects recording that those fields have been initialized.
The rule~\RULE{(T-Assign)} checks that \hq{} is initialized to a
cooked object and then asserts that $\hp.\hf$ is initialized to a
cooked object.
The rule~\RULE{(T-Finish)} can be understood as recording that after a
\hfinish{} has been ``crossed'' all asynchronous initializations $\psi$
can be considered to have been performed $\phi$. Conversely,
the rule~\RULE{(T-Async)} states that any initializations must be
considered asynchronous to the surrounding context.
The rule~\RULE{(T-Seq)} is a slight variation of the stadard rule for
sequential composition that permits $\phi$ to be used in the
antecedent of $\hS_2$, exploiting monotonicity of effects. Note the
effects recorded for $\hS_1~\hS_2$ are a conjunction of the effects
recorded for $\hS_1$ and $\hS_2$.
The rule~\RULE{(T-Invoke)} is routine.

As an example, consider the following classes. Assertions are provided
in-line.
\vspace{-0.2cm}\begin{lstlisting}
class A extends Object {
  var f:Object; var g:Object; var h:Object;
  @NonEscaping def build(a:Object) {
    // inferred decoration: phi => psi
    //   phi= +this.g, +a
    //   psi= -this.h, +this.f
    // checks phi implies +this.g
    val x = this.g;
    async { this.h = x; } // psi= -this.h
    finish {
      // checks phi implies +a
      async { this.f = a; } // psi= -this.h,-this.f
    } // psi= -this.h,+this.f
  }
}
class B extends A { e:Object; }
\end{lstlisting}\vspace{-0.2cm}
Method \code{build} synchronously (asynchronously) initializes fields \code{this.f} (\code{this.h}),
    and it assumes that \code{this.g} and \code{a} are cooked.
The following statement completely initializes \code{b}:
\vspace{-0.2cm}\begin{lstlisting}
val b = new B();
val a = new Object(); // psi= +a
b.g = a; // psi= +a,+b.g
finish {
  b.build(a); // psi= +a,+b.g,+b.f,-b.h
  async { b.e = a; } // psi= +a,+b.g,+b.f,-b.h,-b,-b.e,-b
} // psi= +a,+b
\end{lstlisting}\vspace{-0.2cm}


\Subsection{Reduction}
A {\em heap}~$H$ is a mapping from a given set of locations to {\em
  objects}. An object is a pair $C(F)$ where $C$ is a class (the exact
class of the object), and $F$ is a partial map from the fields of $C$
to locations.
We say the object~\hl is {\em total/cooked} (written~$\cooked_H(\hl)$)
    if its map is total, i.e.,~$H(\hl)=\hC(F) \gap \dom(F)=\fields(\hC)$.

We say that a heap~$H$ {\em satisfies} $\phi$ (written~$H \vdash \phi$)
    if the plus assertions in~$\phi$ (ignoring the minus assertions) are true in~$H$,
    i.e., if~$\phi \vdash +\hl$ then~$\hl$ is cooked in~$H$
    and if~$\phi \vdash +\hl.\hf$ then~$H(\hl)=\hC(F)$ and~$F(\hf)$ is cooked in~$H$.


%An {\em annotation} $N$ for a heap $H$ maps each $l \in \dom(H)$ to a
%possibly empty set of fields $a(H(l))$ of the class of $H(l)$ disjoint
%from $\dom(H(l))$. (These are the fields currently being
%asynchronously initialized.) The logic of initialization described
%above is clearly sound for the obvious interpretation of formulas over
%annotated heaps. For future reference, we say that that a heap $H$
%{\em satisfies} $\phi$ if there is some annotation $N$ (and some
%valuation $v$ assigning locations in $\dom(H)$ to free variables of
%$\phi$) such that $\phi$ evaluates to true.

The reduction relation is described in
Figure~\ref{Figure:reduction}. An S-configuration is of the form
$\hS,H$ where \hS{} is a statement and $H$ is a heap (representing a
computation which is to execute $\hS$ in the heap $H$), or $H$
(representing terminated computation). An
E-configuration is of the form $\he,H$ and represents the
computation which is to evaluate $\he$ in the configuration $H$. The
set of {\em values} is the set of locations; hence E-configurations of
the form $\hl,H$ are terminal.

Two transition relations $\reduce{}$ are defined, one over S-configurations and
the other over E-configurations.
For $X$ a partial function, we use the notation $X[v \mapsto
  e]$ to represent the partial function which is the same as $X$
except that it maps $v$ to $e$.
The rules defining these relations are
standard.
The only minor novelty is in how \hasync{} is defined. The
critical rule is the last rule in~\RULE{(R-Step)}  -- it specifies the
``asynchronous'' nature of \hasync{} by permitting \hS{} to make a step
even if it is preceded by $\async{\hS_1}$.
%
The rule~\RULE{(R-New)} returns a new location that is bound to a new
object that is an instance of \hC{} with none of its fields initialized.
%
The rule~\RULE{(R-Access)} ensures that the field is initialized before it is
read ($\hf_i$ is contained in $\ol{\hf}$).

\begin{figure*}[t]
\begin{center}
\begin{tabular}{|c|}
\hline
$\typerule{
 \hS,H \reduce H'
}{
  \begin{array}{l}
    \finish{\hS},H \reduce H'\\
    \async{\hS},H \reduce H'\\
    \hS~\hS', H \reduce \hS',H'\\
  \end{array}
}$~\RULE{(R-Term)}
~
$\typerule{
 \hS,H \reduce \hS', H'
}{
  \begin{array}{l}
    \finish{\hS},H \reduce \finish{\hS'},H'\\
    \async{\hS},H \reduce \async{\hS'}, H'\\
    \hS~\hS_1, H \reduce \hS'~\hS_1,H'\\
    \async{\hS_1}~\hS, H \reduce \async{\hS_1}~\hS', H'\\
  \end{array}
}$~\RULE{(R-Step)}
~
$\typerule{
  \he,H \reduce \hl,H'
}{
  \hval{\hx}{\he}{\hS},H \reduce \hS[\hl/\hx], H'
}$~\RULE{(R-Val)}
\\\\

$\typerule{
    \hl' \not \in \dom(H)
}{
  \hnew{\hC},H \reduce \hl',H[ \hl' \mapsto \hC()]
}$~\RULE{(R-New)}
\quad
$\typerule{
    H(\hl')=\hC(\ldots)
        \gap
    \mbody{}(\hm,\hC)=\ol{\hx}.\hS
}{
  \hl'.\hm(\ol{\hl}),H \reduce \hS[\ol{\hl}/\ol{\hx},\hl'/\hthis],H
}$~\RULE{(R-Invoke)}
\quad

\\\\

$\typerule{
    H(\hl)=\hC(\ol{\hf}\mapsto\ol{\hl'})
}{
  \hl.\hf_i,H \reduce \hl_i',H
}$~\RULE{(R-Access)}
\quad
$\typerule{
    H(\hl)=\hC(F) \gap \cooked_H(\hl')
}{
  \hl.\hf=\hl',H \reduce H[ \hl \mapsto \hC(F[ \hf \mapsto \hl'])]
}$~\RULE{(R-Assign)}
\\
\hline
\end{tabular}
\end{center}
\caption{FX10 Reduction Rules ($\hS,H \reducesto \hS',H' ~|~H'$ and~$\he,H \reduce \hl,H'$).}
\label{Figure:reduction}
\end{figure*}


\Subsection{Results}

We say a heap $H$ is {\em correctly cooked} (written~$\vdash H$)
    if a field can point only to cooked objects, i.e.,~for every object $o=C(F)$ in the range of $H$ and
    every field $f \in \dom(F)$ it is the case that every object~$\hl=H(F(f))$ is cooked ($\cooked_H(\hl)$).
We shall only consider correctly cooked heaps (valid
programs will only produce correctly cooked heaps).
As the program is executed, the heap monotonically becomes more and more cooked.
Formally,~$H'$ is \emph{more cooked} than~$H$ (written~$H' \vdash H$)
    if for every~$\hl \in \dom(H)$, we have~$H(\hl)=\hC(F)$,~$H'(\hl)=\hC(F')$, and~$\dom(F)\subseteq\dom(F')$.

A {\em heap typing} $\Gamma$ is a mapping from locations to classes. $H$ is
said to be typed by $\Gamma$ if for each $l\in\dom(H)$, the class of $H(l)$
is a subclass of $\Gamma(l)$.  Since our treatment separates out effects
from types, and the treatment of types is standard, we shall assume
that all programs and heaps are typed.

A statement~$\hS$ is \emph{closed} (written~$\vdash \hS$) if it does not contain any free variables.
We say that $\hS$ is \emph{annotable} if there exists $\phi,\psi$ such that
$\phi~\hS~\psi$ can be established.\footnote{An example of a statement that is
{\em not} annotable is $\hval{\hx}{\hnew
  \hC}{\hval{\hy}{\hx.\hf}{\hz.\hg=\hy}}$ where $\hC{}$ has a field
\hf. This attempts to read a field of a variable initialized with a
brand-new object.}

We say that a program $\hP=\ol{\hL}\hS$ is {\em proper} if it is
well-typed and each method in \hL{} can be decorated with pre-post assertions
$(\phi,\psi)$, and \hS is annotable.
The decorations must satisfy the property that under the
assumption that every method satisfies its assertion (this is for use
in recursive calls) we can establish for every method
$\hdef~\hm(\ol{\hx}:\ol{\hC})\{\hS\}$ with assertion $(\phi,\psi)$ that
it is the case that the free variables of $\phi,\psi$ are contained in
$\this,\ol{\hx}$, and that $\phi~\hS~\psi$.

We prove the following theorems. In all these theorems the background
program \hP{} is assumed to be proper. The first theorem is analogous
to subject-reduction for typing systems.
\begin{Theorem}{\textbf{Preservation}}
%\label{Theorem:Preservation}
Let $\phi~\hS~\psi$, $\vdash \hS$, $\vdash H$, $H \vdash \phi$.
(a) If $\hS,H \reducesto H'$ then $\vdash H'$, $H' \vdash H$, $H' \vdash +\psi$.
(b) If $\hS,H \reducesto \hS',H'$ then $\vdash \hS'$, $\vdash H'$, $H' \vdash H$,
there exists $\phi',\psi'$ such
that $H' \vdash \phi'$, $\phi'~\hS'~\psi'$, $\phi' \vdash \phi$, $\psi' \vdash \psi$.
\end{Theorem}

\begin{Theorem}{\textbf{Progress}}
%\label{Theorem:Progress}
Let $\phi~\hS~\psi$, $\vdash \hS$, $\vdash H$, $H \vdash \phi$.
The configuration $\hS,H$ is not stuck.
\end{Theorem}

For proofs, please see associated technical report.

Because our reduction rules only allow reads from initialized fields,
a corollary is that a field can only be read after it was assigned,
and an attempt to read a field will always succeed.


\Section[related-work]{Related Work}
A static analysis \cite{Seo:2007:SBD:1522565.1522587}, has been used to find
some default value reads in Java programs, and supports our belief that default
value reads can be found in real programs and should be considered errors.  Our
approach is stronger (detecting all errors at the expense of some correct
programs) and considers additional language constructs that are not present in
Java.

There has been a study on a large body \cite{Gil:2009:WRS:1615184.1615216} of
Java code, showing that initialization order issues pervade projects from the
real world.  A bytecode verification system for Java initialization has also
been explored \cite{Hubert:2010:ESO:1888881.1888890}.

An early work to support non-null types in Java
\cite{Fahndrich:2003:DCN:949305.949332} has the notion of a type constructor
$raw$ that can be applied to object types and means that the fields of the
object (in X10 terminology) may violate the constraints in their types.  This
simply disables the type-system while an object is partially constructed while
ensuring the rest of the program is typed normally.  Our approach prevents
errors during constructors as it does not disable the type-system, and it also
permits optimisation of the representation of fields whose types are very
constrained, since they will never have to hold a value other than the values
allowed by their type constraint.

A later work \cite{Fahndrich:2007:EOI:1297027.1297052,XinQi:2009} allows
    to specify the time
    (in the type) when the object will be fully constructed.
Field reference types of a partially constructed objects must be fully
constructed by the same time, which allows graphs of objects to be constructed
like our \code{proto} design.  However the system is more complicated, allowing
the object to become fully constructed at a given future time, instead of at
the specific time when its constructor terminates or the last object that links
to it becomes fully constructed (whichever is later).
%We did need this functionality.

Expression of object constructedness has also been explored at the level of
individual fields \cite{XinQi:2009}.  Each type gives the fields which have not
yet been initialised.  Our type system is simpler but less expressive.
%as we only distinguish between the empty set of fields, and some non-empty set of fields.

Ownership types can be used to create immutable cycles~\cite{Zibin:2010:OIG:1869459.1869509}.
This is comparable to
our \code{proto} design because it also allows \code{this} to be linked from an
incomplete object.  However the ownership structure is used to implement a
broader policy, allowing code in the owner to use a reference to its partially
constructed children, whereas we only allow code to use a reference to objects
that are being partially constructed in some nesting stack frame.  However our
approach does not use ownership types.

There is also a time-aware type system \cite{Matsakis:2010:TTS:1869459.1869511}
that allows the detection of data-races, and understands the concept of shared
variables that become immutable only after a certain time
(and can then be accessed without synchronisation).  The same mechanisms can also be
used to express when an object becomes cooked.


\Section[conclusion]{Conclusion}
% 8/17/2011 5:03:31 AM

\subsubsection{Implementation}
An early version of this design was implemented in a branch of the X10
system. Results were obtained for about twenty benchmarks, including
JGF benchmarks such as IDEA, Sor, Series, RayTrace, LUFact,
SparseMatMul, Geometric Mean.  The speed of the programs written using
accumulators and clocked types was compared to the corresponding
programs written in plain X10. On the average we noticed about a 20\%
degradation in performance. 

A new implementation is currently being worked on and we expect to
have more up-to-date results for final submission. This implementation
rides piggyback on the control messages exchanged to implement
\code{finish} in order to update an \code{Acc} from multiple
places. The rules for reads on \code{Acc} guarantee that intermediate
accumulations can be stored at each place and do not need to be
communicated to the place where the \code{Acc} lives until local
computation has quiesced. 
This results in a very efficient implementation. 

\subsection{Adding static effects checking}

Safe X10 can be substantially enriched
through the addition of statically checked effects
\cite{Gifford:1986:IFI:319838.319848},\cite{DPJ}, particularly to
handle non-array-based computations. This section summarizes work
currently in progress.

X10 already implements a very powerful dependent type system based on
constraints. An object \code{o} is of type \code{T\{c\}} if it is of
type \code{T} and further satisfies the constraint
\code{c[o/self]}. Thus \code{Array[T]\{self.rank==R\}} is the type of all
the arrays of \code{T} with rank \code{R}.

Types classify objects, i.e.{} types specify sets of
objects. Therefore we can define a {\em location set} to be of the form
\code{T.f} where \code{T} is a type and \code{f} is a mutable field of
type \code{T}. It stands for the set of locations \code{x.f} where
\code{x} is of type \code{T}. We do not need to introduce any distinct
notion of regions into the language.

The key insight is that two memory locations are distinct if they
either have different names or they have the same name \code{f} and
they live in two objects \code{m} and \code{n} such that for some
property \code{p}, \code{m.p != n.p} -- for then it must be the case
that \code{m != n}. Therefore we introduce enough properties into
classes to ensure that we can distinguish between objects that are
simultaneously being operated on. Specifically, the location sets
\code{S\{c\}.f} and \code{S\{d\}.f} are disjoint if
\code{c[x/self],d[x/self]} is not satisfiable for any \code{x:S}.

Methods are decorated with \code{@Read(L)} and \code{@Write(M)}
annotations, where \code{L},\code{M} are location sets. The annotation
is {\em valid} if every read (write) of a mutable location \code{o.f} in the
body of the method it is the case that \code{o.f} lies in (the set
described by) \code{L} (\code{M}).

We permit location sets to be named:
\begin{lstlisting}
  static locs Cargos = Tree.cargo;
  static locs LeftCargo(up:Tree) = Tree{self.up==up,self.left==true}.cargo;
\end{lstlisting}

Unlike the region system of \cite{DPJ} we do not need to introduce
explicitly named, intensional regions, rather we can work with
extensional representations of location sets (the set of all locations
satisfying a certain condition).  There is no need for a separate
space of region names with constructors. Two location sets are
disjoint if their constraints are mutually unsatisfiable, not because
they are named differently. In particular, we do not need to assume
that the heap is partitioned into a tree of regions.

Two key properties of the X10 type system are worth recalling. The run-time
heap cannot contain cycles involving only properties. This in turn
depends on the fact that the X10 type system prevents \code{this} from
escaping during object construction\cite{X10-object-initialization}.

We mark a property as \code{@ghost} to indicate that space is not
allocated at run-time for this property. Therefore the value of this
property is not accessible at run-time -- it cannot be read and stored
into variables. It may only be accessed in constraints that are
statically checked. Dynamic casts cannot refer to \code{@ghost}
properties.

We enrich the vocabulary of constraints. First, we permit
existentially quantified constraints \code{x:T\^c} -- this represents
the constraint \code{c} in which the variable \code{x} of type
\code{T} is existentially quantified. Second, we permit the {\em
  extended field selector} \code{e.f\$i} where \code{e} is an
expression, \code{f} a field name and \code{i} is a \code{UInt}.

%% Alternatively we just permit regular expressions as field
%% selectors?
%% So take T{self.r==x} where r is a regular expression to mean
%% T{p:??^(p in r, self.p==x)}. So we have to have a notion of a
%% type-consistent path, and then be able to select a path on a given
%% receiver, maybe use some other operator for this, and not .?


%% Hmm dont want to be able to count and do arithmetic, that will
%% take us out of reg exp land.

%% Do we need MONA?
%% May need to permit full-fledged regular expressions.

% Where do we need the flexibility of saying two region parameters are disjoint?

We use an example from \cite{DPJ} to illustrate:
  \begin{lstlisting}
type Tree(t:Tree,l:Boolean)=Tree{self.up==t,self.left==l};
class Tree (up:Tree, left:Boolean) {
    var left:Tree(this,true);
    var right:Tree(this,false);
    var payload:Int;
    //SubTree(t) is the type of all Trees o s.t.
    // for some UInt i, o.up...up=t (i-fold iteration).
    static type SubTree(t:Tree)=Tree{i:UInt^self.up$i==t};
    @Safe
    def makeConstant(x:Int)
        @Write[SubTree(this)].payload
        @Read[SubTree(this)].(left,right) {
          finish {
            this.payload = x;
            if (left != null)
              async left.makeConstant(x);
            if (right != null)
              async right.makeConstant(x);
          }}}
  \end{lstlisting}
In this example the fields \code{left},\code{right} and \code{cargo}
are mutable. It is possible to mutate a tree \code{p} -- replace
\code{p}'s \code{left} child with another tree \code{q}. However, \code{q} cannot be in
\code{p}'s \code{right} subtree, because then its \code{up} field or
\code{left} field would not have the right value. That is, once a
\code{Tree} object is created it can only belong to a specific tree in
a specific position.

Note that a \code{Tree} can be created with \code{null} parent and
children. This is how the root is created:
\code{new Tree(null,true)} or \code{new tree(null, false)}.
%%
%%In checking the validity of the effect annotation on
%%\code{makeConstant} the compiler must check that the effects of the
%%body are contained in the declared effects of the method. This boils
%%down to checking the subtyping relations. e.g.{} for the \code{@Write} annotation it must check:
%%\begin{lstlisting}
%%SubTreeTree
%%Tree{i:UInt^(self.up$i.up==this,self.up$i.left==true} <: Tree{i:UInt^(self.up$i==this}
%%Tree{i:UInt^(self.up$i.up==this,self.up$i.left==false} <: Tree{i:UInt^(self.up$i==this}
%%Tree{self==this} <: Tree{i:UInt^(self.up$i==this)}
%%\end{lstlisting}

This is easily verified.
The notion of ``distinctions from the left'' and ``distinctions from
the right'' of \cite{DPJ} are not needed. These arise naturally
through the use of constraints (distinct access paths, vs distinct fields).

Note that the programmer may not desire to have the fields \code{up}
and \code{left} be available at run-time. These fields can be marked
as \code{@ghost} -- any attempt to access them at run-time will result
in an error.

Constraints such as ``this array must point to distinct objects'' can
also be naturally represented in the dependent type system. For more
details please see \cite{effects-constrained-types}.

\subsection{Conclusion}

This paper presents a lightweight design for a safe language which
permits very rich expression of concurrency idioms.


%\acks
%This material is based in part on work supported by the Defense Advanced Research Projects Agency under its Agreement No. HR0011-07-9-0002.


\bibliographystyle{abbrv} %plain plainnat abbrvnat abbrv
\bibliography{x10-init}

\Appendix{Proofs for the Formalism}
A closed statement is reduced to a closed expression.
\begin{lemma}
\label{Lemma:closed}
If $\vdash \hS$ and $\hS,H \reducesto \hS',H'$ then $\vdash \hS'$.
\end{lemma}
\begin{proof}
The only rules that introduce free variables are \RULE{R-Val}
    and \RULE{R-Invoke}.

In rule \RULE{R-Val} we have \[
    \hval{\hx}{\he}{\hS},H \reduce \hS[\hl/\hx], H'
\]
Therefore, \hS might have a free variable (\hx), but~$\hS[\hl/\hx]$ is closed.

In rule \RULE{R-Invoke} we have \[
    \hl'.\hm(\ol{\hl}),H \reduce \hS[\ol{\hl}/\ol{\hx},\hl'/\hthis],H
\]
The method body can only refer to the free variables~$\ol{\hx}$ and~$\this$,
    therefore~$\hS[\ol{\hl}/\ol{\hx},\hl'/\hthis]$ is closed.
\end{proof}



Monotonicity of cooked heaps: during the process of reduction, the heap stays correctly-cooked and it becomes more cooked.
\begin{lemma}
\label{Lemma:correctly-cooked}
If $\vdash H$ and (i)~$\hS,H \reducesto \hS',H'$ or~(ii)~$\hS,H \reducesto H'$, then $\vdash H'$ and~$H' \vdash H$.
\end{lemma}
\begin{proof}
By definition:
    (i)~$\vdash H$ if for every object $o=C(F)$ in the range of $H$ and
    every field $f \in \dom(F)$ it is the case that every object~$\hl=H(F(f))$ is cooked ($\cooked_H(\hl)$).
    (ii)~$H' \vdash H$
    if for every~$\hl \in \dom(H)$, we have~$H(\hl)=\hC(F)$,~$H'(\hl)=\hC(F')$, and~$\dom(F)\subseteq\dom(F')$.

By structural induction on~\hS.

For \RULE{R-Term} and \RULE{R-Step}, by induction $\vdash H'$ and~$H' \vdash H$.

For \RULE{R-Val}, $\he,H \reduce \hl',H'$ in two possible ways: \RULE{R-New} or \RULE{R-Access}.
In \RULE{R-New}, we created a new location~$\hl'$ that is not connected to any other location, so $\vdash H'$ and~$H' \vdash H$.
In \RULE{R-Access}, $H'=H$.

In \RULE{R-Invoke}, $H'=H$.

Only \RULE{R-Assign} ($\hl.\hf=\hl'$) changes the heap, and it requires that~$\cooked_H(\hl')$ so $\vdash H'$.
Let~$H(\hl)=\hC(F)$ and~$H'(\hl)=\hC(F')$, where~$F'=F[ \hf \mapsto \hl']$, so~$\dom(F)\subseteq\dom(F')$.
Thus~$H' \vdash H$.
\end{proof}



Properties of plus.
\begin{lemma}
\label{Lemma:plus}
For all $\phi$,
    (i)~$+\pm\phi=+\phi$,
    (ii)~$-\pm\phi=-\phi$.
\end{lemma}
\begin{proof}
Routine.
\end{proof}



Properties of the logic system.
\begin{lemma}
\label{Lemma:logic}
Let $\phi \vdash \psi$.
Then
    (i)~$+\phi \vdash +\psi$,
    (ii)~$-\phi \vdash -\psi$,
    (iii)~for all~$\phi'$, we have~$\phi',\phi \vdash \phi',\psi$.
\end{lemma}
\begin{proof}
Routine.
\end{proof}



Properties of heaps and formulas:
\begin{lemma}
\label{Lemma:heaps}
For all heaps~$H$,
(i)~$H \vdash -\phi$,
(ii)~if $H \vdash +\phi$ then $H \vdash \phi$,
(iii)~if $H \vdash \phi$ and $H \vdash \psi$ then $H \vdash \phi,\psi$,
(iv)~if $H \vdash \phi$ and~$H' \vdash H$ then $H' \vdash \phi$.
\end{lemma}
\begin{proof}
Routine.
\end{proof}


The meaning of substitution for statement is to replace all the \emph{free} occurrences of a variable.
For example, for the following statement~\hS \[
\hx.\hf=\hl;\hval{\hx}{\hnew{\hC}}{\hx.\hf=\hl}
\]
if we substitute \hx with $\hl'$, written as~$\hS[\hl'/\hx]$, the result is \[
\hl'.\hf=\hl;\hval{\hx}{\hnew{\hC}}{\hx.\hf=\hl}
\]

Substitution lemma.
\begin{lemma}
\label{Lemma:substitution}
Given a heap $H$, a location~\hl whose type is $\hC$ ($H(\hl)=\hC(F)$), a variable \hx whose type is $\hD$, a statement \hS,
    and \[
    \vdash H \gap \hC \st \hD \gap \phi~\hS~\psi
    \] then~$\phi[\hl/\hx]~\hS[\hl/\hx]~\psi[\hl/\hx]$.
\end{lemma}
\begin{proof}
Routine (by induction on the derivation sequence of~$\phi~\hS~\psi$, considering each rule in \Ref{Figure}{effects}).
The only sensitive point is that the type of~\hl is a subtype \hx, and therefore in \RULE{T-Invoke} we use the fact that when method~$\hm$ is overridden by~$\hm'$
    then their inferred decorations~$(\phi, \psi)$ and~$(\phi', \psi')$ satisfy $\phi \vdash \phi'$ and $\psi' \vdash \psi$.
\end{proof}


Preservation for terminal configurations.
\begin{lemma}
\label{Lemma:Preservation1}
Let $\acute{\phi}~\acute{\hS}~\acute{\psi}$, $\vdash \acute{\hS}$, $\vdash H$, $H \vdash \acute{\phi}$, and $\acute{\hS},H \reducesto H'$.
Then $H' \vdash +\acute{\psi}$.
\end{lemma}
\begin{proof}
By structural induction on~$\acute{\hS}$.
Only two rules result with a terminal configuration:
    \RULE{R-Term} (for \hfinish and \hasync)
    and \RULE{R-Assign}.


In \RULE{R-Term} for \hfinish we have:
\beq{pre1}
    \acute{\hS}=\finish{\hS} \gap \hS,H \reduce H'
\eeq
From~$\acute{\phi}~\acute{\hS}~\acute{\psi}$ and \RULE{T-Finish}
\beq{pre2}
    \acute{\psi}=+\psi \gap \phi~\hS~\psi \gap \acute{\phi}=\phi \gap \phi~\finish{\hS}~ +\psi
\eeq
Because $\hS,H \reduce H'$ and~$\phi~\hS~\psi$ (from \eq{pre2}), by induction we have
\beq{pre3}
    H' \vdash +\phi
\eeq
Because~$\acute{\psi}=+\psi$ and $++\psi=+\psi$ (\Ref{Lemma}{plus}), from~\eq{pre3}, we conclude~$H' \vdash +\acute{\psi}$.

In \RULE{R-Term} for \hasync we do the same proof but use the fact that~$+-\psi=+\psi$ (\Ref{Lemma}{plus}).

In \RULE{R-Assign} we have:
\beqs{pre4}
    \acute{\hS}=\hl.\hf=\hl' \gap H(\hl)=\hC(F) \gap \cooked_H(\hl')
    \\ H'=H[ \hl \mapsto \hC(F[ \hf \mapsto \hl'])]
\eeq
From~$\acute{\phi}~\acute{\hS}~\acute{\psi}$ and \RULE{T-Assign}
\beq{pre5}
    \acute{\phi}=\phi
    \gap
    \acute{\psi}=+\hl.\hf
    \gap
    \phi ~ \hl.\hf=\hl' ~ + \hl.\hf
    \gap
    \phi \vdash +\hl'
\eeq
From~\eq{pre4} and~\eq{pre5}, we conclude~$H' \vdash +\acute{\psi}$.
\end{proof}


Preservation for non-terminal configurations.
\begin{lemma}
\label{Lemma:Preservation2}
Let
\beq{np}
    \acute{\phi}~\acute{\hS}~\acute{\psi}
    \gap
    \vdash \acute{\hS}
    \gap
    \vdash H
    \gap
    H \vdash \acute{\phi}
    \gap
    \acute{\hS},H \reducesto \grave{\hS},H'
\eeq
Then there exists $\grave{\phi},\grave{\psi}$ such
    that
\beq{np-res}
H' \vdash \grave{\phi}
\gap
\grave{\phi}~\grave{\hS}~\grave{\psi}
\gap
\grave{\phi} \vdash \acute{\phi}
\gap
\grave{\psi} \vdash \acute{\psi}
\eeq
\end{lemma}
\begin{proof}
By structural induction on~$\acute{\hS}$.

For \RULE{R-Term} for sequences we have
\beq{np1}
 \acute{\hS} = \hS~\hS' \gap
 \grave{\hS}=\hS' \gap
 \hS,H \reduce H'
\eeq
From \RULE{T-Seq} and~\eq{np} ($\acute{\phi}~\acute{\hS}~\acute{\psi}$),
\beq{np2}
  \acute{\phi}=\phi
    \gap
  \acute{\psi}=\psi_1,\psi_2
    \gap
  \phi~\hS~\psi_1
    \gap
  \phi,\psi_1~\hS'~\psi_2
\eeq
From~\eq{np1} and~\eq{np2}, by \Ref{Lemma}{Preservation1}
\beq{np3}
    H'\vdash +\psi_1.
\eeq
Let~$\grave{\phi}=\phi,\psi_1$, then from~\eq{np2}
\beq{np4}
    \grave{\phi}~\grave{\hS}~\psi_2
\eeq
Because~$\grave{\phi}=\phi,\psi_1$ then~$\grave{\phi} \vdash \psi_1$, and because~$\grave{\psi}=\acute{\psi}=\psi_1,\psi_2$ then
\beq{np4b}
    \grave{\phi}~\grave{\hS}~\grave{\psi}
\eeq
Trivially,
\beq{np5}
    \phi,\psi_1 \vdash \phi
    \gap
    \grave{\psi} \vdash \acute{\psi}
\eeq
From~\eq{np3} and \Ref{Lemma}{heaps},
\beq{np5a}
    H'\vdash \psi_1
\eeq
Because~$H' \vdash H$ (\Ref{Lemma}{correctly-cooked}) and~$H \vdash\phi$, then from \Ref{Lemma}{heaps}~$H' \vdash\phi$.
Together,~$H' \vdash \phi,\psi_1$ and~$\grave{\phi}=\phi,\psi_1$.


For \RULE{R-Step} for \hfinish, we have
\beq{np6}
 \acute{\hS} = \finish{\hS} \gap
 \grave{\hS}=\finish{\hS'} \gap
 \hS,H \reduce \hS', H'
\eeq
Because~$\acute{\phi}~\acute{\hS}~\acute{\psi}$, from \RULE{T-Finish},
\beq{np7}
  \acute{\phi}=\phi
    \gap
  \acute{\psi}=+\psi
    \gap
  \phi~\hS~\psi
\eeq
From~\eq{np6} and~\eq{np7}, by induction, exists~$\phi'$ and~$\psi'$,
\beq{np8}
  \phi'~\hS'~\psi'
    \gap
  \phi' \vdash \phi
    \gap
  \psi' \vdash \psi
    \gap
  H' \vdash \phi'
\eeq
From \RULE{T-Finish} and \eq{np8},
\beq{np9}
  \phi'~\finish{\hS'}~+\psi'
\eeq
Let~$\grave{\phi}=\phi'$ and~$\grave{\psi}=+\psi'$, then from~\eq{np8} and
    from \Ref{Lemma}{logic},
\beq{np10}
    \phi' \vdash \phi
    \gap
    +\psi' \vdash +\psi
    \gap
    \grave{\phi}~\grave{\hS}~\grave{\psi}
    \gap
    H' \vdash \phi'
\eeq

The proof for the first \RULE{R-Step} for \hasync is like that of \hfinish above (but instead we use the fact that
    if~$\psi' \vdash \psi$ then~$-\psi' \vdash -\psi$ in \Ref{Lemma}{logic}).

For the second \hasync \RULE{R-Step}, we have
\beq{np11}
 \acute{\hS} = \async{\hS_1}~\hS \gap
 \grave{\hS}=\async{\hS_1}~\hS' \gap
 \hS,H \reduce \hS', H'
\eeq
Because~$\acute{\phi}~\acute{\hS}~\acute{\psi}$, from \RULE{T-Seq},
\beq{np12}
  \acute{\phi}=\phi
    \gap
  \acute{\psi}=\psi_1,\psi_2
    \gap
  \phi~\async{\hS_1}~\psi_1
        \gap
  \phi,\psi_1~\hS~\psi_2
\eeq
From \RULE{T-Async} we know that there exists~$\psi'_1$ such that~$\psi_1=-\psi'_1$,
    therefore from \Ref{Lemma}{heaps}, we have that for any~$H$, it holds~$H \vdash \psi_1$.
Thus,~$H \vdash \phi,\psi_1$.
From that and~$\phi,\psi_1~\hS~\psi_2$ we can apply the induction and get that there exists~$\phi'$ and~$\psi'$ such that
\beq{np13}
    \phi'~\hS'~\psi'
    \gap
    \phi' \vdash \phi,\psi_1
    \gap
    \psi' \vdash \psi_2
    \gap
    H' \vdash \phi'
\eeq
Then from~\eq{np12},~\eq{np13},
\beq{np14}
    \phi'~\async{\hS_1}~\psi_1
    \gap
    \phi',\psi_1~\hS'~\psi'
\eeq
From~\eq{np14} and \RULE{T-Seq}
\beq{np15}
    \phi'~\async{\hS_1}~\hS'~\psi_1,\psi'
\eeq
Let~$\grave{\phi}=\phi'$ and~$\grave{\psi}=\psi_1,\psi'$, then
\beq{np16}
    \grave{\phi}~\async{\hS_1}~\hS'~\grave{\psi}
    \phi' \vdash \phi
    \gap
    \psi_1,\psi' \vdash \psi_1,\psi_2
    \gap
    H' \vdash \grave{\phi}
\eeq



For \RULE{R-Step} for sequences, we have
\beq{np26}
 \acute{\hS} =\hS~\hS_1 \gap
 \grave{\hS}=\hS'~\hS_1 \gap
 \hS,H \reduce \hS', H'
\eeq
Because~$\acute{\phi}~\acute{\hS}~\acute{\psi}$, from \RULE{T-Seq},
\beq{np27}
  \acute{\phi}=\phi
    \gap
  \acute{\psi}=\psi_1,\psi_2
    \gap
  \phi~\hS~\psi_1
    \gap
  \phi,\psi_1~\hS_1~\psi_2
\eeq
From~\eq{np26} and~\eq{np27}, by induction ($\hS,H \reduce \hS', H'$ and~$H \vdash \phi$ and~$\phi~\hS~\psi_1$),
    exists~$\phi'$ and~$\psi'_1$,
\beq{np28}
  \phi'~\hS'~\psi'_1
    \gap
  \phi' \vdash \phi
    \gap
  \psi'_1 \vdash \psi_1
    \gap
  H' \vdash \phi'
\eeq
Thus, from~\eq{np27} and~\eq{np28},
\beq{np28b}
  \phi'~\hS'~\psi'_1
    \gap
  \phi',\psi'_1~\hS_1~\psi_2
\eeq
\eeq
From \RULE{T-Seq} and \eq{np28b},
\beq{np29}
  \phi'~\hS'~\hS_1~\psi'_1,\psi_2
\eeq
Let~$\grave{\phi}=\phi'$ and~$\grave{\psi}=\psi'_1,\psi_2$, then from~\eq{np28} and~\eq{np29} and
    from \Ref{Lemma}{logic},
\beq{np30}
    \phi' \vdash \phi
    \gap
    \psi'_1,\psi_2 \vdash \psi_1,\psi_2
    \gap
    \grave{\phi}~\grave{\hS}~\grave{\psi}
    \gap
    H' \vdash \phi'
\eeq

For \RULE{R-Val},
\beq{np36}
 \acute{\hS} =\hval{\hx}{\he}{\hS} \gap
 \grave{\hS}= \hS[\hl/\hx] \gap
 \he,H \reduce \hl,H'
\eeq
We have two possible ways to reduce \he:
    either with \RULE{R-New} or \RULE{R-Access}.
Consider first \RULE{R-New}, so
\beq{np36b}
 \he = \hnew{\hC} \gap
 \hl \not \in \dom(H) \gap
 H'=H[ \hl \mapsto \hC()]
\eeq
Because~$\acute{\phi}~\hval{\hx}{\hnew{\hC}}{\hS}~\acute{\psi}$, from \RULE{T-New},
\beq{np37}
  \acute{\phi} = \phi \gap
  \acute{\psi} = \psi \gap
  \phi~\hS~\psi
\eeq
Because~$H$ contains only locations~\hl (not variables~\hx), and~$H \vdash \phi$,
    then~$\phi[\hl/\hx]=\phi$ and~$\psi[\hl/\hx]=\psi$.
From \Ref{Lemma}{substitution} and~\eq{np37}, we have
\beq{np38}
  \phi~\hS[\hl/\hx]~\psi
\eeq
From \Ref{Lemma}{heaps}, because~$H' \vdash H$ (\Ref{Lemma}{correctly-cooked}), then
\beq{np39}
  H' \vdash \phi
\eeq
We choose~$\grave{\phi}=\phi$ and~$\grave{\psi}=\psi$, thus from~\eq{np38} and~\eq{np39},
\beq{np40}
  H' \vdash \grave{\phi} \gap
  \grave{\phi} \vdash \phi  \gap
  \grave{\psi} \vdash \psi  \gap
  \grave{\phi}~\grave{\hS}~\grave{\psi}
\eeq
Now consider \RULE{R-Access}, so
\beq{np41}
 \he = \hl'.\hf_i \gap
 H(\hl')=\hC(\ol{\hf}\mapsto\ol{\hl"}) \gap
 \hl = \hl"_i \gap
 H'=H
\eeq
Because~$\vdash H$ then~$\hl$ is cooked, i.e., \[H \vdash +\hl \]
Because~$\acute{\phi}~\hval{\hx}{\hl'.\hf_i}{\hS}~\acute{\psi}$, from \RULE{T-Access},
\beq{np42}
  \acute{\phi} = \phi \gap
  \acute{\psi} = \psi \gap
    \phi \vdash +\hl'.\hf_i \gap
    \phi, +\hx~\hS~\psi
\eeq
Like for \RULE{R-New}, we have that~$\phi[\hl/\hx]=\phi$ and~$\psi[\hl/\hx]=\psi$.
From \Ref{Lemma}{substitution} and~\eq{np42}, we have
\beq{np43}
  \phi,+\hl~\hS[\hl/\hx]~\psi
\eeq
Because~$H'=H$ and~$H \vdash \phi$ and~$H \vdash +\hl$,
    then
\beq{np44}
  H' \vdash \phi,+\hl
\eeq
We choose~$\grave{\phi}=\phi,+\hl$ and~$\grave{\psi}=\psi$, thus from~\eq{np43} and~\eq{np44},
\beq{np45}
  H' \vdash \grave{\phi} \gap
  \grave{\phi} \vdash \phi  \gap
  \grave{\psi} \vdash \psi  \gap
  \grave{\phi}~\grave{\hS}~\grave{\psi}
\eeq

For \RULE{R-Invoke},
\beqs{np56}
 \acute{\hS} =\hl'.\hm(\ol{\hl}) \gap
 \grave{\hS}=\hS[\ol{\hl}/\ol{\hx},\hl'/\hthis] \\
 H(\hl')=\hC(\ldots) \gap
 \mbody{}(\hm,\hC)=\ol{\hx}.\hS \gap
 H'=H
\eeq
Because~$\acute{\phi}~\acute{\hS}~\acute{\psi}$, from \RULE{T-Invoke},
\beq{np57}
  \acute{\phi}=\phi
    \gap
  \acute{\psi}=\psi'[\hl'/\hthis,\ol{\hl}/\ol{\hx}]
    \gap
  \hm(\ol{\hx}):: \phi' \Rightarrow \psi'
    \gap
  \phi \vdash \phi'[\hl'/\hthis,\ol{\hl}/\ol{\hx}]
\eeq


\hl

congruence

\RULE{R-Assign}
\end{proof}


















\begin{Theorem}{\textbf{Preservation}}
\label{Theorem:Preservation}
Let $\phi~\hS~\psi$, $\vdash \hS$, $\vdash H$, $H \vdash \phi$.
(a) If $\hS,H \reducesto H'$ then $\vdash H'$, $H' \vdash H$, $H' \vdash +\psi$.
(b) If $\hS,H \reducesto \hS',H'$ then $\vdash \hS'$, $\vdash H'$, $H' \vdash H$,
there exists $\phi',\psi'$ such
that $H' \vdash \phi'$, $\phi'~\hS'~\psi'$, $\phi' \vdash \phi$, $\psi' \vdash \psi$.
\end{Theorem}
\begin{proof}
Follows from \Ref{Lemma}{closed},  \Ref{Lemma}{correctly-cooked}, \Ref{Lemma}{Preservation1},
    and \Ref{Lemma}{Preservation2}.
\end{proof}

Progress: a proper non-terminal configuration can always be reduced.
\begin{Theorem}{\textbf{Progress}}
Let $\phi~\hS~\psi$, $\vdash \hS$, $\vdash H$, $H \vdash \phi$.
Then there exists $\hS',H'$ such
    that either $\hS,H \reducesto H'$ or $\hS,H \reducesto \hS',H'$.
\end{Theorem}
\begin{proof}
\end{proof}



\end{document}


Tested my serialization hypotheses using this Java code:
        ObjectOutput out = new ObjectOutputStream(new FileOutputStream("filename.ser"));
        out.writeObject("C:\\cygwin\\home\\Yoav\\intellij\\sourceforge\\x10.tests,C:\\cygwin\\home\\Yoav\\intellij\\sourceforge\\x10.dist\\samples,C:\\cygwin\\home\\Yoav\\intellij\\sourceforge\\x10.runtime\\src-x10"+"C:\\cygwin\\home\\Yoav\\intellij\\sourceforge\\x10.tests,C:\\cygwin\\home\\Yoav\\intellij\\sourceforge\\x10.dist\\samples,C:\\cygwin\\home\\Yoav\\intellij\\sourceforge\\x10.runtime\\src-x10"+"C:\\cygwin\\home\\Yoav\\intellij\\sourceforge\\x10.tests,C:\\cygwin\\home\\Yoav\\intellij\\sourceforge\\x10.dist\\samples,C:\\cygwin\\home\\Yoav\\intellij\\sourceforge\\x10.runtime\\src-x10"+"C:\\cygwin\\home\\Yoav\\intellij\\sourceforge\\x10.tests,C:\\cygwin\\home\\Yoav\\intellij\\sourceforge\\x10.dist\\samples,C:\\cygwin\\home\\Yoav\\intellij\\sourceforge\\x10.runtime\\src-x10"+"C:\\cygwin\\home\\Yoav\\intellij\\sourceforge\\x10.tests,C:\\cygwin\\home\\Yoav\\intellij\\sourceforge\\x10.dist\\samples,C:\\cygwin\\home\\Yoav\\intellij\\sourceforge\\x10.runtime\\src-x10"+"C:\\cygwin\\home\\Yoav\\intellij\\sourceforge\\x10.tests,C:\\cygwin\\home\\Yoav\\intellij\\sourceforge\\x10.dist\\samples,C:\\cygwin\\home\\Yoav\\intellij\\sourceforge\\x10.runtime\\src-x10"+"C:\\cygwin\\home\\Yoav\\intellij\\sourceforge\\x10.tests,C:\\cygwin\\home\\Yoav\\intellij\\sourceforge\\x10.dist\\samples,C:\\cygwin\\home\\Yoav\\intellij\\sourceforge\\x10.runtime\\src-x10"+"C:\\cygwin\\home\\Yoav\\intellij\\sourceforge\\x10.tests,C:\\cygwin\\home\\Yoav\\intellij\\sourceforge\\x10.dist\\samples,C:\\cygwin\\home\\Yoav\\intellij\\sourceforge\\x10.runtime\\src-x10".substring(1));
        out.close(); // 2KB!!
        ObjectOutput out2 = new ObjectOutputStream(new FileOutputStream("filename2.ser"));
        out2.writeObject("C".substring(1));
        out2.close(); // 7bytes!


\subsection{Alternative property design}

Property design (what would happen if we use val fields instead of properties.)

Fields are partition into two: property fields and normal fields. Property fields are assigned together before any other field.
\vspace{-0.2cm}\begin{lstlisting}
class A(x:Int) {
  val y:Int{self!=x};
  val z = x*x;
}
\end{lstlisting}\vspace{-0.2cm}
Vs.
Fields are ordered, and all ctors need to assign in the same order as the fields declaration.
\vspace{-0.2cm}\begin{lstlisting}
class A {
  val x:Int;
  val y:Int{self!=x};
  val z = x*x;
}
\end{lstlisting}\vspace{-0.2cm}



\subsection{Static initialization}
X10 does not support dynamic class loading as opposed to Java,
    and all static fields in X10 are final.
Thus, initialization of static fields is a one-time phase, denoted the static-init phase,
    that is done before the \code{main} method is executed.

During the static-init phase we must finish writing to all static fields,
    and reading a static field \emph{waits} until the field is initialized
    (i.e., the current activity/thread blocks if the field was not written to,
    and it resumes after another activity writes to it).
Obviously, this may lead to deadlock as demonstrated by \Ref{Figure}{Static-init}.
However, in practice, deadlock is rare,
    and usually found quickly the first time a program is executed.

\begin{figure}
\vspace{-0.2cm}\begin{lstlisting}
class A {
  static val a:Int = B.b;
}
class B {
  static val b:Int = A.a;
}
\end{lstlisting}\vspace{-0.2cm}
\caption{Static initialization example:
    the program will deadlock at run-time
    during the static-init phase (before the \code{main} method is executed).
    }
\label{Figure:Static-init}
\end{figure}


\subsection{Memory model and constructor barrier}

todo: discuss final in Java and String class.
(people think that removing final may only hurt performance, but it may be semantic changing.)

type safety and the weak memory model:
* in Java if you don't use final fields correctly or leak this, you will simply see default values (you don't lose type-safety)
* in X10 it could break type safety (if we don't put a barrier at the end of a ctor).

class Box[T] {T haszero} {
  var value:T;
}
class A {
  static val box = new Box[A]();

  var f:Int{self!=0} = 1;
}

var a = new A();
a.f = 2;
A.box.value = a;

Suppose another activity reads A.box.value.
Should the writes to
a.f
and
A.box.value
be ordered? (I don't think we should order them without losing performance)

Therefore, X10 needs a synchronization barrier at the end of a ctor that guarantees that all writes to the fields (both VAL and VAR) of the object has finished before the handle is returned.
(This is different from Java that only promises this for final fields. And the barrier also happens again after deserialization - requiring this weird freeze operation and you could freeze a final field again even after the ctor ended due to deserialization.)


Bowen proposes this transformation in order to inline ctors:
class A {
  static val box = new Box[A]();

  var f:Int{self!=0} = 0;
}


var a = new A();
a.f = 1;
// INSERT BARRIER EXPLICITLY HERE
a.f = 2;
A.box.value = a;
