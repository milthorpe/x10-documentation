\documentclass[9pt]{sigplanconf}


\conferenceinfo{X10'11,} {June 4, 2011, San Jose, California, USA.}
\CopyrightYear{2011}
%\copyrightdata{}

% Doesn't work in a caption!
%\newcounter{RuleCounter}
%\stepcounter{RuleCounter}
% \arabic{RuleCounter}
\newcommand{\myrule}[2]{\textbf{Rule #1:} #2.}

\usepackage{relsize}
\usepackage{amsmath}
\usepackage{url}


% http://en.wikibooks.org/wiki/LaTeX/Packages/Listings
\usepackage{listings}


\usepackage[ruled]{algorithm} % [plain]
\usepackage[noend]{algorithmic} % [noend]
\renewcommand\algorithmiccomment[1]{// \textit{#1}} %


% For fancy end of line formatting.
\usepackage{microtype}


% For smaller font.
\usepackage{pslatex}


\usepackage{xspace}
\usepackage{yglabels}
\usepackage{yglang}
\usepackage{ygequation}
\usepackage{graphicx}
%\usepackage{epstopdf}

\newcommand{\formalrule}[1]{\mbox{\textsc{\scriptsize #1}}}
\newcommand{\myrule}[1]{\textsc{\codesmaller #1 Rule}}
\newcommand{\umyrule}[1]{\textbf{\underline{\textsc{\codesmaller #1 Rule}}}}


% \small \footnotesize \scriptsize \tiny
% \codesize and \scriptsize seem to do the same thing.
% \newcommand{\code}[1]{\texttt{\textup{\footnotesize #1}}}
% \newcommand{\code}[1]{\texttt{\textup{\codesize #1}}}
\newcommand{\normalcode}[1]{\texttt{\textup{#1}}}
\def\codesmaller{\small}
\newcommand{\myCOMMENT}[1]{\COMMENT{\small #1}}
\newcommand{\code}[1]{\texttt{\textup{\codesmaller #1}}}
% \newcommand{\code}[1]{\ifmmode{\mbox{\smaller\ttfamily{#1}}}
%                       \else{\smaller\ttfamily #1}\fi}
\newcommand{\smallcode}[1]{\texttt{\textup{\scriptsize #1}}}
%\newcommand{\myparagraph}[1]{\noindent\textit{\textbf{#1}}~} %\vspace{-1mm}\paragraph{#1}}

% See: \usepackage{bold-extra} if you want to do \textbf
\newcommand{\keyword}[1]{\code{#1}}

% For new, method invocation, and cast:
\newcommand{\hparen}[1]{\code{(}#1\code{)}}

\newcommand{\hgn}[1]{\lt#1\gt} % type parameters and generic method parameters

\newcommand{\Own}{{\it O}}
\newcommand{\Ifn}[1]{\ensuremath{I(#1)}}
\newcommand{\Ofn}[1]{\ensuremath{O(#1)}}
\newcommand{\Cooker}[1]{\ensuremath{{\kappa}(#1)}}
\newcommand{\Owner}[1]{\ensuremath{{\theta}(#1)}}
\newcommand{\Oprec}[0]{\ensuremath{\preceq_{\theta}}}
\newcommand{\Tprec}[0]{\ensuremath{\preceq^T}}
\newcommand{\TprecNotEqual}[0]{\ensuremath{\prec^T}}
\newcommand{\OprecNotEqual}[0]{\ensuremath{\prec_{\theta}}}
\newcommand{\IfnDelta}[1]{\ensuremath{I_\Delta(#1)}}
\newcommand\Abs[1]{\ensuremath{\left\lvert#1\right\rvert}}
\newcommand{\erase}[1]{\ensuremath{\Abs{#1}}}

\newcommand{\CookerH}[1]{\ensuremath{{\kappa}_H(#1)}}
\newcommand{\IfnH}[1]{\ensuremath{I_H(#1)}}
\newcommand{\OwnerH}[1]{\ensuremath{{\theta}_H(#1)}}
\newcommand{\Gdash}[0]{\ensuremath{\Gamma \vdash }}
\newcommand{\reducesto}[0]{\rightsquigarrow}
\newcommand{\reduce}[0]{\rightsquigarrow}

\usepackage{color}
\definecolor{light}{gray}{.75}


\newcommand{\todo}[1]{\textbf{[[#1]]}}
%% To disable, just uncomment this line:
%\renewcommand{\todo}[1]{\relax}

%% Additional todo commands:
\newcommand{\TODO}[1]{\todo{TODO: #1}}

\newcommand\xX[1]{$\textsuperscript{\textit{\text{#1}}}$}


\newcommand{\ol}[1]{\overline{#1}}
\newcommand{\nounderline}[1]{{#1}}


%% Commands used to typeset the FIGJ type system.
\newcommand{\typerule}[2]{
\begin{array}{c}
  #1 \\
\hline
  #2
\end{array}}


%% Commands used to typeset the FOIGJ type system.
\newcommand{\inside}{\prec}
\newcommand{\visible}{{\it visible}}
\newcommand{\placeholderowners}{{\it placeholderowners}}
\newcommand{\nullexpression}{{\tt null}}
\newcommand{\errorexpression}{{\tt error}}
\newcommand{\locations}{{\it locations}} %% \mathop{\mathit{locations}}}
\newcommand{\xo}{{\tt X^O}}
\newcommand{\no}{{\tt N^O}}
\newcommand{\co}{{\tt C^O}}
\newcommand{\I}{\it I}


\newcommand\mynewcommand[2]{\newcommand{#1}{#2\xspace}}


\mynewcommand{\hI}{\code{I}} % iparam

% In the syntax: \hI or ReadOnly or Mutable or Immut
\mynewcommand{\hJ}{\code{J}}
\mynewcommand{\hO}{\code{O}}
\mynewcommand{\ho}{\code{o}}
\mynewcommand{\hnull}{\code{null}}
\mynewcommand{\htrue}{\code{true}}
\mynewcommand{\hfalse}{\code{false}}

\mynewcommand{\hX}{\code{X}} % vars
\mynewcommand{\hY}{\code{Y}} % vars
\mynewcommand{\hB}{\code{B}} % class
\mynewcommand{\hC}{\code{C}} % class
\mynewcommand{\hD}{\code{D}} % class
\mynewcommand{\hL}{\code{L}} % class decl
\mynewcommand{\hM}{\code{M}} % Method decl
\mynewcommand{\hm}{\code{m}} % method
\mynewcommand{\he}{\code{e}} % expression
\mynewcommand{\hl}{\code{l}} % location in the store
\mynewcommand{\hx}{\code{x}} % method parameter
\mynewcommand{\hf}{\code{f}} % field name
\mynewcommand{\hF}{\code{F}} % field
\mynewcommand{\hH}{\code{H}} % Heap
\mynewcommand{\hS}{\code{S}}
\mynewcommand{\hsub}{\code{/}} % substitute (reduction rules)

\mynewcommand{\hand}{\code{~and~}}
\mynewcommand{\hor}{\code{~or~}}
\mynewcommand{\hthis}{\code{this}} % this
\mynewcommand{\super}{\code{super}} % this
\mynewcommand{\hsuper}{\code{super}} % this
\mynewcommand{\hclass}{\code{class}}
\mynewcommand{\hreturn}{\code{return}}
\mynewcommand{\hnew}{\code{new}}
\newcommand{\lt}{\code{<}}%{\mathop{\textrm{\tt <}}}
\newcommand{\gt}{\code{>}}%{\mathop{\textrm{\tt >}}}

\mynewcommand{\this}{\keyword{this}}
\mynewcommand{\ctor}{\keyword{ctor}}
\mynewcommand{\Object}{\code{Object}}
\mynewcommand{\const}{\keyword{const}} %C++ keyword
\mynewcommand{\mutable}{\keyword{mutable}} %C++ keyword
\mynewcommand{\romaybe}{\keyword{romaybe}} %Javari keyword

%% Define the behaviour of the theorem package.
%% Use http://math.ucsd.edu/~jeggers/latex/amsthdoc.pdf for reference.

\newtheorem{theorem}{Theorem}[section]
\newtheorem{definition}[theorem]{Definition}
\newtheorem{lemma}[theorem]{Lemma}
\newtheorem{corollary}[theorem]{Corollary}
\newtheorem{fact}[theorem]{Fact}
\newtheorem{example}[theorem]{Example}
\newtheorem{remark}[theorem]{Remark}


\mynewcommand{\IP}{\code{I}}   % formal type parameter
\mynewcommand{\JP}{\code{J}}   % formal type parameter (for soundness proofs)


\mynewcommand{\Iparam}{Immutability parameter}
\mynewcommand{\iparam}{immutability parameter}
\mynewcommand{\iparams}{immutability parameters}
\mynewcommand{\Iparams}{Immutability parameters}
\mynewcommand{\Iarg}{Immutability argument}
\mynewcommand{\iarg}{immutability argument}
\mynewcommand{\iargs}{immutability arguments}
\mynewcommand{\Iargs}{Immutability arguments}
\mynewcommand{\ReadOnly}{\code{ReadOnly}}
\mynewcommand{\WriteOnly}{\code{WriteOnly}}
\mynewcommand{\None}{\code{None}}
\mynewcommand{\Mutable}{\code{Mutable}}
\mynewcommand{\Immut}{\code{Immut}}
\mynewcommand{\Raw}{\code{Raw}}


\mynewcommand{\This}{\code{This}}
\mynewcommand{\World}{\code{World}}


% Our annotations
\mynewcommand{\OMutable}{\code{@OMutable}}
\mynewcommand{\OI}{\code{@OI}}


\mynewcommand{\InVariantAnnot}{\code{@InVariant}}


\newcommand{\func}[1]{\text{\textnormal{\textit{\codesmaller #1}}}}


\mynewcommand{\st}{\ensuremath{\mathrel{{\leq}}}} %{\mathop{\textrm{\tt <:}}}
\mynewcommand{\notst}{\mathrel{\st\hspace{-1.5ex}\rule[-.25em]{.4pt}{1em}~}}
\mynewcommand{\tl}{\ensuremath{\triangleleft}}
\mynewcommand{\gap}{~ ~ ~ ~ ~ ~}


\newcommand{\RULE}[1]{\textsc{\scriptsize{}#1}} %\RULEhape\scriptsize}


\mynewcommand{\DA}{\texttt{DA}}
\mynewcommand{\ok}{\texttt{OK}}
\mynewcommand{\OK}{\texttt{OK}}
\mynewcommand{\IN}{\texttt{IN}}
\mynewcommand{\subterm}{\func{subterm}}
\mynewcommand{\TP}{\func{TP}} % function that returns type parameters in a type
\mynewcommand{\CT}{\func{CT}} % class table
\mynewcommand{\mtype}{\func{mtype}}
\mynewcommand{\mmodifier}{\func{mmodifier}}
\mynewcommand{\fmodifier}{\func{fmodifier}}
\mynewcommand{\ctortype}{\func{ctortype}}
\mynewcommand{\ctorbody}{\func{ctorbody}}
\mynewcommand{\mbody}{\func{mbody}}
\mynewcommand{\ftype}{\func{ftype}}
\mynewcommand{\fields}{\func{fields}}
\DeclareMathOperator{\dom}{dom}
\mynewcommand{\methodVal}{\func{methodVal}_\Gamma} % val not assigned
\mynewcommand{\ctorVal}{\func{ctorVal}_\Gamma} % val assigned at most once
\mynewcommand{\reductionVal}{\func{reductionVal}} % val assigned at most once



\mynewcommand\xth{\xX{th}}
\mynewcommand\xrd{\xX{rd}}
\mynewcommand\xnd{\xX{nd}}
\mynewcommand\xst{\xX{st}}
\mynewcommand\ith{$i$\xth}
\mynewcommand\jth{$j$\xth}


%\mynewcommand{\emptyline}{\vspace{\baselineskip}}
\mynewcommand{\myindent}{~~}


% Add line between figure and text
\makeatletter
\def\topfigrule{\kern3\p@ \hrule \kern -3.4\p@} % the \hrule is .4pt high
\def\botfigrule{\kern-3\p@ \hrule \kern 2.6\p@} % the \hrule is .4pt high
\def\dblfigrule{\kern3\p@ \hrule \kern -3.4\p@} % the \hrule is .4pt high
\makeatother

\setlength{\textfloatsep}{.75\textfloatsep}


% Remove line between figure and its caption.  (The line is prettier, and
% it also saves a couple column-inches.)
\makeatletter
%\@setflag \@caprule = \@false
\makeatother


% http://www.tex.ac.uk/cgi-bin/texfaq2html?label=bold-extras
\usepackage{bold-extra}


% Left and right curly braces in tt font
\newcommand{\ttlcb}{\texttt{\char "7B}}
\newcommand{\ttrcb}{\texttt{\char "7D}}
\newcommand{\lb}{\ttlcb}
\newcommand{\rb}{\ttrcb}


\setlength{\leftmargini}{.75\leftmargini}


\begin{document}


\lstset{language=java,basicstyle=\ttfamily\small}


\title{Object Initialization in X10}


\authorinfo{Yoav Zibin \and Igor Peshansky \and David Cunningham \and Vijay Saraswat}
           {IBM research in TJ Watson}
           {yzibin$|$igorp$|$dcunnin$|$vsaraswa@us.ibm.com}


\maketitle


\begin{abstract}
\Xten{} is a modern object-oriented language designed for productivity and
performance in concurrent and distributed systems.  In this setting, dependent
types offer significant opportunities for detecting design errors statically,
documenting design decisions, eliminating costly run-time checks (e.g., for
array bounds, null values), and improving the quality of generated code.

We present the design and implementation of {\em constrained types}, a natural,
simple, clean, and expressive extension to object-oriented programming: A type
\xcd{C\{c\}} names a class or interface \xcd{C} and a {\em constraint} \xcd{c}
on the immutable state of \xcd{C} and in-scope final variables.  Constraints
may also be associated with class definitions (representing class invariants)
and with method and constructor definitions (representing preconditions).
Dynamic casting is permitted.  The system is parametric on the underlying
constraint system: the compiler supports a simple equality-based constraint
system but, in addition, supports extension with new constraint systems using
compiler plugins.


\end{abstract}

\category{D.3.3}{Programming Languages}{Language Constructs and Features}
\category{D.1.5}{Programming Techniques}{Object-oriented Programming}

\terms
Asynchronous, Initialization, Types, X10

% Keywords are not required in the paper itself - only in the submission system's meta data.
% \keywords
% Immutability, Ownership, Java


\Section[introduction]{Introduction}
A {\em serial} schedule for a parallel program is one which always
executes the first enabled step in program order. A {\em safe}
parallel program is one that can be executed with a serial schedule
$S$ and for which for every input every schedule produces the same
result (error, correct termination, divergence) as $S$.  Such a
program is semantically a sequential program, hence it is
scheduler-determinate and deadlock-free.

A {\em safe parallel programming language} is an imperative parallel
programming language in which every legal program is a safe
program. Programmers can write code in such a language secure in the
knowledge that they will not encounter a large class of parallel
programming problems. Such a language is particularly useful for
parallelizing sequential (imperative) programs. In such cases ({\em
  contra} reactive programming) the desired application semantics are
sequential, and parallelism is needed solely for efficient
implementation.

The characteristic property of a safe programming language is that the
{\em same} program has a sequential reading and a parallel reading,
and both are compatible with each other. Hence the program can be
developed and debugged as a sequential program, using the serial
scheduler, and then run the unchanged program in parallel.  Parallel
execution is guaranteed to effect only performance, not
correctness. Safety is a very strong property.

There are many challenges in designing an efficient, usable, powerful,
explicitly parallel imperative language that is safe. One central
challenge is that 

One way to get safety is through implicitly parallel languages (e.g.{}
Jade~\cite{Rinard98thedesign},\cite{vonPraun:2007:IPO:1229428.1229443}). One
starts with a sequential programming language, and adds constructs
(e.g.{} tasks) that permit speculative execution while guaranteeing
that the only observable write to a shared variable is the write by
the last task to execute in program order.  While this work is
promising, extracting usable parallelism from a wide variety of
sequential programs remains very hard.

Explicitly parallel programming languages provide a variety of
constructs for spawning tasks in parallel and coordinating between
them. Here the programmer can typically directly control the
granularity of concurrency, and locality of access (e.g. placement of
data-structures in a multi-node computation) and use efficient
concurrent primitives (atomic reads/writes, test and sets, locks etc)
to control their execution. 

For such languages proving that a program is safe -- much less that
the programming language is safe -- now becomes very hard,
particularly for modern object-oriented languages which allow the
programmer to create arbitrarily complicated data-structures in the
shared heap. It becomes very difficult to show that any possible
schedule will produce the same result as the serial schedule.

Our starting point is the language X10 \cite{x10} since it offers a
simple and elegant treatment of concurrency and distribution, with
some nice properties.  In brief, X10
introduces the constructs \code{async S} to spawn an activity to
execute \code{S}; \code{finish S} to execute \code{S} and wait until
such time as all activities spawned during its execution have
terminated; \code{at (p) S} to execute \code{S} at place
\code{p}. These constructs can be nested arbitrarily -- this is a
source of significant elegance and power. Additionally, X10 v 2.2
introduces a simplified version of X10 clocks (adequate for many
practical usages) -- \code{clocked finish S} and \code{clocked async
  S}. Briefly, a clocked finish introduces a new barrier that can be
used by this activity and its children activities for
synchronization.\footnote{X10 also has a conditional atomic construct,
  \code{when (c) S} which permits data-dependent synchronization and
  can introduce deadlocks. We do not consider this construct in this paper.}

\cite{vj-clock} establishes that a large class of programs
in X10, namely those that use \code{finish}, \code{async} and
\code{clock}s are deadlock-free. The central intuition is that a clock
can only be used by the activity that created it and by its children,
and hence the spawn tree structure can be used to avoid depends-for
cycles. 

To obtain determinacy, another idea is required. The central problem
is to ensure that in the statement \code{async \{S\} S1} it is not the
case that \code{S} can write to a location that \code{S1} can read
from or read from a location that \code{S} can read from. Otherwise
the behavior is not determinate. One line of attack has been the pursuit
of {\em effect systems} \cite{Lucassen:1988:PES:73560.73564},
\cite{Leino:2002:UDG:543552.512559},
\cite{boyland:01interdependence},
\cite{DPJ}. At a broad conceptual level, static effects systems call
for a user-specified partitioning of heap into {\em regions} in a
fine-grained enough way to show that operations that may occur
simultaneously work on different regions. For instance \cite{DPJ}
introduces separate syntax for regions, introduces the ability to
specify an arbitrary tree of regions, and new syntax for specifying
which region mutable locations belong to. Methods must be associated
with their read and write effects which capture the set of regions
that can be operated upon during the execution of this
method. Now determinacy can be established if it can be determined
that that in \code{async \{S\} S1} it is the case that \code{S} does
not write any region that \code{S1} reads or writes from, and vice
versa ({\em disjoint parallelism}). 

\cite{DPJ} develops these ideas in the context of a language with
\code{cobegin/coend} and \code{forall} parallelism, and does not
address arbitrary nested or clocked parallelism.  In particular it is
not clear how to adapt these ideas to support some form of pipelined
communication between multiple parallel activities. Once can think of
these computations as requiring ``disjointness across time'' rather
than disjointness across space. A producer is going to write into
locations \code{w(t),w(t+1),w(t+2), \ldots}, but this does not
conflict with a consumer reading from the same locations as long as
the consumer can arrange to read the values in time-staggered order,
i.e. read \code{w(t)} once the produce is writing \code{w(t+1)}.  

We believe it is possible to significantly simplify this approach
(e.g.{} using the dependent-type system of X10) and extend it to cover
all of X10's concurrency constructs (see
Section~\ref{future-work}). Nevertheless the cost of developing these
region assertions is not trivial. Their viability for developing large
commercial-strength software systems is yet to be establishd.

In this paper we chose a different approach, a {\em lightweight}
approach to safety. We introduce two simple ideas -- {\em
  accumulators} and {\em clocked types} -- which require very modest
compiler support and can be implemented efficiently at run-time.

Accumulators arise very naturally in concurrent programming: an
accumulator is a mutable location associated with a commutative and
associative {\em reduction} operator that can be operated on
simultaneously by multiple activities. Multiple values offered by
multiple activities are combined by the reduction operator. We show
that the concrete rules for accumulators can be defined in such a way
that they do not compromise safety: a serial execution precisely
captures results generable from any execution.

Similarly clocked computations (barrier-based synchronization) is
quite common in parallel programming e.g. in the SPMD model, BSP model
etc. We observe that many clocked programs can be written in such a
way that shared variables take on a single value in one phase of the
clock. Further, clocked computations are often iterative and operate
on large aggregate data-structures (e.g. arrays, hash-maps) in a
data-parallel fashion, reading one version of the data-structure (the
``red'' version) while simultaneously writing another version (the
 ``black'' version). To support this widely used idiom, we introduce
the notion of {\em clocked types}. An instance \code{a} of a  clocked type
\code{Clocked[T]} keeps two instances of type \code{T}, the \code{now} and
the \code{next} instance. \code{a} can be operated upon only by
activities registered on the current clock.  All read operations
during the current clock phase are directed to the \code{now} version,
and write operation to the \code{next} version. This ensures that
there are no read-write conflicts. There may be write-write conflicts
-- these must be managed by either using accumulators or an effects
system. Once computation in the current phase has quiesced -- and
before activities start in the next phase -- the \code{now} and
\code{next} versions are switched; \code{now} becomes \code{next} and
\code{next} becomes \code{now}.\footnote{Clearly this idea can be
  extended to $k$-buffered clocked types where each clock tick rotates
  the buffer. This idea is related to the K-bounded Kahn networks of \cite{k-bounded-kahn}.}

We show that clocked types can be defined in a safe way, provided that
accumulators are used to resolve write-write conflicts. The only
dynamic check needed is that a value of this type is being operated
upon only by the activity that created the value or its
descendants. 

In the following by Safe X10 we shall mean the language X10 restricted
to use (\code{clocked}) \code{finish}/\code{async}, \code{at} with
(clocked) accumulators.  All programs in Safe X10 are safe -- they can
be run with a serial schedule and their I/O behavior is identical
under any schedule.  We show that Safe X10 is surprisingly
powerful. Many concurrent idioms can be expressed in this language --
histograms, all-reduce, SpecJBB-style communication Indeed, even some
form of pipelined/systolic communication is expressible.

Since the dynamic conditions introduced on accumulators and clocked
types are not straightforward, we formalize the concurrent and serial
semantics of an abstraction of Safe X10 using Plotkin's structural
operational style. We are able to do this in such a way that the two
proof systems share most of the proof rules, simplifying the proof. We
establish that the language is safe -- for any program and any input,
any execution sequence for the concurrent proof rules can be
transformed into an execution sequence for the sequential proof rules
with the same result. 

In summary the contributions of this paper are as follows.

\begin{itemize}
\item We identify the notion of a {\em safe} program -- one which can
  be executed with a serial schedule and for which 
  every schedule produces the same result. Such a program is
  simultaneously a sequential program and a parallel program with
  identical I/O behavior. 
\item We introduce accumulators and clocked types in the X10
  programming model. These are introduced in such a way that arbitrary
  programs using (\code{clocked}) \code{finish}, \code{async} and
  \code{at} and in which the only variables shared between
  concurrently executing activities are accumulators or clocked
  accumulators are guaranteed to be safe.  
\item We show that many common programming idioms can be expressed in
  this language.
\item We formalize a fairly rich subset of X10 -- including (clocked)
  finish, async, accumulators and clocked accumulators.  This is the
  first formalization of the nested clock design of X10 2.2, and is
  substantially simpler than \cite{vj-clock}. We establish that this
  language is safe. 
\end{itemize}
In companion work we show how these ideas can be extended to support
modularly defined effects analyses, using X10's dependent type system.

The rest of this paper is as follows. In \Ref{related-work} we discuss
related work. In \Ref{constructs} we present the constructs in detail,
followed by examples of their use. \Ref{semantics} presents the
semantics of these constructs. We discuss implementation in
\Ref{implementation} and finally conclude with future work.

%%\cite{Gifford:1986:IFI:319838.319848}
%%
%%
%%Figure~\ref{fig:1} shows  the famous ``parallel Or'' program of Plotkin
%% (in X10 syntax, \cite{x10}). This program can be executed with a
%%depth-first schedule, is partially determinate and deadlock-free, but {\em not}
%%safe. The result of running the sequential schedule is not the same as
%%the result that can be obtained with other schedules. Specifically
%%\code{parallelOr(()=> CONT, ()=>TRUE)} will diverge (exhibit an
%%infinite exection sequence) under the depth-first schedule, but will
%%return \code{true} under any fair schedule that permits the second
%%async to progress.
%%
%%\begin{figure}
%%  \begin{lstlisting}
%%static val CONT=1, TRUE=2, FALSE=3;
%%def run(done:Cell[Boolean], a:()=>Int) {
%% var aa:Int=a();
%% var cont:Boolean=true;
%% for (; aa==CONT && cont;aa=a()) {
%%  atomic cont = !done();
%% }
%% if (aa==TRUE)
%%  atomic done()=true;
%%}
%%def parallelOr(a:()=>Int, b:()=>Int):Boolean {
%% val done=new Cell[Boolean](false);
%% finish {
%%  async run(done, a);
%%  async run(done, b);
%% }
%% return done();
%%}
%%  \end{lstlisting}
%%  \caption{A program that is not sequential}\label{fig:1}
%%\end{figure}



%%{\em
%%\begin{enumerate}
%%\item Use activity registration as a mechanism to tame object graphs.
%%\item Focus on structured concurrency. Using scoping and block-structure
%%    to delimit regions of code that may execute in parallel and affect
%%    the data structure.
%%
%%\item Accumulation can be defined safely by delaying. However, the delay
%%    operation is guaranteed to be deadlock-free.
%%
%%\item Clocked types support phased computation, another common idiom
%%    particularly for stencil computations.
%%\end{enumerate}
%%}
%%
%%Key contributions:
%%{\em
%%\begin{enumerate}
%%\item Identification of determinate, deadlock-free data-structures.
%%\item Discussion of design alternatives which points out the
%%  difficulty of integrating these ideas in a modern OO language.
%%\item Discussion of various idioms expressible using these data-structures.
%%\item Proof of determinacy and deadlock-freedom in an abstract version
%%  of the language.
%%\end{enumerate}
%%These constructs are implemented in \Xten, available as open source from
%%SVN head and will be in the next release of \Xten.
%%}


%Semantics and theorems for an abstract version of the language.





\Section[designs]{Object Initialization Designs}


\subsection{Default values design}

Java first initializes fields with either $0$, \texttt{false}, or \texttt{null}
(depending on the field type) and then running the constructor to initialize
the fields according to the programmer's wishes.  If every X10 type had a
default value that was statically known, then Java's object initialization
scheme could be used in X10.  This would have the advantage of familiarity for
Java programmers that are learning X10.  The disadvantages are that that it is
nonintuitive that final fields can be observed to change value, and that it is
prone to undetectable errors where the field is read before initialization.

Unfortunately it is hard to reconcile the notion of a default value with X10's
type system, because a programmer can define a type which does not contain the
default value.  In the X10 type system, one can define a type with no values at
all, by using a constraint that yields contradiction.

This could be addressed by extending the X10 type to require that the
programmer define a new constant value for any type that has been constrained
enough that the original default value is no longer a member of the type.  This
means every field can be initialized to the value defined in its type.  The
disadvantage of this is that the type system becomes more complex and more type
annotations are required.  We decided that this, in combination with the
disadvantages given above, was too problematic to justify the advantages of
Java-style object initialization.

\subsection{Proto Design}

If we want to allow some of the programs that the Hardhat design rejects, such
as immutable cycles in the object graph, but we do not want to burden the type
system with default value annotations, then one solution is to allow
\texttt{this} to escape in certain cases while still preventing reads from
uninitialized fields.  This can be achieved by annotating reference types with
a keyword \texttt{proto} to indicate that the referenced object is partially
constructed.  Reads of fields where the target object is \texttt{proto}
are not allowed because a partially constructed object may not yet have
initialized its fields.  The advantage of this approach is that it allows a set
of partially constructed objects to establish themselves as a mutually
referential cycle of objects in the heap, which would not otherwise be possible.
The disadvantage is that it requires an additional type annotations, although this
annotation is only required in code that creates immutable cyclic heap
structures.  Also note that there are no additional space or runtime overheads
since these extra type system mechanisms are for static checking only.

An example of an immutable cycle of two nodes is given in
fig.\ref{Figure:Cyclic}.  A more practical but less concise example would be an
immutable doubly-linked list.  Let us assume that we would like to optimize
away any null pointer checks, so we constrain all references to exclude the
null value.  The commented out lines indicate code that would be rejected by
the type system.

\begin{figure}
\begin{lstlisting}
class C {
  val next : C {self!=null};
  var fld : C;
  def this(n:proto C{self!=null}) {
    //Console.OUT.println(n.next); //err1
    //n.f(); //err2
    this.next = n;
  }
  def this() {
    //Console.OUT.println(this.next); //err3
    //this.f(); //err4
    val c = new C(this);
    //Console.OUT.println(c.next.next); //err5
    this.next = c;
  }
  def f() {
    Console.OUT.println(this.next);
  }
  def this(c:C, Int) {
    //c.m(this); //err6
    Console.OUT.println(c.fld.next);
    this.next = c;
  }
  void m (n : proto C) proto {
    this.fld = n;
  }
  static def test() {
    val c:C{self!=null} = new C();
    assert c.next.next==c;
  }
}
\end{lstlisting}
\caption{An immutable cycle of heap references, using \texttt{proto}.}
\label{Figure:Cyclic}
\end{figure}

In the public constructor, \texttt{this} is a pointer to a partially
constructed object.  If the type of \texttt{this} were to be explicit, it would
be \texttt{proto C \{self!=null\}}.  The \texttt{proto} element of the type
forbids any field reads.  It also prevents the reference being leaked (e.g.
into \texttt{f()}), except into variables of \texttt{proto} type where it
follows that there is protection from uninitialized field reads.

The private constructor's \texttt{n} parameter takes a \texttt{proto} pointer
to the original \texttt{C} instance.  It is limited in what it can do with
\texttt{n}, e.g. it cannot read \texttt{n.next}, but it can initialize its own
\texttt{next} field with the passed-in value.

When the public constructor returns, both objects are fully constructed with
all fields initialized.  Thus, the type of the variable \texttt{c} does not
have a proto annotation and the field read \texttt{c.next} is allowed.

If a type has the \texttt{proto} keyword, then its fields (both var and val)
may have partially constructed objects assigned to them, but fields may not be
read.  Conversely, the absence of \texttt{proto} means that the fields may be
read but var fields may not have partially constructed objects assigned to them.
This means that \texttt{proto C} and \texttt{C} are not related by sub-typing.
In other words, \texttt{proto C} means definitely partially constructed and
\texttt{C} means definitely fully constructed.  Consequently it makes no sense
to allow casting between the two types, and one may not extend or implement a
proto type.  The only sources of \texttt{proto} typed objects are via the
\texttt{this} keyword in a constructor and via method parameters of
\texttt{proto} type.  The only way a type can lose its \texttt{proto} is by
becoming fully constructed.

Consider \code{\itshape err5} in fig.~\ref{Figure:Cyclic}.
If we had inferred the type of \code{c} to be non-proto,
    then we could have read the uninitialized field \code{this.next}.
To solve this problem, we must ensure that the whole cycle becomes fully
constructed together.
This can be arranged by changing the type of \texttt{new
C(...)} to be \texttt{proto C} if one of its arguments is of \texttt{proto}
type.  This does not affect the assignment \texttt{this.next = c} because
\texttt{this} is \texttt{proto}.

We do not allow fields to have \texttt{proto} type.  This is because the
referenced object will eventually be fully constructed and then there would be
a variable of \texttt{proto} type pointing to a fully constructed instance.
This admits the possibility of someone assigning a partially constructed object
to a field of the fully constructed object, just as was done in the private
constructor in fig.~\ref{Figure:Cyclic}.  Then, one could accidentally read an
uninitialized field from the partially constructed object by going through the
fully constructed object.  Disallowing \texttt{proto} in fields avoids this
problem.  However local variables are safe because of lexical scoping, they
will go out of scope before the constructor returns and the object becomes
fully constructed.

If one examines the state of the heap as these examples execute, there can be
seen to be a subgraph of 2 partially-constructed objects, which is completely
isolated except for references from the stack of the thread which is executing
the constructors.  The type rules described hitherto ensure the heap has the
property that partially constructed objects are isolated from the rest of the
heap until their construction is completed.

We are not aware of any utility in throwing or catching \texttt{proto} types so
we avoid issues relating to partially constructed objects escaping via the
exception mechanism, by simply forbidding the throwing and catching of
\texttt{proto} exceptions.

There would be an issue calling other instance methods on \texttt{this} from a
constructor, because the type of \texttt{this} in those methods would need to
be \texttt{proto} since the target is still partially constructed.  We support
this by allowing the \texttt{proto} keyword to also be used on a method as an
effect annotation, i.e. it must be preserved by inheritance.  Such methods are
called \texttt{proto} methods and can be called on partially constructed
targets.  The type of \texttt{this} then subjects the body of the method to the
same restrictions as we have already seen in constructor bodies.

However in some cases we would like to avoid code duplication by allowing some
methods to be callable on both \texttt{proto} and non-\texttt{proto} targets.
This violates our principle that the two kinds of objects enjoy different
privileges and are completely distinct.
The error \code{\itshape err6} in fig.~\ref{Figure:Cyclic}
shows how we could potentially read an uninitialized field if we allowed this
relaxation.

To address this, we only allow the method to be called on non-\texttt{proto}
targets if there are no \texttt{proto} parameters to the method.  No such
parameters means the only partially constructed object in scope is
\texttt{this}.  In the case where the method is called on a non-\texttt{proto}
target there is therefore no partially constructed object in scope, and no harm
can be done.

While we believe this type system is correct and usable for writing real
programs in the X10 language, we had to decide whether the additional type
system complexity and annotations were a reasonable price to pay for the
additional expressiveness (i.e. the ability to construct immutable heap
cycles).  We ultimately decided that immutable heap cycles are too rare in
practice to justify including these extra mechanisms in the language.


\Section[implementation]{Hardhat implementation}
implementation design, overheads, some measurements, etc.

outlines our implementation within the X10 compiler using the polyglot framework,
    the compilation time overhead of checking these initialization rules,
    and the annotation overhead in our X10 code base.


Due to page limitation, we mainly focused on the formal effect system for POPL,
but we can easily add an empirical evaluation section that describes some test cases (where minor code refactoring was needed) and shows the annotation burden.
X10 has only two possible method annotations: @NonEscaping, @NoThisAccess.
Methods transitively called from a constructor are implicitly non-escaping (but the compiler issues a warning that they should be marked as @NonEscaping).
SPECjbb and M3R are closed-source whereas the rest is open-source and publicly available at x10-lang.org

------------------------------------------------------------------------------
Programs:           XRX SPECjbb     M3R UTS Other
# of lines          27153   14603       71682   2765    155345
# of files          257 63      294 14  2283
# of constructors       276 267     401 23  1297
# of methods            2216    2475        2831    124 8273
# of non-escaping methods   8   38      34  3   83
# of @NonEscaping       7   7       13  1   62
# of @NoThisAccess      1   0       1   0   12
------------------------------------------------------------------------------
XRX: X10 Runtime (and libraries)
SPECjbb: SPECjbb from 2005 converted to X10
M3R: Map-reduce in X10
UTS: Global load balancing
Other: Programmer guide examples, test suite, issues, samples
------------------------------------------------------------------------------

As can be seen, the annotations burden is minor.

Asynchronous initialization was not used in our applications because they pre-date this feature.
(It is used in our examples and tests 50 times.)
However, it is a useful pattern, especially for local variables.
More importantly, the analysis prevents bugs such as:
val res:Int;
finish {
  async {
    res = doCalculation();
  }
  // WRONG to use res here
}
// OK to use res here

Here are two examples for the use of annotations:
1) In Any.x10 we have:
@NoThisAccess def typeName():String
Method typeName is overridden in subclasses to return a constant string (all structs automatically override this method).
This annotation allows typeName() to be called even during construction.
2) In HashMap.x10, after we added the strict initializations rules, we had to refactor put and rehash methods into:
public def put(k: K, v: V) = putInternal(k,v);
@NonEscaping protected final def putInternal(k: K, v: V) {...}
(Similarly, we have rehash() and rehashInternal())
The reason is that putInternal is called from the deserialization constructor:
def this(x:SerialData) { ... putInternal(...) ... }
And we still want subclasses to be able to override the "put" method.


\Section[case-study]{Case Study}
a lot of Java code was recently translated to X10, and Java is less strict regarding initialization. How did affect the translation?


\Section[related-work]{Related Work}
A static analysis \cite{Seo:2007:SBD:1522565.1522587}, has been used to find
some default value reads in Java programs, and supports our belief that default
value reads can be found in real programs and should be considered errors.  Our
approach is stronger (detecting all errors at the expense of some correct
programs) and considers additional language constructs that are not present in
Java.

There has been a study on a large body \cite{Gil:2009:WRS:1615184.1615216} of
Java code, showing that initialization order issues pervade projects from the
real world.  A bytecode verification system for Java initialization has also
been explored \cite{Hubert:2010:ESO:1888881.1888890}.

An early work to support non-null types in Java
\cite{Fahndrich:2003:DCN:949305.949332} has the notion of a type constructor
$raw$ that can be applied to object types and means that the fields of the
object (in X10 terminology) may violate the constraints in their types.  This
simply disables the type-system while an object is partially constructed while
ensuring the rest of the program is typed normally.  Our approach prevents
errors during constructors as it does not disable the type-system, and it also
permits optimisation of the representation of fields whose types are very
constrained, since they will never have to hold a value other than the values
allowed by their type constraint.

A later work \cite{Fahndrich:2007:EOI:1297027.1297052,XinQi:2009} allows
    to specify the time
    (in the type) when the object will be fully constructed.
Field reference types of a partially constructed objects must be fully
constructed by the same time, which allows graphs of objects to be constructed
like our \code{proto} design.  However the system is more complicated, allowing
the object to become fully constructed at a given future time, instead of at
the specific time when its constructor terminates or the last object that links
to it becomes fully constructed (whichever is later).
%We did need this functionality.

Expression of object constructedness has also been explored at the level of
individual fields \cite{XinQi:2009}.  Each type gives the fields which have not
yet been initialised.  Our type system is simpler but less expressive.
%as we only distinguish between the empty set of fields, and some non-empty set of fields.

Ownership types can be used to create immutable cycles~\cite{Zibin:2010:OIG:1869459.1869509}.
This is comparable to
our \code{proto} design because it also allows \code{this} to be linked from an
incomplete object.  However the ownership structure is used to implement a
broader policy, allowing code in the owner to use a reference to its partially
constructed children, whereas we only allow code to use a reference to objects
that are being partially constructed in some nesting stack frame.  However our
approach does not use ownership types.

There is also a time-aware type system \cite{Matsakis:2010:TTS:1869459.1869511}
that allows the detection of data-races, and understands the concept of shared
variables that become immutable only after a certain time
(and can then be accessed without synchronisation).  The same mechanisms can also be
used to express when an object becomes cooked.


\Section[conclusion]{Conclusion}
% 8/17/2011 5:03:31 AM

\subsubsection{Implementation}
An early version of this design was implemented in a branch of the X10
system. Results were obtained for about twenty benchmarks, including
JGF benchmarks such as IDEA, Sor, Series, RayTrace, LUFact,
SparseMatMul, Geometric Mean.  The speed of the programs written using
accumulators and clocked types was compared to the corresponding
programs written in plain X10. On the average we noticed about a 20\%
degradation in performance. 

A new implementation is currently being worked on and we expect to
have more up-to-date results for final submission. This implementation
rides piggyback on the control messages exchanged to implement
\code{finish} in order to update an \code{Acc} from multiple
places. The rules for reads on \code{Acc} guarantee that intermediate
accumulations can be stored at each place and do not need to be
communicated to the place where the \code{Acc} lives until local
computation has quiesced. 
This results in a very efficient implementation. 

\subsection{Adding static effects checking}

Safe X10 can be substantially enriched
through the addition of statically checked effects
\cite{Gifford:1986:IFI:319838.319848},\cite{DPJ}, particularly to
handle non-array-based computations. This section summarizes work
currently in progress.

X10 already implements a very powerful dependent type system based on
constraints. An object \code{o} is of type \code{T\{c\}} if it is of
type \code{T} and further satisfies the constraint
\code{c[o/self]}. Thus \code{Array[T]\{self.rank==R\}} is the type of all
the arrays of \code{T} with rank \code{R}.

Types classify objects, i.e.{} types specify sets of
objects. Therefore we can define a {\em location set} to be of the form
\code{T.f} where \code{T} is a type and \code{f} is a mutable field of
type \code{T}. It stands for the set of locations \code{x.f} where
\code{x} is of type \code{T}. We do not need to introduce any distinct
notion of regions into the language.

The key insight is that two memory locations are distinct if they
either have different names or they have the same name \code{f} and
they live in two objects \code{m} and \code{n} such that for some
property \code{p}, \code{m.p != n.p} -- for then it must be the case
that \code{m != n}. Therefore we introduce enough properties into
classes to ensure that we can distinguish between objects that are
simultaneously being operated on. Specifically, the location sets
\code{S\{c\}.f} and \code{S\{d\}.f} are disjoint if
\code{c[x/self],d[x/self]} is not satisfiable for any \code{x:S}.

Methods are decorated with \code{@Read(L)} and \code{@Write(M)}
annotations, where \code{L},\code{M} are location sets. The annotation
is {\em valid} if every read (write) of a mutable location \code{o.f} in the
body of the method it is the case that \code{o.f} lies in (the set
described by) \code{L} (\code{M}).

We permit location sets to be named:
\begin{lstlisting}
  static locs Cargos = Tree.cargo;
  static locs LeftCargo(up:Tree) = Tree{self.up==up,self.left==true}.cargo;
\end{lstlisting}

Unlike the region system of \cite{DPJ} we do not need to introduce
explicitly named, intensional regions, rather we can work with
extensional representations of location sets (the set of all locations
satisfying a certain condition).  There is no need for a separate
space of region names with constructors. Two location sets are
disjoint if their constraints are mutually unsatisfiable, not because
they are named differently. In particular, we do not need to assume
that the heap is partitioned into a tree of regions.

Two key properties of the X10 type system are worth recalling. The run-time
heap cannot contain cycles involving only properties. This in turn
depends on the fact that the X10 type system prevents \code{this} from
escaping during object construction\cite{X10-object-initialization}.

We mark a property as \code{@ghost} to indicate that space is not
allocated at run-time for this property. Therefore the value of this
property is not accessible at run-time -- it cannot be read and stored
into variables. It may only be accessed in constraints that are
statically checked. Dynamic casts cannot refer to \code{@ghost}
properties.

We enrich the vocabulary of constraints. First, we permit
existentially quantified constraints \code{x:T\^c} -- this represents
the constraint \code{c} in which the variable \code{x} of type
\code{T} is existentially quantified. Second, we permit the {\em
  extended field selector} \code{e.f\$i} where \code{e} is an
expression, \code{f} a field name and \code{i} is a \code{UInt}.

%% Alternatively we just permit regular expressions as field
%% selectors?
%% So take T{self.r==x} where r is a regular expression to mean
%% T{p:??^(p in r, self.p==x)}. So we have to have a notion of a
%% type-consistent path, and then be able to select a path on a given
%% receiver, maybe use some other operator for this, and not .?


%% Hmm dont want to be able to count and do arithmetic, that will
%% take us out of reg exp land.

%% Do we need MONA?
%% May need to permit full-fledged regular expressions.

% Where do we need the flexibility of saying two region parameters are disjoint?

We use an example from \cite{DPJ} to illustrate:
  \begin{lstlisting}
type Tree(t:Tree,l:Boolean)=Tree{self.up==t,self.left==l};
class Tree (up:Tree, left:Boolean) {
    var left:Tree(this,true);
    var right:Tree(this,false);
    var payload:Int;
    //SubTree(t) is the type of all Trees o s.t.
    // for some UInt i, o.up...up=t (i-fold iteration).
    static type SubTree(t:Tree)=Tree{i:UInt^self.up$i==t};
    @Safe
    def makeConstant(x:Int)
        @Write[SubTree(this)].payload
        @Read[SubTree(this)].(left,right) {
          finish {
            this.payload = x;
            if (left != null)
              async left.makeConstant(x);
            if (right != null)
              async right.makeConstant(x);
          }}}
  \end{lstlisting}
In this example the fields \code{left},\code{right} and \code{cargo}
are mutable. It is possible to mutate a tree \code{p} -- replace
\code{p}'s \code{left} child with another tree \code{q}. However, \code{q} cannot be in
\code{p}'s \code{right} subtree, because then its \code{up} field or
\code{left} field would not have the right value. That is, once a
\code{Tree} object is created it can only belong to a specific tree in
a specific position.

Note that a \code{Tree} can be created with \code{null} parent and
children. This is how the root is created:
\code{new Tree(null,true)} or \code{new tree(null, false)}.
%%
%%In checking the validity of the effect annotation on
%%\code{makeConstant} the compiler must check that the effects of the
%%body are contained in the declared effects of the method. This boils
%%down to checking the subtyping relations. e.g.{} for the \code{@Write} annotation it must check:
%%\begin{lstlisting}
%%SubTreeTree
%%Tree{i:UInt^(self.up$i.up==this,self.up$i.left==true} <: Tree{i:UInt^(self.up$i==this}
%%Tree{i:UInt^(self.up$i.up==this,self.up$i.left==false} <: Tree{i:UInt^(self.up$i==this}
%%Tree{self==this} <: Tree{i:UInt^(self.up$i==this)}
%%\end{lstlisting}

This is easily verified.
The notion of ``distinctions from the left'' and ``distinctions from
the right'' of \cite{DPJ} are not needed. These arise naturally
through the use of constraints (distinct access paths, vs distinct fields).

Note that the programmer may not desire to have the fields \code{up}
and \code{left} be available at run-time. These fields can be marked
as \code{@ghost} -- any attempt to access them at run-time will result
in an error.

Constraints such as ``this array must point to distinct objects'' can
also be naturally represented in the dependent type system. For more
details please see \cite{effects-constrained-types}.

\subsection{Conclusion}

This paper presents a lightweight design for a safe language which
permits very rich expression of concurrency idioms.



\bibliographystyle{plainnat}
\bibliography{x10-init}

\end{document}
