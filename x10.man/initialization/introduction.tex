Constructing an object in a safe way is not easy:
    it is well known that dynamic dispatching
    or leaking \this during object construction
    is error-prone~\cite{Dean:1996,Seo:2007:SBD:1522565.1522587,Gil:2009:WRS:1615184.1615216},
    and various type systems and verifiers have been proposed to
    handle safe object initialization~\cite{Hubert:2010:ESO:1888881.1888890,Zibin:2010:OIG:1869459.1869509,Fahndrich:2007:EOI:1297027.1297052,XinQi:2009}.
As languages become more and more complex,
    new pitfalls are created due to the interactions between
    language features.

X10 is an object oriented programming language with a sophisticated
    type system (constraints, class invariants, non-erased generics, closures)
    and concurrency constructs (asynchronous activities, multiple places, global references).
This paper shows that object initialization is a cross-cutting concern
    that interact with other features in the language.
We discuss several language designs that restrict these interactions,
    and explain why we chose the \emph{hardhat} design for X10.

{Hardhat} was termed in~\cite{Gil:2009:WRS:1615184.1615216}
    and it describes a design that prohibits dynamic dispatching or leaking \this during construction.
A hardhat design limits the user
    but also protects her from future bugs.
X10's hardhat design is even stricter due to additional language features
    such as concurrency, places, and closures.

On the other end of the spectrum,
    Java and C\# allow
    dynamic dispatching and leaking \this.
However, they still maintain type- and runtime- safety
    by relying on the fact that every type has a default zero value
    (whether that zero is 0, false, or \code{null}),
    and all fields are zero-initialized before the constructor begins.
As a consequence,
    a half-baked object can leak before all its fields are set. %\cite{Seo:2007:SBD:1522565.1522587} - reading uninitialized field references
Phrased differently,
    when reading a final field, one can read the default value initially and later read a different value.
Another source of subtle bugs is due to the synchronization barrier
    at the end of a constructor~\cite{JSR133}
    after which all assignments to final fields are guaranteed to be written.
The programmer is warned (in the documentation only!)
    that objects with final fields are thread-safe only if
    \this does not escape its constructor.
%Before JSR 133~\cite{JSR133},
%    immutable objects could have different values in different threads
%    if synchronization was not used properly.
Finally, if the type-system is augmented, for example, with non-null types, then
    a default value no longer exists,
    which leads to complicated type-systems for initialization~\cite{Fahndrich:2007:EOI:1297027.1297052,XinQi:2009}.

\mbox{C++}, as usual, gives you enough rope to hang yourself.
Fields are not zero-initialized, and therefore if \this leaks,
    one can read an uninitialized field.
Moreover, method calls are statically bound during construction,
    which may result in an exception at runtime
    if one tries to invoke a virtual method of an abstract class (see \Ref{Figure}{Dynamic-dispatch} below).
(Determining whether this happens is an intractable problem~\cite{Gil:1998:CTA:646155.679689}.)


The remainder of this introduction presents examples in X10
    that slowly add language features and describes their interaction with
    object initialization.
The rules of the hardhat design are slowly introduced by example.
Violation of the rules are marked with \code{//err},
    and these match errors issued by the X10 compiler.
The hardhat rules are summarized in \Ref{Section}{hardhat}.

\subsection{Constructors and inheritance}
%Methods in X10 are denoted by \code{def}, and \code{val} denotes a final field.
Object initialization begins by invoking a constructor, denoted by \code{def this()}.

Consider \Ref{Figure}{Escaping-this}.
The first leak causes a problem because field \code{a} was not assigned yet.
However, leaking is still a problem even after all fields have been assigned,
    because fields in a subclass (field \code{b}) have not been initialized yet.
Leaking and aliasing is not a problem if \this is not raw, e.g., in \code{m1()}.

The fact that \code{m2} is non-escaping was implicitly inferred from the call to \code{m2}
    in the constructor.
(From Rule 2, because \code{m2} was called on a raw \this receiver.)
The user could also mark \code{m2} explicitly as non-escaping by using the annotation
    \code{@NonEscaping},
    which is also useful for \code{super} calls (see \Ref{Figure}{Dynamic-dispatch}).

\begin{figure}
\begin{lstlisting}
class A {
  val a:Int;
  def this() {
    LeakIt.foo(this); //err
    this.a = 1;
    val me = this; //err
    LeakIt.foo(me);
    this.m2();
  }
  def m1() {
    LeakIt.foo(this);
  }
  final def m2() {
    LeakIt.foo(this); //err
  }
}
class B extends A {
  val b:Int;
  def this() {
    super();
    this.b = 2;
  }
}
\end{lstlisting}
\caption{Escaping \this example.
    Rule 1: \this and \code{super} are raw in \emph{non-escaping} methods and in field initializers.
    Rule 2: A method is \emph{non-escaping} if it is a constructor,
            or a method that is called on a raw \this receiver.
    Rule 3: A raw \this or \code{super} cannot escape or be aliased.
    (The usage of \code{final} on \code{m2} will be explained in \Ref{Figure}{Dynamic-dispatch}.)}
\label{Figure:Escaping-this}
\end{figure}



\subsection{Dynamic dispatch}
Dynamic dispatching interacts with initialization by transferring control to the subclass
    before the superclass completed its initialization.
For example, if X10 would allow calling the non-final \code{m1} in \Ref{Figure}{Dynamic-dispatch},
    then it would dynamically call the implementation in \code{B} that could read the field \code{a}
    before it is initialized.

The hardhat prevents dynamic dispatching by permitting only private and final non-escaping methods
    (so overriding a non-escaping method is impossible).
However, sometimes dynamic dispatching is required during construction.
For example, if a subclass needs to refine initialization
    of the superclass's fields.
Such refinement cannot have any access to \this, and therefore
    such methods are marked with \code{@NoThisAccess}.
(Like in Java, \code{@} is used for annotations in X10.)
\code{@NoThisAccess} prohibits any access to \this,
    however, one could still access the method parameters.



\begin{figure}
\begin{lstlisting}
abstract class A {
  val a:Int;
  def this() {
    this.a = m1(); //err
    this.a = m2();
  }
  abstract def m1():Int;
  @NoThisAccess abstract def m2():Int;
}
class B extends A {
  var b:Int = 3;
  def this(i:Int) {
    super(i);
  }
  def m1() {
    val y = this.b; // would have returned 0
    return 1;
  }
  @NoThisAccess def m2() {
    val x = super.a; //err
    val y = this.b; //err
    return 2;
  }
}
class C {
  @NonEscaping final def m3() {
    LeakIt.foo(this); //err
  }
  final def m4() {
    LeakIt.foo(this);
  }
}
class D extends C {
  def this(i:Int) {
    super.m3();
    super.m4(); //err
  }
}
\end{lstlisting}
\caption{Dynamic dispatch and calling \code{super} methods examples.
    Revised rule 2: A method is \emph{non-escaping} if
        it is annotated with \code{@NoThisAccess} or \code{@NonEscaping}.
    Rule 4: a non-escaping method must be private or final, unless it has \code{@NoThisAccess}.
    Rule 5: A method with \code{@NoThisAccess} cannot access \this or \code{super} (neither read nor write its fields).
    Rule 6: A call on a raw \code{super} is allowed only for a \code{@NonEscaping} method.
    }
\label{Figure:Dynamic-dispatch}
\end{figure}

In C++, the call to \code{m1} is legal, but it gives you a runtime error
    that you called a pure virtual function.

In Java, the call is also legal, but at runtime,
    reading \code{b} in \code{B.m1} would return the default value (0) and not 3.
This behavior is undesired in Java, and Java tries to discourage it by catching some of these cases.
For example, Java prohibits calls to member functions before the super object was initialized,
    as this example shows (it's also illegal in X10):
\begin{lstlisting}
class B extends A { B() {super(f()); }}
\end{lstlisting}



\subsection{Exceptions}


\begin{figure}
\begin{lstlisting}
class B extends A {
  var i:Int=2;
  def this() {
    try { //err
      super(i); //err
    } catch(Throwable e){}
  }
}
\end{lstlisting}
\caption{Exceptions example.
    Rule 7: No statements can preceded a constructor call, i.e., either \code{super(...)} or \code{this(...)}.
    Rule 8: \this and \code{super} can be used only after the (implicit or explicit) constructor call.}
\label{Figure:Exceptions}
\end{figure}

Rules 7 and 8 are also checked in Java.
Failure to implement this verification properly led to a famous attack~\cite{Dean:1996}.


\subsection{Inner classes}
\begin{figure}
\begin{lstlisting}
class A {
  class Inner {
    val b = a;
  }
  val inner =
    this.new Inner(); //err
  val a = 3;
}
\end{lstlisting}
\caption{Inner class example.
    Rule 8: a raw \this cannot be the receiver of \code{new}.}
\label{Figure:InnerClass}
\end{figure}


\subsection{Static initialization}
todo

\subsection{Read and write order for final fields}
Inter-procedural data flow
todo


\subsection{Read and write order for \emph{non}-final fields}
Intra-procedural data flow
todo




We finally turn our attention to features that are unique to X10:
    concurrency, parallelism,
    and a sophisticated type system using constraints and properties.

\subsection{Concurrency}
Explain async+finish.
If the finish was removed, then it would be an error ...

\begin{figure}
\begin{lstlisting}
class A {
  val a:Int;
  val b:Int;
  def this() {
    finish {
      async a = 1;
    }
    async b = 2; //err
  }
}
\end{lstlisting}
\caption{Asynchronously assigned fields example.
    X10: all field assignments must finish when the constructor finishes.}
\label{Figure:Asynchronously-init}
\end{figure}


\subsection{Multiple Places}
Places require serialization and deserialization (both custom and automatic) across "at".

\begin{figure}
\begin{lstlisting}
class A {
  val a:Int;
  def this() {
    // Execute at another place
    at (here.next())
      this.a = 1; //err
  }
}
\end{lstlisting}
\caption{Multi-place initialization example.
    X10: a raw \this cannot cross to another place.}
\label{Figure:Multi-place}
\end{figure}


\subsection{Global references}
Special GlobalRef class: let \this escape and it creates a cycle.
We allow a raw \this to escape iff
* field is private with a field initializer.
* cannot be used with a raw \this receiver.

\begin{figure}
\begin{lstlisting}
class A {
  private val root =
   new GlobalRef(this);
  def me() = root();
}
class B extends A {
  def this() {
    val alias = me(); //err
  }
}
\end{lstlisting}
\caption{\code{GlobalRef} example.
    X10: ...}
\label{Figure:GlobalRef}
\end{figure}

If \code{me()} was prefixed with
\code{@NonEscaping public final}
then accessing \code{root} would be an error.
%Cannot use 'root' because a GlobalRef[...](this) cannot be used in a field initializer, constructor, or methods called from a constructor.

\subsection{Constraints}
Constraints and default values.
The following types do not have a default value:
\code{Int\lb self!=0\rb}
\code{String\lb self!=null\rb}

Therefore the fields of an object cannot be zero-initialized in X10.

\begin{figure}
\begin{lstlisting}
class A {
  var a:Int{self!=0}; //err
}
\end{lstlisting}
\caption{No default value example.
    X10: you must assign to \code{var} fields without a default value.}
\label{Figure:Constraints}
\end{figure}


\subsection{Properties}
Properties are final values that can be used in constraints,
    e.g., \code{Array} has a \code{size} property,
    so an array of size 2 can be expressed as: \code{Array\lb self.size==2\rb}.

Properties are final fields that are initialized before all other fields.


\begin{figure}
\begin{lstlisting}
class A(a:Int) {
  def this() {
    property(42);
  }
}
class B(b:Int) {b!=0} extends A {
  val f1 = a+b;
  val f2:Int;
  def this() {
    super();
    f2 = f1; // err
    property(3);
  }
}
\end{lstlisting}
\caption{Properties example.
    X10: properties must be initialized before fields.}
\label{Figure:Properties}
\end{figure}


%\subsection{Class Invariant}
% Should I talk about interface and abstract property methods? Doesn't relate to initialization...
The \emph{class invariant} may refer only to properties of the class,
    and it must be satisfied after the property call in every constructor.


\subsection{Closures}

\begin{figure}
\begin{lstlisting}
class A {
  val a = 3;
  def this() {
    val local = this.a;
    val closure1 = ()=>local;
    val closure2 = ()=>this.a; //err
    at (here.next())
      closure2();
  }
}
\end{lstlisting}
\caption{Closure capture \this example.
    X10: a closure cannot capture a raw \this.}
\label{Figure:Closures}
\end{figure}



\subsection{Generics}
Generics are not unique to X10,
    however the combination of generics and the lack of default values for all types
    lead to new pitfalls.
In addition, generics in X10 are \emph{not} erased as in Java.

\Ref{Figure}{Constraints} showed that a \code{var} must be assigned if
    it does not contain the zero value.
For generics, we added a \code{haszero} type condition that requires a type parameter to have the zero value.

\begin{figure}
\begin{lstlisting}
class A[T] {
  var a:T; //err
}
class B[T] {T haszero} {
  var a:T;
}
class Usage {
  var b1:B[Int];
  var b1:B[String];
  var b1:B[Int{self!=0}]; //err
}
\end{lstlisting}
\caption{\code{haszero} type condition.
    X10: statically checks a type has the zero value.}
\label{Figure:Generics}
\end{figure}


examples for \code{Array}

 \code{Zero.get[T]()}
