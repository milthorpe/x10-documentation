\documentclass[preprint,nocopyrightspace,9pt]{sigplanconf}
%\documentclass{llncs}

\newif\iflncs
\lncsfalse

\usepackage{times-lite}
\usepackage{mathptm}
\usepackage{txtt}
\usepackage{stmaryrd}
\usepackage{code}
\usepackage{bcprules}
%\usepackage{ttquot}
\usepackage{amsmath}
\usepackage{amssymb}
\usepackage{afterpage}
\usepackage{balance}
\usepackage{floatflt}
\usepackage{defs}
\usepackage{utils}
\usepackage{graphicx}
\usepackage{xspace}
\usepackage{ifpdf}
\usepackage{listings}
\usepackage{x10}
\usepackage{color}

\ifpreprint
\newcommand\todo[1]{\textcolor{red}{#1}}
\else
\newcommand\todo[1]{}
\fi
\renewcommand\todo[1]{}

  %\hyphenpenalty=1000


\newif\ifsemantics
%\semanticsfalse
\semanticstrue

\hfuzz=1pt

\pagestyle{plain}


\ifpdf
\setlength{\pdfpagewidth}{8.5in}
\setlength{\pdfpageheight}{11in}
\fi

\newcommand\val{\mbox{\tt val}}
\newcommand\klass{\mbox{\sf class}}
\newcommand\var{\mbox{\tt var}}
\newcommand\self{\mbox{\tt self}}
\newcommand\this{\mbox{\tt this}}
\newcommand\new{\mbox{\tt new}}
%\newcommand\extends{\unlhd}
%\newcommand\super{\unrhd}
\newcommand\extends{\mathrel{\mbox{\tt \textlt=}}}
\newcommand\super{\mathrel{\mbox{\tt \textgt=}}}
\newcommand\return{\mbox{\tt return}}
\newcommand\true{\mbox{\tt true}}
\newcommand\false{\mbox{\tt false}}
\newcommand\Object{\mbox{\tt Object}}
\newcommand\as{\mbox{\tt as}}
\newcommand\fields{\mbox{\sf fields}}
\newcommand\methods{\mbox{\sf methods}}
\newcommand\type{\mbox{\tt type}}
\newcommand\mtype{\mbox{\sf mtype}}
\newcommand\field{\mbox{\sf field}}
\newcommand\CFJ{{\sf CFJ}\xspace}
\newcommand\FJ{{\sf FJ}\xspace}
\newcommand\Java{Java\xspace}

\newcommand\Xten{{\sf X10}\xspace}
\newcommand\csharp{C$^{\sharp}$\xspace}
\newcommand\hmx{$\mathrm{HM}(X)$\xspace}
\newcommand\clpx{$\mathrm{CLP}(X)$\xspace}

\newcommand\xbar[1]{\ensuremath{\bar{\Xcd{#1}}}}
\newcommand\tbar[1]{\ensuremath{\bar{\tt {#1}}}}
\newcommand\exc[2]{\ensuremath{\exists}#1.~#2}
\newcommand\exty[3]{\ensuremath{\exists}#1\ty#2.~#3}
\newcommand\extyty[5]{\ensuremath{\exists}#1\ty#2,#3\ty#4.~#5}
\newcommand\extytyty[7]{\ensuremath{\exists}#1\ty#2,#3\ty#4,#5\ty#6.~#7}
\def\inv{\mathit{inv}}

\def\FGJ{{\sf FGJ}\xspace}

\newcommand\FXGL[1]{{\sf FXG(${\cal #1}$)}}
\def\FXGZ{\FXGL{\cdot}}
\def\FXGD{\FXGL{Q}}
\def\FXG{{\sf FXG}\xspace}

\def\has{\mbox{\tt has}}
\def\TConstr{\mbox{\sc T-Constr}}
\def\TInv{\mbox{\sc T-Inv}}
\def\TVar{\mbox{\sc T-Var}}
\def\TField{\mbox{\sc T-Field}}
\def\TInvk{\mbox{\sc T-Invk}}
\def\TNew{\mbox{\sc T-New}}
\def\TCast{\mbox{\sc T-Cast}}
\def\TUCast{\mbox{\sc T-UCast}}
\def\TDCast{\mbox{\sc T-DCast}}
\def\TSCast{\mbox{\sc T-SCast}}

\def\RField{\mbox{\sc R-Field}}
\def\RCField{\mbox{\sc RC-Field}}
\def\RInvk{\mbox{\sc R-Invk}}
\def\RCInvkRecv{\mbox{\sc RC-Invk-Recv}}
\def\RCInvkArg{\mbox{\sc RC-Invk-Arg}}
\def\RCNewArg{\mbox{\sc RC-New-Arg}}
\def\RCast{\mbox{\sc R-Cast}}
\def\RCCast{\mbox{\sc RC-Cast}}
\def\uhas{\underline{\has}}

\DeclareTextCommand{\textbraceleft}{OT1}{\texttt{\symbol{`\{}}}
\DeclareTextCommand{\textbraceright}{OT1}{\texttt{\symbol{`\}}}}

% \input{../../../../vj/res/pagesizes}
% \input{../../../../vj/res/vijay-macros}
\newcommand\alt{\bnf}

\newcommand\Implies{\Rightarrow}

\iflncs
\else
\newtheorem{example}{Example}[section]
\newtheorem{theorem}{Theorem}[section]
\newtheorem{lemma}[theorem]{Lemma}
\newtheorem{definition}[theorem]{Definition}
\newenvironment{proof}{
\trivlist
\item[\hskip \labelsep \textsc{Proof.}]
\selectfont
\ignorespaces}{$\Box$}

%\newtheorem{proof}[theorem]{Proof}
\fi

\begin{document}

\title{Constrained Kinds}

\authorinfo{Nathaniel Nystrom}{University of Texas, Arlington}
  {nystrom@uta.edu}

\eat{
\authorinfo{Olivier Tardieu}{IBM Research}
  {tardieu@us.ibm.com}
        
\authorinfo{Igor Peshansky}{IBM Research}
  {igorp@us.ibm.com}

\authorinfo{Vijay Saraswat}{IBM Research}
  {vsaraswa@us.ibm.com}
  }

\authorinfo{Olivier Tardieu \and Igor Peshansky \and Vijay
Saraswat}{IBM T.J.~Watson Research Center}
  {\{tardieu,igorp,vsaraswa\}@us.ibm.com}

% \conferenceinfo{POPL'08}{XXX}
% \copyrightyear{2008}
% \copyrightdata{[to be supplied]}

\maketitle

\begin{abstract}
\Xten{} is a modern object-oriented language designed for productivity and
performance in concurrent and distributed systems.  In this setting, dependent
types offer significant opportunities for detecting design errors statically,
documenting design decisions, eliminating costly run-time checks (e.g., for
array bounds, null values), and improving the quality of generated code.

We present the design and implementation of {\em constrained types}, a natural,
simple, clean, and expressive extension to object-oriented programming: A type
\xcd{C\{c\}} names a class or interface \xcd{C} and a {\em constraint} \xcd{c}
on the immutable state of \xcd{C} and in-scope final variables.  Constraints
may also be associated with class definitions (representing class invariants)
and with method and constructor definitions (representing preconditions).
Dynamic casting is permitted.  The system is parametric on the underlying
constraint system: the compiler supports a simple equality-based constraint
system but, in addition, supports extension with new constraint systems using
compiler plugins.


\end{abstract}


\section{Introduction}
\label{sec:intro}
%A {\em serial} schedule for a parallel program is one which always
executes the first enabled step in program order. A {\em safe}
parallel program is one that can be executed with a serial schedule
$S$ and for which for every input every schedule produces the same
result (error, correct termination, divergence) as $S$.  Such a
program is semantically a sequential program, hence it is
scheduler-determinate and deadlock-free.

A {\em safe parallel programming language} is an imperative parallel
programming language in which every legal program is a safe
program. Programmers can write code in such a language secure in the
knowledge that they will not encounter a large class of parallel
programming problems. Such a language is particularly useful for
parallelizing sequential (imperative) programs. In such cases ({\em
  contra} reactive programming) the desired application semantics are
sequential, and parallelism is needed solely for efficient
implementation.

The characteristic property of a safe programming language is that the
{\em same} program has a sequential reading and a parallel reading,
and both are compatible with each other. Hence the program can be
developed and debugged as a sequential program, using the serial
scheduler, and then run the unchanged program in parallel.  Parallel
execution is guaranteed to effect only performance, not
correctness. Safety is a very strong property.

There are many challenges in designing an efficient, usable, powerful,
explicitly parallel imperative language that is safe. One central
challenge is that 

One way to get safety is through implicitly parallel languages (e.g.{}
Jade~\cite{Rinard98thedesign},\cite{vonPraun:2007:IPO:1229428.1229443}). One
starts with a sequential programming language, and adds constructs
(e.g.{} tasks) that permit speculative execution while guaranteeing
that the only observable write to a shared variable is the write by
the last task to execute in program order.  While this work is
promising, extracting usable parallelism from a wide variety of
sequential programs remains very hard.

Explicitly parallel programming languages provide a variety of
constructs for spawning tasks in parallel and coordinating between
them. Here the programmer can typically directly control the
granularity of concurrency, and locality of access (e.g. placement of
data-structures in a multi-node computation) and use efficient
concurrent primitives (atomic reads/writes, test and sets, locks etc)
to control their execution. 

For such languages proving that a program is safe -- much less that
the programming language is safe -- now becomes very hard,
particularly for modern object-oriented languages which allow the
programmer to create arbitrarily complicated data-structures in the
shared heap. It becomes very difficult to show that any possible
schedule will produce the same result as the serial schedule.

Our starting point is the language X10 \cite{x10} since it offers a
simple and elegant treatment of concurrency and distribution, with
some nice properties.  In brief, X10
introduces the constructs \code{async S} to spawn an activity to
execute \code{S}; \code{finish S} to execute \code{S} and wait until
such time as all activities spawned during its execution have
terminated; \code{at (p) S} to execute \code{S} at place
\code{p}. These constructs can be nested arbitrarily -- this is a
source of significant elegance and power. Additionally, X10 v 2.2
introduces a simplified version of X10 clocks (adequate for many
practical usages) -- \code{clocked finish S} and \code{clocked async
  S}. Briefly, a clocked finish introduces a new barrier that can be
used by this activity and its children activities for
synchronization.\footnote{X10 also has a conditional atomic construct,
  \code{when (c) S} which permits data-dependent synchronization and
  can introduce deadlocks. We do not consider this construct in this paper.}

\cite{vj-clock} establishes that a large class of programs
in X10, namely those that use \code{finish}, \code{async} and
\code{clock}s are deadlock-free. The central intuition is that a clock
can only be used by the activity that created it and by its children,
and hence the spawn tree structure can be used to avoid depends-for
cycles. 

To obtain determinacy, another idea is required. The central problem
is to ensure that in the statement \code{async \{S\} S1} it is not the
case that \code{S} can write to a location that \code{S1} can read
from or read from a location that \code{S} can read from. Otherwise
the behavior is not determinate. One line of attack has been the pursuit
of {\em effect systems} \cite{Lucassen:1988:PES:73560.73564},
\cite{Leino:2002:UDG:543552.512559},
\cite{boyland:01interdependence},
\cite{DPJ}. At a broad conceptual level, static effects systems call
for a user-specified partitioning of heap into {\em regions} in a
fine-grained enough way to show that operations that may occur
simultaneously work on different regions. For instance \cite{DPJ}
introduces separate syntax for regions, introduces the ability to
specify an arbitrary tree of regions, and new syntax for specifying
which region mutable locations belong to. Methods must be associated
with their read and write effects which capture the set of regions
that can be operated upon during the execution of this
method. Now determinacy can be established if it can be determined
that that in \code{async \{S\} S1} it is the case that \code{S} does
not write any region that \code{S1} reads or writes from, and vice
versa ({\em disjoint parallelism}). 

\cite{DPJ} develops these ideas in the context of a language with
\code{cobegin/coend} and \code{forall} parallelism, and does not
address arbitrary nested or clocked parallelism.  In particular it is
not clear how to adapt these ideas to support some form of pipelined
communication between multiple parallel activities. Once can think of
these computations as requiring ``disjointness across time'' rather
than disjointness across space. A producer is going to write into
locations \code{w(t),w(t+1),w(t+2), \ldots}, but this does not
conflict with a consumer reading from the same locations as long as
the consumer can arrange to read the values in time-staggered order,
i.e. read \code{w(t)} once the produce is writing \code{w(t+1)}.  

We believe it is possible to significantly simplify this approach
(e.g.{} using the dependent-type system of X10) and extend it to cover
all of X10's concurrency constructs (see
Section~\ref{future-work}). Nevertheless the cost of developing these
region assertions is not trivial. Their viability for developing large
commercial-strength software systems is yet to be establishd.

In this paper we chose a different approach, a {\em lightweight}
approach to safety. We introduce two simple ideas -- {\em
  accumulators} and {\em clocked types} -- which require very modest
compiler support and can be implemented efficiently at run-time.

Accumulators arise very naturally in concurrent programming: an
accumulator is a mutable location associated with a commutative and
associative {\em reduction} operator that can be operated on
simultaneously by multiple activities. Multiple values offered by
multiple activities are combined by the reduction operator. We show
that the concrete rules for accumulators can be defined in such a way
that they do not compromise safety: a serial execution precisely
captures results generable from any execution.

Similarly clocked computations (barrier-based synchronization) is
quite common in parallel programming e.g. in the SPMD model, BSP model
etc. We observe that many clocked programs can be written in such a
way that shared variables take on a single value in one phase of the
clock. Further, clocked computations are often iterative and operate
on large aggregate data-structures (e.g. arrays, hash-maps) in a
data-parallel fashion, reading one version of the data-structure (the
``red'' version) while simultaneously writing another version (the
 ``black'' version). To support this widely used idiom, we introduce
the notion of {\em clocked types}. An instance \code{a} of a  clocked type
\code{Clocked[T]} keeps two instances of type \code{T}, the \code{now} and
the \code{next} instance. \code{a} can be operated upon only by
activities registered on the current clock.  All read operations
during the current clock phase are directed to the \code{now} version,
and write operation to the \code{next} version. This ensures that
there are no read-write conflicts. There may be write-write conflicts
-- these must be managed by either using accumulators or an effects
system. Once computation in the current phase has quiesced -- and
before activities start in the next phase -- the \code{now} and
\code{next} versions are switched; \code{now} becomes \code{next} and
\code{next} becomes \code{now}.\footnote{Clearly this idea can be
  extended to $k$-buffered clocked types where each clock tick rotates
  the buffer. This idea is related to the K-bounded Kahn networks of \cite{k-bounded-kahn}.}

We show that clocked types can be defined in a safe way, provided that
accumulators are used to resolve write-write conflicts. The only
dynamic check needed is that a value of this type is being operated
upon only by the activity that created the value or its
descendants. 

In the following by Safe X10 we shall mean the language X10 restricted
to use (\code{clocked}) \code{finish}/\code{async}, \code{at} with
(clocked) accumulators.  All programs in Safe X10 are safe -- they can
be run with a serial schedule and their I/O behavior is identical
under any schedule.  We show that Safe X10 is surprisingly
powerful. Many concurrent idioms can be expressed in this language --
histograms, all-reduce, SpecJBB-style communication Indeed, even some
form of pipelined/systolic communication is expressible.

Since the dynamic conditions introduced on accumulators and clocked
types are not straightforward, we formalize the concurrent and serial
semantics of an abstraction of Safe X10 using Plotkin's structural
operational style. We are able to do this in such a way that the two
proof systems share most of the proof rules, simplifying the proof. We
establish that the language is safe -- for any program and any input,
any execution sequence for the concurrent proof rules can be
transformed into an execution sequence for the sequential proof rules
with the same result. 

In summary the contributions of this paper are as follows.

\begin{itemize}
\item We identify the notion of a {\em safe} program -- one which can
  be executed with a serial schedule and for which 
  every schedule produces the same result. Such a program is
  simultaneously a sequential program and a parallel program with
  identical I/O behavior. 
\item We introduce accumulators and clocked types in the X10
  programming model. These are introduced in such a way that arbitrary
  programs using (\code{clocked}) \code{finish}, \code{async} and
  \code{at} and in which the only variables shared between
  concurrently executing activities are accumulators or clocked
  accumulators are guaranteed to be safe.  
\item We show that many common programming idioms can be expressed in
  this language.
\item We formalize a fairly rich subset of X10 -- including (clocked)
  finish, async, accumulators and clocked accumulators.  This is the
  first formalization of the nested clock design of X10 2.2, and is
  substantially simpler than \cite{vj-clock}. We establish that this
  language is safe. 
\end{itemize}
In companion work we show how these ideas can be extended to support
modularly defined effects analyses, using X10's dependent type system.

The rest of this paper is as follows. In \Ref{related-work} we discuss
related work. In \Ref{constructs} we present the constructs in detail,
followed by examples of their use. \Ref{semantics} presents the
semantics of these constructs. We discuss implementation in
\Ref{implementation} and finally conclude with future work.

%%\cite{Gifford:1986:IFI:319838.319848}
%%
%%
%%Figure~\ref{fig:1} shows  the famous ``parallel Or'' program of Plotkin
%% (in X10 syntax, \cite{x10}). This program can be executed with a
%%depth-first schedule, is partially determinate and deadlock-free, but {\em not}
%%safe. The result of running the sequential schedule is not the same as
%%the result that can be obtained with other schedules. Specifically
%%\code{parallelOr(()=> CONT, ()=>TRUE)} will diverge (exhibit an
%%infinite exection sequence) under the depth-first schedule, but will
%%return \code{true} under any fair schedule that permits the second
%%async to progress.
%%
%%\begin{figure}
%%  \begin{lstlisting}
%%static val CONT=1, TRUE=2, FALSE=3;
%%def run(done:Cell[Boolean], a:()=>Int) {
%% var aa:Int=a();
%% var cont:Boolean=true;
%% for (; aa==CONT && cont;aa=a()) {
%%  atomic cont = !done();
%% }
%% if (aa==TRUE)
%%  atomic done()=true;
%%}
%%def parallelOr(a:()=>Int, b:()=>Int):Boolean {
%% val done=new Cell[Boolean](false);
%% finish {
%%  async run(done, a);
%%  async run(done, b);
%% }
%% return done();
%%}
%%  \end{lstlisting}
%%  \caption{A program that is not sequential}\label{fig:1}
%%\end{figure}



%%{\em
%%\begin{enumerate}
%%\item Use activity registration as a mechanism to tame object graphs.
%%\item Focus on structured concurrency. Using scoping and block-structure
%%    to delimit regions of code that may execute in parallel and affect
%%    the data structure.
%%
%%\item Accumulation can be defined safely by delaying. However, the delay
%%    operation is guaranteed to be deadlock-free.
%%
%%\item Clocked types support phased computation, another common idiom
%%    particularly for stencil computations.
%%\end{enumerate}
%%}
%%
%%Key contributions:
%%{\em
%%\begin{enumerate}
%%\item Identification of determinate, deadlock-free data-structures.
%%\item Discussion of design alternatives which points out the
%%  difficulty of integrating these ideas in a modern OO language.
%%\item Discussion of various idioms expressible using these data-structures.
%%\item Proof of determinacy and deadlock-freedom in an abstract version
%%  of the language.
%%\end{enumerate}
%%These constructs are implemented in \Xten, available as open source from
%%SVN head and will be in the next release of \Xten.
%%}


%Semantics and theorems for an abstract version of the language.




\chapter{A Whirlwind Tour of \Xten{}}
In this chapter, we'll look at a couple of quick examples that illustrate
\Xten{} in action. The next two chapters will then fill in a lot of the details.
\section{Hello!}

Enough suspense! You knew it was coming, and here it is!
%%START X10: src/intro/HelloWorld.x10 hello
\fromfile{HelloWorld.x10}
\begin{xtennum}[]
// Copyright 1977 Greeter's Anonymous. All rights reserved 
/** classic first code example */ 
public class HelloWorld { 
   /** //
    * writes "Hello, World" to the console 
    * @param args the command line arguments 
    */ //
   public static def main(args:Array[String](1)) { 
      x10.io.Console.OUT.println("Hello, World"); 
   }
}
\end{xtennum}
%%END X10: src/intro/HelloWorld.x10 hello
The working copy of this code is 
\href{http://dist.codehaus.org/x10/documentation/guide/src/hello/HelloWorld.x10}{\xlfilename{hello}{intro/HelloWorld.x10}}.
Let's step through it, a line at a time:

\begin{description}
\item [line \xlref{hello-cmnt}{1}:] 
\xline{hello-cmnt}{// Copyright 1977 Greeter's Anonymous. All rights reserved}\\
Comments in \Xten{} are the same as Java or C++: they either begin
with ``{\tt //}'' and go through the end of the line, or begin with ``{\tt /*}''
and end at the \emph{first} ``{\tt */}'' that follows.  Because the first ``{\tt */}'' 
ends the comment, ``{\tt /*...*/}'' {\em comments do not nest}. 

\item[line \xlref{hello-x10doc1}{2}:] {\tt /** classic first code example
*/\\}
Comments in the style {\tt /** \ldots */} that immediately preceed a
declaration are called ``X10Doc'' comments. X10Doc is a set of conventions for
publishing the APIs for whole directory trees of \Xten{} source files.  X10Doc
follows that same format as Java's ``JavaDoc'', but instead of using the
``javadoc'' command, you use the ``x10doc'' command to process it.  The final
product is a nicely organized HTML site. 
All of our sample code uses X10Doc documentation, but we usually
do not copy it into the displays in the text.
The documentation for the \Xten{} library 
\href{http://dist.codehaus.org/x10/xdoc/}{http://dist.codehaus.org/x10/xdoc/}
is all generated from X10Doc comments.
\footnote{If you want
  to really get to know JavaDoc, its home page is at  
  \href{http://www.oracle.com/technetwork/java/javase/documentation/index-jsp-135444.html}{oracle.com},
  but for a quick executive summary sufficient for almost all purposes, see
  \href{http://en.wikipedia.org/wiki/Javadoc} {the Wikipedia entry}.
}

\item [line \xlref{hello-code0}{3}:] {\tt class HelloWorld \{\ldots \}}\\
\Xten's classes serve essentially the same purpose as classes in other object
oriented languages, Java and C++ in particular.  There are some differences, of
course, which we'll point out as they arise in the discussion.

A class normally will have the same name as the file in which is
declared---\eg{} {\tt HelloWorld} is found in {\tt HelloWorld.x10}.  C++
programmers should realize that, unlike C++, \Xten{} relies on file names to
find class declarations.  We'll go through the rules in detail in Chapter
\ref{chp:types}.

\item [line \xlref{hello-x10doc2}{6}]{\tt * @param args the command line
arguments}\\
Inside X10Doc comments, lines that start with an asterisk (``*'') are 
copied into the HTML with the asterisk (and leading spaces) deleted. 
This line says that {\tt args} is an argument for the method whose
declaration follows, namely {\tt main()}.

\item [line \xlref{hello-code1}{8}:] {\tt public static def main(args:
Array[String](1))}\\ Program execution starts, as in Java and C++, with a
method named {\tt main}, which takes the command-line arguments as a collection
of strings. But here we start to see some differences in syntax from Java and
C++:
\begin{itemize}
\item The keyword {\tt def} begins a method declaration.   
This makes it easy to tell what is a method and what isn't. 
(In contrast, in Java and C++, you have to look for clues -- sometimes small
ones -- to tell
the difference between a method and a field.)

\item There is no return type specified here.
In fact, {\em you don't normally need to specify the return
type for a method.} This differs from both Java (you must supply it) and C++
(the default is {\tt int}).  The \Xten{} compiler looks for the return
statements in a method and normally can infer the return type. 

If you {\em do} wish to specify it, then, unlike Java and C++, it {\em follows} the
argument list.  For example, ``{\tt def doIt(t: T): U \{\ldots\}}'' declares a
method named {\tt doIt} with one argument of type {\tt T} and a return value of
type {\tt U}.  Notice the `{\tt :}' that preceeds the types in both places:
it is the required syntax.  The return type of {\tt main()} is {\tt void},
meaning that no value is returned.

So why might you specify the return type, if you don't have to?  Basically,
writing down the return type gives valuable documentation, particularly for
longer methods where the return type is not immediately obvious.   It also
prevents some mistakes: if you expect to return a \xcd`Boolean`, but one of
the seventeen \xcd`return` statements accidentally has an \xcd`Int` instead,
X10's type inference will happily pick \xcd`Any`, which can be anything, as
the return type of the method.  If you specify the \xcd`Boolean` return type,
X10 will flag the \xcd`Int` return as an error, which is probably what you want.

Occasionally, the compiler will force you to specify a return
type to resolve type checking difficulties.  For example, X10 might infer that
method \xcd`m` is \xcd`Boolean` --- but in a subclass, you want to have
\xcd`m` return other kinds of things.  So you would need to declare \xcd`m` to
be \xcd`Any`, rather than the inferred \xcd`Boolean`, in the parent class.  

\item \Xten{} has generic types, along the same general lines as Java and C++. 
\Xten{} uses square brackets to hold the actual type, {\em e.g.} {\tt Array[String]}
for declaring an array of {\tt String}s.

\item \Xten{} arrays, like FORTRAN arrays, may be multi-dimensional. The
``{\tt (1)}'' that follows {\tt Array[String]} asserts that the array is
one-di\-men\-sional, or in other words, is just like the usual Java or C++
array.

\item The general syntax for assigning a type to an identifier is 
\begin{center}{\tt {\em iden\-ti\-fi\-er}: {\em type}},\end{center}
as in {\tt args:Array[String](1)}.  White space before or after the `{\tt :}' is 
ignored by the compiler. In fact, white space in \Xten{} is treated as it
is in Java and C++: normally ignored, except to the extent that it is needed to
separate tokens or appears in string literal constants.
\end{itemize}

\item [line \xlref{hello-code2}{9}:]{\tt x10.io.Console.OUT.println("Hello, World");}\\
Like Java, \Xten{} groups classes into units called ``packages''.  For example,
the input-output classes in \Xten's standard library all belong to the package
{\tt x10.io}. The class {\tt Console} is part of that package.  
Package names are used both as prefixes to provide unique names for classes {\em
and to locate the classes}. 
\begin{quote}{\tt Console.IN},  {\tt Console.OUT}, and  {\tt Console.ERR}\end{quote}
are the standard input, output, and error streams. 
The method  {\tt println} prints a string, followed by an operating-system
dependent line-ender: either a single newline character for Unix based systems, 
or a carriage return-newline pair. If you don't want the newline, use {\tt print} instead.

We'll give some more details about packaging in chapter \ref{chp:types}.
\end{description}

Here's the command for compiling {\tt HelloWorld} for execution by a Java-based
runtime (``{\tt \%}'' is the command line prompt):
\begin{verbatim}
% x10c HelloWorld.x10
\end{verbatim}
To run it, you use the {\tt x10} command:
\begin{verbatim}
% x10 HelloWorld
Hello, World
\end{verbatim}
If {\tt HelloWorld} had required some command-line arguments, you could have
added them at the end of the command line.

There is also a C++
runtime.  To use it, you need to compile using {\tt x10c++} rather than {\tt
x10c}.  The usual C compiler convention {\tt -o {\em filename}} for naming the
executable is used.
\begin{verbatim}
% x10c++ HelloWorld.x10 -o hello
% runx10 hello
Hello, World
\end{verbatim} 
You'll need the {\tt runx10}: you cannot invoke {\tt hello} directly, because
some special setup is required to bootstrap the \Xten'c C++ runtime.
We don't actually need anything special
for {\tt hello}, but \Xten{} code that sets up the call to {\tt main()} {\em is}
generic and needs information supplied by {\tt runx10}.

\section{Two CPUs Are Better Than One}

The point of \Xten{} is concurrent programming: giving you control over clusters of 
multiprocessors.   We'll get started on this by parallelizing a simple piece of serial
code that computes an approximation to the number $\pi$.  Along the way, we'll
introduce some more \Xten{} syntax and write our first loops.

\subsection{$\pi$ via Monte Carlo}

The unit circle is the set of points $(x,y)$ in the plane that satisfy $x^{2} + y^{2} \le 1$,
and its area is $\pi$.  We are going to explore a particularly simple method of estimating
$\pi$.  Figure \ref{fig:cis} shows the one-quarter of the unit circle that lies
in the unit square,  $0 \le  x,y \le 1$.  The unit square has area 1, and the
shaded part inside the circle has area $\pi/4$.  
\begin{figure}[!htbp] 
\begin{center} 
\includegraphics[width=2.5in]{"images/cis3"} 
\caption{The intersection of the unit circle with the unit square}
\label{fig:cis}
\end{center}
\end{figure}
Now imagine picking points at random in the unit square.  What fraction
will also lie in the unit circle?  If the points are really random, the
answer ought to be the fraction of the square that lies inside the unit circle,
namely: $\pi/4$.

One way to estimate $\pi/4$, then, is to pick a
large number of points $(x,y)$ in the unit square at random and see what fraction
actually land in the unit circle. This sort of process is called a ``Monte
Carlo'' algorithm.

If ever there 
were an easily parallelized type of algorithm, Monte Carlo is it: if we have 1,000
processors, we let each generate points independently, and at the end, we
just have to merge the results.  The only trick is to make sure that each of the 1,000
processors starts in a way genuinely random with respect to the others, so that
they don't just duplicate each other's efforts. 

\subsection{Getting Started: A Serial Version}
Let's look first at a serial version in Figure \ref{fig:mcpi}, 
because it introduces a number of \Xten{} idioms that we'll need in the
parallel version.  You can find the source in 
\href{http://dist.codehaus.org/x10/documentation/guide/src/montePi/MontePi1.x10}{MontePi1.x10}.
\begin{figure}[!htbp]
\hrulefill
%%START X10: src/intro/MontePi1.x10 mpi1
\fromfile{MontePi1.x10}
\begin{xtennum}[]
import x10.util.Random; 
public class MontePi1 {
    static val N = 10000; 
    public static def main(s: Array[String](1)) { 
        val r = new Random();   
        var inCircle:Double = 0.0; 
        for (j in 1..N) { 
            val x = r.nextDouble(); 
            val y=  r.nextDouble(); 
            if (x*x +y*y <= 1.0) inCircle++; 
        }  
        val pi = 4*(inCircle/N); 
        Console.OUT.println("Our estimate for pi is " + pi); 
    }
}
\end{xtennum}
%%END X10: src/intro/MontePi1.x10 mpi1
\hrulefill
\caption{Serial Monte Carlo Approximation of Pi}\label{fig:mcpi}
\end{figure}
\begin{description}
\item[line \xlref{mpi1-imp}{1}:] 
{\tt Random} is a class from the \Xten{}
standard library. Whenever you need a class that is not implemented in the file
you are editing, the compiler needs to be told how to find it.  There are two ways
to do so, which are the same as in Java:
\begin{enumerate}
\item 
You can specify the {\em fully qualified name} of the class in an {\tt import}
statement, as line \xlref{mpi1-imp}{1} does for {\tt Random}. In the next
chapter, we'll go into more detail about the naming conventions. 

\item Or, you can omit the {\tt import} statement, 
but if you do, you have to write out
the fully qualified class name at each use.  Iif we had left it out
here, we would have had to rewrite line \xlref{mpi1-r}{5}  as
\begin{verbatim}
val r = new x10.util.Random(); 
\end{verbatim}
The standard \Xten{} library provides some types which are so commonly
used that the compiler is kind and does not force you to import them
explicitly or write them out in detail at each use.  {\tt Int}, for example,
is really {\tt x10.lang.Int}, and {\tt Console} in line \xlref{mpi1-out}{13}
is really {\tt x10.io.Console}.
\end{enumerate}

\item[lines  \xlref{mpi1-N}{3},  \xlref{mpi1-r}{5}, 
 \xlref{mpi1-x}{8},  \xlref{mpi1-y}{9},  \xlref{mpi1-pi}{12},]
In line  \xlref{ \xlref{mpi1-N}}{3}, we declare {\tt N} to be a {\tt static val}.
The keyword ``{\tt val}'' means that {\tt N} names a value,
{\tt 10000} in this case.  One cannot assign a new value to 
{\tt N} later on in the code: it is a constant, in the same way the {\tt const}
is used in C and {\tt final} in Java.

The keyword ``{\tt static}'' means that the value is associated with the class
{\tt MontePi} itself. The remaining {\tt val}s, {\tt r}, {\tt x}, {\tt y} and
{\tt pi} are not part of the class: they are just local variables of the method
{\tt main()}.

The compiler will happily figure out the type of a {\tt val} whose
value appears in its declaration, as it does in each of the five declarations
here. You can, if you wish, provide the type yourself.
For example, we could have written line  \xlref{mpi1-r}{5} as
\begin{verbatim}
val r:Random = new Random(); 
\end{verbatim}
There's not much point in this case, though, to spelling things out.
Adding ``{\tt :Random}'' helps neither people nor the compiler read the code. 

\bard{I added a discussion of $<:$.  This probably needs to get moved; it's
too long for here.}
\begin{nonquote}
{\em Most of the time, you should not
specify the type of a {\tt val} whose initial value appears with its
declaration.}  
You can trust the compiler to come up with a correct type for your value, 
or to give you an error message.  

You might want to give a type for the sake of someone reading the program --
someone like yourself six months later, or even the compiler.  You can give an
exact type, and insist that all that the compiler can know about the variable
is the type you give it, with the \xcd`:Random` syntax.  

It's usually better to give {\em approximate} type information, where you tell
{\em some} things about the variable -- the things you expect people reading
your code to care about, and the things you want the compiler to check.  The
compiler will figure out the exact type, but it will check everything that you
said you wanted.  The syntax for this is to use ``\xcd`<:`'' instead of just
``\xcd`:`'', like this:
\begin{verbatim}
val r <: Random = new Random();
\end{verbatim}
For \xcd`Random`, there aren't a lot of choices for what you could say for
partial information.  If you were giving another name to \xcd`s`, the
one-dimensional array of \xcd`String`s declared on line \xlref{mpi1-s}{4}, you
have more choices:
%%START X10: ArgvPartialInfo.x10 argvpartial
\fromfile{ArgvPartialInfo.x10}
\begin{xtennum}[]
public class ArgvPartialInfo {
    public static def main(s: Array[String](1)) {
        val a <: Object = s; 
        val b <: Array[String] = s;
        val c <: Array[String]{rank != 3} = s;
    }
}
\end{xtennum}
%%END X10: ArgvPartialInfo.x10 argvpartial

On line \xlref{argvpartial-a}{3}, we give very little information, just saying
that \xcd`a` is an \xcd`Object`.  On \xlref{argvpartial-b}{4}, we say that
\xcd`b` is an array of strings, but, unlike the declaration of \xcd`s` on line
\xlref{argvpartial-s}{2}, we don't say that it's a {\em one-dimensional} array.  
On \xlref{argvpartial-c}{5}, just because we can, we say that it's 
an array but not a three-dimensional array: it could be one-dimensional or
four-dimensional or eighty-dimensional.  

The difference between ``\xcd`:`'' and ``\xcd`<:`'' in \xcd`val` declarations
is that  ``\xcd`:`'' erases information and ``\xcd`<:`'' doesn't.  
So, if you write 
\begin{verbatim}
val b : Array[String] = s;
\end{verbatim}
\noindent
all that X10 will know about \xcd`b` is that it is an array of strings, of
some unknown dimensionality.  But if, as on line \xlref{argvpartial-b}{4}, you
write 
\begin{verbatim}
val b <: Array[String] = s;
\end{verbatim}
\noindent
X10 will remember that \xcd`b`, like \xcd`s`, is one-dimensional. This is
important information later on if you want to subscript it with a single
index, \xcd`b(0)`.  That's the right thing to do with a one-dimensional array,
but wrong for an array whose dimensionality is unknown.
\bard{This new bit has gotten quite long, and should be put into a separate section}


You {\em do} need to provide the type when you don't want to initialize
the {\tt val} in the declaration itself.  Typically this happens when the {\tt val}
depends on some choices that you can't neatly write in one line:
\begin{verbatim}
     val howMany: Int;
     if (aBoolean) {/* howMany gets set one way here */}
     else {/* and gets set differently here */}    
\end{verbatim}
The {\tt if} and {\tt else} blocks are both free to do any calculation they
need to in arriving at {\tt howMany}'s value, so long as they don't try to
use {\tt howMany} itself before it has been set.  
The rule is
\begin{quote}
{\em Control cannot reach a use of a {\tt val} without first
reaching an assignment that sets the {\tt val}'s value.}
\end{quote}
In other words, there is no such thing as ``default value'' for a {\tt val}.
It must be set explicitly by you, and once set, cannot be changed.
\end{nonquote}

The initializer for a {\tt val} may be any legal \Xten{} expression that can be
evaluated at run-time---they need not be compile-time constants.  Lines  \xlref{mpi1-x}{8},
 \xlref{mpi1-y}{9}, and  \xlref{mpi1-pi}{12} all show examples of initialization expressions.

\item[lines  \xlref{mpi1-incircle}{6}:]  
The keyword {\tt var} introduces the declaration of a variable.
A declaration like ``\xline{mpi1-incircle}{var inCircle:Double = 0.0;}''  says that
{\tt inCircle} names some storage that holds a value of type {\tt Double}
whose initial value is {\tt 0.0}, and this value
may be updated as the code runs.  In the lingo of the trade, one
says that ``{\tt inCircle} {\em references} a {\tt Double}''.  

{\tt Double} values are double-precision IEEE floating points, exactly like Java's and C++'s
{\tt double}.  

You do need to supply
the type for a {\tt var} even when an initial value is provided.
The rationale is a bit involved, so we ask you just to take our word for it that  
for now, the compiler needs to be told the type of every {\tt var}. 

\item[lines  \xlref{mpi1-for}{7}- \xlref{mpi1-endfor}{11}]
Another new ingredient in the code is the ``{\tt for}''  loop, lines
\xlref{mpi1-for}{7} through \xlref{mpi1-endfor}{11}.  
\xline{mpi1-for}{for (j in 1..N) \{} executes its body (lines
\xlref{mpi1-r}{5}-\xlref{mpi1-if}{10}) once with \xcd`j` bound to \xcd`1`, then
once with \xcd`j` bound to \xcd`2`, and so on, ending (if nothing stops it
sooner) with \xcd`j` bound to \xcd`N`.  

\xcd`1..N` is an example of an \xcd`IntRange`, a value that describes a range
of integers.  They're useful for looping over, as we do here.  They are also
useful for declaring the subscripts of arrays, and a variety of other
situations. 


X10 has old-style \xcd`for` loops too, like \xcd`C++` and \xcd`Java`'s
traditional \xcd`for` loops.  We could have written this one:
%%START X10: OldStyleFor.x10 oldstylefor
\fromfile{OldStyleFor.x10}
\begin{xtennum}[]
for(var j:Int = 1; j <= N; j++) {
   //...
}
\end{xtennum}
%%END X10: OldStyleFor.x10 oldstylefor
(\xcd`j` needs to be declared as a \xcd`var`, not a \xcd`val`, because we're
changing it.  \xcd`j++` means ``Add one to \xcd`j`,'' just as it does in C and
Java.)

%%OLDFOR%% The syntax in line  \xlref{mpi1-for}{7} should be familiar.
%%OLDFOR%% It begins with the declaration and initialization of the loop
%%OLDFOR%% variable {\tt j}, ``{\tt var j:Int = 1}''.  Important: you need the ``{\tt var}'', because that
%%OLDFOR%% tells the compiler you want to declare a new variable. ``{\tt j:Int = 1}'' by itself
%%OLDFOR%% won't do: the compiler will complain that {\tt j} is a value, not a variable.
%%OLDFOR%% 
%%OLDFOR%% As you might guess, ``{\tt Int}'' is \Xten's 32-bit integer type. 
%%OLDFOR%% 
%%OLDFOR%% The declaration for {\tt j} is followed by the test ``{\tt j <= N}'' .  This expression is
%%OLDFOR%% evaluated at each iteration (including the first!) and the loop body will be executed
%%OLDFOR%% only so long as it evaluates to ``true.''  
%%OLDFOR%% 
%%OLDFOR%% The third expression in line \xlref{mpi1-for}{7}, ``{\tt j++}'', is the re-initialization of the
%%OLDFOR%% loop variable. \Xten's binary and unary operators are the same as in all of the
%%OLDFOR%% languages in the tradition of C, so things like ``{\tt +=}'' and ``{\tt ++}''
%%OLDFOR%% mean exactly what most of you would expect, but just in case:
%%OLDFOR%% \begin{quote}
%%OLDFOR%% The expression ``{\tt a += b}'' is a short-hand for ``{\tt a = a+b}''.  This
%%OLDFOR%% shorthand works for all binary arithmetic operators and the three bitwise operators
%%OLDFOR%% ({\tt |}, {\tt \&} and \verb+^+).  Operators like {\tt +=} are often referred to as ``update
%%OLDFOR%% assignments''
%%OLDFOR%% 
%%OLDFOR%% ``{\tt n++}'' is another form of update assignment.  As an expression, the
%%OLDFOR%% value of {\tt n++} is the {\em current} value of {\tt n}. As a side-effect of
%%OLDFOR%% evaluating {\tt n++}, though, {\tt n}'s value is incremented by {\tt 1}.  The
%%OLDFOR%% expression {\tt n--} is similiar, but (of course) subtracts {\tt 1}.
%%OLDFOR%% 
%%OLDFOR%% ``{\tt ++n}'' is just a short-hand for ``{\tt n = n+1}''.  That is: {\tt n} is
%%OLDFOR%% incremented by {\tt 1} and the result is used as the value of the expression.
%%OLDFOR%% The expression {\tt --n} is similar.
%%OLDFOR%% \end{quote}  
%%OLDFOR%% 

Each time through the loop, we call the random number generator
{\tt r} twice to get the coordinates of a point
(lines  \xlref{mpi1-x}{8}, and  \xlref{mpi1-y}{9}).
Happily, {\tt r.next\-Double}
returns a value between 0 and 1, so we can use it ``as is.''   In line \xlref{mpi1-if}{10},
we check whether the point lies in the unit circle, and if so, we increment
{\tt inCircle} by 1. 

As it happens here, we don't need {\tt j} in the loop, but we do want to emphasize
that {\em the scope of {\tt j}'s declaration is the loop and nothing but the
loop}, so when you leave the loop, {\tt j} will be unavailable.  
%%FOR%% There is no rule, however,
%%FOR%% that says you {\em have} to declare the loop variable in the {\tt for} statement.
%%FOR%% If you need the value once the loop completes, just declare it in a context surrounding
%%FOR%% the loop, and set it to its initialize value wherever it is convenient to do so, \eg:
%%FOR%% \begin{verbatim}
%%FOR%%    var j: Int = 1;
%%FOR%%    for( ; j <= N; j++) { ... }
%%FOR%%    Console.OUT.println("Is "+j+" == "+(N+1)+"?");
%%FOR%% \end{verbatim}

\item[line  \xlref{mpi1-pi}{12}:]
On exit from the loop at line  \xlref{mpi1-pi}{12}, {\tt inCircle/N} is the fraction of points
in the circle, which is going to be a positive number less than 1, so we
have to use a {\tt Double} (or if we don't care about the precision, a {\tt Float})
to capture the value.  
That's why we made {\tt inCircle} a ``{\tt Double}'' in line  \xlref{mpi1-incircle}{6}.
When we do the division here, the compiler will arrange to convert
{\tt N} to a {\tt Double}, too, and will use double precision floating
point division.  

Suppose we had said to ourselves, ``incrementing a {\tt Double} by 1 inside that
loop has got to
be more expensive than incrementing an {\tt Int}.  So let's declare {\tt inCircle} to be
an {\tt Int}.''  That's fine, but when we get to line \xlref{mpi1-pi}{12}, we have to be careful to
convert it to a {\tt Double}.  One way to do this is with \xcd`as`:
\begin{verbatim}
     val pi = 4 * (inCircle as Double / N);
\end{verbatim}
\end{description}

Time to try compiling and running the code.  Here is our console log for the
run:
\begin{verbatim}
% x10 MonteCarloPi
The value of pi is 3.1368
\end{verbatim}
Not a brilliant guess at $\pi$, but we didn't really try all that many points.  Your
answer might vary: in fact, the answer will vary with each run because, whenever
{\tt Random} creates a new generator, it
uses the current time to create a new starting point for computing its values.
The first two digits, 3.1, though, should be stable.  Good luck!

\subsection{We Can Do Better}\label{subs:wcdb}

There are some pretty primitive aspects to our first cut at $\pi$.  In this section
we'll introduce a few features of \Xten{} that will help us spruce up the code a
little bit.

To begin with, that ``{\tt static val N = 10000;}'' in line  \xlref{mpi1-N}{3} is really, truly rigid. 
We have access to a perfectly good set of command line arguments.  Why not use
the first, if supplied, to set the number of points to try?  That would have
let us try 1,000,000 points right away to see how much better we could do than
10,000. The code we need is simple enough:

%%START X10: MontePi2.x10 mpi2-N
\fromfile{MontePi2.x10}
\begin{xtennum}[]
public static def main(args: Array[String](1)) {
    val N = args.size > 0 ? Int.parse(args(0)) : 10000;
\end{xtennum}
%%END X10: MontePi2.x10 mpi2-N

Some comments:
\begin{description}
\item[line 1:]
The command line arguments come in as the array {\tt args}.  The declaration
\begin{quote} {\tt args:Array[String](1)}  \end{quote}
should, as we have already mentioned, be read: ``{\tt args} {\em is a value whose type
is an array of strings indexed by a single integer.}''   \Xten{} arrays, unlike Java
or C++ arrays, may be indexed by arbitrarily many integers, so you have to 
tell the compiler what sort of indexing you want.  Don't be confused here: the
``{\tt (1)}'' is not the size of the array: it is the {\em kind} of index you
need (a single integer, in this case).
\item[line 2:]
 \Xten{} uses the property {\tt size} to get the number of elements in
an array.  Using ``{\tt length}'', as Java does,
would be misleading, because \Xten{} arrays can be $n$-dimensional, not just
1-dimensional.

\Xten{} uses ordinary parentheses, and not square brackets, to access array elements.  
Thus, since {\tt args} is indexed by a single integer, {\tt args(k)} is the
entry in {\tt args} indexed by the integer {\tt k}.  \Xten, like its relatives
C++ and Java, normally starts array indexing from 0.  However, the \Xten{} programmer can
specify other index domains. There is no reason to get fancy about {\tt args}, however,
so its first element is {\tt args(0)}. Arrays in \Xten{} are a whole subject unto
themselves that we will get to in Chapter \ref{chp:tsl}.

We use {\tt Int}'s static method {\tt parse} to convert the command line input
from a {\tt String} to an {\tt Int}.  Then the conditional operator, 
``{\tt ?:}'' allows us to choose a value:
it begins by testing its first operand, which must must evaluate to \xcd`true`
or 
\xcd`false`
, \ie{} a {\tt Boolean}. If it evaluates to \xcd`true`, the value of the
expression is the second operand; otherwise the value is the third.
Once again, \Xten{} is consistent with Java and C++.
\end{description} 
The next step in improving the code is a little more involved.
We used a random number generator
that the \Xten{} library provided for us.  Suppose, though,
that for some reason we wanted to try another one. 
It would be nice if the generator
were just another parameter to the computation, so we could play with a bunch
of them if we wanted to.  To get there, we are going pull the main loop
out of {\tt main} and put it in its own method, one parameter of which is the
random number generator:
%%START X10:  src/intro/MontePi2.x10 mpi2cp
\fromfile{MontePi2.x10}
\begin{xtennum}[]
public static def countPoints(n: Int, rand: ()=>Double) {
   var inCircle: Int = 0;
   for (j in 1..n) {
      val x = rand();
      val y = rand();
      if (x*x +y*y <= 1.0) inCircle++;
   }
   return inCircle;
}
\end{xtennum}
%%END X10:  src/intro/MontePi2.x10 mpi2cp
 


The new ingredient here is the declaration of {\tt rand} in the first line.
As usual, its type follows
a colon (``:''), but what's there is not just a name, as usual, but a sort
of ``expression'', 
{\tt ()=>Double}, which is read:
``{\em a function that takes no arguments and returns a {\tt Double}.}''
There is no strictly analogous construct yet in Java, although one is planned, 
and the closest thing in C++ is the ``function pointer'' {\tt double (*rand)()}.

Note, too, that we compute {\tt inCircle} as an {\tt Int} here.  It's a little
faster than using {\tt Double} and besides, we know that the result is an
integer, so it makes sense to declare it as such.  

How do we create the function to pass in as {\tt rand}?  Here's one approach: in
{\tt main()}, put
%%START X10: src/intro/MontePi2.x10 mpi2main
\fromfile{MontePi2.x10}
\begin{xtennum}[]
val r = new Random();              
val rand = () => r.nextDouble();   
val inCircle = countPoints(N, rand); 
val pi = (4.0 * inCircle)/N;       
\end{xtennum}
%%END X10: src/intro/MontePi2.x10 mpi2main
 
\begin{description} 
\item[line \xlref{mpi2main-r}{1}] says that {\tt rand} is a function with no arguments whose body
is the expression {\tt r.nextDouble()}, which is its return value.  This is, as you would guess,
just a simple example of a much more general facility, and we'll see a lot of 
examples later that will flesh out how to use it. 

One important thing to understand is that the declaration of {\tt rand} captures the
runtime value of {\tt r}.  If we put this code in the body of a loop, then each time through
the loop, {\tt rand} would use the new value of {\tt r} that is yielded by the constructor
{\tt new Random()}.

The right-hand side of the declaration is often called a
``closure'' in the literature (because of the way variables from the surrounding
context (like {\tt r} here) are captured and kept until needed).  You'll also
see languages like \Xten{} describing themselves as supporting 
``first-class functions'', which is short for ``functions as first-class
data'' and simply means they allow you to work with functions 
in exactly the same way you would with any other sort of data, like \xcd`Int`s
or \xcd`String`s: you
can assign one, pass it as an argument, save it as an array element, \etc{}

\item[line \xlref{mpi2main-incircle}{3}] replaces the whole loop in lines
\xlref{mpi1-for}{7} through \xlref{mpi1-endfor}{11} of
our original with the call to our new method {\tt countPoints}.

\item[line \xlref{mpi2main-pi}{4}] We use a factor of {\tt 4.0} here, rather than {\tt 4} as we did
in our original. The type of {\tt 4.0} is {\tt Double}, which forces the whole
expression to be treated as a {\tt Double}.  
(We could also have used \xcd`as Double`, as we did before.)
We'll say more about this sort of
automatic conversion in section \ref{sec:tng}.
\end{description}  

The cleaned-up version of this code is
\href{http://dist.codehaus.org/x10/documentation/guide/src/montePi/MontePi2.x10}{MontePi2.x10}.

There are one or two things we could do to pretty it up even more, but enough
for now. It is time to look at how to parallelize it.
\subsection{Enter The Second Processor}\label{sec:esp}
We are going to present several parallel versions of our code.  We'll begin with
a version that assumes shared memory: multiple threads running on a single
machine.  Most PCs these days have
dual-processor CPUs, so a factor of 2 speedup is available right out of the box,
{\em if} your code can effectively use both processors.

For our $\pi$ calculator, the changes are simple: we just have to be able to
say ``start $n$ threads going, each with its own independent random number
generator, and when each has done it's share of the work, sum the hits from all
$n$ and divide by the total number of points tried.''  

\Xten{} avoids the term ``thread'', because it (together with ``process'' ) has
a variety of meanings in different contexts.  Instead, \Xten{} uses the term 
{\em activity} to mean a sequential thread of control. We'll be more loose here
and use whichever term seems more natural (to us!) at the moment, but you
should be aware of \Xten's convention when reading other literature.

Figure \ref{fig:mcpm} shows the relevant part of our new, parallel {\tt main}.
We'll go through it line by line.
\begin{figure}[!bthp]
\hrulefill
%%START X10: src/intro/MontePiAsync.x10 mpia
\fromfile{MontePiAsync.x10}
\begin{xtennum}[]
val N = args.size > 0 ? Long.parse(args(0)) : 100000L;  
val threads : Int = args.size > 1 ? Int.parse(args(1)) :  4; 

val nPerThread = N/threads; 
val inCircle = new Array[Long](1..threads);   

finish for(k in 1..threads) { 
   val r = new Random(k*k + k + 1);       
   val rand = () => r.nextDouble();       
   val kval = k;                     
   async inCircle(kval) = countPoints(nPerThread, rand); 
}                                 

var totalInCircle: Long = 0;             
for(k in 1..threads) {      
   totalInCircle += inCircle(k);         
   Console.OUT.println("ic("+k+") = "+inCircle(k)); 
}                                 

val pi = (4.0*totalInCircle)/(nPerThread*threads); 
\end{xtennum}
%%END X10: src/intro/MontePiAsync.x10 mpia
\hrulefill
\caption{Shared memory parallel code for computing $\pi$}\label{fig:mcpm}
\end{figure}
\begin{description}
\item[lines \xlref{mpia-N}{1}-\xlref{mpia-nperthread}{4}:] 
If command-line arguments are supplied, we read the number of points to try, {\tt N},
from the first, and
the number of threads to use, {\tt threads}, from the second.  Otherwise we just use
100,000 points and 4 threads. The number of points each thread will try is {\tt nPerThread}.  Since we
are now using multiple threads, we've switched to {\tt Long} integers---64
bits---instead of ordinary {\tt Ints}.
\item[line \xlref{mpia-incircle}{5}:]
The right-hand side is the \Xten{} idiom for constructing an array that is indexed
by a single integer running from {\tt 1} to {\tt threads}.  
The
entries in the array are initialized to {\tt 0L}, the \xcd`Long` way to say
zero, which is the default value for a \xcd`Long` that doesn't have its value
specified some other way.
You might wonder what
declaring {\tt inCircle} here to be a ``{\tt val}'' implies: what is
it that cannot be changed because {\tt inCircle} is a {\tt val}?  The answer
is that {\tt inCircle}'s value is always going to be the array it is initialized
by the right-hand side of this line, but during the program's run, the individual
elements of that array may be assigned to as needed.

\item [line \xlref{mpia-N}{1}:] The number \xcd`100000L` is the \xcd`Long`
      version of \xcd`100000`.  You can also use a suffix of \xcd`S` for
      \xcd`Short`, and \xcd`Y` for \xcd`Byte`\footnote{\xcd`B` is a
      hexadecimal digit, so X10 couldn't also use it as a byte marker.}, and
      any of these with a \xcd`U` 
      for the unsigned version.



\item[line \xlref{mpia-for}{7}:] There is nothing unusual about the ``{\tt for}'' loop part of
this line. The interesting part is the ``{\tt finish}''.  The whole
point of this loop is to spawn some number of
independent activities, each computing how many hits
out of a possible  {\tt nPer\-Thread} land in the circle.  We can't do any
further processing until we are sure that all of these activities have run to
completion.  That is what ``{\tt finish}'' guarantees: control will not reach
the statement after that guarded by a {\tt finish} until all of the activities
spawned in the {\tt finish}'s statement have completed.  So when we get to line
11, we can be sure that every entry in {\tt inCircle} has been correctly set.

\xcd`finish` is followed by a statement, like \xcd`for` in this example.  It
can be any statement, though.  If you want to start two activities and wait
for them to finish, you can write: 

%%START X10: FinishTwo.x10 finishtwo [numbers=none]
\fromfile{FinishTwo.x10}
\begin{xtennum}[numbers=none]
finish {
  async do_this();
  async do_that();
}
\end{xtennum}
%%END X10: FinishTwo.x10 finishtwo


\item[line \xlref{mpia-r}{8}:]  The constructor {\tt Random()}, if called with no arguments,
uses the number of milliseconds from some fixed time as the ``seed'' to begin
generating its random numbers. Alas, less than a millisecond may elapse between
the creation of two or more of our threads, and when that happens, we get the
same sequences in several threads, not the independent sequences we need.
So we've spiced things up by using a simple
polynomial in the loop counter {\tt k} to generate a unique seed for each activity.
Why not just {\tt k}? Why {\tt k*k + k +1}?  Just to spread the starting points out
a little more in the hope that our threads really
get independent sequences.

\item[lines \xlref{mpia-kval}{10} and \xlref{mpia-async}{11}:]
This is \Xten's idiom for spawning an activity
at the current processor.
The {\tt async} statement is readied for execution in its own thread, and control
may then be returned to the originating activity whenever the \Xten{} runtime's
activity manager wishes: it could be immediately, or it could be after the new
activity has been allowed some time. The important point---particularly if we
are running on a multi-processor---is that we do not have
to wait for the new activity to complete before returning control to the
original activity.  The effect of the {\tt for} loop is, therefore, to get
{\tt threads} activities up and running {\em concurrently}.

\xcd`async`, like \xcd`finish`, can be applied to any statement.  If you want
to start a multi-statement activity, use a \xcd`{` block \xcd`}`: 

%%START X10: MultiLineAsyncExample.x10 multilineasync
\fromfile{MultiLineAsyncExample.x10}
\begin{xtennum}[]
async {
  do_this();
  do_that();
}    
\end{xtennum}
%%END X10: MultiLineAsyncExample.x10 multilineasync


Why do we introduce {\tt kval} in line \xlref{mpia-kval}{10}?
Using {\tt kval}, which is a constant,  in
the {\tt async}, rather than {\tt k}, which is a
variable, ensures that that activity is using the value of {\tt
k} in force when the {\tt async} is spawned.  Were we to use {\tt k} inside the
{\tt async}, and it happened that the execution of the {\tt async} got
sufficiently delayed, the {\tt async} could see a value of {\tt k} greater than
that at time of the {\tt async}'s creation, and then it would set the wrong
array entry in line \xlref{mpia-async}{11}.

Line  \xlref{mpia-async}{11}  ends by calling {\tt countPoints}, as in our serial code.
We've allowed its first argument, the number of points to try, to be a {\tt Long}.
64-bit integers are probably overkill here, but with so many threads at our.
command, we can afford to think {\em big}.

\item[line \xlref{mpia-for2}{15}: ]
When we get to line , we can compute the total number
of hits in the circle out of the {\tt N} points we generated, because we can be sure that every entry in {\tt inCircle} has been correctly set.  

\item[line \xlref{mpia-pi}{20}: ]
Once the loop in lines \xlref{mpia-for2}{15}-\xlref{mpia-endfor2}{18}
has computed the total number of hits in the circle, 
we have to normalize the result to get the final answer.  
{\tt N} is the total number of points
we wanted to try, and each thread actually got {\tt N/threads == nPerThread}
 points to try.
If {\tt threads} doesn't divide {\tt N} evenly, though, we only wind up using
{\tt nPerThread*threads} points, which explains the denominator here.
Using a {\tt Double}, {\tt 4.0}, as the first term in the right hand side forces
the compiler to generate code for converting the two {\tt Long}s to {\tt Doubles} before
the arithmetic is performed.
\end{description}

Here's our console log for running the code from a Unix-style terminal window.
The Unix {\tt time} command
executes the program and then produces three time estimates: total CPU usage (``real''),
wall-clock elapsed time (``user''), and system overhead (``sys'').
This timing is for a 3GHz dual-processor laptop.   
\begin{verbatim}
% time x10 MontePiAsync 10000000 1
The value of pi is 3.1402504

real	0m2.426s
user	0m3.368s
sys	0m0.165s
% time x10 MontePiAsync 10000000 2
The value of pi is 3.1405268

real	0m1.370s
user	0m1.893s
sys	0m0.109s
\end{verbatim}
Not bad: an overall factor of  $3.368/1.893 = 1.78$ speed-up for the observed time-to-completion
out of a best possible of $2$.  

You can find the whole program in
\href{http://dist.codehaus.org/x10/documentation/guide/src/montePi/MontePiAsync.x10}{\tt MontePiAsync.x10}.


\section{A Thousand CPUs Are Better Than Two}
\subsection{Distributing Work}\label{sec:distwork}
To get heavy-duty concurrency, we have to distribute work across many
processors, which usually means we have to scale out to more than one
(shared-memory) machine.

To this end, \Xten{} provides a type, {\tt Place}, that is best thought of as
an address space in which activities may run.
The physical reality is that different {\tt Place}s may refer to the same
physical processor and may share physical memory, but from the programmer's
point of view:
\begin{itemize}
\item The running program has a single address space.
\item The distinct {\tt Places} partition that address space: no two {\tt
Places} have any storage in common.
\item But, since there is a single global address space, an activity at one
{\tt Place} may refer directly to storage at another.
\end{itemize}   

{\tt Place.MAX\_PLACES} is the number of {\tt Places} available to a program.  It is fixed
at program start-up and cannot be altered thereafter.

Each {\tt Place} has an integer id:
if {\tt p} is a {\tt Place}, then {\tt p.id} is its id.  An activity can find
out at which {\tt Place} it is executing by evaluating the expression {\tt
here}.  The keyword {\tt here} is reserved for this purpose alone.  The id of
the current activity's {\tt Place} is {\tt here.id}. The {\tt Place} whose id 
is ``{\tt i}'' can be got by evaluating {\tt Place.place(i)}.\footnote{
If you've programmed using the MPI
library,  {\tt Place.MAX\_PLACES} is analogous to
what you get by calling {\tt MPI\_Comm\_size} and the {\tt id} is analogous to
what {\tt MPI\_Comm\_rank} gives you.  The UPC equivalents are THREADS and MYTHREAD.
}

The \Xten{} runtime begins a program's execution by creating a single activity, the
``root'' activity, that calls
the program's {\tt main()}.  The root activity's home {\tt Place} is called
{\tt Place.FIRST\_PLACE}, and by convention, it is the {\tt Place} whose id is
{\tt 0}.

 So the question is: how does information at one {\tt Place} get to another? 
 One simple way is to use an ``{\tt at(p)}''  statement: if the identifier {\tt
 p} names a {\tt Place}, and if {\tt compute\-An\-Int()} is a method that
 computes an {\tt Int}, then
 \begin{verbatim}
      val anInt = at(p) computeAnInt();
 \end{verbatim} 
 means: 
 \begin{quote}
 Pause the thread for this activity.  Go to the {\tt Place p}, and resume the activity
 by calling {\tt compute\-An\-Int()} in a thread at {\tt p}. 
 Send the result back here to the original {\tt
 Place}, restart the thread there for the activity by assigning the value to {\tt anInt}.
 The activity then continues at the original {\tt Place}.
 \end{quote}
 If you like to think about implementation, you can think of the single
 \Xten{} activity being built out of two threads: the thread for the requesting
 activity and the thread for the remote activity. From the point of view
 of the \Xten{} programmer, though, exactly one activity is going on here,
 because the requesting thread is blocked
 while the remote thread does its thing, so no matter how many ``hardware
 threads'' we use to carry out this computation, it is strictly serial:
 it is one activity. This is one reason why \Xten{} uses the term
 ``activity'' for a serial computation, rather than ``thread.''
 
 If you do not want to wait around for the value to be computed and assigned,
 things are not so simple.  You might think, for instance,
 that something like the obvious ``{\tt async val anInt =  at(p) computeAnInt();}''
 might work, but it doesn't.   Just as with variables declared in {\tt for}  loops,
 the declaration of {\tt anInt} 
 in an {\tt async}'s body means that it is not available outside of it.  This
 is consistent with Java and C++ (and just about every other language): a
 declaration within a statement's body is visible only in that body.
 
The secret is to separate the assignment from the declaration:
%\begin{figure}[!hbtp]
 \begin{verbatim}
 1    val anInt: Int;
 2    finish { 
 3       /* some code not using anInt can go here */
 4       async { anInt = at(p) computeAnInt(); }
 5       /* maybe more code not using anInt here, too! */
 6    }
 7    /* at last: anInt can be used here! */
 \end{verbatim} 
% \caption{Passing a {\tt val} into an {\tt async}}\label{fig:vina}
%\end{figure} 
As the comments in this code suggest, the {\tt async} has to be inside a
 {\tt finish} block, and  {\tt anInt} cannot be used until control leaves the block.
 
Passing a ``{\tt var}'' into an {\tt async}'s block is also possible, but (of
course) risky because of the possibility of race conditions, an example of which
we'll discuss in the next section. If you change {\tt val} to {\tt var} in
line 1, the code works as before, but be aware that the {\tt finish} must be
present in the same scope as the {\tt var} that is used in the {\tt async}. 
That is, you cannot just spawn an activity into which you pass a variable 
unless that activity has a visible bound on its lifetime.  For example, the
code
\begin{verbatim} 
 1    def syncIt() {
 2       var anInt: Int;
 3       async anInt = 3;  // compiler won't accept this
 4   }
\end{verbatim}
will cause the compiler to complain:
\begin{quote}{\tt
Local variable "anInt" cannot be captured in an async if\\
there is no enclosing finish in the same scoping-level\\
as "anInt"; consider changing "anInt" from var to val.}
\end{quote}
What the compiler wants is a {\tt finish} surrounding the {\tt async} within the
scope of the declaration of {\tt anInt}.  To be precise, what will {\em not}
work is:
\begin{verbatim} 
 1    def syncIt() {
 2       finish {
 3         var anInt: Int; // anInt is local to the finish block
 4         async anInt = 3;
 5       }
 6       use(anInt);  // no! anInt is not defined here!
 7    }
\end{verbatim}
What {\em will} work is inserting the {\tt finish} so that {\tt anInt} is alive
and visible when the {\tt finish} completes:
\begin{verbatim} 
 1    def syncIt() {
 2       var anInt: Int;
 3       finish {
 4         async anInt = 3;
 5       }
 6       use(anInt);
 7    }
\end{verbatim}

Now that we know how to move data around and in and out of {\tt asyncs}, we are
ready to rework our code.
 
\subsection{A First Try At Multi-Place Code}\label{ssec:montepierror}
Let's get started with our multi-processor code with some high-level pseudo-code.
We are going to go all out and not only use several {\tt Places}, but at each
{\tt Place}, we'll use several activities.  Here we go:
\begin{description}\label{lbl:mpchl}
\item{\tt main}:
      Read the command line to get the number of places to use.
      For each {\tt Place}, call the function {\tt countAtP} to get
      that one {\tt Place}'s contribution, and add up all of them
      to get the final answer.
\item{\tt countAtP}:
      Add up the counts from several threads at one {\tt Place}. 
\item{\tt countPoints}:
      Called once per thread.  It is the same as its namesake in
      {\tt MontePi2} and {\tt MontePiAsync}.  This is where we actually call
      the random number generator to get the points to test.
\end{description}

In thinking about this code, keep in mind that a really high-performance computer
can provide literally thousands of {\tt Place}s, but,
for this sort of CPU-intensive activity, at any given {\tt Place}, it is likely
to be worthwhile running at most a dozen or so threads, probably fewer.  That
said, we might ask ourselves whether it makes sense to use different strategies
for accumulating our results in {\tt countAtP},  which we expect to have very
few contributors, versus {\tt main}, which may have thousands.

When we only have two or three integers to add together, it might make sense to
use a single {\tt var count:Long} to accumulate the total count, rather than using
an array (as we did in the {\tt main} for {\tt MontePiAsync}).
Here's a first cut:
\begin{verbatim}
 1 public static def countAtP(pId:Int, threads:Int, n:Long) {
 2    var count: Long = 0L;  // 0L == Long integer literal 0
 3    finish for (var j: Int = 1; j<= threads; j++)  {
 4       val jj = j;
 5       async {
 6          val r = new Random(jj*Place.MAX_PLACES + pId));
 7          val rand = () => r.nextDouble();
 8          count +=  countPoints(n,rand); // trouble, as we'll see
 9       }
10    }
11    return count;
12 }
\end{verbatim}
Sadly, this code has a very nasty bug, which is due to the variable
{\tt count} being shared by the whole set of threads.
The trouble is in line 8, where {\tt count}'s value is updated.  Broken
down, line 8 involves the following steps:
\begin{quote}
  {\bf Step 1: } Load the value of {\tt count} into the CPU.\\
  {\bf Step 2: } Call {\tt countPoints}.\\
  {\bf Step 3: } Add the return value from the call to the loaded value of {\tt
  count}.\\
  {\bf Step 4: } Copy the sum from the CPU into {\tt count}.
\end{quote}

If you are a
veteran of the parallel programming wars, you will recognize this as a classic
opportunity for a {\em race} condition. For the newcomers, here is a scenario
that shows what can go wrong:
\begin{quote}
We begin at line 2 with {\tt count}'s value is initially set to 0.
Suppose that the value of {\tt threads} coming in is {\tt 2}.
The {\tt async} in line 5 will then get called twice, so we'll have
two threads, call them {\tt T1} and {\tt T2}, executing the code
in lines 6-8 in parallel.  Figure \ref{fig:tml} is a graphic view
of one possible time-line for the two threads.

Suppose that {\tt T1} begins
executing first.  When it gets to line 8, and does the first step: it loads the
value of {\tt count}, which is still 0. But now comes trouble: suppose that
just after this step completes, the operating system's thread manager suspends
{\tt T1} for some reason.  

If a few nanoseconds later, the thread manager lets {\tt T2} start {\em rather
than restarting {\tt T1}, which, after all, is only fair, since {\tt T1}
already got at least some CPU time}. When {\tt T2} gets to line 8, it, too,
loads {\tt count} into the CPU. But, because {\tt T1} never completed updating
{\tt count}, {\tt T2} finds the same value, 0, stored there that {\tt T1} did.
So when {\tt T2} gets to step 3 in its execution of line 8, it adds
the return value to 0, and stores the result back into {\tt count},
so {\tt count} now is whatever {\tt T2} computed.

At some point {\tt T1} will be restarted, and because it will start executing
exactly where it was suspended, it will be at step 2, the call to {\tt
countPoints.}  {\tt T1} will {\em not} repeat the first step, loading {\tt
count}.  Threads also restart exactly where they where suspended.
So {\tt T1} add its contribution to what it thinks the value of {\tt
count} is---namely, 0.  The result is that {\tt T1} winds up overwriting {\tt
T2}'s value in {\tt count} with its own, not adding the two, which is what
we wanted.

Both activities having now completed, {\tt T2}'s contribution has been
lost.

\begin{figure}[!htbp]
\begin{center} 
\includegraphics[width=3.5in]{"images/timeline"} 
\caption{One possible timeline for {\tt T1} and {\tt T2}}
\label{fig:tml}
\end{center}
\end{figure}

Disaster!
\end{quote}
You can see why this is called a race. It is a particularly insidious sort of bug,
because sometimes you get the right answers, and sometimes you don't.
After all, the operating system did not {\em have} to suspend {\tt T1} at a
bad moment.  It just happened to.  The event that led the thread manager
to suspend {\tt T1} may have come from an external event having nothing to
do with {\tt MontePi} at all.  Maybe it was an I/O event of some kind
that simply had to take precedence over {\tt T1}---so, too bad: {\tt T1} loses.
That's life!

The cure is simple enough: we just need to replace line 8 with some code
that ensures that once one activity starts executing that code, {\em no other
activity can begin that code until the first one finishes.}
\begin{verbatim}
 8          atomic count += countPoints(n, rand));
\end{verbatim}
Guarding an \Xten{} statement this way, with the keyword {\tt atomic},
does just what we want: we are guaranteed that once an activity enters the
statement, no other activity may enter it until the original activity
completes it.

In our original scenario, this means that once {\tt T1}
starts executing line 8,  {\tt T2} will be blocked from entering
line 8 while {\tt T1} is still active there, even if for some reason the
operating system suspends {\tt T1} for a while.  {\tt T2} will be
suspended when {\em it} reaches line 8, at least until {\tt T1} finishes there.
This slows things down, but you get the right answer.

Even though our new line 8 is correct, it really is not a good solution.
The problem is that virtually all the time spent executing the statement is
in the expensive call to {\tt countPoints}.  {\tt countPoints}
does not depend on any resources shared by the two threads,
so there is no problem about the
two executing the call to {\tt countPoints} concurrently.
The only shared resource is {\tt count}, which
doesn't appear in {\tt countPoints} at all.  So what we really need
to do is to split the line into two:
\begin{verbatim}
 8       val countForJ =  countPoints(n, rand);
 8*      atomic count += countForJ;
\end{verbatim}
Because only one activity at a time can execute an atomic statement,
clearly the smart thing is to keep it as small as possible.

Races can occur whenever multiple activities share a resource.  In our example,
they shared a piece of storage, {\tt count}, but they could equally well, for
example, share an output stream.  Suppose three activities
call {\tt Console.OUT.println} at the same time.  What happens?  Answer:
it depends! Or perhaps better: ``We're off to the races!''  Sometimes each line
will print as desired, sometimes the two lines will be interleaved, and
sometimes all three will be. Try the following code, for example, on your
machine a bunch of times:
\begin{verbatim}
public class HelloAsync {
   public static def main(argv:Rail[String]!) {
      async Console.OUT.println("Hello, World");
      async Console.OUT.println("Hola, Mundo");
      async Console.OUT.println("Bonjour, Monde");
   }
}
\end{verbatim}
Here's our console log for our own first shot at running it:
\begin{verbatim}
% x10 HelloAsync
BHelonlo,j oWuorr,l dM
onde
Hola, Mundo
\end{verbatim}
``Hola, Mundo'', got delivered intact, but ``Hello, World'' and
``Bonjour, Monde'' got pretty well interleaved.  Who knows what might have
happened on another run!  Even more amusing: why was the {\em second} of the
three {\tt asyncs} the one that wasn't interrupted?  The first, maybe--the last,
not too surprising--but, the {\em second}?  Moral: where there are races, the
outcomes really are unpredictable. 

Bad as this behavior is, it is all that \Xten{} can reasonably
guarantee, because enforcing atomicity---on calls to {\tt Console.OUT},
for example---is not free, and knowing exactly when and where it is required is
something that really only the programmer can  know. So \Xten's {\tt atomic}
statement makes it easy for the programmer to enforce atomicity, and that's
really the best thing.

Putting all this discussion to use, we get a first cut at {\tt MontePiCluster}
in Figure \ref{fig:mpmtmc} on the next page.
\begin{figure}[!htbp]
\hrulefill
%%START X10: src/intro/MontePiCluster.x10 mpicluster
\fromfile{MontePiCluster.x10}
\begin{xtennum}[]
import x10.util.Random;
/**
 * A parallel version of the Monte Carlo estimate for pi that uses 
 * several Places and several threads at each place.
 */
public class MontePiCluster {
    /**
     * At the current Place, spawn some threads, each of which
     * generates n random points and return the total number
     * (combining all of the threads results) that fell inside
     * the circle.
     * @param pId: this process's id: used to create the seed for the
     *    random number generator.
     * @param threads: how many threads to use at this Place
     * @param n: how many points for each thread to generate
     * @return the total for all the threads of the number of points
     * that landed inside the circle. 
     */
    public static def countAtP(pId: Int, threads: Int, n: Long) {
        var count: Long = 0;
        finish for (j in 1..threads)  {
            val jj = j;
            async {
                val r = new Random(jj*Place.MAX_PLACES + pId);
                val rand = () => r.nextDouble();
                val jCount = countPoints(n, rand);
                atomic count += jCount;
            }
        }
        return count;
    }
    /**
     * Generate n points at random in the unit square, and return
     * the number that fell within the unit circle.
     * @param n the number of points to generate
     * @param rand the function generating the random numbers
     * @return the number of points that landed in the circle
     */
    public static def countPoints(n:Long, rand:()=>Double) {
        var inCircle: Long = 0;
        for (j in 1..n) {
            val x = rand();
            val y = rand();
            if (x*x +y*y <= 1.0) inCircle++;
        }
        return inCircle;
    }
   /**
    * There are three optional command line arguments: args(0) is the
    * number of points to generate, and args(1) is the number of
    * parallel activities to use, and args(2) is the number of 
    * threads to use at each Place.
    */
   public static def main(args: Array[String](1)) {
      val N = args.size > 0 ? Long.parse(args(0)) : 1000000L;
      val places = args.size > 1 ? Int.parse(args(1)) : Place.MAX_PLACES;
      val tPerP = args.size > 2 ? Int.parse(args(2)) : 4;
      val nPerT = N/(places * tPerP);
      val inCircle = new Array[Long](1..places);
      finish for(k in 1..places) {
         val kk = k;
         val pk = Place.place(k-1);
         async inCircle(kk) = at(pk) countAtP(kk, tPerP, nPerT);
      }
      var totalInCircle: Long = 0;
      for(k in 1..places) {
         totalInCircle += inCircle(k);
      }
      val pi = (4.0*totalInCircle)/(nPerT * tPerP * places);
      Console.OUT.println("Our estimate for pi is " + pi);
   }
}
\end{xtennum}
%%END X10: src/intro/MontePiCluster.x10 mpicluster
\hrulefill
\caption{Multi-place, multi-threaded Monte Carlo:
\href{http://dist.codehaus.org/x10/documentation/guide/src/montePi/MontePiCluster.x10}{\em
montePi/MontePiCluster.x10}}\label{fig:mpmtmc}
\end{figure}

You've now seen almost all of the principal ingredients for writing multi-threaded
\Xten{} code: {\tt async} to spawn activities, {\tt finish} to know when a
set of activities is complete, {\tt at} to shift the action to some other
processor, and {\tt atomic} to maintain the integrity of access to shared
data.  We'll fill out the picture in Chapter \ref{chap:concurrency}.
But first we want to take some time to look at how object oriented
programming looks in \Xten.

{\bf Exercise:} You might want to try the following slight variation on the Monte-Carlo
code.   Write a method that
uses one or more {\tt Place}s to sum the values of a function  {\tt f:(d:Double)=>Double} over a 
sequence of {\tt n Doubles}  {\tt d,} {\tt d+delta,} ... {\tt d+n*delta}:

\begin{quote}\begin{equation}
\sum_{k=0}^{n-1}  f(d + k\delta)
\end{equation}
\end{quote}

For debugging sake, try some simple {\tt f}'s to begin, like {\tt (d:Double) =>
1.0}.  Then you can go hog-wild using your own functions (\eg {{\tt
(d:Double)=>d*d}}), or the functions in {\tt x10.Math}, like {\tt sin}, {\tt
log}, and {\tt sqrt}.

The amusing questions are: what part of the sum is a given activity responsible for,
and how do we combine the partial results?  Combining the results is essentially
the same as what we've done for {\tt MontePiCluster} here.  Splitting up the
sum, though, requires some thought.  Here's one approach:

If there are $p_{all}$ {\tt Place}s in all, and there are $a$ activities
in parallel at each {\tt Place}, there will be $ap_{all} = a_{all}$
activities in all. Since there are
$n$ values to be summed, each activity should handle roughly $n/a_{all}$
additions---``roughly'' because $a_{all}$ might not divide $n$ evenly.
One solution is to let the $j$-th
activity $( j=0,1,...,a-1)$ at the {\tt Place} whose id is {\tt p}  take care of the values 
$d+k\delta$, where $k$ runs over $(p a+j)+h a_{all}$ for $h = 0,1,\dots$

We'll come back to this sort of loop in gory detail when we dive deeper into \Xten{} arrays.


\section{Design considerations}
\label{sec:lang}
%A {\em serial} schedule for a parallel program is one which always
executes the first enabled step in program order. A {\em safe}
parallel program is one that can be executed with a serial schedule
$S$ and for which for every input every schedule produces the same
result (error, correct termination, divergence) as $S$.  Such a
program is semantically a sequential program, hence it is
scheduler-determinate and deadlock-free.

A {\em safe parallel programming language} is an imperative parallel
programming language in which every legal program is a safe
program. Programmers can write code in such a language secure in the
knowledge that they will not encounter a large class of parallel
programming problems. Such a language is particularly useful for
parallelizing sequential (imperative) programs. In such cases ({\em
  contra} reactive programming) the desired application semantics are
sequential, and parallelism is needed solely for efficient
implementation.

The characteristic property of a safe programming language is that the
{\em same} program has a sequential reading and a parallel reading,
and both are compatible with each other. Hence the program can be
developed and debugged as a sequential program, using the serial
scheduler, and then run the unchanged program in parallel.  Parallel
execution is guaranteed to effect only performance, not
correctness. Safety is a very strong property.

There are many challenges in designing an efficient, usable, powerful,
explicitly parallel imperative language that is safe. One central
challenge is that 

One way to get safety is through implicitly parallel languages (e.g.{}
Jade~\cite{Rinard98thedesign},\cite{vonPraun:2007:IPO:1229428.1229443}). One
starts with a sequential programming language, and adds constructs
(e.g.{} tasks) that permit speculative execution while guaranteeing
that the only observable write to a shared variable is the write by
the last task to execute in program order.  While this work is
promising, extracting usable parallelism from a wide variety of
sequential programs remains very hard.

Explicitly parallel programming languages provide a variety of
constructs for spawning tasks in parallel and coordinating between
them. Here the programmer can typically directly control the
granularity of concurrency, and locality of access (e.g. placement of
data-structures in a multi-node computation) and use efficient
concurrent primitives (atomic reads/writes, test and sets, locks etc)
to control their execution. 

For such languages proving that a program is safe -- much less that
the programming language is safe -- now becomes very hard,
particularly for modern object-oriented languages which allow the
programmer to create arbitrarily complicated data-structures in the
shared heap. It becomes very difficult to show that any possible
schedule will produce the same result as the serial schedule.

Our starting point is the language X10 \cite{x10} since it offers a
simple and elegant treatment of concurrency and distribution, with
some nice properties.  In brief, X10
introduces the constructs \code{async S} to spawn an activity to
execute \code{S}; \code{finish S} to execute \code{S} and wait until
such time as all activities spawned during its execution have
terminated; \code{at (p) S} to execute \code{S} at place
\code{p}. These constructs can be nested arbitrarily -- this is a
source of significant elegance and power. Additionally, X10 v 2.2
introduces a simplified version of X10 clocks (adequate for many
practical usages) -- \code{clocked finish S} and \code{clocked async
  S}. Briefly, a clocked finish introduces a new barrier that can be
used by this activity and its children activities for
synchronization.\footnote{X10 also has a conditional atomic construct,
  \code{when (c) S} which permits data-dependent synchronization and
  can introduce deadlocks. We do not consider this construct in this paper.}

\cite{vj-clock} establishes that a large class of programs
in X10, namely those that use \code{finish}, \code{async} and
\code{clock}s are deadlock-free. The central intuition is that a clock
can only be used by the activity that created it and by its children,
and hence the spawn tree structure can be used to avoid depends-for
cycles. 

To obtain determinacy, another idea is required. The central problem
is to ensure that in the statement \code{async \{S\} S1} it is not the
case that \code{S} can write to a location that \code{S1} can read
from or read from a location that \code{S} can read from. Otherwise
the behavior is not determinate. One line of attack has been the pursuit
of {\em effect systems} \cite{Lucassen:1988:PES:73560.73564},
\cite{Leino:2002:UDG:543552.512559},
\cite{boyland:01interdependence},
\cite{DPJ}. At a broad conceptual level, static effects systems call
for a user-specified partitioning of heap into {\em regions} in a
fine-grained enough way to show that operations that may occur
simultaneously work on different regions. For instance \cite{DPJ}
introduces separate syntax for regions, introduces the ability to
specify an arbitrary tree of regions, and new syntax for specifying
which region mutable locations belong to. Methods must be associated
with their read and write effects which capture the set of regions
that can be operated upon during the execution of this
method. Now determinacy can be established if it can be determined
that that in \code{async \{S\} S1} it is the case that \code{S} does
not write any region that \code{S1} reads or writes from, and vice
versa ({\em disjoint parallelism}). 

\cite{DPJ} develops these ideas in the context of a language with
\code{cobegin/coend} and \code{forall} parallelism, and does not
address arbitrary nested or clocked parallelism.  In particular it is
not clear how to adapt these ideas to support some form of pipelined
communication between multiple parallel activities. Once can think of
these computations as requiring ``disjointness across time'' rather
than disjointness across space. A producer is going to write into
locations \code{w(t),w(t+1),w(t+2), \ldots}, but this does not
conflict with a consumer reading from the same locations as long as
the consumer can arrange to read the values in time-staggered order,
i.e. read \code{w(t)} once the produce is writing \code{w(t+1)}.  

We believe it is possible to significantly simplify this approach
(e.g.{} using the dependent-type system of X10) and extend it to cover
all of X10's concurrency constructs (see
Section~\ref{future-work}). Nevertheless the cost of developing these
region assertions is not trivial. Their viability for developing large
commercial-strength software systems is yet to be establishd.

In this paper we chose a different approach, a {\em lightweight}
approach to safety. We introduce two simple ideas -- {\em
  accumulators} and {\em clocked types} -- which require very modest
compiler support and can be implemented efficiently at run-time.

Accumulators arise very naturally in concurrent programming: an
accumulator is a mutable location associated with a commutative and
associative {\em reduction} operator that can be operated on
simultaneously by multiple activities. Multiple values offered by
multiple activities are combined by the reduction operator. We show
that the concrete rules for accumulators can be defined in such a way
that they do not compromise safety: a serial execution precisely
captures results generable from any execution.

Similarly clocked computations (barrier-based synchronization) is
quite common in parallel programming e.g. in the SPMD model, BSP model
etc. We observe that many clocked programs can be written in such a
way that shared variables take on a single value in one phase of the
clock. Further, clocked computations are often iterative and operate
on large aggregate data-structures (e.g. arrays, hash-maps) in a
data-parallel fashion, reading one version of the data-structure (the
``red'' version) while simultaneously writing another version (the
 ``black'' version). To support this widely used idiom, we introduce
the notion of {\em clocked types}. An instance \code{a} of a  clocked type
\code{Clocked[T]} keeps two instances of type \code{T}, the \code{now} and
the \code{next} instance. \code{a} can be operated upon only by
activities registered on the current clock.  All read operations
during the current clock phase are directed to the \code{now} version,
and write operation to the \code{next} version. This ensures that
there are no read-write conflicts. There may be write-write conflicts
-- these must be managed by either using accumulators or an effects
system. Once computation in the current phase has quiesced -- and
before activities start in the next phase -- the \code{now} and
\code{next} versions are switched; \code{now} becomes \code{next} and
\code{next} becomes \code{now}.\footnote{Clearly this idea can be
  extended to $k$-buffered clocked types where each clock tick rotates
  the buffer. This idea is related to the K-bounded Kahn networks of \cite{k-bounded-kahn}.}

We show that clocked types can be defined in a safe way, provided that
accumulators are used to resolve write-write conflicts. The only
dynamic check needed is that a value of this type is being operated
upon only by the activity that created the value or its
descendants. 

In the following by Safe X10 we shall mean the language X10 restricted
to use (\code{clocked}) \code{finish}/\code{async}, \code{at} with
(clocked) accumulators.  All programs in Safe X10 are safe -- they can
be run with a serial schedule and their I/O behavior is identical
under any schedule.  We show that Safe X10 is surprisingly
powerful. Many concurrent idioms can be expressed in this language --
histograms, all-reduce, SpecJBB-style communication Indeed, even some
form of pipelined/systolic communication is expressible.

Since the dynamic conditions introduced on accumulators and clocked
types are not straightforward, we formalize the concurrent and serial
semantics of an abstraction of Safe X10 using Plotkin's structural
operational style. We are able to do this in such a way that the two
proof systems share most of the proof rules, simplifying the proof. We
establish that the language is safe -- for any program and any input,
any execution sequence for the concurrent proof rules can be
transformed into an execution sequence for the sequential proof rules
with the same result. 

In summary the contributions of this paper are as follows.

\begin{itemize}
\item We identify the notion of a {\em safe} program -- one which can
  be executed with a serial schedule and for which 
  every schedule produces the same result. Such a program is
  simultaneously a sequential program and a parallel program with
  identical I/O behavior. 
\item We introduce accumulators and clocked types in the X10
  programming model. These are introduced in such a way that arbitrary
  programs using (\code{clocked}) \code{finish}, \code{async} and
  \code{at} and in which the only variables shared between
  concurrently executing activities are accumulators or clocked
  accumulators are guaranteed to be safe.  
\item We show that many common programming idioms can be expressed in
  this language.
\item We formalize a fairly rich subset of X10 -- including (clocked)
  finish, async, accumulators and clocked accumulators.  This is the
  first formalization of the nested clock design of X10 2.2, and is
  substantially simpler than \cite{vj-clock}. We establish that this
  language is safe. 
\end{itemize}
In companion work we show how these ideas can be extended to support
modularly defined effects analyses, using X10's dependent type system.

The rest of this paper is as follows. In \Ref{related-work} we discuss
related work. In \Ref{constructs} we present the constructs in detail,
followed by examples of their use. \Ref{semantics} presents the
semantics of these constructs. We discuss implementation in
\Ref{implementation} and finally conclude with future work.

%%\cite{Gifford:1986:IFI:319838.319848}
%%
%%
%%Figure~\ref{fig:1} shows  the famous ``parallel Or'' program of Plotkin
%% (in X10 syntax, \cite{x10}). This program can be executed with a
%%depth-first schedule, is partially determinate and deadlock-free, but {\em not}
%%safe. The result of running the sequential schedule is not the same as
%%the result that can be obtained with other schedules. Specifically
%%\code{parallelOr(()=> CONT, ()=>TRUE)} will diverge (exhibit an
%%infinite exection sequence) under the depth-first schedule, but will
%%return \code{true} under any fair schedule that permits the second
%%async to progress.
%%
%%\begin{figure}
%%  \begin{lstlisting}
%%static val CONT=1, TRUE=2, FALSE=3;
%%def run(done:Cell[Boolean], a:()=>Int) {
%% var aa:Int=a();
%% var cont:Boolean=true;
%% for (; aa==CONT && cont;aa=a()) {
%%  atomic cont = !done();
%% }
%% if (aa==TRUE)
%%  atomic done()=true;
%%}
%%def parallelOr(a:()=>Int, b:()=>Int):Boolean {
%% val done=new Cell[Boolean](false);
%% finish {
%%  async run(done, a);
%%  async run(done, b);
%% }
%% return done();
%%}
%%  \end{lstlisting}
%%  \caption{A program that is not sequential}\label{fig:1}
%%\end{figure}



%%{\em
%%\begin{enumerate}
%%\item Use activity registration as a mechanism to tame object graphs.
%%\item Focus on structured concurrency. Using scoping and block-structure
%%    to delimit regions of code that may execute in parallel and affect
%%    the data structure.
%%
%%\item Accumulation can be defined safely by delaying. However, the delay
%%    operation is guaranteed to be deadlock-free.
%%
%%\item Clocked types support phased computation, another common idiom
%%    particularly for stencil computations.
%%\end{enumerate}
%%}
%%
%%Key contributions:
%%{\em
%%\begin{enumerate}
%%\item Identification of determinate, deadlock-free data-structures.
%%\item Discussion of design alternatives which points out the
%%  difficulty of integrating these ideas in a modern OO language.
%%\item Discussion of various idioms expressible using these data-structures.
%%\item Proof of determinacy and deadlock-freedom in an abstract version
%%  of the language.
%%\end{enumerate}
%%These constructs are implemented in \Xten, available as open source from
%%SVN head and will be in the next release of \Xten.
%%}


%Semantics and theorems for an abstract version of the language.




\Xten{} is a class-based object-oriented language
that provides both dependent and generic types.
The language has a sequential core similar to Java or Scala, but
also
constructs for concurrency and distribution.

A key feature that interacts with generics is that the type
system provides both reference types and value types.
An instance of a reference type is an object on the
heap.  All reference types are subclasses of \xcd"Object".
Variables of reference type may be \xcd"null".
In contrast, an instance of a value type might be represented in
unboxed form on the call stack
and can never be \xcd"null".
Ideally, the design should support instantiation of generics on both 
reference and value types.

This section describes the design of generics for \Xten,
including several alternative designs.  These alternatives
demonstrate the expressiveness of constrained kinds.

%\footnote{We plan
%to support traits in a future version of the language.}
\todo{emph interactions with data constraints}

\subsection{Type constraints}

To permit genericity, variables \Xcd{X} must be admitted over types.
The choice of type variables is discussed below.  We assume here
that classes have a means of introducing new type variables
either as type parameters or as type members.
For instance,
the class \xcd"List" introduces a type variable \xcd"X"
representing the list's element type.

\Xten already supports constraints over values, so it is natural
to extend these to constraints over types.
Here, we ask: how should type variables be constrained?

Constraints occur in several places in the \Xten syntax.  They
are of course permitted in constrained types \xcd"C{c}".
Constraints may also be used as \emph{class invariants}, 
which are constraints on the class's properties and other
final variables in scope.
The class invariant must be established by the class's
constructor and subsequently holds for all instances of the
class.

Methods and constructors
may also have constraints, or \emph{guards}, on their parameters.
A guard must be satisfied by the caller of the method and
will hold throughout its body.  Type constraints used in the
method guard restrict the types of the arguments or of the method
receiver.  

\subsubsection{Nominal subtyping and equality constraints}

In class-based OO languages such as Java,
types are equipped with a partial
order (the \emph{subtyping} order) generated from the user program
through the ``\Xcd{extends}'' relationship.  
This motivates a very natural constraint system on types.  For a type
variable \Xcd{X} we should be able to assert the constraint
\Xcd{X}~$\extends$~\Xcd{T}: a valuation (mapping from variables to types) realizes
this constraint if it maps \Xcd{X} to a type that extends \Xcd{T}.
Constraints on types can specify either subtype (\xcd"<="),
supertype (\xcd">="), or equality bounds (\xcd"==").

Using subtyping constraints in the class invariant provides a
means to bound the type variables introduced by the class
declaration.  Constraints in constrained types
\xcd"C{c}", can bound
type-valued members of the base type \xcd"C".

These constraints also can be used in method guards.
This feature is similar to optional methods in CLU~\cite{clu} and to generalized type constraints in C$\sharp$~\cite{emir06}.
For instance, given a list of \xcd"T", one could define a
method \xcd"print" with a guard that requires that \xcd"T" be a
subtype of \xcd"Printable":
\begin{xtenmathnoindent}
  def print(){T <= Printable} {
    head.print();
    tail.print();
  }
\end{xtenmathnoindent}
This constraint ensures that the \xcd"head" field of type
\xcd"T" has a \xcd"print()" method.

\subsubsection{Structural constraints}

One should also
be able to require that a type have a
particular member---a field with a given name and type, or a method
with a given name and signature.
We introduce the constraints 
\Xcd{T} \Xcd{has} \Xcd{f:T} and \Xcd{T} \Xcd{has}
\Xcdmath{m($\tbar{x}\ty\tbar{S}$):T} to express this.
These
constraints allow one to define an alternative version of the
\xcd"print" methods above  as:
\begin{xtenmathnoindent}
  def print(){T has print(): void} {
    head.print();
    tail.print();
  }
\end{xtenmathnoindent}
Rather than restricting the actual receiver to lists whose
element type implements \xcd"Printable", with structural
constraints, any list whose element type has a \xcd"print"
method may be used.
\eat{
This feature makes it easier to integrate
third-party libraries, where interface names might not be
compatible.
}

Structural constraints on types are found in many languages.
For instance,
Haskell supports type
classes~\cite{haskell,haskell-type-classes}.
%
%ML's module system allows modules to be constrained by
%structural signatures~\cite{ml}.
In Modula-3, type equivalence is structural
rather than nominal as in object-oriented languages of the C
family (e.g., C++, Java, and \Xten{}).
Unity~\cite{malayeriIntegrating08}
is a Java-like language with both nominal and structural subtyping.

In the class invariant, a structural constraint can bound the
class's type variables, similarly to 
the language PolyJ~\cite{java-popl97}, which allows type
parameters to be
bounded using
structural \emph{where clauses}~\cite{where-clauses}.
For example, a sorted list class
could be written as follows in PolyJ:
{
\begin{xtennoindent}
class SortedList[T] where T {int compareTo(T)} {
  void add(T x) {... x.compareTo(y) ...}
  ...
}
\end{xtennoindent}}
The \xcd"where" clause states that the type parameter
\xcd"T" must have a
method \xcd"compareTo" with the given signature.
\xcd"SortedList" can be instantiated on any type
that provides the method.
With nominal bounds, \xcd"SortedList" could only require that
its parameters implement an interface such as \xcd"Comparable".

\subsubsection{Default values}

Recall that \Xten's type system provides both reference types and value
types.
In languages like Java with
primitive types, every type has a default value---\xcd"null" for
reference types, \xcd"false" or \xcd"0" for primitive
types---used to initialize arrays of that type.
In \Xten, some types do not have an obvious default value.  For example,
\xcd"int{self>0}" does not contain the value \xcd"0".
Consequently, a useful constraint is \xcd"T has default", which
holds if the type \xcd"T" has a default value.

\subsubsection{{\tt instanceof} constraints}

Lastly, we consider constraints of the form \xcd"x" \xcd"instanceof" \xcd"T".
By relating types and values in a single constraint, 
these constraints provide considerable expressive power.
For instance, 
consider the class declaration:
\begin{xtennoindent}
class C {
  def equals[T](x:T) {this instanceof T} = (this==x);
}
\end{xtennoindent}
The \xcd"equals" method can be called with any object
that is a supertype of \xcd"C".

\xcd"instanceof" constraints can be used to build intersection
types, e.g., \xcd"Object{self instanceof A,self instanceof B}"
should be a subtype of \xcd"A" and \xcd"B"..

\subsection{Type variables}
\label{sec:type-properties}
\label{sec:variance}

Languages such as Java~\cite{Java3} and
Scala~\cite{scala} introduce \emph{type parameters} on classes
and methods.
An alternative approach, used by BETA~\cite{beta}, 
Scala~\cite{scala}, and other languages is, to use type members.
In \Xten, one can
generalize properties to include type-valued properties:
A \emph{type property}
is a final object member initialized at construction time with a
concrete type.  

\subsubsection{Type properties}

\label{sec:usability}
\label{sec:parameters-vs-fields}

Like normal value properties, type properties
can be used in constrained types through the variable \xcd"self".
%
This immediately suggests use-site variance
constraints~\cite{unifying-genericity,variant-parametric-types}
on type properties.
The type of a list of integers, say, can be written as
\xcd"List{self.T==int}".  
Nominal subtyping constraints, then, may be used to
provide use-site variance constraints.
%
Consider the following subtypes of \xcd"List" with type property
\xcd"T":
\begin{itemize}
\item \xcd"List".  This type has no constraints on the type
property \xcd"T".
Any type that constrains \xcd"T"
is a subtype of \xcd"List".
\eat{
The type \xcd"List" is equivalent to \xcd"List{true}".
%
For a \xcd"List" \xcd"v", the return type of the \xcd"get" method
is \xcd"v.T".
Since the property \xcd"T" is unconstrained,
the caller can only assign the return value of \xcd"get"
to a variable of type \xcd"v.T".
}

\item \xcd"List{T==float}".
The type property \xcd"T" is bound to \xcd"float".
For a final expression \xcd"v" of this type,
\xcd"v.T" and \xcd"float" are equivalent types and can be used
interchangeably.

\item \xcdmath"List{T$\extends$Collection}".
This type constrains \xcd"T" to be a subtype of \xcd"Collection".
All instances of this type must bind \xcd"T" to a subtype of
\xcd"Collection"; for example \xcd"List[Set]" (i.e.,
\xcd"List{T==Set}") is a subtype of
\xcdmath"List{T$\extends$Collection}" because \xcd"T==Set" entails
\xcdmath"T$\extends$Collection".

\item \xcdmath"List{T$\super$String}".  This type bounds the type property
\xcd"T"
from below. 
\end{itemize}

While expressive,
type properties have a number of usability issues.
The key difference between type parameters and type properties
is that type properties are
instance \emph{members} bound during object construction.  Type
properties are thus accessible through expressions---\xcd"e.T" is
a legal type (if \xcd"e" is final)---and are inherited by subclasses.
These features give type properties more expressive power than
type parameters, as we shall describe below; however, because they 
provide similar functionality with often subtle distinctions,
type properties can be difficult to use, especially for novices,
and require more care in the design.
For instance,
since type properties are inherited,
the language design needs
to account for ambiguities introduced when the same name is
used for different type properties declared in or inherited into a class.

\eat{
Inheriting type properties may also lead to confusion
As an example, in the following hypothetical code extended with
type properties (declared as normal properties with the ``type''
\xcd"*"),
\xcd"HashMap"  inherits the properties \xcd"K" and \xcd"V" from
\xcd"AbstractMap".
\begin{xten}
class AbstractMap(K:*, V:*) {
  abstract def get(K): V;
  abstract def put(K, V): V;
}

class HashMap implements Map {
  def get(k: K): V = ...;
  def put(k: K, v: V): V = ...;
}
\end{xten}
A user more familiar with type parameters might declare
\xcd"HashMap" as follows:
\begin{xten}
class HashMap(K:*,V:*) implements Map(K,V) {
  def get(k: K): V = ...;
  def put(k: K, v: V): V = ...;
}
\end{xten}
This declaration would introduce a new pair of type properties
named \xcd"K" and
\xcd"V" that shadow the inherited properties.
A na{\"\i}ve implementation of type properties would store run-time
type information for all four properties in each instance
of \xcd"HashMap".
}

\paragraph{\normalfont\bf\em Virtual types.}

Type properties provide expressive power much like 
\emph{virtual
types}~\cite{beta,mp89-virtual-classes,ernst06-virtual};
moreover, they can also
be constrained at the use-site,
can be refined on a per-object basis without explicit subclassing,
and can be refined contravariantly as well as covariantly.

Thorup~\cite{thorup97}
proposed adding genericity to Java using virtual types.  For example,
a generic \xcd"List" class can be written as follows:
{
\begin{xten}
abstract class List {
  abstract typedef T;
  T get(int i) { ... }
}
\end{xten}}
\noindent
The virtual type \xcd"T" is unbound in \xcd"List", but 
can be refined by binding \xcd"T" in a subclass:
{
\begin{xten}
abstract class NumberList extends List {
  abstract typedef T as Number;
}
class IntList extends NumberList {
  final typedef T as Integer;
}
\end{xten}}
\noindent
Only classes where \xcd"T" is final bound, such as \xcd"IntList",
can be non-abstract.  Scala~\cite{scala} supports abstract types
and virtual types in a similar way.
%
The analogous definition of 
\xcd"List" in \Xten{} using type properties is as follows:
{
\begin{xten}
class List(T:*) {
  def get(i: int): T { ... }
}
\end{xten}}

\noindent
Unlike the virtual-type version,
the \Xten{} version of \xcd"List" is not abstract;
\xcd"T" need not be instantiated by a subclass because it can be
instantiated (constrained) on a per-object basis.
Rather than declaring subclasses of \xcd"List",
one uses the constrained subtypes
\xcdmath"List{T$\extends$Number}" and \xcd"List{T==Integer}".

Type properties can also be refined contravariantly.
For instance, one can write the type \xcdmath"List{T$\super$Integer}".

\paragraph{Self types.}

Type properties can also be used to support a form of self
type~\cite{bruce-binary,bsg95}.
%
Self types can be implemented by introducing a
type property \Xcd{type} to the root of the class hierarchy,
\Xcd{Object}:
\begin{xtenmath}
class Object(type:*){type <= Object} { $\dots$ }
\end{xtenmath}

\noindent
For any final path expression \Xcd{p}, the type
$\Xcd{p}.\Xcd{type}$ represents all instances of the fixed,
but statically unknown, run-time class referred to by \Xcd{p}.
Scala's path-dependent types~\cite{scala} and J\&'s
dependent classes~\cite{nqm06}
take a similar approach.

Self types are achieved by
constraining types so that if a path expression \Xcd{p}
has type \Xcd{C}, then
$\Xcd{p}.\Xcd{type} \subtype \Xcd{C}$.
In particular, one can add the class invariant
$\Xcd{this}.\Xcd{type} \subtype \Xcd{C}$ to every class \Xcd{C}.
This invariant ensures that
$\Xcd{this}.\Xcd{type}$ is a subtype
of the lexically enclosing class.

The property must be initialized to the given class, so, without
further language support, one must create an instance of
\xcd"Object" with \xcd"new" \xcd"Object(Object)" to initialize
the \xcd"type" property.

\subsubsection{Type parameters}

Most OO languages provide genericity through type parameters on
classes and methods.  The development of a nominal OO type
system with type parameters is now standard (cf.  FGJ~\cite{FJ}).

Scala~\cite{scala} supports definition-site variance
annotations:
a parameter may be declared in-, co-, or
contravariant.
If the parameter \xcd"X" of a class \xcd"C" is covariant,
then \xcd"S" a subtype of
\xcd"T" implies  \xcd"C[S]" is a subtype of \xcd"C[T]".
Similarly, if \xcd"X" is contravariant,
\xcd"C[T]" is a subtype of \xcd"C[S]".
Invariant parameters are the default; a covariant parameter is
declared by prepending ``\xcd"+"'' to the parameter name in the
class header; a contravariant parameter is declared by
prepending ``\xcd"-"''.  The usage of variant parameter types in
the body of their class must be
restricted to ensure the subtyping relation holds.

Java, by contrast, supports use-site variance through wildcards.
This has a number of usability problems~\cite{wildcards-are-evil},
which also occur with constrained type properties, above.

\subsection{Overloading and dispatch}

The next question to address is the overloading semantics for
methods with constraints on formal parameters and with method
guards.  This issue was considered in non-generic \Xten but was
revisited in light of type constraints.

One option is to ignore constraints when checking for
overloading.  This means that \xcd"m(int{self==0})" and
\xcd"m(int{self==1})", for instance, are considered to have the
same signature; if both occur within the same class, a
compile-time error occurs.

Another option is to allow the
overloading: methods are resolved at compile-time, based on the
constraints.  It is an error if a call could resolve to more
than one method.  One question is whether to rule out overlapping
methods (e.g., \xcd"m(int{self>=0})" and \xcd"m(int{self==1})"),
or to permit them and have the caller resolve any
ambiguities.

\todo{Use type constraints, not value constraints}

Allowing the overloading on constraints can also complicate method
overriding by introducing partial overrides.
Consider:
\begin{xtennoindent}
  class A {
    def m(x:int{self<=0}) = ...; // 1
    def m(x:int{self>=0}) = ...; // 2
  }
  class B {
    def m(x:int{self==0}) = ...; // 3
  }
\end{xtennoindent}
\noindent
\xcd"B.m"'s constraint on \xcd"x" partially overrides the
constraint on both \xcd"m" methods of \xcd"A".  Given a variable
\xcd"b" of type \xcd"B": \xcd"b.m(-1)" invokes \xcd"A.m" (method 1),
\xcd"b.m(0)" invokes \xcd"B.m" (method 3), and \xcd"b.m(1)"
invokes \xcd"A.m" (method 2).  Clients of \xcd"B" could get
confused about which method gets invoked.  One option is to
require that when a method with a given name is overridden, all
other methods with that name should be overridden as well.

Finally, one could support a form of predicate
dispatch~\cite{jpred}, selecting the method to invoke by
\emph{dynamically} evaluating the method guard. 
With type constraints and predicate dispatch, multi-method
dispatch can be implemented.  \todo{example}

\subsection{Implementation}

Finally, we turn to the implementation of generics.
To implement a generic class \xcd"C[X]" one can either generate a single 
class for \xcd"C" in the target language (homogeneous translation)
or generate one class per instantiation
\xcdmath"C[T$_1$]", \dots,
\xcdmath"C[T$_k$]" (heterogeneous translation).
The former approach reduces the amount of generated code; the
latter enables specialization based on the type arguments to
\xcd"C".  Hybrid approaches are possible as well.

Java's approach is to erase type parameters and to use the homogeneous
translation.  Erasure admits more dynamic errors because
it permits, for instance, a \xcd"C<A>" to be cast to \xcd"C<B>".
Retrieving a field of static type \xcd"B" could cause a run-time
type error when an \xcd"A" is returned instead.
The homogeneous translation is aided by a restriction that type
parameters cannot be instantiated on primitive types and by
using nominal subtyping bounds on types.
These restrictions ensure type parameters can be represented
with their type bound, or \xcd"Object" if unbounded.
Since, in \Xten, it should be possible to instantiate a generic
type on both value types and reference types, a homogeneous translation
must box value types so that both kinds of types have the same
representation.

PolyJ~\cite{java-popl97} supports structural bounds and uses a
homogeneous translation with adapter objects to allow generic
code to invoke methods on values of its type parameters.

Other languages, such as C++, use a heterogeneous
translation, specializing the generic class for each
instantiation.
C$\sharp$,
NextGen~\cite{nextgen}, and
Fortress~\cite{fortress} takes this approach as well, reducing
(static) code bloat by instantiating generic classes at run time.

A compromise approach is to specialize for only a few parameter
types, for example the primitive types, but to use a homogeneous
translation otherwise.

Representing type variables at run-time allows the language
to support run-time casts to generic types,
including possibly types instantiated on constrained types.

With
non-generic types, a cast such as
\xcd"r"~\xcd"as"~\xcd"Region{rank==k}" can be implemented by
checking the run-time class of the value being
cast---\xcd"r"~\xcd"instanceof"~\xcd"Region"---and then
evaluating the constraint---\xcd"r.rank==k".
%
However, the issue is more subtle with generic casts.
For instance, to do
\xcd"A"~\xcd"as"~\xcd"Array[int{self>=0}]"
one must check at run time that the concrete type used to instantiate
the \xcd"Array"'s type parameter is equivalent to
\xcd"int{self>=0}".  This check could involve a run-time
entailment check, 
breaking the phase distinction between
compile time and run time for constraint solving.

One approach is to restrict the language 
to rule out casts to type parameters 
and to generic types with subtyping constraints, ensuring that
entailment checks are not needed at run time.
Alternatively, 
the constraint solver could be embedded into the runtime system.
However, this
solution can result in inefficient run-time casts
if entailment checking for the given constraint system is expensive.
Finally, one can simply erase the constraints from the run-time
type information, preserving the base type.  As with Java's
erasure semantics, this approach is prone to run-time type
errors.

\subsection{X10 design decisions}

Given these considerations, the \Xten makes the following choices:
\begin{itemize}
\item \Xten supports subtyping and equality constraints on types
\item \Xten does not support structural bounds, but may do so in
the future.  \Xten has closures with structural subtyping, which
can be used in many of the cases structural type bounds would be
used.
\item Classes have type parameters with definition-site variance
rather than type properties with use-site variance annotations.
Properties are just too unfamiliar.
Usability outweighs expressive power. 
\item Run-time type information is preserved, but constraints
are not.  
\end{itemize}


\subsection{Related work}
\label{sec:related}
Constraint-based type systems enjoy a long history.
The use of constraints for type inference and subtyping were
developed by
Mitchell~\cite{mitchell84}
and
Reynolds~\cite{reynolds85}.
%
Trifonov and Smith~\cite{trifonov96}
proposed a type system where types are refined by subtyping
constraints.  Dependent types are not supported.
%
Pottier~\cite{pottier96simplifying,pottier01b}
presents a constraint-based type system for an ML-like language with
subtyping.

HM(X)~\cite{sulzmann97type,pottier01a,pottier-remy-attapl}
is a constraint-based framework
for Hindley--Milner style type systems.
The framework is parameterized on the specific constraint system
X; instantiating X yields extensions of the HM type system.
Constraints in HM(X) are over types, not values.
%
% XXX HM(X) introduced {\em term constraint systems}; constraints in
% CFJ are term constraints?

% Cardelli~\cite{cardelli86}, type checking dependent types and
% subtypes.

% Russell
% \cite{fuh88}
% \cite{curtis90}
% \cite{aiken93}

% Aiken, Wimmers, and Lakshman proposed {\em conditional
% types}~\cite{conditional-types}, which have the ability to
% encode control-flow analysis of {\tt case} expressions.
% Conditional types are not dependent.

% \cite{smith94}



% \cite{palsberg95}
% constraint-based inference algorithm for object calculus, 

% Henglein (TAPOS) set constraints for OO language type-inference.

% Bane~\cite{fahndrich99}

% Pottier

% CLP(X) framework in constraint logic programming (JM94)
% HM(X)

Several systems have been proposed that refine types in a base
type system through constraints.
%
{\em Refinement types}~\cite{refinement-types} extend the 
Hindley--Milner type system with intersection, union, and
constructor types, enabling specification and inference of
more precise type information.
%
{\em Conditional
types}~\cite{conditional-types} extend refinement types to
encode control-flow information in the types.
%
Jones introduced {\em qualified types}, which permit
types to be constrained by a finite set of
predicates~\cite{jones94}.
%
{\em Sized types}~\cite{sized-types}
annotate types with the sizes of recursive data structures.
Sizes are linear functions of size variables.
Size inference is performed using a constraint solver for
Presburger arithmetic~\cite{omega}.
% constraints on types, support primitive recursion only

% Indexed types~\cite{indexed-types}

% Index objects must be pure.
% Singleton types int(n).
% ML$^{\Pi}_0$:
% Refinement of the ML type system: does not affect the
% operational semantics.  Can erase to ML$_0$.

% Jay and Sekanina 1996: array bounds checking based on shape
% types.

With hybrid type-checking~\cite{flanagan-popl06,flanagan-fool06},
types can be constrained by arbitrary boolean expressions.
While typing is undecidable, dynamic checks are inserted into
the program when necessary if the type-checker cannot determine
type safety statically.
In \Xten{} dynamic type checks, including tests of dependent
clauses, are inserted only at explicit casts.

% Ada dependent types.
% Ada has constrained array definitions.  A constraint
% \cite{ada-ref-man}.  Not clear if they're dependent.  Are
% there other dependent types?  Generics are dependent?

Singleton types~\cite{aspinall-singletons,stone00} are dependent
types containing only one value.  
In Stone's formulation~\cite{stone00},
$S(e : \tau)$
is the type of all values of type $\tau$ that are equal to $e$.
Term equivalence is
constructed so that type-checking is decidable.
The singleton $S(e: \tau)$ can be encoded in \Xten{} as
$\tau$\xcd{(:self ==}~$e$\xcd{)}.

        % Used for array bounds by Morrisett et al (I think--need
        % to find paper)

% Singleton types~\cite{aspinall-singletons}.

Several languages---gbeta~\cite{ernst99-gbeta},
Scala~\cite{scala-overview,scala-oopsla05}, J\&~\cite{nqm06}, and
others~\cite{oz01,ocrz-ecoop03,dependent-classes}---provide {\em path-dependent
types}.  For a final access path \xcd{p}, {\tt p.type}
in Scala is the singleton type containing the object \xcd{p}.
In J\&, {\tt p.class} is a type containing all objects
whose run-time class is the same as \xcd{p}'s.
Scala's {\tt p.type} can be encoded in \Xten{} using an equality
constraint \xcd{C(:self == p)}, where \xcd{C} is a supertype of
\xcd{p}'s static type.
\eat{
These types can be encoded in CFJ by introducing a
\xcd{type} property.
\rn{T-constr}, as
described in Section~\ref{sec:examples}.
}

% Where clauses for F-bounded polymorphism~\cite{where-clauses}
% Bounded quantification: Cardelli and Wegner.  Bound T with T'
% In F-bounded polymorphism~\cite{f-bounds}, type variables are bounded by a function of 
% the type variable. 
% Not dependent types.

\eat{
conditional types:

type of an expression can be constrained using information about
the results of run-time tests in the context surrounding the
expression.

e.g., can express that e2 is evaluated only if e1 is true

\begin{verbatim}
\y. case y of true => e1 | false => e2 :
        'a -> (true ? ('a ^ typeof(e1)) U (false ? ('a ^ typeof(e2))
\end{verbatim}

Types include type constructors applied to types.

\begin{verbatim}
        so,  true      : true
        but, (\x . x)  : 'a -> 'a
             node(l,r) : node('a tree, 'a tree)
\end{verbatim}


when checking a case branch, type of the expression being
matched refined to the include the type constructor for that
branch

captures some control flow analysis in the types

types
\begin{verbatim}
        t ::= t1 -> t2
                | c(..ti..) <-- type constructor
                | alpha
                | t1 U t2
                | t1 ^ t2
                | t1 ? t2
                | 0
                | 1

        sigma ::= t | \forall ..alpha.. t where ..ti <= tj..
\end{verbatim}
}


Cayenne~\cite{cayenne} is a Haskell-like language with fully dependent types.
There is no distinction between static and dynamic types.
Type-checking is undecidable.
There is no notion of datatype refinement as in DML.

Epigram~\cite{epigram,epigram-matter}
is a dependently typed functional programming language based on
a type theory with inductive families.
Epigram does not have a phase distinction between values and
types.

\eat{
$\lambda^{\sf Con}$ is a lambda calculus with assertions.
Findler, Felleisen, Contracts for higher-order functions (ICFP02)

  example: int[> 9]

contracts are either simple predicates or function contracts.
defined by (define/contract ...)

enforced at run-time.
}

% Jif~\cite{jif,jflow} is an extension of Java in which
% types are labeled with security policies enforced by the
% compiler.

ESC/Java~\cite{esc-java}
allow programmers to write object invariants and pre- and
post-conditions that are enforced statically
by the compiler using an automated theorem prover.
Static checking is undecidable and, in the presence of loops,
is unsound (but still useful) unless the programmer supplies loop invariants.
ESC/Java can enforce invariants on mutable state.

% and Spec$\sharp$~\cite{specsharp}

Pluggable and optional type systems were proposed by
Bracha~\cite{bracha04-pluggable} and provide another means of
specifying refinement types.
Type annotations, implemented in compiler plugins, serve only to
reject programs statically that might otherwise have dynamic
type errors.
CQual~\cite{foster-popl02} extends C with user-defined type
qualifiers.  These
qualifiers may be flow-sensitive and may be inferred. 
CQual supports only a fixed set of typing rules
for all qualifiers.
In contrast, the {\em semantic type qualifiers} of
Chin, Markstrum, and Millstein~\cite{chin05-qualifiers}
allow programmers to define typing rules for qualifiers
in a meta language that allows type-checking rules to be
specified declaratively.
JavaCOP~\cite{javacop-oopsla06} is a pluggable type system
framework for Java.  Annotations are defined in a meta language
that allows type-checking rules to be specified declaratively.
JSR 308~\cite{jsr308} is a proposal for adding user-defined type qualifiers
to Java.

% Holt, Cordy, the Turing programming language

% Ou, Tan, Mandelbaum, Walker, Dynamic typing with dependent types
% Separate dependent and simple parts of the program.
% Statically type the dependent parts.
% Dynamic checks when passing values into dependent part.

Our work is most closely related to \DML{}, \cite{xi99dependent}, an
extension of ML with dependent types. \DML{} is also built
parametrically on a constraint solver. Types are refinement types;
they do not affect the operational semantics and erasing the
constraints yields a legal ML program.

At a conceptual level the intuitions behind the development of \DML{}
and constrained types are similar. Both are intended for practical
programming by mainstream programmers, both introduce a strict
separation between compile-time and run-time processing, are
parametric on a constraint solver, and deal with mutually recursive
data-structures, mutable state, and higher-order functions (encoded as
objects in the case of constrained types). Both support existential
types.

The most obvious distinction between the two lies in the target
domain: \DML{} is designed for functional programming, specifically
ML, whereas constrained types are designed for imperative, concurrent
OO languages. Hence technically our development of constrained types
takes the route of an extension to \FJ. But there are several other
crucial differences as well.

\lstnewenvironment{displayml}
  {\lstset{language=ML,basicstyle=\tt,tabsize=4,columns=fullflexible,captionpos=b,xleftmargin=1em,xrightmargin=1em,keywordstyle=,keepspaces}}
  {}

First, \DML{} achieves its separation by not permitting program
variables to be used in types. Instead, a parallel set of (universally
or existentially quantified) ``index'' variables are
introduced. For instance the typing of the \xcd{append} operation on
lists is provided by:
\begin{displayml}
fun('a)
  append(nil, ys) = ys
| append(cons(x, xs), ys)
    = cons(x, append(xs,ys))
where append <| {m:nat}{n:nat} 
    'a list(m) * 'a list(n) -> 'a list(m+n)  
\end{displayml}
\noindent in contrast with the direct embedded expression with constrained types:
\begin{xten}
class List(int(:self >= 0) n) {
  Object item;
  List(n-1) tail;
  List(n+a.n) append(final List a) { 
    return n==0
      ? a : new List(item, tail.app(a));
  }
  ...
}
\end{xten}

Second, \DML{} permits only variables of basic index sorts known to
the constraint solver (e.g., \xcd{bool}, \xcd{int}, \xcd{nat}) to
occur in types. In contrast, constrained types permit program
variables at any type to occur in constrained types. As with \DML,
only operations specified by the constraint system are permitted in
types. However, these operations always include field selection and
equality on object references.  (As we have seen permitting arbitrary
type/property graphs may lead to undecidability.) Note that \DML{}
style constraints are easily encoded in constrained types.

% A reviewer says:
% The third criticism of DML is technically correct but highly
% misleading.  Instead of casts, DML allows "if tests" or case
% analysis as dynamic tests that then yield static information
% about the type in the appropriate branch of the if or case.
% Either omit this criticism or describe how DML does the same
% thing--or if DML's system is weaker in some way, give a
% particular example to justify that.

% Third, \DML{} does not permit any runtime checking of constraints
% (dynamic casts).



\eat{\input{constraint-solver.tex}}

%\newpage~\newpage

\section{Semantics}
\label{sec:semantics}
\newcommand\comma{,~}
\newcommand\tj[2]{{#1} \vdash_{\cal T}{#2}}
\newcommand\cj[2]{{#1} \vdash_{\cal C}{#2}}
\newcommand\wj[2]{{#1} \vdash_{\cal W}{#2}}
\newcommand\cdecl{{\tt class}~{\tt C}[\tbar{X}]\{{\tt c}\}(\tbar{f}\ty\tbar{F})~{\tt extends}~{\tt D}[\tbar{E}]~\{~\tbar{M}~\}}
\newcommand\msign[5]{{\tt m}[\tbar{#1}](\tbar{#2}\ty\tbar{#3})\{{\tt #4}\}\ty{\tt #5}}
\newcommand\minst[6]{\msign{#1}{#2}{#3}{#4}{#5}={\tt #6}}
\newcommand\mdecl[6]{{\tt def}~\minst{#1}{#2}{#3}{#4}{#5}{#6}}
\newcommand{\vdashQ}{\vdash_{\cal X}}
\newcommand{\vdashS}{\vdash_{\cal X}}
\newcommand{\Dom}{{\sf Dom}}
\newcommand{\Img}{{\sf Rng}}

In the previous section we presented \Xten. The subtle issues encountered when designing and implementing the \Xten type system exposed the need for a formal framework in which to explore the design space, and to reason about fundamental issues such as soundness, completeness, and decidability. The resulting framework consists of a family of formal languages---the \FXG family---that share a common base language, operational semantics, and type system.

Our framework models many, but not all, relevant features of \Xten.
Following \FJ, our framework does not account for mutations. Proving the soundness of the type system for an imperative calculus is beyond the scope of this paper.
Other features of \Xten not modeled in the framework include interfaces, constructors, and method overloading. Class invariants are simplified and may constrain only type parameters, but not properties of the class.  We believe none of these restrictions affect the type system in a fundamental way. Finally, our subtyping relation is invariant in the type parameters. We leave modeling the definition-site variance annotations of \Xten for future work.

We first describe the base \FXG language and its type system and prove it sound. While its \FGJ-like syntax enables the declaration of generic classes and methods, it does not allow constraints on type parameters. We then demonstrate other languages in the family with features such as \FGJ-like generic types, or structural subtyping constraints.

\eat{
\subsection{The \FXG family}

REVISE

We now describe the semantics of languages in the \FXG family. We start with the \FXGL{X,\bullet} language that is the common core of all languages  in the family. Each language \FXGL{X',T} of the family is rigorously derived from the core language by:
\begin{itemize}
\item replacing the object constraint system $\cal X$ in use in the core type system with a richer object constraint denoted $\cal X'$;
\item extending the core type systems with a set of inference rules denoted $\cal T$.
\end{itemize}

We first specify the grammar, static, and dynamic semantics of \FXGL{X,\bullet}. While its \FGJ-like syntax enables the declaration of generic classes and methods, it does not allow constraints on type parameters. Types in \FXGL{X,\bullet} are therefore are isomorphic to those of \CFJ~\cite{constrained-types}. We then demonstrate other languages in the family with features such as dependent types, \FGJ-like generic types, or structural subtyping constraints. 

We state and prove the soundness of the \FXGL{X,\bullet}. We describe the methodology by which extensions of the languages such as the one we presented can be formalized. We identify the requirements an extension must satisfy to ensure the soundness of its type system (i.e., its proof) can be derived from the soundness of the core type system.
}

\subsection{The \FXG language}

The grammar for \FXG is shown in Figure~\ref{fig:fxg-grammar}. The syntax is essentially that of \FGJ. We use $\bar{x}$ to denote a list $x_1, \dots, x_n$, and use $\bullet$ to denote the empty list.

A program {\tt P} is a set of class declarations \tbar{L}. Class names {\tt C} range over the declared classes in {\tt P} and {\tt Object}. A class declaration has type parameters \tbar{X}, value properties (i.e., immutable fields) \tbar{f}, a supertype {\tt D}[\tbar{E}], methods \tbar{M}, and a guard {\tt c}---a constraint on its type parameters.

Each class has a default constructor. If class $\tt C[\tbar{X}]$ has field $\tt f$ of type $\tt F$ and extends class $\tt D$ with field $\tt g$ of type $\tt G$ then $\tt C$'s constructor has type parameters $\tbar{X}$, a formal of type $\tt G$, and a formal of type $\tt F$. $\Object$ has a nullary constructor.

Methods are introduced with the {\tt def} keyword. A method has both type parameters {\tbar{X}} and value parameters {\tbar{x}}. The method guard {\tt c} is to be thought of as an additional condition that must be satisfied by the receiver and the actual type and value arguments of the method call. We do not consider method overloading: we assume each class declares at most one method with name {\tt m}.

Square brackets may be dropped in the absence of type parameters, as well as \true{} constraints and the surrounding curly braces.

The body of a method is an expression {\tt e}. It is built from the value parameters of the method (including the receiver {\tt this}), field access expressions, constructor calls, method invocations, and casts (written {\tt e}~\as~{\tt G}).

The set of types includes class types {\tt C}[\tbar{A}], type parameters {\tt X}, dependent types ($\tt T\{c\}$), and is closed under existential quantification ($\exty{\tt x}{\tt T}{\tt U}$). Existential types arise in typing judgments but are not permitted in programs. Neither casts nor methods invocations are permitted in constraints. A value {\tt v} is of type {\tt C}[\tbar{A}] if it is an instance of class {\tt C}[\tbar{A}]; it is of type $\tt T\{c\}$ if it is of type {\tt T} and it satisfies the constraint $\tt c[v/self]$; it is of type $\exty{\tt x}{\tt T}{\tt U}$ if there exists some value {\tt w} of type {\tt T} such that {\tt v} is of type $\tt U[w/x]$. In other words, the set of values of $\exty{\tt x}{\tt T}{\tt U}$ is the union of the sets of values $\tt U[w/x]$ for all {\tt x} of type {\tt T}.

Following \FJ{}, we denote values by means of nested constructor calls, e.g.,
\texttt{new} \texttt{Pair[C](new} \texttt{C(),} \texttt{new} \texttt{C())}.

\paragraph{Dynamic semantics.}

The operational semantics, shown in Figures~\ref{fig:members} and~\ref{fig:sos}, is described as a reduction relation on expressions. It enforces a strict left-to-right call-by-value evaluation order.

The use of the subtyping relation to check that the cast is satisfied is the only reason the semantics has to keep track of type parameters. The existence of fields and methods of value $\new~{\tt C}[\tbar{A}](\tbar{v})$ do not depend on the types \tbar{A}. Moreover, even if the applicability of a method depend on the guard hence the type parameters, this is irrelevant to the operational semantics because we will establish that there is no need for a run-time check of the guards in well-typed programs. As a consequence, {\sc R-Invk} does not check method guards. 


{\sc L-Field-I} prevents classes from overriding inherited fields. {\sc L-Method-I} makes sure method lookup goes bottom up, but does not enforce any overriding restrictions. These are not relevant to the operational semantics, hence are introduced later.


{\sc R-Cast} checks that the static type of $\new~{\tt C}[\tbar{A}](\tbar{v})$ is a subtype of $\tt G$.


\paragraph{Conventions.}
In the following, the context $\Gamma$ is always a (finite, possibly empty) sequence of variable declarations $\tt x\ty T$, type parameter declarations $\tt X\ty*$, and constraints $\tt c$ satisfying:
\begin{enumerate}
  \item for any declaration $\phi=\tt x\ty T$ or constraint $\phi=\tt c$ in the sequence, all free variables\footnote{Variables are bound by existential quantifiers. In addition, the type constructor ${\tt c}\mapsto{\tt T}\{{\tt c}\}$ binds the special variable {\self} in {\tt c}.} occurring in $\tt T$ or $\tt c$ are declared in the sequence to the left of $\phi$.

  \item a variable or type parameter is declared at most once in $\Gamma$.
\end{enumerate}
For any formulas $\phi_1$ and $\phi_2$, the judgment $\Gamma \vdash \phi_1\comma\phi_2$ is shorthand for the judgments $\Gamma \vdash \phi_1$ and $\Gamma \vdash \phi_2$. 

An assumption ``{\tt x} fresh'' in an inference rule means that {\tt x} does not occur in the premises to the left of this assumption.

A premise $\theta=\tbar{\alpha}/\tbar{\beta}$ requires $\tbar{\alpha}$ and $\tbar{\beta}$ to have the same length $n$ and defines the substitution of $\beta_i$ by $\alpha_i$ for $1\leq i\leq n$. If $\phi$ is a constraint, a type, an expression, etc., $\phi\theta$ denotes the result of the application of the substitution $\theta$ to the free variables of $\phi$.

\paragraph{Type system.} Typing checking a program {\tt P} involves a series of judgments:
\begin{enumerate}
	\item Constraints:\\
	  $\Gamma\vdash {\tt c}$ \hfill $\Gamma$ entails constraint {\tt c}
	\item Well-formedness:\\
	  $\tj{\Gamma}{\tt t}$ \hfill constraint term {\tt t} is well-formed in $\Gamma$\\
	  $\cj{\Gamma}{\tt c}$ \hfill constraint {\tt c} is well-formed in $\Gamma$\\
	  $\wj{\Gamma}{\tt T}$ \hfill  type {\tt T} is well-formed in $\Gamma$
	\item Lookup:\\
	  $\Gamma\vdash {\tt x}~\has~{\tt I}$ \hfill variable {\tt x} has member {\tt I} in $\Gamma$\\
	  $~$ \hfill where ${\tt I}::= {\tt f}\ty{\tt F} \alt \msign{B}{x}{G}{c}{H}$
	\item Subtyping:\\
	  $\Gamma,{\tt x}\ty{\tt S}\vdash{\tt x}\subtype{\tt T}$ \\ $~$ \hfill type ${\tt S}\{\self=={\tt x}\}$ is a 	subtype of type {\tt T} in $\Gamma,{\tt x}\ty{\tt S}$
	\item Typing:\\
	  $\Gamma\vdash {\tt e}\ty{\tt T}$ \hfill expression {\tt e} has type {\tt T} in $\Gamma$\\
	  $\Gamma\vdash {\tt I}\ll {\tt J}$ \hfill member {\tt I} overrides member {\tt J} in $\Gamma$\\
	  $\vdash {\tt def}~{\tt m}[\tbar{Y}](\tbar{x}\ty \tbar{G})\{{\tt c}\}\ty {\tt H}={\tt e}~{\rm OK~in}~{\tt C}[\tbar{X}]$ \\ $~$ \hfill method {\tt m} in class ${\tt C}[\tbar{X}]$ is well-typed\\
	  $\vdash \cdecl~{\rm OK}$ \\ $~$ \hfill class ${\tt C}[\tbar{X}]$ is well-typed

\end{enumerate}

Notice these judgments depend on the particular program {\tt P}. For readability, we do not make this dependency explicit as we will never consider more than one program at a time.

A program is well-typed iff all its classes are. We now describe in more detail each of these judgments, in turn. 


\paragraph{\normalfont\bf\em 1. Constraints.}
{}\FXG{} is parametrized by an {\em object constraint system} $\cal
X$.  Such a constraint system is required to have terms {\tt t} of the
form ${\tt C}(\tbar{f}=\tbar{t})$ and {\tt t.f}, and an equality
predicate on such terms. (It may have other predicates and terms whose interpretation is left unconstrained, see Section~\ref{sec:Q}.)
The entailment relation for $\cal X$ must respect the interpretation of
(a)~${\tt C}(\tbar{f}=\tbar{t})$ as a finite tree with root
labeled with {\tt C}, $i$th branch labeled with $\tt f_i$ and leading to
$\tt t_i$, and (b)~{\tt t.f} as selection of the child labeled
$\tt f$ for the tree $\tt t$.\footnote{A complete axiomatization of the algebra of finite trees is provided in \cite{maher-tree}.}

In order to expose the current typing context to $\cal X$, we define
the {\em constraint projection} $\sigma(\Gamma)$ that, in essence,
strips out all type information from the $\Gamma$. It also uses
information in the program {\tt P} to translate constraints in the source
program into constraints understood by the constraint system.
 
\begin{quote}
\noindent $\sigma(\epsilon)={\tt true}$\\
$\sigma(\new~{\tt C}[\tbar{A}](\tbar{t}))={\tt C}(\tbar{f}=\tbar{t})$ 
 where $\fields({\tt C}[\tbar{A}])=\tbar{f}:\tbar{G}$\\
$\sigma({\tt t.f}) = {\tt t.f}$\\
$\sigma({\tt f}(\tbar{t}))={\tt f}(\sigma(\tbar{t}))$  for all other functions $f$\\
$\sigma({\tt t==s})=\sigma({\tt t})==\sigma(\tt {s})$\\
$\sigma({\tt p}(\tbar{t}))={\tt p}(\sigma(\tbar{t}))$  for all other predicates $p$\\
$\sigma({\tt c},\Gamma) = \sigma(\tt c), \sigma(\Gamma)$\\
$\sigma({\tt X}\ty*, \Gamma)=\sigma(\Gamma)$\\
$\sigma({\tt x}\ty{\tt C}[\tbar{A}], \Gamma)=\sigma(\Gamma)$\\
$\sigma({\tt x}\ty{\tt X}, \Gamma)=\sigma(\Gamma)$\\
$\sigma({\tt x}\ty{\tt T\{c\}}, \Gamma)={\tt c}\theta,\sigma({\tt x}\ty{\tt T},\Gamma)$  where $\theta={\tt x}/\self$\\
$\sigma({\tt x}\ty\exty{{\tt y}}{{\tt T}}{{\tt U}}, \Gamma)=\sigma({\tt z}\ty{\tt T}, {\tt x}\ty{\tt U}\theta,\Gamma)$  where $\theta={\tt z}/{\tt y}$
\end{quote}
%
In the last rule, we assume that alpha-equivalence is used to choose a variable {\tt z} that does not occur in the context under construction.

We specify that $\Gamma\vdash{\tt c}$ if the constraint projection of $\Gamma$ entails that of $\tt c$ in the input constraint system. %In \FXG, there is no other way to prove entailment.

We say that a context $\Gamma$ is {\em consistent} if it is not the case that $\Gamma\vdash\false$.
In all inference rules presented below, we make the implicit assumption that the context $\Gamma$ of every premise is consistent; if one is inconsistent, the rule cannot be used. In the sequel, we will permit type system extensions to mark contexts as inconsistent, e.g., {\tt X} extends class {\tt C}, {\tt X} extends class {\tt D} entails \false{} if {\tt C} and {\tt D} are not related by the subclassing relation.


\paragraph{2. Well-formedness.} A constraint term, constraint, or type $\alpha$ is well-formed in context $\Gamma$ iff its free variables are declared in $\Gamma$ and all the type parameters of all the generic classes in $\alpha$ satisfy the guards of these classes. Note that this depends on $\cal X$ but not on any typing judgments. The rules of well-formedness of terms and constraints are straightforward and are omitted. The rules for types are specified in Figure~\ref{fig:well}. 

We say a context $\Gamma$ is well-formed if each $\alpha$ in $\Gamma$ is well-formed w.r.t.\ the sequence of $\Gamma$ to its left, that a judgment is well-formed iff its context is well-formed and the consequent is well-formed w.r.t.\ the context.

In all inference rules presented below (except for {\sc OK-Method} and {\sc OK-Class}), we make the implicit assumption that every lookup, subtyping, or typing judgment is well-formed. If it is not then the rule cannot be used.

By design, if the program {\tt P} is well-typed then all the constraints and types in {\tt P} are well-formed in their respective contexts.

\begin{figure*}
\centering
\begin{tabular}{r@{\quad}rcl}
  (Program) & {\tt P} &{::=}& $\tbar{L}$ \\
  (Class declaration) & {\tt L} &{::=}& $ \tt class~C[\tbar{X}]\{c\}(\tbar{f}\ty\tbar{G})~extends~D[\tbar{G}]~\{~\tbar{M}~\}$ \\
  (Method declaration)& {\tt M} &{::=}& $\tt \mdecl{X}{x}{G}{c}{G}{e};$ \\
  (Expression)& {\tt a}, {\tt b}, {\tt e} &{::=}& $\tt x$ \alt $\tt e.f$ \alt $\tt\new~C[\tbar{G}](\tbar{e})$ \alt $\tt e.m[\tbar{G}](\tbar{e})$ \alt $\tt e~\as~G$ \\
  (Constraint term) & {\tt t}, {\tt u} &{::=}& $\tt x$ \alt $\tt t.f$ \alt $\tt\new~C[\tbar{G}](\tbar{t})$ \\
  (Constraint) & {\tt c}, {\tt d} &{::=}& $\true$ \alt $\false$ \alt $\tt c,c$ \alt $\tt t==t$ \\
  (Generic type)& {\tt A}, {\tt B}, {\tt E}, {\tt F}, {\tt G}, {\tt H} &{::=}& $\tt C[\tbar{G}]$ \alt $\tt G\{c\}$ \alt $\tt X$ \\
  (Type)& {\tt S}, {\tt T}, {\tt U} &{::=}& $\tt C[\tbar{G}]$ \alt $\tt T\{c\}$ \alt $\tt X$ \alt $\tt T\{c\}$ \alt $\tt \exists x\ty T.~T$ \\
  (Value)& {\tt v}, {\tt w} &{::=}& $\tt\new~C[\tbar{G}](\tbar{v})$ where \tbar{G} contains no type variables \\
\end{tabular}\smallskip

{\tt C}, {\tt D} range over class names, {\tt f}, {\tt g} over field names, {\tt m} over method names, {\tt x}, {\tt y} over variable names, {\tt X}, {\tt Y} over type variables.
\caption{\FXG productions.}
\label{fig:fxg-grammar}
\end{figure*}


\begin{figure*}
\vspace{-\bigskipamount}
\begin{minipage}{\textwidth}
\quad\typicallabel{XXXXXX}
\infax[L-Fields-B]
  {\fields({\tt Object})=\bullet}

\infrule[L-Fields-I]
  {\cdecl\in{\tt P} \andalso
    \theta=\tbar{A}/\tbar{X} \andalso
    \fields({\tt D}[\tbar{E}\theta])=\tbar{g}\ty\tbar{G} \andalso
    \tbar{f}\cap\tbar{g}=\emptyset}
  {\fields({\tt C}[\tbar{A}])=\tbar{g},\tbar{f}\ty\tbar{G},\tbar{F}\theta}

\infrule[L-Method-B]
  {\cdecl\in{\tt P} \andalso
    \mdecl{Y}{x}{G}{d}{H}{e}\in\tbar{M} \andalso
    \theta=\tbar{A},\tbar{B}/\tbar{X},\tbar{Y}}
  {\methods({\tt C}[\tbar{A}])\ni\minst{B}{x}{G\theta}{d\theta}{H\theta}{e\theta}}

\infrule[L-Method-I]
  {\cdecl\in{\tt P} \andalso
    \theta=\tbar{A}/\tbar{X}
    \andalso
    \methods({\tt D}[\tbar{E}\theta])\ni\minst{B}{x}{G}{d}{H}{e} \andalso
    {\tt m}\not\in\tbar{M}}
  {\methods({\tt C}[\tbar{A}])\ni\minst{B}{x}{G}{d}{H}{e}}
\end{minipage}%
\caption{\FXG fields and methods.}
\label{fig:members}
\end{figure*}


\begin{figure*}
\vspace{-\bigskipamount}
\begin{minipage}{.33\textwidth}
\quad\typicallabel{XXXXXX}
\infrule[\RField]
	{\fields({\tt C}[\tbar{A}])=\tbar{f}\ty\tbar{F}}
	{\new~{\tt C}[\tbar{A}](\tbar{v}).{\tt f}_i \derives {\tt v}_i}

\infrule[\RCField]
	{{\tt e}\derives {\tt e}'}
	{{\tt e}.{\tt f}\derives {\tt e}'.{\tt f}}

\infrule[\RCInvkRecv]
	{{\tt e}\derives {\tt e}'}
	{{\tt e}.{\tt m}(\tbar{a})\derives {\tt e}'.{\tt m}(\tbar{a})}

\infrule[\RCCast]
	{{\tt e}\derives {\tt e}'}
	{{\tt e}~\as~{\tt G}\derives {\tt e}'~\as~{\tt G}}
\end{minipage}%
\begin{minipage}{.67\textwidth}
\quad\typicallabel{XXXXXX}
\infrule[\RCNewArg]
	{{\tt e}_i\derives {\tt e}'_i}
	{\new~{\tt C}[\tbar{A}]({\tt v}_1,\ldots,{\tt v}_{i-1},{\tt e}_i,\ldots,{\tt e}_n)\derives\new~{\tt C}[\tbar{A}]({\tt v}_1,\ldots,{\tt v}_{i-1},{\tt e}'_i,\ldots,{\tt e}_n)}

\infrule[\RInvk]
	{\methods({\tt C}[\tbar{A}])\ni\minst{B}{x}{G}{d}{H}{e} \andalso
	\theta=\new~{\tt C}[\tbar{A}](\tbar{v}),\tbar{w}/\this,\tbar{x}}
	{\new~{\tt C}[\tbar{A}](\tbar{v}).{\tt m}[\tbar{B}](\tbar{w})\derives {\tt e}\theta}

\infrule[\RCInvkArg]
	{{\tt a}_i\derives {\tt a}'_i}
	{{\tt v}.{\tt m}({\tt w}_1,\ldots,{\tt w}_{i-1},{\tt a}_i,\ldots,{\tt a}_n)\derives {\tt v}.{\tt m}({\tt w}_1,\ldots,{\tt w}_{i-1},{\tt a}'_i,\ldots,{\tt a}_n)}

\infrule[\RCast]
	{\Gamma\vdash\tbar{v}\ty\tbar{V} \andalso
	  {\tt x}\ty\exty{\tbar{y}}{\ty\tbar{V}}{{\tt C}[\tbar{A}]\{\self==\new~{\tt C}[\tbar{A}](\tbar{y})\}\vdash{\tt x}}\subtype {\tt G}}
	{\new~{\tt C}[\tbar{A}](\tbar{v})~\as~{\tt G}\derives\new~{\tt C}[\tbar{A}](\tbar{v})}
\end{minipage}
\caption{\FXG operational semantics. \tbar{A} and \tbar{B} are lists of ground types (no type variables, no existentials).}
\label{fig:sos}
\end{figure*}


\begin{figure*}
\vspace{-\bigskipamount}
\begin{minipage}{.24\textwidth}
\quad\typicallabel{XXXX}
\infax[W-Object]
  {\wj{}{\Object}}

\infax[W-Type]
	{\wj{{\tt X}\ty*}{\tt X}}
\end{minipage}%
\begin{minipage}{.38\textwidth}
\quad\typicallabel{XXXX}
\infrule[W-Dep]
  {\wj{\Gamma}{\tt T} \andalso \cj{\Gamma,\self\ty{\tt T}}{\tt c}}
	{\wj{\Gamma}{{\tt T}\{{\tt c}\}}}
\end{minipage}%
\begin{minipage}{.38\textwidth}
\quad\typicallabel{XXXX}
\infrule[W-Exists]
  {\wj{\Gamma}{\tt T} \andalso \wj{\Gamma,{\tt x}\ty{\tt T}}{\tt U}}
	{\wj{\Gamma}{\exty{\tt x}{\tt T}{\tt U}}}
\end{minipage}%

\begin{minipage}{\textwidth}
\quad\typicallabel{XXXXXX}
\infrule[W-Class]
  {\cdecl\in{\tt P} \andalso
    \wj{\Gamma}{\tbar{A}} \andalso
    \cj{\tbar{X}\ty*}{{\tt c}} \andalso 
    \theta=\tbar{A}/\tbar{X} \andalso
    \sigma(\Gamma)\vdashX{\tt c}\theta}
  {\wj{\Gamma}{\tt C}[\tbar{A}]}
\end{minipage}%
\caption{\FXG well-formedness.}
\label{fig:well}
\end{figure*}





\eat{
\begin{figure*}
\vspace{-\bigskipamount}
\begin{minipage}{.24\textwidth}
\quad\typicallabel{XXXX}
\infax[W-Var]
  {\tj{{\tt x}\ty{\tt T}}{\tt x}}

\infax[W-True]
  {\cj{}{\true}}

\infax[W-False]
  {\cj{}{\false}}

\infax[W-Object]
  {\wj{}{\Object}}

\infax[W-Type]
	{\wj{{\tt X}\ty*}{\tt X}}
\end{minipage}%
\begin{minipage}{.38\textwidth}
\quad\typicallabel{XXXX}
\infrule[W-Field]
	{\tj{\Gamma}{\tt t}}
	{\tj{\Gamma}{{\tt t}.{\tt f}}}

\infrule[W-And]
	{\cj{\Gamma}{\tt c} \andalso \cj{\Gamma}{\tt d}}
	{\cj{\Gamma}{{\tt c},{\tt d}}}

\infrule[W-Dep]
  {\wj{\Gamma}{\tt T} \andalso \cj{\Gamma,\self\ty{\tt T}}{\tt c}}
	{\wj{\Gamma}{{\tt T}\{{\tt c}\}}}
\end{minipage}%
\begin{minipage}{.38\textwidth}
\quad\typicallabel{XXXX}
\infrule[W-New]
	{\wj{\Gamma}{\tt C}[\tbar{G}] \andalso \tj{\Gamma}{\tbar{t}}}
	{\wj{\Gamma}{\new~{\tt C}[\tbar{G}](\tbar{t})}}

\infrule[W-Eq]
	{\tj{\Gamma}{\tt t} \andalso \tj{\Gamma}{\tt u}}
	{\cj{\Gamma}{\tt t}=={\tt u}}

\infrule[W-Exists]
  {\wj{\Gamma}{\tt T} \andalso \wj{\Gamma,{\tt x}\ty{\tt T}}{\tt U}}
	{\wj{\Gamma}{\exty{\tt x}{\tt T}{\tt U}}}
\end{minipage}%

\begin{minipage}{\textwidth}
\quad\typicallabel{XXXXXX}
\infrule[W-Class]
  {\cdecl\in{\tt P} \andalso
    \wj{\Gamma}{\tbar{A}} \andalso
    \cj{\tbar{X}\ty*}{{\tt c}} \andalso 
    \theta=\tbar{A}/\tbar{X} \andalso
    \sigma(\Gamma)\vdashX{\tt c}\theta}
  {\wj{\Gamma}{\tt C}[\tbar{A}]}
\end{minipage}%
\caption{\FXG well-formedness.}
\label{fig:well}
\end{figure*}
}

\begin{figure*}
\vspace{-\bigskipamount}
\begin{minipage}{.65\textwidth}
\quad\typicallabel{XXXXXX}
\infrule[H-Field]
  {\fields({\tt C}[\tbar{A}])=\tbar{f}\ty\tbar{F} \andalso \theta={\tt x}/\this}
  {\Gamma,{\tt x}\ty{\tt C}[\tbar{A}]\vdash{\tt x}~\has~{\tt f}_i\ty{\tt F}_i\theta}

\infrule[H-Method]
  {\methods({\tt C}[\tbar{A}])\ni\minst{B}{y}{G}{d}{H}{e} \andalso \theta={\tt x},\tbar{z}/\this,\tbar{y}}
  {\Gamma,{\tt x}\ty{\tt C}[\tbar{A}]\vdash{\tt x}~\has~\msign{B}{z}{G\theta}{d\theta}{H\theta}}
\end{minipage}%
\begin{minipage}{.35\textwidth}
\quad\typicallabel{XXXXXX}
\infrule[H-Dep]
  {\Gamma,{\tt x}\ty{\tt T},{\tt c}\vdash {\tt x}~\has~{\tt I}}
  {\Gamma,{\tt x}\ty{\tt T}\{{\tt c}\}\vdash {\tt x}~\has~{\tt I}}

\infrule[H-Exists]
  {\Gamma,{\tt y}\ty{\tt T},{\tt x}\ty{\tt U}\vdash {\tt x}~\has~{\tt I}}
  {\Gamma,{\tt x}\ty\exty{\tt y}{\tt T}{\tt U}\vdash {\tt x}~\has~{\tt I}}
\end{minipage}%
\caption{\FXG member lookup. {\tt I} ranges over members (fields and methods).}
\label{fig:lookup}
\end{figure*}


\begin{figure*}
\vspace{-\bigskipamount}
\begin{minipage}{.6\textwidth}
\quad\typicallabel{XXXXXX}
\infrule[S-Class]
  {\cdecl\in {\tt P} \theta=\tbar{A}/\tbar{X}}
  {\Gamma,{\tt x}\ty{\tt C}[\tbar{A}]\vdash{\tt x}\subtype {\tt D}[\tbar{E}\theta]}      

\infax[S-Const-L]
	{\Gamma,{\tt x}\ty{\tt T}\{{\tt c}\}\vdash{\tt x}\subtype {\tt T}}

\infrule[S-Exists-L]
  {\Gamma,{\tt y}\ty{\tt U},{\tt x}\ty{\tt S}\vdash {\tt x}\subtype {\tt T} \andalso {\tt y}~\rm not~free~in~{\tt T}}
  {\Gamma,{\tt x}\ty\exty{\tt y}{\tt U}{\tt S}\vdash{\tt x}\subtype {\tt T}}
\end{minipage}%
\begin{minipage}{.4\textwidth}
\quad\typicallabel{XXXXXX}
\infrule[S-Trans]
	{\Gamma\vdash {\tt x}\subtype {\tt T} \andalso \Gamma,{\tt y}\ty{\tt T}\vdash {\tt y}\subtype {\tt U}}
	{\Gamma\vdash {\tt x}\subtype {\tt U}}

\infrule[S-Const-R]
	{\Gamma\vdash {\tt c}[{\tt x}/\self]\comma{\tt x}\subtype {\tt T}}
	{\Gamma\vdash{\tt x}\subtype {\tt T}\{{\tt c}\}}

\infrule[S-Exists-R]
  {\Gamma\vdash {\tt t}\ty{\tt U}\comma{\tt y}\subtype {\tt T}[{\tt t}/{\tt x}]}
  {\Gamma\vdash {\tt y}\subtype\exty{\tt x}{\tt U}{\tt T}}
\end{minipage}%

\caption{\FXG subtyping rules.}\label{fig:subtyping}
\end{figure*}


\begin{figure*}
\vspace{-\bigskipamount}
\begin{minipage}{.3\textwidth}
\quad\typicallabel{XXXX}
\infax[T-Var]
  {\Gamma,{\tt x}\ty{\tt T}\vdash {\tt x}\ty{\tt T}\{\self=={\tt x}\}}
\end{minipage}%
\begin{minipage}{.25\textwidth}
\quad\typicallabel{XXXX}
\infrule[T-Cast]
	{\Gamma\vdash {\tt e}\ty{\tt T}}
	{\Gamma\vdash {\tt e}~\as~{\tt G}\ty{\tt G}}
\end{minipage}%
\begin{minipage}{.45\textwidth}
\quad\typicallabel{XXXX}
\infrule[T-Field]
	{\Gamma\vdash {\tt e}\ty{\tt T} \andalso
	  {\tt x}~{\rm fresh} \andalso
	  \Gamma,{\tt x}\ty{\tt T}\vdash {\tt x}~\has~{\tt f}\ty{\tt F}}
	{\Gamma\vdash {\tt e}.{\tt f}\ty\exty{\tt x}{\tt T}{\tt F}\{\self=={\tt x}.{\tt f}\}}
\end{minipage}

\begin{minipage}{\textwidth}
\quad\typicallabel{XXXXXX}
\infrule[T-New]
	{\Gamma\vdash\tbar{e}\ty\tbar{T} \andalso
	  \fields({\tt C}[\tbar{A}])=\tbar{f}\ty\tbar{F} \andalso 
	  {\tt y},\tbar{x}~{\rm fresh} \andalso 
	  \theta={\tt y}/\this \andalso
	  \Gamma,{\tt y}\ty{\tt C}[\tbar{A}],\tbar{x}\ty\tbar{T},{\tt y}.\tbar{f}==\tbar{x}\vdash\tbar{x}\subtype\tbar{F}\theta}
	{\Gamma\vdash\new~{\tt C}[\tbar{A}](\tbar{e})\ty\exty{\tbar{x}}{\tbar{T}}{\tt C}[\tbar{A}]\{\self==\new~{\tt C}[\tbar{A}](\tbar{x})\}}
        
\infrule[T-Invk]
	{\Gamma\vdash {\tt e}\ty{\tt T}\comma\tbar{a}\ty\tbar{U} \andalso
	  {\tt x},\tbar{y}~{\rm fresh} \andalso
	  \Gamma,{\tt x}\ty{\tt T},\tbar{y}\ty{\tt U}\vdash {\tt x}~\has~\msign{A}{y}{G}{c}{H}\comma{\tt c}\comma\tbar{y}\subtype\tbar{G}}
	{\Gamma\vdash {\tt e}.{\tt m}[\tbar{A}](\tbar{a})\ty\extyty{\tt x}{\tt T}{\tbar{y}}{\tbar{U}}{\tt H}}

\eat{
\infrule[OK-Method]
  {\cdecl \andalso
    {\tt d}={\tt k},{\tt l} \andalso
    \wj{\tbar{X}\ty*,{\tt c},\tbar{Y}\ty*}{{\tt k}} \\
    \wj{\tbar{X}\ty*,{\tt c},\this\ty{\tt C}[\tbar{X}],\tbar{Y}\ty*,{\tt k},\tbar{x}\ty\tbar{G}}{\tbar{G},{\tt l}} \andalso
    \wj{\tbar{X}\ty*,{\tt c},\this\ty{\tt C}[\tbar{X}],\tbar{Y}\ty*,\tbar{x}\ty\tbar{G},{\tt d}}{{\tt H}} \\
    \tbar{X}\ty*,{\tt c},\this\ty{\tt C}[\tbar{X}],\tbar{Y}\ty*,\tbar{x}\ty\tbar{G},{\tt d}\vdash {\tt e}\ty{\tt E} \andalso
    {\tt y}~{\rm fresh} \andalso
    \tbar{X}\ty*,{\tt c},\this\ty{\tt C}[\tbar{X}],\tbar{Y}\ty*,\tbar{x}\ty\tbar{G},{\tt d},{\tt y}\ty{\tt E}\vdash {\tt y}\subtype {\tt H}}
  {\vdash\mdecl{Y}{x}{G}{d}{H}{e}~{\rm OK~in}~{\tt C}[\tbar{X}]}
}
\end{minipage}%

\begin{minipage}{.3\textwidth}
\quad\typicallabel{XXXXXX}
\infax[O-Field]
  {\Gamma\vdash {\tt f}\ty{\tt F} \ll  {\tt f}\ty{\tt F}}
\end{minipage}%
\begin{minipage}{.7\textwidth}
\quad\typicallabel{XXXXXX}

\infrule[O-Method]
  {\Gamma,\tbar{x}\ty\tbar{G},{\tt d'}\vdash{\tt d} \andalso 
    {\tt y}~{\rm fresh} \andalso \Gamma,\tbar{x}\ty\tbar{G},{\tt d},{\tt y}\ty{\tt H}\vdash{\tt y}\subtype{\tt H'}}
  {\Gamma\vdash \msign{Y}{x}{G}{d}{H} \ll  \msign{Y}{x}{G}{d'}{H'}}
\end{minipage}%

\begin{minipage}{\textwidth}
\quad\typicallabel{XXXXXX}
\infrule[OK-Method]
  {\cdecl \andalso
    \Gamma=\tbar{X}\ty*,{\tt c},\this\ty{\tt C}[\tbar{X}],\tbar{Y}\ty*\\
    \cj{\Gamma,\tbar{x}\ty\tbar{G}}{{\tt d}} \andalso
    \wj{\Gamma,\tbar{x}\ty\tbar{G},{\tt d}}{\tbar{G},{\tt H}} \andalso
    \Gamma,\tbar{x}\ty\tbar{G},{\tt d}\vdash {\tt e}\ty{\tt E} \andalso
    {\tt y}~{\rm fresh} \andalso
    \Gamma,\tbar{x}\ty\tbar{G},{\tt d},{\tt y}\ty{\tt E}\vdash {\tt y}\subtype {\tt H} \\
    {\rm if}~~\tbar{X}\ty*,{\tt c},\this\ty{\tt D}[\tbar{E}],\tbar{Y}\ty*\vdash\this~\has~\msign{Y}{x}{G'}{d'}{H'} ~~{\rm then}~~
    \Gamma\vdash \msign{Y}{x}{G}{d}{H} \ll  \msign{Y}{x}{G'}{d'}{H'}}
  {\vdash\mdecl{Y}{x}{G}{d}{H}{e}~{\rm OK~in}~{\tt C}[\tbar{X}]}

\eat{
\infrule[OK-Method]
  {\cdecl \andalso
    \tbar{X}\ty*,{\tt c},\this\ty{\tt C}[\tbar{X}],\tbar{Y}\ty*,\tbar{x}\ty\tbar{G},{\tt d},{\tt y}\ty{\tt E}\vdash {\tt y}\subtype {\tt H} \\
    \tbar{X}\ty*,{\tt c},\this\ty{\tt D}[\tbar{E}],\tbar{Y}\ty*\vdash\this~\has~\msign{Y}{x}{G'}{d'}{H'}~\Rightarrow
    \left\{\begin{array}{@{}l@{}}
    \tbar{G}=\tbar{G'}\\
    \tbar{X}\ty*,{\tt c},\this\ty{\tt C}[\tbar{X}],\tbar{Y}\ty*,\tbar{x}\ty\tbar{G},{\tt d'}\vdash{\tt d}\\
    {\tt z}~{\rm fresh} \andalso \tbar{X}\ty*,{\tt c},\this\ty{\tt C}[\tbar{X}],\tbar{Y}\ty*,\tbar{x}\ty\tbar{G},{\tt d},{\tt z}\ty{\tt H}\vdash{\tt z}\subtype{\tt H'}
    \end{array}\right.}
  {\vdash\mdecl{Y}{x}{G}{d}{H}{e}~{\rm OK~in}~{\tt C}[\tbar{X}]}
}

\infrule[OK-Class]
  {\fields({\tt D}[\tbar{E}])=\tbar{g}\ty\tbar{G} \andalso
    \tbar{f}\cap\tbar{g}=\emptyset \andalso
    \cj{\tbar{X}\ty*}{{\tt c}} \andalso 
    \wj{\tbar{X}\ty*,{\tt c}}{{\tt D}[\tbar{E}]} \andalso 
    \wj{\tbar{X}\ty*,{\tt c},\this\ty{\tt C}[\tbar{X}]}{\tbar{F}} \andalso
    \tbar{M}~{\rm OK~in}~{\tt C}[\tbar{X}]}
  {\vdash \cdecl~\rm OK}
\end{minipage}%
\caption{\FXG typing rules.}\label{fig:FX}
\end{figure*}


\begin{figure*}
\begin{minipage}{.30\textwidth}
\centering
\begin{tabular}{r@{\quad}rcl@{}}
  (Generic type)& {\tt G} &{::=}& {\tt R} \\
  (Expression) & {\tt e} &{::=}& ${\tt q}(\tbar{e})$ \\
  (Values) & {\tt v} &{::=}& ${\tt l}$ \\
  (Constraint term) & {\tt t} &{::=}& ${\tt q}(\tbar{t})$ \alt $\tt l$ \\
  (Constraint) & {\tt c} &{::=}& ${\tt p}(\tbar{t})$
\end{tabular}
\end{minipage}%
\begin{minipage}{.26\textwidth}
\vspace{-\bigskipamount}\quad\typicallabel{XXXX}
\infrule[W-Fun]
  {\tj{\Gamma}{\tbar{t}} \andalso {\tt q}\in\cal Q}
  {\tj{\Gamma}{{\tt q}(\tbar{t})}}

\infrule[W-Pred]
  {\tj{\Gamma}{\tbar{t}} \andalso {\tt p}\in\cal P}
  {\cj{\Gamma}{{\tt p}(\tbar{t})}}
\end{minipage}%
\begin{minipage}{.16\textwidth}
\vspace{-\bigskipamount}
\quad\typicallabel{XXXX}
\infrule[W-Lit]
  {{\tt l}\in\cal L}
  {\tj{}{{\tt l}}}

\infrule[W-Prim]
  {{\tt R}\in\cal R}
  {\wj{}{{\tt R}}}
\end{minipage}%
\begin{minipage}{.28\textwidth}
\vspace{-\bigskipamount}
\quad\typicallabel{XX}
\infrule[R-Fun]
	{\vdashQ{\tt q}(\tbar{v})=={\tt l}}
	{{\tt q}(\tbar{v})\derives {\tt l}}
	
\infax[T-Lit]
	{\vdash{{\tt l}\ty\Dom({\tt l})\{\self=={\tt l}\}}}
\end{minipage}%

\begin{minipage}{.55\textwidth}
\quad\typicallabel{XXXX}
\infrule[RC-Fun]
	{{\tt e}_i\derives {\tt e}'_i}
	{{\tt q}({\tt v}_1,\ldots,{\tt v}_{i-1},{\tt e}_i,\ldots,{\tt e}_n)\derives {\tt q}({\tt v}_1,\ldots,{\tt v}_{i-1},{\tt e}'_i,\ldots,{\tt e}_n)}
\end{minipage}%
\begin{minipage}{.45\textwidth}
\quad\typicallabel{XXXX}
\infrule[T-Fun]
	{\andalso \Gamma\vdash\tbar{e}\ty\Dom({\tt q})}
	{\Gamma\vdash{{\tt q}(\tbar{e})\ty\exty{\tbar{x}}{\Dom({\tt q})}{\Img({\tt q})}\{\self=={\tt q}(\tbar{x})\}}}
\end{minipage}%
\caption{\FXG+primitive types.}
\label{fig:FXGQ}
\end{figure*}

\eat{

\begin{figure*}
\begin{minipage}{.4\textwidth}
\centering
\begin{tabular}{r@{\quad}rcl}
  (Constraint) & {\tt c} &{::=}& $\tt X\extends G$ \\
\end{tabular}
\end{minipage}%
\begin{minipage}{.6\textwidth}
\vspace{-\bigskipamount}
\quad\typicallabel{XXXXXX}
\infrule[H-Bound]
	{\cj{\Gamma}{{\tt X}\extends{\tt G}} \andalso
	  \Gamma\vdash{\tt X}\extends{\tt G}
    \andalso
    \Gamma,{\tt x}\ty{\tt G}\vdash {\tt x}~\has~{\tt I}}
	{\Gamma,{\tt x}\ty{\tt X}\vdash {\tt x}~\has~{\tt I}}
\end{minipage}%

\begin{minipage}{.30\textwidth}
\quad\typicallabel{XXXX}
\infrule[W-Bound]
	{\wj{\Gamma}{\tt X} \andalso \wj{\Gamma}{\tt G}}
	{\cj{\Gamma}{{\tt X}\extends{\tt G}}}
\end{minipage}%
\begin{minipage}{.36\textwidth}
\quad\typicallabel{XXXX}
\infrule[X-Bound]
	{{\tt G}\neq{\tt H}}
	{{\tt X}\extends{\tt G}\comma{\tt X}\extends{\tt H}\vdash\false}
\end{minipage}%
\begin{minipage}{.34\textwidth}
\quad\typicallabel{XXXX}
\infrule[S-Bound]
	{\cj{\Gamma}{{\tt X}\extends{\tt G}} \andalso \Gamma\vdash {\tt X}\extends {\tt G}}
	{\Gamma,{\tt x}\ty{\tt X}\vdash {\tt x}\subtype {\tt G}}
\end{minipage}%
\caption{\FXGL{B}.}
\label{fig:FXGB}
\end{figure*}

}

\begin{figure*}
\begin{minipage}{.45\textwidth}
\centering
\begin{tabular}{r@{\quad}rcl}
  (Constraint) & {\tt c} &{::=}& ${\tt X}~\underline{\has}~\msign{Y}{y}{G}{c}{H}$
\end{tabular}

\end{minipage}%
\begin{minipage}{.55\textwidth}
\vspace{-\bigskipamount}
\quad\typicallabel{XXXXXX}
\infrule[H-Struct]
  {\vdash{\tt X}~\underline\has~{\msign{Y}{y}{G}{c}{H}} \andalso
    \theta={\tt x},\tbar{Z},\tbar{z}/\this,\tbar{Y},\tbar{y}}
  {\Gamma,{\tt x}\ty{\tt X}\vdash {\tt x}~\has~\msign{Z}{z}{G\theta}{c\theta}{H\theta}}
\end{minipage}%

\begin{minipage}{\textwidth}
\quad\typicallabel{XXXXXX}
\infrule[W-Struct]
	{\wj{\Gamma}{\tt X} \andalso
	  \cj{\Gamma,\this\ty{\tt X},\tbar{Y}\ty*,\tbar{y}\ty\tbar{G}}{\tt c} \andalso
	  \wj{\Gamma,\this\ty{\tt X},\tbar{Y}\ty*,\tbar{y}\ty\tbar{G},{\tt c}}{\tbar{G},\tbar{H}}}
	{\cj{\Gamma}{{\tt X}~\underline\has~\msign{Y}{y}{G}{c}{H}}}
\end{minipage}%

\begin{minipage}{.4\textwidth}
\quad\typicallabel{XXXXXX}
\infrule[X-Struct]
  {\methods({\tt C}[\tbar{A}])\ni\minst{Y}{y}{G}{c}{H}{e}}
  {\vdashS{\tt C}[\tbar{A}]~\underline\has~\msign{Y}{y}{G}{c}{H}}
\end{minipage}%
\begin{minipage}{.6\textwidth}
\quad\typicallabel{XXXXXX}
\infrule[KO-Struct]
	{\Gamma,{\tt x}\ty{\tt X}\vdash{\tt x}~\has~\msign{Y}{y}{G}{c}{H}\comma{\tt x}~\has~\msign{Y}{y}{G'}{c'}{H'} \\
	  \tbar{G}\neq\tbar{G'} ~\rm{or}~
	  {\tt H}\neq{\tt H'} ~\rm{or}~
	  \Gamma,{\tt x}\ty{\tt X},\tbar{y}\ty\tbar{G},{\tt c}\not\vdash{\tt c'} ~\rm{or}~
	  \Gamma,{\tt x}\ty{\tt X},\tbar{y}\ty\tbar{G'},{\tt c'}\not\vdash{\tt c}}
	{\Gamma\vdash\false}
\end{minipage}%
\caption{\FXG+structural subtyping constraints.}
\label{fig:FXGS}
\end{figure*}


\paragraph{3. Lookup.} Figure~\ref{fig:lookup} specifies the field and method signatures available on each type. In \FXGL{X}, these are exactly those captured by the \fields{} and \methods{} predicates for class types and none for type parameters. Constraints on type parameters will change that.

\paragraph{4. Subtyping.} The subtyping relation is defined in Figure~\ref{fig:subtyping}.
Because we deal with dependent and existential types, we choose a somewhat unconventional notation for the subtyping relation that makes formal developments less verbose. Rather than judgments of the form $\Gamma\vdash{\tt S}\subtype{\tt T}$, we consider judgments of the form $\Gamma,{\tt x}\ty{\tt S}\vdash{\tt x}\subtype{\tt T}$ which permit {\tt T} to depend on {\tt x}, hence make codependent types easier to work with. In fact, one could define classical subtyping as the following:
\vspace{-\medskipamount}
\infrule
	{\wj{\Gamma}{\tt S} \andalso \Gamma,{\tt x}\ty{\tt S}\vdash{\tt x}\subtype {\tt T} \andalso {\tt x}~\rm not~free~in~{\tt T}}
	{\Gamma\vdash{\tt S}\subtype{\tt T}}
\vspace{-\medskipamount}
or derive our relation from subtyping:
\vspace{-\medskipamount}
\infrule
	{\Gamma,{\tt x}\ty{\tt S}\vdash{\tt S}\{\self=={\tt x}\}\subtype{\tt T}}
	{\Gamma,{\tt x}\ty{\tt S}\vdash{\tt x}\subtype {\tt T}}
\vspace{-\medskipamount}

The intent of the subtyping relation is to combine subclassing and constraint entailment: type ${\tt C}[\tbar{A}]\{{\tt c}\}$ is a subtype of ${\tt D}[\tbar{A}]\{{\tt c}\}$ iff {\tt C} is a subclass of {\tt D} and {\tt c} entails {\tt d} in the underlying constraint system. In the formal model, subtyping is invariant in the type parameters, that is, ${\tt C}[\tbar{A}]\{{\tt c}\}$ is a subtype of ${\tt C}[\tbar{B}]\{{\tt c}\}$ only if $\tbar{A}=\tbar{B}$.

Rules {\sc S-Const-L} and {\sc S-Const-R} let us rearrange constraints in types, e.g., ${\tt x}\ty{\tt T}\{{\tt c},{\tt d}\}\vdash {\tt x}\subtype {\tt T}\{{\tt c}\}\{{\tt d}\}$.


\paragraph{5. Typing.} The typing rules are specified in Figure~\ref{fig:FX}.

{\sc T-Var} is as expected, except that it asserts the constraint {\tt
self==x}, which records that any value of this type is known
statically to be equal to {\tt x}. %It makes it possible to type methods
%calls that expect two parameters to be equal for instance.

%This constraint is actually very
%crucial. As we will see later, a key condition of the soundness of 
%this type system is that constraint terms have singleton types, that is, the
%type system retains enough information in the type of a term to permit
%reconstructing the term from its type.

\eat{---as we shall see in the other rules, once we establish that
an expression {\tt e} is of a given type {\tt T}, we ``transfer'' the
type to a freshly chosen variable {\tt z}.  If, in fact, {\tt e} has a
static ``name'' {\tt x} (i.e., {\tt e} is known statically to be
equal to {\tt x}; that is, it has type {\tt T\{self==x\}}), then
{\sc T-Var} lets us assert that {\tt z:T\{self==x\}}, i.e., that {\tt z}
equals {\tt x}.
Thus {\sc T-Var} provides an important base case for
reasoning statically about equality of values in the environment.}

We do away with the three cast rules in \FJ{} in favor of a single
cast rule, requiring only that {\tt e} be of some type {\tt T}. At run time,
{\tt e} will be checked to see if it is actually of type {\tt G} (see
{\sc R-Cast} in Figure~\ref{fig:sos}).

{\sc T-Field} may be understood through ``proxy'' reasoning.
Given the context $\Gamma$, assume the receiver {\tt e} can
be established to be of type {\tt T}. Now, we do not know the run-time
value of {\tt e}, so we shall assume that it is some fixed but unknown
``proxy'' value {\tt x} (of type {\tt T}) that is ``fresh'' in that it
is not known to be related to any known value (i.e., those recorded
in $\Gamma$).  If we can establish that {\tt x} has a field {\tt f} of
type {\tt F} then we can assert that
${\tt e}.{\tt f}$ has type {\tt F} and, further, that it equals ${\tt x}.{\tt f}$
for some {\tt x} of type {\tt T}.
Hence, we can assert that ${\tt e}.{\tt f}$ has type 
$\exty{\tt x}{\tt T}{\tt F}\{\self=={\tt x}.{\tt f}\}$.

{\sc T-New} and {\sc T-Invk} have a similar structure to {\sc T-Field}: we use
proxy reasoning for the arguments of the constructor call or for the receiver and the arguments of the method
call. Both {\sc T-New} and {\sc T-Invk} check that the argument types are subtypes of the types of the formals.
In addition, {\sc T-Invk} requires the types of the receiver and the arguments to satisfy the method guard.

Fields cannot be overridden thanks to the premise $\tbar{f}\cap\tbar{g}=\emptyset$ in rule {\sc OK-Class}. {\sc O-Method} formalizes method overriding. A method $m1$ may be overridden by a method $m2$ with the same name, type parameters (modulo alpha-renaming), value parameters (modulo alpha-renaming), and value parameter types provided method $m2$ has a return type that is a subtype of $m1$'s return type and $m1$'s guard entails $m2$'s. Rule {\sc O-Field} is redundant at this point. It will matter to type system extensions.

{\sc OK-Method} and {\sc OK-Class} ensure that all types and constraints are well-formed, that overriding rules are observed, and that the body of a method has a type that is a subtype of its declared type. Moreover, the guard on a class must entail the well-formedness of its supertype. which includes entailing the guard of the superclass.


\subsection{\FXG+primitive types}\label{sec:Q}

Since the \FXG design is parametric in the constraint language and system we can easily extend it to support, say, arithmetic constraints, or constraints on primitive types.

First, we assume we are given a constraint system $\cal X$ with a vocabulary of primitive types ${\tt R}\in\cal R$, predicates ${\tt p}\in\cal P$, functions ${\tt q}\in\cal Q$, and literals ${\tt l}\in\cal L$ of these primitive types. Second, we extend the productions, operational semantics and type system of \FXG with the productions and inference rules of Figure~\ref{fig:FXGQ}.

We denote $\Dom({\tt l})$ the primitive type of the literal {\tt l}. We assume each function {\tt q}
is a total mapping from ${\sf Dom}({\tt q})$ (an $n$-tuple of primitive types) to $\Img({\tt q})$, that is, if $\vdash \tbar{v}\ty{\sf Dom}({\tt q})$ then there exists a unique literal $\tt l$ such that $\vdashQ {\tt q}(\tbar{v})=={\tt l}$ and moreover $\Dom({\tt l})=\Img({\tt q})$.

For instance, if $\cal X$ defines the type {\tt int}, integer literals, and the usual arithmetic operators, we can declare:

\begin{xten}
class Count(n:int) extends Object {
  def inc():Count{self.n==this.n+1} =
  	new Count(this.n+1);
}
\end{xten}


\subsection{Results}
\label{sec:results}
The following results hold for \FXG+primitive types. Below, the typing context is always assumed to be consistent and well-formed.


\begin{theorem}[Principal types]
$\Gamma\vdash {\tt t}\ty{\tt T}$ and $\Gamma\vdash {\tt x}\ty{\tt U}$ then ${\tt T}={\tt U}$.
\end{theorem}

%The type system of \FXG has principal types in strong sense (for a constrained type system): if $\Gamma\vdash {\tt t}\ty{\tt T}$ and $\Gamma\vdash {\tt x}\ty{\tt U}$ then {\tt T} and {\tt U} are syntactically identical. So if $\Gamma\vdash {\tt t}\ty{\tt C}[\tbar{A}]\{{\tt c}\}$ then it cannot be proved that $\Gamma\vdash {\tt t}\ty{\tt C}[\tbar{A}]\{{\tt d}\}$ for any other constraint even if equivalent to {\tt c}. However, one can establish the subtyping relation $\Gamma,{\tt x}\ty{\tt C}[\tbar{A}]\{{\tt c}\}\vdash {\tt x}\subtype{\tt C}[\tbar{A}]\{{\tt d}\}$.

Method invocations in a well-typed program do not violate the method guards at run time.

\begin{theorem}[Method guards] If $\Gamma \vdash {\tt e}.{\tt m}[\tbar{B}](\tbar{a})\ty{\tt T}$ and ${\tt e} \derives^*	 \new~{\tt C}[\tbar{A}](\tbar{v})$ and $\tbar{a} \derives^* \tbar{w}$ and $\methods({\tt C}[\tbar{A}])\ni\minst{B}{x}{G}{c}{H}{e}$ then $\Gamma\vdash {\tt c}[\new~{\tt C}[\tbar{A}](\tbar{v}),\tbar{w}/\this,\tbar{x}]$.
\end{theorem}

\begin{lemma}[Subject Reduction] If ${\tt e} \derives {\tt e'}$, and $\Gamma \vdash {\tt e}\ty{\tt T}$ then there exists a type {\tt S} such that $\Gamma\vdash{\tt e'}\ty{\tt S}$. Moreover, $\Gamma,{\tt x}\ty{\tt S} \vdash {\tt x}\subtype{\tt T}$.
\end{lemma}

\begin{lemma}[Progress]
If $\vdash {\tt e}\ty{\tt T}$ then one of the following conditions holds:
\begin{enumerate}
\item {\tt e} is a value,
\item {\tt e} contains a stuck cast sub-expression of the form ``${\tt v}~\as~{\tt G}$'',
\item there exists $\tt e'$ such that $\tt e\derives e'$.
\end{enumerate}
\end{lemma}

\begin{theorem}[Type soundness]
If $\vdash {\tt e}\ty{\tt T}$ and {\tt e}
reduces to a normal form ${\tt e'}$ then either ${\tt e'}$ contains a stuck cast sub-expression of the form ``${\tt v}~\as~{\tt G}$'' or ${\tt e'}$ is a value {\tt v} and there exists {\tt S} such that $\vdash {\tt v}\ty{\tt S}$. Moreover, in that case, ${\tt x}\ty{\tt S}\vdash{\tt x}\subtype{\tt T}$.
\end{theorem}

The proof of these results is detailed in a technical report,
which is available for download at {\tt x10-lang.org/papers}.

It relies on the following two pivotal lemmas:

\begin{lemma}
If $\Gamma,{\tt x}\ty{\tt T}\vdash{\tt x}~\has~{\tt I},{\tt x}~\has~{\tt J}$ where $\tt I$ and $\tt J$ are both fields or both methods with the same name then ${\tt I}={\tt J}$ modulo alpha-renaming of the type and value parameters.
\end{lemma}

\begin{lemma}
If $\Gamma,{\tt x}\ty{\tt T}\vdash{\tt x}\subtype{\tt G}$ and $\Gamma,{\tt x}\ty{\tt G}\vdash{\tt x}~\has~{\tt I}$ then there exists $\tt J$ such that $\Gamma,{\tt x}\ty{\tt T}\vdash{\tt x}~\has~{\tt J}, {\tt J} \ll {\tt I}$.
\end{lemma}

These lemmas makes it possible to separate the main body of the proof from the concrete treatment of type parameters. In the following section, we will further axiomatize type parameters. By making sure these language extensions preserve the two lemmas, we ensure the type system remain sound with minimal additions to the proof.

\subsection{The \FXG family}

In the base \FXG language, if a variable $\tt x$ has type-parameter type $\tt X$ then $\tt x$ has no accessible field or method. In this section, we demonstrate how to make type parameters more expressive while preserving principal and sound types.

The key idea is that information about type parameters can be accumulated through constraints. Then typing rules make use of these constraints to lookup members of variables of type-parameter types or establish subtyping relation about them.

\paragraph{\normalfont\bf\em Bounds.} \FGJ-style bounds on type parameters can be supported in the \FXG family by introducing an ``extends'' constraint \mbox{${\tt X}\extends{\tt G}$} on type parameters.

We need to make several additions to \FXG to make this work.
\begin{enumerate}
\item We extend the vocabulary of constraints:
\begin{center}
\begin{tabular}{r@{\quad}rcl}
  (Constraint) & {\tt c} &{::=}& $\tt X\extends G$ \\
\end{tabular}
\end{center}
\item We adopt a constraint system that can handle the new constraints and add inference rules about entailment if necessary---none in this case (see structural subtyping constraints below for an example). We also deal with inconsistent bounds, here by means of:
\vspace{-\medskipamount}
\infrule
	{{\tt G}\neq{\tt H}}
	{{\tt X}\extends{\tt G}\comma{\tt X}\extends{\tt H}\vdashX\false}
\vspace{-\medskipamount}
In this example, we only need to add entailment relations inside the constraint system. In general, we may want to derive \false{} from the typing judgments as well.

\item We specify well-formedness conditions:
\vspace{-\medskipamount}
\infrule
	{\wj{\Gamma}{\tt X} \andalso \wj{\Gamma}{\tt G}}
	{\cj{\Gamma}{{\tt X}\extends{\tt G}}}
\vspace{-\medskipamount}
\item We specify subtyping:
\vspace{-\medskipamount}
\infrule
	{\cj{\Gamma}{{\tt X}\extends{\tt G}} \andalso \Gamma\vdash {\tt X}\extends {\tt G}}
	{\Gamma,{\tt x}\ty{\tt X}\vdash {\tt x}\subtype {\tt G}}
\vspace{-\medskipamount}
\item We specify lookup:
\vspace{-\medskipamount}
\infrule
	{\cj{\Gamma}{{\tt X}\extends{\tt G}} \andalso \Gamma\vdash {\tt X}\extends {\tt G}}
	{\Gamma,{\tt x}\ty{\tt X}\vdash {\tt x}\subtype {\tt G}}
\vspace{-\medskipamount}
\end{enumerate}

The bounds we have just specified are very elementary. The X10 type system supports lower and upper bounds as well as a much more permissive notion of compatible bounds (i.e., bounds are compatible if they have a common subtype). But the critical observation here is that a more elaborate extension would proceed from the exact same methodology: first add constraints, then subtyping rules and lookup rules, finally worry about inconsistent constraints.

Whatever the extension, to preserve the soundness of the type system and the property of principal types---that is, the proof structure we have previously described---one need only ensure that:
\begin{enumerate}
\item If a type parameter $\tt X$ is established to be a subtype of type $\tt G$ then each member $\tt I$ of $\tt G$ must be matched by a member $\tt J$ of $\tt X$ whose signature overrides that of $\tt I$.

Formally, if $\Gamma,{\tt x}\ty{\tt X}\vdash{\tt x}\subtype{\tt G}$ and $\Gamma,{\tt x}\ty{\tt G}\vdash{\tt x}~\has~{\tt I}$ then there exists $\tt J$ such that $\Gamma,{\tt x}\ty{\tt X}\vdash{\tt x}~\has~{\tt J}, {\tt J} \ll {\tt I}$.

\item If a variable of type parameter $\tt X$ is established to have two members with the same name then they must have the same signature.

Formally, if $\Gamma,{\tt x}\ty{\tt X}\vdash{\tt x}~\has~{\tt I},{\tt x}~\has~{\tt J}$ where $\tt I$ and $\tt J$ are both fields or both methods with the same name then ${\tt I}={\tt J}$ modulo alpha-renaming of the type and value parameters.
\end{enumerate}

Since, the two properties holds for this extension, all results from Section~\ref{sec:results} apply.


\paragraph{Structural subtyping constraints.}
To conclude, we illustrate this methodology with another language extension: structural subtyping constraints. For brevity, we only consider constraint ${\tt X}~\underline{\has}~\msign{Y}{y}{G}{c}{H}$ and forget about fields. We use the underline notation ``$\underline{\has}$'' to distinguish the constraint on types from the predicate on variables. Of course, the two are closely related (by {\sc H-Struct}).

This time, we actually need the constraint system to satisfy a new inference rule {\sc X-Struct} so that generic classes and methods may be instantiated upon the actual class types of the program.

Rule {\sc KO-Struct} precludes constraints inconsistent with requirements 1 and 2 above; all results from Section~\ref{sec:results} apply.

\eat{

\paragraph{Bounds} To conclude this section, we now turn to showing how \FGJ{}-style bounds can be supported in the \FXG family.
We add to \FXG the productions and rules of Figure~\ref{fig:FXGB}.
The key idea is that information about type parameters can be accumulated through constraints. Specifically we introduce the ``extends'' constraint \mbox{${\tt X}\extends{\tt G}$}.

\FXGL{B} is built on top of a slightly more expressive
constraint system $\cal B$ that permits propagating such
constraints viewed as atoms. We then
add four rules to the type system respectively about
well-formedness, inconsistent constraints, subtyping, and member
lookup. {\sc X-Bound} ensures we do not type check programs
that specify multiple bounds for a given type
parameter.%\footnote{More precisely, multiple bounds can be specified provided they are syntactically identical.}

\subsection{Methodology}

While elementary, this last extension demonstrates the extensibility of the \FXGL{X} language and type system and its mechanisms. One can extend the \FXGL{X} language and the following by:
\begin{itemize}
\item extending the vocabulary of constraints;
\item specifying the well-formedness conditions for these constraints;
\item extending the constraint system with constraints on types (or constraints relating types and values);
\item specifying the subtyping relation for variables of parameter type; and
\item specifying the members of variables of parameter type.
\end{itemize}

In order to preserve the well-formedness of the type system and its proof one need only ensure that:
\begin{enumerate}
\item If $\Gamma,{\tt x}\ty{\tt X}\vdash{\tt x}\subtype{\tt G}$ and $\Gamma,{\tt x}\ty{\tt G}\vdash{\tt x}~\has~{\tt I}$ then there exists $\tt J$ such that $\Gamma,{\tt x}\ty{\tt X}\vdash{\tt x}~\has~{\tt J}, {\tt J} \ll {\tt I}$. That is, if a type parameter $\tt X$ is established to be a subtype of type $\tt G$ then each member $\tt I$ of $\tt G$ must be matched by a member $\tt J$ of $\tt X$ whose signature overrides that of $\tt I$.

\item If $\Gamma,{\tt x}\ty{\tt X}\vdash{\tt x}~\has~{\tt I},{\tt x}~\has~{\tt J}$ where $\tt I$ and $\tt J$ are both fields or both methods with the same name then ${\tt I}={\tt J}$ modulo alpha-renaming of the type and value parameters. That is, if a variable of type parameter $\tt X$ is established to have two members with the same name then the must have the same signature.
\end{enumerate}
In order to comply with these two requirements one can use rules of the form:
\infrule {\Gamma\vdash \phi}{\Gamma\vdash\false} whenever $\phi$ (or its consequences) would violate any of the above.

Formally, the proof of the soundness result is built on top of lemmas so that the main body of the proof does not directly depend on member lookup rules and subtyping rules for types parameters, but do it through lemmas that exactly formalize points 1 and 2 above.

}









\eat{



\section{Random bits}


In practice, it makes sense to distinguish the functions of the
constraint language from the functions of the base language.
One would define the {\sc T-Fun} typing judgment on a case-by-case
basis to
relate the interpretation of \xcd"q" as an expression to
its interpretation as a constraint term.

\FXD corresponds to the \CFJ calculus presented
in our prior work on constrained types~\cite{constrained-types}.  As described there, \Xten
supports equality constraints and has been extended with constraint
systems for Presburger arithmetic and for set constraints over
\Xten's array index domains (viz., regions).

\subsection{\FXG}
We now turn to showing how \FGJ{}-style generics can be supported in the \FXG{} family.
\FXGL{B} is the language obtained by adding to \FXGZ{} the
following productions:
\begin{center}
\begin{tabular}{r@{\quad}rcl}
  (Expression)& {\tt e} &{::=}& ${\tt C}\{{\tt c}\}$ \\
  (Value)& {\tt v} &{::=}& ${\tt C}\{{\tt c}\}$ \\
  (Path)& {\tt p} &{::=}& ${\tt x}$ \alt {\tt p}.{\tt f} \\
  (Type)& {\tt T} &{::=}& ${\tt p}$ \alt * \\
  (Constraint term)& {\tt t} &{::=}& ${\tt T}$ \\
  (Constraint) & {\tt c} &{::=}& ${\tt T}\extends {\tt T}$
\end{tabular}
\end{center}
\noindent
and deduction rules of Figure~\ref{fig:FXGB}.

First we introduce the ``type'' *. \FGJ{} method type
parameters are modeled in \FXG{} as normal parameters of type
*.\footnote{In concrete \Xten{} syntax type parameters are
distinguished from ordinary value parameters through the use of
``square'' brackets. This is particularly useful in implementing type
inference for generic parameters. We abstract these concerns away in
the abstract syntax presented in this section.}  Generic class
parameters are modeled as ordinary fields of type *, with
parameter bound information recorded as a constraint in the class
invariant. This decision to use fields rather than parameters is
discussed further in Section~\ref{sec:parameters-vs-fields}. In brief,
it permits powerful idioms using fixed but unknown types without
requiring ``wildcards''.

The set of well-formed types is now enhanced to permit some fixed but unknown
types {\tt x} as well as \emph{path types} (cf. \cite{scala}),
i.e., type-valued fields of objects as types.\footnote{But we will not permit invocations of methods with return type *\ to be 
used as types. This does indeed make sense, but developing
this theory further is beyond the scope of this paper.} We extend $\sigma$ in the obvious way:
%
\begin{center}
\begin{tabular}{l}
$	\sigma({\tt x}:*, \Gamma)=\sigma(\Gamma)$\\
$\sigma({\tt x}:{\tt y}, \Gamma)=\sigma(\Gamma)$\\
$\sigma({\tt x}:{\tt p}.{\tt f}, \Gamma)=\sigma(\Gamma)$
\end{tabular}
\end{center}
%
Reciprocally, we permit class types ${\tt C}\{{\tt c}\}$ to be
used as expressions. We type them accordingly ({\sc T-Type}). In
contrast, the ``type'' *{} is neither a valid expression nor
a class type: it has no field, method, subclass, or superclass.
It may however be constrained as usual as, for instance, in rule
{\sc T-Type}; that is to say, we permit equality constraints over types.\footnote{Type equality is just equality over uninterpreted functions.}

The key idea is that information about type-valued expressions can
be accumulated through constraints. Specifically we introduce 
the ``extends'' constraint ${\tt T}\extends{\tt U}$. It may be used, for
instance, to specify upper bounds on type variables or fields (path
types). In \FXG{}, users are permitted to specify ``$==$'' and ``$\extends$'' constraints
about type variables, fields, and class types.

\begin{example}
The \FGJ{} parametric method

\begin{xten} 
<T> T id(T x) { return x; }
\end{xten}
\noindent can be represented as
\begin{xten} 
def id(T: type, x: T): T = x;
\end{xten}
\end{example}

\begin{example}
\noindent The \FGJ{} class 
\begin{xten} 
class Comparator<B> {
  int compare(B y) { ... } }
class SortedList<T extends Comparator<T>> { 
  int m(T x, T y) { return x.compare(y); } }
\end{xten}
\noindent can be represented as
\begin{xtenmath} 
class Comparator(B: type) {
  def compare(y:B):int = ...; }
class SortedList(T: type)
    {T $\extends$ Comparator{self.B==T}} { 
  def m(x:T, y:T):int = x.compare(y); }
\end{xtenmath}
\end{example}

We require the underlying constraint system $\mathcal{G}$ to treat ``$\extends$'' as a partial order relation (reflexive, antisymmetric, and transitive). It is possible for a program to specify constraints incompatible with the class hierarchy, e.g., ${\tt x}\extends{\tt C}$ and ${\tt x}\extends{\tt D}$ if both class {\tt C} and class {\tt D} extend {\tt Object}. We therefore require $\mathcal{G}$ to treat as inconsistent all sets of constraints on type-valued variables that admit no valuations where these variables take on types as values.

The ``$\extends$'' constraint is used in two deduction rules. If
type {\tt T} extends type {\tt U}, then
\begin{itemize}
\item{\sc S-Extends}. {\tt T} is a subtype of {\tt U}. A method or constructor with argument type {\tt U} may be passed a parameter of type {\tt T}.
\item{\sc L-Extends}. If {\tt x} has type {\tt U} then {\tt x} has all the members of type {\tt T}. Note we only extend the ``$\underline\has$'' predicate that is used in typing judgments. On the other hand, the ``$\has$'' predicate used for method lookup in the operational semantics is not affected.
\end{itemize}

The modification of the lookup predicate is
necessary to permit typing method invocations with receivers of
generic types. It has the unfortunate side effect that we can no
longer ensure that type derivations---and hence types---are unique.
For instance, given the class definitions:
%
\begin{xten}
class A() extends Object { def m():A = new A(); }
class B() extends A { def m():B new B(); }
class C(f:type){this.f<=A} extends Object {}
class D(){this.f<=B} extends C { ..this.f.m().. }
\end{xten}
%
occurrences of $\this.{\tt f}$ in {\tt D} are bounded both by {\tt A} and {\tt B} hence 
$\this.{\tt f}.{\tt m}()$ may either be typed using the declaration of {\tt m} in {\tt A} or {\tt B}.

Another property of \FXG{} worth noticing is that casts can ``erase'' typing information.
Consider the program:
\begin{xten}
class C() extends Object {}
class D(f:type, g:this.f) extends Object {}
\end{xten}
Class {\tt D} has a type parameter {\tt f} and a value field {\tt g} of type {\tt f}.
Thanks to constraints, if
${\tt e}=\new~{\tt D}({\tt C},\new~{\tt C}())$,
then expression ${\tt e}.{\tt g}$ can be shown to
have type {\tt C}.
In contrast $({\tt e}~\as~{\tt D}).{\tt g}$ has type
$\exists {\tt x}:{\tt D}.{\tt x}.{\tt f}\{\self=={\tt x.g}\}$.
The type of $({\tt e}~\as~{\tt D}).{\tt g}$ is essentially ``unknown''
because the cast erased all information about it. In \Xten, we choose to shield users from existential types and only permit casts of the form $({\tt e}~\as~{\tt D}\{\self.{\tt f}=={\tt t}\})$ where {\tt t} is a type in scope (class type, type parameter, or path type).


\subsection{\FXGD} 

No additional rules are needed beyond those of \FXG{} and \FXD{}. This
language permits type and value constraints, supporting \FGJ{} style
generics and value-dependent types. All constraints but existential constraints are now user constraints.

\subsection{Results}
The following results hold for \FXGD supposing the program {\tt P} type checks.

\begin{theorem}[Subject Reduction] If $\Gamma$ is well-formed and $\Gamma \vdash {\tt e:T}$ and ${\tt e} \derives {\tt e'}$, then
for some type {\tt S}, $\Gamma \vdash {\tt {\tt e}':{\tt S}}\comma{\tt S} \subtype {\tt T}$.
\end{theorem}

Values are of the form $\tt v ::= \new\ C(\bar{\tt v}) \alt {\tt d} \alt C\{c\}$.

\begin{theorem}[Progress]
If $\vdash {\tt e:T}$ then one of the following conditions holds:
\begin{enumerate}
\item {\tt e} is a value,
\item {\tt e} contains a cast sub-expression that is stuck,
\item there exists an $\tt e'$ s.t. $\tt e\derives e'$.
\end{enumerate}
\end{theorem}

\begin{theorem}[Type soundness]
If $\vdash {\tt e:T}$ and {\tt e}
reduces to a normal form ${\tt e'}$, then
either $\tt e'$ is a value {\tt v} and $\vdash {\tt v:S},{\tt S\subtype T}$ or
${\tt e'}$ contains  a stuck cast sub-expression.
\end{theorem}

\paragraph{Proof sketch.} The proof of the same results for a
formal language essentially equivalent to \FXD{} has been
reported in \cite{constrained-types}. We discuss here the key
insights that permit us to revise this proof in order to encompass \FXGD{}.
\begin{itemize}
\item Subject reduction. Having potentially multiple types for
an expression does not make the proof any harder as the subject
reduction theorem lets us choose {\tt S} among the possible types of ${\tt e}'$.

The main novelty of the \FXG{} type system is that it permits
the $\underline\has$ predicate to look for methods in arbitrary
superclasses or upper bounds of the type under scrutiny. This is
not so much a concern for fields as they cannot be overridden.
Because methods can, we must adapt the proof of subject reduction for the execution step corresponding to a method invocation (\RInvk).

First, we observe that the operational semantics rule for method
invocations (\RInvk) is required to employ the ``correct''
method for objects of class {\tt C}, that is, the first method
{\tt m} found on the inheritance path from class {\tt C} to
class {\tt Object} from the bottom up. Second, thanks
to overriding restrictions, we know that this method must have a
return type that is a subtype of any other method {\tt m}
defined in any superclass of {\tt C}. Finally, because
constraint sets incompatible with the class hierarchy are
inconsistent, we also know that the type of
$\new~{\tt C}(\tbar{e})$ cannot be constrained to have any upper
bound that is not {\tt C} itself or one of its superclasses.

We
therefore derive that any method instance one could use to type
the expression $\new~{\tt C}(\tbar{e}).{\tt m}(\tbar{a})$ has a
return type that is a supertype of the return type of the only
method instance that can be used to make a step of execution. We
assume the program type checks; hence, by {\sc OK-Method}, we
know that the actual residue ${\tt b}[\new~{\tt C}(\tbar{e}),\tbar{a}/\this,\tbar{x}]$ is guaranteed to have a
type that is a subtype of its declared type. Therefore, by
transitivity of the subtyping relation, we can derive that if
{\tt T} is a type of $\new~{\tt C}(\tbar{e}).{\tt m}(\tbar{a})$,
then there exists a type {\tt S} of
${\tt b}[\new~{\tt C}(\tbar{e}),\tbar{a}/\this,\tbar{x}]$
that is a subtype of {\tt T}.

\item Progress. \FXGD{} only differs from \FXD{} in that it
admits a new kind of expressions: {\tt C}\{{\tt c}\}. But these are also values, so the proof of progress is essentially unchanged.
\eat{
 {\tt d} ~$|$~ {\tt q}(\tbar{e}) ~$|$~ {\tt C}\{{\tt c}\}.
Both literals and class types are values, we just have to establish progress in the case of function calls (in the context of a proof by induction on the structure of the expression {\tt e}).
Assume ${\tt q}(\tbar{e})$ is well-typed then if all \tbar{e} are values progress is possible by rule {\sc R-Fun}. If not, then at least one of the expressions is not a value and by induction hypothesis applied to this expression we can conclude that it is either stuck on a cast or can make a step of execution step. We conclude using rule {\sc RC-Fun} in the second case.}
\item{Type soundness}. Direct consequence of the previous two theorems.
\end{itemize}



\eat{We proceed by induction on the last rule used in the proof of ${\tt e} \derives {\tt e'}$. The key case is {\sc R-Invk}. Because of rule 


\begin{itemize}
\item {\sc RC-Field}. If $\Gamma\vdash S<:T$ then the fields of $T$ are the first fields of $S$.
\item {\sc RC-Invk-Recv}. If $\Gamma\vdash S<:T$ and $\Gamma,x:T\vdash x~\underline\has~m(y:U)\{c\}:V=a$ then there exists $d$ and $W$ such that $\Gamma,x:S\vdash x~\underline\has~m(y:U)\{d\}:W=b$ and $d$ entails $c$ and $W<:V$.
\item {\sc RC-Cast}. Straightforward.
\item {\sc RC-New-Arg}. If $\Gamma\vdash S<:T$ then $\exists x:T.U<:\exists x:S.U$.
\item {\sc RC-Invk-Arg}. Same.
\item {\sc R-Invk}. If $\Gamma,x:C\vdash x~\has~m(y:U)\{c\}:V=a$ then for all $d$ and $W$ such that $\Gamma,x:C\vdash x~\underline\has~m(y:U)\{d\}:W=b$ it is the case that $c$ entails $d$ and $V<:W$.
\item {\sc R-Field}. We have $\Gamma\vdash t:V$ for some $V$, $\Gamma\vdash \new~C(t).f_i:W_i\{\exists x:(\exists y:V.C\{\self==\new~C(y)\}).\self==x.f_i\}$ where $\Gamma\vdash C~\has~f_i:W_i$. We prove $\Gamma\vdash V_i<:W_i\{\exists x:(\exists y:V.C\{\self==\new~C(y)\}).\self==x.f_i\}$.
\item {\sc R-Cast}. Straightforward.
\end{itemize}
}


\eat{




\begin{figure*}
\vspace{-\bigskipamount}
\begin{minipage}{\textwidth}
\quad\typicallabel{XXXXXX}
\infax[W-True]
  {\wj{}{\tt true}}

\infax[W-False]
  {\wj{}{\tt false}}

\infrule[W-Conj]
	{\wj{\Gamma}{\tt c} \andalso 	\wj{\Gamma}{\tt d}}
	{\wj{\Gamma}{{\tt c},{\tt d}}}

\infrule[W-Eq]
	{}
	{\wj{\Gamma}{\tt t}=={\tt u}}

\infrule[W-Dep]
  {\wj{\Gamma}{\tt T} \andalso \wj{\Gamma, {\tt x}:{\tt T}}{\tt c}}
	{\wj{\Gamma}{\exists {\tt x}:{\tt T}.~{\tt c}}}
\end{minipage}
\caption{{\sf FXG} well-formed constraints}
\label{fig:well}
\end{figure*}





\begin{figure*}
\vspace{-\bigskipamount}
\begin{minipage}{\textwidth}
\quad\typicallabel{XXXXXX}
\infax[W-Var]
  {\wj{{\tt x}:{\tt T}}{\tt x}}

\infrule[W-Field]
	{\wj{\Gamma}{\tt t}}
	{\wj{\Gamma}{{\tt t}.{\tt f}}}

\infrule[W-New]
	{\wj{\Gamma}{\tbar{t}} \andalso \wj{\Gamma}{\tbar{G}}}
	{\wj{\Gamma}{\new~{\tt C}[\tbar{G}](\tbar{t})}}
\end{minipage}
\caption{{\sf FXG} well-formed constraint terms}
\label{fig:well}
\end{figure*}
}
\eat{
The syntax for constraints in \FXZ{} is specified in
Figure~\ref{fig:fxg-grammar}. As expected, constraints
relate property fields of objects. Neither casts
nor method invocations are permitted in constraints.

We distinguish a subset of these constraints as
{\em user constraints}---these are permitted to occur in
programs. For \FXZ{} the only user constraint permitted is the vacuous
{\tt true}. Thus the types occurring in user programs are isomorphic
to class types, and class and method definitions specialize to the
standard class and method definitions of \FJ{}. 

The constraints permitted by the syntax in
Figure~\ref{fig:fxg-grammar} that
are not user constraints are used to define the static and
dynamic semantics of \FXZ{} (see, e.g., rule \TField{} in Figure~\ref{fig:FX}).
The use of this richer constraint set as well as constrained and existential types is
not necessary in \FXZ; it simply enables us to present the static and dynamic
semantics once for the entire family of \FX{} languages,
specifying the other members of the family as extensions
to these core semantics.

Existential constraints are introduced for convenience only:
${\tt T}\{\exists {\tt x}:{\tt U}.~{\tt c}\}$ is equivalent to $\exists {\tt y}:{\tt U}.~{\tt T}\{{\tt c}[{\tt y}/{\tt x}]\}$ choosing {\tt y} not free in {\tt T}.
}

\eat{
Because of existential and dependent types we adopt a non-conventional notation for subtyping that we explain and motivate below. 
\{\infrule[S-Sub]
	{\wj{\Gamma}{\tt S} \andalso \Gamma,{\tt x}\ty{\tt S}\vdash{\tt x}\subtype {\tt T} \andalso {\tt x}~\rm not~free~in~{\tt T}}
	{\Gamma\vdash{\tt S}\subtype{\tt T}}

The typing rule for casts ({\sc T-New}) in Figure~\ref{fig:FX} specifies that if the arguments $\tbar{e}$ have type $\tbar{V}$ then $\new~{\tt C}(\tbar{e})$ has type $\exists\tbar{y}:\tbar{V}.~{\tt C}\{\self==\new~{\tt C}(\tbar{y})\}$, therefore {\RCast} requires this particular type to be a subtype of {\tt T}.
In \FXZ, this
test simply involves checking that the class of which the object is an
instance is a subclass of the class specified in the given type; in
languages with richer notions of type this operation may
involve run-time constraint solving using the properties of the object.
See Section~\ref{sec:casts} for further discussion of the casts,
including decidability issues.
}

\eat{
Each language in the family is defined over a given input constraint system $\mathcal{X}$ that is required to support the trivial constraint \true{}, conjunction, and equality on constraint terms. Given a program {\tt P}, we now show how to derive from $\mathcal{X}$ a larger deduction system that captures the object-oriented structure of {\tt P} and lets us decide whether {\tt P} is well-typed.

\medskip
}

}


\eat{
\section{Translation}
\label{sec:translation}
\label{sec:impl}
This section describes an implementation approach for generic types in
\Xten{} on a JVM, with bytecode rewriting.

The design is a hybrid design combining techniques of run-time
instantiation from
NextGen~\cite{nextgen,allen03} and type-passing from
PolyJ~\cite{java-popl97}.  Generic classes are translated
into ``template'' classes that are instantiated on demand at run time by
binding the type properties to concrete types.
%
Constraints on values are erased from type references.
Adapter objects are used to represent type
properties and constraints.  
Run-time type tests (e.g., casts) are translated
into code that checks those constraints at run time.
%
This design has been implemented in the \Xten{} compiler, built
on the Polyglot compiler framework~\cite{ncm03}.  The compiler
translates \Xten{} source to Java source, which is then compiled
to Java bytecode using an off-the-shelf Java compiler.\footnote{There is also
a translation from \Xten{} to C++ source, not described here.}
The \Xten{} runtime is augmented with a class loader
implementation that performs run-time instantiation.

\paragraph{Classes.}
Each class is translated into a \emph{template class}.
The template class is compiled by a Java compiler (e.g., javac)
to produce a class file.
At run time, when a constrained type \xcd"C{c}" is first referenced, a
class loader loads the template class for \xcd"C" and then
transforms its bytecode, specializing it to the constraint
\xcd"c".  The implementation specializes code based on type constraints,
not value constraints; we leave value-constraint specialization to
future work.
%
For example, consider the following classes.
{
\begin{xten}
class A[X] {
  var a: X;
}
\end{xten}}
{
\begin{xten}
class C {
  val x: A[int] = new A[int]();
  val y: int = x.a;
}
\end{xten}}

The compiler generates the following Java code:
{
\begin{xten}
@Parameters({"X"})
class A {
  @TypeProperty public static class X { }
  public x10.runtime.Type X;
  X a;
  @Synthetic public A(Class X) { this(); }
}
\end{xten}}
{
\begin{xten}
class C {
  final A x = new A(int.class);
  final int y = Runtime.to$int(x.a);
}
\end{xten}}

The member class \xcd"A.X" is used in place of the
type property \xcd"X".   The field \xcd"X" of type
\xcd"x10.runtime.Type" captures the actual constrained type on which \xcd"A"
is instantiated, and is used for run-time type tests.
The \xcd"@Parameters" annotation on \xcd"A" is used during
run-time instantiation to identify the type properties.
Synthetic constructors with added \xcd"Class" parameters are
used to pass instantiation arguments to the \xcd"new"
expression.
This code is compiled to Java bytecode.

When an expression (e.g., \xcd"new C()") is evaluated,
the class \xcd"C" is loaded.
The class loader transforms the bytecode as if it had
been written as follows:

{
\begin{xten}
class C {
  final A$$int x = new A$$int();
  final int y = x.a;
}
\end{xten}}

The class loader rewrites allocations of template classes
(e.g., \xcd"new A(int.class)") into allocations of the
instantiated classes (i.e., \xcd"new A$$int()").
The template class name and actual type arguments are mangled to
derive the name of the instantiated class.
This code cannot be generated directly because
class \xcd"A$$int" does not yet exist; the Java source compiler
would fail to compile \xcd"C".

Upon evaluation of the constructor,
the class \xcd"A$$int" is loaded.
The class loader intercepts
this, demangles the name, and loads the bytecode for the
template class \xcd"A".
The bytecode is transformed, replacing the type property \xcd"X"
with the concrete type \xcd"int".

Parameter types are coerced to and from the actual type
\xcd"T" (a Java primitive type or \xcd"Object") using
method \xcd"Runtime.to$T(Object)" and \xcd"Runtime.from(T)",
possibly with additional casts.
Both are eliminated from the transformed
bytecode, but are needed for the template class to type-check.

\eat{
Currently, the class loader instantiates the template for
every encountered combination of parameters.  If desired,
it is possible (and relatively easy) to optimize this scheme
to instantiate only for the Java primitive types and Object,
giving nine possible instantiations per parameter.
}

%Instantiations are used for representation.
%Adapter objects are used for run time type information.
%
%Could do instantiation eagerly, but quickly gets out of hand without
%whole-program analysis to limit the number of instantiations: 9
%instantiations for one type property, 81 for two type
%properties, 729 for three.

%Constructors are translated to static methods of their
%enclosing
%class.
%Constructor calls
%are translated to calls to static methods.

\eat{
Consider the code in Figure~\ref{fig:translation1}.  It contains most of the
features of generics that have to be translated.
\begin{figure*}[tp]
{\footnotesize
\begin{xtenmath}
class C[T] {
    var x: T;
    def this[T](x: T) { this.x = x; }
    def set(x: T) { this.x = x; }
    def get(): T { return this.x; }
    def map[S](f: F[T,S]): S { return f._(this.x); }
    def d() { return new D[T](); }
    def t() { return new T(); } // FIXME
    def isa(y: Object): boolean { return y instanceof C[T]; }
}
abstract class F[T,S] { S _(T x); }

val x : C = new C[String]();
val y : C[Int] = new C[Int]();
val z : C{T $\extends$ Array} = new C[Array[Int]]();
val f : F[String,Int] = ...;
x.map[Int](f);
new C[Int{self==3}]() instanceof C[Int{self<4}];
\end{xten}}
\caption{Code to translate}
\label{fig:translation1}
\end{figure*}

The translated version is shown in Figure~\ref{fig:translation2}.
\begin{figure*}[tp]
{\footnotesize
\begin{xten}
@Parameters({"T"})
class C {
    @TypeProperty public static class T { }
    T x;
    C(T x) { this.x = x; }
    @Synthetic C(Class T, T x) { this(x); }
    @Synthetic public static boolean instanceof\$(Object o, String constraint) { assert(false); return true; }
    public static boolean instanceof\$(Object o, String constraint, boolean b) { /*check constraint*/; return b; }
    public static Object cast\$(Object o, String constraint) { /*check constraint*/; return (C)o; }
    void set(T x) { this.x = x; }
    T get() { return this.x; }
    @Synthetic
    @Parameters("S")
    public static class map {
        public static class S { };
        public C c;
        public map(C c) { this.c = c; }
        @Synthetic
        public map(Class S, C c) { this(c); }
        public S apply(@InstantiateClass({"C\$T", "C\$map\$S"}) F f) { return f._(c.x); }
        @Synthetic
        public T apply(Class T, T x, T y) { return apply(x, y); } // We might only need one
    }
    @Synthetic
    @ParametricMethod("T")
    Object make\$map(Class T) { assert(false); return null; }
    @Synthetic
    Object make\$map(Class T, boolean ignored) {
        Object retval = null;
        try {
            X10RuntimeClassloader cl = (X10RuntimeClassloader)C.class.getClassLoader();
            Class<?> c = cl.instantiate(map.class, T); 
            retval = c.getDeclaredConstructor(new Class[] { C.class }).newInstance(this);
        }
        catch (IllegalAccessException e) { }
        catch (NoSuchMethodException e) { }
        catch (InstantiationException e) { }
        catch (InvocationTargetException e) { }
        return retval;
    }
    @InstantiateClass({"C\$T"}) D d() { return new D(T.class); }
    T t() { return new T(); } // FIXME
    boolean isa(Object y) { return Runtime.instanceof\$(C.instanceof\$(y, null), T.class); }
}
@Parameters({"T","S"})
abstract class F { ... }

C x = new C(String.class);
C y = new C(int.class);
C z = new C(((X10RuntimeClassloader)C.class.getClassLoader()).getClass("Array\$\$int"));
F f = ...;
((C.map)(Object)(C.map)x.make\$map(int.class)).apply(int.class, f);

Runtime.instanceof\$(C.instanceof\$(new C(int.class)(), "self<4"), int.class);
\end{xten}}
\caption{Translated code}
\label{fig:translation2}
\end{figure*}
}

\paragraph{Passing type arguments.}

For types visible at run time, annotations are used to
pass actual type arguments to the class loader.
The annotation \xcd"@InstantiateClass"
specifies the type parameter and
is placed on
fields, methods,
method parameters, and classes to
indicate instantiation parameters for field
types, method return types, method parameters, and superclasses,
respectively.
Interface instantiations are similarly handled
by \xcd"@InstantiateInterfaces".
The annotation
\xcd"@Instantiation"
is used for parametrized exceptions.
The class loader uses the arguments of the annotations to
propagate the instantiation information of the enclosing class
to the instantiation of annotated entities.  It then turns these
entities into references to the appropriate dynamically
instantiated classes.

Type arguments are passed to allocation expressions as
synthetic constructor arguments.  Run-time type tests and casts
receive type parameters via the \xcd"Runtime.cast$" and
\xcd"Runtime.instanceof$" helper methods.

\paragraph{Eliminating method type parameters.}

A parametrized
adapter class with an \xcd"apply" method
is generated for each parametrized method,
The adapter class
is annotated with \xcd"@ParametricMethod".
The parametrized method is invoked by instantiating the adapter
class through a generated factory method
and invoking its \xcd"apply()" method.

\paragraph{Parametrized exceptions.}

Parametrized exceptions are treated just like other classes.
Synthetic local classes, annotated with \xcd"@Instantiation",
are generated for each catch block with an instantiated
generic exception class.  Exception tables in the
bytecode are rewritten with the new exception types.

\paragraph{Run-time instantiation.}

The
\xcd"instanceof" and cast operations on
constrained types or type variables
are translated
to
similar operations on the instantiated type followed by calls
to
methods of the adapter object for the type
that evaluate the constraint.
% run-time constraint solving or other
% complex code that cannot be easily substituted in when rewriting
% the bytecode during instantiation.

}

\section{Conclusions}
\label{sec:conclusions}

We have presented a constraint-based framework \FXG{} for type-
and value-dependent types in an object-oriented language.
%
The use of constraints on type properties allows the design to
capture many features of generics in object-oriented languages
and then to extend these features with more
expressive power.  We have proved the type system sound.

We plan to extend the type system to account for mutable state.
We believe the extension is straightforward, although
cumbersome, because
constraints are only immutable state only and because the formalism
carefully controls occurrences of existential types.

The type system of \FXG formalizes the semantics of the \Xten{}
programming language.  The design admits an efficient
implementation for generics and dependent types in \Xten{},
available at \texttt{x10-lang.org}.
To improve the expressiveness of \Xten{}, we plan to implement
a type inference algorithm that infers constraints over types
and values, and to support user-defined constraints.

\section*{Acknowledgments} 
This material is based upon work supported in part by the Defense Advanced Research Projects Agency under its Agreement No HR0011-07-9-0002.


\eat{
\section*{Acknowledgments} 

The authors thank Bob Blainey,
Doug Lea, Jens Palsberg, and Lex Spoon
for valuable feedback on versions of the language.
We thank
Andrew Myers and
Michael Clarkson for providing us with their implementation of
PolyJ, on which our implementation was based, and for many
discussions over the years about parametrized types in Java.
}

\bibliographystyle{plain}
\bibliography{master}

% \appendix
% \onecolumn

% \section{An extended example}
% {\footnotesize
\begin{verbatim}
/**
   A distributed binary tree.
   @author Satish Chandra 4/6/2006
   @author vj
 */
//                             ____P0
//                            |     |
//                            |     |
//                          _P2  __P0
//                         |  | |   |
//                         |  | |   |
//                        P3 P2 P1 P0
//                         *  *  *  *
// Right child is always on the same place as its parent;
// left child is at a different place at the top few levels of the tree,
// but at the same place as its parent at the lower levels.

class Tree(localLeft: boolean,
           left: nullable Tree(& localLeft => loc=here),
           right: nullable Tree(& loc=here),
           next: nullable Tree) extends Object {
    def postOrder:Tree = {
        val result:Tree = this;
        if (right != null) {
            val result:Tree = right.postOrder();
            right.next = this;
            if (left != null) return left.postOrder(tt);
        } else if (left != null) return left.postOrder(tt);
        this
    }
    def postOrder(rest: Tree):Tree = {
        this.next = rest;
        postOrder
    }
    def sum:int = size + (right==null => 0 : right.sum()) + (left==null => 0 : left.sum)
}
value TreeMaker {
    // Create a binary tree on span places.
    def build(count:int, span:int): nullable Tree(& localLeft==(span/2==0)) = {
        if (count == 0) return null;
        {val ll:boolean = (span/2==0);
         new Tree(ll,  eval(ll => here : place.places(here.id+span/2)){build(count/2, span/2)},
           build(count/2, span/2),count)}
    }
}
\end{verbatim}}

\subsection{Places}
{\footnotesize
\begin{verbatim}
/**

 * This class implements the notion of places in X10. The maximum
 * number of places is determined by a configuration parameter
 * (MAX_PLACES). Each place is indexed by a nat, from 0 to MAX_PLACES;
 * thus there are MAX_PLACES+1 places. This ensures that there is
 * always at least 1 place, the 0'th place.

 * We use a dependent parameter to ensure that the compiler can track
 * indices for places.
 *
 * Note that place(i), for i <= MAX_PLACES, can now be used as a non-empty type.
 * Thus it is possible to run an async at another place, without using arays---
 * just use async(place(i)) {...} for an appropriate i.

 * @author Christoph von Praun
 * @author vj
 */

package x10.lang;

import x10.util.List;
import x10.util.Set;

public value class place (nat i : i <= MAX_PLACES){

    /** The number of places in this run of the system. Set on
     * initialization, through the command line/init parameters file.
     */
    config nat MAX_PLACES;

    // Create this array at the very beginning.
    private constant place value [] myPlaces = new place[MAX_PLACES+1] fun place (int i) {
	return new place( i )(); };

    /** The last place in this program execution.
     */
    public static final place LAST_PLACE = myPlaces[MAX_PLACES];

    /** The first place in this program execution.
     */
    public static final place FIRST_PLACE = myPlaces[0];
    public static final Set<place> places = makeSet( MAX_PLACES );

    /** Returns the set of places from first place to last place.
     */
    public static Set<place> makeSet( nat lastPlace ) {
	Set<place> result = new Set<place>();
	for ( int i : 0 .. lastPlace ) {
	    result.add( myPlaces[i] );
	}
	return result;
    }

    /**  Return the current place for this activity.
     */
    public static place here() {
	return activity.currentActivity().place();
    }

    /** Returns the next place, using modular arithmetic. Thus the
     * next place for the last place is the first place.
     */
    public place(i+1 % MAX_PLACES) next()  { return next( 1 ); }

    /** Returns the previous place, using modular arithmetic. Thus the
     * previous place for the first place is the last place.
     */
    public place(i-1 % MAX_PLACES) prev()  { return next( -1 ); }

    /** Returns the k'th next place, using modular arithmetic. k may
     * be negative.
     */
    public place(i+k % MAX_PLACES) next( int k ) {
	return places[ (i + k) % MAX_PLACES];
    }

    /**  Is this the first place?
     */
    public boolean isFirst() { return i==0; }

    /** Is this the last place?
     */
    public boolean isLast() { return i==MAX_PLACES; }
}
\end{verbatim}}
\subsection{$k$-dimensional regions}
{\footnotesize
\begin{verbatim}
package x10.lang;

/** A region represents a k-dimensional space of points. A region is a
 * dependent class, with the value parameter specifying the dimension
 * of the region.
 * @author vj
 * @date 12/24/2004
 */

public final value class region( int dimension : dimension >= 0 )  {

    /** Construct a 1-dimensional region, if low <= high. Otherwise
     * through a MalformedRegionException.
     */
    extern public region (: dimension==1) (int low, int high)
        throws MalformedRegionException;

    /** Construct a region, using the list of region(1)'s passed as
     * arguments to the constructor.
     */
    extern public region( List(dimension)<region(1)> regions );

    /** Throws IndexOutOfBoundException if i > dimension. Returns the
        region(1) associated with the i'th dimension of this otherwise.
     */
    extern public region(1) dimension( int i )
        throws IndexOutOfBoundException;


    /** Returns true iff the region contains every point between two
     * points in the region.
     */
    extern public boolean isConvex();

    /** Return the low bound for a 1-dimensional region.
     */
    extern public (:dimension=1) int low();

    /** Return the high bound for a 1-dimensional region.
     */
    extern public (:dimension=1) int high();

    /** Return the next element for a 1-dimensional region, if any.
     */
    extern public (:dimension=1) int next( int current )
        throws IndexOutOfBoundException;

    extern public region(dimension) union( region(dimension) r);
    extern public region(dimension) intersection( region(dimension) r);
    extern public region(dimension) difference( region(dimension) r);
    extern public region(dimension) convexHull();

    /**
       Returns true iff this is a superset of r.
     */
    extern public boolean contains( region(dimension) r);
    /**
       Returns true iff this is disjoint from r.
     */
    extern public boolean disjoint( region(dimension) r);

    /** Returns true iff the set of points in r and this are equal.
     */
    public boolean equal( region(dimension) r) {
        return this.contains(r) && r.contains(this);
    }

    // Static methods follow.

    public static region(2) upperTriangular(int size) {
        return upperTriangular(2)( size );
    }
    public static region(2) lowerTriangular(int size) {
        return lowerTriangular(2)( size );
    }
    public static region(2) banded(int size, int width) {
        return banded(2)( size );
    }

    /** Return an \code{upperTriangular} region for a dim-dimensional
     * space of size \code{size} in each dimension.
     */
    extern public static (int dim) region(dim) upperTriangular(int size);

    /** Return a lowerTriangular region for a dim-dimensional space of
     * size \code{size} in each dimension.
     */
    extern public static (int dim) region(dim) lowerTriangular(int size);

    /** Return a banded region of width {\code width} for a
     * dim-dimensional space of size {\code size} in each dimension.
     */
    extern public static (int dim) region(dim) banded(int size, int width);


}

\end{verbatim}}

\subsection{Point}
{\footnotesize
\begin{verbatim}
package x10.lang;

public final class point( region region ) {
    parameter int dimension = region.dimension;
    // an array of the given size.
    int[dimension] val;

    /** Create a point with the given values in each dimension.
     */
    public point( int[dimension] val ) {
        this.val = val;
    }

    /** Return the value of this point on the i'th dimension.
     */
    public int valAt( int i) throws IndexOutOfBoundException {
        if (i < 1 || i > dimension) throw new IndexOutOfBoundException();
        return val[i];
    }

    /** Return the next point in the given region on this given
     * dimension, if any.
     */
    public void inc( int i )
        throws IndexOutOfBoundException, MalformedRegionException {
        int val = valAt(i);
        val[i] = region.dimension(i).next( val );
    }

    /** Return true iff the point is on the upper boundary of the i'th
     * dimension.
     */
    public boolean onUpperBoundary(int i)
        throws IndexOutOfBoundException {
        int val = valAt(i);
        return val == region.dimension(i).high();
    }

    /** Return true iff the point is on the lower boundary of the i'th
     * dimension.
     */
    public boolean onLowerBoundary(int i)
        throws IndexOutOfBoundException {
        int val = valAt(i);
        return val == region.dimension(i).low();
    }
}
\end{verbatim}}

\subsection{Distribution}
{\footnotesize
\begin{verbatim}
package x10.lang;

/** A distribution is a mapping from a given region to a set of
 * places. It takes as parameter the region over which the mapping is
 * defined. The dimensionality of the distribution is the same as the
 * dimensionality of the underlying region.

   @author vj
   @date 12/24/2004
 */

public final value class distribution( region region ) {
    /** The parameter dimension may be used in constructing types derived
     * from the class distribution. For instance,
     * distribution(dimension=k) is the type of all k-dimensional
     * distributions.
     */
    parameter int dimension = region.dimension;

    /** places is the range of the distribution. Guranteed that if a
     * place P is in this set then for some point p in region,
     * this.valueAt(p)==P.
     */
    public final Set<place> places; // consider making this a parameter?

    /** Returns the place to which the point p in region is mapped.
     */
    extern public place valueAt(point(region) p);

    /** Returns the region mapped by this distribution to the place P.
        The value returned is a subset of this.region.
     */
    extern public region(dimension) restriction( place P );

    /** Returns the distribution obtained by range-restricting this to Ps.
        The region of the distribution returned is contained in this.region.
     */
    extern public distribution(:this.region.contains(region))
        restriction( Set<place> Ps );

    /** Returns a new distribution obtained by restricting this to the
     * domain region.intersection(R), where parameter R is a region
     * with the same dimension.
     */
    extern public (region(dimension) R) distribution(region.intersection(R))
        restriction();

    /** Returns the restriction of this to the domain region.difference(R),
        where parameter R is a region with the same dimension.
     */
    extern public (region(dimension) R) distribution(region.difference(R))
        difference();

    /** Takes as parameter a distribution D defined over a region
        disjoint from this. Returns a distribution defined over a
        region which is the union of this.region and D.region.
        This distribution must assume the value of D over D.region
        and this over this.region.

        @seealso distribution.asymmetricUnion.
     */
    extern public (distribution(:region.disjoint(this.region) &&
                                dimension=this.dimension) D)
        distribution(region.union(D.region)) union();

    /** Returns a distribution defined on region.union(R): it takes on
        this.valueAt(p) for all points p in region, and D.valueAt(p) for all
        points in R.difference(region).
     */
    extern public (region(dimension) R) distribution(region.union(R))
        asymmetricUnion( distribution(R) D);

    /** Return a distribution on region.setMinus(R) which takes on the
     * same value at each point in its domain as this. R is passed as
     * a parameter; this allows the type of the return value to be
     * parametric in R.
     */
    extern public (region(dimension) R) distribution(region.setMinus(R))
        setMinus();

    /** Return true iff the given distribution D, which must be over a
     * region of the same dimension as this, is defined over a subset
     * of this.region and agrees with it at each point.
     */
    extern public (region(dimension) r)
        boolean subDistribution( distribution(r) D);

    /** Returns true iff this and d map each point in their common
     * domain to the same place.
     */
    public boolean equal( distribution( region ) d ) {
        return this.subDistribution(region)(d)
            && d.subDistribution(region)(this);
    }

    /** Returns the unique 1-dimensional distribution U over the region 1..k,
     * (where k is the cardinality of Q) which maps the point [i] to the
     * i'th element in Q in canonical place-order.
     */
    extern public static distribution(:dimension=1) unique( Set<place> Q );

    /** Returns the constant distribution which maps every point in its
        region to the given place P.
    */
    extern public static (region R) distribution(R) constant( place P );

    /** Returns the block distribution over the given region, and over
     * place.MAX_PLACES places.
     */
    public static (region R) distribution(R) block() {
        return this.block(R)(place.places);
    }

    /** Returns the block distribution over the given region and the
     * given set of places. Chunks of the region are distributed over
     * s, in canonical order.
     */
    extern public static (region R) distribution(R) block( Set<place> s);


    /** Returns the cyclic distribution over the given region, and over
     * all places.
     */
    public static (region R) distribution(R) cyclic() {
        return this.cyclic(R)(place.places);
    }

    extern public static (region R) distribution(R) cyclic( Set<place> s);

    /** Returns the block-cyclic distribution over the given region, and over
     * place.MAX_PLACES places. Exception thrown if blockSize < 1.
     */
    extern public static (region R)
        distribution(R) blockCyclic( int blockSize)
        throws MalformedRegionException;

    /** Returns a distribution which assigns a random place in the
     * given set of places to each point in the region.
     */
    extern public static (region R) distribution(R) random();

    /** Returns a distribution which assigns some arbitrary place in
     * the given set of places to each point in the region. There are
     * no guarantees on this assignment, e.g. all points may be
     * assigned to the same place.
     */
    extern public static (region R) distribution(R) arbitrary();

}
\end{verbatim}}

\subsection{Arrays}
Finally we can now define arrays. An array is built over a
distribution and a base type.

{\footnotesize
\begin{verbatim}
package x10.lang;

/** The class of all  multidimensional, distributed arrays in X10.

    <p> I dont yet know how to handle B@current base type for the
    array.

 * @author vj 12/24/2004
 */

public final value class array ( distribution dist )<B@P> {
    parameter int dimension = dist.dimension;
    parameter region(dimension) region = dist.region;

    /** Return an array initialized with the given function which
        maps each point in region to a value in B.
     */
    extern public array( Fun<point(region),B@P> init);

    /** Return the value of the array at the given point in the
     * region.
     */
    extern public B@P valueAt(point(region) p);

    /** Return the value obtained by reducing the given array with the
        function fun, which is assumed to be associative and
        commutative. unit should satisfy fun(unit,x)=x=fun(x,unit).
     */
    extern public B reduce(Fun<B@?,Fun<B@?,B@?>> fun, B@? unit);


    /** Return an array of B with the same distribution as this, by
        scanning this with the function fun, and unit unit.
     */
    extern public array(dist)<B> scan(Fun<B@?,Fun<B@?,B@?>> fun, B@? unit);

    /** Return an array of B@P defined on the intersection of the
        region underlying the array and the parameter region R.
     */
    extern public (region(dimension) R)
        array(dist.restriction(R)())<B@P>  restriction();

    /** Return an array of B@P defined on the intersection of
        the region underlying this and the parametric distribution.
     */
    public  (distribution(:dimension=this.dimension) D)
        array(dist.restriction(D.region)())<B@P> restriction();

    /** Take as parameter a distribution D of the same dimension as *
     * this, and defined over a disjoint region. Take as argument an *
     * array other over D. Return an array whose distribution is the
     * union of this and D and which takes on the value
     * this.atValue(p) for p in this.region and other.atValue(p) for p
     * in other.region.
     */
    extern public (distribution(:region.disjoint(this.region) &&
                                dimension=this.dimension) D)
        array(dist.union(D))<B@P> compose( array(D)<B@P> other);

    /** Return the array obtained by overlaying this array on top of
        other. The method takes as parameter a distribution D over the
        same dimension. It returns an array over the distribution
        dist.asymmetricUnion(D).
     */
    extern public (distribution(:dimension=this.dimension) D)
        array(dist.asymmetricUnion(D))<B@P> overlay( array(D)<B@P> other);

    extern public array<B> overlay(array<B> other);

    /** Assume given an array a over distribution dist, but with
     * basetype C@P. Assume given a function f: B@P -> C@P -> D@P.
     * Return an array with distribution dist over the type D@P
     * containing fun(this.atValue(p),a.atValue(p)) for each p in
     * dist.region.
     */
    extern public <C@P, D>
        array(dist)<D@P> lift(Fun<B@P, Fun<C@P, D@P>> fun, array(dist)<C@P> a);

    /**  Return an array of B with distribution d initialized
         with the value b at every point in d.
     */
    extern public static (distribution D) <B@P> array(D)<B@P> constant(B@? b);

}
\end{verbatim}}


\begin{example}
 The code for {\tt List} translates as given in Table~\ref{List-translation}.
\end{example}

\begin{figure*}
{\footnotesize
\begin{verbatim}
  public value class List <Node> {
    public final nat n;   // is a parameter
    nullable Node node = null;
    nullable List<Node> rest = null;  // All assignments must check n = this.n-1.

    /** Returns the empty list. Defined only when the parameter n
        has the value 0. Invocation: new List(0)<Node>().
     */
    public List ( final nat n ) {
      assume n==0;
      this.n = n;
    }

    /** Returns a list of length 1 containing the given node.
        Invocation: new List(1)<Node>( node ).
     */
    public List ( final nat n, Node node ) {
      assume n==1;                         // From the constructor precondition.
      assert 0==0 : "DependentTypeError"; // For the constructor call.
      assert n>=1 : "DependentTypeError"; // For the this call.
      this(n, node, new List<Node>(0));
    }

    public List ( final nat n, Node node, List<Node> rest ) {
      assume n>=1;                               // From the constructor precondition
      assume rest.n==n-1 : "DependentTypeError"; // From the argument type.
      this.n = n;
      this.node = node;
      assert rest.n==n-1 : "DependentTypeError"; // For the field assignment.
      this.rest = rest;
    }

    public  List<Node> append( List<Node> arg ) {
      if (n == 0) {
          final List<Node> result = arg;
          assert n+arg.n == result.n : "DependentTypeError"; // For the return value
          return result;
      } else {
          assume rest.n == n-1;
          final List<Node> argval = rest.append(arg);
          assume argval.n == rest.n+arg.n;
          assert n+arg.n-1== argval.n : "DependentTypeError"; // For the constructor call.
          final List<Node> result = new List<Node>(n+arg.n, node, argval);
          assume result.n == n+arg.n;
          assert n+arg.n == result.n : "DependentTypeError"; // For the return value
          return result;
      }
    }

\end{verbatim}}
\caption{Translation of {\tt List} (contd in Table~\ref{List-translation-2}).}\label{List-translation}
\end{figure*}
\begin{figure*}
{\footnotesize
\begin{verbatim}
    public  List<Node> rev() {
      final List<Node> arg = new List<Node>(0);
      assume arg.n = 0;                           // From the constructor call.
      final List<Node> result = rev( arg );
      assume result.n == n+arg.n;                  // From the method signature
      assert n == result.n : "DependentTypeError"; // For the return value.
      return result;
    }

    public  List(n+arg.n)<Node> rev( final List<Node> arg) {
      if (n==0) {
         assert n+arg.n == arg.n : "DependentTypeError"; // For the return value.
         return arg;
      } else {
        assert 1+arg.n-1=arg.n : "DependentTypeError"; // For the argument to the constructor
        final List<Node> arg2 = new List<Node>(1+arg.n,node, arg));
        assume arg2.n==1+arg.n;                      // From the constructor invocation
        final List<Node> restval = rest;             // Read from a mutable field of parametric type
        assume restval.n == n-1;                     // From the field read.
        final List(restval.n+arg2.n)<Node> result = restval.rev( arg2 );
        assume result.n=restval.n+arg2.n
        assert n+arg.n == result.n                   // For the return value
        return result;
    }

    /** Return a list of compile-time unknown length, obtained by filtering
        this with f. */
    public List<Node> filter(fun<Node, boolean> f) {
         if (n==0) return this;
         if (f(node)) {
           final List<Node> l = rest.filter(f);
           assert l.n+1-1==l.n : "DependentTypeError"; // For the constructor call
           return new List<Node>(l.n+1,node, l);
         } else {
           return rest.filter(f);
         }
    }

    /** Return a list of m numbers from o..m-1. */
    public static  List<nat> gen( final nat m ) {
         assert 0 <= m : "DependentTypeError";        // Precondition for method call.
         final List<nat> result = gen(0,m);
         assume result.n=m-0 : "DependentTypeError";  // From the method signature
         assert m == result.n : "DependentTypeError"; // For the return value
         return result;
    }

    /** Return a list of (m-i) elements, from i to m-1. */
    public static List<nat> gen(final nat i, final nat m) {
      assume i <= m;                                   // Method precondition.
      if (i==m) {
        assert m-i == 0 : "DependentTypeError";        // For the constructor call
        final List result = new List<nat>(m-i);
        assume result.n == 0;                          // From the constructor call.
        assert m-i == result.n : "DependentTypeError"; // For the return value.
        return result;
      } else {
        assert i+1 <= m : "DependentTypeError";        // For the method call.
        final List<nat> arg = gen(i+1,m);
        assume arg.n = m-(i+1);                        // From the method call.
        assert m-i-1 = arg.n;                          // For the constructor invocation.
        final List result = new List<nat>(m-i, i, arg);
        assume result.n = m-i;                         // From the constructor invocation.
        assert m-i == result.n : "DependentTypeError"; // For the return value
        return result;
    }
  }
\end{verbatim}}
\caption{Translation of {\tt List} (continued).}\label{List-translation-2}
\end{figure*}

\section{Type-checking dependent classes}

Each programming language---such as \Xten{}---will specify the base
underlying classes (and the operations on them) which can occur as
types in parameter lists. For instance, in the code for {\tt List}
above, the only type that appears in parameter lists is {\tt int}, and
the only operations on {\tt int} are addition, subtraction, {\tt >=},
{\tt ==}, and the only constants are {\tt 0} and {\tt 1}.  (This
language falls within Presburger arithmetic, a decidable fragment of
arithmetic.)  The compiler must come equipped with a constraint solver
(decision procedure) that can answer questions of the form: does one
constraint entail another?  Constraints are atomic formulas built up
from these operations, using variables. For instance, the compiler
must answer each one of:
{\footnotesize
\begin{verbatim}
  n >= 2 |- n-1 >= 0
  n >= 0, m >= 0 |- m+n >= 0
\end{verbatim}}

Ultimately, the only variables that will occur in constraints are
those that correspond to {\tt config} parameters and those that are
defined by implicit parameter definitions. We need to establish that
the verification of any class will generate only a finite number of
constraints, hence only a finite constraint problem for the constraint
solver.

Second, it should be possible for instances of user-defined classes
(and operations on them) to occur as type parameters. For the compiler
to check conditions involving such values, it is necessary that the
underlying constraint solver be extended.

There are two general ways in which the constraint solver may be
extended.  Both require that the programmer single out some classes
and methods on those classes as {\em pure}. (We shall think of
constants as corresponding to zero-ary methods.) Only instances of
pure classes and expressions involving pure methods on these instances
are allowed in parameter expressions.

How shall constraints be generated for such pure methods? First, the
programmer may explicitly supply with each pure method {\tt T m(T1 x1,
..., Tn xn)} a constraint on {\tt n+2} variables in the constraint
system of the underlying solver that is entailed by {\tt y =
o.m(x1,..., xn)}. Whenever the compiler has to perform reasoning on an
expression involving this method invocation, it uses the constraint
supplied by the programmer. A second more ambitious possibility is
that a symbolic evaluator of the language may be run on the body of
the method to automatically generate the corresponding constraint.

Finally an additional possibility is that the constraint solver itself
be made extensible. In this case, when a user writes a class which is
intended to be used in specifying parameters, he also supplies an
additional program which is used to extend the underlying constraint
solver used by the compiler. This program adds more primitive
constraints and knows how to perform reasoning using these
constraints. This is how I expect we will initially implement the
\Xten{} language. As language designers and implementers we will
provide constraint solvers for finite functions and {\tt Herbrand}
terms on top of arithmetic.





\end{document}
