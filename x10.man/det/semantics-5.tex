%% Includes treatment of clocked variables.
Featherweight X10 (FX10) is a formal calculus for X10 intended to  complement Featherweight Java
(FJ).  It models imperative aspects of X10 including the concurrency
constructs \hfinish{}, \hasync{}, clocks and accumulators.

\paragraph{Overview of formalism}
We present the reduction rules for two schedulers:
    a concurrent and a sequential scheduler.
The sequential scheduler is obtained from the concurrent one by removing two rules.
Intuitively, the sequential scheduler performs a DFS over \code{async}.
%    it executes an activity until it is either terminated, or waits, or it creates a new activity.
%When all clocked activities inside a clocked \finish are ready to advance,
We prove that the concurrent and sequential schedulers always have the same result (i.e., isomorphic heaps).
\Ref{Section}{syntax} presents FX10 syntax,
    \Ref{Section}{reduction} the reduction rules,
    and \Ref{Section}{results} our main result that programs in FX10 are always safe.

\Subsection[syntax]{Syntax}
There are three predefined classes in FX10:
    \code{Object},
    \code{Acc} (representing accumulators),
    and \code{ClockedAcc} (representing clocked accumulators).
X10 has an effect system that guarantees that
    \hasync{} only accesses immutable state, clocked values and accumulators.
The effect system also ensures that clocked values can only be assigned once every phase,
    however clocked accumulators can be assigned multiple times.
FX10 only models the runtime behavior without the effect system.
Therefore, in order to guarantee determinacy without an effect system,
    FX10 does not model field assignment nor clocked values (only clocked accumulators).
Objects are still mutable because accumulators and clocked accumulators
    are mutable.

%todo: talk about stuck configurations.
\begin{figure}[t]
\begin{center}
\begin{tabular}{|l|l|}
\hline

$\hP ::= \ol{\hL},\hS$ & Program. \\

$\hL ::= \hclass ~ \hC~\hextends~\hD~\lb~\ol{\hF};~\ol{\hM}~\rb$
& cLass declaration. \\

$\hF ::= \hvar\,\hf:\hC$
& Field declaration. \\

$\hM ::= \hdef\ \hm(\ol{\hx}:\ol{\hC}):\hC\{\hS\}$
& Method declaration. \\

$\hp ::= \hl ~~|~~ \hx$
& Path. \\ %(location or parameter)

$\he ::=  \hp.\hf  ~|~ \hnew{\hC}(\ol{\hp}) ~|~ \hp()$
& Expressions. \\ %: locations, parameters, field access,  %invocation, \code{new}
$\hB ::= \blockt{\ol{\hl}}{\hS}$
& Blocks. \\
$\hS ::=  \epsilon ~|~ \hadvance; ~|~ \hp.\hm(\ol{\hp}); ~|~$ &  \\
$~~~~~ \valt{\hx}{\he} ~|~ \hp \leftarrow \hp ~|~ \hS~\hS ~|~  $ &\\
$~~~~~ \pclocked~\finishasync{\hB} $ & Statements. \\ %: locations, parameters, field access, invocation, \code{new}

\hline
\end{tabular}
\end{center}
FX10 Syntax.
    The terminals are locations (\hl,\ha), parameters and \hthis (\hx), field name (\hf), method name (\hm), class name (\hB,\hC,\hD,\hObject),
        and keywords. % (\hhnew, \hfinish, \hasync, \code{val}, \hadvance)
    %The non-terminals are statements (\hS), blocks (\hB),
    The program source code cannot contain locations (\hl), because locations are only created during execution/reduction in \RULE{R-New} of \Ref{Figure}{reduction}.

\label{Figure:syntax}
\end{figure}

\Ref{Figure}{syntax} shows the abstract syntax of FX10.
A constructor in FX10 assigns its arguments to the object's fields.

$\epsilon$ is the empty statement. Sequencing associates to the left.
%\code{async}
%and \code{finish} bind tighter than statement sequencing, so
%$\code{async}~\hS~\hS'$ is the sequencing of $\code{async}~\hS$ and
%$\hS'$.
The block syntax in source code is just \code{\{\hS{}\}},
however at run-time the block is augmented with (the initially empty) set of locations
$\pi$ representing the objects registered on the block,
     which are those that were created in that block.

%(\Ref{Section}{val} will later add the \hval and \hvar field modifiers.)
Statement~$\hval{\hx}{\he}{\hS}$ evaluates $\he$, assigns it to a
new variable $\hx$, and then evaluates \hS. The scope of \hx{} is \hS.

Statement $\hp_1 \leftarrow \hp_2$ is legal only when $\hp_1$ is an accumulator,
    and it accumulates $\hp_2$ into $\hp_1$ (given the reduction operator that was supplied when the accumulator was created).
Expression $\hp()$ is legal only when $\hp$ is an accumulator;
    when the accumulator is clocked it immediately returns the old value of the accumulator,
    and when it is not clocked it waits until all activities created in the block have terminated or are ready to advance
    (then it is a safe point to read the accumulator because it cannot be further mutated).
% it checks at runtime that it was executed in the block that created $\hp$


\Subsection[reduction]{Reduction}
A {\em heap}~$\cH$ is a mapping from a given set of locations to {\em
  objects}. An object is a pair $\hC(\ol{\hl})$ where $\hC$ is a class (the exact
class of the object), and $\ol{\hl}$ are the values in the fields.

An {\em configuration} is of the form
$\hS,\cH$ where \hS{} is a statement and $\cH$ is a heap (representing a
computation which is to execute $\hS$ in the heap $\cH$), or $\cH$
(representing a successfully terminated computation).
The reduction rules define transitions between configurations.
We define both a concurrent scheduler, and by removing two rules we also define a sequential scheduler.

The reduction relation $\preduce{}$ is described in
    \Ref{Figure}{reduction}.
Here $\pi$ is a
set of locations which can currently be asynchronously accessed.  Thus
each transition is performed in a context that knows about the current
set of capabilities.

\Ref{Figure}{reduction} also defines the \emph{advance relation} $\hS \ureduce{c,a} \hS'$
    that is used to advance a statement to its next phase.
The integer $c$ represent how many \code{clocked async} can advance to the next phase in \hS,
    and the integer $a$ represent how many non-nested \code{async} are in \hS.
For example, the statement
\begin{lstlisting}
async { S1 }
clocked async {
  clocked async {}
  async { S2 }
  clocked async { advance; S3 }
  advance; S4
}
clocked async { advance; S5 }
\end{lstlisting}
can be advanced to (where $c=3$ and $a=2$)
\begin{lstlisting}
async { S1 }
clocked async {
  clocked async {}
  async { S2 }
  clocked async { S3 }
  S4
}
clocked async { S5 }
\end{lstlisting}
We write  $\hS \ureduce{c,a} \hUnused$ if the result of advancing \hS is not used in the rule
    (i.e., \hUnused is the statement after advancing \hS, but it is unused.)
Finally, we write $\hS \ureduce{c+,a+} \hS'$ if $\hS \ureduce{c',a'} \hS'$ and $c'\geq c$ and $a' \geq a$.
For example, \RULE{(R-Seq)+} uses $\hS \ureduce{1+,0} \hUnused$,
    meaning that \hS has at least one \code{clocked async} waiting on \xadvance
    and it does not have any non-clocked \code{async}.
(This rule is used for the sequential scheduler, to allow it to progress across stuck activities.
    The concurrent scheduler can just use \RULE{(R-Async)} instead.)
The advance relation is also used in the following places:
(i)~$\hS_1 \ureduce{0+,0+} \hUnused$ is used in \RULE{(R-Trans-B)}, \RULE{(R-Adv)}, \RULE{(R-New)}
    to give context (it ensures that we can execute the statement after $\hS_1$).
(ii)~$\hS_1 \ureduce{0+,0} \hUnused$ in \RULE{(R-Acc-R)} ensures that $\hS_1$ cannot progress
    and therefore it is a safe point to read the accumulator (because no other activity can mutate the accumulator).
(iii)~$\ureduce{1+,0+}$ in \RULE{(R-Adv)} guarantees that we can advance to the next phase
    (there must be at least one \xadvance that is removed).


For $X$ a partial function, we use the notation
$X[v \mapsto e]$ to represent the partial function which is the same
as $X$ except that it maps $v$ to $e$.

We now explain the rules in detail.
The rules for \RULE{(R-Epsilon)}, \RULE{(R-Trans)}, \RULE{(R-Access)}, and \RULE{(R-Invoke)}
    are standard.

One minor novelty is in how \hasync{} is
defined. The critical rule is the last rule in~\RULE{(R-Async)-} -- it
specifies the ``asynchronous'' nature of \hasync{} by permitting \hS{}
to make a step even if it is preceded by $\async{\hS_1}$.
For the sequential scheduler, we define \RULE{(R-Seq)+}
    that permits \hS{}
to make a step only if it is preceded by $\async{\hS_1}$ that cannot progress
    (it has at least one \code{clocked async} waiting on \xadvance and no un-clocked \async).

Further,
each block $\blockt{\ol{\hl}}{\hS}$ records the set of newly created objects~$\ol{\hl}$, which we say have registered on the block.
When descending into the body in \RULE{(R-Trans)}, the newly registered objects are added to
    those obtained from the environment.
When descending into an \async or \finish in \RULE{(R-Trans-B)}, we need to take special care of the clocked
    accumulators (because they cannot be used in a non-clocked construct).
Given $\ol{\hl}$ and a heap $\cH$,
    we define
    $\hAcc(\ol{\hl})$ to return only the non-clocked accumulators in~$\ol{\hl}$
    (dropping all the other objects).
According to \RULE{(R-Trans-B)}, when descending into a non-clocked
    \async or \finish, we keep only the non-clocked accumulators in $\pi$
    (i.e., the clocked accumulators cannot be accessed).
When descending into a clocked \async we keep $\pi$ unchanged.
Finally, the most complicated case is descending into a clocked \finish:
    then we keep all the non-clocked accumulators and also the newly created clocked-accumulators in the enclosing block
    (these clocked accumulators will be advanced in rule \RULE{(R-Adv)} later).

The rule~\RULE{(R-New)} returns a new location that is bound to a new
    object that is an instance of \hC{} with its fields set to the argument;
    the new object is registered with the block.
(It is enough to register only clocked-values and accumulators, but
    for simplicity we register all newly created objects.)


%todo: give an example of registration and legality.


The rule~\RULE{(R-Acc-W)} updates the current contents of the
accumulator provided that the current set of capabilities permit
asynchronous access to the accumulator.
Rule~\RULE{(R-Clocked-W)} is similar.

The rule~\RULE{(R-Acc-R)} permits the accumulator to be read in a
block provided that the current set of capabilities permit it (if not,
an error is thrown), and provided that the only statements prior to
the read are re is no nested async prior to the read are clocked
asyncs that are stuck at an \hadvance.
Rule~\RULE{(R-Clocked-R)} immediately returns the old value of the accumulator (without waiting like in \RULE{(R-Acc-R)}).

The rule~\RULE{(R-Adv)} permits a \code{clocked finish} to advance
only if all the top-level \code{clocked asyncs} in the scope of the
\code{clocked finish} are stuck at an
\code{advance}.
Note that an un-clocked \code{aync} may exist in the
scope of the \code{clocked finish}; they do not come in the way of the
    \code{clocked finish} advancing.
The resulting heap~$\cH'=\hswitch(\cH,\ol{\hl})$ is obtained from $\cH$ by switching
    the current and next values in all clocked accumulators in $\ol{\hl}$.


\begin{figure*}[t]

\begin{center}
\begin{tabular}{|c|}
\hline
$\typerule{
}{
 \epsilon,\cH \preduce \cH
}$~\RULE{(R-Epsilon)}
~
$\typerule{
 \hS,\cH \preduce \hS',\cH' ~|~ \cH'
}{
  \begin{array}{rcl}
    \blockt{\ol{\hl}}{\hS},\cH&\ptreduce &\blockt{\ol{\hl}}{\hS'},\cH'
    ~|~ \cH' \gap \pi=\pi',\ol{\hl} \\
    \hS~\hS_1, \cH &\preduce& \hS'~\hS_1,\cH' ~|~ \hS_1,\cH' \\
  \end{array}
}$~\RULE{(R-Trans)}
~
$\typerule{
\hS_1\ureduce{1+,0}\hUnused \gap\gap \hS,\cH \preduce \hS',\cH'
}{
   \hS_1~\hS,\cH \preduce \hS_1~\hS',\cH'
}$~\RULE{(R-Seq)+}
\\\\
$\typerule{
 \hB,\cH \preduce \hB',\cH' ~|~ \cH'
}{
    \finishasync{\hB},\cH \ptreduce  \finishasync{\hB'},\cH'  ~|~ \cH'  \gap \pi=\hAcc(\pi')\\
    %\finish{}{\hB},\cH \ptreduce \finish{}{\hB'},\cH'  ~|~ \cH'  \gap \pi=\hAcc(\pi')\\
    \vspace{-0.5em}
    \hclocked~\async{\hB},\cH \preduce \hclocked~\async{\hB'},\cH'  ~|~ \cH'  \\
   \blockt{\ol{\hl}}{\hS_1~\hclocked~\finish{\hB}~\hS},\cH \ptreduce
  \blockt{\ol{\hl}}{\hS_1~\hclocked~\finish{\hB'}~\hS},\cH' ~|~ \blockt{\ol{\hl}}{\hS_1~\hS},\cH' \gap \pi=\hAcc(\pi') \cup \ol{\hl} \gap \hS_1\ureduce{0+,0+}\hUnused
}$~\RULE{(R-Trans-B)}
\\\\

$\typerule{
    \cH(\hl)=\hC(\ol{\hl'})
}{
  \hval{\hx}{\hl.\hf_i}{\hS},\cH \preduce \hS[\hl_i'/\hx],\cH
}$~\RULE{(R-Access)}
\gap
$\typerule{
    \cH(\hl')=\hC(\ldots)
        \gap
    \mbody{}(\hm,\hC)=\ol{\hx}.\hS
}{
  \hl'.\hm(\ol{\hl}),\cH \preduce \hS[\ol{\hl}/\ol{\hx},\hl'/\hthis],\cH
}$~\RULE{(R-Invoke)}
\\\\
$\typerule{
  \cH(\ha)=\hClockedAcc(\hr,\ho,\he) \gap \ha\in\pi
}{
  \hval{\hx}{\ha()}{\hS},\cH \preduce \hS[\ho/\hx], \cH \\
}$~\RULE{(R-Clocked-R)}
\gap
$\typerule{
    \cH(\ha)=\hClockedAcc(\hr,\ho,\hv) \gap \hw=\hr(\hv,\hl) \gap \hl\in \pi
}{
  \ha\leftarrow\hl,\cH \preduce \cH[ \hl \mapsto \hClockedAcc(\hr,\ho,\hw)]
}$~\RULE{(R-Clocked-W)}
\\\\
$\typerule{
\hS_1\ureduce{0+,0}\hUnused \gap\cH(\ha)=\hAcc(\hr,\hv) \gap  \gap \ha\in \ol{\hl}
}{
\blockt{\ol{\hl}}{\hS_1~\hval{\hx}{\ha()}~\hS},\cH \preduce
\blockt{\ol{\hl}}{\hS_1~\hS[\hv/\hx]}, \cH'
}$~\RULE{(R-Acc-R)}
~
$\typerule{
  \cH(\ha)=\hAcc(\hr,\hv)\gap \hw=\hr(\hv,\hl)  \gap \ha\in \pi
}{
  \ha \leftarrow \hl,\cH \preduce \cH[\ha \mapsto \hAcc(\hr,\hw)]
}$~\RULE{(R-Acc-W)}
\\\\
$\typerule{
    \hS_1\ureduce{0+,0+}\hUnused    \gap\hl' \not \in \dom(\cH)
}{
\blockt{\ol{\ha}}{\hS_1~\valt{\hx}{\hnew{\hC(\ol{\hl})}}~\hS},\cH \preduce
   \blockt{\ol{\ha},\hl}{\hS_1~\hS[\hl'/\hx]},\cH[ \hl' \mapsto \hC(\ol{\hl})] \\
%\blockt{\pi_1}{\acc{\hx}{\hnew{\hAcc(\hr,\hz)}}},\cH \preduce
%   \cH[ \hl \mapsto \hAcc(\hr,\hz)]
}$~\RULE{(R-New)}
\\\\

$\typerule{
}{
    \epsilon \ureduce{0,0} \epsilon
}$~\RULE{(R-Adv-Epsilon)}
\gap

$\typerule{
    \hS_1 \ureduce{c_1,a_1} \hS_1' \gap     \hS_2 \ureduce{c_2,a_2} \hS_2'
}{
  \hS_1~\hS_2\ureduce{c_1+c_2,a_1+a_2} \hS_1'~\hS_2'
}$~\RULE{(R-Adv-S)}
\\\\

$\typerule{
  \hS_1 \ureduce{c,a} \hS'_1
}{
    \hclocked~\async~\blockt{\ol{\hl}}{\hS_1} \ureduce{c,a} \hclocked~\async~\blockt{\ol{\hl}}{\hS'_1} \\
    \hclocked~\async~\blockt{\ol{\hl}}{\hS_1~\xadvance~\hS_2} \ureduce{c+1,a} \hclocked~\async~\blockt{\ol{\hl}}{\hS'_1~\xadvance~\hS_2}
}$~\RULE{(R-Adv-C-A)}
\\\\


$\typerule{
    \hS_1 \ureduce{0+,0+}\hUnused \gap \hclocked~\async~\hB \ureduce{1+,0+} \hclocked~\async~\hB' \gap \cH'=\hswitch(\cH,\ol{\hl})
}{
  \blockt{\ol{\hl}}{\hS_1~\hclocked~\finish{\hB}~\hS"},\cH\preduce
  \blockt{\ol{\hl}}{\hS_1~\hclocked~\finish{\hB'}~\hS"},\cH'
}$~\RULE{(R-Adv)}
\\\\
\hline

$\typerule{
}{
  \async{\hB} \ureduce{0,1} \async{\hB}
}$~\RULE{(R-Adv-A)-}
\gap
$\typerule{
  \hS,\cH \preduce \hS', \cH'
}{
  \pclocked~\async{\hB}~\hS, \cH \preduce \pclocked~\async{\hB}~\hS', \cH'\\
}$~\RULE{(R-Async)-}
\\
\hline
\end{tabular}
\end{center}


\caption{FX10 Reduction Rules ($\hS,\cH \preduce \hS',\cH' ~|~\cH'$) for the \emph{concurrent} scheduler
    (\RULE{(R-Seq)+} is not mandatory in the concurrent scheduler; it was added so that the sequential scheduler will progress if an \hasync cannot advance.)
    The \emph{sequential} scheduler is obtained by removing \RULE{(R-Adv-A)-} and \RULE{(R-Async)-}.
    }
\label{Figure:reduction}
\end{figure*}


\Subsection[results]{Results}
Our main result is that all programs in FX10 are safe, i.e.,
    all schedulers are deadlock-free and result in the same heap (up to isomorphism of locations).

\begin{Theorem}
Given a well-typed program $\ol{\hL},\hS$ in heap $\cH$,
    if $\hS,\cH \stackrel{*}\reduce \cH'$ and $\hS,\cH \stackrel{*}\reduce \cH"$
    then $\cH'$ and $\cH"$ are isomorphic.
\end{Theorem}
