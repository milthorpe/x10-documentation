%\title{Sequential semantics for concurrent programs}


Imperative memory locations are famously indeterminate under
concurrent read/write operations. Further, to support reliable
communication most concurrent programming languages support
synchronization mechanisms that enable a thread to wait until some
appropriate condition is true, thus leading to the possibility of
deadlock. Sequential programs are fundamentally determinate and
deadlock-free; thus it is very difficult to systematically write
concurrent programs with sequential semantics, even if the program is
in fact a parallelization of a sequential program. 

In this paper we develop a fragment of the programming language
\Xten{} in which all programs have sequential semantics.  \Xten{} is
based on a very powerful imperative calculus supporting fine-grained
asynchronous concurrency, multi-place execution, messaging and clocked
synchronization. We introduce two new abstractions -- {\em
  accumulators} and {\em clocked values} -- with lightweight compiler
and run-time support. Accumulators of type \code{T} permit multiple
concurrent writes, these are reduced into a single value by a
user-specified reduction operator. Clocked values of type \code{T}
operate on two values of \code{T} (the {\em current} and the {\em
  next}). Read operations are performed on the current value and write
operations on the next. Such values are implicitly associated with a
clock and the current and next values are switched (determinately) on
quiescence of the clock. Clocked values capture the common ``double
buffering'' or ``red/black'' concurrency idiom.

We introduce
the notion 

In this paper we present some lightweight techniques that enable a
large class of concurrent programs to be written in such a way that
their semantics 

The former can lead to run-time overheads and the latter to brittle
programs since it is very difficult to smoothly extend the rich type
systems underlying modern imperative languages (supporting objects,
inheritance, subtyping, genericity etc) to support determinacy. The
fundamental problem is the possibility of aliasing in arbitrary
unstructured heaps. This makes it very difficult to get a static
handle on concurrent access to a shared location.

In this paper we show that a middle path can be successful for a
language based on structured concurrency, such as \Xten.  We introduce
two new abstractions -- 


We show that these abstractions (by design) determinate and
deadlock-free in any usage, though executions may throw exceptions
under certain circumstances. We show that these data-structures are
very natural to use and successfully capture many common patterns of
expression that are semantically determinate (e.g. histograms,
all-to-all reductions, stencil computations etc). We show that there
are simple statically-checkable rules that can establish for many
common idioms that concurrency-related exceptions will not be thrown
at run-time, and that some potentially costly synchronization checks
can be avoided.

A key technical innovation is the introduction of an {\em implicit
  ownership domain} for objects. This provides a way around the
unstructured heap by permitting only the current activity and its
spawned asyncs to access the objects. Now the block structure of
\code{finish}, \code{async}, \code{at} and the clock construct of
\Xten{} can be used to establish determinacy in a local way,
independent of the context in which the code is being used. 

