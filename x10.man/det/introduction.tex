A serial schedule for a parallel program is one which always executes
the first enabled step in program order. A {\em safe} parallel program
is one that can be executed with a serial schedule $S$ and for which
every schedule produces the same result as $S$.  Such a program is
semantically a sequential program, hence it is scheduler-determinate
and deadlock-free.

A {\em safe programming language} is an imperative parallel
programming language in which every legal program is a safe
program. Programmers can write code in such a language secure in the
knowledge that they will not encounter a large class of parallel
programming problems. Such a language is particularly useful for
parallelizing sequential (imperative) programs. In such cases ({\em
  contra} reactive programming) the desired application semantics are
sequential, and parallelism is needed purely for efficient
implementation.

The key property of a safe programming language is that the {\em same}
program can be developed and debugged as a sequential program and then
safely run in parallel. Parallel execution is guaranteed to effect
only performance, not correctness. Safety is a very strong property.

Figure~\ref{fig:1} shows  the famous ``parallel Or'' program of Plotkin 
 (in X10 syntax, \cite{x10}). This program can be executed with a
depth-first schedule, is partially determinate and deadlock-free, but {\em not}
safe. The result of running the sequential schedule is not the same as
the result that can be obtained with other schedules. Specifically
\code{parallelOr(()=> CONT, ()=>TRUE)} will diverge (exhibit an
infinite exection sequence) under the depth-first schedule, but will
return \code{true} under any fair schedule that permits the second
async to progress.

\begin{figure}
  \begin{lstlisting}
static val CONT=1, TRUE=2, FALSE=3;
def run(done:Cell[Boolean], a:()=>Int) {
 var aa:Int=a();
 var cont:Boolean=true;         
   for (; aa==CONT && cont;aa=a()) {
      atomic cont = !done();  
   }
   if (aa==TRUE)
     atomic done()=true;
}
def parallelOr(a:()=>Int, b:()=>Int):Boolean {
  val done=new Cell[Boolean](false);
  finish {
    async run(done, a);
    async run(done, b);
  }
  return done();
}
  \end{lstlisting}
  \caption{A program that is not sequential}\label{fig:1}
\end{figure}
In this paper we establish that a very rich fragment of X10 is safe.
The fragment supports multi-place, fine-grained concurrency
over an arbitrary heap, clocked computations, concurrent accumulators,
and {\em clocked types} that safely capture the ``red black'' idiom
for iterative computations. It also uses a very lightweight effects
mechanism to reason about disjointness of acccess. We show that a
large variety of concurrency and communication idioms can be naturally
expressed in Safe X10.

Determinism in parallel programming is a very active area of research.
Guava \cite{guava} introduces restrictions on shared memory Java
programs that ensure no data-races primarily by distinguishing
monitors (all access is synchronized) from values (immutable) and
objects (private to a thread). However Guava is not safe since Guava
programs may use \code{wait}/\code{notify} for arbitrary concurrent
signalling and hence may not executable with a sequential
schedule. The Revisions programming model \cite{Revisions} guarantees
determinism by isolating asynchronous tasks but merging their writes
determinately. However, the model explicitly does not require that
a sequential schedule be valid (c.f. Figure~1 in \cite{Revisions}).

DPJ develops the ``determinacy-by-default'' slogan using a static
type-and-effects system to establish commutativity of concurrent
actions.  The deterministic fragment of DPJ is safe according to the
definition above. Safe X10 offers a much richer concurrency model
which guarantees the safety of common idioms such as accumulators and
cyclic tasks (clocks) without relying on effects annotations. The
lightweight effects mechanism in X10 can be extended to support a much
richer effects framework (along the lines of DPJ) using X10's
constrained type system.  We leave this as future work.

The SafeJava language \cite{SafeJava} is unfortunately not safe
according to our definition, even though it guarantees determinacy and
deadlock freedom, using ownership types, unique pointers and partially
ordered lock levels. Again, a sequential scheduler is not admissible
for the model.

Some data-flow synchronization based languages and frameworks (e.g.{}
Kahn style process networks \cite{kahn,kahn-mcqueen}, concurrent
constraint programming \cite{ccp}, \cite{SHIM}) are guaranteed
determinate but not safe according to our definition since they do not
permit sequential schedules. Indeed they permit the possibility of
deadlock. (The notion of safety is also not quite relevant since these
frameworks do not support shared mutable variables.)

Determinizing run-times support coarse-grained fork-join concurrency
by maintaining a different copy of memory for each activity and
merging them determinately at finish points (\cite{grace},
\cite{core-det}, \cite{dmp}, \cite{kendo},\cite{determinator}). Safe
X10 can run on such systems in principle, but does not require them.
To execute Safe X10, such systems need to support fine-grained
asynchrony (with some form of work-stealing or fork-joining
scheduler), clocks and accumulators.




%% Anything from Blelloch?

Desiderata:

\begin{itemize}
\item Data-structures should by design by dynamically determinate and
   deadlock-free.
\item They should be first-class -- they can be stored in
  data-structures/read from them, passed as arguments to
  procedures/returned etc with no restrictions. 
\item Usable -- common idioms should be naturally and elegantly expressible.
\item Additional static type-checking can provide extra guarantees
  (e.g.{} no concurrency related run-time exceptions) that may aid
  efficient implementation.
\end{itemize}

We discuss three examples

  * clocks
  * accumulators
  * clocked types

Challenge
  Arbitrary nature of object graphs.


{\em 
\begin{enumerate}
\item Use activity registration as a mechanism to tame object graphs.
\item Focus on structured concurrency. Using scoping and block-structure
    to delimit regions of code that may execute in parallel and affect
    the data structure.

\item Accumulation can be defined safely by delaying. However, the delay
    operation is guaranteed to be deadlock-free.

\item Clocked types support phased computation, another common idiom
    particularly for stencil computations.
\end{enumerate}
}

Key contributions:
{\em 
\begin{enumerate}
\item Identification of determinate, deadlock-free data-structures. 
\item Discussion of design alternatives which points out the
  difficulty of integrating these ideas in a modern OO language.
\item Discussion of various idioms expressible using these data-structures.
\item Proof of determinacy and deadlock-freedom in an abstract version
  of the language.
\end{enumerate}
These constructs are implemented in \Xten, available as open source from
SVN head and will be in the next release of \Xten.
}


Semantics and theorems for an abstract version of the language.


\subsection{Related Work}

\cite{Steele:1989:MAP:96709.96731} introduced the idea of 
