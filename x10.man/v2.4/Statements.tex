%% vj Wed Sep 18 19:27:18 EDT 2013
%% Updtaed to v 2.4
\chapter{Statements}\label{XtenStatements}\index{statement}

This chapter describes the statements in the sequential core of
\Xten{}.  Statements involving concurrency and distribution
are described in \Sref{XtenActivities}.

\section{Empty statement}

The empty statement \xcd";" does nothing.  

\begin{ex}
Sometimes, the syntax of X10 requires a statement in some position, but you do
not actually want to do any computation there.   
The following code searches the rail \xcd`a` for the value \xcd`v`, assumed
to appear somewhere in \xcd`a`, and returns the index at which it was found.  
There is no computation to do in the loop body, so we use an empty statement
there. 
%~~gen ^^^ Statements10
% package statements.emptystatement;
% class EmptyStatementExample {
%~~vis
\begin{xten}
static def search[T](a: Rail[T], v: T):Long {
  var i : Long;
  for(i = 0L; a(i) != v; i++)
     ;
  return i;
}
\end{xten}
%~~siv
%}
%~~neg

\end{ex}

\section{Local variable declaration}
\label{sect:LocalVarDecln}
\index{variable!declaration}
\index{var}
\index{val}

%##(LocVarDecln LocVarDeclnStmt VariableDeclaratorsWithType  VariableDeclarators VariableInitializer FormalDeclarators
\begin{bbgrammar}
%(FROM #(prod:LocVarDecln)#)
         LocVarDecln \: Mods\opt VarKeyword VariableDeclrs & (\ref{prod:LocVarDecln}) \\
                     \| Mods\opt VarDeclsWType \\
                     \| Mods\opt VarKeyword FormalDeclrs \\
%(FROM #(prod:LocVarDeclnStmt)#)
     LocVarDeclnStmt \: LocVarDecln \xcd";" & (\ref{prod:LocVarDeclnStmt}) \\
%(FROM #(prod:VarDeclsWType)#)
       VarDeclsWType \: VarDeclWType & (\ref{prod:VarDeclsWType}) \\
                     \| VarDeclsWType \xcd"," VarDeclWType \\
%(FROM #(prod:VariableDeclrs)#)
      VariableDeclrs \: VariableDeclr & (\ref{prod:VariableDeclrs}) \\
                     \| VariableDeclrs \xcd"," VariableDeclr \\
%(FROM #(prod:VariableInitializer)#)
 VariableInitializer \: Exp & (\ref{prod:VariableInitializer}) \\
%(FROM #(prod:FormalDeclrs)#)
        FormalDeclrs \: FormalDeclr & (\ref{prod:FormalDeclrs}) \\
                     \| FormalDeclrs \xcd"," FormalDeclr \\
\end{bbgrammar}
%##)

Short-lived variables are introduced by local variables declarations, as
described in \Sref{sect:LocalVarDecln}. Local variables may be declared only
within a block statement (\Sref{Blocks}). The scope of a local variable
declaration is the subsequent statements in the
block.   
%~~gen ^^^ Statements20
% package statements.should.have.locals;
% class LocalExample {
% def example(a:Long) {
%~~vis
\begin{xten}
  if (a > 1) {
     val b = a/2;
     var c : Long = 0;
     // b and c are defined here
  }
  // b and c are not defined here.
\end{xten}
%~~siv
%} }
%~~neg

Variables declared in such statements shadow variables of the same
name declared elsewhere.
A local variable of a given name, say \xcd`x`, cannot shadow another local
variable or parameter named \xcd`x` unless there is an intervening method,
constructor, initializer, or
closure declaration.
%%, or unless the inner \xcd`x` is declared inside an
%%\xcd`async` or \xcd`at` statement and the outer variable is declared outside
%% of that.   Strictly, \xcd`at` introduces a new scope that does not share the
%%variables of the external scope, so its variables do not actually shadow those
%%outside.  

\begin{ex}
The following code illustrates both legal and illegal uses of shadowing.
Note that a shadowed {\em field} name \xcd`x` can still be accessed 
as \xcd`this.x`. 

%%AT-COPY%% %~~gen ^^^ Statements4h6p
%%AT-COPY%% % package Statements4h6p;
%%AT-COPY%% %~~vis
%%AT-COPY%% \begin{xten}
%%AT-COPY%% class Shadow{
%%AT-COPY%%   var x : Long; 
%%AT-COPY%%   def this(x:Long) { 
%%AT-COPY%%      // Parameter can shadow field
%%AT-COPY%%      this.x = x; 
%%AT-COPY%%   }
%%AT-COPY%%   def example(y:Long) {
%%AT-COPY%%      val x = "shadows a field";
%%AT-COPY%%      // ERROR: val y = "shadows a param";
%%AT-COPY%%      val z = "local";
%%AT-COPY%%      for (a in [1,2,3]) {
%%AT-COPY%%         // ERROR: val x = "can't shadow local var";
%%AT-COPY%%      }
%%AT-COPY%%      async {
%%AT-COPY%%         val x = "can shadow through async";
%%AT-COPY%%      }        
%%AT-COPY%%      at (here;) {
%%AT-COPY%%         val x = "at gives a whole new namespace";
%%AT-COPY%%      }        
%%AT-COPY%%      val f = () => { 
%%AT-COPY%%        val x = "can shadow through closure";
%%AT-COPY%%        x
%%AT-COPY%%      };
%%AT-COPY%%   }
%%AT-COPY%% }
%%AT-COPY%% \end{xten}
%%AT-COPY%% %~~siv
%%AT-COPY%% %
%%AT-COPY%% %~~neg
%%AT-COPY%% 
%~~gen ^^^ Statements4h6p
% package Statements4h6p;
% // NOTEST-stupid-packaging-issue
%~~vis
\begin{xten}
class Shadow{
  var x : Long; 
  def this(x:Long) { 
     // Parameter can shadow field
     this.x = x; 
  }
  def example(y:Long) {
     val x = "shadows a field";
     // ERROR: val y = "shadows a param";
     val z = "local";
     for (a in [1,2,3]) {
        // ERROR: val x = "can't shadow local var";
     }
     async {
        // ERROR: val x = "can't shadow through async";
     }        
     val f = () => { 
       val x = "can shadow through closure";
       x
     };
     class Local {
        val f = at(here.next()){ val x = "can here"; x };
        def this() { val x = "can here, too"; }
     }
  }
}
\end{xten}
%~~siv
%
%~~neg



\end{ex}

\begin{ex}
Note that recursive definitions of local variables is not allowed.  There are
few useful recursive declarations of objects and structs; \xcd`x`, in the
following example, has no meaningful definition.  Recursive declarations of
local functions is forbidden, even though (like \xcd`f` below) there are
meaningful uses of it.  
\begin{xten}
val x : Long = x + 1; // ERROR: recursive local declaration
val f : (Long)=>Long 
      = (n:Long) => (n <= 2) ? 1 : f(n-1) + f(n-2);
      // ERROR: recursive local declaration
\end{xten}

\end{ex}



\section{Block statement}
\index{block}
\label{Blocks}

%##(Block BlockStatements BlockInteriorStatement
\begin{bbgrammar}
%(FROM #(prod:Block)#)
               Block \: \xcd"{" BlockStmts\opt \xcd"}" & (\ref{prod:Block}) \\
%(FROM #(prod:BlockStmts)#)
          BlockStmts \: BlockInteriorStmt & (\ref{prod:BlockStmts}) \\
                     \| BlockStmts BlockInteriorStmt \\
%(FROM #(prod:BlockInteriorStmt)#)
   BlockInteriorStmt \: LocVarDeclnStmt & (\ref{prod:BlockInteriorStmt}) \\
                     \| ClassDecln \\
                     \| StructDecln \\
                     \| TypeDefDecln \\
                     \| Stmt \\
\end{bbgrammar}
%##)


A block statement consists of a sequence of statements delimited by
``\xcd"{"'' and ``\xcd"}"''. When a block is evaluated, the statements inside
of it are evaluated in order.  Blocks are useful for putting several
statements in a place where X10 asks for a single one, such as the consequent
of an \xcd`if`, and for limiting the scope of local variables.
%~~gen ^^^ Statements30
% package statements.FOR.block.heads;
% class Example {
% def example(b:Boolean, S1:(Long)=>void, S2:(Long)=>void ) {
%~~vis
\begin{xten}
if (b) {
  // This is a block
  val v = 1;
  S1(v); 
  S2(v);
}
\end{xten}
%~~siv
%  } } 
%~~neg



\section{Expression statement}

Any expression may be used as a statement.

%##(ExpStatement StatementExp
\begin{bbgrammar}
%(FROM #(prod:ExpStmt)#)
             ExpStmt \: StmtExp \xcd";" & (\ref{prod:ExpStmt}) \\
%(FROM #(prod:StmtExp)#)
             StmtExp \: Assignment & (\ref{prod:StmtExp}) \\
                     \| PreIncrementExp \\
                     \| PreDecrementExp \\
                     \| PostIncrementExp \\
                     \| PostDecrementExp \\
                     \| MethodInvo \\
                     \| ObCreationExp \\
\end{bbgrammar}
%##)

The expression statement evaluates an expression. The value of the expression
is not used. Side effects of the expression occur, and may produce results
used by following statements. Indeed, statement expressions which terminate
without side effects cannot have any visible effect on the results of the
computation. 


\begin{ex}
%~~gen ^^^ Statements40
% package Sta.tem.ent.s.expressions;
% import x10.util.*;
%~~vis
\begin{xten}
class StmtEx {
  def this() { 
     x10.io.Console.OUT.println("New StmtEx made");  }
  static def call() { 
     x10.io.Console.OUT.println("call!");}
  def example() {
     var a : Long = 0;
     a = 1; // assignment
     new StmtEx(); // allocation
     call(); // call
  }
}
\end{xten}
%~~siv
%
%~~neg
\end{ex}



\section{Labeled statement}
\index{label}
\index{statement label}


\begin{bbgrammar}
    LabeledStatement \: Id \xcd":" Statement 
\end{bbgrammar}


Statements may be labeled. The label may be used to describe the target of a
\xcd`break` statement appearing within a substatement (which, when executed,
ends the labeled statement), or, in the case of a loop, a \xcd`continue` as
well (which, when executed, proceeds to the next iteration of the loop). The
scope of a label is the statement labeled.

\begin{ex}
The label on the outer \xcd`for` statement allows \xcd`continue` and
\xcd`break` statements to continue or break it.  Without the label,
\xcd`continue` or \xcd`break` would only continue or break the inner \xcd`for`
loop. 
%~~gen ^^^ Statements50
% package state.meant.labe.L;
% class Example {
% def example(a:(Long,Long) => Long, do_things_to:(Long)=>Long) {
%~~vis
\begin{xten}
lbl : for (i in 1..10) {
   for (j in i..10) {  
      if (a(i,j) == 0) break lbl;
      if (a(i,j) == 1) continue lbl;
      if (a(i,j) == a(j,i)) break lbl;
   }
}
\end{xten}
%~~siv
%} } 
%~~neg
\end{ex}

In particular, a block statement may be labeled: \xcd` L:{S}`.  This allows
the use of \xcd`break L` within \xcd`S` to leave \xcd`S`, which can, if
carefully used, avoid deeply-nested \xcd`if`s. 

\begin{ex}
%~~gen ^^^ Statements51
% package statements51;
% abstract class Example {
% abstract def phase1(String):void;
% abstract def phase2(String):void;
% abstract def phase3(String):void;
% abstract def suitable_for_phase_2(String):Boolean;
% abstract def suitable_for_phase_3(String):Boolean;
% def example(filename: String) {
% KNOWNFAIL-labelled-blocks
%~~vis
\begin{xten}
multiphase: {
  if (!exists(filename)) break multiphase;
  phase1(filename);
  if (!suitable_for_phase_2(filename)) break multiphase;
  phase2(filename);
  if (!suitable_for_phase_3(filename)) break multiphase;
  phase3(filename);
}
// Now the file has been phased as much as possible
\end{xten}
%~~siv
%}
%~~neg
\end{ex}

\limitation{Blocks cannot currently be labeled.}

\section{Break statement}
\index{break}

%##(BreakStatement
\begin{bbgrammar}
%(FROM #(prod:BreakStmt)#)
           BreakStmt \: \xcd"break" Id\opt \xcd";" & (\ref{prod:BreakStmt}) \\
\end{bbgrammar}
%##)


An unlabeled break statement exits the currently enclosing loop or switch
statement. A labeled break statement exits the enclosing 
statement with the given label.
It is illegal to break out of a statement not defined in the current
method, constructor, initializer, or closure.  
\xcd`break` is only allowed in sequential code.

\begin{ex}
The following code searches for an element of a C-style two-dimensional
array and breaks out of the loop when it is found:
%~~gen ^^^ Statements60
% package statements.come.from.banks.and.cranks;
% class LabelledBreakeyBreakyHeart {
% def findy(a:Rail[Rail[Long]], v:Long): Boolean {
%~~vis
\begin{xten}
var found: Boolean = false;
outer: for (i in a.range)
    for (j in a(i).range)
        if (a(i)(j) == v) {
            found = true;
            break outer;
        }
\end{xten}
%~~siv
% return found;
%}}
%~~neg
\end{ex}

\section{Continue statement}
\index{continue}

%##(ContinueStatement
\begin{bbgrammar}
%(FROM #(prod:ContinueStmt)#)
        ContinueStmt \: \xcd"continue" Id\opt \xcd";" & (\ref{prod:ContinueStmt}) \\
\end{bbgrammar}
%##)
An unlabeled \xcd`continue` skips the rest of the current iteration of the
innermost enclosing loop, and proceeds on to the next.  A labeled
\xcd`continue` does the same to the enclosing loop with that label.
It is illegal to continue a loop not defined in the current
method, constructor, initializer, or closure.
\xcd`continue` is only allowed in sequential code.



\section{If statement}
\index{if}

%##(IfThenStatement IfThenElseStatement
\begin{bbgrammar}
%(FROM #(prod:IfThenStmt)#)
          IfThenStmt \: \xcd"if" \xcd"(" Exp \xcd")" Stmt & (\ref{prod:IfThenStmt}) \\
%(FROM #(prod:IfThenElseStmt)#)
      IfThenElseStmt \: \xcd"if" \xcd"(" Exp \xcd")" Stmt  \xcd"else" Stmt  & (\ref{prod:IfThenElseStmt}) \\
\end{bbgrammar}
%##)

An if statement comes in two forms: with and without an else
clause.

The if-then statement evaluates a condition expression, which must be of type
\xcd`Boolean`. If the condition is \xcd`true`, it evaluates the then-clause.
If the condition is \xcd"false", the if-then statement completes normally.

The if-then-else statement evaluates a \xcd`Boolean` expression and 
evaluates the then-clause if the condition is
\xcd"true"; otherwise, the \xcd`else`-clause is evaluated.

As is traditional in languages derived from Algol, the if-statement is syntactically
ambiguous.  That is, 
\begin{xten}
if (B1) if (B2) S1 else S2
\end{xten}
could be intended to mean either 
\begin{xten}
if (B1) { if (B2) S1 else S2 }
\end{xten} 
or 
\begin{xten}
if (B1) {if (B2) S1} else S2
\end{xten}
X10, as is traditional, attaches an \xcd`else` clause to the most recent
\xcd`if` that doesn't have one.
This example is interpreted as 
\xcd`if (B1) { if (B2) S1 else S2 }`. 



\section{Switch statement}
\index{switch}

%##(SwitchStatement SwitchBlock SwitchBlockGroups SwitchBlockGroup SwitchLabels SwitchLabel
\begin{bbgrammar}
%(FROM #(prod:SwitchStmt)#)
          SwitchStmt \: \xcd"switch" \xcd"(" Exp \xcd")" SwitchBlock & (\ref{prod:SwitchStmt}) \\
%(FROM #(prod:SwitchBlock)#)
         SwitchBlock \: \xcd"{" SwitchBlockGroups\opt SwitchLabels\opt \xcd"}" & (\ref{prod:SwitchBlock}) \\
%(FROM #(prod:SwitchBlockGroups)#)
   SwitchBlockGroups \: SwitchBlockGroup & (\ref{prod:SwitchBlockGroups}) \\
                     \| SwitchBlockGroups SwitchBlockGroup \\
%(FROM #(prod:SwitchBlockGroup)#)
    SwitchBlockGroup \: SwitchLabels BlockStmts & (\ref{prod:SwitchBlockGroup}) \\
%(FROM #(prod:SwitchLabels)#)
        SwitchLabels \: SwitchLabel & (\ref{prod:SwitchLabels}) \\
                     \| SwitchLabels SwitchLabel \\
%(FROM #(prod:SwitchLabel)#)
         SwitchLabel \: \xcd"case" ConstantExp \xcd":" & (\ref{prod:SwitchLabel}) \\
                     \| \xcd"default" \xcd":" \\
\end{bbgrammar}
%##)

A switch statement evaluates an index expression and then branches to
a case whose value is equal to the value of the index expression.
If no such case exists, the switch branches to the 
\xcd"default" case, if any.

Statements in each case branch are evaluated in sequence.  At the
end of the branch, normal control-flow falls through to the next case, if
any.  To prevent fall-through, a case branch may be exited using
a \xcd"break" statement.

The index expression must be of type \xcd"Int".
Case labels must be of type \xcd"Int", \xcd`Byte`, or \xcd`Short`, 
and must be compile-time 
constants.  Case labels cannot be duplicated within the
\xcd"switch" statement.

\begin{ex}
In this \xcd`switch`, case \xcd`1` falls through to case \xcd`2`.  The
other cases are separated by \xcd`break`s.
%~~gen ^^^ Statements70
% package Statement.Case;
% class Example {
% def example(i : Int, println: (String)=>void) {
%~~vis
\begin{xten}
switch (i) {
  case 1n: println("one, and ");
  case 2n: println("two"); 
          break;
  case 3n: println("three");
          break;
  default: println("Something else");
           break;
}
\end{xten}
%~~siv
% } } 
%~~neg
\end{ex}

\section{While statement}
\index{while}

%##(WhileStatement
\begin{bbgrammar}
%(FROM #(prod:WhileStmt)#)
           WhileStmt \: \xcd"while" \xcd"(" Exp \xcd")" Stmt & (\ref{prod:WhileStmt}) \\
\end{bbgrammar}
%##)

A while statement evaluates a \xcd`Boolean`-valued condition and executes a
loop body if \xcd"true". If the loop body completes normally (either by
reaching the end or via a \xcd"continue" statement with the loop header as
target), the condition is reevaluated and the loop repeats if \xcd"true". If
the condition is \xcd"false", the loop exits.

\begin{ex}
A loop to execute the process in the Collatz conjecture (a.k.a. 3n+1 problem,
Ulam conjecture, Kakutani's problem, Thwaites conjecture, Hasse's algorithm,
and Syracuse problem) can be written as follows:
%~~gen ^^^ Statements80
% package Statements.AreFor.Flatements;
% class Example {
% def example(var n:Long) {
%~~vis
\begin{xten}
  while (n > 1) {
     n = (n % 2 == 1) ? 3*n+1 : n/2;
  }
\end{xten}
%~~siv
% } } 
%~~neg
\end{ex}
\section{Do--while statement}
\index{do}

%##(DoStatement
\begin{bbgrammar}
%(FROM #(prod:DoStmt)#)
              DoStmt \: \xcd"do" Stmt \xcd"while" \xcd"(" Exp \xcd")" \xcd";" & (\ref{prod:DoStmt}) \\
\end{bbgrammar}
%##)


A \Xcd{do-while} statement executes the loop body, and then evaluates a
\xcd`Boolean`-valued condition expression. If \xcd"true", the loop repeats.
Otherwise, the loop exits.


\section{For statement}
\index{for}

%##(ForStatement BasicForStatement ForInit ForUpdate StatementExpList EnhancedForStatement
\begin{bbgrammar}
%(FROM #(prod:ForStmt)#)
             ForStmt \: BasicForStmt & (\ref{prod:ForStmt}) \\
                     \| EnhancedForStmt \\
%(FROM #(prod:BasicForStmt)#)
        BasicForStmt \: \xcd"for" \xcd"(" ForInit\opt \xcd";" Exp\opt \xcd";" ForUpdate\opt \xcd")" Stmt & (\ref{prod:BasicForStmt}) \\
%(FROM #(prod:ForInit)#)
             ForInit \: StmtExpList & (\ref{prod:ForInit}) \\
                     \| LocVarDecln \\
%(FROM #(prod:ForUpdate)#)
           ForUpdate \: StmtExpList & (\ref{prod:ForUpdate}) \\
%(FROM #(prod:StmtExpList)#)
         StmtExpList \: StmtExp & (\ref{prod:StmtExpList}) \\
                     \| StmtExpList \xcd"," StmtExp \\
%(FROM #(prod:EnhancedForStmt)#)
     EnhancedForStmt \: \xcd"for" \xcd"(" LoopIndex \xcd"in" Exp \xcd")" Stmt & (\ref{prod:EnhancedForStmt}) \\
                     \| \xcd"for" \xcd"(" Exp \xcd")" Stmt \\
\end{bbgrammar}
%##)

\xcd`for` statements provide bounded iteration, such as looping over a list.
It has two forms: a basic form allowing near-arbitrary iteration, {\em a la}
C, and an enhanced form designed to iterate over a collection.

A basic \xcd`for` statement provides for arbitrary iteration in a somewhat
more organized fashion than a \xcd`while`.  The loop 
\xcd`for(init; test; step)body` is
similar to: 
\begin{xten}
{
   init;
   while(test) {
      body;
      step;
   }
}
\end{xten}
\noindent
except that \xcd`continue` statements which continue the \xcd`for` loop will
perform the \xcd`step`, which, in the \xcd`while` loop, they will not do. 

\xcd`init` is performed before the loop, and is traditionally used to declare
and/or initialize the loop variables. It may be a single variable binding
statement, such as \xcd`var i:Long = 0` or \xcd`var i:Long=0, j:Long=100`. (Note
that a single variable binding statement may bind multiple variables.)
Variables introduced by \xcd`init` may appear anywhere in the \xcd`for`
statement, but not outside of it.  Or, it may be a sequence of expression
statements, such as \xcd`i=0, j=100`, operating on already-defined variables.
If omitted, \xcd`init` does nothing.

\xcd`test` is a Boolean-valued expression; an iteration of the loop will only
proceed if \xcd`test` is true at the beginning of the loop, after \xcd`init`
on the first iteration or after \xcd`step` on later ones. If omitted, \xcd`test`
defaults to \xcd`true`, giving a loop that will run until stopped by some
other means such as \xcd`break`, \xcd`return`, or \xcd`throw`.

\xcd`step` is performed after the loop body, between one iteration and the
next. It traditionally updates the loop variables from one iteration to the
next: \eg, \xcd`i++` and \xcd`i++,j--`.  If omitted, \xcd`step` does nothing.

\xcd`body` is a statement, often a code block, which is performed whenever
\xcd`test` is true.  If omitted, \xcd`body` does nothing.




\label{ForAllLoop}


An enhanced for statement is used to iterate over a collection, or other
structure designed to support iteration by implementing the interface
\xcd`Iterable[T]`.    The loop variable must be of type \xcd`T`, 
or destructurable from a value of type \xcd`T`
(\Sref{exploded-syntax}).  
Each iteration of the loop
binds the iteration variable to another element of the collection.
The loop \xcd`for(x in c)S` behaves like: 
%~~gen ^^^ Statements5e4u
% package Statements5e4u;
% class ForAll {
% def forall[T](c:Iterable[T], S: () => void) {
%~~vis
\begin{xten}
val iterator: Iterator[T] = c.iterator();
while (iterator.hasNext()) {
  val x : T = iterator.next();
  S();
}
\end{xten}
%~~siv
%} }
%~~neg

A number of library classes implement \xcd`Iterable`, and thus can be iterated
over.  For example, iterating over a \xcd`Rail` iterates the elements 
stored in the rail.

The type of the loop variable may be supplied as \xcd`x <: T`.  In this case
the iterable \xcd`c` must have type \xcd`Iterable[U}` for some \xcd`U <: T`,
and \xcd`x` will be given the type \xcd`U`.

\begin{ex}
This loop adds up the elements of a \xcd`List[Long]`.
Note that iterating over a list yields the elements of the list, as specified
in the \xcd`List` API. 
%~~gen ^^^ Statements3d9l
% package Statements3d9l;
% class Example {
%~~vis
\begin{xten}
static def sum(a:x10.util.List[Long]):Long {
  var s : Long = 0;
  for(x in a) s += x;
  return s;
}
\end{xten}
%~~siv
%}
%~~neg

The following code sums the elements of an integer rail. 
%~~gen ^^^ Statements2d4h
% package Statements2d4h;
% class Example { 
%~~vis
\begin{xten}
static def sum(a: Rail[Long]): Long {
  var s : Long = 0;
  for(v in a) s += v;
  return s;
}
\end{xten}
%~~siv
%}
%~~neg

Iteration over an \xcd`LongRange` (\Sref{sect:intrange}) is quite common. This
allows looping while varying a long index: 
%~~gen ^^^ Statements3o9s
% package Statements3o9s;
% class Example { static def example() {
%~~vis
\begin{xten}
var sum : Long = 0;
for(i in 1..10) sum += i;
assert sum == 55;
\end{xten}
%~~siv
%} } 
% class Hook { def run() { Example.example(); return true; } }
%~~neg


\end{ex}

Iteration variables have the \xcd`for` statement as scope.  They shadow other
variables of the same names.


\section{Return statement}
\label{ReturnStatement}
\index{return}

%##(ReturnStatement
\begin{bbgrammar}
%(FROM #(prod:ReturnStmt)#)
          ReturnStmt \: \xcd"return" Exp\opt \xcd";" & (\ref{prod:ReturnStmt}) \\
\end{bbgrammar}
%##)

Methods and closures may return values using a \xcd`return` statement. 
\xcd"void" methods must return without a value; other methods must return a value of the return type. 

\begin{ex}
The following code illustrates returning values from a closure and a method.
The \xcd`return` inside of \xcd`closure` returns from \xcd`closure`, not from
\xcd`method`.  
%~~gen ^^^ Statements2j1d
% package Statements2j1d;
% class Example {
%~~vis
\begin{xten}
def method(x:Long) {
  val closure = (y:Long) => {return x+y;}; 
  val res = closure(0);
  assert res == x;
  return res == x;
}
\end{xten}
%~~siv
%}
%~~neg


\end{ex}


\section{Assert statement} 
\index{assert}

%##(AssertStatement
\begin{bbgrammar}
%(FROM #(prod:AssertStmt)#)
          AssertStmt \: \xcd"assert" Exp \xcd";" & (\ref{prod:AssertStmt}) \\
                     \| \xcd"assert" Exp  \xcd":" Exp  \xcd";" \\
\end{bbgrammar}
%##)

The statement \xcd`assert E` checks that the Boolean expression \xcd`E`
evaluates to true, and, if not, throws an \xcd`x10.lang.Error`  exception.  
The annotated assertion statement \xcd`assert E : F;` checks \xcd`E`, and, if
it is 
false, throws an \xcd`x10.lang.Error` exception with \xcd`F`'s value attached
to it. 

\begin{ex}
The following code compiles properly.  
%~~gen ^^^ Statements100
% package Statements.Assert;
% 
%~~vis
\begin{xten}
class Example {
  public static def main(argv:Rail[String]) {
    val a = 1;
    assert a != 1 : "Changed my mind about a.";
  }
}
\end{xten}
%~~siv
%~~neg
\noindent
However, when run, it 
prints a stack trace starting with 
\begin{xten}
x10.lang.Error: Changed my mind about a.
\end{xten}
\end{ex}

\section{Exceptions in X10}
\index{exception}
\index{termination!abrupt}
\index{termination!normal}

X10 programs can throw {\em exceptions} to indicate unusual or problematic
situations; this is {\em abrupt termination}.  Exceptions, as data values, are
objects which which inherit from 
\xcd`x10.lang.CheckedThrowable`.    Exceptions may be thrown intentionally with the
\xcd`throw` statement. Many primitives and library functions throw exceptions
if they encounter problems; \eg, dividing by zero throws an instance of
\xcd`x10.lang.ArithmeticException`. 

When an exception is thrown, dynamically enclosing
\xcd`try`-\xcd`catch` blocks in the same activity can attempt to handle it.   If the throwing
statement in inside some \xcd`try` clause, and some matching \xcd`catch`
clause catches that type of exception, the corresponding \xcd`catch` body will
be executed, and the process of throwing is finished.  
If no statically-enclosing \xcd`try`-\xcd`catch` block can handle the
exception, the current method call returns (abnormally), throwing the same
exception from the point at which the method was called.  

This process continues until the exception is handled or there are no more
calling methods in the activity. In the latter case, the activity will
terminate abnormally, and the exception will propagate to the activity's root;
see \Sref{ExceptionModel} for details.

\Xten{} supports both {\em checked} and {\em unchecked}
exceptions. Methods are obligated to declare via a \xcd{throws} clause
any checked exceptions that they might throw.  However, in 
\Xten{}, the class library design favors unchecked exceptions:
virtually all exceptions in the standard library are unchecked.  Checked
exceptions are defined to be any subclass of
\xcd{x10.lang.CheckedThrowable} that are not also subclasses of
either \xcd{x10.lang.Exception} or \xcd{x10.lang.Error}. All of the
concrete exception classes in the \Xten{} standard library are
subclasses of either \xcd{Exception} or \xcd{Error}.


\section{Throw statement}
\index{throw}

%##(ThrowStatement
\begin{bbgrammar}
%(FROM #(prod:ThrowStmt)#)
           ThrowStmt \: \xcd"throw" Exp \xcd";" & (\ref{prod:ThrowStmt}) \\
\end{bbgrammar}
%##)

\index{Exception}
\xcd"throw E" throws an exception whose value is \xcd`E`, which must be an
instance of a subtype of \xcd`x10.lang.CheckedThrowable`. 

\begin{ex}
The following code checks if an index is in range and
throws an exception if not.

%~~gen ^^^ Statements110
% package Statements_index_check;
% class ThrowStatementExample {
% def thingie(i:Long, x:Rail[Boolean])  {
%~~vis
\begin{xten}
if (i < 0 || i >= x.size)
    throw new MyIndexOutOfBoundsException();
\end{xten}
%~~siv
%} }
% class MyIndexOutOfBoundsException extends Exception {}
%~~neg
\end{ex}

\section{Try--catch statement}
\index{try}
\index{catch}
\index{finally}
\index{exception}

%##(TryStatement Catches CatchClause Finally
\begin{bbgrammar}
%(FROM #(prod:TryStmt)#)
             TryStmt \: \xcd"try" Block Catches & (\ref{prod:TryStmt}) \\
                     \| \xcd"try" Block Catches\opt Finally \\
%(FROM #(prod:Catches)#)
             Catches \: CatchClause & (\ref{prod:Catches}) \\
                     \| Catches CatchClause \\
%(FROM #(prod:CatchClause)#)
         CatchClause \: \xcd"catch" \xcd"(" Formal \xcd")" Block & (\ref{prod:CatchClause}) \\
%(FROM #(prod:Finally)#)
             Finally \: \xcd"finally" Block & (\ref{prod:Finally}) \\
\end{bbgrammar}
%##)

Exceptions are handled with a \xcd"try" statement.
A \xcd"try" statement consists of a \xcd"try" block, zero or more
\xcd"catch" blocks, and an optional \xcd"finally" block.

First, the \xcd"try" block is evaluated.  If the block throws an
exception, control transfers to the first matching \xcd"catch"
block, if any.  A \xcd"catch" matches if the value of the
exception thrown is a subclass of the \xcd"catch" block's formal
parameter type.

The \xcd"finally" block, if present, is evaluated on all normal
and exceptional control-flow paths from the \xcd"try" block.
If the \xcd"try" block completes normally
or via a \xcd"return", a \xcd"break", or a
\xcd"continue" statement, 
the \xcd"finally"
block is evaluated, and then control resumes at
the statement following the \xcd"try" statement, at the branch target, or at
the caller as appropriate.
If the \xcd"try" block completes
exceptionally, the \xcd"finally" block is evaluated after the
matching \xcd"catch" block, if any, and when and if the \xcd`finally` block
finishs normally, the
exception is rethrown.


The parameter of a \xcd`catch` block has the block as scope.  It shadows other
variables of the same name.

\begin{ex}
The \xcd`example()` method below executes without any assertion errors
%~~gen ^^^ Statements9x3m
% package Statements9x3m;
% 
%~~vis
\begin{xten}
class Example {
  class ThisExn extends Exception {}
  class ThatExn extends Exception {}
  var didFinally : Boolean = false;
  def example(b:Boolean) {
    try {
       throw b ? new ThatExn() : new ThisExn();
    }
    catch(ThatExn) {return true;}
    catch(ThisExn) {return false;}
    finally {
       this.didFinally = true;
    }
  }
  static def doExample() {
    val e = new Example();
    assert e.example(true);
    assert e.didFinally == true;
  }
}
\end{xten}
%~~siv
% 
% class Hook { def run() { Example.doExample(); return true; } }
%~~neg

\end{ex}

\limitation{Constraints on exception types in \xcd`catch` blocks are not
currently supported. 
}

\section{Assert}

The \xcd`assert` statement 
%~~stmt~~`~~`~~B:Boolean ~~
\xcd`assert B;` 
checks that the Boolean expression \xcd`B` evaluates to true.  If so,
computation proceeds.  If not, it throws \xcd`x10.lang.AssertionError`.

The extended form 
%~~stmt~~`~~`~~B:Boolean, A:Any ~~ 
\xcd`assert B:A;`
is similar, but provides more debugging information.  The value of the
expression \xcd`A` is available as part of the \xcd`AssertionError`, \eg, to
be printed on the console.

\begin{ex}
\xcd`assert` is useful for confirming properties that you believe to be true
and wish to rely on.  In particular, well-chosen \xcd`assert`s make a program
robust in the face of code changes and unexpected uses of methods.
For example, the following method compute percent differences, but asserts
that it is not dividing by zero.  If the mean is zero, it throws an exception,
including the values of the numbers as potentially useful debugging
information. 
%~~gen ^^^ StmtAssert10
%package StmtAssert10;
% class Example {
%~~vis
\begin{xten}
static def percentDiff(x:Double, y:Double) {
  val diff = x-y;
  val mean = (x+y)/2;
  assert mean != 0.0  : [x,y]; 
  return Math.abs(100 * (diff / mean));
}
\end{xten}
%~~siv
% }
%~~neg

\end{ex}


At times it may be considered important not to check \xcd`assert` statements;
\eg, if the test is expensive and the code is sufficiently well-tested.  The
\xcd`-noassert` command line option causes the compiler to ignore all
\xcd`assert` statements. 
