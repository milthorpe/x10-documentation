\chapter{Interoperability with Other Languages}
\label{NativeCode}
\index{native code}
\label{Interoperability}
\index{interoperability}

The ability to interoperate with other programming languages is an
essential feature of the \Xten{} implementation.  Cross-language
interoperability enables both the incremental adoption of \Xten{} in
existing applications and the usage of existing libraries and
frameworks by newly developed \Xten{} programs. 

There are two primary interoperability scenarios that are supported by
\XtenCurrVer{}: inline substitution of fragments of foreign code for
\Xten program fragments (expressions, statements) and external linkage
to foreign code.

\section{Embedded Native Code Fragments}

The
\xcd`@Native(lang,code) Construct` annotation from \xcd`x10.compiler.Native` in
\Xten{} can be used to tell the \Xten{} compiler to substitute \xcd`code` for
whatever it would have generated when compiling \xcd`Construct`
with the \xcd`lang` back end.

The compiler cannot analyze native code the same way it analyzes \Xten{} code.  In
particular, \xcd`@Native` fields and methods must be explicitly typed; the
compiler will not infer types.

\subsection{Native {\tt static} Methods}

\xcd`static` methods can be given native implementations.  Note that these
implementations are syntactically {\em expressions}, not statements, in C++ or
Java.   Also, it is possible (and common) to provide native implementations
into both Java and C++ for the same method.
%~~gen ^^^ extern10
% package Extern.or_current_turn;
%~~vis
\begin{xten}
import x10.compiler.Native;
class Son {
  @Native("c++", "printf(\"Hi!\")")
  @Native("java", "System.out.println(\"Hi!\")")
  static def printNatively():void = {};
}
\end{xten}
%~~siv
%
%~~neg

If only some back-end languages are given, the \Xten{} code will be used for the
remaining back ends: 
%~~gen ^^^ extern20
% package Extern.or.burn;
%~~vis
\begin{xten}
import x10.compiler.Native;
class Land {
  @Native("c++", "printf(\"Hi from C++!\")")
  static def example():void = {
    x10.io.Console.OUT.println("Hi from X10!");
  };
}
\end{xten}
%~~siv
%
%~~neg

The \xcd`native` modifier on methods indicates that the method must not have
an \Xten{} code body, and \xcd`@Native` implementations must be given for all back
ends:
%~~gen ^^^ extern30
% package Extern.or_maybe_getting_back_together;
%~~vis
\begin{xten}
import x10.compiler.Native;
class Plants {
  @Native("c++", "printf(\"Hi!\")")
  @Native("java", "System.out.println(\"Hi!\")")
  static native def printNatively():void;
}
\end{xten}
%~~siv
%
%~~neg


Values may be returned from external code to \Xten{}.  Scalar types in Java and
C++ correspond directly to the analogous types in \Xten{}.  
%~~gen ^^^ extern40
% package Extern.hardy;
%~~vis
\begin{xten}
import x10.compiler.Native;
class Return {
  @Native("c++", "1")
  @Native("java", "1")
  static native def one():Int;
}
\end{xten}
%~~siv
%
%~~neg
Types are {\em not} inferred for methods marked as \xcd`@Native`.

Parameters may be passed to external code.  \xcd`(#1)`  is the first parameter,
\xcd`(#2)` the second, and so forth.  (\xcd`(#0)` is the name of the enclosing
class, or the \xcd`this` variable.)  Be aware that this is macro substitution
rather than normal parameter 
passing; \eg, if the first actual parameter is \xcd`i++`, and \xcd`(#1)`
appears twice in the external code, \xcd`i` will be incremented twice.
For example, a (ridiculous) way to print the sum of two numbers is: 
%~~gen ^^^ extern50
% package Extern.or_turnabout_is_fair_play;
%~~vis
\begin{xten}
import x10.compiler.Native;
class Species {
  @Native("c++","printf(\"Sum=%d\", ((#1)+(#2)) )")
  @Native("java","System.out.println(\"\" + ((#1)+(#2)))")
  static native def printNatively(x:Int, y:Int):void;
}
\end{xten}
%~~siv
%
%~~neg


Static variables in the class are available in the external code.  For Java,
the static variables are used with their \Xten{} names.  For C++, the names
must be mangled, by use of the \xcd`FMGL` macro.  

%~~gen ^^^ extern60
%package Extern.or.Die;
%~~vis
\begin{xten}
import x10.compiler.Native;
class Ability {
  static val A : Int = 1n;
  @Native("java", "A+2")
  @Native("c++", "Ability::FMGL(A)+2")
  static native def fromStatic():Int;
}
\end{xten}
%~~siv
%
%~~neg




\subsection{Native Blocks}

Any block may be annotated with \xcd`@Native(lang,stmt)`, indicating that, in
the given back end, it should be implemented as \xcd`stmt`. All 
variables  from the surrounding context are available inside \xcd`stmt`. For
example, the method call \xcd`born.example(10)`, if compiled to Java, changes
the field \xcd`y` of a \xcd`Born` object to 10. If compiled to C++ (for which
there is no \xcd`@Native`), it sets it to 3. 
%~~gen ^^^ extern70
%package Extern.me.plz; 
%~~vis
\begin{xten}
import x10.compiler.Native;
class Born {
  var y : Int = 1n; 
  public def example(x:Int):Int{
    @Native("java", "y=x;") 
    {y = 3n;}
    return y;
  }
}
\end{xten}
%~~siv
%
%~~neg

Note that the code being replaced is a statement -- the block \xcd`{y = 3;}`
in this case -- so the replacement should also be a statement. 


Other \Xten{} constructs may or may not be available in Java and/or C++ code.  For
example, type variables do not correspond exactly to type variables in either
language, and may not be available there.  The exact compilation scheme is
{\em not} fully specified.  You may inspect the generated Java or C++ code and
see how to do specific things, but there is no guarantee that fancy external
coding will continue to work in later versions of \Xten{}.



The full facilities of C++ or Java are available in native code blocks.
However, there is no guarantee that advanced features behave sensibly. You
must follow the exact conventions that the code generator does, or you will
get unpredictable results.  Furthermore, the code generator's conventions may
change without notice or documentation from version to version.  In most cases
the  code should either be a very simple expression, or a method
or function call to external code.


\section{Interoperability with External Java Code}

With Managed X10, we can seamlessly call existing Java code from \Xten{},
and call \Xten{} code from Java.  We call this the 
\emph{Java interoperability} feature.

By combining Java interoperability with X10's distributed
execution features, we can even make existing Java applications, which
are originally designed to run on a single Java VM, scale-out with
minor modifications.

\subsection{How Java program is seen in X10}

Managed X10 does not pre-process the existing Java code to make it
accessible from X10.  X10 programs compiled with Managed X10 will call
existing Java code as is.

\paragraph{Types}

In X10, both at compile time and run time, there is no way to
distinguish Java types from X10 types.  Java types can be referred to
with regular \xcd{import} statement, or their qualified names.  The
package \xcd{java.lang} is not auto-imported into \Xten.  In Managed
x10, the resolver is enhanced to resolve types with X10 source files
in the source path first, then resolve them with Java class files in
the class path. Note that the resolver does not resolve types with
Java source files, therefore Java source files must be compiled in
advance.  To refer to Java types listed in
Tables~\ref{tab:specialtypes}, and \ref{tab:otherspecialtypes}, which
include all Java primitive types, use the corresponding X10 type
(e.g. use \xcd{x10.lang.Int} (or in short, \xcd{Int}) instead of
\xcd{int}).

\paragraph{Fields}

Fields of Java types are seen as fields of X10 types.

Managed X10 does not change the static initialization semantics of
Java types, which is per-class, at load time, and per-place (Java VM),
therefore, it is subtly different than the per-field lazy
initialization semantics of X10 static fields.

\paragraph{Methods}

Methods of Java types are seen as methods of X10 types.

\paragraph{Generic types}

Generic Java types are seen as their raw types 
(\S 4.8 in~\cite{java-lang-spec2005}).  Raw type is a mechanism to handle generic
Java types as non-generic types, where the type parameters are assumed
as \verb|java.lang.Object| or their upperbound if they have it.  Java
introduced generics and raw type at the same time to facilitate
generic Java code interfacing with non-generic legacy Java code.
Managed X10 uses this mechanism for a slightly different purpose.
Java erases type parameters at compile time, whereas X10 preserves
their values at run time.  To manifest this semantic gap in generics,
Managed X10 represents Java generic types as raw types and eliminates
type parameters at source code level.  For more detailed discussions,
please refer to~\cite{TakeuchiX1011,TakeuchiX1012}.

\paragraph{Arrays}

X10 rail and array types are generic types whose representation is different
from Java array types.

Managed X10 provides a special X10 type
\xcd{x10.interop.Java.array[T]} which represents Java array type
\xcd{T[]}.  Note that for X10 types in Table~\ref{tab:specialtypes},
this type means the Java array type of their primary type.  For
example, \xcd{Java.array[Int]} and \xcd{Java.array[String]} mean
\xcd{int[]} and \xcd{java.lang.String[]}, respectively.  Managed X10
also provides conversion methods between X10 \xcd`Rail`s and Java
arrays (\xcd{Java.convert[T](a:Rail[T]):Java.array[T]} and
\xcd{Java.convert[T](a:Java.array[T]):Rail[T]}), between X10 \`Array`s 
of rank 1 and Java arrays
(\xcd{Java.convert[T](a:Array[T](1)):Java.array[T]} and
\xcd{Java.convert[T](a:Java.array[T]):Array[T](1)}),
and creation methods for Java arrays 
(\xcd{Java.newArray[T](d0:Int):Java.array[T]}
etc.).

\paragraph{Exceptions}

The \Xten{} 2.3 exception hierarchy has been designed so that there is a
natural correspondence with the Java exception hierarchy. As shown in
Table~\ref{tab:otherspecialtypes}, many commonly used Java
exception types are directly mapped to X10 exception types. 
Exception types that are thus aliased may be caught (and thrown) using
either their Java or \Xten types.  In \Xten code, it is stylistically
preferable to use the \Xten type to refer to the exception as shown in 
Figure~\ref{fig:javaexceptions}.

%----------------
\begin{figure}
\begin{xten}
import x10.interop.Java;
public class XClass {   
  public static def main(args:Rail[String]):void {
    try {
      val a = Java.newArray[Int](2);
      a(0) = 0;
      a(1) = 1;
      a(2) = 2;
    } catch (e:x10.lang.ArrayIndexOutOfBoundsException) {
      Console.OUT.println(e);
    }
  }
}
\end{xten}
%\vspace{-2mm}%@@ADJUST
\begin{verbatim}
> x10c -d bin src/XClass.x10
> x10 -cp bin XClass
x10.lang.ArrayIndexOutOfBoundsException: Array index out of range: 2
\end{verbatim}
\caption{Java exceptions in X10}
%\vspace{-4mm}%@@ADJUST
\label{fig:javaexceptions}
\end{figure}
%----------------

\paragraph{Compiling and executing X10 programs}

We can compile and run X10 programs that call existing Java code with
the same \verb|x10c| and \verb|x10| command by specifying the location
of Java class files or jar files that your X10 programs refer to, with
\verb|-classpath| (or in short, \verb|-cp|) option.

\subsection{How X10 program is translated to Java}

Managed X10 translates X10 programs to Java class files. 

X10 does not provide a Java reflection-like mechanism to resolve X10
types, methods, and fields with their names at runtime, nor a code
generation tool, such as \verb|javah|, to generate stub code to access
them from other languages.  Java programmers, therefore, need to
access X10 types, methods, and fields in the generated Java code
directly as they access Java types, methods, and fields.  To make it
possible, Java programmers need to understand how X10 programs are
translated to Java.

Some aspects of the X10 to Java translation scheme may change in
future version of \Xten{}; therefore in this document only a stable
subset of translation scheme will be explained.  Although it is a
subset, it has been extensively used by X10 core team and proved to be
useful to develop Java Hadoop interop layer for a Main-memory Map
Reduce (M3R) engine~\cite{Shinnar12M3R} in X10.

In the following discussions, we deliberately ignore generic X10
types because the translation of generics is an area of active
development and will undergo some changes in future versions of \Xten{}.
For those who are interested in the implementation of generics
in Managed X10, please consult~\cite{TakeuchiX1012}.  We also don't
cover function types, function values, and all non-static methods.
Although slightly outdated, another paper~\cite{TakeuchiX1011}, which
describes translation scheme in X10 2.1.2, is still useful to
understand the overview of Java code generation in Managed X10.


\paragraph{Types}

X10 classes and structs are translated to Java classes with the same
names.  X10 interfaces are translated to Java interfaces with the same
names.

Table~\ref{tab:specialtypes} shows the list of special types that are
mapped to Java primitives.  Primitives are their primary
representations that are useful for good performance.  Wrapper classes
are used when the reference types are needed.  Each wrapper class has
two static methods \verb|$box()| and \verb|$unbox()| to convert its
value from primary representation to wrapper class, and vice versa,
and Java backend inserts their calls as needed.  An you notice, every
unsigned type uses the same Java primitive as its corresponding signed
type for its representation.

Table~\ref{tab:otherspecialtypes} shows a non-exhaustive list of
another kind of special types that are mapped (not translated) to Java
types.  As you notice, since an interface \verb|Any| is mapped to a
class |java.lang.Object| and \verb|Object| is hidden from the
language, there is no direct way to create an instance of
\verb|Object|. As a workaround, \verb|Java.newObject()| is provided.

As you also notice, \verb|x10.lang.Comparable[T]| is mapped to \verb|java.lang.Comparable|.
This is needed to map \verb|x10.lang.String|, which implements \verb|x10.lang.Compatable[String]|, to \verb|java.lang.String| for performance, but as a trade off, this mapping results in the lost of runtime type information for \verb|Comparable[T]| in Managed X10.
The runtime of Managed X10 has built-in knowledge for \verb|String|, but for other Java classes that implement \verb|java.lang.Comparable|, \verb|instanceof Comparable[Int]| etc. may return incorrect results.
In principle, it is impossible to map X10 generic type to the existing Java generic type without losing runtime type information.

%----------------
\begin{table}
%\scriptsize
\small
\centering
\mbox{
%\hspace{-4mm}%@@ADJUST
\begin{tabular}{|lr|lr|l|}												   \hline
\multicolumn{2}{|c|}{\textbf{X10}}	& \multicolumn{2}{|c|}{\textbf{Java (primary)}}	& \textbf{Java (wrapper class)}	\\ \hline
															   \hline
{\tt x10.lang.Byte}	& {\tt 1y}	& {\tt byte}		& {\tt (byte)1}		& {\tt x10.core.Byte}		\\ \hline
{\tt x10.lang.UByte}	& {\tt 1uy}	& {\tt byte}		& {\tt (byte)1}		& {\tt x10.core.UByte}		\\ \hline
{\tt x10.lang.Short}	& {\tt 1s}	& {\tt short}		& {\tt (short)1}	& {\tt x10.core.Short}		\\ \hline
{\tt x10.lang.UShort}	& {\tt 1us}	& {\tt short}		& {\tt (short)1}	& {\tt x10.core.UShort} 	\\ \hline
{\tt x10.lang.Int}	& {\tt 1}	& {\tt int}		& {\tt 1}		& {\tt x10.core.Int}		\\ \hline
{\tt x10.lang.UInt}	& {\tt 1u}	& {\tt int}		& {\tt 1}		& {\tt x10.core.UInt}		\\ \hline
{\tt x10.lang.Long}	& {\tt 1l}	& {\tt long}		& {\tt 1l}		& {\tt x10.core.Long}	 	\\ \hline
{\tt x10.lang.ULong}	& {\tt 1ul}	& {\tt long}		& {\tt 1l}		& {\tt x10.core.ULong}	 	\\ \hline
{\tt x10.lang.Float}	& {\tt 1.0f}	& {\tt float}		& {\tt 1.0f}		& {\tt x10.core.Float}	 	\\ \hline
{\tt x10.lang.Double}	& {\tt 1.0}	& {\tt double}		& {\tt 1.0}		& {\tt x10.core.Double} 	\\ \hline
{\tt x10.lang.Char}	& {\tt 'c'}	& {\tt char}		& {\tt 'c'}		& {\tt x10.core.Char}		\\ \hline
{\tt x10.lang.Boolean}	& {\tt true}	& {\tt boolean}		& {\tt true}		& {\tt x10.core.Boolean}	\\ \hline
%{\tt x10.lang.String} 	& {\tt "abc"}	& {\tt java.lang.String}& {\tt "abc"}		& {\tt x10.core.String}		\\ \hline
\end{tabular}
}
\caption{X10 types that are mapped to Java primitives}
%\vspace{-4mm}%@@ADJUST
\label{tab:specialtypes}
\end{table}
%----------------


%----------------
\begin{table}
%\scriptsize
\small
\centering
\mbox{
%\hspace{-4mm}%@@ADJUST
\begin{tabular}{|l|l|}										   \hline
\multicolumn{1}{|c|}{\textbf{X10}}		& \multicolumn{1}{|c|}{\textbf{Java}}		\\ \hline
												   \hline
{\tt x10.lang.Any} 				& {\tt java.lang.Object}			\\ \hline
{\tt x10.lang.Comparable[T]} 			& {\tt java.lang.Comparable}			\\ \hline
{\tt x10.lang.String}		 		& {\tt java.lang.String}			\\ \hline
{\tt x10.lang.CheckedThrowable}		 	& {\tt java.lang.Throwable}			\\ \hline
{\tt x10.lang.CheckedException}		 	& {\tt java.lang.Exception}			\\ \hline
{\tt x10.lang.Exception} 			& {\tt java.lang.RuntimeException}		\\ \hline
{\tt x10.lang.ArithmeticException} 		& {\tt java.lang.ArithmeticException}		\\ \hline
{\tt x10.lang.ClassCastException} 		& {\tt java.lang.ClassCastException}		\\ \hline
{\tt x10.lang.IllegalArgumentException} 	& {\tt java.lang.IllegalArgumentException}	\\ \hline
{\tt x10.util.NoSuchElementException}	 	& {\tt java.util.NoSuchElementException}	\\ \hline
{\tt x10.lang.NullPointerException} 		& {\tt java.lang.NullPointerException}		\\ \hline
{\tt x10.lang.NumberFormatException} 		& {\tt java.lang.NumberFormatException}		\\ \hline
{\tt x10.lang.UnsupportedOperationException} 	& {\tt java.lang.UnsupportedOperationException}	\\ \hline
{\tt x10.lang.IndexOutOfBoundsException} 	& {\tt java.lang.IndexOutOfBoundsException}	\\ \hline
{\tt x10.lang.ArrayIndexOutOfBoundsException} 	& {\tt java.lang.ArrayIndexOutOfBoundsException}\\ \hline
{\tt x10.lang.StringIndexOutOfBoundsException} 	& {\tt java.lang.StringIndexOutOfBoundsException}\\ \hline
{\tt x10.lang.Error} 				& {\tt java.lang.Error}				\\ \hline
{\tt x10.lang.AssertionError} 			& {\tt java.lang.AssertionError}		\\ \hline
{\tt x10.lang.OutOfMemoryError} 		& {\tt java.lang.OutOfMemoryError}		\\ \hline
{\tt x10.lang.StackOverflowError} 		& {\tt java.lang.StackOverflowError}		\\ \hline
{\tt void} 					& {\tt void}					\\ \hline
\end{tabular}
}
\caption{X10 types that are mapped (not translated) to Java types}
%\vspace{-4mm}%@@ADJUST
\label{tab:otherspecialtypes}
\end{table}
%----------------


\paragraph{Fields}

As shown in Figure~\ref{fig:fields}, instance fields of X10 classes and structs are translated to the instance fields of the same names of the generated Java classes.
Static fields of X10 classes and structs are translated to the static methods of the generated Java classes, whose name has \verb|get$| prefix.
Static fields of X10 interfaces are translated to the static methods of the special nested class named \verb|$Shadow| of the generated Java interfaces.

%----------------
\begin{figure}
\begin{xten}
class C {
  static val a:Int = ...;
  var b:Int;
}
interface I {
  val x:Int = ...;
}
\end{xten}
%\vspace{-4mm}%@@ADJUST
\begin{xten}
class C {
  static int get$a() { return ...; }
  int b;
}
interface I {
  abstract static class $Shadow {
    static int get$x() { return ...; }
  }
}
\end{xten}
%\vspace{-2mm}%@@ADJUST
\caption{X10 fields in Java}
%\vspace{-4mm}%@@ADJUST
\label{fig:fields}
\end{figure}
%----------------


\paragraph{Methods}

As shown in Figure~\ref{fig:methods}, methods of X10 classes or structs are translated to the methods of the same names of the generated Java classes.
Methods of X10 interfaces are translated to the methods of the same names of the generated Java interfaces.

Every method whose return type has two representations, such as the types in Table~\ref{tab:specialtypes}, will have \verb|$O| suffix with its name.
For example, \verb|def f():Int| in X10 will be compiled as \verb|int f$O()| in Java.
For those who are interested in the reason, please look at our paper~\cite{TakeuchiX1012}.

%----------------
\begin{figure}
\begin{xten}
interface I {
  def f():Int;
  def g():Any;
}
class C implements I {
  static def s():Int = 0;
  static def t():Any = null;
  public def f():Int = 1;
  public def g():Any = null;
}
\end{xten}
%\vspace{-4mm}%@@ADJUST
\begin{xten}
interface I {
  int f$O();
  java.lang.Object g();
}
class C implements I {
  static int s$O() { return 0; }
  static java.lang.Object t() { return null; }
  public int f$O() { return 1; }
  public java.lang.Object g() { return null; }
}
\end{xten}
%\vspace{-2mm}%@@ADJUST
\caption{X10 methods in Java}
%\vspace{-4mm}%@@ADJUST
\label{fig:methods}
\end{figure}
%----------------


\paragraph{Compiling Java programs}

To compile Java program that calls X10 code, you should use
\verb|x10cj| command instead of javac (or whatever your Java
compiler). It invokes the post Java-compiler of \verb|x10c| with the
appropriate options. You need to specify the location of X10-generated
class files or jar files that your Java program refers to.

\verb|x10cj -cp MyX10Lib.jar MyJavaProg.java|


\paragraph{Executing Java programs}

Before executing any X10-generated Java code, the runtime of Managed
X10 needs to be set up properly at each place.  To set up the runtime,
a special launcher named \verb|runjava| is used to run Java programs.
All Java programs that call X10 code should be launched with it, and
no other mechanisms, including direct execution with java command, are
supported.

In current implementation, the setup process includes preloading of
all reachable classes in order to assign a globally unique id for each
type.  The \verb|runjava| command ensures this for the X10 standard
library. However all the X10 types that are not in the standard
library must be specified as command-line arguments to \verb|runjava|
as follows.  To make preloading successfully preload all used types in
a program run, users must use one of the following two techniques.
One technique is to specify X10 types that are unreachable by other
means as command-line arguments as follows.

\begin{verbatim}
Usage: runjava [options] <Java-main-class> [arg0 arg1 ...]
where [options] includes all x10 options, and:
    -preload <class>         class to be preloaded
\end{verbatim}

The other technique is to insert dummy references in Java programs to
the X10 types that are unreachable by other means, and make them
reachable.

We anticipate the future versions of \Xten{} will use a more dynamic
id assignment algorithm and that therefore this preloading requirement
will be eliminated.

\section{Interoperability with External C and C++ Code}

C and C++ code can be linked to X10 code, either by writing auxiliary C++ files and
adding them with suitable annotations, or by linking libraries.

\subsection{Auxiliary C++ Files}

Auxiliary C++ code can be written in \xcd`.h` and \xcd`.cc` files, which
should be put in the same directory as the the X10 file using them.
Connecting with the library uses the \xcd`@NativeCPPInclude(dot_h_file_name)`
annotation to include the header file, and the 
\xcd`@NativeCPPCompilationUnit(dot_cc_file_name)` annotation to include the
C++ code proper.  For example: 

{\bf MyCppCode.h}: 
\begin{xten}
void foo();
\end{xten}


{\bf MyCppCode.cc}:
\begin{xten}
#include <cstdlib>
#include <cstdio>
void foo() {
    printf("Hello World!\n");
}
\end{xten}

{\bf Test.x10}:
\begin{xten}
import x10.compiler.Native;
import x10.compiler.NativeCPPInclude;
import x10.compiler.NativeCPPCompilationUnit;

@NativeCPPInclude("MyCPPCode.h")
@NativeCPPCompilationUnit("MyCPPCode.cc")
public class Test {
    public static def main (args:Rail[String]) {
        { @Native("c++","foo();") {} }
    }
}
\end{xten}

\subsection{C++ System Libraries}

If we want to additionally link to more libraries in \xcd`/usr/lib` for
example, it is necessary to adjust the post-compilation directly.  The
post-compilation is the compilation of the C++ which the X10-to-C++ compiler
\xcd`x10c++` produces.  

The primary mechanism used for this is the \xcd`-cxx-prearg` and
\xcd`-cxx-postarg` command line arguments to
\xcd`x10c++`. The values of any \xcd`-cxx-prearg` commands are placed
in the post compiler command before the list of .cc files to compile.
The values of any \xcd`-cxx-postarg` commands are placed in the post
compiler command after the list of .cc files to compile. Typically
pre-args are arguments intended for the C++ compiler itself, while
post-args are arguments intended for the linker. 

The following example shows how to compile \xcd`blas` into the
executable via these commands. The command must be issued on one line.

\begin{xten}
x10c++ Test.x10 -cxx-prearg -I/usr/local/blas 
  -cxx-postarg -L/usr/local/blas -cxx-postarg -lblas'
\end{xten}

