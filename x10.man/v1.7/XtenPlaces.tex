\chapter{Places}\label{XtenPlaces}\index{places}

An \Xten{} place is a repository for data and activities. Each place
is to be thought of as a locality boundary: the activities running in
a place may access data items located at that place with the
efficiency of on-chip access. Accesses to remote places may take
orders of magnitude longer.

{}\Xten{} provides a built-in value class, \xcd"x10.lang.place"; all
places are instances of this class.  This class is \xcd"final" in
{}\XtenCurrVer.

In \XtenCurrVer{}, the set of places available to a computation is
determined at the time that the program is run and remains fixed
through the run of the program. The number of places available 
may be determined by reading (\xcd"Place.MAX_PLACES"). (This number
is specified from the command line/configuration information; 
see associated {\tt README} documentation.)

All scalar objects created during program execution are located in one
place, though they may be referenced from other places. Aggregate
objects (arrays) may be distributed across multiple places using
distributions.

The set of all places in a running instance of an \Xten{} program may
be obtained through the \xcd"const" field \xcd"Place.places".  (This
set may be used to define distributions, for instance,
\Sref{XtenDistributions}.) 


The set of all places is totally ordered.  The first place may be
obtained by reading \xcd"Place.FIRST_PLACE". The initial activity for
an \Xten{} computation starts in this place
(\Sref{initial-computation}). For any place, the operation \xcd"next()"
returns the next place in the total order (wrapping around at the
end). Further details on the methods and fields available on this
class may be obtained by consulting the API documentation.

\begin{note}
Future versions of the language may permit user-definable
places, and the ability to dynamically create places.
\end{note}

\begin{staticrule*}
Variables of type \xcd"Place" must be initialized and are implicitly
\xcd"final".  
\end{staticrule*}

\section{Place expressions}
Any expression of type \xcd"Place" is called a place expression. 
Examples of place expressions are \xcd"this.location" (the place
at which the current object lives), \xcd"here"
(the place where the current activity is executing), etc.

Place expressions are used in the following contexts: 
\begin{itemize}
%\item As a  place type in a type (\Sref{PlaceTypes}).
\item As a target for an \xcd"async" activity or a future
(\Sref{AsyncActivity}).
\item In a cast expression (\Sref{ClassCast}).
\item In an \xcd"instanceof" expression (\Sref{instanceOf}).
\item In stable equality comparisons, at type \xcd"Place".
\end{itemize}

Like values of any other type, places may be passed as arguments
to methods, returned from methods, stored in fields etc.

\section{\Xcd{here}}\index{here}\label{Here}
\Xten{} supports a special indexical constant\footnote{
An indexical constant is one whose value depends on its context
of use.} \xcd"here":

\begin{grammar}
ExpressionName \: \xcd"here" \\
\end{grammar}

The constant evaluates to the place at which the current activity is
running. Unlike other place expressions, this constant cannot be 
used as the placetype of fields, since the type of a field 
should be independent of the activity accessing it.

\paragraph{Example.}
The code:
\begin{xten}
public class F {
  public def m(a: F) {
    val OldHere: place = here;
    async (a) {
      Console.OUT.println("OldHere == here:" 
                          + (OldHere == here));
    }
  }
  public static void main(s: array[String]) {
    new F().m( (future(Place.FIRST_PLACE.next()) new F())() );
  }
}  
\end{xten}
\noindent will print out \xcd"true" iff the computation was configured
to start with the number of places set to \xcd"1". 


