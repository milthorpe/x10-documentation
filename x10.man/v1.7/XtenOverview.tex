\chapter{Overview of \Xten}

\Xten{} is a statically typed object-oriented language, extending a
sequential core language with 
\emph{places},
\emph{activities}, \emph{clocks},
(distributed, multi-dimensional) \emph{arrays} and \emph{value}
types. All these changes are motivated by the desire to use the new
language for high-end, high-performance, high-productivity computing.

\section{Object-oriented features}

The sequential core of \Xten{} is a class-based object-oriented language
similar to \java{} or Scala.
Programmers write \Xten{} code by writing
\emph{interfaces}
(\Sref{XtenInterfaces}) and
\emph{classes}
(\Sref{XtenClasses}).

An \Xten{} class has fields, methods and
inner types (interfaces, classes), subclasses another class, and
implements one or more interfaces. Thus \Xten{} classes live in a
single-inheritance code hierarchy.  An interface specifies a set
of methods, constant fields, and inner types.  Interfaces are
multiply inherited.

\paragraph{Dependent types}
Classes and interfaces may declare \emph{properties}: immutable object
members bound at object construction.
Types may be defined by constraining a
class or interface's properties.
\emph{Properties} enable the definition of \emph{dependent types}.

For example, the following code declares a class for a two-dimensional 
\xcd"Point" class with an \xcd"add" method.
\begin{xten}
class Point(x: Int, y: Int) {
    def add(p: Point): Point { return new Point(x+p.x,y+p.y); }
}
\end{xten}
The class has integer properties \xcd"x" and \xcd"y".
The \xcd"add" method creates and returns a new point by
element-wise addition.

% @@@### This sentence is out of place.  Define.
From this class, we can define the
the dependent type \xcd"Point{x==0}".  This is the type of all points 
with \xcd"x" set to \xcd"0"; that is, all points along the $y$-axis.

\paragraph{Generic types}

Classes and interfaces may have type parameters, permitting the definition of
{\em generic types}.
For example,
the following code declares a simple \xcd"List" class with a
type parameter \xcd"T".

\begin{xten}
class List[T] {
    var head: T;
    var tail: List[T];
    def this(h: T, t: List[T]) { head = h; tail = t; }
    def add(x: T) {
        if (this.tail == null)
            this.tail = new List(x, null);
        else
            this.tail.add(x);
    }
}
\end{xten}
The constructor (\xcd"def this") initializes the fields of the new object.
The \xcd"add" method appends an element to the list.
\xcd"List" is a generic type.  When  instances of \xcd"List" are
allocated, the type \param{} \xcd"T" must be bound to a concrete
type.  \xcd"List[Int]" is the type of lists of element type
\xcd"Int", \xcd"List[String]" is the type of lists of element
type \xcd"String".

\paragraph{Reference and value classes}

There are two kinds of classes: \emph{reference} classes
(\Sref{ReferenceClasses}) and \emph{value} classes
(\Sref{ValueClasses}).
A reference class typically has updatable fields.
Objects of such a
class may not be freely copied from place to place. Methods may be
invoked on such an object only by an activity in the same place.
The \xcd"null" reference is a value of any reference type.

A value class (\Sref{ValueClasses}) has no updatable fields (defined
directly or through inheritance), and allows no reference
subclasses. (Fields may be typed at reference classes, so may contain
references to objects with mutable state.) Objects of such a class may
be freely copied from place to place, and may be implemented very
efficiently. Methods may be invoked on such an object from any place.

\Xten{} has no primitive classes. However, the standard library
\xcd"x10.lang" supplies (final) value classes \xcd"Boolean", \xcd"Byte",
\xcd"Short", \xcd"Char", \xcd"Int", \xcd"Long", \xcd"Float",
\xcd"Double", \xcd"Complex" and \xcd"String". The user may define
additional arithmetic value classes using the facilities of the
language.

\section{The sequential core}

\paragraph{Control flow.}  \Xten{} supports standard sequential control flow
constructs: \xcd"if" statements, \xcd"while" loops, \xcd"for" loops,
\xcd"switch" statements, etc.  \Xten{} also supports exceptions: exceptions are
raised by \xcd"throw" statements and are handled by \xcd"try"--\xcd"catch"
statements.

\paragraph{Primitive operations.}  The language provides syntax for performing
binary and unary operations on values of types defined in the standard library.

\paragraph{Closures.}
\Xten{} provides closures (\Sref{Closures}) to allow code to be used
as values.  The body of a closure may capture variables in the closure's
environment.  Closures are used implicitly in asyncs, futures, and array
initializers.
%
For example, the following method uses a closure to increment
elements of an array by invoking the \xcd"lift" method on the
array.  The syntax \xcd"(x: Int) => x+1" specifies a closure
that takes an integer argument \xcd"x" and returns \xcd"x+1".
\begin{xten}
def incr(A: Array[Int]): Array[Int] = {
    val f = (x: Int) => x+1; // e.g., f(1) == 2
    return A.lift(f);
}
\end{xten}

\paragraph{Allocation.}
Objects are allocated with the \xcd"new" operator
(\Sref{ClassCreation}), which takes
a class name and type and value arguments to pass to the class's
constructor.  The constructor must ensure that all properties of
the class and its superclasses are bound.

\paragraph{Coercions and conversions}
\Xten{} supports implicit and explicit coercions and
conversions (\Sref{XtenConversions}).

Values of one type can be converted to another type using the
\xcd"as" operation:
\begin{xten}
val x: Int = 65535;
val y: Byte = x as Byte; // convert to Byte,
                         // retaining the lower 8 bits
\end{xten}
The \xcd"as" operation does not necessarily preserve equality
and for numeric values may 
result in a loss of precision.

References may be coerced to another type, preserving object
identity.  A run-time check is performed to ensure the reference
is to an object of the target type.  If not,
a \xcd"ClassCastException" is thrown.
\begin{xten}
// C and D are immediate subclasses of B.
val x: B = new C();
val y: C = x as C; // run-time check succeeds
val z: D = x as D; // run-time check fails
\end{xten}

\section{Places and activities}
An \Xten{} program is intended to run on a wide range of computers,
from uniprocessors to large clusters of parallel processors supporting
millions of concurrent operations. To support this scale, \Xten{}
introduces the central concept of \emph{place} (\Sref{XtenPlaces}).
Conceptually, a place is a ``virtual shared-memory multi-processor'':
a computational unit with a finite (though perhaps changing) number of
hardware threads and a bounded amount of shared memory, uniformly
accessible by all threads.

An \Xten{} computation acts on \emph{data
objects}(\Sref{XtenObjects}) through the execution of lightweight
threads called \emph{activities}(\Sref{XtenActivities}).  Objects are
of two kinds. A \emph{scalar} object has a small, statically fixed set
of fields, each of which has a distinct name. A scalar object is
located at a single place and stays at that place throughout its
lifetime.  An \emph{aggregate} object has many fields (the number may
be known only when the object is created), uniformly accessed through
an index (e.g., an integer) and may be distributed across many
places. The distribution of an aggregate object remains unchanged
throughout the computation. \Xten{} assumes an underlying garbage
collector will dispose of (scalar and aggregate) objects and reclaim
the memory associated with them once it can be determined that these
objects are no longer accessible from the current state of the
computation. (There are no operations in the language to allow a
programmer to explicitly release memory.)

{}\Xten{} has a \emph{unified} or \emph{global address space}. This
means that an activity can reference objects at other places.
However, an activity may synchronously access data items only in the
current place (the place in which the activity is running). It may
atomically update one or more data items, but only in the current
place.  To read a remote location, an activity must spawn another
activity \emph{asynchronously} (\Sref{AsynchronousActivity}). This
operation returns immediately, leaving the spawning activity with a
\emph{future} (\Sref{XtenFutures}) for the result. Similarly, remote
location can be written into only by asynchronously spawning an
activity to run at that location.

Throughout its lifetime an activity executes at the same place. An
activity may dynamically spawn activities in the current or remote
places.

\paragraph{Place casts.}
The programmer may use the standard type cast mechanism
(\Sref{ClassCast}) to cast a value to a located type. A
\xcd"BadPlaceException" is thrown if the value is not of the given
type. This is the only language construct that throws a \xcd"BadPlaceException".

\paragraph{Atomic blocks.}

\Xten{} introduces statements of the form \xcd"atomic S" where \xcd"S"
is a statement.  The type system ensures that such a statement will
dynamically access only local data. (The statement may throw
a \xcd"BadPlaceException"---but only because of a failed place cast.)
Such a statement is executed by the activity as if in a single step
during which all other activities are frozen.

\paragraph{Asynchronous activities.}

An asynchronous activity is created by a statement \xcd"async (P) S"
where \xcd"P" is a place expression and \xcd"S" is a statement.  Such
a statement is executed by spawning an activity at the place
designated by \xcd"P" to execute statement \xcd"S".

An asynchronous expression of type \xcd"Future[T]" has the form
\xcd"future (p) e" where \xcd"e" is an expression of type
\xcd"T".  The expression \xcd"e"
may reference final and shared variables declared in the lexically
enclosing environment.  It executes the expression \xcd"e" at the
place \xcd"p" as an asynchronous activity, immediately returning with
a future. The future may later be forced causing the activity to be
blocked until the return value has been computed by the asynchronous
activity.

\section{Clocks}
The MPI style of coordinating the activity of multiple processes with
a single barrier is not suitable for the dynamic network of (possibly
diverse) activities in an \Xten{} computation. Instead, it becomes
necessary to allow a computation to use multiple barriers. \Xten{}
\emph{clocks} (\Sref{XtenClocks}) are designed to offer the
functionality of multiple barriers in a dynamic context while still
supporting determinate, deadlock-free parallel computation.

Activities may use clocks to repeatedly detect quiescence of arbitrary
programmer-specified, data-dependent set of activities. Each activity
is spawned with a known set of clocks and may dynamically create new
clocks. At any given time an activity is \emph{registered} with zero or
more clocks. It may register newly created activities with a clock,
un-register itself with a clock, suspend on a clock or require that a
statement (possibly involving execution of new async activities) be
executed to completion before the clock can advance.  At any given
step of the execution a clock is in a given phase. It advances to the
next phase only when all its registered activities have \emph{quiesced}
(by executing a \xcd"next" operation on the clock).
When a clock advances, all its activities may now resume execution.

Thus clocks act as \emph{barriers} for a dynamically varying collection
of activities. They generalize the barriers found in MPI style program
in that an activity may use multiple clocks simultaneously. Yet
programs using clocks are guaranteed not to suffer from
deadlock.

\emph{In future versions of the language,
clocks will be integrated into the \Xten{} type system,
permitting variables to be declared so that they are final in each
phase of a clock.}

\section{Arrays, regions and distributions}

An \Xten{} array type is a map from a \emph{distribution}
(\Sref{XtenDistributions}) to a type, which may itself be an
array type.

% POINT REGION
A distribution is a map from a \emph{region} (\Sref{XtenRegions}) to
places.  A region is a collection of \emph{points} or
\emph{indices}. For instance, the region \xcd"[0..200,1..100]" specifies
a collection of two-dimensional points \xcd"[i,j]" with 
\xcd"i" ranging from \xcd"0" to \xcd"200" and \xcd"j" ranging
from \xcd"1" to \xcd"100". Points are used in array index expressions
to pick out a particular array element.

Operations are provided to construct regions from other regions, and
to iterate over regions. Standard set operations, such as union,
disjunction and set difference are available for regions.

A primitive set of distributions is provided, together with operations
on distributions. A \emph{sub-distribution} of a distribution is one
defined on a smaller region and agrees with the distribution
at all points.  The standard operations on regions are extended to
distributions.

% XXX views

A new array can be created by restricting an existing array to a
sub-distribution, by combining multiple arrays, and by performing
pointwise operations on arrays with the same distribution.

\Xten{} allows array constructors to iterate over the underlying
distribution and specify a value at each item in the underlying
region. Such a constructor may spawn activities at multiple places.

\emph{In future versions of the language, a programmer may specify new
distributions, and new operations on distributions.}

\section{Annotations}

\Xten{} supports annotations on classes and interfaces, methods
and constructors,
variables, types, expressions and statements.
These annotations may be processed by compiler plugins.

\section{Translating MPI programs to \Xten{}}

While \Xten{} permits considerably greater flexibility in writing
distributed programs and data structures than MPI, it is instructive
to examine how to translate MPI programs to \Xten.

Each separate MPI process can be translated into an \Xten{}
place. Async activities may be used to read and write variables
located at different processes. A single clock may be used for barrier
synchronization between multiple MPI processes. \Xten{} collective
operations may be used to implement MPI collective operations.
\Xten{} is more general than MPI in (a)~not requiring synchronization
between two processes in order to enable one to read and write the
other's values, (b)~permitting the use of high-level atomic blocks
within a process to obtain mutual exclusion between multiple
activities running in the same node (c)~permitting the use of multiple
clocks to combine the expression of different physics (e.g.,
computations modeling blood coagulation together with computations
involving the flow of blood), (d)~not requiring an SPMD style of
computation.


%\note{Relaxed exception model}
\section{Summary and future work}
\subsection{Design for scalability}
\Xten{} is designed for scalability. An activity may atomically
access only multiple locations in the current place. Unconditional
atomic blocks are statically guaranteed to be non-blocking, and may
be implemented using non-blocking techniques that avoid mutual
exclusion bottlenecks. Data-flow synchronization permits point-to-point
coordination between reader/writer activities, obviating the need for
barrier-based or lock-based synchronization in many cases.

\subsection{Design for productivity}
\Xten{} is designed for productivity. 

\paragraph{Safety and correctness.} 
Programs written in \Xten{} are guaranteed to be statically
\emph{type
safe}, \emph{memory safe} and \emph{pointer safe}. Static type safety
guarantees that at run time a location contains only those values who's
dynamic type satisfies the constraints imposed by the location's
static type and every run-time operation performed on the value in a
location is permitted by the static type of the location.

Memory safety guarantees that an object may only access memory within
its representation, and other objects it has a reference to. \Xten{}
supports no pointer arithmetic, and bound-checks array accesses
dynamically if necessary. \Xten{} uses dynamic garbage collection to
collect objects no longer referenced by the computation. \Xten{}
guarantees that no object can retain a reference to an object
whose memory has been reclaimed.  Further, \Xten{} guarantees that
every location is initialized at run time before it is read, 
and every value read from a location has previously been written into
that location. 

%XXX
%Pointer safety guarantees that a null pointer exception cannot be
%thrown by an operation on a value of a non-nullable type.

Because places are reflected in the type system, static type safety
also implies \emph{place safety}: a location may contain references to only
those objects whose location satisfies the restrictions of the static
place type of the location.

\Xten{} programs that use only clocks and unconditional atomic
blocks are guaranteed not to deadlock. Unconditional atomic blocks
are non-blocking, hence cannot introduce deadlocks (assuming the
implementation is correct).

Many concurrent programs can be shown to be determinate (hence
race-free) statically.

\paragraph{Integration.}
A key issue for any new programming language is how well it can be
integrated with existing (external) languages, system environments,
libraries and tools.

We believe that \Xten{}, like \java{}, will be able to support a large
number of libraries and tools. An area where we expect future versions
of \Xten{} to improve on \java{} like languages is \emph{native
integration} (\Sref{extern}). Specifically, \Xten{} will permit
permit multi-dimensional local arrays to be operated on natively by
native code.

%% Portability measures the amount of effort required to move an
%% application across multiple platforms, architectures and system
%% generations. The performance portability of applications across widely
%% different computer architectures (distributed cluster vs. vector
%% processor) depends significantly on inherent properties of the
%% underlying algorithms used in the application.
%% 
\subsection{Conclusion}
{}\Xten{} is considerably higher-level than thread-based languages in
that it supports dynamically spawning very lightweight activities, the
use of atomic operations for mutual exclusion, and the use of clocks
for repeated quiescence detection.

Yet it is much more concrete than languages like HPF in that it forces
the programmer to explicitly deal with distribution of data
objects. In this the language reflects the designers' belief that
issues of locality and distribution cannot be hidden from the
programmer of high-performance code in high-end computing.  A
performance model that distinguishes between computation and
communication must be made explicit and transparent.\footnote{In this
\Xten{} is similar to more modern languages such as ZPL \cite{zpl}.}
At the same time we believe that the place-based type system and
support for generic programming will allow the \Xten{} programmer to
be highly productive; many of the tedious details of
distribution-specific code can be handled in a generic fashion.

We expect the next version of the language to be significantly
informed by experience in implementing and using the language. We
expect it to have constructs to support continuous program
optimization, and allow the programmer to provide guidance on
clustering places to (hardware) nodes. For instance, we may introduce
a notion of hierarchical clustering of places.




