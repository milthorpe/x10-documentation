

\subsection{Default values design}
every type has a default value.
(when reading a val field, you may get the default value or its final value.)

\subsection{proto design}
A proto modifier for locals
allowed cyclic immutable objects.

\subsection{Hardhat design}
\label{Section:hardhat}

Discuss why we didn't allow aliasing of \this (to avoid an alias analysis).

A method is non-escaping, if it is called from a constructor or from another non-escaping method.
A non-escaping method must be private, final, or in a final class.
Constructors and non-escaping methods cannot leak \this.


- "this" cannot escape, no dynamic dispatching, etc.
- Dataflow for definite-assignment for locals (similarity to Java, with the extension for async-initialization)
- Dataflow for definite-assignment for fields:
  + @NoThisAccess and @NonEscaping
  + The fixed-point algorithm to infer the sets of fields that are written, async-written, and read, on each private method that is transitively called from a ctor or field-init. Maybe also add a nice proof :)


Property design (what would happen if we use val fields instead of properties.)

Fields are partition into two: property fields and normal fields. Property fields are assigned together before any other field.
\begin{lstlisting}
class A(x:Int) {
  val y:Int{self!=x};
  val z = x*x;
}
\end{lstlisting}
Vs.
Fields are ordered, and all ctors need to assign in the same order as the fields declaration.
\begin{lstlisting}
class A {
  val x:Int;
  val y:Int{self!=x};
  val z = x*x;
}
\end{lstlisting}



what can you read before (inclusive), and what after.


Hardhat claims:
* \this is the only accessible raw object (there could be several raw objects in the heap but only \this is accessible). Reason: \this cannot be aliased or leaked.
* only cooked objects cross places. Reason: \this is the only non-cooked object and it cannot have any aliases and it cannot cross an \code{at}.
* there is no dynamic dispatch during construction. Reason: only calling private or final methods.
* all field writes finish by the time the ctor ends. Reason: data flow ensures that any write within an async has an enclosing finish.
* one cannot read an uninitialized field. Reason: reading from \this is ok by data-flow, any other read is from a cooked object.
