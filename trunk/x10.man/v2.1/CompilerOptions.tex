\chapter{Options}

\subsection{Compiler Options}

The X10 compilers have many useful options. 

% -CHECK_INVARIANTS seems to check some internal compiler invariants.

\subsection{Optimization: {\tt -O} or {\tt -optimize}}

This flag causes the compiler to generate optimized code.


\subsection{Debugging: {\tt -DEBUG=boolean}}

This flag, if true, causes the compiler to generate debugging information.  It
is false by default.

\subsection{Call Style: {\tt -STATIC\_CALLS=boolean, -VERBOSE\_CALLS=boolean}}
\index{STATIC\_CALLS}
\index{VERBOSE\_CALLS}
\index{dynamic checks}

By default, if a method call {\em could} be correct but is not {\em
necessarily} correct, the X10 compiler generates a dynamic check to ensure
that it is correct before it is performed.  For example, the following code: 
\begin{xten}
def use(n:Int{self == 0}) {}
def test(x:Int) { 
   use(x); // creates a dynamic cast
}
\end{xten}
compiles with \xcd`-STATIC_CALLS=false`, even though it is possible that 
\xcd`x!=0` when \xcd`use(x)` is called.  In this case, the compiler inserts a
cast, which has the effect of checking that the call is correct before it
happens: 
\begin{xten}
def use(n:Int{self == 0}) {}
def test(x:Int) { 
   use(x as Int{self == 0}); 
}
\end{xten}
The compiler produces a warning that it inserted some dynamic casts.
If you then want to see what it did, use \xcd`-VERBOSE_CALLS`.

You may also turn on static checking, with the \xcd`-STATIC_CALLS` flag.  With
static checking, calls that cannot be proved correct statically will be
marked as errors.  The program without the dynamic cast fails to compile with
\xcd`-STATIC_CALLS`.  





\subsection{Help: {\tt -help} and {\tt -- -help}}

These options cause the compiler to print a list of all command-line options.


\subsection{Source Path: {\tt -sourcepath {\em path}}}

This option tells the compiler where to look for X10 source code.  


\subsection{Class Path: {\tt -classpath {\em path}}}

This option tells the compiler where to look for compiled code in {\tt class}
files.

\subsection{Output Directory: {\tt -d {\em directory}}}

This option tells the compiler to produce its output files in the specified directory.

\subsection{Runtime {\tt -x10rt {\em impl}}}

This option tells which runtime implementation to use.  The choices are
\xcd`lapi`, \xcd`pgp`, \xcd`sockets`, \xcd`mpi`, and \xcd`standalone`.



\section{Execution Options: Java}

The Java execution command \xcd`x10` has a number of options as well. 

\subsection{{\tt -NUMBER\_OF\_LOCAL\_PLACES={\em number}}}

This option controls how many \xcd`Place`s the system will run on.  The
default is four, but you may control it as you wish.

\subsection{Heap Size: {\tt -mx {\em size}}}

Sets the maximum size of the heap. 

\subsection{Help: {\tt -h}}

Prints a listing of all execution options.



%\subsection{{\tt }}
