\documentclass[10pt,a4paper]{article}

%% For typesetting theorems and some math symbols.
\usepackage{amssymb}
\usepackage{amsthm}

\usepackage{fullpage}

\title{Featherweight X10}

\author{}

\date{}




\usepackage{relsize}
\usepackage{amsmath}
\usepackage{url}


% http://en.wikibooks.org/wiki/LaTeX/Packages/Listings
\usepackage{listings}


\usepackage[ruled]{algorithm} % [plain]
\usepackage[noend]{algorithmic} % [noend]
\renewcommand\algorithmiccomment[1]{// \textit{#1}} %


% For fancy end of line formatting.
\usepackage{microtype}


% For smaller font.
\usepackage{pslatex}


\usepackage{xspace}
\usepackage{yglabels}
\usepackage{yglang}
\usepackage{ygequation}
\usepackage{graphicx}
%\usepackage{epstopdf}

\newcommand{\formalrule}[1]{\mbox{\textsc{\scriptsize #1}}}
\newcommand{\myrule}[1]{\textsc{\codesmaller #1 Rule}}
\newcommand{\umyrule}[1]{\textbf{\underline{\textsc{\codesmaller #1 Rule}}}}


% \small \footnotesize \scriptsize \tiny
% \codesize and \scriptsize seem to do the same thing.
% \newcommand{\code}[1]{\texttt{\textup{\footnotesize #1}}}
% \newcommand{\code}[1]{\texttt{\textup{\codesize #1}}}
\newcommand{\normalcode}[1]{\texttt{\textup{#1}}}
\def\codesmaller{\small}
\newcommand{\myCOMMENT}[1]{\COMMENT{\small #1}}
\newcommand{\code}[1]{\texttt{\textup{\codesmaller #1}}}
% \newcommand{\code}[1]{\ifmmode{\mbox{\smaller\ttfamily{#1}}}
%                       \else{\smaller\ttfamily #1}\fi}
\newcommand{\smallcode}[1]{\texttt{\textup{\scriptsize #1}}}
%\newcommand{\myparagraph}[1]{\noindent\textit{\textbf{#1}}~} %\vspace{-1mm}\paragraph{#1}}

% See: \usepackage{bold-extra} if you want to do \textbf
\newcommand{\keyword}[1]{\code{#1}}

% For new, method invocation, and cast:
\newcommand{\hparen}[1]{\code{(}#1\code{)}}

\newcommand{\hgn}[1]{\lt#1\gt} % type parameters and generic method parameters

\newcommand{\Own}{{\it O}}
\newcommand{\Ifn}[1]{\ensuremath{I(#1)}}
\newcommand{\Ofn}[1]{\ensuremath{O(#1)}}
\newcommand{\Cooker}[1]{\ensuremath{{\kappa}(#1)}}
\newcommand{\Owner}[1]{\ensuremath{{\theta}(#1)}}
\newcommand{\Oprec}[0]{\ensuremath{\preceq_{\theta}}}
\newcommand{\Tprec}[0]{\ensuremath{\preceq^T}}
\newcommand{\TprecNotEqual}[0]{\ensuremath{\prec^T}}
\newcommand{\OprecNotEqual}[0]{\ensuremath{\prec_{\theta}}}
\newcommand{\IfnDelta}[1]{\ensuremath{I_\Delta(#1)}}
\newcommand\Abs[1]{\ensuremath{\left\lvert#1\right\rvert}}
\newcommand{\erase}[1]{\ensuremath{\Abs{#1}}}

\newcommand{\CookerH}[1]{\ensuremath{{\kappa}_H(#1)}}
\newcommand{\IfnH}[1]{\ensuremath{I_H(#1)}}
\newcommand{\OwnerH}[1]{\ensuremath{{\theta}_H(#1)}}
\newcommand{\Gdash}[0]{\ensuremath{\Gamma \vdash }}
\newcommand{\reducesto}[0]{\rightsquigarrow}
\newcommand{\reduce}[0]{\rightsquigarrow}

\usepackage{color}
\definecolor{light}{gray}{.75}


\newcommand{\todo}[1]{\textbf{[[#1]]}}
%% To disable, just uncomment this line:
%\renewcommand{\todo}[1]{\relax}

%% Additional todo commands:
\newcommand{\TODO}[1]{\todo{TODO: #1}}

\newcommand\xX[1]{$\textsuperscript{\textit{\text{#1}}}$}


\newcommand{\ol}[1]{\overline{#1}}
\newcommand{\nounderline}[1]{{#1}}


%% Commands used to typeset the FIGJ type system.
\newcommand{\typerule}[2]{
\begin{array}{c}
  #1 \\
\hline
  #2
\end{array}}


%% Commands used to typeset the FOIGJ type system.
\newcommand{\inside}{\prec}
\newcommand{\visible}{{\it visible}}
\newcommand{\placeholderowners}{{\it placeholderowners}}
\newcommand{\nullexpression}{{\tt null}}
\newcommand{\errorexpression}{{\tt error}}
\newcommand{\locations}{{\it locations}} %% \mathop{\mathit{locations}}}
\newcommand{\xo}{{\tt X^O}}
\newcommand{\no}{{\tt N^O}}
\newcommand{\co}{{\tt C^O}}
\newcommand{\I}{\it I}


\newcommand\mynewcommand[2]{\newcommand{#1}{#2\xspace}}


\mynewcommand{\hI}{\code{I}} % iparam

% In the syntax: \hI or ReadOnly or Mutable or Immut
\mynewcommand{\hJ}{\code{J}}
\mynewcommand{\hO}{\code{O}}
\mynewcommand{\ho}{\code{o}}
\mynewcommand{\hnull}{\code{null}}
\mynewcommand{\htrue}{\code{true}}
\mynewcommand{\hfalse}{\code{false}}

\mynewcommand{\hX}{\code{X}} % vars
\mynewcommand{\hY}{\code{Y}} % vars
\mynewcommand{\hB}{\code{B}} % class
\mynewcommand{\hC}{\code{C}} % class
\mynewcommand{\hD}{\code{D}} % class
\mynewcommand{\hL}{\code{L}} % class decl
\mynewcommand{\hM}{\code{M}} % Method decl
\mynewcommand{\hm}{\code{m}} % method
\mynewcommand{\he}{\code{e}} % expression
\mynewcommand{\hl}{\code{l}} % location in the store
\mynewcommand{\hx}{\code{x}} % method parameter
\mynewcommand{\hf}{\code{f}} % field name
\mynewcommand{\hF}{\code{F}} % field
\mynewcommand{\hH}{\code{H}} % Heap
\mynewcommand{\hS}{\code{S}}
\mynewcommand{\hsub}{\code{/}} % substitute (reduction rules)

\mynewcommand{\hand}{\code{~and~}}
\mynewcommand{\hor}{\code{~or~}}
\mynewcommand{\hthis}{\code{this}} % this
\mynewcommand{\super}{\code{super}} % this
\mynewcommand{\hsuper}{\code{super}} % this
\mynewcommand{\hclass}{\code{class}}
\mynewcommand{\hreturn}{\code{return}}
\mynewcommand{\hnew}{\code{new}}
\newcommand{\lt}{\code{<}}%{\mathop{\textrm{\tt <}}}
\newcommand{\gt}{\code{>}}%{\mathop{\textrm{\tt >}}}

\mynewcommand{\this}{\keyword{this}}
\mynewcommand{\ctor}{\keyword{ctor}}
\mynewcommand{\Object}{\code{Object}}
\mynewcommand{\const}{\keyword{const}} %C++ keyword
\mynewcommand{\mutable}{\keyword{mutable}} %C++ keyword
\mynewcommand{\romaybe}{\keyword{romaybe}} %Javari keyword

%% Define the behaviour of the theorem package.
%% Use http://math.ucsd.edu/~jeggers/latex/amsthdoc.pdf for reference.

\newtheorem{theorem}{Theorem}[section]
\newtheorem{definition}[theorem]{Definition}
\newtheorem{lemma}[theorem]{Lemma}
\newtheorem{corollary}[theorem]{Corollary}
\newtheorem{fact}[theorem]{Fact}
\newtheorem{example}[theorem]{Example}
\newtheorem{remark}[theorem]{Remark}


\mynewcommand{\IP}{\code{I}}   % formal type parameter
\mynewcommand{\JP}{\code{J}}   % formal type parameter (for soundness proofs)


\mynewcommand{\Iparam}{Immutability parameter}
\mynewcommand{\iparam}{immutability parameter}
\mynewcommand{\iparams}{immutability parameters}
\mynewcommand{\Iparams}{Immutability parameters}
\mynewcommand{\Iarg}{Immutability argument}
\mynewcommand{\iarg}{immutability argument}
\mynewcommand{\iargs}{immutability arguments}
\mynewcommand{\Iargs}{Immutability arguments}
\mynewcommand{\ReadOnly}{\code{ReadOnly}}
\mynewcommand{\WriteOnly}{\code{WriteOnly}}
\mynewcommand{\None}{\code{None}}
\mynewcommand{\Mutable}{\code{Mutable}}
\mynewcommand{\Immut}{\code{Immut}}
\mynewcommand{\Raw}{\code{Raw}}


\mynewcommand{\This}{\code{This}}
\mynewcommand{\World}{\code{World}}


% Our annotations
\mynewcommand{\OMutable}{\code{@OMutable}}
\mynewcommand{\OI}{\code{@OI}}


\mynewcommand{\InVariantAnnot}{\code{@InVariant}}


\newcommand{\func}[1]{\text{\textnormal{\textit{\codesmaller #1}}}}


\mynewcommand{\st}{\ensuremath{\mathrel{{\leq}}}} %{\mathop{\textrm{\tt <:}}}
\mynewcommand{\notst}{\mathrel{\st\hspace{-1.5ex}\rule[-.25em]{.4pt}{1em}~}}
\mynewcommand{\tl}{\ensuremath{\triangleleft}}
\mynewcommand{\gap}{~ ~ ~ ~ ~ ~}


\newcommand{\RULE}[1]{\textsc{\scriptsize{}#1}} %\RULEhape\scriptsize}


\mynewcommand{\DA}{\texttt{DA}}
\mynewcommand{\ok}{\texttt{OK}}
\mynewcommand{\OK}{\texttt{OK}}
\mynewcommand{\IN}{\texttt{IN}}
\mynewcommand{\subterm}{\func{subterm}}
\mynewcommand{\TP}{\func{TP}} % function that returns type parameters in a type
\mynewcommand{\CT}{\func{CT}} % class table
\mynewcommand{\mtype}{\func{mtype}}
\mynewcommand{\mmodifier}{\func{mmodifier}}
\mynewcommand{\fmodifier}{\func{fmodifier}}
\mynewcommand{\ctortype}{\func{ctortype}}
\mynewcommand{\ctorbody}{\func{ctorbody}}
\mynewcommand{\mbody}{\func{mbody}}
\mynewcommand{\ftype}{\func{ftype}}
\mynewcommand{\fields}{\func{fields}}
\DeclareMathOperator{\dom}{dom}
\mynewcommand{\methodVal}{\func{methodVal}_\Gamma} % val not assigned
\mynewcommand{\ctorVal}{\func{ctorVal}_\Gamma} % val assigned at most once
\mynewcommand{\reductionVal}{\func{reductionVal}} % val assigned at most once



\mynewcommand\xth{\xX{th}}
\mynewcommand\xrd{\xX{rd}}
\mynewcommand\xnd{\xX{nd}}
\mynewcommand\xst{\xX{st}}
\mynewcommand\ith{$i$\xth}
\mynewcommand\jth{$j$\xth}


%\mynewcommand{\emptyline}{\vspace{\baselineskip}}
\mynewcommand{\myindent}{~~}


% Add line between figure and text
\makeatletter
\def\topfigrule{\kern3\p@ \hrule \kern -3.4\p@} % the \hrule is .4pt high
\def\botfigrule{\kern-3\p@ \hrule \kern 2.6\p@} % the \hrule is .4pt high
\def\dblfigrule{\kern3\p@ \hrule \kern -3.4\p@} % the \hrule is .4pt high
\makeatother

\setlength{\textfloatsep}{.75\textfloatsep}


% Remove line between figure and its caption.  (The line is prettier, and
% it also saves a couple column-inches.)
\makeatletter
%\@setflag \@caprule = \@false
\makeatother


% http://www.tex.ac.uk/cgi-bin/texfaq2html?label=bold-extras
\usepackage{bold-extra}


% Left and right curly braces in tt font
\newcommand{\ttlcb}{\texttt{\char "7B}}
\newcommand{\ttrcb}{\texttt{\char "7D}}
\newcommand{\lb}{\ttlcb}
\newcommand{\rb}{\ttrcb}


\setlength{\leftmargini}{.75\leftmargini}



\begin{document}


\maketitle


\lstset{language=java,basicstyle=\ttfamily\small}

%\chapter{Featherweight Ownership and Immutability Generic Java}
\section{Introduction}


Main lemma:
(i)~Progress and Preservation.
(ii)~After a location is cooked, val fields are never assigned;
    Before a location is cooked, val (and var) fields are never read.

Conclusion:
val fields have a single unique value that is read by all threads.
Like in Java, we will need a barrier whenever the constructor of some cooker is finished.
However, Java has the problem that if this escapes, then even this barrier does not guarantee immutability.
In contrast, in X10, we do make such guarantee.
We could also have a different implementation technique, where we flush only the newly created objects (no need to flush the entire memory).



We begin with some definitions.

FX10 program consists of class declarations followed by the program's expression.

An expression~$\he$ is called \emph{closed} (denoted $\closed{\he}$) if $\he$ does not contain \proto nor
    free variables (but it may contain \cooked or other locations).
For example, the expression~$\code{new Foo<l>()}$ is closed, but $\code{new Foo<proto>()}$ is not closed.
Similarly, we define a closed type.

Given a field $\code{A}~\code{C'}~\hf$ and a method $\code{U m(}\ol{\hV}~\ol{\hx}\code{) K \lb~return e;~\rb}$ in class~\code{C"},
    then for any subclass \code{C} of \code{C"}
    we define:
\beqst
%    \code{\mtype{}(m,C)} &\Fdef \ol{\hV}\rightarrow\hU \\
    \code{\mtype{}(m,C<K>)} &\Fdef  [\hK/\proto](\ol{\hV}\rightarrow\hU)\\
    \code{\mbody{}(m,C)} &\Fdef  \he\\
    \code{\mcooker{}(m,C)} &\Fdef  \hK\\
    \code{\fclass{}(f,C)} &\Fdef  \code{C'}\\
    \code{\isVar{}(f,C)} &\Fdef  (\code{A}=\code{var})\\
\eeq
Function~$\fields{}$ returns the fields of a class, i.e.,~$\fields{}(\hC)=\ol{\hf}$.

Function~$\cooker{\hK}$ receives a proto set~$P$ and a cooker~$\hK ::= \proto ~|~ \cooked ~|~ {\hl}$,
    and return whether it is cooked or proto:
\beqst
\cooker{\hK} \Fdef
\begin{cases}
\code{cooked} & \hK=\code{cooked} \\
\code{proto} & \hK=\code{proto} \\
\code{cooked} & \hK=\code{l}\text{~~and~~}\hl \not \in P \\
\code{proto} & \hK=\code{l}\text{~~and~~}\hl \in P \\
\end{cases}
\eeq

Given an expression~\he, we define~$R(\he)$ to be the set
    of all ongoing constructors in~\he, i.e., all locations in subexpressions~\code{e;return l}.
Formally,  $R(\he) \Fdef  \{ \hl ~|~ \code{return l} \in \he \}$.
%Formally,
%\[
%R(\he) =
%\begin{cases}
%    R(\code{e'}) \cup \{ l \} & \text{if~}\he=(\code{e';return l}) \\
%    R(\code{e'}) & \text{if~}\he=(\code{e'.f}) \\
%    R(\code{e'}) \cup R(\code{e"}) & \text{if~}\he=(\code{e'.f=e"}) \\
%    \bigcup_{\code{e'}\in\ol{\he'}} R(\code{e'}) & \text{if~}\he=(\code{new T}\hparen{\ol{\code{e'}}}) \\
%    R(\code{e"}) \cup (\bigcup_{\code{e'}\in\ol{\code{e'}}} R(\ol{\code{e'}})) & \text{if~}\he=(\code{e".m}\hparen{\ol{\code{e'}}}) \\
%    \end{cases}
%\]

Function~$\NPE(\he)$ returns whether an expression will throw a \code{null}-pointer exception:
\beqs{NPE}
\NPE(\he) & \Fdef \NPE(\he,\code{false}) \\
\NPE(\he,\code{b})& \Fdef
\begin{cases}
    \NPE(\code{e'},\code{false}) & \text{if~}\he=(\code{e';return l}) \\
    \NPE(\code{e'},\code{true}) & \text{if~}\he=(\code{e'.f}) \\
    \NPE(\code{e"},\code{true}) \vee \NPE(\code{e'},\code{false}) & \text{if~}\he=(\code{e".f=e'}) \\
    \vee \NPE(\ol{\code{e'}},\code{false}) & \text{if~}\he=(\code{new T}\hparen{\ol{\code{e'}}}) \\
    \NPE(\code{e'},\code{true}) \vee \NPE(\ol{\code{e"}},\code{false}) & \text{if~}\he=(\code{e'.m}\hparen{\ol{\code{e"}}}) \\
    \code{b} & \text{if~}\he=(\code{null}) \\
    \code{false} & \text{if~}\he=(\code{l}) \\
    \end{cases}\\
&\text{An alternative definition (though longer) that takes into account the order of evaluation: (which one is better?)}\\
\NPE(\he) & \Fdef
\begin{cases}
    \NPE(\code{e'}) & \text{if~}\he=(\code{e';return l}) \\
    \code{true} & \text{if~}\he=(\code{null.f}) \\
    \NPE(\code{e`}) & \text{if~}\he=(\code{e`.f}) \\
    \code{true} & \text{if~}\he=(\code{null.f=e'}) \\
    \NPE(\code{e'}) & \text{if~}\he=(\code{l.f=e'}) \\
    \NPE(\code{e"}) & \text{if~}\he=(\code{e".f=e'}) \\
    \NPE(\code{e'}) & \text{if~}\he=(\code{new T}\hparen{\ol{v},\code{e"},\ol{\code{e'}}}) \\
    \code{true} & \text{if~}\he=(\code{null.m}\hparen{\ol{\code{e'}}}) \\
    \NPE(\code{e"}) & \text{if~}\he=(\code{l.m}\hparen{\ol{v},\code{e"},\ol{\code{e'}}}) \\
    \NPE(\code{e"}) & \text{if~}\he=(\code{e".m}\hparen{\ol{\code{e'}}}) \\
    \code{false} & \text{if~}\he=(\code{v}) \\
    \end{cases}
    \gap \code{e"}\neq\hv
    \gap \code{e`}\neq\code{null}\\
\eeq

Summary of syntax used:
The comma operator ($,$) represents disjoint union.
Environment $\Gamma$ maps variables to types, i.e., $\Gamma(\hx)=\hT$.

A \emph{location}~\hl is a pointer to an object on the heap.
An \emph{object} has the form~$\code{C<l>}\hparen{\ol{\hv}}$, where~\hC is a class,~$\hl$ is the object's cooker, and~$\ol{\hv}$ are the values of the object's fields.
A \emph{heap}~$H$ maps locations to objects, i.e., $H(\hl)=\code{C<l'>}\hparen{\ol{\hv}}$.
The proto-set $P \in \dom(H)$ is a set of locations whose constructor has not finished yet.

We define two relations over cookers:
    (i)~$\Pequals{\hK}{\code{K'}}$ meaning that the two cookers are equivalent, and
    (ii)~$\Ppoints{\hK}{\code{K'}}$ meaning that an object with cooker~$\hK$ can point to an object with cooker~$\code{K'}$.
\beqst
\Pequals{\hK}{\code{K'}} & \Fdef \hK=\code{K'} \text{~~or~~} \cooker{\hK}=\cooker{\code{K'}}=\code{cooked}\\
\Ppoints{\hK}{\code{K'}} & \Fdef \hK=\code{K'} \text{~~or~~} \cooker{\code{K'}}=\code{cooked}\\
\eeq

A heap~$H$ is well-typed for~$P$, written~$P \vdash H$, if it satisfies:
    (i)~there is a linear order~$\Tprec$ over~$\dom{}(H)$ such that for every location~\hl, $H(\hl)=\code{C<l'>}\hparen{\ol{\hv}}$,
        we have~$\hl' \Tprec \hl$,
        and
    (ii)~for each object~$\code{C<l>}\hparen{\ol{\hv}}$ with fields~$\fields{}(\hC)=\ol{\hf}$, and for each non-null field~$\hv_i\neq\code{null}$,
        we have that~$H(\hv_i) = \code{C'<l'>}\hparen{\ldots}$, $\code{C'} \st \fclass(\hf_i,\hC)$, and~$\Ppoints{\hl}{\code{l'}}$.

Summary of judgements:
\beqst
\cooker{\hK} & \quad \text{Returns either \code{cooked} or \code{proto}}\\
\closed{\he} & \quad \text{Expression \he is closed}\\
\NPE(\he) & \quad \text{Expression \he is about to throw a \code{null}-pointer exception}\\
\hC \st \code{C'} & \quad \text{Class \hC is a subclass of \code{C'}}\\
P \vdash \hT \st \code{T'} & \quad \text{Type \hT is a subtype of \code{T'}} \\
\Ppoints{\hK}{\code{K'}} & \quad \text{An object with cooker \hK can point to another with cooker \code{K'}}\\
\Pequals{\hK}{\code{K'}} & \quad \text{Cooker \hK is equivalent to \code{K'}}\\
\Gamma,H,P \vdash \he : \hT & \quad \text{Expression \he has type \code{T}}\\
P \vdash H,\he \rightsquigarrow H',\code{e'} & \quad \text{Expression \he reduces (in one step) to \code{e'}, and heap~$H$ reduces to~$H'$}\\
P \vdash H & \quad \text{Heap $H$ is well-typed for~$P$}\\
\eeq

\begin{smaller}

\begin{figure*}[htpb!]
\begin{center}
\begin{tabular}{|l|l|}
\hline

$\hK ::= \proto ~|~ \cooked ~|~ \textbf{\hl}$ & cooKer. \\

$\code{T} ::= \code{C<K>}$ & Type. \\

$\code{A} ::= \code{var}~|~\code{val}$ & Assignable (\code{var}) or final (\code{val}) field. \\

$\code{F} ::= \code{A}~\hC~\hf\texttt{;}$ & Field declaration. \\

$\hM ::= \code{T} ~ \hm\hparen{\ol{\code{T}} ~ \ol{\hx}}~\hK ~ \lb\ \hreturn ~ \he\texttt{;}~\rb$
& Method declaration. \\

$\hL ::= \hclass ~ \hC\code{~extends~C'} \lb\ \ol{\code{F}}~\ol{\hM}~\rb$
& cLass declaration. \\


$\hv ::= \code{null} ~|~ \textbf{\hl} $
& Values: either \code{null} or a location~\hl. \\


% No cast: \hparen{\hT} ~ \he ~|~
$\he ::= \hv ~|~ \hx ~|~ \he.\hf ~|~ \he.\hf = \he ~|~ \he.\hm\hparen{\ol{\he}} ~|~ \hnew ~ \hT\hparen{\ol{\he}}  ~|~ \textbf{\he\code{;return l}}$
& Expressions. \\ %: values, parameters, field access\&assignment, invocation, \code{new} start\&finish

\hline
\end{tabular}
\end{center}
\caption{FX10 Syntax. Class declarations in FX10 cannot contain locations~\hl (marked with a boldface).
    Such locations are created during the reduction process (see \RULE{R-New} in \Ref{Figure}{reduction}).}
\label{Figure:syntax}
\end{figure*}


\begin{figure*}[!bt]
\begin{center}
\begin{tabular}{|c|}
\hline


$\typerule{
}{
  \code{C} \st \code{C}
}$
~\RULE{(S1)}\quad

$\typerule{
  \code{C} \st \code{C'}
   \gap
  \code{C'} \st \code{C"}
}{
  \code{C} \st \code{C"}
}$
~\RULE{(S2)}\quad
$\typerule{
  \code{class C extends C'}\lb\ \ol{\code{F}}~\ol{\hM}~\rb
}{
  \code{C} \st \code{C'}
}$
~\RULE{(S3)}

\\

$\typerule{
  \code{C} \st \code{C'}
  \gap
  \Pequals{\hK}{\code{K'}}
}{
  P \vdash \code{C<K>} \st \code{C<K'>}
}$
~\RULE{(S4)}\quad
\\


\hline
\end{tabular}
\end{center}
\caption{FX10 subclassing and subtyping rules.}
\label{Figure:subtyping}
\end{figure*}


\begin{figure*}[t]
\begin{center}
\begin{tabular}{|c|}
\hline
$\typerule{
  \Gamma,H,P \vdash \he:\hT
  \gap
  P \vdash \hT \st \code{T'}
}{
  \Gamma,H,P \vdash \code{e} : \code{T'}
}$
\quad \RULE{(T-subtype)}
\qquad
$\typerule{
  \Gamma,H,P \cup \{ \hl \} \vdash \he:\hT
}{
  \Gamma,H,P \vdash \code{e;return l} : \Gamma(\hl)
}$
\quad \RULE{(T-return)}
\\\\

$\typerule{
\hK'=
\begin{cases}
\bot & \hK=\cooked \\
\hK & \text{otherwise} \\
\end{cases}
    \gap
  \mtype{}(\code{build},\code{C<K'>})=\ol{\code{T}}\rightarrow\code{Object}
    \gap
  \Gamma,H,P \vdash \ol{\he}:\ol{\code{V}}
    \gap
  P \vdash \ol{\code{V}} \st \ol{\code{T}}
}{
  \Gamma,H,P \vdash \code{new C<K>(}\ol{\he}\code{)} : \code{C<K>}
}$
\quad \RULE{(T-New)}
\\\\

$\typerule{
}{
  \Gamma,H,P \vdash \code{null} : \hT
}$
\quad \RULE{(T-null)}
\qquad

$\typerule{
  \Gamma(\hx)=\hT
}{
  \Gamma,H,P \vdash \hx : \hT
}$
\quad \RULE{(T-Var)}
\qquad
$\typerule{
  H(\hl)=\code{C<l'>}\hparen{\ldots}
}{
  \Gamma,H,P \vdash \hl : \code{C<l'>}
}$
\quad \RULE{(T-Location)}\\\\

$\typerule{
  \Gamma,H,P \vdash \he:\code{C<K>}
    \gap
  \cooker{\hK}=\code{cooked}
    \gap
  \fclass{}(\hf,\hC)=\code{C'}
}{
  \Gamma,H,P \vdash \he.\hf : \code{C'<K>}
}$
\quad \RULE{(T-Field-Access)}\\\\


$\typerule{
  \Gamma,H,P \vdash \he:\code{C<K>}
    \gap
  \fclass{}(\hf,\hC)=\code{C"}
    \gap
  \Gamma,H,P \vdash \code{e'}:\code{C'<K'>}
    \gap
  \code{C'} \st \code{C"}
    \gap
  \Ppoints{\hK}{\code{K'}}
    \\
  \big(\cooker{\hK}=\code{proto} \text{~~or~~} \isVar{}(\hf,\hC)\big)
}{
  \Gamma,H,P \vdash \he.\hf = \code{e'} : \code{T'}
}$
\quad \RULE{(T-Field-Assignment)}\\\\

$\typerule{
  \Gamma,H,P \vdash \he':\code{C<K>}
    \gap
  \mtype{}(\hm,\code{C<K>})=\ol{\code{T}}\rightarrow\code{U}
    \gap
  \Gamma,H,P \vdash \ol{\he}:\ol{\code{V}}
    \gap
  P \vdash \ol{\code{V}} \st \ol{\code{T}}
    \\
  \cooker{\hK}=\mcooker{}(\hm,\code{C})
}{
  \Gamma,H,P \vdash \he'\code{.m(}\ol{\he}\code{)} : \code{U}
}$
\quad \RULE{(T-Invoke)}\\


\hline
\end{tabular}
\end{center}
\caption{FX10 Expression Typing Rules.}
\label{Figure:expressions}
\end{figure*}


\begin{figure*}[t]
\begin{center}
\begin{tabular}{|c|}
\hline

$\typerule{
  \hl \not \in \dom(H)
    \gap
  \code{l'} =
    \begin{cases}
    \hl & \text{if~}\cooker{\hK}=\code{cooked} \\
    \hK & \text{otherwise} \\
    \end{cases}
}{
  P \vdash H,\code{new C<K>}\hparen{\ol{\hv}} \rightsquigarrow H[\hl \mapsto \code{C<l'>}\hparen{\ol{\code{null}}}],\hl\code{.build}\hparen{\ol{\hv}}\code{;return l}
}$
\quad \RULE{(R-New)}\\\\

$\typerule{
  H(\hl) = \code{C<K>}\hparen{\ol{\hv}}
    \gap
  \fields{}(\hC)=\ol{\hf}
}{
  P \vdash H,\hl.\hf_i \rightsquigarrow H,\hv_i
}$
\quad \RULE{(R-Field-Access)}
\\\\

$\typerule{
  H(\hl) = \code{C<K>}\hparen{\ol{\hv}}
    \gap
  \fields{}(\hC)=\ol{\hf}
}{
  P \vdash H,\hl.\hf_i = \hv' \rightsquigarrow H[\hl \mapsto \code{C<K>}\hparen{[\hv'/\hv_i]\ol{\hv}}],\hv'
}$
\quad \RULE{(R-Field-Assignment)}\\\\


$\typerule{
}{
  P \vdash H,\code{v;return l} \rightsquigarrow H,\hl
}$
\quad \RULE{(R-return)}
\gap

$\typerule{
  H(\hl) = \code{C<K>}\hparen{\ldots}
    \gap
  \mbody{}(\hm,\code{C})=\ol{\hx}.\he'
}{
  P \vdash H,\hl\code{.m(}\ol{\hv}\code{)} \rightsquigarrow H, [\ol{\hv}/\ol{\hx}, \hl/\this, \hl/\proto]\he'
}$
\quad \RULE{(R-Invoke)}\\\\



$\typerule{
  P \cup \{\hl\} \vdash H,\he \rightsquigarrow H',\code{e'}
}{
  P \vdash H,\code{e;return l} \rightsquigarrow H',\code{e';return l}
}$
\quad \RULE{(R-c1)}
\gap

$\typerule{
  P \vdash H,\he \rightsquigarrow H',\code{e'}
}{
  P \vdash H,\code{e.f} \rightsquigarrow H',\code{e'.f}
}$
\quad \RULE{(R-c2)}
\\\\

$\typerule{
  P \vdash H,\he \rightsquigarrow H',\code{e'}
}{
  P \vdash H,\code{e.f=e"} \rightsquigarrow H',\code{e'.f=e"}
}$
\quad \RULE{(R-c3)}
\gap

$\typerule{
  P \vdash H,\he \rightsquigarrow H',\code{e'}
}{
  P \vdash H,\code{l.f=e} \rightsquigarrow H',\code{l.f=e'}
}$
\quad \RULE{(R-c4)}
\\\\

$\typerule{
  P \vdash H,\he \rightsquigarrow H',\code{e'}
}{
  P \vdash H,\code{new C<K>}\hparen{\ol{\hv},\he,\ol{\code{e"}}} \rightsquigarrow H',\code{new C<K>}\hparen{\ol{\hv},\code{e'},\ol{\code{e"}}}
}$
\quad \RULE{(R-c5)}
\\\\


$\typerule{
  P \vdash H,\he \rightsquigarrow H',\code{e'}
}{
  P \vdash H,\he\code{.m(}\ol{\code{e"}}\code{)} \rightsquigarrow H',\code{e'}\code{.m(}\ol{\code{e"}}\code{)}
}$
\quad \RULE{(R-c6)}
\gap

$\typerule{
  P \vdash H,\he \rightsquigarrow H',\code{e'}
}{
  P \vdash H,\code{l.m(}\ol{\hv},\he,\ol{\code{e"}}\code{)} \rightsquigarrow H',\code{l.m(}\ol{\hv},\code{e'},\ol{\code{e"}}\code{)}
}$
\quad \RULE{(R-c7)}
\gap

\\
\hline
\end{tabular}
\end{center}
\caption{FX10 Reduction Rules. The congruence rules have the initial \RULE{R-c}.}
\label{Figure:reduction}
\end{figure*}

\end{smaller}


Next we describe the syntax (\Ref{Figure}{syntax}),
    subtyping rules (\Ref{Figure}{subtyping}),
    expression typing rules (\Ref{Figure}{expressions}),
    and reduction rules (\Ref{Figure}{reduction}).

\section{Syntax}
Obviously, class declarations cannot contain locations.

\section{Subtyping}


\section{Typing}
\paragraph{Method typing}
If \proto appears in $\mtype{}(\hm,\hC)$ then $\mcooker{}(\hm,\hC)=\proto$.

An overriding method must keep the same $\mtype$ and $\mcooker$.

In class~\hC, when typing a method:
        $\code{U} ~ \hm\hparen{\ol{\code{V}} ~ \ol{\hx}} ~ \hK~ \lb\ \hreturn ~ \he\texttt{;} \rb$\\
        we use an environment~$\Gamma=\{\ol{\hx}:\ol{\code{T}}, \this:\code{C<K>}\}$,
        and we must prove that~$\Gamma,\emptyset,\emptyset \vdash \he:\code{S}$
        and~$\emptyset \vdash \code{S} \st \code{U}$.

\paragraph{Expression typing}
See \Ref{Figure}{expressions}.


\section{Reduction}
See \Ref{Figure}{reduction}.

\begin{Theorem}[preservation]
  \textbf{(Progress and Preservation)}
    For every expression~$\he$, heap~$H$, and proto-set~$P$,
        there exists~$H'$ and~$\he'$
        such that
        \[
        \begin{cases}
        & \he \neq \hv\\
        & \NPE(\he)=\code{false}\\
        & \closed{\he}\\
        & P \cup R(\he) \vdash H\\
        & \emptyset,H,P \vdash \he : \hT\\
        \end{cases}
        \Longrightarrow
        \begin{cases}
        & P \vdash H,\he \rightarrow H',\he'\\
        &\closed{\he'}\\
        &P \cup R(\he') \vdash H'\\
        &\emptyset,H',P \vdash \he':\hT\\
        \end{cases}
        \]
\end{Theorem}


(ii)~After a location is cooked, val fields are never assigned;
    Before a location is cooked, val (and var) fields are never read.


\end{document}
